\cutname{litmus.html}
Traditionally, a \emph{litmus test} is a small parallel program designed
to exercise the memory model of a parallel, shared-memory, computer.
Given a litmus test in assembler (X86, X86\_64, Power, ARM, MIPS, RISC-V), \litmus{}
runs the test.

Using \litmus{} thus requires a parallel machine,
which must additionally feature \prog{gcc} and the \prog{pthreads} library.
Our tool \litmus{} has some limitations especially
as regards recognised instructions.
Nevertheless, \litmus{} should accept all tests
produced by the companion test generators (see Part~\ref{part:diy})
and has been successfully used on Linux, MacOS, AIX and~Android.


\section{A tour of~\litmus{}}

\subsection{A \label{litmus:simple} simple run}
\aname{SB}{Consider} the following (rather classical, store buffering)
\afile{SB.litmus} litmus test for X86:
\verbatiminput{SB.litmus}
A litmus test source has three main sections:
\begin{enumerate}
\item The initial state defines the initial values of registers
and memory locations. Initialisation to zero may be omitted.
\item The code section defines the code to be run concurrently
--- above there are two threads.
Yes we know, our X86 assembler syntax is a mistake.
\item The final condition applies to the final values
of registers and memory locations.
\end{enumerate}

\label{x86:classic}Run the test by\ifhevea{}
(complete \ahref{SB.log}{run log})\fi:
\begin{verbatim}
% litmus7 SB.litmus
%%%%%%%%%%%%%%%%%%%%%%%%%
% Results for SB.litmus %
%%%%%%%%%%%%%%%%%%%%%%%%%
X86 SB
"Fre PodWR Fre PodWR"

{x=0; y=0;}

 P0          | P1          ;
 MOV [x],$1  | MOV [y],$1  ;
 MOV EAX,[y] | MOV EAX,[x] ;

exists (0:EAX=0 /\ 1:EAX=0)
Generated assembler
#START _litmus_P1
	movl $1,(%r10)
	movl (%r9),%eax
#START _litmus_P0
	movl $1,(%r9)
	movl (%r10),%eax

Test SB Allowed
Histogram (4 states)
40    *>0:EAX=0; 1:EAX=0;
499923:>0:EAX=1; 1:EAX=0;
500009:>0:EAX=0; 1:EAX=1;
28    :>0:EAX=1; 1:EAX=1;
Ok

Witnesses
Positive: 40, Negative: 999960
Condition exists (0:EAX=0 /\ 1:EAX=0) is validated
Hash=7dbd6b8e6dd4abc2ef3d48b0376fb2e3
Observation SB Sometimes 40 999960
Time SB 0.44
...
\end{verbatim}
The  litmus test is first reminded, followed by actual assembler
--- the machine is a 64~bits one, in-line address references disappeared,
registers may change, and assembler syntax is now more familiar.
The test has run one million times, producing one million final states,
or \emph{outcomes} for the registers \verb+EAX+ of threads \P{0} and~\P{1}.
The test run validates the condition, with $40$~positive witnesses.



\subsection{Cross\label{cross} compilation}

With option \opt{-o <name.tar>}, \litmus{} does not run the test.
Instead, it produces a tar archive that contains the C~sources for
the test.

Consider \afile{SB-PPC.litmus}, a Power version
of the previous test:
\verbatiminput{SB-PPC.litmus}


Our target machine (ppc) runs MacOS, which we specify with the \opt{-os}~option:
\begin{verbatim}
% litmus7 -o /tmp/a.tar -os mac SB-PPC.litmus
% scp /tmp/a.tar ppc:/tmp
\end{verbatim}
Then, on the remote machine ppc:
\begin{verbatim}
ppc% mkdir SB && cd SB
ppc% tar xf /tmp/a.tar
ppc% ls
comp.sh  Makefile  outs.c  outs.h  README.txt  run.sh  SB-PPC.c  show.awk  utils.c  utils.h
\end{verbatim}
Test is compiled by the shell script \file{comp.sh} (or
by (Gnu) \texttt{make}, at user's choice) and run by the shell script
\file{run.sh}:
\begin{verbatim}
ppc% sh comp.sh
ppc% sh run.sh
...
Test SB-PPC Allowed
Histogram (3 states)
1784  *>0:r3=0; 1:r3=0;
498564:>0:r3=1; 1:r3=0;
499652:>0:r3=0; 1:r3=1;
Ok

Witnesses
Positive: 1784, Negative: 998216
Condition exists (0:r3=0 /\ 1:r3=0) is validated
Hash=4edecf6abc507611612efaecc1c4a9bc
Observation SB-PPC Sometimes 1784 998216
Time SB-PPC 0.55
...
\end{verbatim}
%$
\ifhevea(Complete \ahref{SB-PPC.log}{run log}.) \fi
As we see, the condition validates also on Power.
Notice that compilation produces an executable file, \file{SB-PPC.exe},
which can be run directly, for a less verbose output.


\subsection{Running several tests at once}

\label{stfw}Consider the additional test~\afile{STFW-PPC.litmus}:
\verbatiminput{STFW-PPC.litmus}

To compile the two tests together,
we can give two file names as arguments to litmus:
\begin{verbatim}
$ litmus7 -o /tmp/a.tar -os mac SB-PPC.litmus STFW-PPC.litmus
\end{verbatim}
Or, more conveniently, list the litmus sources in a file whose
name starts with \file{@}:
\begin{verbatim}
$ cat @ppc
SB-PPC.litmus
STFW-PPC.litmus
$ litmus7 -o /tmp/a.tar -os mac @ppc
\end{verbatim}
%$
To run the test on the remote ppc machine, the same sequence
of commands as in the one test case applies:
\begin{verbatim}
ppc% tar xf /tmp/a.tar && make && sh run.sh
...
Test SB-PPC Allowed
Histogram (3 states)
1765  *>0:r3=0; 1:r3=0;
498741:>0:r3=1; 1:r3=0;
499494:>0:r3=0; 1:r3=1;
Ok

Witnesses
Positive: 1765, Negative: 998235
Condition exists (0:r3=0 /\ 1:r3=0) is validated
Hash=4edecf6abc507611612efaecc1c4a9bc
Observation SB-PPC Sometimes 1765 998235
Time SB-PPC 0.57
...
Test STFW-PPC Allowed
Histogram (4 states)
480   *>0:r3=1; 0:r4=0; 1:r3=1; 1:r4=0;
499560:>0:r3=1; 0:r4=1; 1:r3=1; 1:r4=0;
499827:>0:r3=1; 0:r4=0; 1:r3=1; 1:r4=1;
133   :>0:r3=1; 0:r4=1; 1:r3=1; 1:r4=1;
Ok

Witnesses
Positive: 480, Negative: 999520
Condition exists (0:r3=1 /\ 0:r4=0 /\ 1:r3=1 /\ 1:r4=0) is validated
Hash=92b2c3f6332309325000656d0632131e
Observation STFW-PPC Sometimes 480 999520
Time STFW-PPC 0.56
...
\end{verbatim}
\ifhevea(Complete \ahref{SB-PPC2.log}{run log}.) \fi
Now, the output of \file{run.sh} shows the result of two tests.

\section{Controlling\label{litmus:control} test parameters}
Users can control some of testing conditions.
Those impact efficiency and outcome variability.

Sometimes one looks for a particular outcome
--- for instance, one may seek to get the
outcome \verb+0:r3=1; 1:r3=1;+ that is missing
in the previous experiment for test~\ltest{SB-PPC}.
To that aim, varying test conditions may help.


\subsection{\label{sec:arch}Architecture of tests}

Consider a test \file{a.litmus}
designed to run on $t$ threads \P{0},\ldots, \P{t-1}.
The structure of the executable \file{a.exe} that performs
the experiment is as follows:
\begin{itemize}
\item \label{defn}\label{defa}So as to benefit from parallelism,
we run $n = \max(1,a/t)$ (integer division)
tests concurrently on a machine where $a$~logical processors are available.
\item \label{defr}Each of these (identical)
tests consists in repeating $r$ times
the following sequence:
\begin{itemize}
\item Fork ~$t$ (POSIX) threads $T_0,\ldots T_{t-1}$
for executing \P{0},\ldots, \P{t-1}.
Which thread executes which code is either fixed, or changing,
controlled by the \emph{launch mode}.
In our experience, the launch mode has marginal impact.

In \emph{cache mode} the $T_k$ threads are re-used.
As a consequence, $t$~threads only are forked.

\item \label{defs}Each thread $T_k$ executes a loop of size~$s$.
Loop iteration number~$i$ executes the code of \P{k} (in fixed mode)
and saves
the final contents of its observed registers in some arrays indexed by~$i$.
Furthermore, still for iteration~$i$, memory location~$x$ is in fact
an array cell.

\label{defmemorymode}\label{defstride}How this array cell is accessed depends
upon the \emph{memory mode}.
In \emph{direct mode} the array cell is accessed directly as~$x[i]$;
as a result, cells are accessed sequentially and false sharing effects
are likely.
In \emph{indirect mode} the array cell is accessed by the means of a
shuffled array of pointers;
as a result we observed a much greater variability of outcomes.
Additionally, the increment of the main loop (of size~$s$)
can be set to a value or \emph{stride} different from the default of~one.
Running a test several times with changing the stride value also
proved quite effective in favouring outcome variability.



\label{defpreload}If the \emph{random preload mode} is enabled,
a preliminary loop of size~$s$ reads
a random subset of the memory locations accessed by~\P{k}.
Preload has a noticeable effect and the random preload mode is
enabled by default.
Starting from version~5.0, we provide a more precise control
over preloading memory locations --- See Sec.~\ref{preload:custom}.


\label{defsynchronisation}The iterations performed
by the different threads~$T_k$ may be unsynchronised,
exactly synchronised by a pthread based barrier, or approximately synchronised
by specific code.
Absence of synchronisation may be interesting when $t$ exceeds~$a$.
As a matter of fact, in this situation,
any kind of synchronisation leads to prohibitive running times.
However, for a large value of parameter~$s$ and small $t$ we have observed
spontaneous concurrent execution of some iterations amongst many.
Pthread based barriers are exact but they are slow
and in fact offers poor synchronisation for short code sequences.
The approximate synchronisation is thus the preferred technique.

Starting from version 5.0, we provide a slightly altered
user synchronisation mode: \emph{userfence}, which alters
user mode by executing memory fences to speedup write propagation.
The new mode features overall better synchronisation, yielding dramatic
improvements on some examples. However,
outcome variability may suffer from this more accurate synchronisation,
hence user mode remains the default.

\label{timebase:intro}More importantly,
we provide an additional exact, \emph{timebase}
synchronisation
technique: test threads will first synchronise using polling synchronisation
barrier code,
agree on a target timebase\footnote{Power and x86-based systems
provide a user accessible timebase counter that should provide
consistent times to all cores and processors.}  value and then loop
reading the timebase until it exceeds the target value.
This technique yields very good synchronisation and allows
fine synchronisation tuning by assigning different starting delays to
different threads --- see Sec.~\ref{timebase}.
As ARM does not provide timebase counters,
notice that ``timebase'' synchronisation for ARM silently degrades
to synchronisation by the means of the polling synchronisation barrier.


\item Wait for the $t$ threads to terminate and collect outcomes
in some histogram like structure.
\end{itemize}
\item Wait for the $n$~tests to terminate and sum their histograms.
\end{itemize}

Hence, running \file{a.exe} produces $n \times r \times s$ outcomes.
Parameters $n$, $a$, $r$ and~$s$ can first be set directly while
invoking \file{a.exe}, using the appropriate command line options.
For instance, assuming $t=2$,
\verb+./a.exe -a 20 -r 10k -s 1+ and \verb+./a.exe -n 1 -r 1 -s 1M+
will both produce one million outcomes, but the latter is probably
more efficient.
If our machine has $8$~cores,
\verb+./a.exe -a 8 -r 1 -s 1M+ will yield $4$~millions outcomes,
in a time that we hope not to exceed too much the one experienced
with~\verb+./a.exe -n 1+.
Also observe that the memory allocated is roughly proportional
to $n \times s$, while the number of $T_k$~threads created will be
$t \times n \times r$ ($t \times n$ in cache mode).
The \file{run.sh} shell script transmits its command line to all
the executable (\file{.exe}) files
it invokes, thereby providing a convenient means
to control testing condition for several tests.
Satisfactory test parameters are found by experimenting and
the control of executable files by command line options is designed for
that purpose.

Once satisfactory parameters are found, it is a nuisance to repeat them
for every experiment.
Thus, parameters $a$, $r$ and~$s$ can also be set while invoking litmus,
with the same command line options. In fact those settings command
the default values of \file{.exe}~files controls.
Additionally, the synchronisation technique for iterations,
the memory mode, and several others compile time parameters
can be selected by appropriate \litmus{} command line options.
Finally, users can record frequently used parameters in configuration files.

\subsection{Affinity\label{sec:affinity}}

We view affinity as
a scheduler property that binds a (software, POSIX) thread to
a given (hardware) \emph{logical processor}.
In the most simple situation a logical processor is a core.
However in the presence of hyper-threading (x86) or simultaneous multi threading
(SMT, Power) a given core can host several logical processors.

\subsubsection{Introduction to affinity}
In our experience,
binding the threads of test programs to selected logical processors
yields significant speedups and, more importantly, greater outcome variety.
We illustrate the issue by the means of an example.

We consider the test~\afile{ppc-iriw-lwsync.litmus}:
\verbatiminput{ppc-iriw-lwsync.litmus}
The test consists of four threads.
There are  two writers (P0 and P2) that write the value
one into two different locations (\texttt{x} and \texttt{y}),
and two readers that read the contents of \texttt{x} and \texttt{y}
in different orders --- P1 reads \texttt{x} first, while P3 reads
\texttt{y} first.
The load instructions \texttt{lwz} in reader threads are separated
by a lightweight barrier instruction~\texttt{lwsync}.
The final condition \verb+exists (1:r1=1 /\ 1:r3=0 /\ 3:r1=1 /\ 3:r3=0)+
characterises the situation where the reader threads see the writes
by P0 and P2 in opposite order.
The corresponding outcome  \verb+1:r1=1; 1:r3=0; 3:r1=1; 3:r3=0;+
is the only non-sequential consistent (non-SC, see Part~\ref{part:diy}) possible outcome.
By any reasonable memory model for Power,
one expects the condition to validate,
\emph{i.e.} the non-SC outcome to show up.

The tested machine
\ahref{http://www.idris.fr/su/Scalaire/vargas/hw-vargas.html}{\texttt{vargas}}
is a Power~6 featuring 32~cores (\emph{i.e.}
64 logical processors, since SMT is enabled) and running AIX in 64 bits mode.
So as not to disturb other users, we run only one instance of the test,
thus specifying four available processors.
The \litmus{} tool is absent on \texttt{vargas}.
All these conditions command the following invocation of \litmus{},
performed on our local machine:
\begin{verbatim}
$ litmus7 -r 1000 -s 1000 -a 4 -os aix -ws w64 ppc-iriw-lwsync.litmus -o ppc.tar
$ scp ppc.tar vargas:/var/tmp
\end{verbatim}
On \texttt{vargas} we unpack the archive and compile the test:
\begin{verbatim}
vargas% tar xf /var/tmp/ppc.tar && sh comp.sh
\end{verbatim}
Then we run the test:
\begin{verbatim}
vargas% ./ppc-iriw-lwsync.exe
Test ppc-iriw-lwsync Allowed
Histogram (15 states)
163674:>1:r1=0; 1:r3=0; 3:r1=0; 3:r3=0;
34045 :>1:r1=1; 1:r3=0; 3:r1=0; 3:r3=0;
40283 :>1:r1=0; 1:r3=1; 3:r1=0; 3:r3=0;
95079 :>1:r1=1; 1:r3=1; 3:r1=0; 3:r3=0;
33848 :>1:r1=0; 1:r3=0; 3:r1=1; 3:r3=0;
72201 :>1:r1=0; 1:r3=1; 3:r1=1; 3:r3=0;
32452 :>1:r1=1; 1:r3=1; 3:r1=1; 3:r3=0;
43031 :>1:r1=0; 1:r3=0; 3:r1=0; 3:r3=1;
73052 :>1:r1=1; 1:r3=0; 3:r1=0; 3:r3=1;
1     :>1:r1=0; 1:r3=1; 3:r1=0; 3:r3=1;
42482 :>1:r1=1; 1:r3=1; 3:r1=0; 3:r3=1;
90470 :>1:r1=0; 1:r3=0; 3:r1=1; 3:r3=1;
30306 :>1:r1=1; 1:r3=0; 3:r1=1; 3:r3=1;
43239 :>1:r1=0; 1:r3=1; 3:r1=1; 3:r3=1;
205837:>1:r1=1; 1:r3=1; 3:r1=1; 3:r3=1;
No

Witnesses
Positive: 0, Negative: 1000000
Condition exists (1:r1=1 /\ 1:r3=0 /\ 3:r1=1 /\ 3:r3=0) is NOT validated
Hash=4fbfaafa51f6784d699e9bdaf5ba047d
Observation ppc-iriw-lwsync Never 0 1000000
Time ppc-iriw-lwsync 1.32
\end{verbatim}
%$
The non-SC outcome does not show up.

Altering parameters may yield this outcome.
In particular, we may try using all the available logical processors
with option \texttt{-a 64}.
Affinity control offers an alternative, which is enabled at compilation time
with \litmus{} option \texttt{-affinity}:
\begin{verbatim}
$ litmus7 ... -affinity incr1 ppc-iriw-lwsync.litmus -o ppc.tar
$ scp ppc.tar vargas:/var/tmp
\end{verbatim}
Option \texttt{-affinity} takes one argument (\texttt{incr1} above)
that specifies the increment used while allocating
logical processors to test threads.
Here, the  (POSIX) threads created by the test
(named $T_0$, $T_1$, $T_2$ and $T_3$ in Sec.~\ref{sec:arch})
will get bound to logical processors
$0$, $1$, $2$, and~$3$, respectively.

\label{defi}Namely, by default, the logical processors are
ordered as the sequence $0, 1, \ldots, A-1$ ---
where $A$ is the number of available logical processors, which is
inferred by the test executable\footnote{Parameter $A$ is not to be confused with~$a$ of section~\ref{sec:arch}. The former  serves to compute logical threads while the latter governs the number of tests that run simultaneously. However
parameters~$a$ will be set to~$A$ when affinity control is enabled and when
$a$~value is~$0$.}.
Furthermore, logical processors are allocated to threads by
applying the affinity increment while scanning the logical processor sequence.
Observe that since the launch mode is changing (the default) threads
$T_k$ correspond to different test threads~$P_i$ at each run.
The unpack compile and run sequence on \texttt{vargas} now yields
the non-SC outcome, better outcome variety and a lower running time:
\begin{verbatim}
vargas% tar xf /var/tmp/ppc.tar && make
vargas% ./ppc-iriw-lwsync.exe
Test ppc-iriw-lwsync Allowed
Histogram (16 states)
180600:>1:r1=0; 1:r3=0; 3:r1=0; 3:r3=0;
3656  :>1:r1=1; 1:r3=0; 3:r1=0; 3:r3=0;
18812 :>1:r1=0; 1:r3=1; 3:r1=0; 3:r3=0;
77692 :>1:r1=1; 1:r3=1; 3:r1=0; 3:r3=0;
2973  :>1:r1=0; 1:r3=0; 3:r1=1; 3:r3=0;
9     *>1:r1=1; 1:r3=0; 3:r1=1; 3:r3=0;
28881 :>1:r1=0; 1:r3=1; 3:r1=1; 3:r3=0;
75126 :>1:r1=1; 1:r3=1; 3:r1=1; 3:r3=0;
20939 :>1:r1=0; 1:r3=0; 3:r1=0; 3:r3=1;
30498 :>1:r1=1; 1:r3=0; 3:r1=0; 3:r3=1;
1234  :>1:r1=0; 1:r3=1; 3:r1=0; 3:r3=1;
89993 :>1:r1=1; 1:r3=1; 3:r1=0; 3:r3=1;
75769 :>1:r1=0; 1:r3=0; 3:r1=1; 3:r3=1;
76361 :>1:r1=1; 1:r3=0; 3:r1=1; 3:r3=1;
87864 :>1:r1=0; 1:r3=1; 3:r1=1; 3:r3=1;
229593:>1:r1=1; 1:r3=1; 3:r1=1; 3:r3=1;
Ok

Witnesses
Positive: 9, Negative: 999991
Condition exists (1:r1=1 /\ 1:r3=0 /\ 3:r1=1 /\ 3:r3=0) is validated
Hash=4fbfaafa51f6784d699e9bdaf5ba047d
Observation ppc-iriw-lwsync Sometimes 9 999991
Time ppc-iriw-lwsync 0.68
\end{verbatim}


One may change the affinity increment with the command line option
\texttt{-i} of executable files. For instance, one binds the test threads
to logical processors $0$, $2$, $4$ and~$6$ as follows:
\begin{verbatim}
vargas% ./ppc-iriw-lwsync.exe -i 2
Test ppc-iriw-lwsync Allowed
Histogram (15 states)
160629:>1:r1=0; 1:r3=0; 3:r1=0; 3:r3=0;
33389 :>1:r1=1; 1:r3=0; 3:r1=0; 3:r3=0;
43725 :>1:r1=0; 1:r3=1; 3:r1=0; 3:r3=0;
93114 :>1:r1=1; 1:r3=1; 3:r1=0; 3:r3=0;
33556 :>1:r1=0; 1:r3=0; 3:r1=1; 3:r3=0;
64875 :>1:r1=0; 1:r3=1; 3:r1=1; 3:r3=0;
34908 :>1:r1=1; 1:r3=1; 3:r1=1; 3:r3=0;
43770 :>1:r1=0; 1:r3=0; 3:r1=0; 3:r3=1;
64544 :>1:r1=1; 1:r3=0; 3:r1=0; 3:r3=1;
4     :>1:r1=0; 1:r3=1; 3:r1=0; 3:r3=1;
54633 :>1:r1=1; 1:r3=1; 3:r1=0; 3:r3=1;
92617 :>1:r1=0; 1:r3=0; 3:r1=1; 3:r3=1;
34754 :>1:r1=1; 1:r3=0; 3:r1=1; 3:r3=1;
54027 :>1:r1=0; 1:r3=1; 3:r1=1; 3:r3=1;
191455:>1:r1=1; 1:r3=1; 3:r1=1; 3:r3=1;
No

Witnesses
Positive: 0, Negative: 1000000
Condition exists (1:r1=1 /\ 1:r3=0 /\ 3:r1=1 /\ 3:r3=0) is NOT validated
Hash=4fbfaafa51f6784d699e9bdaf5ba047d
Observation ppc-iriw-lwsync Never 0 1000000
Time ppc-iriw-lwsync 0.92
\end{verbatim}
One observes that the non-SC outcome does not show up
with the new affinity setting.

One may also bind test thread to logical processors randomly with
executable option \texttt{+ra}.
\begin{verbatim}
vargas% ./ppc-iriw-lwsync.exe +ra
Test ppc-iriw-lwsync Allowed
Histogram (15 states)
...
No

Witnesses
Positive: 0, Negative: 1000000
Condition exists (1:r1=1 /\ 1:r3=0 /\ 3:r1=1 /\ 3:r3=0) is NOT validated
Hash=4fbfaafa51f6784d699e9bdaf5ba047d
Observation ppc-iriw-lwsync Never 0 1000000
Time ppc-iriw-lwsync 1.85
\end{verbatim}

As we see, the condition does not validate either with random affinity.
As a matter of fact, logical processors are taken at random in the
sequence $0$, $1$, \ldots, $63$;
while the successful run with \texttt{-i 1} took
them in the sequence $0$, $1$, $2$, $3$.
One can limit the sequence of  logical processor with option \texttt{-p},
which takes a sequence of logical processors numbers as argument:
\begin{verbatim}
vargas% ./ppc-iriw-lwsync.exe +ra -p 0,1,2,3
Test ppc-iriw-lwsync Allowed
Histogram (16 states)
...
8     *>1:r1=1; 1:r3=0; 3:r1=1; 3:r3=0;
...
Ok

Witnesses
Positive: 8, Negative: 999992
Condition exists (1:r1=1 /\ 1:r3=0 /\ 3:r1=1 /\ 3:r3=0) is validated
Hash=4fbfaafa51f6784d699e9bdaf5ba047d
Observation ppc-iriw-lwsync Sometimes 8 999992
Time ppc-iriw-lwsync 0.70
\end{verbatim}
The condition now validates.

\subsubsection{Study of affinity}
As illustrated by the previous example, both the running time and the outcomes
of a test are sensitive to affinity settings.
We measured running time for increasing values of the affinity increment
from $0$ (which disables affinity control)
to~$20$, producing the following figure:
\begin{center}\image[height=75ex]{m1l-time}\end{center}
As regards outcome variety,
we get all of the $16$ possible
outcomes only for an affinity increment of~$1$.

The differences in running times can be explained by reference to the mapping
of logical processors to hardware.
The machine~\texttt{vargas} consists in four MCM's (Multi-Chip-Module), each MCM
consists in four ``chips'', each chip consists in two cores, and
each core may support two logical processors.
As far as we know, by querying \texttt{vargas}
with the AIX commands
\texttt{lsattr}, \texttt{bindprocessor}
and \texttt{llstat},
the MCM's hold the logical processors
$0$--$15$, $16$--$31$, $32$--$47$ and~$48$--$63$,
each chip holds the logical processors $4k, 4k+1, 4k+2, 4k+3$
and each core holds the logical processors $2k, 2k+1$.

The measure of running times for varying increments
reveals two noticeable slowdowns:
from an increment of~$1$ to an increment of~$2$ and from $5$ to~$6$.
The gap between~$1$ and~$2$ reveals the benefits of
SMT for our testing application.
An increment of~$1$ yields both the greatest outcome
variety and the minimal running time.
The other gap may perhaps be explained by reference to MCM's:
for a value of~$5$ the tests runs on the logical processors
$0, 5, 10, 15$, all belonging to the same
MCM; while the next affinity increment of~$6$ results in
running the test on two different MCM ($0, 6, 12$ on the one hand
and $18$ on the other).


As a conclusion, affinity control provides users with a certain level
of control over thread placement, which is likely to yield faster tests when
threads are constrained to run on logical processors that are ``close'' one
to another.
The best results are obtained when SMT is effectively enforced.
However, affinity control is no panacea, and the memory system may
be stressed by other means, such as, for instance, allocating important
chunks of memory (option~\texttt{-s}).

\subsubsection{Advanced\label{affinity:advanced} control}
For specific experiments, the technique of
allocating logical processors sequentially by following a fixed increment
may be two rigid. \litmus{} offers a finer control on affinity by allowing
users to supply the logical processors sequence.
Notice that most users will probably not need this advanced feature.

Anyhow, so as to confirm that testing \afile{ppc-iriw-lwsync}
benefits from not crossing chip boundaries, one may wish to confine
its four threads to logical processors $16$ to~$19$,
that is to the first chip of the second MCM.
This can be done by overriding  the default logical processors sequence
by an user supplied one given as an argument to command-line
option~\texttt{-p}:
\begin{verbatim}
vargas% ./ppc-iriw-lwsync.exe -p 16,17,18,19 -i 1
Test ppc-iriw-lwsync Allowed
Histogram (16 states)
169420:>1:r1=0; 1:r3=0; 3:r1=0; 3:r3=0;
1287  :>1:r1=1; 1:r3=0; 3:r1=0; 3:r3=0;
17344 :>1:r1=0; 1:r3=1; 3:r1=0; 3:r3=0;
85329 :>1:r1=1; 1:r3=1; 3:r1=0; 3:r3=0;
1548  :>1:r1=0; 1:r3=0; 3:r1=1; 3:r3=0;
3     *>1:r1=1; 1:r3=0; 3:r1=1; 3:r3=0;
27014 :>1:r1=0; 1:r3=1; 3:r1=1; 3:r3=0;
75160 :>1:r1=1; 1:r3=1; 3:r1=1; 3:r3=0;
19828 :>1:r1=0; 1:r3=0; 3:r1=0; 3:r3=1;
29521 :>1:r1=1; 1:r3=0; 3:r1=0; 3:r3=1;
441   :>1:r1=0; 1:r3=1; 3:r1=0; 3:r3=1;
93878 :>1:r1=1; 1:r3=1; 3:r1=0; 3:r3=1;
81081 :>1:r1=0; 1:r3=0; 3:r1=1; 3:r3=1;
76701 :>1:r1=1; 1:r3=0; 3:r1=1; 3:r3=1;
93623 :>1:r1=0; 1:r3=1; 3:r1=1; 3:r3=1;
227822:>1:r1=1; 1:r3=1; 3:r1=1; 3:r3=1;
Ok

Witnesses
Positive: 3, Negative: 999997
Condition exists (1:r1=1 /\ 1:r3=0 /\ 3:r1=1 /\ 3:r3=0) is validated
Hash=4fbfaafa51f6784d699e9bdaf5ba047d
Observation ppc-iriw-lwsync Sometimes 3 999997
Time ppc-iriw-lwsync 0.63
\end{verbatim}
%$
Thus we get results similar to the previous experiment on logical processors
$0$ to~$3$ (option \texttt{-i 1} alone).

We may also run four simultaneous instances (\texttt{-n 4}, parameter~$n$ of
section~\ref{sec:arch}) of the test on
the four available MCM's:
\begin{verbatim}
vargas% ./ppc-iriw-lwsync.exe -p 0,1,2,3,16,17,18,19,32,33,34,35,48,49,50,51 -n 4 -i 1
Test ppc-iriw-lwsync Allowed
Histogram (16 states)
...
57    *>1:r1=1; 1:r3=0; 3:r1=1; 3:r3=0;
...
Ok

Witnesses
Positive: 57, Negative: 3999943
Condition exists (1:r1=1 /\ 1:r3=0 /\ 3:r1=1 /\ 3:r3=0) is validated
Hash=4fbfaafa51f6784d699e9bdaf5ba047d
Observation ppc-iriw-lwsync Sometimes 57 3999943
Time ppc-iriw-lwsync 0.75
\end{verbatim}
%$
Observe that, for a negligible penalty in running time, the number
of non-SC outcomes increases significantly.

By contrast, binding threads of a given instance of the test
to different MCM's results in poor running time and no non-SC outcome.
\begin{verbatim}
vargas% ./ppc-iriw-lwsync.exe -p 0,1,2,3,16,17,18,19,32,33,34,35,48,49,50,51 -n 4 -i 4
Test ppc-iriw-lwsync Allowed
Histogram (15 states)
...
Witnesses
Positive: 0, Negative: 4000000
Condition exists (0:r1=1 /\ 0:r3=0 /\ 2:r1=1 /\ 2:r3=0) is NOT validated
Time ppc-iriw-lwsync 1.48
\end{verbatim}
%$
In the experiment above, the increment is~$4$, hence the logical processors
allocated to the first
instance of the test are $0, 16, 32, 48$,
of which indices in the logical processors sequence are $0, 4, 8, 12$,
respectively.
The next allocated index in the sequence is~$12+4 = 16$.
However, the sequence has $16$ items.
Wrapping around yields index~$0$ which happens to be
the same as the starting index.
Then, so as to allocate fresh processors, the starting index is incremented
by one, resulting in allocating processors $1, 17, 33, 49$
(indices $1, 5, 9, 13$) to the second instance --- see section~\ref{incr:full}
for the full story.
Similarly, the third and fourth instances will get processors
$2, 18, 34, 50$ and $3, 19, 35, 51$, respectively.
Attentive readers may have noticed that the same experiment can
be performed with option \opt{-i 16} and no \opt{-p} option.

Finally, users should probably be aware that at least some versions of Linux
for x86 feature a less obvious mapping of logical processors to hardware.
On a bi-processor, dual-core, 2-ways hyper-threading, Linux,  AMD64 machine,
we have checked that logical processors residing on the same core
are $k$ and $k+4$, where $k$ is an arbitrary core number ranging from
$0$ to~$3$.
As a result, a proper choice for favouring effective hyper-threading
on such a machine is \opt{-i 4} (or \opt{-p 0,4,1,5,2,6,3,7 -i 1}).
More worthwhile noticing, perhaps,
the straightforward choice \opt{-i 1} disfavours effective hyper-threading\ldots

\subsubsection{Custom\label{affinity:custom} control}
Most tests run by \litmus{} are produced by the litmus test generators
described in Part~\ref{part:diy}.
Those tests include meta-information that may direct affinity control.
For instance we generate one test with the
\diyone{} tool, see Sec.~\ref{diyone:intro}.
More specifically we generate
\ahref{IRIW+lwsyncs.litmus}{\ltest{IRIW+lwsyncs}} for Power
(\ahref{ppc-iriw-lwsync.litmus}{\ltest{ppc-iriw-lwsync}} in the previous
section) as follows:
\begin{verbatim}
% diyone7 -arch PPC -name IRIW+lwsyncs Rfe LwSyncdRR Fre Rfe LwSyncdRR Fre
\end{verbatim}
We get the new source file~\afile{IRIW+lwsyncs.litmus}:
\verbatiminput{IRIW+lwsyncs.litmus}
The relevant meta-information  is the ``\texttt{Com}'' line that describes
how test threads are related --- for instance, thread~0 stores a value
to memory that is read by thread~1, written ``\texttt{Rf}'' (see Part~\ref{part:diy} for more details).
Custom affinity control will tend to run threads related by ``\texttt{Rf}''
on ``close'' logical processors, where we can for instance consider
that close logical processors belong to the same physical core (SMT for Power).
This minimal logical processor topology is described by two \litmus{}
command-line option: \texttt{-smt <n>} that specifies $n$-way SMT;
and \texttt{-smt\_mode (seq|end)} that specifies how logical processors
from the same core are numbered.
For a 8-cores 4-ways SMT power7 machine we invoke \litmus{} as follows:
\begin{verbatim}
% litmus7 -mem direct -smt 4 -smt_mode seq -affinity custom -o a.tar IRIW+lwsyncs.litmus
\end{verbatim}
Notice that memory mode is direct and that the number of available
logical processors is unspecified, resulting in running one instance of
the test. More importantly, notice that affinity control is enabled
\texttt{-affinity custom}, additionally specifying custom affinity mode.

We then upload the archive \texttt{a.tar} to our Power7 machine,
unpack, compile and run the test:
\begin{verbatim}
power7% tar xmf a.tar
power7% make
...
power7% ./IRIW+lwsyncs.exe -v
./IRIW+lwsyncs.exe  -v
IRIW+lwsyncs: n=1, r=1000, s=1000, +rm, +ca, p='0,1,2,3,4,5,6,7,8,9,10,11,12,13,14,15,16,17,18,19,20,21,22,23,24,25,26,27,28,29,30,31'
thread allocation:
[23,22,3,2] {5,5,0,0}
\end{verbatim}
Option \texttt{-v} instructs the executable to show settings of the test harness: we see that one instance of the test is run (\texttt{n=1}),
size parameters are reminded (\texttt{r=1000, s=1000}) and
shuffling of indirect memory mode is performed (\texttt{+rm}).
Affinity settings are also given: mode is custom (\texttt{+ca}) and
the logical processor sequence inferred is given (\texttt{-p 0,1,\ldots,31}).
Additionally, the allocation  of test threads to logical processors is
given, as \texttt{[\ldots]}, as well as the allocation  of test threads
to physical cores, as \texttt{\{\ldots\}}.

Here is the run output proper:
\begin{verbatim}
Test IRIW+lwsyncs Allowed
Histogram (15 states)
2700  :>1:r1=0; 1:r3=0; 3:r1=0; 3:r3=0;
142   :>1:r1=1; 1:r3=0; 3:r1=0; 3:r3=0;
37110 :>1:r1=0; 1:r3=1; 3:r1=0; 3:r3=0;
181257:>1:r1=1; 1:r3=1; 3:r1=0; 3:r3=0;
78    :>1:r1=0; 1:r3=0; 3:r1=1; 3:r3=0;
15    *>1:r1=1; 1:r3=0; 3:r1=1; 3:r3=0;
103459:>1:r1=0; 1:r3=1; 3:r1=1; 3:r3=0;
149486:>1:r1=1; 1:r3=1; 3:r1=1; 3:r3=0;
30820 :>1:r1=0; 1:r3=0; 3:r1=0; 3:r3=1;
9837  :>1:r1=1; 1:r3=0; 3:r1=0; 3:r3=1;
2399  :>1:r1=1; 1:r3=1; 3:r1=0; 3:r3=1;
204629:>1:r1=0; 1:r3=0; 3:r1=1; 3:r3=1;
214700:>1:r1=1; 1:r3=0; 3:r1=1; 3:r3=1;
5186  :>1:r1=0; 1:r3=1; 3:r1=1; 3:r3=1;
58182 :>1:r1=1; 1:r3=1; 3:r1=1; 3:r3=1;
Ok

Witnesses
Positive: 15, Negative: 999985
Condition exists (1:r1=1 /\ 1:r3=0 /\ 3:r1=1 /\ 3:r3=0) is validated
Hash=836eb3085132d3cb06973469a08098df
Com=Rf Fr Rf Fr
Orig=Rfe LwSyncdRR Fre Rfe LwSyncdRR Fre
Affinity=[2, 3] [0, 1] ; (1,2) (3,0)
Observation IRIW+lwsyncs Sometimes 15 999985
Time IRIW+lwsyncs 0.70
\end{verbatim}
As we see, the test validates. Namely we observe the non-SC behaviour of
\ltest{IRIW} in spite of the presence of two \texttt{lwsync} barriers.
We may also notice, in the executable output some meta-information related
to affinity: it reads that threads 2 and~3 on the one hand and threads 0
and~1 on the other are considered ``close'' (\emph{i.e.} will run on the
same physical core); while threads 1 and~2 on the one hand and threads 3 and~0
on the other are considered ``far'' (\emph{i.e.} will run on different cores).


Custom affinity can be disabled by enabling another affinity mode.
For instance with \texttt{-i 0} we specify an affinity increment of zero.
That is, affinity control is disabled altogether:
\begin{verbatim}
power7% ./IRIW+lwsyncs.exe -i 0 -v
./IRIW+lwsyncs.exe -i 0  -v
IRIW+lwsyncs: n=1, r=1000, s=1000, +rm, i=0, p='0,1,2,3,4,5,6,7,8,9,10,11,12,13,14,15,16,17,18,19,20,21,22,23,24,25,26,27,28,29,30,31'
Test IRIW+lwsyncs Allowed
Histogram (15 states)
...
No

Witnesses
Positive: 0, Negative: 1000000
Condition exists (1:r1=1 /\ 1:r3=0 /\ 3:r1=1 /\ 3:r3=0) is NOT validated
Hash=836eb3085132d3cb06973469a08098df
Com=Rf Fr Rf Fr
Orig=Rfe LwSyncdRR Fre Rfe LwSyncdRR Fre
Observation IRIW+lwsyncs Never 0 1000000
Time IRIW+lwsyncs 0.90
\end{verbatim}
As we see, the test does not validate under those conditions.

Notice that section~\ref{affinity:experiment} describes a complete experiment
on affinity control.

\subsection{Controlling\label{exec:control} executable files}

\paragraph{Test conditions}
Any executable file produced by \litmus{} accepts the following
command line options.
\begin{description}
\item[{\tt -v}] Be verbose, can be repeated to increase verbosity.
Specifying \opt{-v} is a convenient way to look at the default of options.
\item[{\tt -q}] Be quiet.
\item[{\tt -a <n>}] Run maximal number of tests concurrently for $n$~available
logical processors --- parameter~$a$ in Sec.~\ref{defa}.
Notice that if affinity control is enabled (see below), \opt{-a 0} will
set parameter~$a$ to the number of logical processors effectively available.
\item[{\tt -n <n>}] Run $n$~tests concurrently --- parameter~$n$
in Sec.~\ref{defn}.
\item[{\tt -r <n>}] Perform $n$ runs --- parameter~$r$ in Sec.~\ref{defr}.
\item[{\tt -fr <f>}] Multiply~$r$ by~$f$ ($f$~is a floating point number).
\item[{\tt -s <n>}] Size of a run --- parameter~$s$ in Sec.~\ref{defs}.
\item[{\tt -fs <f>}] Multiply~$s$ by~$f$.
\item[{\tt -f <f>}] Multiply~$s$ by~$f$ and divide $r$ by~$f$.
\end{description}
\label{generalized:integer}Notice that options \opt{-s} and~\opt{-r}
accept a generalised
syntax for their integer argument: when suffixed by~\opt{k} (resp.~\opt{M})
the integer gets multiplied by~$10^3$ (resp.~$10^6$).

\label{rm}The following options are accepted only for tests compiled
in indirect memory mode (see Sec.~\ref{defmemorymode}):
\begin{description}
\item[{\tt -rm}] Do not shuffle pointer arrays, resulting a behaviour
similar do direct mode, without recompilation.
\item[{\tt +rm}] Shuffle pointer arrays, provided for regularity.
\end{description}

\label{st}The following option is accepted only for tests compiled
with a specified stride value (see Sec.~\ref{defstride}).
\begin{description}
\item[{\tt -st <n>}] Change stride to~\opt{<n>}.
The default stride is specified at compile time by \litmus{}
option \opt{-stride}.
A value of zero or less is replaced by the number of threads in the test.
\end{description}


The following option is accepted when enabled at compile time:
\begin{description}
\item[{\tt -l <n>}]
Insert the assembly code of each thread in a loop of size \opt{<n>}.
\end{description}

\paragraph{\aname{affinity:runopt}{Affinity}}
If affinity control has been enabled at compilation time
(for instance, by supplying option \texttt{-affinity incr1}
to \litmus),
the  executable file produced by \litmus{} accepts the following
command line options.
\begin{description}
\item[{\tt -p <ns>}] Logical processors sequence.
The sequence \texttt{<ns>} is a comma separated list of integers,
The default sequence is inferred by the executable as $0,1,\ldots,A-1$,
where $A$ is the number of logical processors featured by the tested machine;
or is a sequence specified at compile time
with \litmus{} option~\texttt{-p}.
\item[{\tt -i <n>}] Increment for allocating logical processors to threads.
Default is specified at compile time by \litmus{} option~\texttt{-affinity
incr<\textit{n}>}.
Notice that \texttt{-i 0} disable affinity
and that \texttt{.exe} files reject the \texttt{-i} option when affinity
control has not been enabled at compile~time.
\item[{\tt +ra}] Perform random allocation of affinity at each test round.
\item[{\tt +ca}] Perform custom affinity.
\end{description}
Notice that when custom affinity is not available, would it be that
the test source lacked meta-information or that logical processor
topology was not specified at compile-time, then \texttt{+ca}
behaves as \texttt{+ra}.

\label{incr:full}Logical processors are allocated
test instance by test instance
(parameter~$n$ of Sec.~\ref{sec:arch}) and then
thread by thread, scanning the logical processor sequence
left-to-right by steps of the given increment.
More precisely, assume a logical processor sequence
$P = p_0, p_1, \ldots, p_{A-1}$ and an increment~$i$.
The first processor allocated is $p_0$, then $p_i$, then $p_{2i}$ etc,
Indices in the sequence~$P$ are reduced modulo $A$ so as to wrap around.
The starting index of the allocation sequence (initially $0$) is recorded,
and coincidence with the index of the next processor to be
allocated is checked.
When coincidence occurs, a new index is computed, as the previous
starting index plus one, which also becomes the new starting index.
Allocation then proceeds from this new starting index.
That way, all the processors in the sequence
will get allocated to different threads naturally, provided of
course that less than $A$~threads are scheduled to run.
See Sec.~\ref{affinity:advanced} for an example with $A=16$ and~$i=4$.

\section{Advanced\label{advanced:control} control of test parameters}

\subsection{Timebase \label{timebase}synchronisation mode}
Timebase synchronisation of the testing loop iterations
(see Sec.~\ref{timebase:intro}) is selected by \litmus{} command line option
\opt{-barrier timebase}.
In that mode,
test threads will first synchronise using polling synchronisation
barrier code, agree on a target timebase value and then loop
reading the timebase until it exceeds the target value.
Some tests demonstrate that timebase synchronisation
is more precise than user synchronisation (\opt{-barrier user} and default).

For instance, consider the x86 test \atest{6.SB},
a 6-thread analog of the \ahrefloc{SB}{\ltest{SB}}~test:
\verbatiminput{6.SB.litmus}
As for \atest{SB}, the final condition of
\atest{6.SB} identifies executions where each thread loads the initial
value~$0$ of a location that is writtent into by another thread.
\begin{center}\img{6.SB}\end{center}

We first compile the test in user synchronisation mode, saving
\litmus{} output files into the directory~\file{R}:
\begin{verbatim}
% mkdir -p R
% litmus7 -barrier user -vb true -o R 6.SB.litmus
% cd R
% make
\end{verbatim}
The additional command line option \opt{-vb true} activates the printing
of some timing information on synchronisations.

We then directly run the test executable \file{6.SB.exe}:
\begin{verbatim}
% ./6.SB.exe
Test 6.SB Allowed
Histogram (62 states)
7569  :>0:EAX=1; 1:EAX=0; 2:EAX=0; 3:EAX=0; 4:EAX=0; 5:EAX=0;
8672  :>0:EAX=0; 1:EAX=1; 2:EAX=0; 3:EAX=0; 4:EAX=0; 5:EAX=0;
...
326   :>0:EAX=1; 1:EAX=0; 2:EAX=1; 3:EAX=1; 4:EAX=1; 5:EAX=1;
907   :>0:EAX=0; 1:EAX=1; 2:EAX=1; 3:EAX=1; 4:EAX=1; 5:EAX=1;
No

Witnesses
Positive: 0, Negative: 1000000
Condition exists (0:EAX=0 /\ 1:EAX=0 /\ 2:EAX=0 /\ 3:EAX=0 /\ 4:EAX=0 /\ 5:EAX=0) is NOT validated
Hash=107f1303932972b3abace3ee4027408e
Observation 6.SB Never 0 1000000
Time 6.SB 0.85
\end{verbatim}
The targeted outcome --- reading zero in the \verb+EAX+ registers
of the $6$ threads --- is not observed.
We can observe synchronisation times for all tests runs
with the executable  command line option \opt{+vb}:
\begin{verbatim}
% ./6.SB.exe  +vb
99999: 162768 420978 564546   -894 669468
99998:    -93      3     81   -174   -651
99997:   -975    -30    -33     93   -192
99996:    990   1098    852   1176    774
...
\end{verbatim}
We see five columns of numbers that list, for each test run,
the starting delays of \P{1}, \P{2} etc. with respect to \P{0}, expressed
in timebase ticks. Obviously, synchronisation is rather loose,
there are always two threads whose starting delays differ of
about $1000$~ticks.

We now compile the same test in timebase synchronisation mode,
saving \litmus{} output files into the pre-existing directory~\file{RT}:
\begin{verbatim}
% mkdir -p RT
% litmus7 -barrier timebase -vb true -o RT 6.SB.litmus
% cd RT
% make
\end{verbatim}
And we run the test directly
(option \opt{-vb} disable the printing of any
synchronisation timing information):
\begin{verbatim}
% ./6.SB.exe -vb
Test 6.SB Allowed
Histogram (64 states)
60922 *>0:EAX=0; 1:EAX=0; 2:EAX=0; 3:EAX=0; 4:EAX=0; 5:EAX=0;
38299 :>0:EAX=1; 1:EAX=0; 2:EAX=0; 3:EAX=0; 4:EAX=0; 5:EAX=0;
...
598   :>0:EAX=0; 1:EAX=1; 2:EAX=1; 3:EAX=1; 4:EAX=1; 5:EAX=1;
142   :>0:EAX=1; 1:EAX=1; 2:EAX=1; 3:EAX=1; 4:EAX=1; 5:EAX=1;
Ok

Witnesses
Positive: 60922, Negative: 939078
Condition exists (0:EAX=0 /\ 1:EAX=0 /\ 2:EAX=0 /\ 3:EAX=0 /\ 4:EAX=0 /\ 5:EAX=0) is validated
Hash=107f1303932972b3abace3ee4027408e
Observation 6.SB Sometimes 60922 939078
Time 6.SB 1.62
\end{verbatim}
We now see that the test validates. Moreover
all of the $64$~possible outcomes are observed.

Timebase synchronisation works as follows: at every iteration,
\begin{enumerate}
\item one of the threads reads timebase~$T$;
\item all threads synchronise by the means of a polling synchronisation
barrier;
\item each thread computes $T_i = T + \delta_i$, where $\delta_i$ is
\emph{the timebase delay}, a thread
specific constant;
\item each thread loops, reading the timebase until the read value exceeds
$T_i$.
\end{enumerate}
By default the timebase delay $\delta_i$ is $2^{11} = 2048$ for all threads.

The precision of timebase synchronisation can be illustrated
by enabling the printing of all synchronisation timings:
\begin{verbatim}
% ./6.SB.exe +vb
99999: 672294[1] 671973[1] 672375[1] 672144[1] 672303[1] 672222[1]
99998:  4524[1]  4332[1]  4446[1]  2052[65]  2064[73]  4095[1]
...
99983:  4314[1]  3036[1]  3141[1]  2769[1]  4551[1]  3243[1]
99982:* 2061[36]  2064[33]  2067[11]  2079[12]  2064[14]  2064[24]
99981:  2121[1]  2382[1]  2586[1]  2643[1]  2502[1]  2592[1]
...
\end{verbatim}
For each test iteration and each thread, two numbers are shown (1) the last
timebase value read by and~(2) (in brackets \verb+[+\ldots\verb+]+) how many iterations of loop~4. were performed.
Additionally a star ``\verb+*+'' indicates the occurrence
of the targeted outcome.
Here, we see that a nearly perfect synchronisation can be achieved
(cf. line \verb+99982:+ above).


Once timebase synchronisation have been selected
(\litmus{} option \opt{-barrier timebase}),
test executable behaviour can be altered by the following
two command line options:
\begin{description}
\item[{\tt -ta <n>}]
Change the timebase delay~$\delta_i$ of all threads.
\item[{\tt -tb <0:n$_0$;1:$n_1$;$\cdots$>}]
Change the timebase delay~$\delta_i$ of individual threads.
\end{description}

The \litmus{} command line option~\opt{-vb true}
(verbose barrier)
governs the printing of synchronisation timings. It comes handy when
choosing values for the \opt{-ta} and \opt{-tb} options.
When set, the executable show synchronisation timings
for outcomes that validate the test final condition.
This default behaviour can be altered with
the following two command line options:
\begin{description}
\item[{\tt -vb}] Do not show synchronisation timings.
\item[{\tt +vb}] Show synchronisation timings for all outcomes.
\end{description}
Synchronisation timings are expressed in timebase ticks.
The format depends on the synchronisation mode
(\litmus{} option \opt{-barrier}).
This section just gave two examples for user mode
(timings are show as differences from thread~\P{0}); and for
timebase mode (timings are shown as differences
from a commonly agreed by all thread timebase value).
Notice that, when affinity control is enabled,
the running logical processors of threads are also shown.

\subsection{Advanced \label{preload:custom}prefetch control}
Supplying the tags
\opt{custom}, \opt{static}, \opt{static1} or~\opt{static2} to
\litmus{} command line option \opt{-preload}
commands the insertion of cache prefetch or flush
instructions before every test instance.

In custom mode the execution of such cache management instruction
is under total user control, the other, ``static'', modes offer
less control to the user, for the sake of not altering test code proper.

\subsubsection{Custom prefetch}
Custom prefetch mode offers complete control over
cache management instructions.
Users enable this mode by supplying the command line option
\opt{-preload custom} to \litmus{}. For instance one may compile
the x86 test \afile{6.SB.litmus} as follows:
\begin{verbatim}
% mkdir -p R
% litmus7 -mem indirect -preload custom -o R 6.SB.litmus
% cd R
% make
\end{verbatim}
Notice the test is compiled in \ahrefloc{defmemorymode}{indirect memory mode},
in order to reduce false sharing effects.

The executable \texttt{6.SB.exe} accepts two new command line options:
\opt{-prf} and~\opt{-pra}. Those options takes arguments that describe
cache management instructions.
The option \opt{-pra} takes one letter that stands for a cache
management instruction as we here describe:
\begin{center}
\texttt{I}: do nothing,
\texttt{F}: cache flush,
\texttt{T}: cache touch,
\texttt{W}: cache touch for a write.
\end{center}
All those cache management instructions are not provided by all architectures,
in case some instruction is missing, the letters behave as follows:
\begin{center}
\texttt{F}: do nothing,
\texttt{T}: do nothing,
\texttt{W}: behave as \texttt{T}.
\end{center}

With \opt{-pra $X$} the commanded action applies to all
threads and all variables, for instance:
\begin{verbatim}
% ./6.SB.exe -pra T
\end{verbatim}
will perform a run where every test thread touches the test locations
that it refers to
(\emph{i.e.} \texttt{x} and~\texttt{y} for Thread~0, \texttt{y}
and \texttt{z} for Thread~1, etc.)
before executing test code proper.
Although one may achieve interesting results by using
this \opt{-pra} option, the more selective
\opt{-prf} option should prove more useful.
The \opt{-prf} option takes a comma separated list of cache managment
directives.
A cache management directive is $n$:\textit{loc}=$X$,
where $n$ is a thread number, \textit{loc} is a program variable,
and $X$ is a cache management controle letter.
For instance, \texttt{-prf 0:y=T} instructs
thread~0 to touch location~\texttt{y}.
More generally, having each thread of the test
\ahref{6.SB.litmus}{\ltest{6.SB}} to touch the memory location
it reads with its second instruction would favor reading the initial value
of these locations,
and thus validating
the final condition of the test
``\verb+(0:EAX=0 /\ 1:EAX=0 /\ 2:EAX=0 /\ 3:EAX=0 /\ 4:EAX=0 /\ 5:EAX=0)+''.

Notice that those locations can be found by looking
at the \ahref{6.SB.litmus}{test code}
or at the \ahref{6.SB.png}{diagram} of the target execution.
Let us have a try:
\begin{verbatim}
./6.SB.exe -prf 0:y=T,1:z=T,2:a=T,3:b=T,4:c=T,5:x=T
Test 6.SB Allowed
Histogram (63 states)
10    *>0:EAX=0; 1:EAX=0; 2:EAX=0; 3:EAX=0; 4:EAX=0; 5:EAX=0;
...
Witnesses
Positive: 10, Negative: 999990
...
Prefetch=0:y=T,1:z=T,2:a=T,3:b=T,4:c=T,5:x=T
...
\end{verbatim}
As can be seen, the final condition is validated. Also notice
that the prefetch directives used during the run are reminded.
If given several times, \opt{-prf} options cumulate,
the rightmost directives taking precedence in case of ambiguity.
As a consequence, one may achieve the same prefetching
effect as above with:
\begin{verbatim}
% ./6.SB.exe -prf 0:y=T -prf 1:z=T -prf 2:a=T -prf 3:b=T -prf 4:c=T -prf 5:x=T
\end{verbatim}


\subsubsection{Prefetch metadata}
The source code of tests may include prefetch directives as metadata
prefixed with ``\verb+Prefetch=+''.
In particular, the generators of the \diy{}
suite  (see Part~\ref{part:diy}) produce such metadata.
For instance in the case of the
\ltest{6.SB} test (generated source \afile{6.SB+Prefetch.litmus}),
this metadata reads:
\begin{verbatim}
Prefetch=0:x=F,0:y=T,1:y=F,1:z=T,2:z=F,2:a=T,3:a=F,3:b=T,4:b=F,4:c=T,5:c=F,5:x=T
\end{verbatim}
That is, each thread flushes the location it stores to and touches
each location it reads from.
Notice that each thread starts with a memory location access
(here a store) and ends with another (here a load).
The idea simply is to accelerate the exit access (with a cache touch)
while delaying the entry access (with a cache flush).


When prefetch metadata is available, it acts as the default of
prefetch directives:
\begin{verbatim}
% litmus7 -mem indirect -preload custom -o R 6.SB+Prefetch.litmus
% cd R
% make
\end{verbatim}
Then we run the test by:
\begin{verbatim}
% ./6.SB+Prefetch.exe
Test 6.SB Allowed
Histogram (63 states)
674   *>0:EAX=0; 1:EAX=0; 2:EAX=0; 3:EAX=0; 4:EAX=0; 5:EAX=0;
...
Witnesses
Positive: 674, Negative: 999326
...
Prefetch=0:x=F,0:y=T,1:y=F,1:z=T,2:a=T,2:z=F,3:a=F,3:b=T,4:b=F,4:c=T,5:c=F,5:x=T
...
\end{verbatim}
One may notice that the prefetch directives from the source file
medata found its way to the test executable.

As with any kind of metadata,
one can change the prefetch metadata by editing the litmus source file,
or better by using the \opt{-hints} command line option.
The  \opt{-hints} command line option takes a filename as  argument.
This file is a \ahrefloc{defmapping}{mapping} that associates
new metadata to test names.
As an example, we reverse \diy{} scheme for cache management directives:
accelerating entry accesses and delaying exit accesses:
\begin{verbatim}
% cat map.txt
6.SB Prefetch=0:x=W,0:y=F,1:y=W,1:z=F,2:a=F,2:z=W,3:a=W,3:b=F,4:b=W,4:c=F,5:c=W,5:x=F
% litmus7 -mem indirect -preload custom -hints map.txt -o R 6.SB.litmus
% cd R
% make
...
% ./6.SB.exe
Test 6.SB Allowed
Histogram (63 states)
24    *>0:EAX=0; 1:EAX=0; 2:EAX=0; 3:EAX=0; 4:EAX=0; 5:EAX=0;
...
Prefetch=0:x=W,0:y=F,1:y=W,1:z=F,2:a=F,2:z=W,3:a=W,3:b=F,4:b=W,4:c=F,5:c=W,5:x=F
...
\end{verbatim}
As we see above, the final condition validates.
It does so in spite of
the apparently unfavourable cache management directives.

We can experiment further without recompilation, by
using the \opt{-pra} and~\opt{-prf} command line options of
the test executable. Those are parsed left-to-right, so that we can
(1) cancel any default cache management directive with
\opt{-pra I} and~(2) enable cache touch for the stores:
\begin{verbatim}
% ./6.SB.exe -pra I -prf 0:x=W -prf 1:y=W -prf 2:z=W -prf 3:a=W -prf 4:b=W -prf 5:c=W
Test 6.SB Allowed
...
Witnesses
Positive: 0, Negative: 1000000
...
Prefetch=0:x=W,1:y=W,2:z=W,3:a=W,4:b=W,5:c=W
\end{verbatim}
As we see, the final condition does not validate.

By contrast, flushing or touching the locations that the threads load permit to
repetitively achieve validation:
\begin{verbatim}
chi% ./6.SB.exe -pra I -prf 0:y=F -prf 1:z=F -prf 2:a=F -prf 3:b=F -prf 4:c=F -prf 5:x=F
Test 6.SB Allowed
Histogram (63 states)
211   *>0:EAX=0; 1:EAX=0; 2:EAX=0; 3:EAX=0; 4:EAX=0; 5:EAX=0;
...
% ./6.SB.exe -pra I -prf 0:y=T -prf 1:z=T -prf 2:a=T -prf 3:b=T -prf 4:c=T -prf 5:x=T
Test 6.SB Allowed
Histogram (63 states)
10    *>0:EAX=0; 1:EAX=0; 2:EAX=0; 3:EAX=0; 4:EAX=0; 5:EAX=0;
...
\end{verbatim}
As a conclusion, interpreting the impact of cache management
directives is not easy. However, custom preload mode
(litmus command line option \opt{-preload custom}) and test executable
options \opt{-pra} and~\opt{-prf} allow experimentation on specific tests.


\subsubsection{``Static'' prefetch control}
Custom prefetch mode comes handy when one wants to tailor cache management
directives for a particular test.
In practice, we run batches of tests using source metadata for
prefetch directives.
In such a setting, the code that interprets the
prefetch directives is useless,
as we do not use the \opt{-prf} option of the test executables.
As this code get executed before each test thread code, it may impact test
results.
It is desirable to supress this code from test executables, still
performing cache management instructions.
To that aim, \litmus{} provides some ``static'' preload modes, enabled
with command line options \opt{-preload static},
\opt{-preload static1} and~\opt{-preload static2}.

In the former mode  \opt{-preload static} and without any further user
intervention, each test thread executes the cache management
instructions commanded by the \verb+Prefetch+ metadata:
\begin{verbatim}
% mkdir -p S
% litmus7 -mem indirect -preload static -o R 6.SB+Prefetch.litmus
% make -C S
% S/6.SB+Prefetch.exe
Test 6.SB Allowed
Histogram (63 states)
804   *>0:EAX=0; 1:EAX=0; 2:EAX=0; 3:EAX=0; 4:EAX=0; 5:EAX=0;
...
Observation 804 999196
...
\end{verbatim}
As we can see above, the effect of the  cache management
instructions looks more favorable than in custom preload mode.

Users still have a limited control on the
execution of cache management instructions: produced executable accept
a new \opt{-prs <n>} option, which take a positive or null integer as argument.
Then, each test thread executes the cache
management instructions commanded by source metadata with probability~$1/n$,
the special value $n=0$ disabling prefetch altogether.
The default for the \opt{-prs} options is ``\opt{1}'' (always execute
the cache management instructions).
Let us try:
\begin{verbatim}
% S/6.SB+Prefetch.exe -prs 0 | grep Observation
Observation 6.SB Never 0 1000000
% S/6.SB+Prefetch.exe -prs 1 | grep Observation
Observation 6.SB Sometimes 901 999099
% S/6.SB+Prefetch.exe -prs 2 | grep Observation
Observation 6.SB Sometimes 29 999971
% S/6.SB+Prefetch.exe -prs 3 | grep Observation
Observation 6.SB Sometimes 16 999984
\end{verbatim}
In those experiments we show the ``\verb+Observation+'' field of \litmus{}
output: this field gives the count of outcomes that validate the final
condition, followed by the count of outcomes that do not validate
the final condition. The above counts confirm that
cache management instructions favor validation.

The remaining preload modes \opt{static1} and~\opt{static2} are similar,
except that they produce executable files that do not accept
the \opt{-prs} option. Furthermore, in the former
mode \opt{-preload static1} cache management instructions are always executed,
while in the latter mode \opt{-preload static2}
cache management instructions are  executed with probability~$1/2$.
Those modes thus act as pure static mode
(\litmus{} option \opt{-preload static}),
with runtime options \opt{-prs 1} and~\opt{-prs 2} respectively.
Moreover,
as the test scaffold includes no code to interpret the \opt{-prs <n>}
switch, the test code is less perturbed.
In practice and for the \ltest{6.SB} example, there is little difference:
\begin{verbatim}
% mkdir -p S1 S2
% litmus7 -mem indirect -preload static1 -o S1 6.SB+Prefetch.litmus
% litmus7 -mem indirect -preload static2 -o S2 6.SB+Prefetch.litmus
% make -C S1 && make -C S2
...
% S1/6.SB+Prefetch.exe | grep Observation
Observation 6.SB Sometimes 1119 998881
% S2/6.SB+Prefetch.exe | grep Observation
Observation 6.SB Sometimes 16 999984
\end{verbatim}

\section{Usage of \litmus{}}

\subsection*{Arguments\label{file:argument}}
\litmus{} takes file names as command line arguments.
Those files are either a single litmus test,
when having extension \file{.litmus},
or a list of file names, when prefixed by \file{@}.
Of course, the file names in \file{@}files can themselves be
\file{@}files.

\subsection*{Options}
There are many command line options.
We describe the more useful ones:

\paragraph*{General \label{general:bahaviour}behaviour}
\begin{description}
\item[{\tt -version}] Show version number and exit.
\item[{\tt -libdir}] Show installation directory and exit.
\item[{\tt -v}] Be verbose, can be repeated to increase verbosity.
\item[{\tt -mach <name>}] Read configuration file~\file{name.cfg}.
See the \ahrefloc{config:sec}{next section}
for the syntax of configuration files.
\item[{\tt -o <dest>}]
Save C-source of test files into \opt{<dest>} instead of running them.
If  argument \opt{<dest>} is an archive (extension \opt{.tar})
or a compressed archive (extension \opt{.tgz}),
\litmus{} builds an archive: this is the ``cross~compilation feature''
demonstrated in Sec.~\ref{cross}.
Otherwise, \opt{<dest>} is interpreted as the name of an
existing directory and tests are saved in it.
\item[{\tt -mode (std|presi|kvm)}]
Style for output C files, default is \texttt{std}.
Standard mode (\texttt{std}) is the operating mode described in this manual.
Other modes are experimental. Mode \texttt{presi} (for ``pre-silicium``) is an attempt not to rely on pre-existing operating system calls and libraries, while mode \texttt{kvm} is for running system level tests on top of the
\ahref{https://www.linux-kvm.org/page/KVM-unit-tests}{kvm-unit-tests}
infrastructure. See section~\ref{sec:nostd}.
\item[{\tt \aname{driver}{-driver (shell|C|XCode)}}]
Choose the driver that will run the tests.
In the ``\opt{shell}'' (and default) mode,
each test will be compiled into an executable. A dedicated shell script
\file{run.sh} will launch the test executables.
In the ``\opt{C}'' mode, one executable \file{run.exe} is produced, which
will launch the tests.
%(see Sec.~\ref{driverc:example} for an example).
Finally, the \opt{XCode} mode is for inclusion of the tests into
a dedicated iOS App, which we do not distribute at the moment.
\item[{\tt -crossrun <(user@)?host(:port)?|adb|qemu(:exec)?>}]
When the shell driver is used (\opt{-driver shell} above),
the first two possible arguments \aname{crossrun}{instruct}
the \texttt{run.sh} script to run individual tests on a remote machine.
The remote host can be contacted by the means of \texttt{ssh}
or the Android Debug Bridge.
\begin{description}
\item[\tt ssh]
\opt{user} is a login name on the the remote host,
\opt{<host>} is the name of the remote host,
and \texttt{port} is a port-number which can be omitted when standard~(22).
\item[\tt adb]
Tests will be run in the remote directory \opt{/data/tmp}.
\item[\tt qemu] Tests will be run with a simulator, whose executable name
can be optionally specified after ``\texttt{:}''.
By default, the executable name is \texttt{qemu}.
\end{description}
This option may be useful when the tested machine has little disk space or a
crippled installation.
Default is disabled --- \emph{i.e.} run tests on
the machine where the \file{run.sh} script runs.

The argument \opt{qemu} instruct the shell script to use the \texttt{qemu}
emulator. The \texttt{QEMU} environment variable must be defined to the
emulator path before running the \texttt{run.sh} script, as in (\texttt{sh} syntax): \verb+QEMU=qemu sh run.sh+.


\item[{\tt -index <@name>}] Save the source names of compiled files in index
file~\file{@name}.
\item[\aname{opt:hexa}{{\tt -hexa}}] Hexadecimal output, default is decimal.
\item[\aname{opt:stdio}{{\tt -stdio (true|false)}}] Command whether
using or not using the \texttt{printf} function in C~output
files. Default is \opt{true}.
\end{description}

\paragraph*{Test conditions\label{litmus:option:sec}}
The following options set the default values of
the options of the executable files produced:
\begin{description}
\item[\aname{avail}{{\tt -a <n>}}]Run maximal number of tests concurrently for $n$~available
logical processors ---
set default value for \opt{-a} of Sec.~\ref{exec:control}.
Default is special and will command running exaclty one test instance.
When affinity control is enabled, the value~$0$ has the special
meaning of having executables to set the
number of available logical processors according to
how many are actually present.
\item[{\tt -limit <bool>}] Do not process tests with more than~$n$
threads, where $n$ is the number of available cores defined above.
Default is \opt{true}.  Does not apply when option \texttt{-a} is absent.
\item[{\tt -r <n>}] Perform $n$ runs --- set default value for
option~\opt{-r} of Sec.~\ref{exec:control}.
The option accepts generalised syntax for integers and
default is~$10$.
\item[\aname{sizeoftest}{{\tt -s <n>}}]Size of a run --- set default value for
option \opt{-s} of Sec.~\ref{exec:control}.
The option accepts generalised syntax for integers and
default is $100000$ (or \opt{100k}).
\end{description}


The following additional options control the various modes described
in Sec.~\ref{sec:arch}, and more.
Those cannot be changed without running \litmus{} again:
\begin{description}
\item[{\tt -barrier (user|userfence|pthread|none|timebase)}] Set synchronisation mode, default \opt{user}. Synchronisation modes are described
in Sec.~\ref{defsynchronisation}
\item[{\tt -launch (changing|fixed)}] Set launch mode,
default \opt{changing}.
\item[{\tt -mem (indirect|direct)}] Set memory mode,
default \opt{indirect}.
It is possible to instruct executables compiled in indirect mode
to behave almost as if compiled in direct mode, see Sec.~\ref{rm}.
\item[{\tt -stride (none|adapt|<n>)}]
Specify a stride value of \opt{<n>} --- set default value for option \opt{-st} of Sec.~\ref{st}. See Sec.~\ref{defstride} for details on the stride parameter.
The default is \opt{none}\footnote{For backward compatibility, \opt{<n>} being negative or zero is equivalent to \opt{none}.}
--- \emph{i.e.} tests have no
\opt{-st} option. The special tag \opt{adapt} sets the default
stride to be the number of threads in test.

\item[{\tt -st <n>}] Alias for \opt{-stride <n>}.
\item[{\tt -para (self|shell)}]
Perform several tests concurrently, either by forking POSIX
threads (as described in Sec.~\ref{sec:arch}), or by forking
Unix processes. Only applies for cross compilation.
Default is \opt{self}.
\item[\aname{opt:alloc}{{\tt -alloc (dynamic|static|before)}}]
Set memory allocation mode. In ``dynamic'' and ``before'' modes, the memory
used by test threads is allocated with \texttt{malloc} --- in ``before'' mode,
memory is allocated before forking test instances.
In ``static'' mode, the memory is pre-allocated as static arrays.
In that latter case, the size of allocated arrays depend upon
compile time defined parameters: the number of available logical processors
(see option \ahrefloc{avail}{\opt{-a <n>}})
and the size of a run (see option \ahrefloc{sizeoftest}{\opt{-s <n>}}).
It remains possible to change those at execution time, provided
the resulting memory size does not exceed the compile time value.
Default is \texttt{dynamic} most often and
static for options \texttt{-mode presi/kvm -driver shell}.

\item[{\tt -preload (no|random|custom|static|static1|static2)}]
Specify preload mode (see Sec.~\ref{defpreload}), default is \texttt{random}.
Starting from version 5.0 we provide additional ``custom'' and ``static''
modes for a finer control of prefetching and flushing of some memory locations
by some threads. See Sec~\ref{preload:custom}.
\item[{\tt -safer (no|all|write)}] Specify safer mode,
default is \texttt{write}.
When instructed to do so, executable files perform some consistency checks.
Those are intended both for debugging and for dynamically checking
some assumptions on POSIX threads that we rely upon.
More specifically the test harness checks for the stabilisation of
memory locations after a test round in the ``\texttt{all}'' and
``\texttt{write}'' mode, while
the initial values of memory locations are checked in ``\texttt{all}'' mode.
\item[{\tt -speedcheck (no|some|all)}]
Quick condition check mode,
default is ``\texttt{no}''.
In mode ``\texttt{some}'', test executable will stop as soon as its
condition is settled.
In mode ``\texttt{all}'', the \texttt{run.sh} script will additionally
not run the test if invoked once more later.
\end{description}

\aname{affinity:control}{The} following option commands affinity control:
\begin{description}
\item[{\tt -affinity (none|incr<n>|random|custom)}]
Enable (or disable with tag \opt{none}) affinity control,
specifying default affinity mode of executables.
Default is \opt{none}, \emph{i.e.} executables do not
include affinity control code.
The various tags are interpreted as follows:
\begin{enumerate}
\item \opt{incr<n>}:
integer \opt{<n>} is the increment  for allocating logical
processors to threads --- see Sec.~\ref{sec:affinity}.
Notice that with \opt{-affinity incr0}
the produced code features affinity control, which executable
files do not exercise by default.
\item \opt{random}: executables perform random allocation of
test threads to logical processors.
\item \opt{custom}: executables perform custom allocation of
test threads to logical processors.
\end{enumerate}
Notice that the default for executables can be overridden using
options \opt{-i},\opt{+ra} and~\opt{+ca}
of Sec.~\ref{exec:control}.

\item[{\tt -i <n>}] Alias for \opt{-affinity incr<n>}.
\end{description}
Notice that affinity control is not implemented for MacOs.

The following options are significant when affinity control is enabled.
Otherwise they are silent no-ops.
\begin{description}
\item[{\tt -p <ns>}]
Specify the sequence of logical processors.
The notation \opt{<ns>} stands for a comma separated list of integers.
Set default value for option \opt{-p} of Sec.~\ref{exec:control}.
Default for this \opt{-p} option
will let executable files compute the logical processor sequence
themselves.
\item[{\tt -force\_afffinity <bool>}]
Code that sets affinity will spin until all specified
cores (as given with option \opt{-avail <n>}) processors
are up. This option is necessary on devices that let core sleep
when the computing load is low. Default is false.
\end{description}

\aname{desc:custom}{Custom} affinity control (see Sec.~\ref{affinity:custom}) is enabled,
first by enabling affinity control (\emph{e.g.} with \opt{-affinity \ldots}),
and then by specifying a logical processor topology with options \texttt{-smt}
and \texttt{-smt\_mode}.
\begin{description}
\item[{\tt -smt <n>}] Specify that logical processors are close by groups
of $n$, default is \texttt{1}.
\item[{\tt -smt\_mode (none|seq|end)}] Specify how ``close'' logical processors
are numbered, default is \texttt{none}.
In mode~``\texttt{end}'', logical processors of the same core
are numbered as $c$, $c+A_c$ etc. where $c$ is a physical core number and
$A_c$ is the number of physical cores available.
In mode~``\texttt{seq}'', logical processors of the same core
are numbered in sequence.
\end{description}
Notice that custom affinity works only for those tests that include the proper
meta-information. Otherwise, custom affinity silently degrades
to random affinity.

Finally, a few miscellaneous options are documented:
\begin{description}
\item[{\tt -l <n>}]
Insert the assembly code of each thread in test in a loop of size \opt{<n>}.
Accepts generalised integer syntax, disabled by default.
Sets default value for option \opt{-l} of Sec.~\ref{exec:control}.

This feature may prove useful for measuring running times that are not
too much perturbed by the test harness, in combination
with options~\opt{-s 1 -r 1}.
\item[{\tt -vb <bool>}]
Disable/enable the printing of synchronisation timings, default is \opt{false}.

This feature may prove useful for analysing the synchronisation behaviour of
a specific test, see Sec.~\ref{timebase}.
\item[{\tt -ccopts <flags>}] Set \prog{gcc} compilation flags
(defaults: X86=\opt{"-fomit-frame-pointer -O2"}, PPC/ARM=\opt{"-O2"}).
\item[\aname{gcc}{{\tt -gcc <name>}}]
Change the name of C compiler, default \opt{gcc}.
\item[\aname{linkopt}{{\tt -linkopt <flags>}}] Set \prog{gcc} linking flags.
(default: void).
\item[\aname{gas}{{\tt -gas <bool>}}]
Emit Gnu as extensions  (default Linux/Mac=\opt{true}, AIX=\opt{false})
\end{description}

\paragraph*{Target architecture description}
Litmus compilation chain may slightly vary depending on the following
parameters:
\begin{description}
\item[{\tt -os (linux|mac|aix)}] Set target operating system.
This parameter mostly impacts some of \prog{gcc} options. Default \opt{linux}.
\item[{\tt -ws (w32|w64)}] Set word size.
This option first selects \prog{gcc} $32$ or~$64$ bits mode,
by providing it with the appropriate option (\texttt{-m32}
or \texttt{-m64} on linux, \texttt{-maix32}
or \texttt{-maix64} on AIX).
It also slightly impacts code generation in the corner case
where memory locations hold other memory locations.
Default is a bit contrived: it acts as \opt{w32} as regards code
generation, while it provides no $32$/$64$ bits mode selection option
to~\texttt{gcc}.
\item[\tt -c11 <bool>] Enable the C11 standard, default \texttt{false}.
\end{description}

\paragraph*{Change\label{change:input} input}
Some items in
the source of tests can be changed at the very last moment.
\aname{defmapping}{The}
new items are defined in mapping files whose names are arguments to
the appropriate command line options.
Mapping files simply are lists of pairs, with one line starting with a test
name, and the rest of line defining the changed item.
The changed item may also contain several lines: in that case it should be
included in double quotes~``\texttt{"}.''.
\begin{description}
\item[{\tt -names <file>}] Run \litmus{} only on tests whose names are
listed in \texttt{<file>}.
\item[{\tt -excl <file>}] Do not run \litmus{} on tests whose names are
listed in \texttt{<file>}.
\item[{\tt -rename <file>}] Change test names.
\item[{\tt -kinds <file>}] Change test kinds.
This amounts to changing the quantifier of final conditions, with
kind \texttt{Allow} being \verb+exists+,
kind \texttt{Forbid} being \verb+~exists+
and kind \texttt{Require} being \verb+forall+.
\item[{\tt -conds <file>}] Change the final condition of tests.
\item[{\tt -hints <file>}] Change meta-data, or hints.
Hints command advanced features such as custom affinity
(option \opt{-affinity custom} and Sec.~\ref{affinity:custom})
and prefech control
(option \opt{-preload custom} and Sec.~\ref{preload:custom}).

\end{description}
Observe that the rename mapping is applied first. As a result kind or condition
change must refer to new names. For instance, we can highlight
that a X86 machine is not sequentially consistent by first
renaming \atest{SB} into \ltest{SB+SC}, and then changing the
final condition.
The new condition expresses
that the first instruction (a store)
of one of the threads must come first:
\begin{center}
\begin{tabular}{p{.3\linewidth}p{.2\linewidth}p{.3\linewidth}}
\multicolumn{1}{c}{\afile{rename.txt}} & &
\multicolumn{1}{c}{\afile{cond.txt}}\\ \hline
\begin{verbatim}
SB SB+SC
\end{verbatim}
& \quad\quad\quad &
\begin{verbatim}
SB+SC "forall (0:EAX=1 \/ 1:EAX=1)"
\end{verbatim}
\end{tabular}
\end{center}
Then, we run litmus:
\begin{verbatim}
% litmus7 -mach x86 -rename rename.txt -conds cond.txt SB.litmus
%%%%%%%%%%%%%%%%%%%%%%%%%
% Results for SB.litmus %
%%%%%%%%%%%%%%%%%%%%%%%%%
X86 SB+SC
"Fre PodWR Fre PodWR"

{x=0; y=0;}

 P0          | P1          ;
 MOV [x],$1  | MOV [y],$1  ;
 MOV EAX,[y] | MOV EAX,[x] ;

forall (0:EAX=1 \/ 1:EAX=1)
Generated assembler
#START _litmus_P1
        movl $1,(%r8,%rdx)
        movl (%rdx),%eax
#START _litmus_P0
        movl $1,(%rdx)
        movl (%r8,%rdx),%eax

Test SB+SC Required
Histogram (4 states)
39954 *>0:EAX=0; 1:EAX=0;
3979407:>0:EAX=1; 1:EAX=0;
3980444:>0:EAX=0; 1:EAX=1;
195   :>0:EAX=1; 1:EAX=1;
No

Witnesses
Positive: 7960046, Negative: 39954
Condition forall (0:EAX=1 \/ 1:EAX=1) is NOT validated
Hash=7dbd6b8e6dd4abc2ef3d48b0376fb2e3
Observation SB+SC Sometimes 7960046 39954
Time SB+SC 0.48
\end{verbatim}
One sees that the test name and final condition have changed.

\paragraph*{Miscellaneous\label{misc}}
\begin{description}
\item[{\tt -sleep <n>}] Insert a delay of $n$ seconds between each individual test run.
\item[{\tt -exit <bool>}] Exit status of each individual test executable reflects the final condition success or failure, default \texttt{false} (test exit status always is success in absence  of errors).
\end{description}

\subsection*{\aname{config:sec}{Configuration} files}
The syntax of configuration files is minimal:
lines ``\textit{key} \texttt{=} \textit{arg}'' are interpreted
as setting the value of parameter~\textit{key} to \textit{arg}.
Each parameter has a corresponding option,
usually \opt{-}\textit{key}, except for single-letter options:
\begin{center}
\newenvironment{opts}
{\begin{tabular}{lll}
\hline \hline
\multicolumn{1}{c}{\quad\textit{option}\quad} &
\multicolumn{1}{c}{\quad\textit{key}\quad} &
\multicolumn{1}{c}{\quad\textit{arg}\quad} \\
\hline \hline}
{\hline \hline \end{tabular}}
\begin{opts}
\opt{-a} & avail & integer \\
\opt{-s} & size\_of\_test & integer\\
\opt{-r} & number\_of\_run & integer\\
\opt{-p} & procs & list of integers\\
\opt{-l} & loop & integer\\
\end{opts}
\end{center}
Notice that \litmus{} in fact accepts long versions of options
(\emph{e.g.} \opt{-avail} for~\opt{-a}).

As command line option are processed left-to-right,
settings from a configuration file (option \opt{-mach})
can be overridden by a later
command line option.
Some configuration files for the machines we have tested
are present in the distribution. As an example here is the configuration
file \file{hpcx.cfg}.
\verbatiminput{hpcx.cfg}
Lines introduced by \verb+#+ are comments and are thus ignored.

Configuration files are searched first in the current directory;
then in any directory specified
by setting the shell environment variable \texttt{LITMUSDIR};
and then in litmus installation directory, which is defined
while compiling~\litmus{}.

\section{\label{klitmus}Running kernel tests with \klitmus}

The tool~\klitmus{} is specialised for running kernel tests as kernel modules.
Kernel modules are object files that can be loaded dynamically in a
running kernel. The modules are then launched by reading the
pseudo-file~\texttt{/proc/litmus}.

\subsection{Compiling and running a kernel litmus test}
Kernel tests are normally written in C, augmented with specific kernel macros.
As an example, here is \afile{SB+onces.litmus},
the kernel version of the \ahrefloc{SB}{store buffering litmus test}:
\verbatiminput{SB+onces.litmus}

As can be seen above, C litmus tests are formatted differently from assembler
tests: the code of threads is presented as functions, not as columns of
instructions. One may also notice that the test uses the kernel macros
\verb+WRITE_ONCE+ (guaranteed memory write) and \verb+READ_ONCE+
(guaranteed memory read).

The tool \klitmus{} offers only the cross compilation mode:
\emph{i.e.}, \klitmus{} translates the tests to some C files, plus
\texttt{Makefile} and run script (\texttt{run.sh}).
\begin{verbatim}
% klitmus7 -o X.tar SB+onces.litmus
\end{verbatim}
The above command outputs the archive \verb+X.tar+.

Then, users compile the contents of the archive on the target machine, which
need not be the machine where \klitmus{} has run.
\begin{verbatim}
% mkdir -p X && cd X && tar xmf ../X.tar
% make make
make -C /lib/modules/3.13.0-108-generic/build/ M=/tmp/X modules
make[1]: Entering directory `/usr/src/linux-headers-3.13.0-108-generic'
  CC [M]  /tmp/X/litmus000.o
  Building modules, stage 2.
  MODPOST 1 modules
  CC      /tmp/X/litmus000.mod.o
  LD [M]  /tmp/X/litmus000.ko
make[1]: Leaving directory `/usr/src/linux-headers-3.13.0-108-generic'
\end{verbatim}
Notice that the final compilation relies on kernel headers being installed.

One then run the test. This step must be performed with root
privileges, and requires the \texttt{insmod} and \texttt{rmmod}
commands to be installed. We get:
\begin{verbatim}
% sudo sh run.sh
Fri Aug 18 16:37:43 CEST 2017
Compilation command: klitmus7 -o X.tar SB+onces.litmus
OPT=
uname -r=3.13.0-108-generic

Test SB+onces Allowed
Histogram (4 states)
184992  *>0:r0=0; 1:r0=0;
909016  :>0:r0=1; 1:r0=0;
858961  :>0:r0=0; 1:r0=1;
47031   :>0:r0=1; 1:r0=1;
Ok

Witnesses
Positive: 184992, Negative: 1815008
Condition exists (0:r0=0 /\ 1:r0=0) is validated
Hash=2487db0f8d32aeb7ec0eacf5d0e301d7
Observation SB+onces Sometimes 184992 1815008
Time SB+onces 0.18

Fri Aug 18 16:37:43 CEST 2017
\end{verbatim}
Output is similar to \litmus{} output. One sees that the store buffer idiom
is also observed in kernel mode.

In this simple example, we compile and run one single test.
However, usual practice is, as it is for \litmus{}, to compile and run
several tests at once.
One achieves this naturally, by supplying several source files as command
line arguments to \klitmus,
or by using index files, see Sec~\ref{file:argument}.

\subsection{Runtime control of kernel modules}
The value of some test parameters
(see Sec.\ref{litmus:control}) can be set differently from defaults.
This change is performed by the means
of a key value interface.\footnote{This interface is the one provided for
configuring kernel modules at load time.}
For instance, one sets the number of threads devoted
to the test, the size of a run
and the number of runs by using keywords \texttt{avail},
\texttt{size} and \texttt{nruns}, respectively:
\begin{verbatim}
% sudo sh run.sh avail=2 size=100 nruns=1
Mon Aug 21 11:04:12 CEST 2017
Compilation command: klitmus7 -o /tmp/X.tar SB+onces.litmus
OPT=avail=2 size=100 nruns=1
uname -r=3.13.0-108-generic

Test SB+onces Allowed
Histogram (3 states)
12      *>0:r0=0; 1:r0=0;
48      :>0:r0=1; 1:r0=0;
40      :>0:r0=0; 1:r0=1;
Ok

Witnesses
Positive: 12, Negative: 88
Condition exists (0:r0=0 /\ 1:r0=0) is validated
Hash=2487db0f8d32aeb7ec0eacf5d0e301d7
Observation SB+onces Sometimes 12 88
Time SB+onces 0.00

Mon Aug 21 11:04:12 CEST 2017
\end{verbatim}

\aname{module:parameter}{The kernel modules}
produced by \klitmus{} can be controlled by
using the following keys=\texttt{\em{<int>}} settings.
The default values of those parameters are set at compile time,
by the means of command line options passed to~\klitmus,
\ahrefloc{klitmus:parameter:default}{see below}.

\begin{description}
\item[{\tt size}] Size of a run.

\item[{\tt nruns}] Number of runs.

\item[{\tt ninst}] Number of test instances run.
When the value of this parameter is zero (usual case)
it is unused (see \texttt{avail} below).

Otherwise, \texttt{ninst} instance of each test is run.
However, if the total number of threads used by a given test exceeds
the number of online logical processors of the machine, the test is not run.

\item[{\tt avail}] Number of threads devoted to testing.
This parameter is active when \texttt{ninst} is zero.

If this parameter is zero (or exceeds the number of online logical processors)
then it is changed to the number of online logical processors of the machine.
Finally, \texttt{avail} divided by the number of threads in test are run.
Since this is euclidean division the resulting number of instances can be zero.
In that case, the test is not run.


\item[{\tt stride}] Stride for scanning locations arrays ---
see Sec.~\ref{defstride}. If this parameter is zero, it is silently
changed to one.

\item[{\tt affincr}] Affinity control. Any value strictly greater
than zero is an affinity increment (see section~\ref{defi}).
A value of zero means no affinity control. A strictly negative value commands
random affinity control.
\end{description}

It is important to notice that the values of keywords are integers in a
strict sense.
Namely, the generalized syntax of integer (such as \texttt{10k}, \texttt{1M}
etc.) is not accepted.
Moreover, except in the affinity case, where negative values are accepted,
only positive (or null) values are accepted.

\subsection{Usage of \klitmus}

Given that \klitmus{} purpose is running kernel litmus tests, \klitmus{}
accepts tests written (1) in~C or (2) in LISA, an experimental
generic assembler langage.

As regards options  \klitmus{} accepts a restricted subset of the options
accepted by \litmus{}, plus a few specific options.

\paragraph*{General behaviour}

\begin{description}
\item[{\tt -version}] Show version number and exit.
\item[{\tt -libdir}] Show installation directory and exit.
\item[{\tt -v}] Be verbose, can be repeated to increase verbosity.
\item[{\tt -o <dest>}]
Save C-source of test files into \opt{<dest>}.
Argument \opt{<dest>} can be an archive (extension \opt{.tar}),
a compressed archive (extension \opt{.tgz}),
or an existing directory. This option is mandatory for \klitmus,
while it is optional for~\litmus.
\item[{\tt -hexa}] Hexadecimal output, default is decimal.
\item[{\tt -pad <n>}] Sets the padding for C litmus filenames, default 3.
Namely, \klitmus{} dumps the code for tests into files that
are named systematically \texttt{litmus\textit{<i>}.c}, where \textit{i} is
an index number formated with
\texttt{"\%0\textit{<n>}i"} (zero left padding,
with minimal size of  \texttt{\textit{<n>}}).
As a result, by default, file names are
\texttt{litmus000.c}, \texttt{litmus001.c}, \texttt{litmus002.c}, \ldots
\end{description}

\paragraph*{\aname{klitmus:parameter:default}{Test conditions}}

Most of these options set the default values of the parameters
\ahrefloc{module:parameter}{described above}.
\begin{description}
\item[{\tt -a <n>}]Run maximal number of tests concurrently for $n$~available
logical processors ---
set default value for parameter~\texttt{avail}.
Default is~$0$. As a result,
the runtime parameter~\texttt{avail}
will default to the number of online logical processors.
\item[{\tt -limit <bool>}] Do not process tests with more than~$n$
threads, where $n$ is the number of available logical processors
specified above. Default is \opt{true}.
Does not apply  when option \texttt{-a} is absent.
\item[{\tt -r <n>}] Perform $n$ runs --- set default value
of the parameter~\opt{nruns}.
The option accepts generalised syntax for integers and
default is~$10$.
\item[{\tt -s <n>}]Size of a run --- set default value for
parameter \opt{size}.
The option accepts generalised syntax for integers and
default is $100000$ (or \opt{100k}).
\item[{\tt -st (adapt|<n>)}]
Set stride value, default is~\texttt{1}.
Set default value of parameter~\texttt{stride}.
In case the command line argument is \texttt{adapt}, the parameter default
value will be the number of threads in test.

\item[{\tt -affinity (none|incr<n>|random)}]
Enable (or disable with tag \opt{none}) affinity control,
specifying default value of parameter~\opt{affincr};
with tag \opt{none} yielding a default value of~\texttt{0},
tag \opt{incr<n>} yielding a default value of~\texttt{<n>},
and tag \opt{random} yielding a default value of~\opt{-1}.
Default is none, \emph{i.e.}, modules do not perform affinity control,
although they include code to do so ---
this is a significant difference with
\ahref{affinity:control}{litmus7 default}).
The various tags are interpreted as follows:
\begin{enumerate}
\item \opt{incr<n>}:
integer \opt{<n>} is the increment  for allocating logical
processors to threads --- see Sec.~\ref{sec:affinity}.
Setting \opt{-affinity incr0} is special and defined as
equivalent to \opt{-affinity none}.
\item \opt{random} modules perform random allocation of
test threads to logical processors.
\end{enumerate}
\item[{\tt -i <n>}] Alias for \opt{-affinity incr<n>}.

\item[{\tt -barrier (userfence|tb)}]
Set synchronisation mode. Default is \opt{userfence}.
One may notice that this option offers less variety than
the corresponding \litmus{} option (see Sec.~\ref{defsynchronisation},
experience having proved to us
that those two setting are the most useful ones.
\end{description}

\paragraph*{Change input}
\begin{description}
\item[{\tt -names <file>}] Run \litmus{} only on tests whose names are
listed in \texttt{<file>}.
\item[{\tt -excl <file>}] Do not run \litmus{} on tests whose names are
listed in \texttt{<file>}.
\item[{\tt -rename <file>}] Change test names.
\item[{\tt -rcu (yes|no|only)}]
Accept or not tests that contain RCU (read-copy-update) primitives,
which can be quite long to run. Default is \texttt{no}.
The tags have the following meaning:
\begin{enumerate}
\item \opt{yes}: Accept RCU tests.
\item \opt{no}: Reject  RCU tests (silently).
\item \opt{only}: Accept RCU tests and reject others.
\end{enumerate}
\item[\tt -expedited <bool>]  translate the RCU primitive
\verb+syncronize_rcu+ to
\verb+synchronize_expedited+ (which may be much faster).
Default is \opt{true}.
\end{description}

\paragraph*{Miscellaneous option}
\begin{description}
\item[{\tt -ccopt <string>}]Additional option for the C compiler.
\end{description}

\section{\label{sec:nostd} Non-standard modes}
The non-standard modes are selected by \litmus{} options \opt{-mode
presi} and \opt{-mode kvm}.

\subsection{\label{sec:presi}Mode \texttt{presi}}
In standard mode, system threads are spawned frequently
--- see Section~\ref{sec:arch}.
By contrast, in presi mode, harness threads  are created once for
all and then collaborate  to execute  litmus tests, a technique better
adapted to running tests over limiter system support. Non-standard
modes also permit finer control over some of test parameters and the
collection of statistics on their effect.

During test run, each harness thread runs various tests threads,
with various parameters being selected randomly.
For instance, consider the following test~\atest{R}:
\verbatiminput{R.litmus}
On a X86\_64  machine with four cores two-ways hyper-threaded cores,
we compile the test~\atest{R} in presi mode as follows:
\begin{verbatim}
% mkdir -p R
% litmus7 -mach x86_64 -mode presi -avail 8 -o R R.litmus
% cd R
% make ...
\end{verbatim}
Namely, the machine has 8 ``logical'' cores, \emph{i.e}
4~two-way hyper-threaded cores. Hence the harness spawns 8 threads,
whether those threads being attached to hardware logical cores or not
depends on the target system offering affinity control (Linux) or not (MacOs).

We now run the  test:
\begin{verbatim}
% ./R.exe -s 1k -r 1k
Test R Allowed
Histogram (4 states)
1889769:>1:rax=0; [y]=1;
1653920:>1:rax=1; [y]=2;
322740*>1:rax=0; [y]=2;
133571:>1:rax=1; [y]=1;
Ok

Witnesses
Positive: 322740, Negative: 3677260
Condition exists ([y]=2 /\ 1:rax=0) is validated
Hash=b66b3ce129b44b61416afa21c7cc6972
Observation R Sometimes 322740 3677260
Topology 56913 :> part=0 [[0],[1]]
Topology 265827:> part=1 [[0,1]]
Vars 205786:> {xp=0}
Vars 116954:> {xp=1}
Cache 86602 :> {c_0_x=0, c_0_y=1, c_1_x=0, c_1_y=0}
Cache 67620 :> {c_0_x=0, c_0_y=2, c_1_x=0, c_1_y=0}
Cache 41920 :> {c_0_x=0, c_0_y=2, c_1_x=0, c_1_y=1}
Cache 51438 :> {c_0_x=2, c_0_y=1, c_1_x=0, c_1_y=0}
Cache 49888 :> {c_0_x=2, c_0_y=2, c_1_x=0, c_1_y=0}
Time R 0.85

\end{verbatim}
First observe the command lines options ``\texttt{-s 1k -r 1k}'',
thereby setting the size parameter~$s$  and the number of
runs~$r$. The interpretation of  these settings differ from  standard
mode: in presi mode and for each test instance, $s$~experiments are
performed for a given random allocation of test threads to harness
threads. This process is performed $r$~times, resulting in $n \times
r \times s$ experiments, where $n$ is the number of test instances
(see Section~\ref{sec:arch}).

In test output above. we see that the final state of the test is
observed about $323$k~times over four billions tries. More
significantly, we see some record of successful parameters. Each line
consists in a number of occurrences of and of a parameter choice,
called \aname{statistics}{\emph{parameter statistics}}. For instance,
the first line ``\texttt{Topology 56913 :> part=0 [[0],[1]]}'' signals
there are about $57$k occurrences of the target
outcome \texttt{1:rax=0; [y]=2;} with topology
parameter \texttt{part=0}. We now list parameters:
\begin{description}
\item[{\tt Topology}]  The configuration file \texttt{x86\_64.cfg}
specifies the standard Intel topology description:
\begin{verbatim}
...
# Standard Intel parameters
smt=2
...
\end{verbatim}
Thus, ``logical'' core are grouped by two. Hence, the two threads of
the R~test can either be running on cores in the same group, or from
different groups. This corresponds to parameters \texttt{part=1
[[0,1]]} and \texttt{part=0 [[0],[1]]}, respectively.
\item[{\tt Vars}] In sharp contrast to standard mode, where
experiments operate on array cells,  all experiments (with a given
Vars setting) operate on the same memory locations. The Vars
parameters signals the relative placement of the two memory
locations~\texttt{x} and~\texttt{y}. Parameter \texttt{xp} is to be
interpreted as a cache line offset relative to the cache line of the
missing location (here~\texttt{y}). That is, \texttt{xp=0} means
that \texttt{x} and~\texttt{y} share the same cache line,
while \texttt{xp=1} means that \texttt{x} resides in the cache line
next to the one of~\texttt{y}.
\item[{\tt Cache}] Finally, the cache  parameter describes the cache
management performed by each test thread (\texttt{0} or~\texttt{1}) on
each variable (\texttt{x} or~\texttt{y}) before running the actual
experiment. For instance, given \texttt{\{c\_0\_x=0, c\_0\_y=1,
c\_1\_x=0, c\_1\_y=0\}} thread~1 performs operation number one (cache
flush) on \texttt{y}, while thread zero performs operation number zero
(\emph{i.e.} no operation or ignore). For completeness, operation
number two is cache touch and, if available, operation number three is
cache touch for store.
\end{description}

As a distinctive feature of presi mode, users can set fixed values for
parameters by giving settings on the command line. For instance, one
can set the topology and vars parameters to their most productive
values from a first test run:
\begin{verbatim}
% ./R.exe -s 1k -r 1k part=1 xp=0
Test R Allowed
...
Observation R Sometimes 660262 3339738
Topology 660262:> part=1 [[0,1]]
Vars 660262:> {xp=0}
Cache 387980:> {c_0_x=0, c_0_y=1, c_1_x=0, c_1_y=0}
Cache 272282:> {c_0_x=0, c_0_y=2, c_1_x=0, c_1_y=0}
Time R 0.52
\end{verbatim}
We witness an improvement of productivity by a factor of two.

Most command line options were designed for standard mode first and presi mode
is in some sense more automated. As a consequence, few command line
settings apply to the presi mode: we have already seen
the \texttt{-r}, \texttt{-s} and \texttt{-a} setting, of which interpretation
significantly differs from standard mode.

All options related to affinity are ignored: topology variation
applies even when harness  threads cannot be attached to hardware
threads. However, one can void the effect of topology variation with
the option \texttt{-smt 1} (resulting in as many groups as available
harness threads),  or by giving the same number argument to the
options \opt{-smt} and~\opt{-a} (resulting in one group).

The barrier setting option \texttt{-barrier tb} of
Section~\ref{timebase} is effective. The  timebase delay value
is expressed in ``ticks'', depends on the target architecture and can
be changed at runtime with option \texttt{-tb <num>}.  Moreover, this
option introduces a new randomly selected parameter. The delay
parameter of threads w.r.t. to the starting time of test thread~$T_0$,
as a (possibly negative) number of one eighth of the timebase delay
value in the interval~$[-4..4]$.

Option \ahrefloc{driver}{\texttt{-driver C}} is also implemented. It
commands to the production of one executable \texttt{run.exe} in place
of one executable per test  with the default \texttt{-driver
shell}.

Observe that non-implemented options are silently ignored.


\subsection{\aname{sec:kvm}{Mode \texttt{kvm}}}
The kvm (or vmsa) mode is based upon the \ahrefloc{sec:presi}{presi mode}.
This mode permits running tests at the system level under the
control of the KVM virtualisation infrastructure, by using the adequate
qemu virtual machine. To that aim, it leverages the
pre-existing \footahref{https://gitlab.com/kvm-unit-tests/kvm-unit-tests}{\texttt{kvm-unit-tests}}
framework, available for Linux boxes and, with a little effort, on
ARM-based Apple machines. Our extension has been developed for ARM
AArch64 and from now we shall consider this  architecture. However,
there is also a partial implementation for X86\_64. Refer to
\footahref{https://gitlab.com/kvm-unit-tests/kvm-unit-tests}{\texttt{kvm-unit-tests}}
or to OS documentation for qemu with kvm support and kvm-unit-tests
installation instructions (instructions
for \footahref{https://wiki.debian.org/KVM}{Debian Linux}).

Let us consider the following two system tests
\atest{MP-TTD+DMB.ST+DMB.LD} and~\atest{MP-TTD+DMB.ST+DSB-ISB}:
\begin{center}
\begin{tabular}{p{.45\linewidth}p{.4\linewidth}}
\verbatiminput{MP-TTD+DMB.ST+DMB.LD.litmus}
&
\verbatiminput{MP-TTD+DMB.ST+DSB-ISB.litmus}
\end{tabular}
\end{center}
The tests are specific ``message-passing tests'' acting on the signal
location \texttt{y} and the data location~\texttt{x}. However, instead
of changing the data content, we change its status or, more precisely,
its TTD (Translation Table Descriptors). That is, location~\texttt{x} is
invalid initially and $T_0$ restore validity with its first
instruction \texttt{STR X1,[X2]}, where the register \texttt{X1} holds
the valid descriptor and the register \texttt{X2} points to the page table
entry of~\texttt{x}. The thread $T_0$ then signals by writing the
value one into the location~\texttt{y}. Concurrently, $T_1$ reads the
signal location~\texttt{y} and then attemps to read~\texttt{x}.
The final condition
\verb+exists(1:X7=1 /\ Fault(P1:L0,x,MMU:Translation)+
describes the situation when the signal has been received by~$T_1$
(\texttt{1:X7=1}), while $T_1$ has not yet noticed that $T_0$ has made
accessing to~\texttt{x} valid
--- The atom \texttt{Fault(P1:L0,x,MMU:Translation)} corresponds to a
faulting attempt by~$T_1$ to read the location~\texttt{x}.
The two tests differ by their synchronisation
instructions: \atest{MP-TTD+DMB.ST+DMB.LD} is synchronised by fences
sufficient to synchronise the ordinary message passing test,
while  \atest{MP-TTD+DMB.ST+DSB-ISB} synchronisation is more
heavyweight.

Compilation to qemu binary file arguments is complex.
For user convenience, we provide litmus configuration
files \texttt{kvm-aarch64.cfg} (generic
ARMv8.0), \texttt{kvm-armv8.1.cfg} (for ARMv8.1)
and  \texttt{kvm-m1.cfg} (for Apple machines).
As an example, we target some ARMv8.0 machine MACH, with a
functional \texttt{gcc} compiler and \texttt{kvm-unit-tests} installed
in some homonymous directory.  We execute \litmus{} on another machine
and run the tests on MACH following the usual
``litmus-cross-compilation''~\ahrefloc{crosscompile}{scheme}.
\begin{verbatim}
% litmus7 -mach kvm-aarch64 -variant fatal -o M.tar MP-TTD+DMB.ST+DMB.LD.litmus  MP-TTD+DMB.ST+DSB-ISB.litmus
% scp M.tar MACH:
\end{verbatim}
Notice the \opt{-variant fatal} option that
controls \ahrefloc{faultcontrol}{fault management}. Here,
if some instruction faults, control is transferred at the end of thread~code. See 

On the target machine, we unpack the transferred archive into some
sub-directory of the \texttt{kvm-unit-tests} installation, and compile
the test:
\begin{verbatim}
mach% cd kvm-litmus-tests
mach% mkdir -p M
mach% cd M
mach% tar xmf ../../M.tar
mach% make
...
\end{verbatim}
We are now ready to run the tests. It is important to observe that
tests can be run from the \texttt{kvm-unit-tests} directory and only from
there:
\begin{verbatim}
% sh M/run.sh
...
Test MP-TTD+DMB.ST+DMB.LD Allowed
Histogram (4 states)
70    *>1:X7=1; fault(P1:L0,x,MMU:Translation);
1975326:>1:X7=0; ~fault(P1:L0,x,MMU:Translation);
21197 :>1:X7=0; fault(P1:L0,x,MMU:Translation);
3407  :>1:X7=1; ~fault(P1:L0,x,MMU:Translation);
Ok

Witnesses
Positive: 70, Negative: 1999930
Condition exists (1:X7=1 /\ fault(P1:L0,x,MMU:Translation)) is validated
Hash=e6d0396aa32c40b307db5f6921404c07
EL0=P1
Observation MP-TTD+DMB.ST+DMB.LD Sometimes 70 1999930
Topology 70    :> part=0 [[0],[1]]
Faults MP-TTD+DMB.ST+DMB.LD 21267 P1:21267
Time MP-TTD+DMB.ST+DMB.LD 15.14

...

Test MP-TTD+DMB.ST+DSB-ISB Allowed
Histogram (3 states)
1991985:>1:X7=0; ~fault(P1:L0,x,MMU:Translation);
5938  :>1:X7=0; fault(P1:L0,x,MMU:Translation);
2077  :>1:X7=1; ~fault(P1:L0,x,MMU:Translation);
No

Witnesses
Positive: 0, Negative: 2000000
Condition exists (1:X7=1 /\ fault(P1:L0,x,MMU:Translation)) is NOT validated
Hash=18e9682c0e7ed43f7f059cbfc47d781d
EL0=P1
Observation MP-TTD+DMB.ST+DSB-ISB Never 0 2000000
Faults MP-TTD+DMB.ST+DSB-ISB 5938 P1:5938
Time MP-TTD+DMB.ST+DSB-ISB 17.29
...
\end{verbatim}
%HEVEA The above log has been elided for brevity, \ahref{kvm-log.txt}{complete log}.
We see that the lightweight synchronisation pattern
of \atest{MP-TTD+DMB.ST+DMB.LD} does not suffice to forbid the non-SC
message passing execution, while the heavyweight  synchronisation
pattern of \atest{MP-TTD+DMB.ST+DSB-ISB} does suffice.

Due to harness structure and kvm-unit-tests limited system
support, \litmus{} options and runtime controls are restricted. Harness
structure is similar to one of \ahrefloc{sec:presi}{presi mode},
with the noticeable exception that each memory location resides in its
own (virtual memory) page, where the memory location of the presi mode
reside in cache lines. The following options are worth mentioning:
\begin{description}
\item[{\tt -s  <n>} and {\tt -r <n>}] Control the size and number of
runs. As in presi mode, a run consists in a series of single
experiments with a fixed allocation of tests threads to qemu (or harness)
threads. The \litmus{} tool accepts the long forms
({\tt -size  <n>} and {\tt -nruns <n>}) and sets default values that
can be changed by giving the same options \opt{-s} and~\opt{-r}
to test executables.

\item[{\tt -a <n>}] Define the number of cores (or hardware threads)
allocated to the qemu virtual machine. The \litmus{} tool also accepts
the long form \texttt{-avail <n>}.

\item[{\tt -smt <n>} and {\tt smtmode (none|seq|end)}]
Define (simple) machine topology: threads gather in groups of
``close'' siblings of size \opt{<n>} and threads of the same
group are numbered in sequence (\opt{-smt seq}) or by gaps of size
number of threads divided by group size (\opt{end}). For instance,
given $6$ available threads and a value of $2$ for the \opt{smt}
parameter, groups are $\{0,1\}, \{2,3\},  \{4,5\}$ in the first case and
$\{0,3\}, \{1,4\}, \{2,5\}$ in the second. Those parameters command
the possible allocations of test threads to harness threads. Those
allocations aim at testing all partitions of test threads among
groups. Default is a neutral topology (\opt{-smt 1}) that will result
in one  single possibility (all test threads in different groups).

It is worth noticing that, once  harness threads are selected
according to their groups, allocation of test threads to them is
performed randomly, unless the test executable is given the \opt{+fix}
option. For instance, considering a test with two threads $T_0$
and~$T_1$ running on the above described machine, let us further
assume the allocation of a test instance to the group $\{2,5\}$.
Then, the two possible random allocations are
$\{ T_0 \rightarrow 2, T_1 \rightarrow 5\}$
and
$\{ T_0 \rightarrow 5, T_1 \rightarrow 2\}$. With option \opt{+fix}
one only of those allocations is performed.

\item[{\tt -barrier (user|userfence|pthread|none|timebase)}]
Control test thread starting synchronisation. The default and
behaviour for all arguments except \texttt{timebase} is for test
threads to synchronise by using
standard \ahref{https://en.wikipedia.org/wiki/Barrier\_(computer\_science)}
{sense-reversing barriers}. Argument \texttt{timebase} (which can be abbreviated
as \texttt{tb}) uses the same synchronisation barrier for threads to
agree on starting time based upon the hardware timebase. Timebase
offers a finer control on test thread starting time: given a timebase
delay ($d=32$ for AArch64) and a a starting ``timebase tick
time''~$t_0$ for the starting time of the test first thread $T_0$,
another thread~$T_k$ will start at time $t_k = t_0+(i_k*d)/8$, where
$i_k$ is a random integer in the interval $[-4..+4]$. Test executables
accept the option \texttt{-tb <n>} for overriding the default value
of the timebase delay~$d$.

\item[{\tt -driver (shell|C)}] Choose the driver that will run the tests.
In the \opt{shell} (and default) mode, each test will be compiled into
a \texttt{.flat} binary given as argument to qemu. By contrast, in the
``\opt{C}'' mode, there is only one binary \texttt{run.flat} and one
call of the qemu virtual machine. As a result, qemu initial runtime
costs are amortized for batches of many tests. It is worth noticing
that tests are executed by running a shell script called
\texttt{run.sh} in all modes.

However notice that \ahref{statistics}{parameter statistics}
are not collected and thus not  printed in \opt{-driver C} mode.

\item[\aname{faultcontrol}{\tt -variant (handled|fatal|skip)}] When
an instruction faults, control is transferred to a \emph{fault
handler} that records the fault occurrence and returns to the faulting
code. The above three variants control where:
\begin{description}
\item[{\tt handled}] Attempt to re-execute the faulting
instruction. For the test not to loop forever, one of the test threads
should supppress the cause of the fault, usually by changing some page
table entry. This is the
default. Synonymous: \texttt{imprecise}, \texttt{async}  and~\texttt{asynchronous}.
\item[{\tt fatal}] Jump to the end of the thread code. Notice that
threads other than the faulting thread are not
affected. Synonymous: \texttt{precise}, \texttt{sync} and~\texttt{synchronous}.
\item[{\tt skip}] Execute the instruction that follows the faulting
instruction. Synonymous: \texttt{faultToNext}.
\end{description}
\end{description}
Other miscellaneous options are implemented, such
as \ahrefloc{opt:stdio}{\opt{-stdio (true|false)}},
\ahrefloc{opt:hexa}{\opt{-hexa}},
\ahrefloc{opt:alloc}{\opt{-alloc (dynamic|static)}},
\ldots{}
Non recognised options should be ignored silently.
