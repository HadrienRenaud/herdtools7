\cutname{developer.html}
\section{Introducing new languages in \texttt{herdtools7}}
A language can be a high-level programming language (e.g. C++, Java), an
assembly language (e.g. AArch64, x86), or a specialised language (e.g. BPF).

To handle a target language (TL) (e.g. run litmus tests against the
\herd{} simulator, or against hardware using \litmus{}, or generate
litmus tests using \diy{}), it is required to introduce the following
files in the \source{lib/} folder:

\begin{itemize}
\item \source{TLBase.ml}
\item \source{TLLexer.\{ml,mll\}}
\item \source{TLParser.mly}
\end{itemize}

The \source{lib/} folder contains modules that are shared between the three
other relevant directories: \source{herd/}, \source{gen/} and \source{litmus/}.

The \source{TLBase.ml} file defines the basic types that all tools
(\herd{}, \diy{} and \litmus{}) use. Using
\source{AArch64Base.ml} as an example, it defines the following:

\begin{itemize}
\item Register formats: \texttt{type gpr}, \texttt{type reg}
\item Barriers: \texttt{type barrier}
\item Instruction: \texttt{type kinstruction}
\item Datatype of the AST
\item Various pretty printing functions: \source{do_pp_instruction}
\end{itemize}

It might be further required to define memory orders for the language: an
example can be found in \source{lib/memOrder.ml}, which is relative to C++.

After defining the \source{lib/} files, the next steps are:

\begin{itemize}
\item \source{herd/}: Define the semantics of commands which link to cat files.
\item \source{gen/}: Enable multiple tests generation.
\item \source{litmus/}: Enable running litmus tests.
\end{itemize}

It does not matter which tool you should to update first: some people
prioritise the simulator \herd{} for rapid modeling development, others
prioritise the test harness \litmus{} for experimenting with hardware.  

\subsection{The \herd{} tool}

In \source{herd/}, three files are required:
\begin{itemize}
\item \source{TLArch_herd.ml}: Define the annotations that the language Effects
(sometimes called events) can have and the sets that the cat file uses. For
example: \texttt{type annot} and \texttt{type annot\_sets} in
\source{AArch64Arch_herd.ml}

\item \source{TLMem.ml}: Simple call to \source{machModelChecker}. It might be
further required to define \source{TLBarrier.mli} which has the same type
declaration as \source{TLBase.ml}.

\item \source{TLSem.ml}: Using monadic operators (see
Section~\ref{sec:monadic_operators}), such as \source{>>=} or \source{>>|},
assemble the semantics of a given instruction. See function
\source{build_semantics} in \source{herd/AArch64Sem.ml}.
\end{itemize}

\subsubsection{Monadic operators in herd \label{sec:monadic_operators}}

Monadic operators are defined in \source{herd/eventsMonad.ml} and are used to
combine different Effects and add relations among them. Memory accesses and
ordering constraints enforced by synchronization operations are examples of
Effects.

The monadic operators use various base operators, such as \source{=**=},
\source{=*$=}, \source{=$$=}, \source{+|+}, \source{=|=}, which are defined in
\source{herd/event.ml}. These operators provide various ways of combining two
Effect graphs.

Effect graphs can either represent the execution of a whole litmus test or be
very small, corresponding to some of the Effects contributing to the semantics
of a single instruction or command.

The operators build on three definitions:
\begin{itemize}
\item \source{para_comp}
\item \source{data_comp}
\item \source{control_comp}
\end{itemize}

We detail each in turn below.

\paragraph{Operator \source{para_comp}} Unions Effects from both Effect graphs, and their relations.

For example, \source{para_comp} is used in the semantics of a store (see
\source{do_str} and its call in the definition of \source{str} in
\source{herd/AArch64Sem.ml)}. A store takes as input a source register
\source{rs}, which contains the value \source{v} to be stored, and a target
register \source{rd}, which contains the address \source{a} to be written to.
In \source{do_str}, two things happen in parallel using \source{>>|}, which
makes use of \source{para_comp}: an address \source{a} is extracted from a
register \source{rd} (\source{ma} in \source{do_str}, instantiated to
\source{get_ea} in \source{str}) and a value \source{v} is extracted from
register \source{rs} (via \source{read_reg_data} in \source{do_str}).

Base operators that build on \source{para_comp} are as follows:
\begin{itemize}
\item \source{=|=}: Corresponds to \source{para_comp} over disjoint graphs
\item \source{+|+}: Applies \source{para_comp} if there are two disjoint
graphs, and asserts false otherwise (via \source{check_both})
\end{itemize}

\paragraph{Operator \source{data_comp}} Introduces a causality link between an
Effect graph \source{es1} and an Effect graph \source{es2} (see
\source{intra_causality_data}). This means that the Effect(s) of \source{es1}
is (intrinsically-)before the Effect(s) of \source{es2} in the order of
operation of the instruction or command described.

Using the same \source{str} example as above, after extracting the value
\source{v} and the address \source{a} in parallel, the value \source{v} is
written to the address \source{a} via \source{do_write_mem} in \source{do_str}.
This last Effect \source{do_write_mem} occurs after the two extracting Effects.
In the herd diagrams, this is depicted with an \source{iico} arrow, which
stands for intra-instruction causality order. 

Base operators that build on \source{data_comp} are as follows:
\begin{itemize}
\item \source{=*$=}: Applies \source{data_comp} over disjoint graphs. This
means that two graphs \source{es1} and \source{es2} are composed and an
\source{iico} causality link from \source{es1} to \source{es2} is added.

\item \source{=$$=}: Similar to \source{=*$=}, but additionally performs some
transitive closure on the \source{iico} links (via \source{get_output}).

\item \source{control_comp}: Introduces another type of causality link between
the first and second Effect graphs (see \source{intra_causality_control}). This
is similar to \source{data_comp}, but it distinguishes which type of causality
is at stake. This is used to indicate whether there is a Intrinsic control
dependency.
\end{itemize}


Base operators that build on \source{control_comp} are as follows:
\begin{itemize}
\item \source{=**=}: Applies \source{control_comp} over disjoint graphs,
meaning that two graphs \source{es1} and \source{es2} are composed and an
\source{iico_control} causality link from \source{es1} to \source{es2} is
added.

\item \source{=*$$=}: Similar to \source{=**=}, but additionally performs some
transitive closure on the \source{iico_control} links (via
\source{get_output}).
\end{itemize}

\subsection{Memento of frequently used operators}

\begin{showcenter}
\begin{tabular}{c|p{.5\linewidth}}
Operator      & Description  \\
\source{>>|}  & Intuitively \source{a >>| b} means ``do a in parallel with b''.
This is used when accesses do not need to be ordered. The code snippet  below
reads the register \source{rs} and in parallel reads the register \source{rd},
gets a value \source{v} from \source{rs} and a location \source{x} from
\source{rd}, and writes that value \source{v} back to that memory location
\source{x}.                    

\begin{verbatim}
(read_reg rs >>| read_reg rd) >>= fun (v,x) -> write_mem x v    
\end{verbatim}                                                             

\\

\source{>>||} & \source{>>||} is similar to \source{>>|}, but this operator
does not check               whether the two argument Effect graphs are
disjoint.            

\\

\source{>>=}  & Intuitively \source{a >>= b} means ``do a, collect its result
and feed it into b''. The code snippet below reads the register \source{rs},
gets a value \source{v}, and writes that value \source{v} back to memory
location \source{x}.  

\begin{verbatim}
read_reg rs >>= fun v -> write_mem x v                          
\end{verbatim}

\\

\source{>>==} & The operator \source{>>=} corresponds to \source{=*$=}, while
\source{>>==} corresponds to \source{=$$=}. They both do \source{data_comp}
over disjoint graphs, with the latter performing transitive closure on
\source{iico} causality link.                                        
    
\\

\source{>>*=} and \source{>>*==} & The operator \source{>>*=} corresponds to
\source{=**=}, while \source{>>*==} corresponds to \source{=*$$=}. They both do
\source{control_comp} over disjoint graphs, with the latter performing
transitive closure on \source{iico_control} links.                                          
\end{tabular}
\end{showcenter}

\subsection{The \diy{} tool}

The \source{gen/} directory is used for the \diy{} tool suite which
enables one or multiple tests generation.

The algorithm that generates litmus tests from cycles of relaxations is
parametric in the architecture, resulting in little to no additions required
when introducing a new language.

To add a new backend to the general test generation algorithm, two modules are required:
\begin{itemize}
\item \source{TLArch_gen.ml}: Define the syntax of the tests. For example, what
is the syntax of a load?
\item \source{TLCompile_gen.ml}: Define how the relaxations of a given cycle
combine, and what sequence of code to output on a given relaxation.
\end{itemize}

To better understand \source{TLArch_gen.ml}, an example of such file is
\source{CArch_gen.ml}, with function \source{dump_typ} and \source{type exp} as
example candidates.

To better understand \source{TLCompile_gen.ml}, an example of such file is
\source{CCompile_gen.ml}. It contains the function \source{type_event}, which
checks whether an extremity is atomic or not, in the sense defined by the
architecture. It also contains the function \source{compile_access}, which
could, for example, output a store using function \source{compile_store} as per
the architecture definition, when given \source{W} as an input. See also
functions \source{dump_exp}, \source{dump_ins}, \source{dump_c_test} which
print the right syntax. There needs to be an equivalent of these functions for
each new language.

\subsection{The \litmus{} tool}

The \source{litmus/} directory contains the sources of the \litmus{}
tool which enables running litmus tests on hardware.

To enable the \litmus{} tool to run litmus tests on hardware, it is
required to add new skeletons that the litmus assembly can be slotted into.

In this case, two files are required:
\begin{itemize}
\item \source{TLArch_litmus}
\item \source{TLCompile_Litmus}: Performs the code emission
\end{itemize}

These enable the \litmus{} tool to create a program in the programming
language that is being introduced and then run it on the target machine.
