\documentclass{book}
\usepackage{amsmath}  % Classic math package
\usepackage{amssymb}  % Classic math package
\usepackage{mathtools}  % Additional math package
\usepackage{amssymb}  % Classic math package
\usepackage{mathtools}  % Additional math package
\usepackage{graphicx}  % For figures
\usepackage{caption}  % For figure captions
\usepackage{subcaption}  % For subfigure captions
\usepackage{url}  % Automatically escapes urls
\usepackage{hyperref}  % Insert links inside pdfs
\hypersetup{
    colorlinks=true,
    linkcolor=blue,
    filecolor=magenta,
    urlcolor=cyan,
}
\makeatletter
\makeatletter
  \newcommand{\linkdest}[1]{\Hy@raisedlink{\hypertarget{#1}{}}}
% \newcommand{\linkdest}[1]{\Hy@raisedlink{\hypertarget{#1}{}}}
% \newcommand{\pac}[2]{\hyperlink{{#1}1}{\Hy@raisedlink{\hypertarget{{#1}0}{}}{#2}}}
% \newcommand{\jac}[1]{\Hy@raisedlink{\hypertarget{{#1}1}{}}{\hyperlink{{#1}0}{\ac{#1}}}}
\makeatother
\usepackage[inline]{enumitem}  % For inline lists
\usepackage[export]{adjustbox}  % For centering too wide figures
\usepackage[export]{adjustbox}  % For centering too wide figures
\usepackage{mathpartir}  % For deduction rules and equations paragraphs
\usepackage{comment}
\usepackage{fancyvrb}
\usepackage[
  % Even pages have notes on the left, odd on the right
  twoside,
  % Notes on the right, should be less than outer
  marginparwidth=100pt,
  % margins
  top=4.5cm, bottom=4.5cm, inner=3.5cm, outer=4.5cm
  % To visualize:
  % showframe
]{geometry}
%\usepackage{stmaryrd} % for \llbracket and \rrbracket
\input{ifempty}
\input{ifformal}
\input{ifcode}
%% Should be functional
\ifempty
\newcommand{\isempty}[1]{#1}
\else
\newcommand{\isempty}[1]{}
\fi

%%%Safety net
\iffalse
\makeatletter
\newcommand{\isempty}[1]{#1}
\makeatother
\fi


\newcommand\herd[0]{\texttt{herd7}}

\fvset{fontsize=\small}

\newcommand\tododefine[1]{\hyperlink{tododefine}{\color{red}{\texttt{#1}}}}
\newcommand\lrmcomment[1]{}

\usepackage{enumitem}
\renewlist{itemize}{itemize}{20}
\setlist[itemize,1]{label=\textbullet}
\setlist[itemize,2]{label=\textasteriskcentered}
\setlist[itemize,3]{label=\textendash}
\setlist[itemize,4]{label=$\triangleright$}
\setlist[itemize,5]{label=+}
\setlist[itemize,6]{label=\textbullet}
\setlist[itemize,7]{label=\textasteriskcentered}
\setlist[itemize,8]{label=\textendash}
\setlist[itemize,9]{label=$\triangleright$}
\setlist[itemize,10]{label=+}
\setlist[itemize,11]{label=\textbullet}
\setlist[itemize,12]{label=\textasteriskcentered}
\setlist[itemize,13]{label=\textendash}
\setlist[itemize,14]{label=\textendash}
\setlist[itemize,15]{label=$\triangleright$}
\setlist[itemize,16]{label=+}
\setlist[itemize,17]{label=\textbullet}
\setlist[itemize,18]{label=\textasteriskcentered}
\setlist[itemize,19]{label=\textendash}
\setlist[itemize,20]{label=\textendash}

\ifcode
% First argument is \<rule>Begin, second is \<rule>End, third is the file name.
% Example: for SemanticsRule.Lit, use the following:
% \CodeSubsection{\LitBegin}{\LitEnd}{../Interpreter.ml}
\newcommand\CodeSubsection[3]{\subsection{Code} \VerbatimInput[firstline=#1, lastline=#2]{#3}}
\else
\newcommand\CodeSubsection[3]{}
\fi

\ifcode
% First argument is \<rule>Begin, second is \<rule>End, third is the file name.
% Example: for SemanticsRule.Lit, use the following:
% \CodeSubsubsection{\LitBegin}{\LitEnd}{../Interpreter.ml}
\newcommand\CodeSubsubsection[3]{\subsubsection{Code} \VerbatimInput[firstline=#1, lastline=#2]{#3}}
\else
\newcommand\CodeSubsubsection[3]{}
\fi

%%%%%%%%%%%%%%%%%%%%%%%%%%%%%%%%%%%%%%%%%%%%%%%%%%
% Typesetting macros
\newtheorem{definition}{Definition}
\newcommand\defref[1]{Def.~\ref{def:#1}}
\newcommand\secref[1]{Section~\ref{sec:#1}}
\newcommand\chapref[1]{Chapter~\ref{chap:#1}}
\newcommand\ie{i.\,e.}
\newcommand\eg{e.\,g.}

%%%%%%%%%%%%%%%%%%%%%%%%%%%%%%%%%%%%%%%%%%%%%$%%%%%
%% Mathematical notations and Inference Rule macros
%%%%%%%%%%%%%%%%%%%%%%%%%%%%%%%%%%%%%%%%%%%%$%%%%%%
\usepackage{relsize}
\newcommand\view[0]{\hyperlink{def-deconstruction}{view}}
%\newcommand\defpoint[1]{\underline{#1}}
\newcommand\defpoint[1]{#1}
\newcommand\eqname[0]{\hyperlink{def-deconstruction}{\stackrel{\mathsmaller{\mathsf{is}}}{=}}}
\newcommand\eqdef[0]{\hyperlink{def-eqdef}{:=}}
\newcommand\overname[2]{\overbracket{#1}^{#2}}
\newcommand\overtext[2]{\overbracket{#1}^{\text{#2}}}
\newcommand\emptyfunc[0]{\hyperlink{def-emptyfunc}{{\emptyset}_\lambda}}
\newcommand\restrictfunc[2]{{#1}\hyperlink{def-restrictfunc}{|}_{#2}}

\newcommand\choice[3]{\hyperlink{def-choice}{\textsf{choice}}(#1,#2,#3)}
\newcommand\equal[0]{\hyperlink{def-equal}{\texttt{equal}}}
\newcommand\equalarrow[0]{\rightarrow}
\newcommand\equallength[0]{\hyperlink{def-equallength}{\texttt{equal\_length}}}
\newcommand\listrange[0]{\hyperlink{def-listrange}{\texttt{indices}}}
\newcommand\listlen[1]{\hyperlink{def-listlen}{|}#1\hyperlink{def-listlen}{|}}
\newcommand\cardinality[1]{\hyperlink{def-cardinality}{|}#1\hyperlink{def-cardinality}{|}}
\newcommand\splitlist[0]{\hyperlink{def-splitlist}{\texttt{split}}}
\newcommand\concat[0]{\hyperlink{def-concat}{+}}
\newcommand\prepend[0]{\hyperlink{def-prepend}{{+}{+}}}
\newcommand\listcomprehension[2]{[#1 : #2]} % #1 is a list element predicate, #2 is the output list element.
\newcommand\setcomprehension[2]{\{#1 : #2\}} % #1 is a set element predicate, #2 is the output set element.

\newcommand\Ignore[0]{\hyperlink{def-ignore}{\underline{\;\;}}}
\newcommand\None[0]{\hyperlink{def-none}{\texttt{None}}}

% Set types
\newcommand\N[0]{\hyperlink{def-N}{\mathbb{N}}}
\newcommand\Npos[0]{\hyperlink{def-Npos}{\mathbb{N}^{+}}}
\newcommand\Q[0]{\hyperlink{def-Q}{\mathbb{Q}}}
\newcommand\Z[0]{\hyperlink{def-Z}{\mathbb{Z}}}
\newcommand\Bool[0]{\hyperlink{def-bool}{\mathbb{B}}}
\newcommand\Identifiers[0]{\hyperlink{def-identifier}{\mathbb{I}}}
\newcommand\Strings[0]{\hyperlink{def-strings}{\mathbb{S}}}
\newcommand\astlabels[0]{\hyperlink{def-astlabels}{\mathbb{L}}}
\newcommand\literals[0]{\mathcal{L}}

\newcommand\pow[1]{\hyperlink{def-pow}{\mathcal{P}}(#1)}
\newcommand\partialto[0]{\hyperlink{def-partialfunc}{\rightharpoonup}}
\newcommand\rightarrowfin[0]{\hyperlink{def-finfunction}{\rightarrow_{\text{fin}}}}
\newcommand\funcgraph[0]{\hyperlink{def-funcgraph}{\texttt{func\_graph}}}
\DeclareMathOperator{\dom}{\hyperlink{def-dom}{dom}}
\newcommand\sign[0]{\hyperlink{def-sign}{\texttt{sign}}}

\newcommand\configdomain[1]{\hyperlink{def-configdomain}{\texttt{config\_domain}}({#1})}

\newcommand\sslash[0]{\mathbin{/\mkern-6mu/}}
\newcommand\terminateas[0]{\hyperlink{def-terminateas}{\sslash}\;}
\newcommand\booltrans[1]{\hyperlink{def-booltrans}{\texttt{bool\_transition}}(#1)}
\newcommand\booltransarrow[0]{\longrightarrow}
\newcommand\checktrans[2]{\hyperlink{def-checktrans}{\texttt{check}}(#1, \texttt{#2})}
\newcommand\checktransarrow[0]{\longrightarrow}

%%%%%%%%%%%%%%%%%%%%%%%%%%%%%%%%%%%%%%%%%%%%%%%%%%
% Abstract Syntax macros
% These are used by the AST reference, typing reference, and semantics reference.
\newcommand\specification[0]{\textsf{specification}}

\newcommand\emptylist[0]{\hyperlink{def-emptylist}{[\ ]}}

\newcommand\BNOT[0]{\texttt{BNOT}} % Boolean inversion
\newcommand\NEG[0]{\texttt{NEG}} % Integer or real negation
\newcommand\NOT[0]{\texttt{NOT}} % Bitvector bitwise inversion

\newcommand\AND[0]{\texttt{AND}} % Bitvector bitwise and
\newcommand\BAND[0]{\texttt{BAND}} % Boolean and
\newcommand\BEQ[0]{\texttt{BEQ}} % Boolean equivalence
\newcommand\BOR[0]{\texttt{BOR}} % Boolean or
\newcommand\DIV[0]{\texttt{DIV}} % Integer division
\newcommand\DIVRM[0]{\texttt{DIVRM}} % Inexact integer division, with rounding towards negative infinity.
\newcommand\EOR[0]{\texttt{XOR}} % Bitvector bitwise exclusive or
\newcommand\EQOP[0]{\texttt{EQ\_OP}} % Equality on two base values of same type
\newcommand\GT[0]{\texttt{GT}} % Greater than for int or reals
\newcommand\GEQ[0]{\texttt{GEQ}} % Greater or equal for int or reals
\newcommand\IMPL[0]{\texttt{IMPL}} % Boolean implication
\newcommand\LT[0]{\texttt{LT}} % Less than for int or reals
\newcommand\LEQ[0]{\texttt{LEQ}} % Less or equal for int or reals
\newcommand\MOD[0]{\texttt{MOD}} % Remainder of integer division
\newcommand\MINUS[0]{\texttt{MINUS}} % Subtraction for int or reals or bitvectors
\newcommand\MUL[0]{\texttt{MUL}} % Multiplication for int or reals or bitvectors
\newcommand\NEQ[0]{\texttt{NEQ}} % Non equality on two base values of same type
\newcommand\OR[0]{\texttt{OR}} % Bitvector bitwise or
\newcommand\PLUS[0]{\texttt{PLUS}} % Addition for int or reals or bitvectors
\newcommand\POW[0]{\texttt{POW}} % Exponentiation for ints
\newcommand\RDIV[0]{\texttt{RDIV}} % Division for reals
\newcommand\SHL[0]{\texttt{SHL}} % Shift left for ints
\newcommand\SHR[0]{\texttt{SHR}} % Shift right for ints

\newcommand\UNKNOWN[0]{\texttt{UNKNOWN}}

% For loop direction
\newcommand\UP[0]{\texttt{Up}}
\newcommand\DOWN[0]{\texttt{Down}}

% Non-terminal names
\newcommand\unop[0]{\textsf{unop}}
\newcommand\binop[0]{\textsf{binop}}
\newcommand\literal[0]{\textsf{literal}}
\newcommand\expr[0]{\textsf{expr}}
\newcommand\lexpr[0]{\textsf{lexpr}}
\newcommand\slice[0]{\textsf{slice}}
\newcommand\arrayindex[0]{\textsf{array\_index}}
\newcommand\leslice[0]{\texttt{LE\_Slice}}

\newcommand\ty[0]{\textsf{ty}}
\newcommand\pattern[0]{\textsf{pattern}}
\newcommand\intconstraints[0]{\textsf{int\_constraints}}
\newcommand\intconstraint[0]{\textsf{int\_constraint}}
\newcommand\unconstrained[0]{\textsf{Unconstrained}}
\newcommand\wellconstrained[0]{\textsf{WellConstrained}}
\newcommand\underconstrained[0]{\textsf{Underconstrained}}
\newcommand\constraintexact[0]{\textsf{Constraint\_Exact}}
\newcommand\constraintrange[0]{\textsf{Constraint\_Range}}
\newcommand\bitfield[0]{\textsf{bitfield}}
\newcommand\version[0]{\textsf{version}}
\newcommand\spec[0]{\textsf{spec}}
\newcommand\typedidentifier[0]{\textsf{typed\_identifier}}
\newcommand\localdeclkeyword[0]{\textsf{local\_decl\_keyword}}
\newcommand\globaldeclkeyword[0]{\textsf{global\_decl\_keyword}}
\newcommand\localdeclitem[0]{\textsf{local\_decl\_item}}
\newcommand\globaldecl[0]{\textsf{global\_decl}}
\newcommand\fordirection[0]{\textsf{for\_direction}}
\newcommand\stmt[0]{\textsf{stmt}}
\newcommand\decl[0]{\textsf{decl}}
\newcommand\casealt[0]{\textsf{case\_alt}}
\newcommand\catcher[0]{\textsf{catcher}}
\newcommand\subprogramtype[0]{\textsf{sub\_program\_type}}
\newcommand\subprogrambody[0]{\textsf{sub\_program\_body}}
\newcommand\func[0]{\textsf{func}}
\newcommand\Field[0]{\textsf{field}}

% Expression labels
\newcommand\ELiteral[0]{\textsf{E\_Literal}}
\newcommand\EVar[0]{\textsf{E\_Var}}
\newcommand\EATC[0]{\textsf{E\_ATC}}
\newcommand\EBinop[0]{\textsf{E\_Binop}}
\newcommand\EUnop[0]{\textsf{E\_Unop}}
\newcommand\ECall[0]{\textsf{E\_Call}}
\newcommand\ESlice[0]{\textsf{E\_Slice}}
\newcommand\ECond[0]{\textsf{E\_Cond}}
\newcommand\EGetArray[0]{\textsf{E\_GetArray}}
\newcommand\EGetField[0]{\textsf{E\_GetField}}
\newcommand\EGetItem[0]{\textsf{E\_GetItem}}
\newcommand\EGetFields[0]{\textsf{E\_GetFields}}
\newcommand\ERecord[0]{\textsf{E\_Record}}
\newcommand\EConcat[0]{\textsf{E\_Concat}}
\newcommand\ETuple[0]{\textsf{E\_Tuple}}
\newcommand\EUnknown[0]{\textsf{E\_Unknown}}
\newcommand\EPattern[0]{\textsf{E\_Pattern}}

% Left-hand-side expression labels
\newcommand\LEConcat[0]{\textsf{LE\_Concat}}
\newcommand\LEDiscard[0]{\textsf{LE\_Discard}}
\newcommand\LEVar[0]{\textsf{LE\_Var}}
\newcommand\LESlice{\textsf{LE\_Slice}}
\newcommand\LESetArray[0]{\textsf{LE\_SetArray}}
\newcommand\LESetField[0]{\textsf{LE\_SetField}}
\newcommand\LESetFields[0]{\textsf{LE\_SetFields}}
\newcommand\LEDestructuring[0]{\textsf{LE\_Destructuring}}

% Statement labels
\newcommand\SPass[0]{\textsf{S\_Pass}}
\newcommand\SAssign[0]{\textsf{S\_Assign}}
\newcommand\SReturn[0]{\textsf{S\_Return}}
\newcommand\SSeq[0]{\textsf{S\_Seq}}
\newcommand\SCall[0]{\textsf{S\_Call}}
\newcommand\SCond[0]{\textsf{S\_Cond}}
\newcommand\SCase[0]{\textsf{S\_Case}}
\newcommand\SDecl[0]{\textsf{S\_Decl}}
\newcommand\SAssert[0]{\textsf{S\_Assert}}
\newcommand\SWhile[0]{\textsf{S\_While}}
\newcommand\SRepeat[0]{\textsf{S\_Repeat}}
\newcommand\SFor[0]{\textsf{S\_For}}
\newcommand\SThrow[0]{\textsf{S\_Throw}}
\newcommand\STry[0]{\textsf{S\_Try}}
\newcommand\SPrint[0]{\textsf{S\_Print}}

% Literal labels
\newcommand\lint[0]{\textsf{L\_Int}}
\newcommand\lbool[0]{\textsf{L\_Bool}}
\newcommand\lreal[0]{\textsf{L\_Real}}
\newcommand\lbitvector[0]{\textsf{L\_Bitvector}}
\newcommand\lstring[0]{\textsf{L\_String}}

\newcommand\True[0]{\hyperlink{def-true}{\texttt{TRUE}}}
\newcommand\False[0]{\hyperlink{def-false}{\texttt{FALSE}}}

% Type labels
\newcommand\TInt[0]{\textsf{T\_Int}}
\newcommand\TReal[0]{\textsf{T\_Real}}
\newcommand\TString[0]{\textsf{T\_String}}
\newcommand\TBool[0]{\textsf{T\_Bool}}
\newcommand\TBits[0]{\textsf{T\_Bits}}
\newcommand\TEnum[0]{\textsf{T\_Enum}}
\newcommand\TTuple[0]{\textsf{T\_Tuple}}
\newcommand\TArray[0]{\textsf{T\_Array}}
\newcommand\TRecord[0]{\textsf{T\_Record}}
\newcommand\TException[0]{\textsf{T\_Exception}}
\newcommand\TNamed[0]{\textsf{T\_Named}}

\newcommand\BitFieldSimple[0]{\textsf{BitField\_Simple}}
\newcommand\BitFieldNested[0]{\textsf{BitField\_Nested}}
\newcommand\BitFieldType[0]{\textsf{BitField\_Type}}

\newcommand\ConstraintExact[0]{\textsf{Constraint\_Exact}}
\newcommand\ConstraintRange[0]{\textsf{Constraint\_Range}}

% Array index labels
\newcommand\ArrayLengthExpr[0]{\textsf{ArrayLength\_Expr}}
\newcommand\ArrayLengthEnum[0]{\textsf{ArrayLength\_Enum}}

% Slice labels
\newcommand\SliceSingle[0]{\textsf{Slice\_Single}}
\newcommand\SliceRange[0]{\textsf{Slice\_Range}}
\newcommand\SliceLength[0]{\textsf{Slice\_Length}}
\newcommand\SliceStar[0]{\textsf{Slice\_Star}}

% Pattern labels
\newcommand\PatternAll[0]{\textsf{Pattern\_All}}
\newcommand\PatternAny[0]{\textsf{Pattern\_Any}}
\newcommand\PatternGeq[0]{\textsf{Pattern\_Geq}}
\newcommand\PatternLeq[0]{\textsf{Pattern\_Leq}}
\newcommand\PatternNot[0]{\textsf{Pattern\_Not}}
\newcommand\PatternRange[0]{\textsf{Pattern\_Range}}
\newcommand\PatternSingle[0]{\textsf{Pattern\_Single}}
\newcommand\PatternMask[0]{\textsf{Pattern\_Mask}}
\newcommand\PatternTuple[0]{\textsf{Pattern\_Tuple}}

% Local declarations
\newcommand\LDIDiscard[0]{\textsf{LDI\_Discard}}
\newcommand\LDIVar[0]{\textsf{LDI\_Var}}
\newcommand\LDITyped[0]{\textsf{LDI\_Typed}}
\newcommand\LDITuple[0]{\textsf{LDI\_Tuple}}

\newcommand\LDKVar[0]{\textsf{LDK\_Var}}
\newcommand\LDKConstant[0]{\textsf{LDK\_Constant}}
\newcommand\LDKLet[0]{\textsf{LDK\_Let}}

\newcommand\GDKConstant[0]{\textsf{GDK\_Constant}}
\newcommand\GDKConfig[0]{\textsf{GDK\_Config}}
\newcommand\GDKLet[0]{\textsf{GDK\_Let}}
\newcommand\GDKVar[0]{\textsf{GDK\_Var}}

\newcommand\SBASL[0]{\textsf{SB\_ASL}}
\newcommand\SBPrimitive[0]{\textsf{SB\_Primitive}}

\newcommand\STFunction[0]{\texttt{ST\_Function}}
\newcommand\STGetter[0]{\texttt{ST\_Getter}}
\newcommand\STEmptyGetter[0]{\texttt{ST\_EmptyGetter}}
\newcommand\STSetter[0]{\texttt{ST\_Setter}}
\newcommand\STEmptySetter[0]{\texttt{ST\_EmptySetter}}
\newcommand\STProcedure[0]{\texttt{ST\_Procedure}}

% Fields
\newcommand\funcname[0]{\text{name}}
\newcommand\funcparameters[0]{\text{parameters}}
\newcommand\funcargs[0]{\text{args}}
\newcommand\funcbody[0]{\text{body}}
\newcommand\funcreturntype[0]{\text{return\_type}}
\newcommand\funcsubprogramtype[0]{\text{subprogram\_type}}
\newcommand\GDkeyword[0]{\text{keyword}}
\newcommand\GDname[0]{\text{name}}
\newcommand\GDty[0]{\text{ty}}
\newcommand\GDinitialvalue[0]{\text{initial\_value}}

\newcommand\DFunc[0]{\texttt{D\_Func}}
\newcommand\DGlobalStorage[0]{\texttt{D\_GlobalStorage}}
\newcommand\DTypeDecl[0]{\texttt{D\_TypeDecl}}

\newcommand\identifier[0]{\textsf{identifier}}
%%%%%%%%%%%%%%%%%%%%%%%%%%%%%%%%%%%%%%%%%%%%%%%%%%

\newcommand\torexpr[0]{\hyperlink{def-rexpr}{\textsf{rexpr}}}

%%%%%%%%%%%%%%%%%%%%%%%%%%%%%%%%%%%%%%%%%%%%%%%%%%
%% Type System macros
%%%%%%%%%%%%%%%%%%%%%%%%%%%%%%%%%%%%%%%%%%%%%%%%%%
\newcommand\TypeErrorConfig[0]{\hyperlink{def-typeerrorconfig}{\texttt{\#TE}}}
\newcommand\TTypeError[0]{\hyperlink{def-ttypeerror}{\textsf{TTypeError}}}
\newcommand\TypeError[0]{\hyperlink{def-typeerror}{\textsf{TypeError}}}
\newcommand\TypeErrorVal[1]{\TypeError(\texttt{#1})}

\newcommand\Val[0]{\textsf{Val}}
\newcommand\aslto[0]{\longrightarrow}

\newcommand\staticenvs[0]{\hyperlink{def-staticenvs}{\mathbb{SE}}}
\newcommand\emptytenv[0]{\hyperlink{def-emptytenv}{\emptyset_{\tenv}}}
\newcommand\tstruct[0]{\hyperlink{def-tstruct}{\texttt{get\_structure}}}
\newcommand\astlabel[0]{\hyperlink{def-astlabel}{\textsf{ast\_label}}}
\newcommand\typesat[0]{\hyperlink{def-typesatisfies}{\texttt{type\_satisfies}}}

\newcommand\constantvalues[0]{\hyperlink{def-constantvalues}{\textsf{constant\_values}}}
\newcommand\globalstoragetypes[0]{\hyperlink{def-globalstoragetypes}{\textsf{global\_storage\_types}}}
\newcommand\localstoragetypes[0]{\hyperlink{def-localstoragetypes}{\textsf{local\_storage\_types}}}
\newcommand\returntype[0]{\hyperlink{def-returntype}{\textsf{return\_type}}}
\newcommand\declaredtypes[0]{\hyperlink{def-declaredtypes}{\textsf{declared\_types}}}
\newcommand\subtypes[0]{\hyperlink{def-subtypes}{\textsf{subtypes}}}
\newcommand\subprograms[0]{\hyperlink{def-subprograms}{\textsf{subprograms}}}
\newcommand\subprogramrenamings[0]{\hyperlink{def-subprogramrenamings}{\textsf{subprogram\_renamings}}}
\newcommand\parameters[0]{\hyperlink{def-parameters}{\textsf{parameters}}}

\newcommand\unopliterals[0]{\hyperlink{def-unopliterals}{\texttt{unop\_literals}}}
\newcommand\binopliterals[0]{\hyperlink{def-binopliterals}{\texttt{binop\_literals}}}

%%%%%%%%%%%%%%%%%%%%%%%%%%%%%%%%%%%%%%%%%%%%%%%%%%
%% Semantics macros
%%%%%%%%%%%%%%%%%%%%%%%%%%%%%%%%%%%%%%%%%%%%%%%%%%
\newcommand\aslrel[0]{\bigtimes}
\newcommand\vals[0]{\hyperlink{def-vals}{\mathbb{V}}}
\newcommand\nvliteral[1]{\hyperlink{def-nvliteral}{\texttt{NV\_Literal}}(#1)}
\newcommand\nvvector[1]{\hyperlink{def-nvvector}{\texttt{NV\_Vector}}(#1)}
\newcommand\nvrecord[1]{\hyperlink{def-nvrecord}{\texttt{NV\_Record}}(#1)}
\newcommand\nvint[0]{\hyperlink{def-nvint}{\texttt{Int}}}
\newcommand\nvbool[0]{\hyperlink{def-nvbool}{\texttt{Bool}}}
\newcommand\nvreal[0]{\hyperlink{nv-real}{\texttt{Real}}}
\newcommand\nvstring[0]{\hyperlink{def-nvstring}{\texttt{String}}}
\newcommand\nvbitvector[0]{\hyperlink{def-nvbitvector}{\texttt{Bitvector}}}
\newcommand\tint[0]{\hyperlink{def-tint}{\mathcal{Z}}}
\newcommand\tbool[0]{\hyperlink{def-tbool}{\mathcal{B}}}
\newcommand\treal[0]{\hyperlink{def-treal}{\mathcal{R}}}
\newcommand\tstring[0]{\hyperlink{def-tstring}{\mathcal{S}\mathcal{T}\mathcal{R}}}
\newcommand\tbitvector[0]{\hyperlink{def-tbitvector}{\mathcal{B}\mathcal{V}}}
\newcommand\tvector[0]{\hyperlink{def-tvector}{\mathcal{V}\mathcal{E}\mathcal{C}}}
\newcommand\trecord[0]{\hyperlink{def-trecord}{\mathcal{R}\mathcal{E}\mathcal{C}}}

\newcommand\evalrel[0]{\hyperlink{def-evalrel}{\textsf{eval}}}
\newcommand\evalarrow[0]{\xrightarrow{\evalrel}}
\newcommand\envs[0]{\hyperlink{def-envs}{\mathbb{E}}}
\newcommand\dynamicenvs[0]{\hyperlink{def-dynamicenvs}{\mathbb{DE}}}
\newcommand\xgraph[0]{\textsf{g}}
\newcommand\emptygraph[0]{{\hyperlink{def-emptygraph}{\emptyset_\xgraph}}}
\newcommand\asldata[0]{\hyperlink{def-asldata}{\mathtt{asl\_data}}}
\newcommand\aslctrl[0]{\hyperlink{def-aslctrl}{\mathtt{asl\_ctrl}}}
\newcommand\aslpo[0]{\hyperlink{def-aslpo}{\mathtt{asl\_po}}}

%%%%%%%%%%%%%%%%%%%%%%%%%%%%%%%%%%%%%%%%%%%%%%%%%%
% Semantics/Typing Shared macros
\newcommand\ProseTerminateAs[1]{\hyperlink{def-proseterminateas}{${}^{\sslash #1 }$}}
\newcommand\tenv[0]{\textsf{tenv}}
\newcommand\denv[0]{\textsf{denv}}
\newcommand\aslsep[0]{\mathbf{,}}

% Glossary
\newcommand\head[0]{\hyperlink{def-head}{\texttt{head}}}
\newcommand\tail[0]{\hyperlink{def-tail}{\texttt{tail}}}

%%%%%%%%%%%%%%%%%%%%%%%%%%%%%%%%%%%%%%%%%%%%%%%%%%
% LRM Ident info
\newcommand\ident[2]{\texttt{#1}\textsubscript{\texttt{\MakeUppercase{#2}}}}
\newcommand\identi[1]{\ident{I}{#1}}
\newcommand\identr[1]{\ident{R}{#1}}
\newcommand\identd[1]{\ident{D}{#1}}
\newcommand\identg[1]{\ident{G}{#1}}

%%%%%%%%%%%%%%%%%%%%%%%%%%%%%%%%%%%%%%%%%%%%%%%%%%
% Variable Names
\newcommand\id[0]{\texttt{id}}
\newcommand\idone[0]{\texttt{id1}}
\newcommand\idtwo[0]{\texttt{id2}}
\newcommand\idthree[0]{\texttt{id3}}
\newcommand\op[0]{\texttt{op}}
\newcommand\vx[0]{\texttt{x}}
\newcommand\va[0]{\texttt{a}}
\newcommand\vb[0]{\texttt{b}}
\newcommand\vbfour[0]{\texttt{b4}}
\newcommand\ve[0]{\texttt{e}}
\newcommand\vi[0]{\texttt{i}}
\newcommand\vj[0]{\texttt{j}}
\newcommand\vn[0]{\texttt{n}}
\newcommand\vtone[0]{\texttt{t1}}
\newcommand\vttwo[0]{\texttt{t2}}
\newcommand\vtthree[0]{\texttt{t3}}
\newcommand\vle[0]{\texttt{le}}
\newcommand\vargs[0]{\texttt{args}}
\newcommand\vnewargs[0]{\texttt{new\_args}}
\newcommand\vgone[0]{\texttt{g1}}
\newcommand\vgtwo[0]{\texttt{g2}}
\newcommand\envone[0]{\texttt{env1}}
\newcommand\envtwo[0]{\texttt{env2}}
\newcommand\newenv[0]{\texttt{new\_env}}
\newcommand\vv[0]{\texttt{v}}
\newcommand\vk[0]{\texttt{k}}
\newcommand\vvl[0]{\texttt{vl}}
\newcommand\vr[0]{\texttt{r}}
\newcommand\fieldmap[0]{\textit{field\_map}}
\newcommand\length[0]{\texttt{length}}
\newcommand\factor[0]{\texttt{factor}}
\newcommand\width[0]{\texttt{width}}
\newcommand\widthp[0]{\texttt{width'}}
\newcommand\vcs[0]{\texttt{vcs}}
\newcommand\vis[0]{\texttt{vis}}
\newcommand\vd[0]{\texttt{d}}
\newcommand\isone[0]{\texttt{is1}}
\newcommand\istwo[0]{\texttt{is2}}
\newcommand\dt[0]{\texttt{dt}}
\newcommand\ds[0]{\texttt{ds}}
\newcommand\widtht[0]{\texttt{width\_t}}
\newcommand\widths[0]{\texttt{width\_s}}
\newcommand\ws[0]{\texttt{w\_s}}
\newcommand\wt[0]{\texttt{w\_t}}
\newcommand\bfss[0]{\texttt{bfs\_s}}
\newcommand\bfst[0]{\texttt{bfs\_t}}
\newcommand\bfsone[0]{\texttt{bfs1}}
\newcommand\bfstwo[0]{\texttt{bfs2}}
\newcommand\vlengtht[0]{\texttt{length\_t}}
\newcommand\vlengths[0]{\texttt{length\_s}}
\newcommand\vlengthexprt[0]{\texttt{length\_expr\_t}}
\newcommand\vlengthexprs[0]{\texttt{length\_expr\_s}}
\newcommand\vtyt[0]{\texttt{ty\_t}}
\newcommand\vtys[0]{\texttt{ty\_s}}
\newcommand\vnamet[0]{\texttt{name\_t}}
\newcommand\vnames[0]{\texttt{name\_s}}
\newcommand\vlis[0]{\texttt{lis\_s}}
\newcommand\vlit[0]{\texttt{lis\_t}}
\newcommand\vfieldss[0]{\texttt{fields\_s}}
\newcommand\vfieldst[0]{\texttt{fields\_t}}
\newcommand\vnamest[0]{\texttt{names\_t}}
\newcommand\vnamess[0]{\texttt{names\_s}}
\newcommand\vsstruct[0]{\texttt{s\_struct}}
\newcommand\vtstruct[0]{\texttt{t\_struct}}
\newcommand\vtystruct[0]{\texttt{ty\_struct}}
\newcommand\calltype[0]{\texttt{call\_type}}
\newcommand\eqsone[0]{\texttt{eqs1}}
\newcommand\eqstwo[0]{\texttt{eqs2}}
\newcommand\eqsthree[0]{\texttt{eqs3}}
\newcommand\eqsthreep[0]{\texttt{eqs3'}}
\newcommand\eqsfour[0]{\texttt{eqs4}}
\newcommand\retty[0]{\texttt{ret\_ty}}
\newcommand\rettyone[0]{\texttt{ret\_ty1}}
\newcommand\rettyopt[0]{\texttt{ret\_ty\_opt}}
\newcommand\exprs[0]{\texttt{exprs}}
\newcommand\exprsone[0]{\texttt{exprs1}}
\newcommand\typedexpr[0]{\texttt{typed\_exprs}}
\newcommand\typedexprs[0]{\texttt{typed\_exprs}}
\newcommand\typedexprsone[0]{\texttt{typed\_exprs1}}
\newcommand\cst[0]{\texttt{cs\_t}}
\newcommand\css[0]{\texttt{cs\_s}}
\newcommand\vlione[0]{\texttt{li1}}
\newcommand\vlitwo[0]{\texttt{li2}}
\newcommand\structone[0]{\texttt{struct1}}
\newcommand\structtwo[0]{\texttt{struct2}}
\newcommand\newty[0]{\texttt{new\_ty}}
\newcommand\tenvp[0]{\texttt{tenv'}}
\newcommand\tenvthree[0]{\texttt{tenv3}}
\newcommand\tenvthreep[0]{\texttt{tenv3'}}
\newcommand\tenvfour[0]{\texttt{tenv4}}
\newcommand\tenvfive[0]{\texttt{tenv5}}
\newcommand\funcsig[0]{\texttt{func\_sig}}
\newcommand\funcsigp[0]{\texttt{func\_sig'}}
\newcommand\gsd[0]{\texttt{gsd}}
\newcommand\gsdp[0]{\texttt{gsd'}}
\newcommand\newgsd[0]{\texttt{new\_gsd}}
\newcommand\decls[0]{\texttt{decls}}
\newcommand\declsp[0]{\texttt{decls'}}
\newcommand\ordereddecls[0]{\texttt{ordered\_decls}}
\newcommand\constraints[0]{\texttt{constraints}}
\newcommand\constraintsp[0]{\texttt{constraints'}}
\newcommand\ewidth[0]{\texttt{e\_width}}
\newcommand\ewidthp[0]{\texttt{e\_width'}}
\newcommand\bitfieldsp[0]{\texttt{bitfields'}}
\newcommand\bitfieldspp[0]{\texttt{bitfields''}}
\newcommand\fieldsp[0]{\texttt{fields'}}
\newcommand\vstmt[0]{\texttt{stmt}}
\newcommand\newstmt[0]{\texttt{new\_stmt}}
\newcommand\newstmtp[0]{\texttt{new\_stmt'}}
\newcommand\newopt[0]{\texttt{new\_opt}}
\newcommand\callerargtyped[0]{\texttt{caller\_arg\_typed}}
\newcommand\callerargtypes[0]{\texttt{caller\_arg\_types}}
\newcommand\callee[0]{\texttt{callee}}
\newcommand\calleeargtypes[0]{\texttt{callee\_arg\_types}}
\newcommand\calleeparam[0]{\texttt{callee\_param}}
\newcommand\calleeparams[0]{\texttt{callee\_params}}
\newcommand\calleeparamsone[0]{\texttt{callee\_params1}}
\newcommand\extranargs[0]{\texttt{extra\_nargs}}
\newcommand\veq[0]{\texttt{eq}}
\newcommand\newconstraints[0]{\texttt{new\_constraints}}
\newcommand\newc[0]{\texttt{new\_c}}
\newcommand\tysp[0]{\texttt{tys'}}
\newcommand\ttyp[0]{\texttt{ty'}}
\newcommand\field[0]{\texttt{field}}
\newcommand\newfield[0]{\texttt{new\_field}}
\newcommand\newfields[0]{\texttt{new\_fields}}
\newcommand\names[0]{\texttt{names}}
\newcommand\bits[0]{\texttt{bits}}
\newcommand\veonep[0]{\texttt{e1'}}
\newcommand\vetwop[0]{\texttt{e2'}}
\newcommand\vethreep[0]{\texttt{e2'}}
\newcommand\vepp[0]{\texttt{e''}}
\newcommand\vtpp[0]{\texttt{t''}}
\newcommand\vsp[0]{\texttt{s'}}
\newcommand\vspp[0]{\texttt{s''}}
\newcommand\econdp[0]{\texttt{e\_cond'}}
\newcommand\etruep[0]{\texttt{e\_true'}}
\newcommand\efalsep[0]{\texttt{e\_false'}}
\newcommand\slicesp[0]{\texttt{slices'}}
\newcommand\tep[0]{\texttt{t\_e'}}
\newcommand\structtep[0]{\texttt{struct\_t\_e'}}
\newcommand\structtleone[0]{\texttt{struct\_t\_le1}}
\newcommand\varg[0]{\texttt{arg}}
\newcommand\vnewarg[0]{\texttt{new\_arg}}
\newcommand\vargsp[0]{\texttt{args'}}
\newcommand\eqsp[0]{\texttt{eqs'}}
\newcommand\namep[0]{\texttt{name'}}
\newcommand\namepp[0]{\texttt{name''}}
\newcommand\eindexp[0]{\texttt{e\_index'}}
\newcommand\tindexp[0]{\texttt{t\_index'}}
\newcommand\fieldtypes[0]{\texttt{field\_types}}
\newcommand\tspecp[0]{\texttt{t\_spec'}}
\newcommand\pairs[0]{\texttt{pairs}}
\newcommand\initializedfields[0]{\texttt{initialized\_fields}}
\newcommand\vteone[0]{\texttt{t\_e1}}
\newcommand\vtetwo[0]{\texttt{t\_e2}}
\newcommand\vtethree[0]{\texttt{t\_e3}}
\newcommand\vtefour[0]{\texttt{t\_e4}}
\newcommand\vtefive[0]{\texttt{t\_e5}}
\newcommand\ldip[0]{\texttt{ldi'}}
\newcommand\newldip[0]{\texttt{new\_ldi'}}
\newcommand\newldis[0]{\texttt{new\_ldis'}}
\newcommand\newlocalstoragetypes[0]{\texttt{new\_local\_storagetypes}}
\newcommand\eoffset[0]{\texttt{offset}}
\newcommand\eoffsetp[0]{\texttt{offset'}}
\newcommand\elength[0]{\texttt{length}}
\newcommand\elengthp[0]{\texttt{length'}}
\newcommand\efactor[0]{\texttt{factor}}
\newcommand\toffset[0]{\texttt{t\_offset}}
\newcommand\tlength[0]{\texttt{t\_length}}
\newcommand\prelength[0]{\texttt{pre\_length}}
\newcommand\prelengthp[0]{\texttt{pre\_length'}}
\newcommand\preoffset[0]{\texttt{pre\_offset}}
\newcommand\newp[0]{\texttt{new\_p}}
\newcommand\newq[0]{\texttt{new\_q}}
\newcommand\newli[0]{\texttt{new\_li}}
\newcommand\vlp[0]{\texttt{l'}}
\newcommand\useset[0]{\texttt{use\_set}}
\newcommand\gdk[0]{\texttt{gdk}}
\newcommand\testruct[0]{\texttt{t\_e\_struct}}
\newcommand\vtestruct[0]{\texttt{t\_e\_struct}}
\newcommand\vq[0]{\texttt{q}}
\newcommand\vteonestruct[0]{\texttt{t\_e1\_struct}}
\newcommand\vtetwostruct[0]{\texttt{t\_e2\_struct}}
\newcommand\vlip[0]{\texttt{li'}}
\newcommand\vfp[0]{\texttt{f'}}
\newcommand\body[0]{\texttt{body}}
\newcommand\newbody[0]{\texttt{new\_body}}
\newcommand\newd[0]{\texttt{new\_d}}
\newcommand\funcsigone[0]{\texttt{func\_sig1}}
\newcommand\namesubs[0]{\texttt{name\_s}}
\newcommand\namesubt[0]{\texttt{name\_t}}
\newcommand\vstructt[0]{\texttt{struct\_t}}
\newcommand\vstructs[0]{\texttt{struct\_s}}
\newcommand\vcsnew[0]{\texttt{cs\_new}}
\newcommand\vbot[0]{\texttt{bot}}
\newcommand\vvtop[0]{\texttt{top}}
\newcommand\vneg[0]{\texttt{neg}}
\newcommand\vtonestruct[0]{\texttt{t1\_struct}}
\newcommand\vttwostruct[0]{\texttt{t2\_struct}}
\newcommand\vtoneanon[0]{\texttt{t1\_anon}}
\newcommand\vttwoanon[0]{\texttt{t2\_anon}}
\newcommand\vcsone[0]{\texttt{cs1}}
\newcommand\vcstwo[0]{\texttt{cs2}}
\newcommand\vanons[0]{\texttt{anon\_s}}
\newcommand\vanont[0]{\texttt{anon\_t}}
\newcommand\vet[0]{\texttt{e\_t}}
\newcommand\ves[0]{\texttt{e\_s}}
\newcommand\vtsupers[0]{\texttt{t\_supers}}
\newcommand\vindex[0]{\texttt{index}}
\newcommand\widthsum[0]{\texttt{width\_sum}}
\newcommand\ttyone[0]{\texttt{ty1}}
\newcommand\ttytwo[0]{\texttt{ty2}}
\newcommand\vpat[0]{\texttt{pat}}
\newcommand\vpatp[0]{\texttt{pat'}}
\newcommand\lesp[0]{\texttt{les'}}
\newcommand\lesone[0]{\texttt{les1}}
\newcommand\vteeq[0]{\texttt{t\_e\_eq}}
\newcommand\vtleonestruct[0]{\texttt{t\_le1\_struct}}
\newcommand\tsubs[0]{\texttt{t\_s}}
\newcommand\vdebug[0]{\texttt{debug}}
\newcommand\vreone[0]{\texttt{re1}}
\newcommand\vtre[0]{\texttt{t\_re}}
\newcommand\vtep[0]{\texttt{t\_e'}}
\newcommand\vsonep[0]{\texttt{s1'}}
\newcommand\vstwop[0]{\texttt{s2'}}
\newcommand\vcases[0]{\texttt{cases}}
\newcommand\vcasesone[0]{\texttt{cases1}}
\newcommand\vcase[0]{\texttt{case}}
\newcommand\vcaseone[0]{\texttt{case1}}
\newcommand\vpzero[0]{\texttt{p0}}
\newcommand\vwzero[0]{\texttt{w0}}
\newcommand\vszero[0]{\texttt{s0}}
\newcommand\vewzero[0]{\texttt{e\_w0}}
\newcommand\vewone[0]{\texttt{e\_w1}}
\newcommand\vtwe[0]{\texttt{twe}}
\newcommand\otherwisep[0]{\texttt{otherwise'}}
\newcommand\vcp[0]{\texttt{c'}}
\newcommand\catchersp[0]{\texttt{catchers'}}
\newcommand\nameopt[0]{\texttt{name\_opt}}
\newcommand\botcs[0]{\texttt{bot\_cs}}
\newcommand\topcs[0]{\texttt{top\_cs}}
\newcommand\ebot[0]{\texttt{e\_bot}}
\newcommand\etop[0]{\texttt{e\_top}}
\newcommand\vbf[0]{\texttt{bf}}
\newcommand\vbfone[0]{\texttt{bf1}}
\newcommand\vbftwo[0]{\texttt{bf2}}
\newcommand\bfstwop[0]{\texttt{bfs2'}}
\newcommand\vpp[0]{\texttt{p'}}
\newcommand\acc[0]{\texttt{acc}}
\newcommand\accp[0]{\texttt{acc'}}
\newcommand\annotateddecls[0]{\texttt{annotated\_decls}}
\newcommand\funcdef[0]{\texttt{func\_def}}
\newcommand\funcdefone[0]{\texttt{func\_def1}}
\newcommand\newfuncdef[0]{\texttt{new\_func\_def}}
\newcommand\newfuncsig[0]{\texttt{new\_func\_sig}}
\newcommand\bd[0]{\texttt{bd}}
\newcommand\initialvalue[0]{\texttt{initial\_value}}
\newcommand\keyword[0]{\texttt{keyword}}
\newcommand\subpgmtype[0]{\texttt{subpgm\_type}}
\newcommand\subpgmtypeone[0]{\texttt{subpgm\_type1}}
\newcommand\subpgmtypetwo[0]{\texttt{subpgm\_type2}}
\newcommand\formals[0]{\texttt{formals}}
\newcommand\formaltypes[0]{\texttt{formal\_types}}
\newcommand\nameformaltypes[0]{\texttt{name\_formal\_types}}
\newcommand\nameargs[0]{\texttt{name\_args}}
\newcommand\nameformals[0]{\texttt{name\_formals}}
\newcommand\namesubpgmtype[0]{\texttt{name\_subpgmtype}}
\newcommand\othernames[0]{\texttt{other\_names}}
\newcommand\newrenamings[0]{\texttt{new\_renamings}}
\newcommand\issetter[0]{\texttt{is\_setter}}
\newcommand\rettype[0]{\texttt{ret\_type}}
\newcommand\argtypes[0]{\texttt{arg\_types}}
\newcommand\argtypesp[0]{\texttt{arg\_types'}}
\newcommand\watendgettertype[0]{\texttt{wanted\_getter\_type}}
\newcommand\ret[0]{\texttt{ret}}
\newcommand\renamingset[0]{\texttt{renaming\_set}}
\newcommand\formaltys[0]{\texttt{f\_tys}}
\newcommand\argtys[0]{\texttt{a\_tys}}
\newcommand\candidates[0]{\texttt{candidates}}
\newcommand\candidatesone[0]{\texttt{candidates1}}
\newcommand\matches[0]{\texttt{matches}}
\newcommand\matchesone[0]{\texttt{matches1}}
\newcommand\matchingrenamings[0]{\texttt{matching\_renamings}}
\newcommand\vres[0]{\texttt{res}}
\newcommand\matchedname[0]{\texttt{matched\_name}}
\newcommand\potentialparams[0]{\texttt{potential\_params}}
\newcommand\declaredparams[0]{\texttt{declared\_params}}
\newcommand\argparams[0]{\texttt{arg\_params}}
\newcommand\vreturntype[0]{\texttt{return\_type}}
\newcommand\vparameters[0]{\texttt{parameters}}
\newcommand\vparametersone[0]{\texttt{parameters1}}
\newcommand\vparams[0]{\texttt{params}}
\newcommand\vparam[0]{\texttt{param}}
\newcommand\vused[0]{\texttt{used}}
\newcommand\vusedone[0]{\texttt{used1}}
\newcommand\tyoptp[0]{\texttt{ty\_opt'}}
\newcommand\expropt[0]{\texttt{expr\_opt}}
\newcommand\exproptp[0]{\texttt{expr\_opt'}}
\newcommand\initialvaluetype[0]{\texttt{initial\_value\_type}}
\newcommand\initialvaluep[0]{\texttt{initial\_value'}}
\newcommand\typeannotation[0]{\texttt{type\_annotation}}
\newcommand\declaredt[0]{\texttt{declared\_t}}
\newcommand\ids[0]{\texttt{ids}}
\newcommand\extrafields[0]{\texttt{extra\_fields}}
\newcommand\vsuper[0]{\texttt{super}}
\newcommand\idsone[0]{\texttt{ids1}}
\newcommand\idstwo[0]{\texttt{ids2}}
\newcommand\newreturntype[0]{\texttt{new\_return\_type}}
\newcommand\bvtwo[0]{\texttt{bv2}}
\newcommand\positionsone[0]{\texttt{positions1}}
\newcommand\positionstwo[0]{\texttt{positions2}}
\newcommand\vms[0]{\texttt{vms}}
\newcommand\vmsone[0]{\texttt{vms1}}
\newcommand\vmstwo[0]{\texttt{vms2}}
\newcommand\monoms[0]{\texttt{monoms}}
\newcommand\monome[0]{\texttt{monome}}
\newcommand\monomsone[0]{\texttt{monoms1}}
\newcommand\compare[0]{\texttt{compare}}
\newcommand\vqone[0]{\texttt{q1}}
\newcommand\vqtwo[0]{\texttt{q2}}
\newcommand\tty[0]{\texttt{ty}}
\newcommand\aritymatch[0]{\texttt{arity\_match}}
\newcommand\paramname[0]{\texttt{param\_name}}
\newcommand\calleeargs[0]{\texttt{callee\_args}}
\newcommand\callerargstyped[0]{\texttt{caller\_args\_typed}}
\newcommand\callerargstypedone[0]{\texttt{caller\_args\_typed1}}
\newcommand\calleex[0]{\texttt{callee\_x}}
\newcommand\callerty[0]{\texttt{caller\_ty}}
\newcommand\callere[0]{\texttt{caller\_e}}
\newcommand\calleearg[0]{\texttt{callee\_arg}}
\newcommand\calleeargone[0]{\texttt{callee\_arg1}}
\newcommand\callerarg[0]{\texttt{caller\_arg}}
\newcommand\calleeargname[0]{\texttt{callee\_arg\_name}}
\newcommand\calleeargisparam[0]{\texttt{callee\_arg\_is\_param}}
\newcommand\calleeargsone[0]{\texttt{callee\_args\_one}}
\newcommand\callerargtypesone[0]{\texttt{caller\_arg\_types1}}
\newcommand\calleeparamt[0]{\texttt{callee\_param\_t}}
\newcommand\calleeparamtrenamed[0]{\texttt{callee\_param\_t\_renamed}}
\newcommand\callerparame[0]{\texttt{caller\_param\_e}}
\newcommand\callerparamt[0]{\texttt{caller\_param\_t}}
\newcommand\callerparamname[0]{\texttt{caller\_param\_name}}
\newcommand\calleerettyopt[0]{\texttt{callee\_ret\_ty\_opt}}
\newcommand\newtys[0]{\texttt{new\_tys}}
\newcommand\substs[0]{\texttt{substs}}
\newcommand\vesp[0]{\texttt{es'}}
\newcommand\paramargs[0]{\texttt{param\_args}}

\input{ASLTypingLines}
\input{ASLTypeSatisfactionLines}
\input{ASLASTLines}
\newcommand{\tests}{../tests/ASLTypingReference.t/}

\newcommand\todocomment[1]{}

%%%%%%%%%%%%%%%%%%%%%%%%%%%%%%%%%%%%%%%%%%%%%%%%%%

% The following macros will be moved to ASLmacros.tex when we unify all reference documents.
\newcommand\ReadEffect[0]{\textsf{ReadEffect}}
\newcommand\Normal[0]{\textsf{Normal}}
\newcommand\ThrowingConfig[0]{\texttt{\#T}}
\newcommand\OrAbnormal[0]{\terminateas \ThrowingConfig, \ErrorConfig}
\newcommand\vg[0]{\texttt{g}}
\newcommand\env[0]{\texttt{env}}
\newcommand\parallelcomp[0]{\parallel}
\newcommand\binoprel[0]{\texttt{binop}}
\newcommand\evalexprsef[1]{\hyperlink{def-evalexprsef}{\texttt{eval\_expr\_sef}}(#1)}
\newcommand\XGraphs[0]{\mathcal{G}}
\newcommand\TError[0]{\textsf{TDynError}}
\newcommand\ErrorConfig[0]{\hyperlink{def-errorconfig}{\texttt{\#DE}}}

\newcommand\dynamicdomain[0]{\hyperlink{def-dyndomain}{\textsf{dyn\_dom}}}
\newcommand\literals[0]{\mathcal{L}}

\newcommand\TypeError[0]{\hyperlink{def-typeerror}{\textsf{TypeError}}}
\newcommand\TypeErrorVal[1]{\TypeError(\texttt{"ERROR[#1]"})}
\newcommand\TTypeError[0]{\hyperlink{def-ttypeerror}{\textsf{TTypeError}}}
\newcommand\TypeErrorConfig[0]{\hyperlink{def-typeerrorconfig}{\texttt{\#TE}}}
\newcommand\OrTypeError[0]{\terminateas \TypeErrorConfig}
%\newcommand\ProseOrTypeError[0]{or a type error that short-circuits the entire rule}
\newcommand\ProseOrTypeError[0]{\hyperlink{def-proseortypeerror}{$^{\TypeErrorConfig}$}}

\newcommand\staticbinop[0]{\texttt{binop}}

\newcommand\annotaterel[0]{\hyperlink{def-annotaterel}{\textsf{type}}}
\newcommand\typearrow[0]{\xrightarrow{\annotaterel}}
\newcommand\isbuiltinsingular[0]{\hyperlink{def-isbuiltinsingular}{\texttt{is\_builtin\_singular}}}
\newcommand\isbuiltinaggregate[0]{\hyperlink{def-isbuiltinaggregate}{\texttt{is\_builtin\_aggregate}}}
\newcommand\isbuiltin[0]{\hyperlink{def-isbuiltin}{\texttt{is\_builtin}}}
\newcommand\isnamed[0]{\hyperlink{def-isnamed}{\texttt{is\_named}}}
\newcommand\isanonymous[0]{\hyperlink{def-isanonymous}{\texttt{is\_anonymous}}}
\newcommand\issingular[0]{\hyperlink{def-issingular}{\texttt{is\_singular}}}
\newcommand\isaggregate[0]{\hyperlink{def-isaggregate}{\texttt{is\_aggregate}}}
\newcommand\isnonprimitive[0]{\hyperlink{def-isnonprimitive}{\texttt{is\_non\_primitive}}}
\newcommand\isprimitive[0]{\hyperlink{def-isprimitive}{\texttt{is\_primitive}}}

\newcommand\isunconstrainedinteger[0]{\hyperlink{def-isunconstrainedinteger}{\textsf{is\_unconstrained\_integer}}}
\newcommand\isunderconstrainedinteger[0]{\hyperlink{def-isunderconstrainedinteger}{\textsf{is\_under\_constrained\_integer}}}
\newcommand\iswellconstrainedinteger[0]{\hyperlink{def-iswellconstrainedinteger}{\textsf{is\_well\_constrained\_integer}}}
\newcommand\unconstrainedinteger[0]{\hyperlink{def-unconstrainedinteger}{\textsf{unconstrained\_integer}}}

\newcommand\checkconstrainedinteger[0]{\hyperlink{def-checkconstrainedinteger}{\texttt{check\_constrained\_integer}}}

\newcommand\staticeval[0]{\hyperlink{def-staticeval}{\texttt{static\_eval}}}
\newcommand\isstaticallyevaluable[0]{\texttt{check\_statically\_evaluable}}

\newcommand\makeanonymous[0]{\hyperlink{def-makeanonymous}{\texttt{make\_anonymous}}}
\newcommand\subtypesrel[0]{\hyperlink{def-subtypesrel}{\texttt{is\_subtype}}}
\newcommand\structsubtypesat[0]{\hyperlink{def-structsubtypesat}{\texttt{structural\_subtype\_satisfies}}}
\newcommand\domsubtypesat[0]{\hyperlink{def-domsubtypesat}{\texttt{domain\_subtype\_satisfies}}}
\newcommand\subtypesat[0]{\hyperlink{def-subtypesat}{\texttt{subtype\_satisfies}}}
\newcommand\checktypesat[0]{\hyperlink{def-checktypesat}{\texttt{checked\_typesat}}}
\newcommand\typeclashes[0]{\hyperlink{def-typeclashes}{\texttt{type\_clashes}}}
\newcommand\lca[0]{\hyperlink{def-lowestcommonancestor}{\texttt{lowest\_common\_ancestor}}}
\newcommand\namedlca[0]{\hyperlink{def-namedlowestcommonancestor}{\texttt{named\_lowest\_common\_ancestor}}}
\newcommand\Supers{\textsf{Supers}}
\newcommand\bitfieldsincluded[0]{\hyperlink{def-bitfieldsincluded}{\texttt{bitfields\_included}}}
\newcommand\membfs[0]{\hyperlink{def-membfs}{\texttt{mem\_bfs}}}
\newcommand\instantiate[0]{\texttt{instantiate}}
\newcommand\canbeinitializedwith[0]{\texttt{can\_be\_initialized\_with}}
\newcommand\getbitvectorwidth[0]{\hyperlink{def-getbitvectorwidth}{\texttt{get\_bitvector\_width}}}
\newcommand\checkbitsequalwidth[0]{\hyperlink{def-checkbitsequalwidth}{\texttt{check\_bits\_equal\_width}}}
\newcommand\findsubprogram[0]{\hyperlink{def-findsubprogram}{\texttt{find\_subprogram}}}
\newcommand\subprogramtypeclash[0]{\texttt{subprogram\_type\_clash}}
\newcommand\hassubprogramtypeclash[0]{\texttt{subprogram\_type\_clash}}
\newcommand\subprogramclash[0]{\texttt{subprogram\_clash}}
\newcommand\argsclash[0]{\texttt{args\_clash}}
\newcommand\bitfieldgetname[0]{\hyperlink{def-bitfieldgetname}{\texttt{bitfield\_get\_name}}}
\newcommand\pairstomap[0]{\hyperlink{def-pairstomap}{\texttt{pairs\_to\_map}}}
\newcommand\assocopt[0]{\hyperlink{def-assocopt}{\texttt{assoc\_opt}}}
\newcommand\annotatefieldinit[0]{\hyperlink{def-annotatefieldinit}{\texttt{annotate\_field\_init}}}
\newcommand\annotatestaticconstrainedinteger[0]{\hyperlink{def-annotatestaticconstrainedinteger}{\texttt{annotate\_static\_constrained\_integer}}}
\newcommand\checkstructurelabel[0]{\hyperlink{def-checkstructurelabel}{\texttt{check\_structure}}}
\newcommand\checkstructureinteger[0]{\hyperlink{def-checkstructureinteger}{\texttt{check\_structure\_integer}}}
\newcommand\checkstaticallyevaluable[0]{\hyperlink{def-checkstaticallyevaluable}{\texttt{check\_statically\_evaluable}}}
\newcommand\storageispure[0]{\hyperlink{def-storageispure}{\texttt{storage\_is\_pure}}}
\newcommand\negateconstraint[0]{\hyperlink{def-negateconstraint}{\texttt{negate\_constraint}}}
\newcommand\getwellconstrainedstructure[0]{\hyperlink{def-getwellconstrainedstructure}{\texttt{get\_well\_constrained\_structure}}}
\newcommand\towellconstrained[0]{\hyperlink{def-towellconstrained}{\texttt{to\_well\_constrained}}}
\newcommand\annotatelebits[0]{\hyperlink{def-annotatelebits}{\texttt{annotate\_lebits}}}
\newcommand\annotatecase[1]{\hyperlink{def-annotatecase}{\texttt{annotate\_case}}(#1)}
\newcommand\getforconstraints[0]{\hyperlink{def-getforconstraints}{\texttt{for\_constraints}}}
\newcommand\minconstraints[0]{\hyperlink{def-minconstraints}{\texttt{min\_constraints}}}
\newcommand\maxconstraints[0]{\hyperlink{def-maxconstraints}{\texttt{max\_constraints}}}
\newcommand\minconstraint[0]{\hyperlink{def-minconstraint}{\texttt{min\_constraint}}}
\newcommand\maxconstraint[0]{\hyperlink{def-maxconstraint}{\texttt{max\_constraint}}}
\newcommand\findbitfieldopt[0]{\hyperlink{def-findbitfieldopt}{\texttt{find\_bitfield\_opt}}}
\newcommand\lookupconstant[0]{\hyperlink{def-lookupconstant}{\texttt{lookup\_constant}}}
\newcommand\typeof[0]{\hyperlink{def-typeof}{\texttt{type\_of}}}

% Symbolic equivalence testing macros
\newcommand\typeequal[0]{\hyperlink{def-typeequal}{\texttt{type\_equal}}}
\newcommand\exprequal[0]{\hyperlink{def-exprequal}{\texttt{expr\_equal}}}
\newcommand\bitwidthequal[0]{\hyperlink{def-bitwidthequal}{\texttt{bitwidth\_equal}}}
\newcommand\bitfieldequal[0]{\hyperlink{def-bitfieldequal}{\texttt{bitfield\_equal}}}
\newcommand\bitfieldsequal[0]{\hyperlink{def-bitfieldsequal}{\texttt{bitfields\_equal}}}
\newcommand\reduceexpr[0]{\texttt{reduce\_expr}}

\newcommand\subsumes[0]{\hyperlink{def-subsumes}{\texttt{subsumes}}}
\newcommand\symsubsumes[0]{\hyperlink{def-symsubsumes}{\texttt{sym\_subsumes}}}

\newcommand\toir[0]{\hyperlink{def-toir}{\texttt{to\_ir}}}
\newcommand\toircase[0]{\hyperlink{def-toircase}{\texttt{to\_ir\_case}}}
\newcommand\Prod[0]{\hyperlink{def-prod}{\textsf{Prod}}}
\newcommand\Sum[0]{\hyperlink{def-sum}{\textsf{Sum}}}
\newcommand\monomial[0]{\hyperlink{def-monomial}{\textsf{monomial}}}
\newcommand\polynomial[0]{\hyperlink{def-polynomial}{\textsf{polynomial}}}
\newcommand\addpolynomials[0]{\hyperlink{def-addpolynomials}{\texttt{add\_polynomials}}}
\newcommand\mulpolynomials[0]{\hyperlink{def-mulpolynomials}{\texttt{mul\_polynomials}}}
\newcommand\mulmonomials[0]{\hyperlink{def-mulmonomials}{\texttt{mul\_mononimials}}}

\newcommand\declaredtype[0]{\hyperlink{def-declaredtype}{\texttt{declared\_type}}}

\newcommand\fieldnames[0]{\hyperlink{def-fieldnames}{\texttt{field\_names}}}
\newcommand\fieldtype[0]{\hyperlink{def-fieldtype}{\texttt{field\_type}}}

\newcommand\eliteral[1]{\textsf{E\_Literal}(#1)}

% Glossary
\newcommand\structure[0]{\hyperlink{def-structure}{structure}}
\newcommand\underlyingtype[0]{\hyperlink{def-underlyingtype}{underlying type}}
\newcommand\symbolicdomain[0]{\hyperlink{def-symbolicdomain}{symbolic domain}}
\newcommand\typesatisfies[0]{\hyperlink{def-typesatisfies}{type-satisfies}}
\newcommand\typesatisfy[0]{\hyperlink{def-typesatisfies}{type-satisfy}}
\newcommand\checkedtypesatisfies[0]{\hyperlink{def-checktypesat}{checked-type-satisfies}}
\newcommand\typesatisft[0]{\hyperlink{def-typesatisfies}{type-satisft}}
\newcommand\subtypesatisfies[0]{\hyperlink{def-subtypesatisfies}{subtype-satisfies}}
\newcommand\subtypesatisfy[0]{\hyperlink{def-subtypesatisfies}{subtype-satisfisfy}}
\newcommand\typeequivalent[0]{\hyperlink{def-typeequal}{type-equivalent}}
\newcommand\bitwidthequivalent[0]{\hyperlink{def-bitwidthequal}{bitwidth-equivalent}}
\newcommand\typeclash[0]{\hyperlink{def-typeclashes}{type-clash}}
\newcommand\constrainedinteger[0]{\hyperlink{def-checkconstrainedinteger}{constrained integer}}
\newcommand\structureofinteger[0]{\hyperlink{def-checkstructureinteger}{structure of an integer}}
\newcommand\staticallyevaluable[0]{\hyperlink{def-staticallyevaluable}{statically evaluable}}
\newcommand\pure[0]{\hyperlink{def-storageispure}{pure}}
\newcommand\wellconstrainedstructure[0]{\hyperlink{def-getwellconstrainedstructure}{well-constrained structure}}
\newcommand\wellconstrainedversion[0]{\hyperlink{def-towellconstrained}{well-constrained version}}
\newcommand\namedlowestcommonancestor[0]{\hyperlink{def-namedlowestcommonancestor}{named lowest common ancestor}}

%%%%%%%%%%%%%%%%%%%%%%%%%%%%%%%%%%%%%%%%%%%%%%%%%%%%%%%%%%%%%%%%%%%%%%%%%%%%%%%%
%% Type functions
\newcommand\CheckUnop[0]{\hyperlink{def-checkunop}{\texttt{check\_unop}}}
\newcommand\CheckBinop[0]{\hyperlink{def-checkbinop}{\texttt{check\_binop}}}
\newcommand\constraintbinop[0]{\texttt{constraint\_binop}}
\newcommand\sliceswidth[0]{\texttt{slices\_width}}
\newcommand\annotatetype[1]{\hyperlink{def-annotatetype}{\texttt{annotate\_type}}(#1)}
\newcommand\annotateexpr[1]{\hyperlink{def-annotateexpr}{\texttt{annotate\_expr}}(#1)}
\newcommand\annotateexprlist[1]{\hyperlink{def-annotateexprs}{\texttt{annotate\_exprs}}(#1)}
\newcommand\annotatelexpr[1]{\hyperlink{def-annotatelexpr}{\texttt{annotate\_lexpr}}(#1)}
\newcommand\annotatearrayindex[0]{\texttt{annotate\_array\_index}}
\newcommand\annotateslice[0]{\hyperlink{def-annotateslice}{\texttt{annotate\_slice}}}
\newcommand\annotateslices[0]{\hyperlink{def-annotateslices}{\texttt{annotate\_slices}}}
\newcommand\annotatepattern[0]{\hyperlink{def-annotatepattern}{\texttt{annotate\_pattern}}}
\newcommand\annotatelocaldeclitem[1]{\hyperlink{def-annotatelocaldeclitem}{\texttt{annotate\_local\_decl\_item}}(#1)}
\newcommand\annotatestmt[1]{\hyperlink{def-annotatestmt}{\texttt{annotate\_stmt}}(#1)}
\newcommand\annotateblock[1]{\hyperlink{def-annotateblock}{\texttt{annotate\_block}}(#1)}
\newcommand\inlinesetter[1]{\texttt{setter\_should\_reduce\_to\_call\_s}(#1)}
\newcommand\annotatecall[1]{\hyperlink{def-annotatecall}{\texttt{annotate\_call}}(#1)}
\newcommand\annotatecatcher[1]{\hyperlink{def-annotatecatcher}{\texttt{annotate\_catcher}}(#1)}
\newcommand\reduceconstants[1]{\texttt{reduce\_constants}(#1)}
\newcommand\declarelocalconstant[1]{\texttt{declare\_local\_constant}(#1)}
\newcommand\annotatelocaldeclitemuninit[1]{\texttt{annotate\_local\_decl\_item\_uninit}(#1)}
\newcommand\checkvarnotinenv[1]{\hyperlink{def-checkvarnotinenv}{\texttt{check\_var\_not\_in\_env}}(#1)}
\newcommand\varinenv[1]{\hyperlink{def-varinenv}{\texttt{var\_in\_env}}(#1)}
\newcommand\annotatesubprogram[1]{\hyperlink{def-annotatesubprogram}{\texttt{annotate\_subprogram}}(#1)}
\newcommand\annotatedecl[1]{\hyperlink{def-annotatedecl}{\texttt{annotate\_decl}}(#1)}
\newcommand\declaredecl[1]{\hyperlink{def-declaredecl}{\texttt{declare\_decl}}(#1)}
\newcommand\annotatespec[1]{\texttt{annotate\_spec}(#1)}
\newcommand\evalexpr[1]{\texttt{eval\_expr}(#1)}
\newcommand\evalconstraint[1]{\texttt{eval\_constraint}(#1)}
\newcommand\annotateliteral[1]{\hyperlink{def-annotateliteral}{\texttt{annotate\_literal}}(#1)}
\newcommand\exprequalcase[0]{\hyperlink{def-exprequalcase}{\texttt{expr\_equal\_case}}}
\newcommand\exprequalnorm[0]{\hyperlink{def-exprequalnorm}{\texttt{expr\_equal\_norm}}}
\newcommand\slicesequal[0]{\hyperlink{def-slicesequal}{\texttt{slices\_equal}}}
\newcommand\sliceequal[0]{\hyperlink{def-sliceequal}{\texttt{slice\_equal}}}
\newcommand\constraintsequal[0]{\hyperlink{def-constraintsequal}{\texttt{constraints\_equal}}}
\newcommand\constraintequal[0]{\hyperlink{def-constraintequal}{\texttt{constraint\_equal}}}
\newcommand\arraylengthequal[0]{\hyperlink{def-arraylengthequal}{\texttt{array\_length\_equal}}}
\newcommand\literalequal[0]{\hyperlink{def-literalequal}{\texttt{literal\_equal}}}
\newcommand\findcheckdeduce[0]{\hyperlink{def-findcheckdeduce}{\texttt{find\_check\_deduce}}}
%\newcommand\renametyeqs[0]{\texttt{rename\_ty\_eqs}}
\newcommand\annotatebitfield[0]{\hyperlink{def-annotatebitfield}{\texttt{annotate\_bitfield}}}
\newcommand\annotatebitfields[0]{\hyperlink{def-annotatebitfields}{\texttt{annotate\_bitfields}}}
\newcommand\checknoduplicates[0]{\hyperlink{def-checknoduplicates}{\texttt{check\_no\_duplicates}}}
\newcommand\reduceslicestocall[0]{\hyperlink{def-reduceslicestocall}{\texttt{reduce\_slices\_to\_call}}}
\newcommand\typeofarraylength[0]{\hyperlink{def-typeofarraylength}{\texttt{type\_of\_array\_length}}}
\newcommand\addlocal[0]{\hyperlink{def-addlocal}{\texttt{add\_local}}}

% Symbolic domain subsumption
\newcommand\symdom[0]{\hyperlink{def-symdom}{\textsf{sym\_dom}}}
\newcommand\intset[0]{\hyperlink{def-intset}{\textsf{int\_set}}}
\newcommand\syntax[0]{\hyperlink{def-syntax}{\textsf{syntax}}}

\newcommand\DBool[0]{\hyperlink{def-dbool}{\textsf{D\_Bool}}}
\newcommand\DString[0]{\hyperlink{def-dstring}{\textsf{D\_String}}}
\newcommand\DReal[0]{\hyperlink{def-dreal}{\textsf{D\_Real}}}
\newcommand\DSymbols[0]{\hyperlink{def-dsymbols}{\textsf{D\_Symbols}}}
\newcommand\DInt[0]{\hyperlink{def-dint}{\textsf{D\_Int}}}
\newcommand\DBits[0]{\hyperlink{def-dbits}{\textsf{D\_Bits}}}

\newcommand\Finite[0]{\hyperlink{def-finite}{\textsf{Finite}}}
\newcommand\Top[0]{\hyperlink{def-top}{\textsf{Top}}}
\newcommand\FromSyntax[0]{\hyperlink{def-fromsymtax}{\textsf{FromSyntax}}}

\newcommand\symdomoftype[0]{\hyperlink{def-symdomoftype}{\texttt{symdom\_of\_type}}}
\newcommand\symdomofexpr[0]{\hyperlink{def-symdomofexpr}{\texttt{symdom\_of\_expr}}}
\newcommand\symdomofliteral[0]{\hyperlink{def-symdomofliteral}{\texttt{symdom\_of\_literal}}}
\newcommand\intsetofintconstraints[0]{\hyperlink{def-intsetofintconstraintse}{\texttt{intset\_of\_intconstraints}}}
\newcommand\symdomissubset[0]{\hyperlink{def-symdomissubset}{\texttt{symdom\_is\_subset}}}
\newcommand\symintsetsubset[0]{\hyperlink{def-symintsetsubset}{\texttt{sym\_intset\_subset}}}

%%%%%%%%%%%%%%%%%%%%%%%%%%%%%%%%%%%%%%%%%%%%%%%%
%% Typeset variable names
\newcommand\newtenv[0]{\texttt{new\_tenv}}
\newcommand\tenvone[0]{\texttt{tenv1}}
\newcommand\tenvtwo[0]{\texttt{tenv2}}
\newcommand\va[0]{\texttt{a}}
\newcommand\vc[0]{\texttt{c}}
\newcommand\vcone[0]{\texttt{c1}}
\newcommand\vctwo[0]{\texttt{c2}}
\newcommand\vione[0]{\texttt{i1}}
\newcommand\vitwo[0]{\texttt{i2}}
\newcommand\vf[0]{\texttt{f}}
\newcommand\vfone[0]{\texttt{f1}}
\newcommand\vftwo[0]{\texttt{f2}}
\newcommand\vl[0]{\texttt{l}}
\newcommand\vm[0]{\texttt{m}}
\newcommand\vmone[0]{\texttt{m1}}
\newcommand\vmtwo[0]{\texttt{m2}}
\newcommand\vy[0]{\texttt{y}}
\newcommand\vp[0]{\texttt{p}}
\newcommand\vo[0]{\texttt{o}}
\newcommand\vvone[0]{\texttt{v1}}
\newcommand\vvtwo[0]{\texttt{v2}}
\newcommand\vt[0]{\texttt{t}}
\newcommand\vte[0]{\texttt{t\_e}}
\newcommand\veone[0]{\texttt{e1}}
\newcommand\vetwo[0]{\texttt{e2}}
\newcommand\vethree[0]{\texttt{e3}}
\newcommand\vefour[0]{\texttt{e4}}
\newcommand\vefive[0]{\texttt{e5}}
\newcommand\vleone[0]{\texttt{le1}}
\newcommand\vletwo[0]{\texttt{le2}}
\newcommand\vtleone[0]{\texttt{t\_le1}}
\newcommand\vre[0]{\texttt{re}}
\newcommand\vs[0]{\texttt{s}}
\newcommand\vtsone[0]{\texttt{ts1}}
\newcommand\vtstwo[0]{\texttt{ts2}}
\newcommand\vlt[0]{\texttt{l\_t}}
\newcommand\vls[0]{\texttt{l\_s}}
\newcommand\vtt[0]{\texttt{t\_t}}
\newcommand\vts[0]{\texttt{t\_s}}
\newcommand\vsone[0]{\texttt{s1}}
\newcommand\vstwo[0]{\texttt{s2}}
\newcommand\vz[0]{\texttt{z}}
\newcommand\vw[0]{\texttt{w}}
\newcommand\vwone[0]{\texttt{w1}}
\newcommand\vwtwo[0]{\texttt{w2}}
\newcommand\vwidth[0]{\texttt{width}}
\newcommand\size[0]{\texttt{size}}
\newcommand\vfield[0]{\texttt{field}}
\newcommand\vfieldone[0]{\texttt{field1}}
\newcommand\vfieldtwo[0]{\texttt{field2}}
\newcommand\vfieldsone[0]{\texttt{fields1}}
\newcommand\vfieldstwo[0]{\texttt{fields2}}
\newcommand\bitfields[0]{\texttt{bitfields}}
\newcommand\bfone[0]{\texttt{bf1}}
\newcommand\bftwo[0]{\texttt{bf2}}
\newcommand\bfoneone[0]{\texttt{bf1\_1}}
\newcommand\bftwoone[0]{\texttt{bf2\_1}}
\newcommand\vslices[0]{\texttt{slices}}
\newcommand\newle[0]{\texttt{new\_le}}
\newcommand\ldi[0]{\texttt{ldi}}
\newcommand\ldk[0]{\texttt{ldk}}
\newcommand\tty[0]{\texttt{ty}}
\newcommand\tsy[0]{\texttt{sy}}
\newcommand\tyopt[0]{\texttt{ty\_opt}}
\newcommand\ldis[0]{\texttt{ldis}}
\newcommand\newldi[0]{\texttt{new\_ldi}}
\newcommand\news[0]{\texttt{new\_s}}
\newcommand\newsone[0]{\texttt{new\_s1}}
\newcommand\newstwo[0]{\texttt{new\_s2}}
\newcommand\newargs[0]{\texttt{new\_args}}
\newcommand\eqs[0]{\texttt{eqs}}
\newcommand\neweqs[0]{\texttt{new\_eqs}}
\newcommand\reduced[0]{\texttt{reduced}}
\newcommand\tcond[0]{\texttt{t\_cond}}
\newcommand\econd[0]{\texttt{e\_cond}}
\newcommand\vcond[0]{\texttt{v\_cond}}
\newcommand\etrue[0]{\texttt{e\_true}}
\newcommand\efalse[0]{\texttt{e\_false}}
\newcommand\ttrue[0]{\texttt{t\_true}}
\newcommand\tfalse[0]{\texttt{t\_false}}
\newcommand\dir[0]{\texttt{dir}}
\newcommand\eindex[0]{\texttt{e\_index}}
\newcommand\wantedtindex[0]{\texttt{wanted\_t\_index}}
\newcommand\tindex[0]{\texttt{t\_index}}
\newcommand\fieldname[0]{\texttt{field\_name}}
\newcommand\fields[0]{\texttt{fields}}
\newcommand\fieldsone[0]{\texttt{fields1}}
\newcommand\fieldstwo[0]{\texttt{fields2}}
\newcommand\slices[0]{\texttt{slices}}
\newcommand\newe[0]{\texttt{new\_e}}
\newcommand\ta[0]{\texttt{ta}}
\newcommand\les[0]{\texttt{les}}
\newcommand\subtys[0]{\texttt{sub\_tys}}
\newcommand\catchers[0]{\texttt{catchers}}
\newcommand\otherwise[0]{\texttt{otherwise}}
\newcommand\csone[0]{\texttt{cs1}}
\newcommand\cstwo[0]{\texttt{cs2}}
\newcommand\irone[0]{\texttt{ir1}}
\newcommand\irtwo[0]{\texttt{ir2}}
\newcommand\vpone[0]{\texttt{p1}}
\newcommand\vptwo[0]{\texttt{p2}}
\newcommand\vps[0]{\texttt{ps}}
\newcommand\opone[0]{\texttt{op1}}
\newcommand\optwo[0]{\texttt{op2}}
\newcommand\vep[0]{\texttt{e'}}
\newcommand\veoneone[0]{\texttt{e1\_1}}
\newcommand\veonetwo[0]{\texttt{e1\_2}}
\newcommand\veonethree[0]{\texttt{e1\_3}}
\newcommand\vetwoone[0]{\texttt{e2\_1}}
\newcommand\vetwotwo[0]{\texttt{e2\_2}}
\newcommand\vetwothree[0]{\texttt{e2\_3}}
\newcommand\vbone[0]{\texttt{b1}}
\newcommand\vbtwo[0]{\texttt{b2}}
\newcommand\vbthree[0]{\texttt{b3}}
\newcommand\nameone[0]{\texttt{name1}}
\newcommand\nametwo[0]{\texttt{name2}}
\newcommand\vargsone[0]{\texttt{args1}}
\newcommand\vargstwo[0]{\texttt{args2}}
\newcommand\vargone[0]{\texttt{arg1}}
\newcommand\vargtwo[0]{\texttt{arg2}}
\newcommand\vlone[0]{\texttt{l1}}
\newcommand\vltwo[0]{\texttt{l2}}
\newcommand\sliceone[0]{\texttt{slice1}}
\newcommand\slicetwo[0]{\texttt{slice2}}
\newcommand\slicesone[0]{\texttt{slices1}}
\newcommand\slicestwo[0]{\texttt{slices2}}
\newcommand\positions[0]{\texttt{positions}}
\newcommand\posmax[0]{\texttt{pos\_max}}
\newcommand\bv[0]{\texttt{bv}}
\newcommand\bvone[0]{\texttt{bv1}}
\newcommand\vtp[0]{\texttt{t'}}
\newcommand\vli[0]{\texttt{li}}
\newcommand\tys[0]{\texttt{tys}}
\newcommand\name[0]{\texttt{name}}
\newcommand\newname[0]{\texttt{new\_name}}

% Increase indentation of sections in the table of contents
% to allow a space between the section numbers and their titles.
\makeatletter
\renewcommand{\l@section}{\@dottedtocline{1}{1.5em}{2.6em}}
\makeatother
\setcounter{tocdepth}{1}

\author{Arm Architecture Technology Group}
\title{ASL Typing Reference \\
       DDI 0622}
\begin{document}
\maketitle

\tableofcontents{}

\chapter{Non-Confidential Proprietary Notice}

This document is protected by copyright and other related rights and the
practice or implementation of the information contained in this document may be
protected by one or more patents or pending patent applications. No part of
this document may be reproduced in any form by any means without the express
prior written permission of Arm. No license, express or implied, by estoppel or
otherwise to any intellectual property rights is granted by this document
unless specifically stated.

Your access to the information in this document is conditional upon your
acceptance that you will not use or permit others to use the information for
the purposes of determining whether implementations infringe any third party
patents.

THIS DOCUMENT IS PROVIDED “AS IS”. ARM PROVIDES NO REPRESENTATIONS AND NO
WARRANTIES, EXPRESS, IMPLIED OR STATUTORY, INCLUDING, WITHOUT LIMITATION, THE
IMPLIED WARRAN-TIES OF MERCHANTABILITY, SATISFACTORY QUALITY, NON-INFRIN-GEMENT
OR FITNESS FOR A PARTICULAR PURPOSE WITH RESPECT TO THE DOCUMENT. For the
avoidance of doubt, Arm makes no representation with respect to, and has
undertaken no analysis to identify or understand the scope and content of, any
patents, copyrights, trade secrets, trademarks, or other rights.

This document may include technical inaccuracies or typographical errors.

TO THE EXTENT NOT PROHIBITED BY LAW, IN NO EVENT WILL ARM BE LIABLE FOR ANY
DAMAGES, INCLUDING WITHOUT LIMITATION ANY DIRECT, INDIRECT, SPECIAL,
INCIDENTAL, PUNITIVE, OR CONSEQUENTIAL DAMAGES, HOWEVER CAUSED AND REGARDLESS
OF THE THEORY OF LIABILITY, ARISING OUT OF ANY USE OF THIS DOCUMENT, EVEN IF
ARM HAS BEEN ADVISED OF THE POSSIBILITY OF SUCH DAMAGES.

This document consists solely of commercial items. You shall be responsible for
ensuring that any use, duplication or disclosure of this document complies
fully with any relevant export laws and regulations to assure that this
document or any portion thereof is not exported, directly or indirectly, in
violation of such export laws. Use of the word “partner” in reference to Arm’s
customers is not intended to create or refer to any partnership relationship
with any other company. Arm may make changes to this document at any time and
without notice.

This document may be translated into other languages for convenience, and you
agree that if there is any conflict between the English version of this
document and any translation, the terms of the English version of this document
shall prevail.

The Arm corporate logo and words marked with ® or ™ are registered trademarks
or trademarks of Arm Limited (or its affiliates) in the US and/or elsewhere.
All rights reserved.  Other brands and names mentioned in this document may be
the trademarks of their respective owners. Please follow Arm’s trademark usage
guidelines at

\url{https://www.arm.com/company/policies/trademarks.}

Copyright © [2023,2024] Arm Limited (or its affiliates). All rights reserved.

Arm Limited. Company 02557590 registered in England.  110 Fulbourn Road,
Cambridge, England CB1 9NJ.  (LES-PRE-20349)


\chapter{Disclaimer}

This document is part of the ASLRef material.

This material covers ASLv1, a new, experimental, and as yet unreleased version of ASL.

The development version of ASLRef can be found here: \\
\url{https://github.com/herd/herdtools7}.

A list of open items being worked on can be found here: \\
\url{https://github.com/herd/herdtools7/blob/master/asllib/doc/ASLRefProgress.tex}.

This material is work in progress, more precisely at Alpha quality as
per Arm’s quality standards. In particular, this means that it would be
premature to base any production tool development on this material.

However, any feedback, question, query and feature request would be most
welcome; those can be sent to Arm’s Architecture Formal Team Lead Jade Alglave
\texttt{(jade.alglave@arm.com)} or by raising issues or PRs to the herdtools7
github repository.


%%%%%%%%%%%%%%%%%%%%%%%%%%%%%%%%%%%%%%%%%%%%%%%%%%%%%%%%%%%%%%%%%%%%%%%%%%%%%%%%%%%%
\chapter{Introduction}
%%%%%%%%%%%%%%%%%%%%%%%%%%%%%%%%%%%%%%%%%%%%%%%%%%%%%%%%%%%%%%%%%%%%%%%%%%%%%%%%%%%%

The purpose of this document is to describe, in a formal and authoritative way,
which ASL specifications are considered \emph{well-typed}.
Whether a specification is well-typed is defined in terms of a \emph{type system}~\cite{TypeSystemsLucaCardelli}.
That is, a set of \emph{typing rules}.

An ASL parser accepts an ASL specification and checks whether it is valid with respect to the syntax of ASL,
which is defined in~\cite{ASLAbstractSyntaxReference}.
If the specification is syntactically valid, the parser returns an \emph{abstract syntax tree} (AST, for short),
which represents the specification as a labelled structured tree. Otherwise, it returns a syntax error.
When an ASL specification is successfully parsed, we refer to the resulting AST as the \emph{parsed AST}.

A \emph{type checker} is an implementation of the ASL type system, which accepts a parsed AST and applies the
rules of the type system to the parsed AST. If it is successful, the specification
is considered \emph{well-typed} and the result is a pair consisting of
a \emph{static environment} and a \emph{typed AST}, which are used in defining the ASL semantics~\cite{ASLSemanticsReference}.
Otherwise, the type checker returns a type error.

\paragraph{Related documents:}
\begin{itemize}
  \item The ASL Language Reference Manual~\cite{LRM} (LRM, for short) introduces the concrete syntax and intent
  of all ASL language constructs.
  Please note that the LRM will be retired in due course. For ease of reviewing, we currently indicate which statement
  of the LRM the present rules correspond to.
  \item The Abstract Syntax Reference~\cite{ASLAbstractSyntaxReference} defines the abstract syntax, parsed AST, and typed AST.
  \item The ASL Semantics Reference~\cite{ASLSemanticsReference} defines all valid behaviors of a well-typed ASL specification.
\end{itemize}

\paragraph{Understanding the Typing Formalization:}
We assume basic familiarity with the ASL language.
The ASL type system is defined in terms of its AST,
and familiarity with the AST is \underline{required} to understand it.
The mathematical background needed to understand the mathematical formalization
of the ASL semantics appears in \chapref{formal} and \chapref{typesystembuildingblocks}.

\chapter{Formal System \label{chap:FormalSystem}}

In this part, we define the mathematical concepts and notations used throughout.
We start by defining general mathematical concepts
and then describe how sets of rules formally define functions and relations.

\section{Mathematical Definitions and Notations}

\hypertarget{def-triangleq}{}
We use $\triangleq$ to define mathematical concepts.

We define the following sets:
\begin{itemize}
\item \hypertarget{def-N}{
    $\N$ is the set of natural numbers, including $0$.
}

\item \hypertarget{def-Npos}{
    $\Npos$ is the set of natural numbers, excluding $0$.
}

\item
\hypertarget{def-Z}{
    $\Z$ is the set of integers.
}

\item
\hypertarget{def-Q}{
    $\Q$ is the set of rationals.
}

\hypertarget{def-bool}{}
\hypertarget{def-false}{}
\hypertarget{def-true}{}
\item $\Bool$ is the set of ASL Boolean literals, which consists of $\True$ and $\False$.
We employ these literals to represent the corresponding mathematical truth values,
which are used to denote whether logical assertions hold or not.
\hypertarget{def-land}{}
\hypertarget{def-lor}{}
We also employ the mathematical meaning of logical conjunction $\land$, logical disjunction $\lor$,
and logical negation $\neg$, given next.
For a set of Boolean values $A$:
\[
  \begin{array}{rcl}
  \land A &\triangleq&
  \begin{cases}
    \True & \text{if all values in A are }\True\\
    \False & \text{otherwise}
  \end{cases}\\
  \lor A &\triangleq&
  \begin{cases}
    \False & \text{if all values in A are }\False\\
    \True & \text{otherwise}
  \end{cases}\\
\end{array}
\]
\hypertarget{def-neg}{}
For a pair of Boolean values $a,b\in\Bool$, we define $a \land b \triangleq \land\{a, b\}$
and $a \lor b \triangleq \lor\{a, b\}$.
Finally, $\neg\True\triangleq\False$ and $\neg\False\triangleq\True$.

\item
\hypertarget{def-identifier}{}
    $\Identifiers$ is the set of all ASL identifiers.

\item
\hypertarget{def-astlabels}{}
    $\astlabels$ is the set of all labels of Abstract Syntax Tree (AST) nodes.

\item
\hypertarget{def-strings}{}
    $\Strings$ is the set of all ASCII strings.
\end{itemize}

We utilize the notation $\overname{a}{b}$ to enable us to name the mathematical term $a$ as $b$ so that
we can refer to it in text. We especially use this to name the input arguments and
output results of functions and relations. For example, the input argument of $\sign$,
which is defined next is named $q$.

\hypertarget{def-sign}{}
\begin{definition}[Sign of a Rational Number]
  The function $\sign : \overname{\Q}{q} \rightarrow \{-1,0,1\}$ returns the sign of $\vq$:
\[
\sign(q) \triangleq \begin{cases}
  1 & \text{if }q > 0\\
  0 & \text{if }q = 0\\
  -1 & \text{if }q < 0
\end{cases}
\]
\end{definition}

\begin{definition}[Empty Set]
  The \emph{empty set} --- the set that does not contain any element --- is denoted as $\emptyset$.
\end{definition}

\hypertarget{def-cardinality}{}
\begin{definition}[Set Cardinality]
  For a set $S$, the notation $\cardinality{S}$ stands for the number of elements in $S$.
\end{definition}

\hypertarget{def-pow}{}
\begin{definition}[Powerset]
  The \emph{powerset} of a set $A$, denoted as $\pow{A}$, is the set of all subsets of $A$, including the empty set and $A$ itself:
  \[
     \pow{A} \triangleq \{ B \;|\; B \subseteq A\} \enspace.
  \]
\end{definition}

\hypertarget{def-powfin}{}
\begin{definition}[Powerset of Finite Subsets]
  The \emph{powerset of finite subsets} of a set $A$, denoted as $\powfin{A}$, is the set of all finite subsets (including the empty set) of $A$:
  \[
     \powfin{A} \triangleq \{ B \;|\; B \subseteq A, |B| \in \N\} \enspace.
  \]
\end{definition}

\hypertarget{def-cartimes}{}
\begin{definition}[Cartesian Product]
    The \emph{Cartesian product} of sets $A$ and $B$, denoted $A \cartimes B$
    is $A \cartimes B \triangleq \{(a,b) \;|\; a \in A, b \in B\}$.
\end{definition}

\hypertarget{def-partialfunc}{}
\hypertarget{def-dom}{}
\begin{definition}[Partial Function\label{def:PartialFunction}]
  A \emph{partial function}, denoted $f : A \partialto B$, is a function from a \underline{subset} of $A$ to $B$.
  The \emph{domain} of a partial function $f$, denoted $\dom(f)$, is the subset of $A$ for which it is defined.
  We write $f(x) = \bot$ to denote that $x$ is not in the domain of $f$, that is, $x \not\in \dom(f)$.
\end{definition}

Notice that the domain of a partial function need not be finite, which is what the following definition covers.

\hypertarget{def-finfunction}{}
\begin{definition}[Finite-domain Function]
The notation $\rightarrowfin$ stands for a function \\ whose domain is finite.
\end{definition}

\hypertarget{def-emptyfunc}{}
\begin{definition}[Empty Function\label{def:EmptyFunction}]
The function with an empty domain is denoted as $\emptyfunc$.
\end{definition}

\begin{definition}[Function Update\label{def:FunctionUpdate}]
  The function denoted as $f[x \mapsto v]$ is a function identical to $f$, except that $x$ is bound
  to $v$. That is, if  $g = f[x \mapsto v]$ then
  \[
    g(z) =
  \begin{cases}
    v     & \text{if } z = x\\
    f(z)  & \text{otherwise } \enspace.
  \end{cases}
  \]

  The notation $\{i=1..k: a_i\mapsto b_i\}$ stands for the function formed from the corresponding input-output pairs:
  $\emptyfunc[a_1\mapsto b_1]\ldots[a_k\mapsto b_k]$.
\end{definition}

\begin{definition}[Function Restriction]
\hypertarget{def-restrictfunc}{}
The \emph{restriction} of a function $f : X \rightarrow Y$ to a subset of its domain
$A \subseteq \dom(f)$, denoted as $\restrictfunc{f}{A}$, is defined
in terms of the set of input-output pairs:
\[
  \restrictfunc{f}{A} \triangleq \{ (x, f(x)) \;|\; x \in A \} \enspace.
\]
\end{definition}

\begin{definition}[Function Graph]
\hypertarget{def-funcgraph}{}
The \emph{graph} of a finite-domain function $f : X \rightarrowfin Y$
is the list of input-output pairs for $f$, given in any order:
\[
\funcgraph(f) \triangleq \{ (x, f(x)) \;|\; x \in \dom(f) \} \enspace.
\]
\end{definition}

Throughout this document, we will annotate arguments of relations and functions, wherever it is useful,
by writing a name or an expression above the corresponding argument type.
This makes convenient to refer to arguments by referring to the corresponding names and helps identify
the expressions corresponding to the arguments.
For example,
\[
    \textsf{choice} : \overname{\Bool}{b} \cartimes \overname{T}{x} \cartimes \overname{T}{y} \rightarrow \overname{T}{z}
\]
defines a function type and lets us refer to the first argument as $b$, the second argument as $x$,
the third argument as $y$, and to the result as $z$.

A \emph{parametric function} is a function whose domain is not a priori fixed but rather
parameterized by the type of its arguments. An example is the $\textsf{choice}$ function where the type $T$ of
$x$, $y$, and $z$ is unspecified and inferred from the context where the function is used.

\hypertarget{def-choice}{}
\begin{definition}[Choice]
The parametric function $\textsf{choice} : \overname{\Bool}{b} \cartimes \overname{T}{x} \cartimes \overname{T}{y} \rightarrow \overname{T}{z}$,
is defined as follows:
\[
  \choice{\vb}{x}{y} \triangleq
  \begin{cases}
    x & \text{ if }\vb \text{ is }\True\\
    y & \text{ otherwise}\\
  \end{cases}
\]
\end{definition}

\subsection{Lists}
In the remainder of this document, we use the term \emph{list} and \emph{sequence} interchangeably.

A list of elements \hypertarget{def-emptylist}{is either empty, denoted by $\emptylist$}, or non-empty.
A non-empty list is either denoted by listing the elements in sequence, $v_1 \ldots\ v_k$,
or in bracketed form, $[v_1,\ldots,v_k]$, which is used to aesthetically separate it from surrounding mathematical expressions.
The commas carry no special meaning.

\hypertarget{def-head}{}
\hypertarget{def-tail}{}
For a non-empty list $v_1 \ldots\ v_k$, the \emph{\head} of the list is the first element --- $v_1$ ---
and the \emph{\tail} of the list is the suffix obtained by removing $v_1$ from the list.

We refer to individual elements of a non-empty list $V$ by the index notation $V[i]$ where $i\in\Npos$.

\hypertarget{def-listlen}{}
\begin{definition}[List Length]
The \emph{length} of a list is the number of elements in that list:
$\listlen{\emptylist} \triangleq 0$ and $\listlen{v_1,\ldots,v_k}=k$.
\end{definition}

We use the notation $a..b$, where $a,b\in\Z$ and as a shorthand for the interval $[a\ldots b]$
(counting up when $a \leq b$ and counting down when $a \geq b$).
We write $x_{a..b}$ as a shorthand for the sequence $x_a \ldots x_b$.
%
We write $i=1..k: V(i)$, where $V(i)$ is a mathematical expression parameterized by $i$,
to denote the sequence of expressions $V(1) \ldots V(k)$.
The notation $a \in A: V(a)$, where $A$ is a set and $V$ is an expression parameterized by the free variable $a$,
stands for $V(a_1) \ldots V(a_k)$ where $a_{1..k}$ is an arbitrary ordering of the elements of $A$.

We write $T^*$ to denote a the type of a possibly-empty list of elements of type $T$,
and $T^+$ for a non-empty list of elements of type $T$.

\hypertarget{def-listset}{}
\begin{definition}[Listing a Set]
The parametric relation $\listset : \pow{T} \times T^*$
lists the elements of a set in an arbitrary order:
\[
\begin{array}{c}
  \listset(X) = x_{1..k}\\
  |X| = k\\
  \forall x\in X.\ \exists 1 \leq i \leq k.\ x = x_i
\end{array}
\]
\end{definition}

\hypertarget{def-concat}{}
\begin{definition}[List Concatenation]
The parametric function $\concat : T^* \cartimes T^* \rightarrow T^*$ concatenates two lists:
\[
    \begin{array}{rcl}
    \emptylist \concat L &\triangleq& L\\
    L \concat \emptylist &\triangleq& L\\
    l_{1..k} \concat m_{1..n} &\triangleq& [l_{1..k}, m_{1..n}]
    \end{array}
\]
\end{definition}

\hypertarget{def-equallength}{}
\begin{definition}[Equating List Lengths]
The parametric function
\[
  \equallength : \overname{L}{a} \cartimes \overname{L}{b} \rightarrow \Bool
\]
compares the length of two lists:
\[
\equallength(a, b) \triangleq \listlen{a}=\listlen{b} \enspace.
\]
\end{definition}

\hypertarget{def-listprefix}{}
\begin{definition}[List Prefix]
The parametric function $\listprefix : \overname{T^*}{\vlone} \cartimes \overname{T^*}{\vltwo} \rightarrow \Bool$ checks whether
the list $\vlone$ is a \emph{prefix} of the list $\vltwo$:
\[
\listprefix(\vlone, \vltwo) \triangleq \exists \vlthree.\ \vltwo = \vlone \concat \vlthree \enspace.
\]
\end{definition}

\hypertarget{def-listrange}{}
\begin{definition}[Indices of a List]
The parametric function $\listrange : T^* \rightarrow \N^*$ returns the ($1$-based) list of indices for a given list:
\[
    \begin{array}{rcl}
        \listrange(\emptylist) &\triangleq& \emptylist\\
        \listrange(v_{1..k}) &\triangleq& [1..k] \enspace.
    \end{array}
\]
\end{definition}

% \hypertarget{def-filterlist}{}
% \begin{definition}[Filtering a List]
% The parametric function
% \[
% \filterlist : \overname{T^*}{l} \times \overname{(T \rightarrow \Bool)}{p} \rightarrow T^*
% \]
% retains the elements of the list $l$ for which the predicate $p$ returns $\True$:
% \[
% \begin{array}{rcl}
% \filterlist(\emptylist, p) &\triangleq& \emptylist\\
% \filterlist([h] \concat t, p) &\triangleq& \begin{cases}
% [h]\concat \filterlist(t) & \text{if }p(h) = \True\\
%  \filterlist(t) & \text{else}\\
% \end{cases}\\
% \end{array}
% \]
% \end{definition}

\hypertarget{def-unziplist}{}
\begin{definition}[Unzipping a List of Pairs]
The parametric function
\[
\unziplist : (T_1 \cartimes T_2)^* \rightarrow (T_1^* \cartimes T_2^*)
\]
transforms a list of pairs into the corresponding pair of lists:
\[
  \unziplist(\pairs) \triangleq \begin{cases}
    (\emptylist, \emptylist)  & \text{if }\pairs = \emptylist\\
    (a_{1..k}, b_{1..k})      & \text{else }\pairs = (a_1,b_1) \ldots (a_k,b_k)  \enspace.
  \end{cases}
\]
\end{definition}

\hypertarget{def-unziplistthree}{}
\begin{definition}[Unzipping a List of Triples]
The parametric function
\[
\unziplistthree : (T_1 \cartimes T_2 \cartimes T_3)^* \rightarrow (T_1^* \cartimes T_2^* \cartimes T_3^*)
\]
transforms a list of triples into the corresponding triple of lists:
\[
  \unziplistthree(\triples) \triangleq \begin{cases}
    (\emptylist, \emptylist, \emptylist)  & \text{if }\triples = \emptylist\\
    (a_{1..k}, b_{1..k}, c_{1..k})      & \text{else }\triples = (a_1,b_1,c_1) \ldots (a_k,b_k,c_k)  \enspace.
  \end{cases}
\]
\end{definition}

\hypertarget{def-uniquelist}{}
\hypertarget{def-uniquep}{}
\begin{definition}[Finding unique elements of a list]
The parametric function
\[
\uniquelist : \overname{T^*}{l} \rightarrow T^*
\]
retains only the first occurrence of each element of the list $l$.
It relies on the helper function $\uniquep$:
\[
\begin{array}{rcl}
\uniquelist(l) &\triangleq& \uniquep(l, \emptylist)\\\\
\uniquep(\emptylist, \acc) &\triangleq& \acc\\
\uniquep([h] \concat t, \acc) &\triangleq&
  \begin{cases}
    \uniquep(t, \acc) & \text{if }h \in t\\
    \uniquep(t, \acc \concat [h]) & \text{otherwise}\\
  \end{cases}
\end{array}
\]
\end{definition}

\subsection{Strings}
\hypertarget{def-stringconcat}{}
The function $\stringconcat : \Strings \times \Strings \rightarrow \Strings$
concatenates two strings.

\hypertarget{def-stringofnat}{}
The function $\stringofnat : \N \rightarrow \Strings$ converts a natural number
to the corresponding string.

\subsection{OCaml-style Notations}
We use the following notations, which are in the style of the OCaml programming language,
to facilitate correspondence with our
\href{https://github.com/herd/herdtools7/tree/master/asllib}{reference implementation}.

The notation $L(v_{1..k})$ is a compound term where $L$ is a label and $v_{1..k}$ is a (possibly singleton) list of mathematical values.
We also write $L(T_{1..k})$, where $T_{1..k}$ denotes mathematical types of values, to stand for the type
$\{ L(v_{1..k}) \;|\; v_1\in T_1,\ldots,v_k\in T_k \}$.

\hypertarget{def-optional}{}
\begin{definition}[Optional]
\hypertarget{def-none}{}
\hypertarget{def-some}
The notation $\Some{\cdot}$ stands for either an empty set or a singleton set,
where $\None\triangleq\langle\rangle$ denotes an empty set
and $\Some{v}$ denotes a set containing the single element $v$.
%
The notation $\Some{T}$, where $T$ denotes a mathematical type, stands for
$\{ \None \} \cup \{\Some{v} \;|\; v \in T\}$.
%
We refer to $\Some{T}$ as an \emph{\optional}.
\end{definition}

%%%%%%%%%%%%%%%%%%%%%%%%%%%%%%%%%%%%%%%%%%%%%%%%%%%%%%%%%%%%%%%%%%%%%%%%%%%%%%%%
\section{Inference Rules}
%%%%%%%%%%%%%%%%%%%%%%%%%%%%%%%%%%%%%%%%%%%%%%%%%%%%%%%%%%%%%%%%%%%%%%%%%%%%%%%%
\hypertarget{def-inferencerule}{}
An \emph{\inferencerule} (rule, for short) is an implication between a set of logical assertions,
called the \emph{premises} of the rule,
and a \emph{conclusion} assertion.
The conclusion holds when the \underline{conjunction} of its premises holds.

We use the following rule notation, where $P_{1..k}$ are the rule premises and $C$ is the conclusion:
\begin{mathpar}
  \inferrule{P_1 \and \ldots \and P_k}{C}
\end{mathpar}

For example, the rule \TypingRuleRef{ELit} has one premise:
\begin{mathpar}
\inferrule{
  \annotateliteral{\vv} \typearrow \vt
}{
  \annotateexpr{\tenv, \ELiteral(\vv)} \typearrow (\vt, \ELiteral(\vv))
}
\end{mathpar}

and the rule \TypingRuleRef{EBinop} (somewhat simplified here) has three premises:
\begin{mathpar}
\inferrule{
  \annotateexpr{\tenv, \veone} \typearrow (\vtone, \veone') \\\\
  \annotateexpr{\tenv, \vetwo} \typearrow (\vttwo, \vetwo') \\\\
  \applybinoptypes(\tenv, \op, \vtone, \vttwo) \typearrow \vt
}{
  \annotateexpr{\tenv, \EBinop(\op, \veone, \vetwo)} \typearrow (\vt, \EBinop(\op, \veone', \vetwo'))
}
\end{mathpar}

The free variables appearing in the premises and conclusion are interpreted \underline{universally}.
That is, the rules apply to any values (of the appropriate types) assigned to their free variables.
%
For example, the rule \TypingRuleRef{EBinop} applies to any choice of values for the free variables
$\tenv$ (a static environment),
$\veone$, $\vetwo$, $\veone'$, $\vetwo'$ (expressions),
$\vt$, $\vtone$, and $\vttwo$ (types).

\begin{definition}[Grounding]
Assertions can be \emph{grounded} by substituting their free variables with values.
A \emph{ground rule} is a rule with all its assertions (premises and conclusion) grounded.
\end{definition}
For example,
the following is a grounding of \TypingRuleRef{EBinop}
\begin{mathpar}
\inferrule{
  \annotateexpr{\emptytenv, \ELiteral(\lint(2))} \typearrow (\TInt, \ELiteral(\lint(2))) \\\\
  \annotateexpr{\emptytenv, \ELiteral(\lint(3))} \typearrow (\TInt, \ELiteral(\lint(3))) \\\\
  \applybinoptypes(\emptytenv, \MUL, \TInt, \TInt) \typearrow \TInt
}{
  \annotateexpr{\emptytenv, \EBinop(\MUL, \ELiteral(\lint(2)), \ELiteral(\lint(3)))} \typearrow \\ (\TInt, \EBinop(\MUL, \ELiteral(\lint(2)), \ELiteral(\lint(3))))
}
\end{mathpar}
obtained by the following substitutions:
\begin{tabular}{ll}
  \textbf{free variable} & \textbf{value}\\
  \hline
  $\tenv$   & $\emptytenv$\\
  $\veone$  & $\ELiteral(\lint(2))$\\
  $\veone'$  & $\ELiteral(\lint(2))$\\
  $\vetwo$  & $\ELiteral(\lint(3))$\\
  $\vetwo'$  & $\ELiteral(\lint(3))$\\
  $\vt$    & $\TInt$\\
  $\vtone$    & $\TInt$\\
  $\vttwo$    & $\TInt$\\
  $\op$       & $\MUL$
\end{tabular}

A set of rules is interpreted \underline{disjunctively}. That is, each rule is used to determine whether its conclusion
holds independently of other rules.

\begin{definition}[Axiom]
An \emph{axiom} is a rule with an empty set of premises.
An axiom is denoted by simply stating its conclusion.
\end{definition}

An example of an axiom in the ASL type system is \TypingRuleRef{SPass}:
\begin{mathpar}
\inferrule{}{\annotatestmt(\tenv, \SPass) \typearrow (\SPass,\tenv)}
\end{mathpar}
\hypertarget{SemanticsRule.PAll-example}{}
An example of an axiom in the ASL semantics is \SemanticsRuleRef{PAll}:
\begin{mathpar}
\inferrule{}{
  \evalpattern{\env, \Ignore, \PatternAll} \evalarrow \Normal(\nvbool(\True), \emptygraph)
}
\end{mathpar}

To show that a specification is correct, with respect to the set of type rules,
or to show that a specification evaluates to a certain value, with respect to
the set of semantic rules, we must apply rules to form a \emph{derivation tree}.

\hypertarget{def-derivationtree}{}
\begin{definition}[Derivation Tree]
  A \emph{derivation tree} is a tree whose vertices correspond to ground assertions.
  More specifically, the leaves of a derivation tree correspond to ground axioms,
  and an internal vertex corresponds to a ground conclusion of a rule with its children
  corresponding to the ground premises of the same rule.
\end{definition}

\subsection{Transitions\label{sec:transitions}}

We use rules as a structured way for defining relations (and therefore functions, as a special case).

To define a relation $R \subseteq X \cartimes Y$, we use assertions of the form $\termx \rulearrow \termy$
where $\termx$ and $\termy$ are logical terms denoting sets of elements from $X$ and $Y$, respectively.
%
We call such assertions \emph{transitions}.
A set of rules $M$ with transition assertions defines the relation
\[
    R = \{ (x,y) \;|\; x \rulearrow y \text{ can be derived from rules in } M\} \enspace.
\]

For example, the rule \TypingRuleRef{ELit} defines a relation
between the infinite set of elements of the form
$\annotateexpr{\tenv, \ELiteral(\vv)}$ (for the
infinite choice of values for the free variables $\tenv$ and
$\vv$) to the infinite set of pairs of the form $(\vt,
\ELiteral(\vv))$, such that the premise holds.

\paragraph{Mutual Exclusion Principle:}
Our rules follow (with very few deviations, which we point out
in context) a mutual exclusion principle, where each rule
defines a relation disjoint from the ones defined by the other
rules.  This makes it easy to determine the rule responsible
for a given transition.

\hypertarget{def-configuration}{}
\subsection{Configurations}

\hypertarget{def-configdomain}{}
Our relations range over compound values. That is, values that often nest tuples and lists inside other tuples and lists.
We refer to such values as \emph{configurations}. To make it easier to distinguish between different configurations,
we will sometimes attach labels to tuples using the OCaml-style notation discussed earlier.
We refer to those labels as \emph{configuration domains}.
The domain of a configuration $C=L(\ldots)$, denoted $\configdomain{C}$, is the label $L$.

We refer to configurations at the origin of a transition as \emph{input configurations} and to the
configurations at the destination of a transition as \emph{output transitions}.

For example, the conclusion of the rule \TypingRuleRef{ELit} has \\
$\annotateexpr{\tenv, \ELiteral(\vv)}$ as its input configuration
and $(\vt, \ELiteral(\vv))$ as its output configuration.
Further, \\
$\configdomain{\annotateexpr{\tenv, \ELiteral(\vv)}} = \textit{annotate\_expr}$,
while the output configuration does not have a configuration domain, since it is an unlabelled pair.

Our rules always make use of labelled input configurations. This makes it easier to ensure
the mutual exclusion rule principle.

Our rules always define relations whose sets of input configurations and output configurations are disjoint.

\hypertarget{def-freshvariables}{}
\begin{definition}[Fresh Element]
  Premises of the form \texttt{$x\in T$ is fresh} mean that in any
  instantiation in a derivation tree, the value of $x$ is unique.
  That is, different from all other values instantiated for any other variable.
\end{definition}

\hypertarget{def-ignore}{}
\begin{definition}[Ignore Variable]
To keep rules succinct, we write $\Ignore$ for a mathematical variable whose name is
irrelevant for understanding the rule, and can thus be omitted.
Each \underline{occurrence} of $\Ignore$ represents a variable whose name is
different from any other free variable in the rule.
\end{definition}

For example, the rule \SemanticsRuleRef{PAll}, shown \hyperlink{SemanticsRule.PAll-example}{above},
uses an ignore variable to stand for the value being matched by a \texttt{-} pattern.
Since the rule does not need to refer to the value, we do not name it and use an ignore variable
instead.

\subsection{Flavors of Equality In Rules \label{sec:FlavoursOfEqualityInRules}}
We now explain equality notations in rules, two of which are used in \SemanticsRuleRef{ECond},
shown here:
\begin{mathpar}
\inferrule{
  \evalexpr{\env, \econd} \evalarrow \Normal(\mcond, \envone) \OrAbnormal\\\\
  \mcond \eqname (\nvbool(\vb), \vgone)\\
  \vep \eqdef \choice{\vb}{\veone}{\vetwo}\\\\
  \evalexpr{\envone, \vep} \evalarrow \Normal((\vv, \vgtwo), \newenv)  \OrAbnormal\\\\
  \vg \eqdef \ordered{\vgone}{\aslctrl}{\vgtwo}
}{
  \evalexpr{\env, \overname{\ECond(\econd, \veone, \vetwo)}{\ve}} \evalarrow
  \Normal((\vv, \vg), \newenv)
}
\end{mathpar}

\begin{description}
  \item[Range:] we write $i=1..k$ to allow listing premises parameterized by $i$ or constructing
  lists from expressions parameterized by $i$.
  For example, given two lists $a$ and $b$,
  \[
    i=1..k: a[i] > b[i]
  \]
  is the list of premises
  \[
    \begin{array}{l}
    a[1] > b[1]\\
    \ldots\\
    a[k] > b[k] \enspace.
    \end{array}
  \]

  \item[Predicate:] we write $a = b$ as an assertion of the equality of $a$ and $b$.
  For example, the mathematical identity $x \times (y + z) = x \times y + x \times z$.

  \hypertarget{def-deconstruction}{}
  \item[Deconstruction / ``View as'':] some values, such as tuples, are compound. In order to refer to the structure
  of compound values, we write $v \eqname \textit{f}(u_{1..k})$ where the expression on the right
  hand side exposes the internal structure of $v$ by introducing the variables
  $u_{1..k}$, allowing us to alias internal components of $v$.
  Intuitively, $v$ is re-interpreted as $\textit{f}(u_{1..k})$.
  For example, suppose we know that $v$ is a pair of values.
  Then, $v \eqname (a, b)$ allows us to alias $a$ and $b$.
  In \SemanticsRuleRef{ECond}, we know from the definition of $\evalexpr$ that
  $\mcond$ is a pair datatype.
  Therefore, writing $\mcond \eqname (\nvbool(\vb), \vgone)$ allows us to name each component of this pair
  and then refer to it, while \hyperlink{def-ignore}{ignoring} the static environment component.
  Similarly, if $v$ is a non-empty list, then $v \eqname [h] + t$ deconstructs the list into the
  head of the list $h$ and its tail $t$.
  Given that a variable $v$ represents a list, we write $v \eqname v_{1..k}$ to list its elements and allow
  referring to them by index.

  \hypertarget{def-eqdef}{}
  \item[Definition / ``Define as'':] the notation $\vx \eqdef \ve$ denotes that $\vx$ is a new name serving as an alias for the expression $\ve$.
  For example, in the rule \SemanticsRuleRef{ECond}, we use $\vg$ to name the mathematical expression
  $\ordered{\vgone}{\aslctrl}{\vgtwo}$.
  Aliases allow us to break down complex expressions, but rules can always be rewritten without them,
  by inlining their right-hand sides:
\begin{mathpar}
\inferrule{
  \evalexpr{\env, \econd} \evalarrow \Normal(\mcond, \envone) \OrAbnormal\\\\
  \mcond \eqname (\nvbool(\vb), \vgone)\\
  \evalexpr{\envone, \choice{\vb}{\veone}{\vetwo}} \evalarrow \Normal((\vv, \vgtwo), \newenv) \OrAbnormal
}{
  \evalexpr{\env, \overname{\ECond(\econd, \veone, \vetwo)}{\ve}} \evalarrow
  \Normal((\vv, \ordered{\vgone}{\aslctrl}{\vgtwo}), \newenv)
}
\end{mathpar}
\end{description}

\subsection{AST-related Notations}

When deconstructing AST record nodes such as $\{f_1:t_1,\ldots,f_k:t_k\}$,
we sometimes only care about a subset of the fields $\{f_{i_1},\ldots,f_{i_m}\} \subset \{f_{1..k}\}$.
In such cases, we write $\{f_{i_1}:t_{i_1},\ldots,f_{i_m}:t_{i_m},\ldots\}$,
where $\ldots$ stands for fields that are irrelevant for the rule.

For example\footnote{This example is from \SemanticsRuleRef{FCall}.}, the \func\ non-terminal is of a record type and has the following fields:
$\textsf{name}$, $\textsf{parameters}$, $\textsf{args}$, $\textsf{body}$, $\textsf{return\_type}$, and $\textsf{subprogram\_type}$.
The notation $\{ \textsf{body}:\SBASL(\texttt{body}),\ \textsf{args}:\texttt{arg\_decls}, \ldots \}$
allows us to deconstruct a given \func\ node by matching only the \textsf{body} and \textsf{args} fields.

Recall that a subset of AST nodes are either labels or labelled tuples.
\hypertarget{def-astlabel}{}
The partial function $\astlabel$ returns the label $\vl\in\astlabels$ of an AST node, when it exists.
For example, $\astlabel(\TBool) = \TBool$ and $\astlabel(\TNamed(\texttt{x})) = \TNamed$.

\subsection{How to Parse Rules Efficiently}
Consider the following examples, which is a simplified version of \SemanticsRuleRef{Binop}
\begin{mathpar}
  \inferrule{\op \not\in \{\BAND, \BOR, \IMPL\}\\\\
    \evalexpr{ \env, \veone} \evalarrow \Normal(\vmone, \envone) \\\\
    \evalexpr{ \envone, \vetwo } \evalarrow \Normal(\vmtwo, \newenv) \\\\
    \vmone \eqname (\vvone, \vgone) \\
    \vmtwo \eqname (\vvtwo, \vgtwo) \\
    \binoprel(\op, \vvone, \vvtwo) \evalarrow \vv \\\\
    \vg \eqdef \vgone \parallelcomp \vgtwo
  }
  {
    \evalexpr{ \env, \EBinop(\op, \veone, \vetwo) } \evalarrow
    \Normal((\vv, \vg), \newenv)
  }
\end{mathpar}

To parse a rule, start by examining the conclusion and the variables appearing in the rule.
In this case, the rule describes a transition from an input configuration \\
$\evalexpr{ \env, \EBinop(\op, \veone, \vetwo) }$,
whose configuration domain is \texttt{eval\_expr}, to an output configuration $\Normal((\vv, \vg), \newenv)$
whose configuration domain is $\Normal$.
%
A rule uses the free variables appearing in the input configuration of the conclusion
($\env$, $\op$, $\veone$, and $\vetwo$ in our example),
with the goal of assigning values to the free variables in the output configuration
of the conclusion ($\vv$, $\vg$, and $\newenv$, in our example).

Now, scan the premises in order to see where $\env$, $\op$, $\veone$, and $\vetwo$ are used and how
premises assign values to $\vv$, $\vg$, and $\newenv$.
%
In this case, $\vv$ is assigned as the result of the transition assertion
$\binoprel(\op, \vvone, \vvtwo) \evalarrow \vv$,
$\vg$ is assigned the expression $\vgone \parallelcomp \vgtwo$,
and $\newenv$ is assigned as the result of the transition assertion
$\evalexpr{ \envone, \vetwo } \evalarrow \Normal(\vmtwo, \newenv)$.
%
Notice that to assign values to the variables $\vv$, $\vg$, and $\newenv$,
intermediate values have to be assigned first.
For example, $\evalexpr{ \env, \veone} \evalarrow \Normal(\vmone, \envone)$
assigned values to $\envone$, which is then used by the transition \\
$\evalexpr{ \envone, \vetwo } \evalarrow \Normal(\vmtwo, \newenv)$.
Similarly, $\vg$ requires first assigning values to $\vgone$ and $\vgtwo$,
which are components of the previously assigned variables $\vmone$ and $\vmtwo$.

\subsection{Short-Circuit Rule Macros}

\emph{Short-circuit rule macros}, or \emph{rule macros}, for short, allow us to succinctly define sets of rules.
Specifically, they allow us to capture situations where
transitions have two alternative output configurations.
If the transition results in the first of the alternative output configurations, the following premises are considered.
However, if the result is the second, short-circuit output configuration, then the following premises are ignored
and the conclusion transitions into the short-circuit output configuration.
These short-circuit output configurations are typically, but not always, due to (type or dynamic) errors.

\hypertarget{def-terminateas}{}
In the following, $\XP$ and $\XQ$ stand for, possibly empty, sequences of premises.
%
A rule macro includes the special premise form $C \rulearrow C' \terminateas E$,
which introduces alternative output configurations $C'$ and short-circuit $E$:
\begin{mathpar}
  \inferrule{
    \XP\\\\
    C \rulearrow C' \terminateas E\\\\
    \XQ\\
  }
  {
    V \rulearrow V'
  }
\end{mathpar}
Such a rule macro expands to the following pair of rules:
\begin{mathpar}
  \inferrule[(Option 1)]{
    \XP\\\\
    C \rulearrow C' \\\\
    \XQ\\
  }
  {
    V \rulearrow V'
  }
  \and
  \inferrule[(Option 2:Short-circuited)]{
    \XP\\\\
    C \rulearrow E
  }
  {
    V \rulearrow E
  }
\end{mathpar}
Intuitively, if $C$ transitions to $C'$ then $\sslash E$ can be ignored
and the rule is interpreted as usual (Option 1).
However, if $C$ transitions into $E$ (Option 2) then the premises $\XQ$ are ignored,
thereby short-circuiting the rule, and the input configuration
in the conclusion also transitions into $E$.

We allow more than one premise to include short-circuiting alternatives and also
a single premise to include several alternatives.
That is, a rule macro of the form
\begin{mathpar}
  \inferrule{
    \XP\\\\
    C \rulearrow C' \terminateas E_{1...m}\\\\
    \XQ\\
  }
  {
    V \rulearrow V'
  }
\end{mathpar}
Stands for the set of rule macros
\begin{mathpar}
  \inferrule{
    \XP\\\\
    C \rulearrow C' \terminateas E_1\\\\
    \XQ\\
  }
  {
    V \rulearrow V'
  }
\and
\inferrule{\ldots}{}
\and
\inferrule{
  \XP\\\\
  C \rulearrow C' \terminateas E_m\\\\
  \XQ\\
}
{
  V \rulearrow V'
}
\end{mathpar}

Notice that after all rule macros are expanded, in a top-to-bottom and left-to-right order, into normal rules,
they behave like normal rules where the order of premises does
not matter.

\hypertarget{def-proseterminateas}{}
\paragraph{Alternative Outcomes Expressed in English Prose:}
In English prose, we use
\ProseTerminateAs{x, y, \ldots} to mean
``if the outcome is one of $x, y, \ldots$ then the result short-circuits the rule.

As an example, consider the rule \SemanticsRuleRef{Binop}.
This time, not simplified:
\begin{mathpar}
  \inferrule{\op \not\in \{\BAND, \BOR, \IMPL\}\\\\
    \evalexpr{ \env, \veone} \evalarrow \Normal(\vmone, \envone) \OrAbnormal \\\\
    \evalexpr{ \envone, \vetwo } \evalarrow \Normal(\vmtwo, \newenv) \OrAbnormal \\\\
    \vmone \eqname (\vvone, \vgone) \\
    \vmtwo \eqname (\vvtwo, \vgtwo) \\
    \binoprel(\op, \vvone, \vvtwo) \evalarrow \vv \terminateas \DynErrorConfig\\\\
    \vg \eqdef \vgone \parallelcomp \vgtwo
  }
  {
    \evalexpr{ \env, \EBinop(\op, \veone, \vetwo) } \evalarrow
    \Normal((\vv, \vg), \newenv)
  }
\end{mathpar}

In this rule, $\ThrowingConfig$ and $\DynErrorConfig$ are just shorthand notations for
actual configurations, which are properly defined in \chapref{Semantics}.
Intuitively, the alternative configurations $\ThrowingConfig$ and $\DynErrorConfig$
represent situations where a transition may result in a raised exception and a dynamic error,
respectively.

One may first read the rule ignoring these alternative configurations, to see how the
goal of transitioning into the output configuration appearing in the conclusion ---
$\Normal((\vv, \vg), \newenv)$ --- is achieved.
Then, re-reading the rule would indicate where exceptions and dynamic errors may result
in other output configurations.
%
For example, if the first transition assertion results in a throwing configuration $\ThrowingConfig$
then the output configuration of the conclusion is also $\ThrowingConfig$.
This corresponds to the following rule in the expanded macro:

\begin{mathpar}
  \inferrule{\op \not\in \{\BAND, \BOR, \IMPL\}\\\\
    \evalexpr{ \env, \veone} \evalarrow \ThrowingConfig
  }
  {
    \evalexpr{ \env, \EBinop(\op, \veone, \vetwo) } \evalarrow
    \ThrowingConfig
  }
\end{mathpar}

Similarly, if the first transition assertion results in a dynamic error, the output configuration of
the conclusion is that dynamic error, which corresponds to the following rule in the expansion:
\begin{mathpar}
  \inferrule{\op \not\in \{\BAND, \BOR, \IMPL\}\\\\
    \evalexpr{ \env, \veone} \evalarrow \DynErrorConfig
  }
  {
    \evalexpr{ \env, \EBinop(\op, \veone, \vetwo) } \evalarrow
    \DynErrorConfig
  }
\end{mathpar}

The following rules correspond to the cases where the first transition results in \\
$\Normal(\vmone, \envone)$, but the second transition assertion results in either
$\ThrowingConfig$ or $\DynErrorConfig$, respectively:
\begin{mathpar}
  \inferrule{\op \not\in \{\BAND, \BOR, \IMPL\}\\\\
    \evalexpr{ \env, \veone} \evalarrow \Normal(\vmone, \envone) \\\\
    \evalexpr{ \envone, \vetwo } \evalarrow \ThrowingConfig
  }
  {
    \evalexpr{ \env, \EBinop(\op, \veone, \vetwo) } \evalarrow
    \ThrowingConfig
  }
\end{mathpar}

\begin{mathpar}
  \inferrule{\op \not\in \{\BAND, \BOR, \IMPL\}\\\\
    \evalexpr{ \env, \veone} \evalarrow \Normal(\vmone, \envone) \\\\
    \evalexpr{ \envone, \vetwo } \evalarrow \DynErrorConfig
  }
  {
    \evalexpr{ \env, \EBinop(\op, \veone, \vetwo) } \evalarrow
    \DynErrorConfig
  }
\end{mathpar}

Expanding the last transition assertion, gives us the case:
\begin{mathpar}
  \inferrule{\op \not\in \{\BAND, \BOR, \IMPL\}\\\\
    \evalexpr{ \env, \veone} \evalarrow \Normal(\vmone, \envone) \\\\
    \evalexpr{ \envone, \vetwo } \evalarrow \Normal(\vmtwo, \newenv) \\\\
    \vmone \eqname (\vvone, \vgone) \\
    \vmtwo \eqname (\vvtwo, \vgtwo) \\
    \binoprel(\op, \vvone, \vvtwo) \evalarrow \DynErrorConfig
  }
  {
    \evalexpr{ \env, \EBinop(\op, \veone, \vetwo) } \evalarrow
    \DynErrorConfig
  }
\end{mathpar}

All these cases are succinctly encoded in a single rule with the alternative output configurations.

\subsection{Boolean Transition Assertions}
\hypertarget{def-booltrans}{}
We define the following rules to allow us to treat assertions as transition assertions:
\begin{mathpar}
  \inferrule[bool\_trans\_true]{}{ \booltrans{\True} \booltransarrow\True }
  \and
  \inferrule[bool\_trans\_false]{}{ \booltrans{\False} \booltransarrow\False }
\end{mathpar}
This is useful in that it allows us to use assertions in rule macros.

\subsection{Assertions Over Optional Data Types}
\hypertarget{def-mapopt}{}
Optional data types are prevalent in the AST.
To facilitate transition assertions over optional data types,
we introduce the parametric function,
which accepts a one-argument relation (or function) $f : A \aslrel B$
and applies it to an optional value $A?$:
\[
\mapopt{\cdot} : \overname{A?}{\vvopt} \aslrel \overname{B?}{\vvoptnew}
\]

\ProseParagraph
\OneApplies
\begin{itemize}
  \item \AllApplyCase{Some}
  \begin{itemize}
    \item $\vvopt$ consists of the value $v$;
    \item applying $f$ to $v$ yields $v'$;
    \item \Proseeqdef{$\vvoptnew$}{the singleton set consisting of $v'$}.
  \end{itemize}

  \item \AllApplyCase{None}
  \begin{itemize}
    \item $\vvopt$ is $\None$;
    \item \Proseeqdef{$\vvoptnew$}{$\None$}.
  \end{itemize}
\end{itemize}

\FormallyParagraph
\begin{mathpar}
\inferrule[Some]{
  f(v) \longrightarrow v'
}{
  \mapopt{f}(\overname{\Some{v}}{\vvopt}) \longrightarrow{r} \Some{v'}
}
\and
\inferrule[None]{}{
  \mapopt{f}(\overname{\None}{\vvopt}) \longrightarrow \None
}
\end{mathpar}

\subsection{Rule Naming}
To name a rule, we place it in a section with its name.
However, some relations are defined by a group of rules.
\hypertarget{def-caserules}{}
In such cases, we refer to the individual rules in a group as \emph{case rules},
or simply \emph{cases}. We annotate case rules by names
appearing above and to the left of the rule. The name of these case rules
is the name of the group, given by its section, followed by the name of the case.

For example, the rule \TypingRuleRef{BaseValue} is defined via multiple
cases. Two of these cases are the following ones:
\begin{mathpar}
\inferrule[t\_bool]{}{
    \basevalue(\tenv, \overname{\TBool}{\vt}) \typearrow \overname{\ELiteral(\lbool(\False))}{\veinit}
}
\end{mathpar}

\begin{mathpar}
\inferrule[t\_real]{}{
    \basevalue(\tenv, \overname{\TReal}{\vt}) \typearrow \overname{\ELiteral(\lreal(0))}{\veinit}
}
\end{mathpar}

The full name of the first case is then \TypingRuleRef{BaseValue}.BOOL
and the full name of the second case is \TypingRuleRef{BaseValue}.REAL.

When explaining rules in English prose, we include the name of the case rules
in parenthesis to make it easier to relate the prose to the corresponding mathematical
definitions (see, for example, the Prose paragraph of \TypingRuleRef{BaseValue}
or that of \TypingRuleRef{ApplyUnopType}).

\subsection{Generic Notations}
\hypertarget{def-wrapline}{}
\begin{itemize}
\item
The notation $\wrappedline$ denotes that a line that is longer than the page width continues on the next line.

\hypertarget{def-commonprefixline}{}
\item The notation $\commonprefixline$ serves as a visual aid to delimit a common prefix of premises shared by rule cases.

\hypertarget{def-commonsuffixline}{}
\item The notation $\commonsuffixline$ serves as a visual aid to delimit a common suffix of premises shared by rule cases.

\hypertarget{tododefine}{}
\item \tododefine{Missing:} Red hyperlinks indicate items that are yet to be defined.
\end{itemize}


%%%%%%%%%%%%%%%%%%%%%%%%%%%%%%%%%%%%%%%%%%%%%%%%%%%%%%%%%%%%%%%%%%%%%%%%%%%%%%%%%%%%
\chapter{Type System Building Blocks}
\label{chap:typesystembuildingblocks}
%%%%%%%%%%%%%%%%%%%%%%%%%%%%%%%%%%%%%%%%%%%%%%%%%%%%%%%%%%%%%%%%%%%%%%%%%%%%%%%%%%%%
This chapter defines necessary mathematical types and concepts for the ASL system.

Types are represented by ASTs derived from the non-terminal $\ty$ (see \cite{ASLAbstractSyntaxReference}
for the precise definition of $\ty$).

\section{Static Environments}

A \emph{static environment} (also called a \emph{type environment}) is what the typing rules operate over:
a structure, which amongst other things, associates types to variables.
Throughout this document, we will use the term environment for static environment, unless otherwise stated.
Intuitively, the typing of a
specification makes an initial environment evolve, with new types as given by the
variable declarations of the specification.

\begin{definition}
\hypertarget{def-staticenvs}{}
Static environments, denoted as $\staticenvs$, are defined as follows (referring to symbols defined by the abstract syntax):
\[
\begin{array}{rcl}
\staticenvs 	          &\triangleq& \mathbb{G} \times \mathbb{L}\\
\mathbb{G} 	            &\triangleq& \declaredtypes \times \constantvalues \times \globalstoragetypes\\
  			                &          & \times\ \subtypes \times \subprograms \times \subprogramrenamings\\
\mathbb{L} 	            &\triangleq& \constantvalues \times \localstoragetypes \times \returntype\\
\hline
\declaredtypes	        &\triangleq& \identifier \partialto \ty\\
\constantvalues         &\triangleq& \identifier \partialto \literal\\
\globalstoragetypes     &\triangleq& \identifier \partialto \ty \times \globaldeclkeyword\\
\localstoragetypes      &\triangleq& \identifier \partialto \ty \times \localdeclkeyword\\
\subtypes		            &\triangleq& \identifier \partialto \identifier\\
\subprograms	          &\triangleq& \identifier \partialto \func\\
\subprogramrenamings	  &\triangleq& \identifier \rightarrow \pow{\func}\\
\returntype             &\triangleq& \langle \ty \rangle
\end{array}
\]
\end{definition}

We use $\tenv$ to range over static environments.

A static environment $\tenv=(G^\tenv, L^\tenv)$ consists of two
distinct components: the global environment $G^\tenv$---pertaining to AST nodes
appearing outside of a given subprogram, and the local environment
$L^\tenv$---pertaining to AST nodes appearing inside a given subprogram.
This separation allows us to typecheck subprograms by using an empty local environment.

The intuitive meaning of each component is as follows:
\begin{itemize}
  \hypertarget{def-declaredtypes}{}
  \item $\declaredtypes$ assigns types to their declared names;
  \hypertarget{def-constantvalues}{}
  \item $\constantvalues$ assigns literals to the their declaring (constant) names;
  \hypertarget{def-globalstoragetypes}{}
  \item $\globalstoragetypes$ associates names of global storage elements to their inferred type
  and how they were declared --- as constants, configuration variables, \texttt{let} variables,
  or mutable variables;
  \hypertarget{def-localstoragetypes}{}
  \item $\localstoragetypes$ associates names of local storage elements to their inferred type
  and how they were declared --- as variables, constants, or as \texttt{let} variables;
  \hypertarget{def-subtypes}{}
  \item $\subtypes$ associates type names to the names that their type subtypes;
  \hypertarget{def-subprograms}{}
  \item $\subprograms$ associates names of subprograms to the $\func$ AST node they were
  declared with;
  \hypertarget{def-subprogramrenamings}{}
  \item $\subprogramrenamings$ associates names of subprograms to the set of overloading
  subprograms ---  $\func$ AST nodes that share the same name;
  \hypertarget{def-returntype}{}
  \item $\returntype$ contains the name of the type that a subprogram declares, if it is
  a function.
\end{itemize}

\hypertarget{def-emptytenv}{}
\begin{definition}[Empty Static Environment]
  The \emph{empty static environment}, \\ denoted as $\emptytenv$, is defined as follows:
  \[
    \emptytenv \triangleq \left(
      \begin{array}{c}
        \overname{
      (\overname{\emptyfunc}{\declaredtypes},
      \overname{\emptyfunc}{\constantvalues},
      \overname{\emptyfunc}{\globalstoragetypes},
      \overname{\emptyfunc}{\subtypes},
      \overname{\emptyfunc}{\subprograms},
      \overname{\emptyfunc}{\subprogramrenamings})}{G},\\
      \overname{
      (
        \overname{\emptyfunc}{\constantvalues},
        \overname{\emptyfunc}{\localstoragetypes},
        \overname{\None}{\returntype}
      )}{L}
      \enspace.
    \end{array}
    \right)
  \]
\end{definition}

The global environment and local environment consist of various components.
We use the notation $G^\tenv.m$ and $L^\tenv.m$ to access the $m$ component of a given environment.

To update a function component $f$ (e.g., $\declaredtypes$) of an environment $\tenv$ (either local or global)
with a new mapping $x \mapsto v$, we use the notation $\tenv.f[x \mapsto v]$ to stand for $\tenv[f \mapsto E.f[x \mapsto v]]$.

This is related to \identd{JRXM} and \identi{ZTMQ}.

\section{Constrained Types}
\begin{itemize}
  \item A \emph{constrained type} is a type whose definition relies on an expression, for example, certain integers and bitvectors.
  \item A type which is not constrained is \emph{unconstrained}.
  \item A constrained type with a non-empty constraint is \emph{well-constrained}.
  \item An \emph{under-constrained integer type} is an implicit type of a subprogram parameter.
  \end{itemize}
The widths of bitvector storage elements are constrained integers.

\hypertarget{def-isunconstrainedinteger}{}
\hypertarget{def-isunderconstrainedinteger}{}
\hypertarget{def-iswellconstrainedinteger}{}
We define the following helper predicates to classify integer types:
\[
  \begin{array}{rcl}
  \isunconstrainedinteger(\overname{\ty}{\vt}) &\aslto& \Bool\\
  \isunderconstrainedinteger(\overname{\ty}{\vt}) &\aslto& \Bool\\
  \iswellconstrainedinteger(\overname{\ty}{\vt}) &\aslto& \Bool
  \end{array}
\]

We define the following shorthands for classifying integers:
\[
  \begin{array}{rcl}
  \isunconstrainedinteger(\vt) &\triangleq& \vt = \TInt(\unconstrained)\\
  \isunderconstrainedinteger(\vt) &\triangleq& \vt = \TInt(\Ignore)\\
  \iswellconstrainedinteger(\vt) &\triangleq& \vt = \TInt(c) \land \astlabel(c)=\wellconstrained\\
\end{array}
\]

\subsection{Comments}
    This is related to \identd{ZTPP}, \identr{WJYH}, \identr{HJPN}, \identr{CZTX}, \identr{TPHR}

\section{ASL Type System}
\hypertarget{def-annotaterel}{}
The type system of ASL is given by the relation $\annotaterel$, which is defined as the disjoint union
of the partial functions and relations defined in this document.
The \hyperlink{def-partialfunc}{partial functions} and relations in this document are defined, in turn, via type system rules.

The output configurations of type assertions have two flavors:
\begin{description}
  \item[Normal Outputs.] \hypertarget{def-normal-type-outputs}{}
  Configurations are typically tuples with different combinations
  of \emph{static environments}, types, and Boolean values.

  \hypertarget{def-typeerror}{}
  \item[Type Errors.] Configurations in $\TypeError(\texttt{<string>})$
  represent type errors, for example, using an integer type as a condition expression, as in \verb|if 5 then 1 else 2|.
  The ASL type system is designed such that when these \emph{type error configurations} appear,
  the typing of the entire specification terminates by outputting them.
\end{description}

We define the mathematical type of type error configurations
(which is needed to define the types of functions in the ASL type system)
as follows:
\hypertarget{def-ttypeerror}{}
\[
  \TTypeError \triangleq \{\TypeErrorVal{\vs} \;|\; \vs \in \texttt{<string>} \} \enspace.
\]

\hypertarget{def-typeerrorconfig}{}
and the shorthand $\TypeErrorConfig \triangleq \TypeError(\vs)$ for type error configurations.

\hypertarget{def-proseortypeerror}{}
\ProseOrTypeError\ means: ``or a type error configuration $\TypeErrorConfig$, which short-circuits the rule,
making it transition into the type error configuration $\TypeErrorConfig$.''\\
We use this when explaining rules in English prose.
%
When several \hyperlink{def-caserules}{case rules} for the same function use the same short-circuiting transition assertion,
we do not repeat the \ProseOrTypeError, but rather include it only in the first rule.

% \subsection*{Rule Example}
% The following rule is used to type a sequence of two statements:
% \[
% \inferrule{
%   \annotatestmt{\tenv, \vs1} = (\newsone, \tenvone)\\
%   \annotatestmt{\tenvone, \vs2} = (\newstwo, \tenvtwo)\\
% }
% {
%   \annotatestmt{\tenv, \SSeq(\vsone, \vstwo)} = (\SSeq(\newsone, \newstwo), \tenvtwo)
% }
% \]
% The rule uses the annotation function $\annotatestmt{\cdot}$, which
% accepts an environment $\tenv$ and two statements and returns a new statement and a new environment.
% The function returns a new statement in order to implement certain code transformations, such as
% inlining setter functions.

\section{Annotation}

Typing a specification consists of annotating the root of its AST with the rules defined
in the remainder of this document.

% The rules are organized into groups that define functions to annotate different types of nodes:
% \begin{itemize}
% \item \texttt{annotate\_expr} annotates expressions;
% \item \texttt{annotate\_slices} annotates slices;
% \item \texttt{annotate\_pattern} annotates pattern;
% \item \texttt{annotate\_local\_decl\_item} annotates local declarations;
% \item \texttt{annotate\_lexpr} annotates left-hand sides of assignments;
% \item \texttt{annotate\_stmt} annotates statements;
% \item \texttt{annotate\_block} annotates blocks;
% \item \texttt{annotate\_catcher} annotates catchers;
% \item \texttt{annotate\_call} annotates functions calls;
% \item \texttt{annotate\_func} annotates functions.
% \end{itemize}

\paragraph{Shorthand Notations:}
\hypertarget{def-elit}{}
\newcommand\ELInt[1]{\hyperlink{def-elit}{\texttt{ELInt}}(#1)}
We use the shorthand $\ELInt{n}$ for the expression denoting the literal integer value $n$. That is, $\ELiteral(\lint(n))$.

\hypertarget{def-unconstrainedinteger}{}
We use the shorthand notation $\unconstrainedinteger$ to denote the unconstrained integer type: $\TInt(\unconstrained)$.

Note that throughout this document we use $\tty$ to denote a type variable, which should not be confused with the abstract syntax variable $\ty$.

%%%%%%%%%%%%%%%%%%%%%%%%%%%%%%%%%%%%%%%%%%%%%%%%%%%%%%%%%%%%%%%%%%%%%%%%%%%%%%%%%%%%
\chapter{Reading guide}
%%%%%%%%%%%%%%%%%%%%%%%%%%%%%%%%%%%%%%%%%%%%%%%%%%%%%%%%%%%%%%%%%%%%%%%%%%%%%%%%%%%%

The definition of each \texttt{annotation\_<label>} function is given by a number of
rules, which follow the possible shapes the \texttt{label} can have. For
example, an expression can be a literal, or a binary operator, amongst other
things. Each of those has its own evaluation rule: TypingRule.ELit in
\secref{TypingRule.ELit} and
Typing.Binop in \secref{TypingRule.Binop}, respectively.

Each rule is presented using the following template:
\begin{itemize}
\item a Prose paragraph gives the rule in English, and corresponds as much as possible to the code of the reference implementation ASLRef given at
\href{https://github.com/herd/herdtools7//tree/master/asllib}{/herdtools7/asllib};
\item one or several Example paragraphs, which as much as possible are also given as regression tests in
\href{https://github.com/herd/herdtools7//tree/master/asllib/tests/ASLTypingReference.t}{/herdtools7/asllib/tests/ASLTypingReference.t};
\ifcode
\item a Code paragraph which gives a verbatim of the corresponding OCaml implementation in the type-checker of ASLRef
\href{https://github.com/herd/herdtools7//tree/master/asllib/Typing.ml}{/herdtools7/asllib/Typing.ml};
\fi
\item Formal paragraphs which give formal definitions of the rule.
\item Comments paragraphs which may refer to one or more statements from the Language Reference Manual~\cite{LRM}
      and may add more information.
\end{itemize}

%%%%%%%%%%%%%%%%%%%%%%%%%%%%%%%%%%%%%%%%%%%%%%%%%%%%%%%%%%%%%%%%%%%%%%%%%%%%%%%%%%%%
\chapter{Domain of Values for Types}
%%%%%%%%%%%%%%%%%%%%%%%%%%%%%%%%%%%%%%%%%%%%%%%%%%%%%%%%%%%%%%%%%%%%%%%%%%%%%%%%%%%%
This chapter formalises the concept of the set of values for a given type.
The formalism is given in the form of rules.
%
The chapter also defines the concept of checking whether the set of values
for one type is included in the set of values for another type.

\hypertarget{def-vals}{}
\section{Native Values \label{sec:nativevalues}}

Types define sets of values that variables can take when a specification is executed.
To formalize this, we define the set of \emph{native values}, denoted $\vals$,
as the minimal set defined by the following recursive rules (\texttt{NV} stands for Native Value):
\hypertarget{def-nvliteral}{}
\hypertarget{def-nvvector}{}
\hypertarget{def-nvrecord}{}
\begin{mathpar}
  \inferrule[(Basis Set: Integers, Reals, Booleans, Strings, and Bitvectors)]
  {\vv \in \literal}
  { \nvliteral{\vv} \in \vals }
  \and
  \inferrule[(Tuple Values and Array Values)]{\vvl \in \vals^*}
  { \nvvector{\vvl} \in \vals }
  \and
  \inferrule[(Record Values)]
  {\vr : \Identifiers \rightarrowfin \vals}
  { \nvrecord{\vr} \in \vals }
\end{mathpar}

We define the following shorthands for native value literals:
\hypertarget{def-nvint}{}
\[
\begin{array}{rcl}
\nvint(z)       &\triangleq& \nvliteral{\lint(z)}           \hypertarget{def-nvbool}{}\\
\nvbool(b)      &\triangleq& \nvliteral{\lbool(b)}          \hypertarget{def-nvreal}{}\\
\nvreal(r)      &\triangleq& \nvliteral{\lreal(r)}          \hypertarget{def-nvstring}{}\\
\nvstring(s)    &\triangleq& \nvliteral{\lstring(s)}        \hypertarget{def-nvbitvector}{}\\
\nvbitvector(v) &\triangleq& \nvliteral{\lbitvector(v)}\\
\end{array}
\]

We define the following types of native values:
\hypertarget{def-tint}{}
\[
\begin{array}{rcl}
  \tint       &\triangleq& \{ \nvint(z) \;|\; z \in \Z\}                                        \hypertarget{def-tbool}{}\\
  \tbool      &\triangleq& \{ \nvbool(\True), \nvbool(\False) \}                                \hypertarget{def-treal}{}\\
  \treal      &\triangleq& \{ \nvreal(r) \;|\; r \in \Q\}                                       \hypertarget{def-tstring}{}\\
  \tstring    &\triangleq& \{ \nvstring(s) \;|\; \texttt{"}s\texttt{"} \in \texttt{<string>}\}  \hypertarget{def-tbitvector}{}\\
  \tbitvector &\triangleq& \{ \nvbitvector(\textit{bits}) \;|\; \textit{bits} \in \{0,1\}^*\}   \hypertarget{def-tvector}{}\\
  \tvector    &\triangleq& \{ \nvvector{\textit{vals}} \;|\; \textit{vals} \in \vals^*\}        \hypertarget{def-trecord}{}\\
  \trecord  &\triangleq& \{ \nvrecord{\fieldmap} \;|\; \fieldmap \in \Identifiers\rightarrow\vals\}\\
\end{array}
\]

\section{Dynamic Domain of a Type}
\hypertarget{def-dyndomain}{}

We now define the concept of a \emph{dynamic domain} of a type
and the \emph{static domain} of a type.
Intuitively, domains assign potentially infinite sets of native values to types.
Dynamic domains are used by the semantics to evaluate expressions of the form \texttt{UNKNOWN: t}
by choosing a single value from the dynamic domain of $\vt$.
Static domains are used to define (domain) subtype satisfaction in \secref{TypingRule.DomainSubtypeSatisfaction}.

\hypertarget{def-dynamicenvs}{}
The definition of a dynamic domain refers to \emph{dynamic environments}, denoted $\dynamicenvs$,
which assigns native values to identifiers~\cite{ASLSemanticsReference}.

\hypertarget{def-envs}{}
We define \emph{environments} as pairs of static environments and dynamic environments:
$\envs \triangleq \staticenvs \times \dynamicenvs$.

Formally, the partial function
\[
  \dynamicdomain : \overname{\envs}{\env} \times \overname{\ty}{\vt}
  \partialto \overname{\pow{\vals}}{\vd}
\]
assigns the set of values that a type $\vt$ can hold in a given environment $\env$.
%
We say that $\dynamicdomain(\env, \vt)$ is the \emph{dynamic domain} of $\vt$
in the environment $\env$.
%
The \emph{static domain} of a type is the set of values which storage elements of that type may hold
\underline{across all possible dynamic environments}.
%
The reason for this distinction is that the sets of values
of integer types, bitector types, array types can depend on the dynamic values of variables.

Types that do not refer to variables whose values are only known dynamically have
a static domain that is equal to any of their dynamic domains.
In those cases, we simply refer to their \emph{domain}.

Associating a set of values to a type is done by evaluating any expression appearing
in the type definitions. Evaluation is defined by the ASL semantics~\cite{ASLSemanticsReference}
via the relation
\hypertarget{def-evalexprsef}{}
\[
  \evalexprsef{\overname{\envs}{\env} \aslsep \overname{\expr}{\ve}} \;\aslrel\;
  \Normal(\overname{\vals}{\vv}\aslsep\overname{\XGraphs}{\vg}) \cup
  \overname{\TError}{\ErrorConfig}
\]
\hypertarget{def-errorconfig}{}
which evaluates side-effect-free expressions and either returns
a configuration of the form $\Normal(\vv,\vg)$ or a dynamic error configuration $\ErrorConfig$.
In the first case, $\vv$ is a native value and $\vg$
is an \emph{execution graph}. Execution graphs are related to the concurrent semantics
and can be ignored in the context of defining dynamic domains.
In the latter case (which can occur if, for example, an expression attempts to divide
\texttt{8} by \texttt{0}), a dynamic error configuration, for which we use the notation
$\ErrorConfig$, is returned.
%
The dynamic domain is empty in cases where evaluating side-effect-free expressions
results in a dynamic error.
%
The dynamic domain is undefined if the type $\vt$ is not well-typed in $\tenv$.
That is, if $\annotatetype{\tenv, \vt} \typearrow \TypeErrorConfig$.

As part of the definition, we also associate dynamic domains to integer constraints
by overloading $\dynamicdomain$:
\[
  \dynamicdomain : \overname{\envs}{\env} \times \overname{\intconstraint}{\vc}
  \partialto \overname{\pow{\vals}}{\vd}
\]

\subsection{Prose}
For an environment $\env \in \envs$ and a type $\vt$, the domain is $\vd$ and one of the following applies:
\begin{itemize}
  \item All of the following apply (\textsc{t\_bool}):
  \begin{itemize}
    \item $\vt$ is the Boolean type, $\TBool$;
    \item $\vd$ is the set of native Boolean values, $\tbool$.
  \end{itemize}

  \item All of the following apply (\textsc{t\_string}):
  \begin{itemize}
    \item $\vt$ is the string type, $\TString$;
    \item $\vd$ is the set of all native string values, $\tstring$.
  \end{itemize}

  \item All of the following apply (\textsc{t\_real}):
  \begin{itemize}
    \item $\vt$ is the real type, $\TReal$;
    \item $\vd$ is the set of all native real values, $\treal$.
  \end{itemize}

  \item All of the following apply (\textsc{t\_enumeration}):
  \begin{itemize}
    \item $\vt$ is the enumeration type with labels $\id_{1..k}$, that is $\TEnum(\id_{1..k})$;
    \item $\vd$ is the set of all native integer values for $1..k$.\\
    \textbf{Why represent enumeration domains via integers:}
    Conceptually, enumeration labels carry two pieces of information --- the identifiers themselves
    and their position in the list of identifiers, which are used for accessing arrays.
    For the purpose of type-checking, we use the identifiers, but for the purpose of the semantics
    and the domain of values, only the positions are relevant.
  \end{itemize}

  \item All of the following apply (\textsc{t\_int\_unconstrained}):
  \begin{itemize}
    \item $\vt$ is the unconstrained integer type, $\TInt(\unconstrained)$;
    \item $\vd$ is the set of all native integer values, $\tint$.
  \end{itemize}

  \item All of the following apply (\textsc{t\_int\_well\_constrained}):
  \begin{itemize}
    \item $\vt$ is the well-constrained integer type $\TInt(\wellconstrained(\vc_{1..k}))$;
    \item $\vd$ is the union of the dynamic domains of each of the constraints $vc_{1..k}$ in $\env$.
  \end{itemize}

  \item All of the following apply (\textsc{constraint\_exact\_okay}):
  \begin{itemize}
    \item $\vc$ is a constraint consisting of a single side-effect-free expression $\ve$, that is, $\ConstraintExact(\ve)$;
    \item evaluating $\ve$ in $\env$, results in a configuration with the native integer for $n$;
    \item $\vd$ is the set containing the single native integer value for $n$.
  \end{itemize}

  \item All of the following apply (\textsc{constraint\_exact\_dynamic\_error}):
  \begin{itemize}
    \item $\vc$ is a constraint consisting of a single side-effect-free expression $\ve$, that is, $\ConstraintExact(\ve)$;
    \item evaluating $\ve$ in $\env$, results in a dynamic error configuration;
    \item $\vd$ is the empty set.
  \end{itemize}

  \item All of the following apply (\textsc{constraint\_range\_okay}):
  \begin{itemize}
    \item $\vc$ is a range constraint consisting of a two side-effect-free expressions $\veone$ and $\vetwo$, that is, $\ConstraintRange(\veone, \vetwo)$;
    \item evaluating $\veone$ in $\env$, results in a configuration with the native integer for $a$;
    \item evaluating $\vetwo$ in $\env$, results in a configuration with the native integer for $b$;
    \item $a$ is less than or equal to $b$;
    \item $\vd$ is the set containing all native integer values for integers from $a$ to $b$, inclusive.
  \end{itemize}

  \item All of the following apply (\textsc{constraint\_range\_dynamic\_error1}):
  \begin{itemize}
    \item $\vc$ is a range constraint consisting of a two side-effect-free expressions $\veone$ and $\vetwo$, that is, $\ConstraintRange(\veone, \vetwo)$;
    \item evaluating $\veone$ in $\env$, results in a dynamic error configuration;
    \item $\vd$ is the empty set.
  \end{itemize}

  \item All of the following apply (\textsc{constraint\_range\_dynamic\_error2}):
  \begin{itemize}
    \item $\vc$ is a range constraint consisting of a two side-effect-free expressions $\veone$ and $\vetwo$, that is, $\ConstraintRange(\veone, \vetwo)$;
    \item evaluating $\veone$ in $\env$, results in a configuration with the native integer for $a$;
    \item evaluating $\vetwo$ in $\env$, results in a dynamic error configuration;
    \item $\vd$ is the empty set.
  \end{itemize}

  \item All of the following apply (\textsc{constraint\_non\_range\_error}):
  \begin{itemize}
    \item $\vc$ is a range constraint consisting of a two side-effect-free expressions $\veone$ and $\vetwo$, that is, $\ConstraintRange(\veone, \vetwo)$;
    \item evaluating $\veone$ in $\env$, results in a configuration with the native integer for $a$;
    \item evaluating $\vetwo$ in $\env$, results in a configuration with the native integer for $b$;
    \item $a$ is greater than $b$;
    \item $\vd$ is the empty set.
  \end{itemize}

  \item All of the following apply (\textsc{t\_int\_underconstrained}):
  \begin{itemize}
    \item $\vt$ is an under constrained integer for parameter $\id$, \\ $\TInt(\underconstrained(\id))$;
    \item the native value associated with $\id$ in the local dynamic environment is the native integer value for $n$;
    \item $\vd$ is the set containing the single integer value for $n$.
  \end{itemize}

  \item All of the following apply (\textsc{t\_bits\_dynamic\_error}):
  \begin{itemize}
    \item $\vt$ is a bitvector type with size expression $\ve$, $\TBits(\ve, \Ignore)$;
    \item evaluating $\ve$ in $\env$, results in a dynamic error configuration;
    \item $\vd$ is the empty set.
  \end{itemize}

  \item All of the following apply (\textsc{t\_bits\_negative\_width\_error}):
  \begin{itemize}
    \item $\vt$ is a bitvector type with size expression $\ve$, $\TBits(\ve, \Ignore)$;
    \item evaluating $\ve$ in $\env$, results in a configuration with the native integer for $k$;
    \item $k$ is negative;
    \item $\vd$ is the empty set.
  \end{itemize}

  \item All of the following apply (\textsc{t\_bits\_empty}):
  \begin{itemize}
    \item $\vt$ is a bitvector type with size expression $\ve$, $\TBits(\ve, \Ignore)$;
    \item evaluating $\ve$ in $\env$, results in a configuration with the native integer for $0$;
    \item $\vd$ is the set containing the single native value for an empty bitvector.
  \end{itemize}

  \item All of the following apply (\textsc{t\_bits\_non\_empty}):
  \begin{itemize}
    \item $\vt$ is a bitvector type with size expression $\ve$, $\TBits(\ve, \Ignore)$;
    \item evaluating $\ve$ in $\env$, results in a configuration with the native integer for $k$;
    \item $k$ is greater than $0$;
    \item $\vd$ is the set containing all native values for bitvectors of size exactly $k$.
  \end{itemize}

  \item All of the following apply (\textsc{t\_tuple}):
  \begin{itemize}
    \item $\vt$ is a tuple type over types $\vt_i$, for $i=1..k$, $\TTuple(\vt_{1..k})$;
    \item the domain of each element $\vt_i$ is $D_i$, for $i=1..k$;
    \item evaluating $\ve$ in $\env$, results in a configuration with the native integer for $k$;
    \item $\vd$ is the set containing all native vectors of $k$ values, where the value at position $i$
    is from $D_i$.
  \end{itemize}

  \item All of the following apply (\textsc{t\_array\_dynamic\_error}):
  \begin{itemize}
    \item $\vt$ is an array type with length expression $\ve$ and element type $\vt_i$, for $i=1..k$, $\TArray(\ve, \vtone)$;
    \item evaluating $\ve$ in $\env$, results in a dynamic error configuration;
    \item $\vd$ is the empty set.
  \end{itemize}

  \item All of the following apply (\textsc{t\_array\_negative\_length\_error}):
  \begin{itemize}
    \item $\vt$ is an array type with length expression $\ve$ and element type $\vt_i$, for $i=1..k$, $\TArray(\ve, \vtone)$;
    \item evaluating $\ve$ in $\env$, results in a configuration with the native integer for $k$;
    \item $k$ is negative;
    \item $\vd$ is the empty set.
  \end{itemize}

  \item All of the following apply (\textsc{t\_array\_okay}):
  \begin{itemize}
    \item $\vt$ is an array type with length expression $\ve$ and element type $\vt_i$, for $i=1..k$, $\TArray(\ve, \vtone)$;
    \item evaluating $\ve$ in $\env$, results in a configuration with the native integer for $k$;
    \item $k$ is greater than or equal to $0$;
    \item the domain of $\vtone$ is $D_\vtone$;
    \item $\vd$ is the set containing all native vectors of $k$ values taken from $D_\vtone$.
  \end{itemize}

  \item All of the following apply (\textsc{t\_structured}):
  \begin{itemize}
    \item $\vt$ is either a record type or an exception type with typed fields $(\id_i, \vt_i$, for $i=1..k$, that is $L([i=1..k: (\id_i,\vt_i))]$
    where $L\in\{\TRecord, \TException\}$;
    \item the domain of each type $\vt_i$ is $D_i$, for $i=1..k$;
    \item $\vd$ is the set containing all native records where $\id_i$ is mapped to a value taken from $D_i$.
  \end{itemize}

  \item All of the following apply (\textsc{t\_named}):
  \begin{itemize}
    \item $\vt$ is a named type with name $\id$, $\TNamed(\id)$;
    \item the type associated with $\id$ in $\tenv$ is $\tty$;
    \item $\vd$ is the domain of $\tty$ in $\env$.
  \end{itemize}
\end{itemize}

\subsection{Formally}

\begin{mathpar}
\inferrule[t\_bool]{}{ \dynamicdomain(\env, \TBool) = \tbool }
\and
\inferrule[t\_string]{}{ \dynamicdomain(\env, \TString) = \tstring }
\and
\inferrule[t\_real]{}{ \dynamicdomain(\env, \TReal) = \treal }
\and
\inferrule[t\_enumeration]{}{
  \dynamicdomain(\env, \TEnum(\id_{1..k})) = \{\nvint(1),\ldots,\nvint(k)\}
}
\end{mathpar}

\begin{mathpar}
  \inferrule[t\_int\_unconstrained]{}{
  \dynamicdomain(\env, \TInt(\unconstrained)) = \tint
}
\end{mathpar}

\begin{mathpar}
\inferrule[t\_int\_well\_constrained]{}{
  \dynamicdomain(\env, \TInt(\wellconstrained(\vc_{1..k}))) = \bigcup_{i=1}^k \dynamicdomain(\env, \vc_i)
}
\end{mathpar}

\begin{mathpar}
\inferrule[constraint\_exact\_okay]{
  \evalexprsef{\env, \ve} \evalarrow \Normal(\nvint(n), \Ignore)
}{
  \dynamicdomain(\env, \ConstraintExact(\ve)) = \{ \nvint(n) \}
}
\and
\inferrule[constraint\_exact\_dynamic\_error]{
  \evalexprsef{\env, \ve} \evalarrow \ErrorConfig
}{
  \dynamicdomain(\env, \ConstraintExact(\ve)) = \emptyset
}
\end{mathpar}

\begin{mathpar}
\inferrule[constraint\_range\_okay]{
  \evalexprsef{\env, \veone} \evalarrow \Normal(\nvint(a), \Ignore)\\
  \evalexprsef{\env, \vetwo} \evalarrow \Normal(\nvint(b), \Ignore)\\
  a \leq b
}{
  \dynamicdomain(\env, \ConstraintRange(\veone, \vetwo)) = \{ \nvint(n) \;|\;  a \leq n \leq b\}
}
\and
\inferrule[constraint\_range\_dynamic\_error1]{
  \evalexprsef{\env, \veone} \evalarrow \ErrorConfig
}{
  \dynamicdomain(\env, \ConstraintRange(\veone, \vetwo)) = \emptyset
}
\and
\inferrule[constraint\_range\_dynamic\_error2]{
  \evalexprsef{\env, \veone} \evalarrow \Normal(\Ignore, \Ignore)\\
  \evalexprsef{\env, \vetwo} \evalarrow \ErrorConfig
}{
  \dynamicdomain(\env, \ConstraintRange(\veone, \vetwo)) = \emptyset
}
\and
\inferrule[constraint\_range\_non\_range\_error]{
  \evalexprsef{\env, \veone} \evalarrow \Normal(\nvint(a), \Ignore)\\
  \evalexprsef{\env, \vetwo} \evalarrow \Normal(\nvint(b), \Ignore)\\
  a > b
}{
  \dynamicdomain(\env, \ConstraintRange(\veone, \vetwo)) = \emptyset
}
\end{mathpar}

The notation $L^\denv(\id)$ denotes the native value associated with the identifier $\id$
in the \emph{local dynamic environment} of $\denv$.
\begin{mathpar}
  \inferrule[t\_int\_underconstrained]{
  L^\denv(\id) = \nvint(n)
}{
  \dynamicdomain(\env, \TInt(\underconstrained(\id))) = \{ \nvint(n) \}
}
\end{mathpar}

\begin{mathpar}
\inferrule[t\_bits\_dynamic\_error]{
  \evalexprsef{\env, \ve} \evalarrow \ErrorConfig
}{
  \dynamicdomain(\env, \TBits(\ve, \Ignore)) = \emptyset
}
\and
\inferrule[t\_bits\_negative\_width\_error]{
  \evalexprsef{\env, \ve} \evalarrow \Normal(\nvint(k), \Ignore)\\
  k < 0
}{
  \dynamicdomain(\env, \TBits(\ve, \Ignore)) = \emptyset
}
\and
\inferrule[t\_bits\_empty]{
  \evalexprsef{\env, \ve} \evalarrow \Normal(\nvint(0), \Ignore)
}{
  \dynamicdomain(\env, \TBits(\ve, \Ignore)) = \{ \nvbitvector(\emptylist) \}
}
\and
\inferrule[t\_bits\_non\_empty]{
  \evalexprsef{\env, \ve} \evalarrow \Normal(\nvint(k), \Ignore)\\
  k > 0
}{
  \dynamicdomain(\env, \TBits(\ve, \Ignore)) = \{ \nvbitvector(\vb_{1..k}) \;|\; \vb_1,\ldots,\vb_k \in \{0,1\} \}
}
\end{mathpar}

\begin{mathpar}
\inferrule[t\_tuple]{
  i=1..k: \dynamicdomain(\env, \vt_i) = D_i
}{
  \dynamicdomain(\env, \TTuple(\vt_{1..k})) =
  \{ \nvvector{\vv_{1..k}} \;|\; \vv_i \in D_i \}
}
\end{mathpar}

\begin{mathpar}
\inferrule[t\_array\_dynamic\_error]{
  \evalexprsef{\env, \ve} \evalarrow \ErrorConfig
}{
  \dynamicdomain(\env, \TArray(\ve, \vtone)) = \emptyset
}
\and
\inferrule[t\_array\_negative\_length\_error]{
  \evalexprsef{\env, \ve} \evalarrow \Normal(\nvint(k), \Ignore)\\
  k < 0
}{
  \dynamicdomain(\env, \TArray(\ve, \vtone)) = \emptyset
}
\and
\inferrule[t\_array\_okay]{
  \evalexprsef{\env, \ve} \evalarrow \Normal(\nvint(k), \Ignore)\\
  \dynamicdomain(\env, \vtone) = D_\vtone
}{
  \dynamicdomain(\env, \TArray(\ve, \vtone)) =
  \{ \nvvector{\vv_{1..k}} \;|\; \vv_{1..k} \in D_{\vtone} \}
}
\end{mathpar}

\begin{mathpar}
\inferrule[structured]{
  L \in \{\TRecord, \TException\}\\
  i=1..k: \dynamicdomain(\env, \vt_i) = D_i
}{
  \dynamicdomain(\env, L([i=1..k: (\id_i,\vt_i))]) = \\
  \{ \nvrecord{\{i=1..k: \id_i\mapsto \vv_i\}} \;|\; \vv_i \in D_i \}
}
\end{mathpar}

\begin{mathpar}
\inferrule[t\_named]{
  G^\tenv.\declaredtypes(\id)=\tty
}{
  \dynamicdomain(\env, \TNamed(\id)) = \dynamicdomain(\env, \tty)
}
\end{mathpar}

\subsection{Example}
The domain of \texttt{integer} is the infinite set of all integers.

The domain of \texttt{bits(1)} is the set $\{\nvint(1), \nvint(2)\}$.

The domain of \verb|integer {2,16}| is the set $\{\nvint(2), \nvint(16)\}$.

The domain of \verb|enumeration {GREEN, ORANGE, RED}| is the set \\
$\{\nvint(1), \nvint(2), \nvint(3)\}$ and so is the domain
of \\
\verb|type TrafficLights of enumeration {GREEN, ORANGE, RED}|.

The domain of \texttt{bits({2,16})} is the set containing native bitvectors of all 2-bit and all 16-bit binary sequences.

The domain of \texttt{(integer, integer)} is the set containing all pairs of native integer values.

The domain of \verb|record {a: integer;  b: boolean}| contains all native records
that map \texttt{a} to a native integer value and \texttt{b} to a native Boolean value.

The dynamic domain of a subprogram parameter \texttt{N: integer} is the (singleton) set containing
the native integer value $c$,
which is assigned to \texttt{N} by a given dynamic environment. The static domain of that parameter
is the infinite set of all native integer values.

This is related to \identd{BMGM}, \identr{PHRL}, \identr{PZNR},
\identr{RLQP}, \identr{LYDS}, \identr{SVDJ}, \identi{WLPJ}, \identr{FWMM},
\identi{WPWL}, \identi{CDVY}, \identi{KFCR}, \identi{BBQR}, \identr{ZWGH},
\identr{DKGQ}, \identr{DHZT}, \identi{HSWR}, \identd{YZBQ}.

\section{Subsumption Testing}
Whether an assignment statement is well-typed depends on whether the dynamic domain of the
right hand side type is contained in the dynamic domain of the left hand side type,
for any given dynamic environment
(see \secref{TypingRule.DomainSubtypeSatisfaction} where this is checked).

\begin{definition}[Subsumption]
For any given types $\vt$ and $\vs$ and static environment $\tenv$,
we say that $\vt$ \emph{subsumes} $\vs$ in $\tenv$,
if the following condition holds:
\hypertarget{def-subsumes}{}
\begin{equation}
  \subsumes(\tenv, \vt, \vs) \triangleq \forall \denv\in\dynamicenvs.\ \dynamicdomain((\tenv, \denv), \vt) \supseteq \dynamicdomain((\tenv, \denv), \vs) \enspace.
\end{equation}
\end{definition}

For example, consider the assignment
\begin{center}
\verb|var x : integer{1,2,3} = UNKNOWN : integer{1,2};|
\end{center}

It is legal, since (in any static environment), the domain of \verb|integer{1,2,3}|
is \\
$\{\nvint(1), \nvint(2), \nvint(3)\}$, which subsumes
the domain of \verb|integer{1,2}|, which is \\ $\{\nvint(1), \nvint(2)\}$.

Since dynamic domains are potentially infinite, this requires \emph{symbolic reasoning}.
Furthermore, since any (statically evaluable) expressions may appear inside integer and bitvector
types, testing subsumption is undecidable.
We therefore approximate subsumption testing \emph{conservatively} via the predicate $\symsubsumes(\tenv, \vt, \vs)$.

\begin{definition}[Sound Subsumption Test]
A predicate
\[
  \symsubsumes(\overname{\staticenvs}{\tenv} \aslsep \overname{\ty}{\vt} \aslsep \overname{\ty}{\vs}) \aslto \Bool
\]
is \emph{sound} if the following condition holds:
\begin{equation}
  \begin{array}{l}
  \forall \vt,\vs\in\ty.\ \tenv\in\staticenvs. \\
  \;\;\;\; \symsubsumes(\tenv, \vt, \vs) \typearrow \True \;\Longrightarrow\; \subsumes(\tenv, \vt, \vs)  \enspace.
  \end{array}
\end{equation}
\end{definition}

That is, if a sound subsumption test returns a positive answer, it means that
$\vt$ definitely \emph{subsumes} $\vs$ in the static environment $\tenv$.
This is referred to as a \emph{true positive}.
However, a negative answer means one of two things:
\begin{description}
  \item[True Negative:] indeed, $\vt$ does not subsume $\vs$ in the static environment $\tenv$; or
  \item[False Negative:] the symbolic reasoning is unable to decide.
\end{description}

In other words, $\symsubsumes(\tenv, \vt, \vs)$ errs on the \emph{safe side} ---
it never answers $\True$ when the real answer is $\False$, which would (undesirably)
determine the following statement as well-typed:
\begin{center}
  \verb|var x : integer{1,2} = integer;|
\end{center}

A sound but trivial subsumption test is one that always returns $\False$.
However, that would make many useful examples, such as the one above,
be considered as not well-typed. Indeed, it has the maximal set of false negatives.
Reducing the set of false negatives requires stronger symbolic reasoning algorithms,
which inevitably leads to higher computational complexity.
%
The symbolic subsumption test in \chapref{symbolicsubsumptiontesting}
attempts to accept a large enough set of true positives, based on empirical trial and error,
while maintaining the computational complexity of the symbolic reasoning relatively low.

%%%%%%%%%%%%%%%%%%%%%%%%%%%%%%%%%%%%%%%%%%%%%%%%%%%%%%%%%%%%%%%%%%%%%%%%%%%%%%%%%%%%
\chapter{Basic Type Attributes}
%%%%%%%%%%%%%%%%%%%%%%%%%%%%%%%%%%%%%%%%%%%%%%%%%%%%%%%%%%%%%%%%%%%%%%%%%%%%%%%%%%%%

This chapter defines some basic predicates for classifying types as well as
functions that inspect the structure of types:
\begin{itemize}
  \item Builtin singular types (\secref{TypingRule.BuiltinSingularType})
  \item Builtin aggregate types (\secref{TypingRule.BuiltinAggregateType})
  \item Buitin types (\secref{TypingRule.BuiltinSingularOrAggregate})
  \item Named types (\secref{TypingRule.NamedType})
  \item Anonymous types (\secref{TypingRule.AnonymousType})
  \item Singular types (\secref{TypingRule.SingularType})
  \item Aggregate types (\secref{TypingRule.AggregateType})
  \item Non-primitive types (\secref{TypingRule.NonPrimitiveType})
  \item Primitive types (\secref{TypingRule.PrimitiveType})
  \item The structure of a type (\secref{structure})
  \item The underlying type of a type (\secref{anonymize})
  \item Checked constrained integers (\secref{TypingRule.CheckConstrainedInteger})
\end{itemize}

\section{TypingRule.BuiltinSingularType \label{sec:TypingRule.BuiltinSingularType}}
\hypertarget{def-isbuiltinsingular}{}
The predicate
\[
  \isbuiltinsingular(\overname{\ty}{\tty}) \;\aslto\; \Bool
\]
tests whether the type $\tty$ is a \emph{builtin singular type}.

\subsection{Prose}
The \emph{builtin singular types} are:
\begin{itemize}
\item \texttt{integer};
\item \texttt{real};
\item \texttt{string};
\item \texttt{boolean};
\item \texttt{bits} (which also represents \texttt{bit}, as a special case);
\item \texttt{enumeration}.
\end{itemize}

\subsection{Example}

In this example:
\VerbatimInput[firstline=3,lastline=8]{\tests/TypingRule.BuiltinSingularTypes.asl}

Variables of builtin singular types \texttt{integer}, \texttt{real},
\texttt{boolean}, \texttt{bits(4)}, \\ and~\texttt{bits(2)} are defined.

\subsection{Example}
\VerbatimInput{\tests/TypingRule.EnumerationType.asl}
The builtin singular type \texttt{Color} consists in two constants
\texttt{RED}, and~\texttt{BLACK}.

\CodeSubsection{\BuiltinSingularBegin}{\BuiltinSingularEnd}{../types.ml}

\begin{emptyformal}
\subsection{Formally}
\begin{mathpar}
\inferrule{
  \vb \eqdef \astlabel(\tty) \in \{\TReal, \TString, \TBool, \TBits, \TEnum, \TInt\}
}{
  \isbuiltinsingular(\tty) \typearrow \vb
}
\end{mathpar}
\end{emptyformal}

\subsection{Comments}
This is related to \identd{PQCK} and \identd{NZWT}.

\section{TypingRule.BuiltinAggregateType \label{sec:TypingRule.BuiltinAggregateType}}
\hypertarget{def-isbuiltinaggregate}{}
The predicate
\[
  \isbuiltinaggregate(\overname{\ty}{\tty}) \;\aslto\; \Bool
\]
tests whether the type $\tty$ is a \emph{builtin aggregate type}.

\subsection{Prose}
The builtin aggregate types are:
\begin{itemize}
\item tuple;
\item \texttt{array};
\item \texttt{record};
\item \texttt{exception}.
\end{itemize}

\subsection{Example}
\VerbatimInput{\tests/TypingRule.BuiltinAggregateTypes.asl}
Type \texttt{Pair} is the type of integer and boolean pairs.

Arrays are declared with indices that are either integer-typed
or enumeration-typed.  In the example above, \texttt{T} is
declared as an array with an integer-typed index (as indicated
by the used of the integer-typed constant \texttt{3}) whereas
\texttt{PointArray} is declared with the index of
\texttt{Coord}, which is an enumeration type.

Arrays declared with integer-typed indices can be accessed only by integers ranging from $0$ to
the size of the array minus $1$. In the example above, $\texttt{T}$ can be accessed with
one of $0$, $1$, and $2$.

Arrays declared with an enumeration-typed index can only be accessed with labels from the corresponding
enumeration. In the example above, \texttt{PointArray} can only be accessed with one of the labels
\texttt{CX}, \texttt{CY}, and \texttt{CZ}.

The (builtin aggregate) type \verb|{ x : real, y : real, z : real }| is a record type with three fields
\texttt{x}, \texttt{y} and \texttt{z}.

\subsection{Example}
\VerbatimInput{\tests/TypingRule.BuiltinExceptionType.asl}
Two (builtin aggregate) exception types are defined:
\begin{itemize}
\item \verb|exception{}| (for \texttt{Not\_found}), which carries no value; and
\item \verb|exception { message:string }| (for \texttt{SyntaxException}), which carries a message.
\end{itemize}
Notice the similarity with record types and that the empty field list \verb|{}| can be
omitted in type declarations, as is the case for \texttt{Not\_found}.

\CodeSubsection{\BuiltinAggregateBegin}{\BuiltinAggregateEnd}{../types.ml}

\begin{emptyformal}
\subsection{Formally}
\begin{mathpar}
\inferrule{ \vb \eqdef \astlabel(\tty) \in \{\TTuple, \TArray, \TRecord, \TException\} }
{ \isbuiltinaggregate(\tty) \typearrow \vb }
\end{mathpar}
\end{emptyformal}

\subsection{Comments}
This is related to \identd{PQCK} and \identd{KNBD}.

\section{TypingRule.BuiltinSingularOrAggregate \label{sec:TypingRule.BuiltinSingularOrAggregate}}
\hypertarget{def-isbuiltin}{}
The predicate
\[
  \isbuiltin(\overname{\ty}{\tty}) \;\aslto\; \Bool
\]
tests whether the type $\tty$ is a \emph{builtin type}.

\subsection{Prose}
$\tty$ is a builtin type and one of the following applies:
\begin{itemize}
\item $\tty$ is singular;
\item $\tty$ is builtin aggregate.
\end{itemize}

\subsection{Example}
In ``\texttt{type ticks of integer;}'', the type \texttt{integer} is a builtin type but the named type \texttt{ticks} is not.

\CodeSubsection{\BuiltinSingularOrAggregateBegin}{\BuiltinSingularOrAggregateEnd}{../types.ml}

\begin{emptyformal}
\subsection{Formally}
\begin{mathpar}
  \inferrule{
    \isbuiltinsingular(\tty) \typearrow \vbone\\
    \isbuiltinaggregate(\tty) \typearrow \vbtwo
  }{
    \isbuiltin(\tty) \typearrow \vbone \lor \vbtwo
  }
\end{mathpar}
\end{emptyformal}

\isempty{\subsection{Comments}}

\section{TypingRule.NamedType \label{sec:TypingRule.NamedType} }
\hypertarget{def-isnamed}{}
The predicate
\[
  \isnamed(\overname{\ty}{\tty}) \;\aslto\; \Bool
\]
tests whether the type $\tty$ is a \emph{named type}.

\subsection{Prose}
A named type is a type that is declared by using the \texttt{type of} syntax.

\subsection{Example}
The type \texttt{ticks} in ``\texttt{type ticks of integer;}'' is a named type.

\CodeSubsection{\NamedBegin}{\NamedEnd}{../types.ml}

\begin{emptyformal}
\subsection{Formally}
\begin{mathpar}
\inferrule{
  \vb \eqdef \astlabel(\tty) = \TNamed
}{
  \isnamed(\tty) \typearrow \vb
}
\end{mathpar}
\end{emptyformal}

\subsection{Comments}
This is related to \identd{vmzx}.

\section{TypingRule.AnonymousType \label{sec:TypingRule.AnonymousType}}
\hypertarget{def-isanonymous}{}
The predicate
\[
  \isanonymous(\overname{\ty}{\tty}) \;\aslto\; \Bool
\]
tests whether the type $\tty$ is an \emph{anonymous type}.

\subsection{Prose}
An anonymous type is a type that is not declared using the type syntax.

\subsection{Example}
The tuple type \texttt{(integer, integer)} is an anonymous type.

\CodeSubsection{\AnonymousBegin}{\AnonymousEnd}{../types.ml}

\begin{emptyformal}
\subsection{Formally}
\begin{mathpar}
\inferrule{ \vb \eqdef \astlabel(\tty) \neq \TNamed }
{ \isanonymous(\tty) \typearrow \vb }
\end{mathpar}
\end{emptyformal}

\subsection{Comments}
This is related to \identd{VMZX}.

\section{TypingRule.SingularType \label{sec:TypingRule.SingularType}}
\hypertarget{def-issingular}{}
The predicate
\[
  \issingular(\overname{\staticenvs}{\tenv} \aslsep \overname{\ty}{\tty}) \;\aslto\;
  \overname{\Bool}{\vb} \cup \overname{\TTypeError}{\TypeErrorConfig}
\]
tests whether the type $\tty$ is a \emph{singular type} in the static environment $\tenv$.

\subsection{Prose}
A type $\tty$ is singular if and only if all of the following apply:
\begin{itemize}
  \item obtaining the \underlyingtype\ of $\tty$ in the environment $\tenv$ yields $\vtone$ \ProseOrTypeError;
  \item $\vtone$ is a builtin singular type.
\end{itemize}

\subsection{Example}
In the following example, the types \texttt{A}, \texttt{B}, and \texttt{C} are all singular types:
\begin{verbatim}
type A of integer;
type B of A;
type C of B;
\end{verbatim}

\CodeSubsection{\SingularBegin}{\SingularEnd}{../types.ml}
\begin{emptyformal}
\subsection{Formally}
\begin{mathpar}
\inferrule{
  \makeanonymous(\tenv, \vt) \typearrow \vtone \OrTypeError\\\\
  \isbuiltinsingular(\vtone) \typearrow \vb
}{
\issingular(\tenv, \tty) \typearrow \vb
}
\end{mathpar}
\end{emptyformal}

\subsection{Comments}
This is related to \identr{GVZK}.

\section{TypingRule.AggregateType \label{sec:TypingRule.AggregateType}}
\hypertarget{def-isbuiltinaggregate}{}
The predicate
\[
  \isaggregate(\overname{\staticenvs}{\tenv} \aslsep \overname{\ty}{\tty}) \;\aslto\;
  \overname{\Bool}{\vb} \cup \overname{\TTypeError}{\TypeErrorConfig}
\]
tests whether the type $\tty$ is an \emph{aggregate type} in the static environment $\tenv$.

\subsection{Prose}
A type $\tty$ is aggregate in an environment $\tenv$ if and only if all of the following apply:
\begin{itemize}
  \item obtaining the \underlyingtype\ of $\tty$ in the environment $\tenv$ yields $\vtone$ \ProseOrTypeError;
  \item $\vtone$ is a builtin aggregate.
\end{itemize}

\subsection{Example}
In the following example, the types \texttt{A}, \texttt{B}, and \texttt{C} are all aggregate types:
\begin{verbatim}
type A of (integer, integer);
type B of A;
type C of B;
\end{verbatim}

\CodeSubsection{\AggregateBegin}{\AggregateEnd}{../types.ml}

\begin{emptyformal}
\subsection{Formally}
\begin{mathpar}
\inferrule{
  \makeanonymous(\tenv, \tty) \typearrow \vtone \OrTypeError\\\\
  \isbuiltinaggregate(\vtone) \typearrow \vb
}{
  \isaggregate(\tenv, \tty) \typearrow \vb
}
\end{mathpar}
\end{emptyformal}

\subsection{Comments}
This is related to \identr{GVZK}.

\section{TypingRule.NonPrimitiveType \label{sec:TypingRule.NonPrimitiveType}}
\hypertarget{def-isnonprimitive}{}
The predicate
\[
  \isnonprimitive(\overname{\ty}{\tty}) \;\aslto\; \overname{\Bool}{\vb}
\]
tests whether the type $\tty$ is a \emph{non-primitive type}.

\subsection{Prose}
One of the following applies:
\begin{itemize}
  \item All of the following apply (\textsc{singular}):
  \begin{itemize}
  \item $\tty$ is a builtin singular type;
  \item $\vb$ is $\False$.
  \end{itemize}
  \item All of the following apply (\textsc{named}):
  \begin{itemize}
    \item $\tty$ is a named type;
    \item $\vb$ is $\True$.
  \end{itemize}
  \item All of the following apply (\textsc{tuple}):
  \begin{itemize}
    \item $\tty$ is a tuple type $\vli$;
    \item $\vb$ is $\True$ if and only if there exists a non-primitive type in $\vli$.
  \end{itemize}
  \item All of the following apply (\textsc{array}):
    \begin{itemize}
    \item $\tty$ is an array of type $\tty'$
    \item $\vb$ is $\True$ if and only if $\tty'$ is non-primitive.
    \end{itemize}
  \item All of the following apply (\textsc{structured}):
    \begin{itemize}
    \item $\tty$ is a record or exception with fields $\fields$;
    \item $\vb$ is $\True$ if and only if there exists a non-primitive type in $\fields$.
    \end{itemize}
\end{itemize}

\subsection{Example}
The following types are non-primitive:

\begin{tabular}{ll}
\textbf{Type definition} & \textbf{Reason for being non-primitive}\\
\hline
\texttt{type A of integer}  & Named types are non-primitive\\
\texttt{(integer, A)}       & The second component, \texttt{A}, has non-primitive type\\
\texttt{array[6] of A}      & Element type \texttt{A} has a non-primitive type\\
\verb|record { a : A }|     & The field \texttt{a} has a non-primitive type
\end{tabular}

\CodeSubsection{\NonPrimitiveBegin}{\NonPrimitiveEnd}{../types.ml}

\begin{emptyformal}
\subsection{Formally}
The cases \textsc{tuple} and \textsc{structured} below, use the notation $\vb_\vt$ to name
Boolean variables by using the types denoted by $\vt$ as a subscript.
\begin{mathpar}
  \inferrule[singular]{
    \astlabel(\tty) \in \{\TReal, \TString, \TBool, \TBits, \TEnum, \TInt\}
  }
  {
    \isnonprimitive(\tty) \typearrow \False
  }
  \and
  \inferrule[named]{\astlabel(\tty) = \TNamed}{\isnonprimitive(\tty) \typearrow \True}
  \and
  \inferrule[tuple]{
    \vt \in \vli: \isnonprimitive(\vt) \typearrow \vb_{\vt}\\
    \vb \eqdef \bigvee_{\vt \in \vli} \vb_{\vt}
  }{
    \isnonprimitive(\TTuple(\vli)) \typearrow \vb
  }
  \and
  \inferrule[array]{
    \isnonprimitive(\tty') \typearrow \vb
  }{
    \isnonprimitive(\TArray(\Ignore, \tty')) \typearrow \vb
  }
  \and
  \inferrule[structured]{
    L \in \{\TRecord, \TException\}\\
    (\Ignore,\vt) \in \fields : \isnonprimitive(\vt) \typearrow \vb_\vt\\
    \vb \eqdef \bigvee_{\vt \in \vli} \vb_{\vt}
  }{
    \isnonprimitive(L(\fields)) \typearrow \vb
  }
\end{mathpar}
\end{emptyformal}

\subsection{Comments}
This is related to \identd{GWXK}.

\section{TypingRule.PrimitiveType \label{sec:TypingRule.PrimitiveType}}
\hypertarget{def-isprimitive}{}
The predicate
\[
  \isprimitive(\overname{\ty}{\tty}) \;\aslto\; \Bool
\]
tests whether the type $\tty$ is a \emph{primitive type}.

\subsection{Prose}
A type $\tty$ is primitive if it is not non-primitive.

\subsection{Example}
The following types are primitive:

\begin{tabular}{ll}
\textbf{Type definition} & \textbf{Reason for being primitive}\\
\hline
\texttt{integer} & Integers are primitive\\
\texttt{(integer, integer)} & All tuple elements are primitive\\
\texttt{array[5] of integer} & The array element type is primitive\\
\verb|record {ticks : integer}| & The single field \texttt{ticks} has a primitive type
\end{tabular}

\CodeSubsection{\PrimitiveBegin}{\PrimitiveEnd}{../types.ml}

\begin{emptyformal}
\subsection{Formally}
\begin{mathpar}
\inferrule{
  \isnonprimitive(\tty) \typearrow \vb
}{
  \isprimitive(\tty) \typearrow \neg\vb
}
\end{mathpar}
\end{emptyformal}

\subsection{Comments}
This is related to \identd{GWXK}.

\section{TypingRule.Structure \label{sec:structure}}
\hypertarget{def-structure}{}
The function
\[
  \tstruct(\overname{\staticenvs}{\tenv} \aslsep \overname{\ty}{\tty}) \aslto \overname{\ty}{\vt} \cup \overname{\TTypeError}{\TypeErrorConfig}
\]
assigns a type to its \hypertarget{def-tstruct}{\emph{\structure}}, which is the type formed by
recursively replacing named types by their type definition in the static environment $\tenv$.
If a named type is not associated with a declared type in $\tenv$, a type error is returned.

TypingRule.Specification ensures the absence of circular type definitions,
which ensures that TypingRule.Structure terminates\footnote{In mathematical terms,
this ensures that TypingRule.Structure is a proper \emph{structural induction.}}.

\subsection{Prose}
One of the following applies:
\begin{itemize}
\item All of the following apply (\textsc{named}):
  \begin{itemize}
  \item $\tty$ is a named type $\vx$;
  \item obtaining the declared type associated with $\vx$ in the static environment $\tenv$ yields $\vtone$ \ProseOrTypeError;
  \item obtaining the structure of $\vtone$ static environment $\tenv$ yields $\vt$ \ProseOrTypeError;
  \end{itemize}
\item All of the following apply (\textsc{builtin\_singular}):
  \begin{itemize}
  \item $\tty$ is a builtin singular type;
  \item $\vt$ is $\tty$.
  \end{itemize}
\item All of the following apply (\textsc{tuple}):
  \begin{itemize}
  \item $\tty$ is a tuple type with list of types $\tys$;
  \item the types in $\tys$ are indexed as $\vt_i$, for $i=1..k$;
  \item obtaining the structure of each type $\vt_i$, for $i=1..k$, in $\tys$ in the static environment $\tenv$,
  yields $\vtp_i$ \ProseOrTypeError;
  \item $\vt$ is a tuple type with the list of types $\vtp_i$, for $i=1..k$.
  \end{itemize}
\item All of the following apply (\textsc{array}):
  \begin{itemize}
    \item $\tty$ is an array type of length $\ve$ with element type $\vt$;
    \item obtaining the structure of $\vt$ yields $\vtone$ \ProseOrTypeError;
    \item $\vt$ is is an array type with of length $\ve$ with element type $\vtone$.
  \end{itemize}
\item All of the following apply (\textsc{structured}):
  \begin{itemize}
  \item $\tty$ is either a record or an exception with fields $\fields$;
  \item obtaining the structure for each type $\vt$ associated with field $\id$ yields a type $\vt_\id$ \ProseOrTypeError;
  \item $\vt$ is a record or an exception, in correspondence to $\tty$, with the list of pairs $(\id, \vt\_\id)$;
  \end{itemize}
\end{itemize}

\subsection{Example}
In this example:
\texttt{type T1 of integer;} is the named type \texttt{T1}
whose structure is \texttt{integer}.

In this example:
\texttt{type T2 of (integer, T1);}
is the named type \texttt{T2} whose structure is (integer, integer). In this
example, \texttt{(integer, T1)} is non-primitive since it uses \texttt{T1}, which is builtin aggregate.

In this example:
\texttt{var x: T1;}
the type of $\vx$ is the named (hence non-primitive) type \texttt{T1}, whose structure
is \texttt{integer}.

In this example:
\texttt{var y: integer;}
the type of \texttt{y} is the anonymous primitive type \texttt{integer}.

In this example:
\texttt{var z: (integer, T1);}
the type of \texttt{z} is the anonymous non-primitive type
\texttt{(integer, T1)} whose structure is \texttt{(integer, integer)}.

\CodeSubsection{\StructureBegin}{\StructureEnd}{../types.ml}

\begin{emptyformal}
\subsection{Formally}
\begin{mathpar}
\inferrule[named]{
  \declaredtype(\tenv, \vx) \typearrow \vtone \OrTypeError\\\\
  \tstruct(\tenv, \vtone)\typearrow\vt \OrTypeError
}{
  \tstruct(\tenv, \TNamed(\vx)) \typearrow \vt
}
\and
\inferrule[builtin\_singular]{
  \isbuiltinsingular(\tty) \typearrow \True
}{
  \tstruct(\tenv, \tty) \typearrow \tty
}
\and
\inferrule[tuple]{
  \tys \eqname \vt_{1..k}\\
  i=1..k: \tstruct(\tenv, \vt_i) \typearrow \vtp_i \OrTypeError
}{
  \tstruct(\tenv, \TTuple(\tys)) \typearrow  \TTuple(i=1..k: \vtp_i)
}
\and
\inferrule[array]{
  \tstruct(\tenv, \vt) \typearrow \vtone \OrTypeError
}{
  \tstruct(\tenv, \TArray(\ve, \vt)) \typearrow \TArray(\ve, \vtone)
}
\and
\inferrule[structured]{
  L \in \{\TRecord, \TException\}\\\\
  (\id,\vt) \in \fields : \tstruct(\tenv, \vt) \typearrow \vt_\id \OrTypeError
}{
  \tstruct(\tenv, L(\fields)) \typearrow
 L([ (\id,\vt) \in \fields : (\id,\vt_\id) ])
}
\end{mathpar}
\end{emptyformal}

\subsection{Comments}
This is related to \identd{FXQV}.

\section{TypingRule.Anonymize \label{sec:anonymize}}
\hypertarget{def-makeanonymous}{}
\hypertarget{def-underlyingtype}{}
The function
\[
  \makeanonymous(\overname{\staticenvs}{\tenv} \aslsep \overname{\ty}{\tty}) \aslto \overname{\ty}{\vt} \cup \overname{\TTypeError}{\TypeErrorConfig}
\]
returns the \emph{\underlyingtype} --- $\vt$ --- of the type $\tty$ in the static environment $\tenv$ or a type error.
Intuitively, $\tty$ is the first non-named type that is used to define $\tty$. Unlike $\tstruct$,
$\makeanonymous$ replaces named types by their definition until the first non-named type is found but
does not recurse further.

\subsection{Prose}
One of the following applies:
\begin{itemize}
  \item All of the following apply (\textsc{named}):
  \begin{itemize}
    \item $\tty$ is a named type $\vx$;
    \item obtaining the type declared for $\vx$ yields $\vtone$ \ProseOrTypeError;
    \item the \underlyingtype\ of $\vtone$ is $\vt$.
  \end{itemize}

  \item All of the following apply (\textsc{non-named}):
  \begin{itemize}
    \item $\tty$ is not a named type $\vx$;
    \item $\vt$ is $\tty$.
  \end{itemize}
\end{itemize}

\subsection{Example}
Consider the following example:
\begin{verbatim}
type T1 of integer;
type T2 of T1;
type T3 of (integer, T2);
\end{verbatim}

The underlying types of \texttt{integer}, \texttt{T1}, and \texttt{T2} is \texttt{integer}.

The underlying type of \texttt{(integer, T2)} and \texttt{T3} is
\texttt{(integer, T2)}.  Notice how the underlying type does not replace
\texttt{T2} with its own underlying type, in contrast to the structure of
\texttt{T2}, which is \texttt{(integer, integer)}.

\CodeSubsection{\AnonymizeBegin}{\AnonymizeEnd}{../types.ml}

\begin{emptyformal}
\subsection{Formally}
\begin{mathpar}
\inferrule[named]{
  \tty \eqname \TNamed(\vx) \\
  \declaredtype(\tenv, \vx) \typearrow \vtone \OrTypeError \\\\
  \makeanonymous(\tenv, \vtone) \typearrow \vt
}{
  \makeanonymous(\tenv, \tty) \typearrow \vt
}
\and
\inferrule[non-named]{
  \astlabel(\tty) \neq \TNamed
}{
  \makeanonymous(\tenv, \tty) \typearrow \tty
}
\end{mathpar}
\end{emptyformal}
\subsection{Comments}

\section{TypingRule.CheckConstrainedInteger \label{sec:TypingRule.CheckConstrainedInteger}}
\hypertarget{def-checkconstrainedinteger}{}
The function
\[
  \checkconstrainedinteger(\overname{\staticenvs}{\tenv} \aslsep \overname{\ty}{\tty}) \aslto \True \cup \overname{\TTypeError}{\TypeErrorConfig}
\]
checks whether the type $\vt$ is a \constrainedinteger. If so, the result is $\True$, otherwise a type error is returned.

\subsection{Prose}
One of the following applies:
\begin{itemize}
  \item All of the following apply (\textsc{well-constrained}):
  \begin{itemize}
    \item $\vt$ is a well-constrained integer;
    \item the result is $\True$.
  \end{itemize}

  \item All of the following apply (\textsc{underconstrained}):
  \begin{itemize}
    \item $\vt$ is an underconstrained integer;
    \item the result is $\True$.
  \end{itemize}

  \item All of the following apply (\textsc{unconstrained}):
  \begin{itemize}
    \item $\vt$ is an unconstrained integer;
    \item the result is a type error indicating that a constrained integer type is expected.
  \end{itemize}

  \item All of the following apply (\textsc{conflicting\_type}):
  \begin{itemize}
    \item $\vt$ is not an integer type;
    \item the result is a type error indicating the type conflict.
  \end{itemize}
\end{itemize}

\subsection{Example}

\CodeSubsection{\CheckConstrainedIntegerBegin}{\CheckConstrainedIntegerEnd}{../Typing.ml}

\begin{emptyformal}
\subsection{Formally}
\begin{mathpar}
\inferrule[well-constrained]{}
{
  \checkconstrainedinteger(\tenv, \TInt(\wellconstrained(\Ignore))) \typearrow \True
}
\and
\inferrule[underconstrained]{}
{
  \checkconstrainedinteger(\tenv, \TInt(\underconstrained(\Ignore))) \typearrow \True
}
\and
\inferrule[unconstrained]{}
{
  \checkconstrainedinteger(\tenv, \TInt(\unconstrained(\Ignore))) \typearrow \\
  \TypeErrorVal{ConstrainedIntegerExpected}
}
\and
\inferrule[conflicting\_type]{
  \astlabel(\vt) \neq \TInt
}
{
  \checkconstrainedinteger(\tenv, \vt) \typearrow \TypeErrorVal{TypeConflict}
}
\end{mathpar}
\end{emptyformal}
\subsection{Comments}

%%%%%%%%%%%%%%%%%%%%%%%%%%%%%%%%%%%%%%%%%%%%%%%%%%%%%%%%%%%%%%%%%%%%%%%%%%%%%%%%%%%%
\chapter{Relations Over Types \label{chap:relationsovertypes}}
%%%%%%%%%%%%%%%%%%%%%%%%%%%%%%%%%%%%%%%%%%%%%%%%%%%%%%%%%%%%%%%%%%%%%%%%%%%%%%%%%%%%

We define the following relations over types and operators:
\begin{itemize}
  \item Subtype (\secref{TypingRule.Subtype})
  \item Structural Subtype Satisfaction (\secref{TypingRule.StructuralSubtypeSatisfaction})
  \item Domain Subtype Satisfaction (\secref{TypingRule.DomainSubtypeSatisfaction})
  \item Subtype Satisfaction (\secref{TypingRule.SubtypeSatisfaction})
  \item Type Satisfaction (\secref{TypingRule.TypeSatisfaction})
  \item Type Clash (\secref{TypingRule.TypeClash})
  \item Lowest Common Ancestor (\secref{TypingRule.LowestCommonAncestor})
  \item Checking adequacy of a unary operator for a type (\secref{TypingRule.CheckUnop})
  \item Checking adequacy of a binary operator for a pair of types (\secref{TypingRule.CheckBinop})
\end{itemize}

We also define the helper rule Typing.FindNamedLCA (\secref{Typing.FindNamedLCA}).

\section{TypingRule.Subtype\label{sec:TypingRule.Subtype}}
The \emph{subtype} relation is a partial order over \underline{named types}.
The \emph{supertype} is the symmetric relation. That is, \tty\ is a supertype of \tsy\ if and only if \tsy\ is a subtype of \tty.

\hypertarget{def-subtypesrel}{}
The predicate
\[
  \subtypesrel(\overname{\staticenvs}{\tenv} \aslsep \overname{\ty}{\vtone} \aslsep \overname{\ty}{\vttwo})
  \aslto \overname{\Bool}{\vb}
\]

\subsection{Prose}
One of the following applies:
\begin{itemize}
  \item all of the following apply (\textsc{reflexive}):
  \begin{itemize}
    \item $\vtone$ and $\vttwo$ are both the same named type;
    \item $\vb$ is $\True$.
  \end{itemize}

  \item all of the following apply (\textsc{transitive}):
  \begin{itemize}
    \item $\vtone$ is a named type with name $\idone$, that is $\TNamed(\idone)$;
    \item $\vttwo$ is a named type with name $\idtwo$, that is $\TNamed(\idtwo)$, such that $\idone$ is different from $\idtwo$;
    \item the global static environment maintains that $\idone$ is a subtype of $\idthree$;
    \item testing whether the type named $\idthree$ is a subtype of $\vttwo$ in the static environment $\tenv$
    gives $\vb$.
  \end{itemize}

  \item all of the following apply (\textsc{no\_supertype}):
  \begin{itemize}
    \item $\vtone$ is a named type with name $\idone$, that is $\TNamed(\idone)$;
    \item $\vttwo$ is a named type with name $\idtwo$, that is $\TNamed(\idtwo)$, such that $\idone$ is different from $\idtwo$;
    \item the global static environment maintains that $\idone$ does subtype any named type;
    \item $\vb$ is $\False$.
  \end{itemize}

  \item all of the following apply (\textsc{not\_named}):
  \begin{itemize}
    \item at least one of $\vtone$ and $\vttwo$ is not a named type;
    \item $\vb$ is $\False$.
  \end{itemize}
\end{itemize}
\subsection{Example}
In the following example \texttt{subInt} is a subtype of itself and of \texttt{superInt}:
\begin{verbatim}
type superInt of integer;
type subInt of integer subtypes superInt;
\end{verbatim}

\CodeSubsection{\SubtypeBegin}{\SubtypeEnd}{../types.ml}

\begin{emptyformal}
\subsection{Formally}
\begin{mathpar}
  \inferrule[reflexive]{}{
    \subtypesrel(\tenv, \TNamed(\id), \TNamed(\id)) \typearrow \True
  }
  \and
  \inferrule[transitive]{
    \idone \neq \idtwo\\
    G^\tenv.\subtypes(\idone) = \idthree\\
    \subtypesrel(\tenv, \TNamed(\idthree), \vttwo) \typearrow \vb
  }{
    \subtypesrel(\tenv, \TNamed(\idone), \TNamed(\idtwo)) \typearrow \vb
  }
  \and
  \inferrule[no\_supertype]{
    \idone \neq \idtwo\\
    G^\tenv.\subtypes(\idone) = \bot
  }{
    \subtypesrel(\tenv, \TNamed(\idone), \TNamed(\idtwo)) \typearrow \False
  }
  \and
  \inferrule[not\_named]{
    (\astlabel(\vtone) \neq \TNamed \lor \astlabel(\vttwo) \neq \TNamed)
  }{
    \subtypesrel(\tenv, \vtone, \vttwo) \typearrow \False
  }
\end{mathpar}
\end{emptyformal}

\subsection{Comments}
% Since the subtype relation is a partial order, it is reflexive, viz,
% every type is also a subtype of itself.

% Since the subtype relation is a partial order, it is transitive, viz, if A is
% a subtype of B and B is a subtype of C then A is a subtype of C.

% As a consequence, there is no need to declare the reflexive and transitive
% subtype relations explicitly. All other subtype relations must be explicitly
% declared.

% Since the subtype relation is a partial order, it is antisymmetric. Therefore
% it is an error if all of the following apply:
% \begin{itemize}
% \item \texttt{id1} is a subtype of \texttt{id2};
% \item \texttt{id2} is a subtype of \texttt{id1}.
% \end{itemize}
This is related to \identr{NXRX}, \identi{KGKS}, \identi{MTML}, \identi{JVRM}, \identi{CHMP}.

\section{TypingRule.StructuralSubtypeSatisfaction\label{sec:TypingRule.StructuralSubtypeSatisfaction}}
\hypertarget{def-structsubtypesat}{}
The predicate
\[
  \structsubtypesat(\overname{\staticenvs}{\tenv} \aslsep \overname{\ty}{\vt} \aslsep \overname{\ty}{\vs})
  \aslto \overname{\Bool}{\vb} \cup \overname{\TTypeError}{\TypeErrorConfig}
\]
tests whether a type $\vt$ \emph{structurally-subtype-satisfies} a type $\vs$ in environment $\tenv$,
returning the result $\vb$ or a type error, if one is detected.
The function assumes that both $\vt$ and $\vs$ are well-typed according to \chapref{typingoftypes}.

\subsection{Prose}
One of the following applies:
\begin{itemize}
\item All of the following apply (\textsc{error1}):
  \begin{itemize}
  \item obtaining the \underlyingtype\ of $\vt$ gives a type error;
  \item the rule results in a type error.
  \end{itemize}

\item All of the following apply (\textsc{error2}):
  \begin{itemize}
    \item obtaining the \underlyingtype\ of $\vt$ gives a type $\vttwo$;
    \item obtaining the \underlyingtype\ of $\vs$ gives a type error;
    \item the rule results in a type error.
    \end{itemize}

\item All of the following apply (\textsc{different\_labels}):
  \begin{itemize}
  \item the underlying types of $\vt$ and $\vs$ have different AST labels
  (for example, \texttt{integer} and \texttt{real});
  \item $\vb$ is $\False$.
  \end{itemize}

\item All of the following apply (\textsc{simple}):
  \begin{itemize}
  \item the \underlyingtype\ of $\vt$, $\vttwo$, is either \texttt{integer} (any kind), \texttt{real}, \texttt{string}, or \texttt{bool};
  \item the \underlyingtype\ of $\vs$, $\vstwo$, is either \texttt{integer} (any kind), \texttt{real}, \texttt{string}, or \texttt{bool};
  \item $\vb$ is $\True$ if and only if both $\vttwo$ and $\vstwo$ have the same ASL label.
  \end{itemize}

\item All of the following apply (\textsc{t\_enum}):
  \begin{itemize}
  \item the \underlyingtype\ of $\vt$, $\vttwo$, is an enumeration type;
  \item the \underlyingtype\ of $\vs$ is $\vstwo$;
  \item $\vb$ is $\True$ if and only if $\vttwo$ is equal to $\vstwo$.
  \end{itemize}

\item All of the following apply (\textsc{t\_bits}):
  \begin{itemize}
  \item the \underlyingtype\ of $\vs$ is a bitvector type with width $\ws$ and bit fields $\bfss$, that is $\TBits(\ws, \bfss)$;
  \item the \underlyingtype\ of $\vt$ is a bitvector type with width $\wt$ and bit fields $\bfst$, that is $\TBits(\wt, \bfst)$;
  \item $\vbone$ is $\True$ if and only if bitwidth $\ws$ is determined to be equivalent to $\wt$ in $\tenv$;
  \item determining whether the bit fields $\bfss$ are included in the bit fields $\bfst$ in $\tenv$ yields $\vbtwo$ \ProseOrTypeError;
  \item $\vb$ is $\True$ if and only if both $\vbone$ and $\vbtwo$ are $\True$.
  \end{itemize}

\item All of the following apply (\textsc{t\_array\_expr}):
  \begin{itemize}
  \item $\vs$ has the \underlyingtype\ of an array with index $\vlengths$ and element type $\vtys$, that is $\TArray(\vlengths, \vtys)$;
  \item $\vt$ has the \underlyingtype\ of an array with index $\vlengtht$ and element type $\vtyt$, that is $\TArray(\vlengtht, \vtyt)$;
  \item determining whether $\vtys$ and $\vtyt$ are equivalent in $\tenv$ is either $\True$
  or $\False$, which short-circuits the entire rule with $\vb=\False$;
  \item either the AST labels of $\vlengths$ and $\vlengtht$ are the same or the rule short-circuits with $\vb=\False$;
  \item $\vlengths$ is an array length expression with $\vlengthexprs$, that is \\ $\ArrayLengthExpr(\vlengthexprs)$;
  \item $\vlengtht$ is an array length expression with $\vlengthexprt$, that is \\ $\ArrayLengthExpr(\vlengthexprt)$;
  \item determining whether expressions $\vlengthexprs$ and $\vlengthexprt$ are equivalent gives $\vb$.
  \end{itemize}

  \item All of the following apply (\textsc{t\_array\_enum}):
  \begin{itemize}
  \item $\vs$ has the \underlyingtype\ of an array with index $\vlengths$ and element type $\vtys$, that is $\TArray(\vlengths, \vtys)$;
  \item $\vt$ has the \underlyingtype\ of an array with index $\vlengtht$ and element type $\vtyt$, that is $\TArray(\vlengtht, \vtyt)$;
  \item determining whether $\vtys$ and $\vtyt$ are equivalent in $\tenv$ is either $\True$
  or $\False$, which short-circuits the entire rule with $\vb=\False$;
  \item either the AST labels of $\vlengths$ and $\vlengtht$ are the same or the rule short-circuits with $\vb=\False$;
  \item $\vlengths$ is an array with indices taken from the enumeration $\vnames$, that is $\ArrayLengthEnum(\vnames, \Ignore)$;
  \item $\vlengtht$ is an array with indices taken from the enumeration $\vnamet$, that is $\ArrayLengthEnum(\vnamet, \Ignore)$;
  \item $\vb$ is $\True$ if and only if $\vnames$ and $\vnamet$ are the same.
  \end{itemize}

\item All of the following apply (\textsc{t\_tuple}):
  \begin{itemize}
  \item $\vs$ has the \underlyingtype\ of a tuple with type list $\vlis$, that is $\TTuple(\vlis)$;
  \item $\vt$ has the \underlyingtype\ of a tuple with type list $\vlit$, that is $\TTuple(\vlit)$;
  \item equating the lengths of $\vlis$ and $\vlit$ is either $\True$ or $\False$, which short-circuits
  the entire rule with $\vb=\False$;
  \item checking at each index $\vi$ of the list $\vlis$ whether the type $\vlit[\vi]$ \typesatisfies\ the type $\vlis[\vi]$
  yields $\vb_\vi$ \ProseOrTypeError;
  \item $\vb$ is $\True$ if and only if all $\vb_\vi$ are $\True$;
  \end{itemize}

\item All of the following apply (\textsc{structured}):
  \begin{itemize}
  \item $\vs$ has the \underlyingtype\ $L(\vfieldss)$, which is either a record type or an exception type;
  \item $\vt$ has the \underlyingtype\ $L(\vfieldst)$, which is either a record type or an exception type;
  \item since both underlying types have the same AST label they are either both record types or both exception types;
  \item $\vb$ is $\True$ if and only if for each field in $\vfieldss$ with type $\vtys$
  there exists a field in $\vfieldst$ with type $\vtyt$ such that both $\vtys$ and $\vtyt$
  are determined to be \typeequivalent\ in $\tenv$.
  \end{itemize}
\end{itemize}

\subsection{Example}

\CodeSubsection{\StructuralSubtypeSatisfactionBegin}{\StructuralSubtypeSatisfactionEnd}{../types.ml}

\begin{emptyformal}
\subsection{Formally}
\begin{mathpar}
  \inferrule[error1]{
    \makeanonymous(\tenv, \vt) \typearrow \TypeErrorConfig
  }
  {
    \structsubtypesat(\tenv, \vt, \vs) \typearrow \TypeErrorConfig
  }
  \and
  \inferrule[error2]{
    \makeanonymous(\tenv, \vt) \typearrow \vttwo\\
    \makeanonymous(\tenv, \vs) \typearrow \TypeErrorConfig
  }
  {
    \structsubtypesat(\tenv, \vt, \vs) \typearrow \TypeErrorConfig
  }
  \and
  \inferrule[different\_labels]{
    \makeanonymous(\tenv, \vt) \typearrow \vttwo\\
    \makeanonymous(\tenv, \vs) \typearrow \vstwo\\
    \astlabel(\vttwo) \neq \astlabel(\vstwo)
  }
  {
    \structsubtypesat(\tenv, \vt, \vs) \typearrow \False
  }
\end{mathpar}

\begin{mathpar}
  \inferrule[simple]{
    \makeanonymous(\tenv, \vt) \typearrow \vttwo\\
    \makeanonymous(\tenv, \vs) \typearrow \vstwo\\
    \astlabel(\vttwo) \in \{\TInt, \TReal, \TString, \TBool\}\\
    \vb \eqdef \astlabel(\vstwo) = \astlabel(\vttwo)
  }{
    \structsubtypesat(\tenv, \vt, \vs) \typearrow \vb
  }
\end{mathpar}

\begin{mathpar}
  \inferrule[t\_enum]{
    \makeanonymous(\tenv, \vt) \typearrow \TEnum(\Ignore)\\
    \makeanonymous(\tenv, \vs) \typearrow \vstwo\\
    \vb \eqdef \vstwo = \vttwo
  }{
    \structsubtypesat(\tenv, \vt, \vs) \typearrow \vb
  }
\end{mathpar}

\begin{mathpar}
\inferrule[t\_bits]{
  \makeanonymous(\tenv, \vs) \typearrow \TBits(\ws, \bfss)\\
  \makeanonymous(\tenv, \vt) \typearrow \TBits(\wt, \bfst)\\
  \bitwidthequal(\tenv, \ws, \wt) \typearrow \vbone\\
  \bitfieldsincluded(\tenv, \bfss, \bfst) \typearrow \vbtwo \OrTypeError\\\\
  \vb \eqdef \vbone \land \vbtwo
}{
  \structsubtypesat(\tenv, \vt, \vs) \typearrow \vb
}
\end{mathpar}

\begin{mathpar}
\inferrule[t\_array\_expr]{
  \makeanonymous(\tenv, \vs) \typearrow \TArray(\vlengths,\vtys) \\
  \makeanonymous(\tenv, \vt) \typearrow \TArray(\vlengtht,\vtyt) \\
  \typeequal(\tenv, \vtys, \vtyt) \typearrow \True \terminateas \False\\
  \booltrans{\astlabel(\vlengths) = \astlabel(\vlengtht)} \booltransarrow \True \terminateas \False\\
  \vlengths \eqname \ArrayLengthExpr(\vlengthexprs)\\
  \vlengtht \eqname \ArrayLengthExpr(\vlengthexprt)\\
  \exprequal(\tenv, \vlengthexprs, \vlengthexprt) \typearrow \vb
}
{
  \structsubtypesat(\tenv, \vt, \vs) \typearrow \vb
}
\and
\inferrule[t\_array\_enum]{
  \makeanonymous(\tenv, \vs) \typearrow \TArray(\vlengths,\vtys) \\
  \makeanonymous(\tenv, \vt) \typearrow \TArray(\vlengtht,\vtyt) \\
  \typeequal(\tenv, \vtys, \vtyt) \typearrow \True\\
  \booltrans{\astlabel(\vlengths) = \astlabel(\vlengtht)} \typearrow \True\\
  \vlengths \eqname \ArrayLengthEnum(\vnames, \Ignore)\\
  \vlengtht \eqname \ArrayLengthEnum(\vnamet, \Ignore)\\
  \vb \eqdef \vnames = \vnamet
}
{
  \structsubtypesat(\tenv, \vt, \vs) \typearrow \vb
}
\end{mathpar}

\begin{mathpar}
\inferrule[t\_tuple]
{ \makeanonymous(\tenv, \vs) \typearrow\TTuple(\vlis)\\
  \makeanonymous(\tenv, \vt) \typearrow\TTuple(\vlit)\\
  \equallength(\vlis, \vlit) \typearrow\True \terminateas \False\\
  \vi\in\listrange(\vlis): \typesat(\tenv, \vlit[\vi], \vlis[\vi]) \typearrow \vb_i \terminateas \TTypeError\\
  \vb \eqdef \bigwedge_{\vi=1}^k \vb_\vi
}{
  \structsubtypesat(\tenv, \vt, \vs) \typearrow \vb
}
\end{mathpar}

\hypertarget{def-fieldnames}{}
For a list of typed fields $\fields$, we define the set of its field identifiers as:
\[
  \fieldnames(\fields) \triangleq \{ \id \;|\; (\id, \vt) \in \fields\}
\]
\hypertarget{def-fieldtype}{}
We define the type associated with the field name $\id$ in a list of typed fields $\fields$,
if there is a unique one, as follows:
\[
  \fieldtype(\fields, \id) \triangleq
  \begin{cases}
  \vt  & \text{ if }\{ \vtp \;|\; (\id,\vtp) \in \fields\} = \{\vt\}\\
  \bot & \text{ otherwise}
  \end{cases}
\]

\begin{mathpar}
\inferrule[structured]{
  L \in \{\TRecord, \TException\}\\
  \makeanonymous(\tenv, \vs)\typearrow L(\vfieldss) \\
  \makeanonymous(\tenv, \vt)\typearrow L(\vfieldst) \\
  \vnamess \eqdef \fieldnames(\vfieldss)\\
  \vnamest \eqdef \fieldnames(\vfieldst)\\
  \booltrans{\vnamess \subseteq \vnamest} \booltransarrow \True \terminateas \False\\
  (\id,\vtys)\in\vfieldss: \typeequal(\tenv, \vtys, \fieldtype(\vfieldst, \id)) \typearrow \vb_\id\\
  \vb \eqdef \bigwedge_{\id \in \vnamess} \vb_\id
}{
  \structsubtypesat(\tenv, \vs, \vt) \typearrow \vb
}
\end{mathpar}
\end{emptyformal}

\subsection{Comments}
  This is related to \identd{TRVR}, \identi{SJDC}, \identi{MHYB}, \identi{TWTZ}, \identi{GYSK}, \identi{KXSD}.

\section{TypingRule.DomainSubtypeSatisfaction\label{sec:TypingRule.DomainSubtypeSatisfaction}}
\hypertarget{def-domsubtypesat}{}
The predicate
\[
  \domsubtypesat(\overname{\staticenvs}{\tenv} \aslsep \overname{\ty}{\vt} \aslsep \overname{\ty}{\vs})
  \aslto \overname{\Bool}{\vb} \cup \overname{\TTypeError}{\TypeErrorConfig}
\]
tests whether a type $\vt$ \emph{domain-subtype-satisfies} a type $\vs$ in environment $\tenv$,\\
\underline{assuming that $\vt$ structurally-subtype-satisfies $\vs$},
returning the result $\vb$ or a type error, if one is detected.

\subsection{Prose}
One of the following applies:
\begin{itemize}
  \item All of the following apply (\textsc{error1}):
  \begin{itemize}
    \item obtaining the \structure\ of $\vs$ results in a type error;
    \item the rule gives a type error.
  \end{itemize}

  \item All of the following apply (\textsc{error2}):
  \begin{itemize}
    \item obtaining the \structure\ of $\vs$ results in a type $\vsstruct$;
    \item obtaining the \structure\ of $\vt$ results in a type error;
    \item the rule gives a type error.
  \end{itemize}

  \item All of the following apply (\textsc{simple}):
  \begin{itemize}
    \item obtaining the \structure\ of $\vs$ results in a type $\vsstruct$;
    \item the AST label of $\vsstruct$ is either $\TTuple$, $\TArray$, $\TRecord$, or $\TException$;
    \item $\vb$ is $\True$.
  \end{itemize}

  \item All of the following apply (\textsc{symbolic}):
  \begin{itemize}
    \item obtaining the \structure\ of $\vs$ results in a type $\vsstruct$;
    \item obtaining the \structure\ of $\vt$ results in a type $\vtstruct$;
    \item the AST label of $\vsstruct$ is either $\TReal$, $\TString$, $\TBool$, $\TEnum$, or $\TInt$;
    \item determining whether $\vs$ subsumes $\vt$ in $\tenv$ via symbolic reasoning results in $\vb$.
  \end{itemize}
\end{itemize}

\subsection{Example}

\CodeSubsection{\DomainSubtypeSatisfactionBegin}{\DomainSubtypeSatisfactionEnd}{../types.ml}

\begin{emptyformal}
\subsection{Formally}
\begin{mathpar}
\inferrule[error1]{
  \tstruct(\tenv, \vs) \typearrow \TypeErrorConfig\\
}{
  \domsubtypesat(\tenv, \vt, \vs) \typearrow \TypeErrorConfig
}
\and
\inferrule[error2]{
  \tstruct(\tenv, \vs) \typearrow \vsstruct\\
  \tstruct(\tenv, \vt) \typearrow \TypeErrorConfig\\
}{
  \domsubtypesat(\tenv, \vt, \vs) \typearrow \TypeErrorConfig
}
\and
\inferrule[simple]{
  \tstruct(\tenv, \vs) \typearrow \vsstruct\\
  \astlabel(\vsstruct) \in \{\TTuple, \TArray, \TRecord, \TException\}
}{
  \domsubtypesat(\tenv, \vt, \vs) \typearrow \True
}
\and
\inferrule[symbolic]{
  \tstruct(\tenv, \vs) \typearrow \vsstruct\\
  \tstruct(\tenv, \vt) \typearrow \vtstruct\\
  \astlabel(\vsstruct) \in \{\TReal, \TString, \TBool, \TEnum, \TInt\}\\
  \symsubsumes(\tenv, \vsstruct, \vtstruct) \typearrow \vb
}{
  \domsubtypesat(\tenv, \vt, \vs) \typearrow \vb
}
\end{mathpar}
\end{emptyformal}

\subsection{Comments}
    This is related to \identd{TRVR}.

\section{TypingRule.SubtypeSatisfaction\label{sec:TypingRule.SubtypeSatisfaction}}
\hypertarget{def-subtypesat}{}
The predicate
\[
  \subtypesat(\overname{\staticenvs}{\tenv} \aslsep \overname{\ty}{\vt} \aslsep \overname{\ty}{\vs})
  \aslto \overname{\Bool}{\vb} \cup \overname{\TTypeError}{\TypeErrorConfig}
\]
tests whether a type $\vt$ \emph{subtype-satisfies} a type $\vs$ in environment $\tenv$,
returning the result $\vb$ or a type error, if one is detected.

\subsection{Prose}
All of the following apply:
\begin{itemize}
  \item determining whether $\vt$ structurally-subtype-satisfies $\vs$ yields $\vbone$ \ProseOrTypeError;
  \item determining whether $\vt$ domain-subtype-satisfies $\vs$ yields $\vbtwo$;
  \item $\vb$ is $\True$ if and only if both $\vbone$ and $\vbtwo$ are $\True$.
\end{itemize}

\subsection{Example}

\CodeSubsection{\SubtypeSatisfactionBegin}{\SubtypeSatisfactionEnd}{../types.ml}

\begin{emptyformal}
\begin{mathpar}
\inferrule{
  \structsubtypesat(\tenv, \vt, \vs) \typearrow \vbone \OrTypeError\\
  \domsubtypesat(\tenv, \vt, \vs) \typearrow \vbtwo\\
  \vb \eqdef \vbone \land \vbtwo
}{
  \subtypesat(\tenv, \vt, \vs) \typearrow \vb
}
\end{mathpar}

\end{emptyformal}

\subsection{Comments}
    This is related to \identd{TRVR}, \identi{KNXJ}.

\section{TypingRule.TypeSatisfaction \label{sec:TypingRule.TypeSatisfaction}}
\hypertarget{def-typesatisfies}{}
The predicate
\[
  \typesat(\overname{\staticenvs}{\tenv} \aslsep \overname{\ty}{\vt} \aslsep \overname{\ty}{\vs})
  \aslto \overname{\Bool}{\vb} \cup \overname{\TTypeError}{\TypeErrorConfig}
\]
tests whether a type $\vt$ \emph{\typesatisfies} a type $\vs$ in environment $\tenv$,
returning the result $\vb$ or a type error, if one is detected.

\hypertarget{def-checktypesat}{}
We also define
\[
  \checktypesat(\overname{\staticenvs}{\tenv} \aslsep \overname{\ty}{\vt} \aslsep \overname{\ty}{\vs})
  \aslto \True \cup \overname{\TTypeError}{\TypeErrorConfig}
\]
which is the same as $\typesat$, but yields a type error when \\ $\typesat(\tenv, \vt, \vs)$ is $\False$.

\subsection{Prose}
One of the following applies:
 \begin{itemize}
  \item All of the following apply (\textsc{subtypes}):
    \begin{itemize}
    \item $\vt$ subtypes $\vs$ in $\tenv$ ;
    \item $\vb$ is $\True$.
  \end{itemize}

  \item All of the following apply (\textsc{anonymous1}, \textsc{anonymous2}):
  \begin{itemize}
    \item $\vt$ does not subtype $\vs$ in $\tenv$;
    \item at least one of $\vt$ and $\vs$ is an anonymous type in $\tenv$;
    \item determining whether $\vt$ \subtypesatisfies\ $\vs$ in $\tenv$ yields $\True$ \ProseOrTypeError;
    \item $\vb$ is $\True$.
  \end{itemize}

  \item All of the following apply (\textsc{t\_bits}):
  \begin{itemize}
    \item $\vt$ does not subtype $\vs$ in $\tenv$;
    \item determining whether $\vt$ is anonymous yields $\vbone$;
    \item determining whether $\vs$ is anonymous yields $\vbtwo$;
    \item determining whether $\vt$ \subtypesatisfies\ $\vs$ in $\tenv$ yields $\vbthree$;
    \item $(\vbone \lor \vbtwo) \land \vbthree$ is $\False$;
    \item determining whether $\vt$ is a bitvector type yields $\True$ \ProseOrTypeError;
    \item $\vt$ is a bitvector type with width $\widtht$;
    \item obtaining the \structure\ of $\vs$ in $\tenv$ yields a bitvector type with width $\widths$ \ProseOrTypeError;
    \item determining whether $\widtht$ and $\widths$ are \bitwidthequivalent\ yields $\vb$.
  \end{itemize}

  \item All of the following apply (\textsc{otherwise}):
  \begin{itemize}
    \item $\vt$ does not subtype $\vs$ in $\tenv$;
    \item neither one of $\vt$ and $\vs$ is an anonymous type in $\tenv$;
    \item determining whether $\vt$ \subtypesatisfies\ $\vs$ in $\tenv$ yields $\False$;
    \item obtaining the \structure\ of $\vt$ in $\tenv$ yields a type $\vtone$;
    \item obtaining the \structure\ of $\vs$ in $\tenv$ yields a type $\vsone$;
    \item at least one of $\vtone$ and $\vsone$ is not a bitvector type;
    \item obtaining the \structure\ of $\vs$ in $\tenv$ yields a bitvector type of length $\widths$ \ProseOrTypeError;
    \item $\vb$ is $\False$.
  \end{itemize}
\end{itemize}

\subsection{Example}
In the specification:
\VerbatimInput{../tests/ASLTypingReference.t/TypingRule.TypeSatisfaction1.asl}
\texttt{var pair: pairT = (1, dataT1)} is legal since the right-hand-side has
anonymous, non-primitive type \texttt{(integer, T1)}.

\subsection{Example}
In the specification:
\VerbatimInput{../tests/ASLTypingReference.t/TypingRule.TypeSatisfaction2.asl}
\texttt{pair = (1, dataAsInt);} is legal since the right-hand-side has anonymous,
primitive type \texttt{(integer, integer)}.

\subsection{Example}
In the specification:
\VerbatimInput{../tests/ASLTypingReference.t/TypingRule.TypeSatisfaction3.asl}
\texttt{pair = (1, dataT2);} is illegal since the right-hand-side has anonymous,
non-primitive type \texttt{(integer, T2)} which does not subtype-satisfy named
type \texttt{pairT}.

\CodeSubsection{\TypeSatisfactionBegin}{\TypeSatisfactionEnd}{../types.ml}

\begin{emptyformal}
\subsection{Formally}
\begin{mathpar}
\inferrule[subtypes]{
  \subtypesrel(\tenv, \vt, \vs) \typearrow \True
}{
  \typesat(\tenv, \vt, \vs) \typearrow \True
}
\and
\inferrule[anonymous1]{
  \subtypesrel(\tenv, \vt, \vs) \typearrow \False\\
  \isanonymous(\tenv, \vt) \typearrow \True\\
  \subtypesat(\tenv, \vt, \vs) \typearrow \True \OrTypeError
}{
  \typesat(\tenv, \vt, \vs) \typearrow \True
}
\and
\inferrule[anonymous2]{
  \subtypesrel(\tenv, \vt, \vs) \typearrow \False\\
  \isanonymous(\tenv, \vs) \typearrow \True\\
  \subtypesat(\tenv, \vt, \vs) \typearrow \True
}{
  \typesat(\tenv, \vt, \vs) \typearrow \True
}
\and
\inferrule[t\_bits]{
  \subtypesrel(\tenv, \vt, \vs) \typearrow \False\\
  \isanonymous(\tenv, \vt) \typearrow \vbone\\
  \isanonymous(\tenv, \vs) \typearrow \vbtwo\\
  \subtypesat(\tenv, \vt, \vs) \typearrow \vbthree\\
  \neg((\vbone \lor \vbtwo) \land \vbthree)\\
  \checktrans{\astlabel(\vt) = \TBits}{ExpectedBitvectorType} \checktransarrow \True \OrTypeError\\\\
  \vt \eqname \TBits(\widtht, \Ignore)\\
  \tstruct(\tenv, \vs) \typearrow \TBits(\widths, \Ignore) \OrTypeError\\
  \bitwidthequal(\tenv, \widtht, \widths) \typearrow \vb
}{
  \typesat(\tenv, \vt, \vs) \typearrow \vb
}
\and
\inferrule[otherwise]{
  \subtypesrel(\tenv, \vt, \vs) \typearrow \False\\
  \isanonymous(\tenv, \vt) \typearrow \False\\
  \isanonymous(\tenv, \vs) \typearrow \False\\
  \subtypesat(\tenv, \vt, \vs) \typearrow \False\\
  \tstruct(\tenv, \vt) \typearrow \vtone\\
  \tstruct(\tenv, \vs) \typearrow \vsone\\
  \astlabel(\vtone) \neq \TBits \lor \astlabel(\vsone) \neq \TBits
}{
  \typesat(\tenv, \vt, \vs) \typearrow \False
}
\end{mathpar}

The rules for the checked type-satisfy predicate are:
\begin{mathpar}
\inferrule[true]{
  \typesat(\tenv, \vt, \vs) \typearrow \True \OrTypeError\\
}{
  \checktypesat(\tenv, \vt, \vs) \typearrow \True
}
\and
\inferrule[error]{
  \typesat(\tenv, \vt, \vs) \typearrow \False
}{
  \checktypesat(\tenv, \vt, \vs) \typearrow \TypeErrorVal{TypeConflict}
}
\end{mathpar}
\end{emptyformal}

\subsection{Comments}
Since the subtype relation is a partial order, it is reflexive. Therefore
every type $\vt$ is a subtype of itself, and as a consequence, every type $\vt$
\typesatisfies\  itself.

This is related to \identr{FMXK} and \identi{NLFD}.

\section{TypingRule.TypeClash\label{sec:TypingRule.TypeClash}}
\hypertarget{def-typeclashes}{}
The predicate
\[
  \typeclashes(\overname{\staticenvs}{\tenv} \aslsep \overname{\ty}{\vt} \aslsep \overname{\ty}{\vs})
  \aslto \overname{\Bool}{\vb} \cup \overname{\TTypeError}{\TypeErrorConfig}
\]
tests whether a type $\vt$ \emph{type-clashes} with a type $\vs$ in environment $\tenv$,
returning the result $\vb$ or a type error, if one is detected.

\subsection{Prose}
 One of the following applies:
\begin{itemize}
  \item All of the following apply (\textsc{subtype1}, \textsc{subtype2}):
  \begin{itemize}
    \item either $\vs$ subtypes $\vt$ or $\vt$ subtypes $\vs$;
    \item $\vb$ is $\True$.
  \end{itemize}

  \item All of the following apply (\textsc{simple}):
  \begin{itemize}
    \item neither $\vs$ subtypes $\vt$ nor $\vt$ subtypes $\vs$;
    \item obtaining the \structure\ of $\vt$ in $\tenv$ yields $\vtstruct$ \ProseOrTypeError;
    \item obtaining the \structure\ of $\vs$ in $\tenv$ yields $\vsstruct$ \ProseOrTypeError;
    \item both $\vtstruct$ and $\vsstruct$ are of the following types: \\ \texttt{integer}, \texttt{real}, \texttt{string};
    \item $\vb$ is $\True$.
  \end{itemize}

  \item All of the following apply (\textsc{t\_enum}):
  \begin{itemize}
    \item neither $\vs$ subtypes $\vt$ nor $\vt$ subtypes $\vs$;
    \item obtaining the \structure\ of $\vt$ in $\tenv$ yields an enumeration type with labels $\vlit$;
    \item obtaining the \structure\ of $\vs$ in $\tenv$ yields an enumeration type with labels $\vlis$;
    \item $\vb$ is $\True$ if and only if $\vlis$ and $\vlit$ are equal.
  \end{itemize}

  \item All of the following apply (\textsc{t\_array}):
  \begin{itemize}
    \item neither $\vs$ subtypes $\vt$ nor $\vt$ subtypes $\vs$;
    \item obtaining the \structure\ of $\vt$ in $\tenv$ yields an array type with element type $\vtyt$;
    \item obtaining the \structure\ of $\vs$ in $\tenv$ yields an array type with element type $\vtys$;
    \item $\vb$ is $\True$ if and only if $\vtyt$ and $\vtys$ type-clash.
  \end{itemize}

  \item All of the following apply (\textsc{t\_tuple}):
  \begin{itemize}
    \item neither $\vs$ subtypes $\vt$ nor $\vt$ subtypes $\vs$;
    \item obtaining the \structure\ of $\vt$ in $\tenv$ yields a tuple type with element types $\vt_{1..k}$;
    \item obtaining the \structure\ of $\vs$ in $\tenv$ yields a tuple type with element types $\vs_{1..n}$;
    \item if $n \neq k$ the rule short-circuits with $\vb=\False$;
    \item $\vb$ is $\True$ if and only if $\vt_i$ type-clashes with $\vs_i$, for all $i=1..k$.
  \end{itemize}

  \item All of the following apply (\textsc{otherwise}):
  \begin{itemize}
    \item neither $\vs$ subtypes $\vt$ nor $\vt$ subtypes $\vs$;
    \item obtaining the \structure\ of $\vt$ in $\tenv$ yields a tuple type with element types $\vt_{1..k}$;
    \item obtaining the \structure\ of $\vs$ in $\tenv$ yields a tuple type with element types $\vs_{1..n}$;
    \item either $\vsstruct$ and $\vtstruct$ have different AST labels or one of their labels
    is $\TRecord$ or $\TException$;
    \item $\vb$ is $\False$.
  \end{itemize}
\end{itemize}

\subsection{Example}

\CodeSubsection{\TypeClashBegin}{\TypeClashEnd}{../types.ml}

\begin{emptyformal}
\subsection{Formally}
\begin{mathpar}
\inferrule[subtype1]{
  \subtypesrel(\tenv, \vs, \vt) \typearrow \True
}{
  \typeclashes(\tenv, \vt, \vs) \typearrow \True
}
\and
\inferrule[subtype2]{
  \subtypesrel(\tenv, \vs, \vt) \typearrow \False\\
  \subtypesrel(\tenv, \vt, \vs) \typearrow \True
}{
  \typeclashes(\tenv, \vt, \vs) \typearrow \True
}
\end{mathpar}

\begin{mathpar}
\inferrule[simple]{
  \subtypesrel(\tenv, \vs, \vt) \typearrow \False\\
  \subtypesrel(\tenv, \vt, \vs) \typearrow \False\\
  \tstruct(\tenv, \vt) \typearrow \vtstruct \OrTypeError \\
  \tstruct(\tenv, \vs) \typearrow \vsstruct \OrTypeError \\
  \astlabel(\vtstruct)=\astlabel(\vsstruct) \in \{\TInt, \TReal, \TString, \TBits\}
}{
  \typeclashes(\tenv, \vt, \vs) \typearrow \True
}
\end{mathpar}

\begin{mathpar}
\inferrule[t\_enum]{
  \subtypesrel(\tenv, \vs, \vt) \typearrow \False\\
  \subtypesrel(\tenv, \vt, \vs) \typearrow \False\\
  \tstruct(\tenv, \vt) \typearrow \TEnum(\Ignore, \vlis) \\
  \tstruct(\tenv, \vs) \typearrow \TEnum(\Ignore, \vlit) \\
  \vb \eqdef \vlis = \vlit
}{
  \typeclashes(\tenv, \vt, \vs) \typearrow \vb
}
\end{mathpar}

\begin{mathpar}
\inferrule[t\_array]{
  \subtypesrel(\tenv, \vs, \vt) \typearrow \False\\
  \subtypesrel(\tenv, \vt, \vs) \typearrow \False\\
  \tstruct(\tenv, \vt) \typearrow \TArray(\Ignore, \vtyt) \\
  \tstruct(\tenv, \vs) \typearrow \TArray(\Ignore, \vtys) \\
  \typeclashes(\tenv, \vtyt, \vtys) \typearrow \vb
}{
  \typeclashes(\tenv, \vt, \vs) \typearrow \vb
}
\end{mathpar}

\begin{mathpar}
\inferrule[t\_tuple]{
  \subtypesrel(\tenv, \vs, \vt) \typearrow \False\\
  \subtypesrel(\tenv, \vt, \vs) \typearrow \False\\
  \tstruct(\tenv, \vt) \typearrow \TTuple(\vt_{1..k}) \\
  \tstruct(\tenv, \vs) \typearrow \TTuple(\vs_{1..n}) \\
  \booltrans{n = k} \booltransarrow \True \terminateas \False\\
  i=1..k: \typeclashes(\tenv, \vt_i, \vs_i) \typearrow \vb_i\\
  \vb \eqdef \bigwedge_{\vi=1}^k \vb_i
}{
  \typeclashes(\tenv, \vt, \vs) \typearrow \vb
}
\end{mathpar}

\begin{mathpar}
\inferrule[otherwise]{
  \subtypesrel(\tenv, \vs, \vt) \typearrow \False\\
  \subtypesrel(\tenv, \vt, \vs) \typearrow \False\\
  \tstruct(\tenv, \vt) \typearrow \vtstruct \\
  \tstruct(\tenv, \vs) \typearrow \vsstruct \\
  \booltrans{\astlabel(\vtstruct) \neq \astlabel(\vsstruct)} \booltransarrow \True \terminateas \False\\
  \vb \eqdef \astlabel(\vtstruct) \not\in \{\TRecord, \TException\}\\
}{
  \typeclashes(\tenv, \vt, \vs) \typearrow \False
}
\end{mathpar}
\end{emptyformal}

\subsection{Comments}
Note that if $\vt$ subtype-satisfies $\vs$ then $\vt$ and $\vs$ type-clash, but not the other
way around.

Note that type-clashing is an equivalence relation. Therefore if $\vt$
type-clashes with \texttt{A} and \texttt{B} then it is also the case that \texttt{A} and \texttt{B} type-clash.

This is related to \identd{VPZZ}, \identi{PQCT} and \identi{WZKM}.

\section{TypingRule.LowestCommonAncestor \label{sec:TypingRule.LowestCommonAncestor}}
\hypertarget{def-lowestcommonancestor}{}
Annotating a conditional expression (see \secref{TypingRule.ECond}),
requires finding a single type that can be used to annotate the results of both sub-expressions.
The function
\[
  \lca(\overname{\staticenvs}{\tenv} \aslsep \overname{\ty}{\vt} \aslsep \overname{\ty}{\vs})
  \aslto \overname{\ty}{\tty} \cup \overname{\TTypeError}{\TypeErrorConfig}
\]
returns the \emph{lowest common ancestor} of types $\vt$ and $\vs$ in $\tenv$ --- $\tty$.
The result is a type error if a lowest common ancestor does not exist or a type error is detected.

\subsection{Prose}
One of the following applies:
\begin{itemize}
  \item All of the following apply (\textsc{type\_equal}):
  \begin{itemize}
    \item $\vt$ is \typeequal\ to $\vs$ in $\tenv$;
    \item $\tty$ is $\vs$ (can as well be $\vt$).
  \end{itemize}

  \item All of the following apply:
  \begin{itemize}
    \item $\vt$ is not \typeequal\ to $\vs$ in $\tenv$ and one of the following applies:

    \item All of the following apply (\textsc{named\_subtype1}):
    \begin{itemize}
      \item $\vt$ is a named type with identifier $\namesubt$, that is, $\TNamed(\namesubt)$;
      \item $\vs$ is a named type with identifier $\namesubs$, that is, $\TNamed(\namesubs)$;
      \item there is no \namedlowestcommonancestor\ of $\namesubs$ and $\namesubt$ in $\tenv$;
      \item obtaining the \underlyingtype\ of $\vs$ yields $\vanons$ \ProseOrTypeError;
      \item obtaining the \underlyingtype\ of $\vt$ yields $\vanont$ \ProseOrTypeError;
      \item obtaining the lowest common ancestor of $\vanons$ and $\vanont$ in $\tenv$ yields $\tty$ \ProseOrTypeError.
    \end{itemize}

    \item All of the following apply (\textsc{named\_subtype2}):
    \begin{itemize}
      \item $\vt$ is a named type with identifier $\namesubt$, that is, $\TNamed(\namesubt)$;
      \item $\vs$ is a named type with identifier $\namesubs$, that is, $\TNamed(\namesubs)$;
      \item the \namedlowestcommonancestor\ of $\namesubs$ and $\namesubt$ in $\tenv$ is $\name$ \ProseOrTypeError;
      \item $\tty$ is the named type with identifier $\name$, that is, $\TNamed(\name)$.
    \end{itemize}

    \item All of the following apply (\textsc{one\_named1}):
    \begin{itemize}
      \item only one of $\vt$ or $\vs$ is a named type;
      \item obtaining the \underlyingtype\ of $\vs$ yields $\vanons$ \ProseOrTypeError;
      \item obtaining the \underlyingtype\ of $\vt$ yields $\vanont$ \ProseOrTypeError;
      \item $\vanont$ is \typeequal\ to $\vanons$;
      \item $\tty$ is $\vt$ if it is a named type (that is, $\astlabel(\vt)=\TNamed$), and $\vs$ otherwise.
    \end{itemize}

    \item All of the following apply (\textsc{one\_named2}):
    \begin{itemize}
      \item only one of $\vt$ or $\vs$ is a named type;
      \item obtaining the \underlyingtype\ of $\vs$ yields $\vanons$ \ProseOrTypeError;
      \item obtaining the \underlyingtype\ of $\vt$ yields $\vanont$ \ProseOrTypeError;
      \item $\vanont$ is not \typeequal\ to $\vanons$;
      \item the lowest common ancestor of $\vanont$ and $\vanons$ in $\tenv$ is $\tty$ \ProseOrTypeError.
    \end{itemize}

    \item All of the following apply (\textsc{t\_int\_unconstrained}):
    \begin{itemize}
      \item at least one of $\vt$ or $\vs$ is an unconstrained integer type;
      \item $\tty$ is the unconstrained integer type.
    \end{itemize}

    \item All of the following apply (\textsc{t\_int\_underconstrained}):
    \begin{itemize}
      \item neither of $\vt$ and $\vs$ are the unconstrained integer type;
      \item the \wellconstrainedversion\ of $\vt$ is $\vtone$;
      \item the \wellconstrainedversion\ of $\vs$ is $\vsone$;
      \item $\tty$ the lowest common ancestor of $\vtone$ and $\vsone$ in $\tenv$ is $\tty$ \ProseOrTypeError.
    \end{itemize}

    \item All of the following apply (\textsc{t\_int\_wellconstrained}):
    \begin{itemize}
      \item $\vt$ is a well-constrained integer type with constraints $\cst$;
      \item $\vs$ is a well-constrained integer type with constraints $\css$;
      \item $\tty$ is the well-constrained integer type with constraints $\cst \concat \css$.
    \end{itemize}

    \item All of the following apply (\textsc{t\_bits\_equal}):
    \begin{itemize}
      \item $\vt$ is a bitvector type with with length expression $\vet$, that is, $\TBits(\vet, \Ignore)$;
      \item $\vs$ is a bitvector type with with length expression $\ves$, that is, $\TBits(\ves, \Ignore)$;
      \item $\vet$ is equivalent to $\ves$ in $\tenv$;
      \item $\tty$ is a bitvector type with length expression $\vet$ and an empty bitfield list, that is, $\TBits(\vet, \emptylist)$.
    \end{itemize}

    \item All of the following apply (\textsc{t\_bits\_not\_equal}):
    \begin{itemize}
      \item $\vt$ is a bitvector type with length expression $\vet$;
      \item $\vs$ is a bitvector type with length expression $\ves$;
      \item $\vet$ is not equivalent to $\ves$ in $\tenv$;
      \item the result is a type error indicating the lack of a lowest common ancestor.
    \end{itemize}

    \item All of the following apply (\textsc{t\_array\_equal}):
    \begin{itemize}
      \item $\vt$ is an array type with width expression $\widtht$ and element type $\vtyt$;
      \item $\vs$ is an array type with width expression $\widths$ and element type $\vtys$;
      \item $\widtht$ is equivalent to $\widths$ in $\tenv$;
      \item the lowest common ancestor of $\vtyt$ and $\vtys$ is $\vtone$ \ProseOrTypeError;
      \item $\tty$ is an array type with width expression $\widths$ and element type $\vtone$.
    \end{itemize}

    \item All of the following apply (\textsc{t\_array\_not\_equal}):
    \begin{itemize}
      \item $\vt$ is an array type with width expression $\widtht$ and element type $\vtyt$;
      \item $\vs$ is an array type with width expression $\widths$ and element type $\vtys$;
      \item $\widtht$ is not equivalent to $\widths$ in $\tenv$;
      \item the result is a type error indicating the lack of a lowest common ancestor.
    \end{itemize}

    \item All of the following apply (\textsc{t\_tuple}):
    \begin{itemize}
      \item $\vt$ is a tuple type with type list $\vlit$;
      \item $\vs$ is a tuple type with type list $\vlis$;
      \item checking whether $\vlit$ and $\vlis$ have the same number of elements yields $\True$
            or a type error, which short-circuits the entire rule (indicating that the number of elements in both tuples is expected
            to be the same and thus there is no lowest common ancestor);
      \item $\vli[\vi]$ is the of types lowest common ancestor of $\vlit[\vi]$ and $\vlis[\vi]$, for every position of $\vlit$;
      \item $\tty$ is the tuple type with list of types $\vli$, that is, $\TTuple(\vli)$.
    \end{itemize}

    \item All of the following apply (\textsc{labels}):
    \begin{itemize}
      \item either the AST labels of $\vt$ and $\vs$ are different, or one of them is $\TEnum$, $\TRecord$, or $\TException$;
      \item the result is a type error indicating the lack of a lowest common ancestor.
    \end{itemize}
  \end{itemize}
\end{itemize}

\subsection{Example}

\CodeSubsection{\LowestCommonAncestorBegin}{\LowestCommonAncestorEnd}{../types.ml}

\begin{emptyformal}
\subsection{Formally}
Since we do not impose a canonical representation on types (e.g., \verb|integer {1, 2}| is equivalence to \verb|integer {1..2}|),
the lowest common ancestor is not unique.
We define $\lca(\tenv, \vt, \vs)$ to be any type $\vtp$ that is \typeequivalent\ to the lowest common ancestor of $\vt$ and $\vs$.

\begin{mathpar}
\inferrule[type\_equal]{
  \typeequal(\tenv, \vt, \vs) \typearrow \True
}{
  \lca(\tenv, \vt, \vs) \typearrow \vs
}
\and
\inferrule[named\_subtype1]{
  \typeequal(\tenv, \vt, \vs) \typearrow \False\\
  \namedlca(\tenv, \namesubs, \namesubt) \typearrow \None \OrTypeError\\
  \makeanonymous(\tenv, \vs) \typearrow \vanons \OrTypeError\\
  \makeanonymous(\tenv, \vt) \typearrow \vanont \OrTypeError\\
  \lca(\tenv, \vanont, \vanons) \typearrow \tty \OrTypeError
}{
  \lca(\tenv, \TNamed(\namesubs), \TNamed(\namesubt)) \typearrow \tty
}
\and
\inferrule[named\_subtype2]{
  \typeequal(\tenv, \vt, \vs) \typearrow \False\\
  \namedlca(\tenv, \namesubs, \namesubt) \typearrow \langle\name\rangle \OrTypeError\\
}{
  \lca(\tenv, \TNamed(\namesubs), \TNamed(\namesubt)) \typearrow \TNamed(\name)
}
\and
\inferrule[one\_named1]{
  \typeequal(\tenv, \vt, \vs) \typearrow \False\\
  (\astlabel(\vt) = \TNamed \lor \astlabel(\vs) = \TNamed)\\
  \astlabel(\vt) \neq \astlabel(\vs)\\
  \makeanonymous(\tenv, \vs) \typearrow \vanons \OrTypeError\\
  \makeanonymous(\tenv, \vt) \typearrow \vanont \OrTypeError\\
  \typeequal(\tenv, \vanont, \vanons) \typearrow \True\\
  \tty \eqdef \choice{\astlabel(\vt) = \TNamed}{\vt}{\vs}
}{
  \lca(\tenv, \vt, \vs) \typearrow \tty
}
\and
\inferrule[one\_named2]{
  \typeequal(\tenv, \vt, \vs) \typearrow \False\\
  (\astlabel(\vt) = \TNamed \lor \astlabel(\vs) = \TNamed)\\
  \astlabel(\vt) \neq \astlabel(\vs)\\
  \makeanonymous(\tenv, \vs) \typearrow \vanons \OrTypeError\\
  \makeanonymous(\tenv, \vt) \typearrow \vanont \OrTypeError\\
  \typeequal(\tenv, \vanont, \vanons) \typearrow \False\\
  \lca(\tenv, \vanont, \vanons) \typearrow \tty \OrTypeError
}{
  \lca(\tenv, \vt, \vs) \typearrow \tty
}
\end{mathpar}

\begin{mathpar}
\inferrule[t\_int\_unconstrained]{
  \typeequal(\tenv, \vt, \vs) \typearrow \False\\
  \astlabel(\vt) = \astlabel(\vs) = \TInt\\
  \isunconstrainedinteger(\vt) \lor \isunconstrainedinteger(\vs)
}{
  \lca(\tenv, \vt, \vs) \typearrow \unconstrainedinteger
}
\and
\inferrule[t\_int\_underconstrained]{
  \typeequal(\tenv, \vt, \vs) \typearrow \False\\
  \astlabel(\vt) = \astlabel(\vs) = \TInt\\
  \neg\isunconstrainedinteger(\vt)\\
  \neg\isunconstrainedinteger(\vs)\\
  \isunderconstrainedinteger(\vt) \lor \isunderconstrainedinteger(\vs)\\
  \towellconstrained(\tenv, \vt) \typearrow \vtone\\
  \towellconstrained(\tenv, \vs) \typearrow \vsone\\
  \lca(\tenv, \vtone, \vsone) \typearrow \tty \OrTypeError
}{
  \lca(\tenv, \vt, \vs) \typearrow \tty
}
\and
\inferrule[t\_int\_wellconstrained]
{
  \typeequal(\tenv, \vt, \vs) \typearrow \False\\
  \astlabel(\vt) = \astlabel(\vs) = \TInt\\
  \vt \eqname \TInt(\wellconstrained(\cst))\\
  \vs \eqname \TInt(\wellconstrained(\css))
}{
  \lca(\tenv, \vt, \vs) \typearrow \TInt(\wellconstrained(\cst \concat \css))
}
\end{mathpar}

\begin{mathpar}
\inferrule[t\_bits\_equal]{
  \typeequal(\tenv, \vt, \vs) \typearrow \False\\
  \vt \eqname \TBits(\vet, \Ignore)\\
  \vs \eqname \TBits(\ves, \Ignore)\\
  \exprequal(\tenv, \vet, \ves) \typearrow \True
}{
  \lca(\tenv, \vt, \vs) \typearrow \TBits(\vet, \emptylist)
}
\and
\inferrule[t\_bits\_not\_equal]{
  \typeequal(\tenv, \vt, \vs) \typearrow \False\\
  \vt \eqname \TBits(\vet, \Ignore)\\
  \vs \eqname \TBits(\ves, \Ignore)\\
  \exprequal(\tenv, \vet, \ves) \typearrow \False
}{
  \lca(\tenv, \vt, \vs) \typearrow \TypeErrorVal{NoLCA}
}
\end{mathpar}

\begin{mathpar}
\inferrule[t\_array\_equal]{
  \typeequal(\tenv, \vt, \vs) \typearrow \False\\
  \vt \eqname \TArray(\widtht, \vtyt)\\
  \vs \eqname \TArray(\widths, \vtys)\\
  \arraylengthequal(\tenv, \widtht, \widths) \typearrow \True\\
  \lca(\tenv, \vtyt, \vtys) \typearrow \vtone \OrTypeError
}{
  \lca(\tenv, \vt, \vs) \typearrow \TArray(\widtht, \vtone)
}
\and
\inferrule[t\_array\_not\_equal]{
  \typeequal(\tenv, \vt, \vs) \typearrow \False\\
  \vt \eqname \TArray(\widtht, \vtyt)\\
  \vs \eqname \TArray(\widths, \vtys)\\
  \arraylengthequal(\tenv, \widtht, \widths) \typearrow \False
}{
  \lca(\tenv, \vt, \vs) \typearrow \TypeErrorVal{NoLCA}
}
\end{mathpar}

\begin{mathpar}
\inferrule[t\_tuple]{
  \typeequal(\tenv, \vt, \vs) \typearrow \False\\
  \vt \eqname \TTuple(\vlit)\\
  \vs \eqname \TTuple(\vlis)\\
  \equallength(\vlit, \vlis) \typearrow \vb\\
  \checktrans{\vb}{NoLCA/TuplesHaveDifferentLengths} \typearrow \True \OrTypeError\\\\
  \vi\in\listrange(\vlit): \lca(\tenv, \vlit[\vi], \vlis[\vi]) \typearrow \vli[\vi] \OrTypeError\\
  \vli \eqdef [\vi\in\listrange(\vlit): \vli[\vi]]
}{
  \lca(\tenv, \vt, \vs) \typearrow \TTuple(\vli)
}
\end{mathpar}

\begin{mathpar}
\inferrule[labels]{
  \typeequal(\tenv, \vt, \vs) \typearrow \False\\
  (\astlabel(\vt) \neq \astlabel(\vs)) \lor
  \astlabel(\vt) \in \{\TEnum, \TRecord, \TException\}
}
{
  \lca(\tenv, \vt, \vs) \typearrow \TypeErrorVal{NoLCA}
}
\end{mathpar}
\end{emptyformal}

\subsection{Comments}
    This is related to \identr{YZHM}.

\section{TypingRule.CheckUnop \label{sec:TypingRule.CheckUnop}}
\hypertarget{def-checkunop}{}
The function
\[
  \CheckUnop(\overname{\staticenvs}{\tenv} \aslsep \overname{\unop}{\op} \aslsep \overname{\ty}{\vt})
  \aslto \overname{\ty}{\vs}
\]
determines the result type of applying a unary operator when the type of its operand is known.
Similarly, we determine the negation of integer constraints.

\subsection{Prose}
One of the following applies:
\begin{itemize}
\item All of the following apply (\textsc{bnot\_t\_bool}):
  \begin{itemize}
    \item $\op$ is $\BNOT$;
    \item determining whether $\vt$ \typesatisfies\ $\TBool$ yields $\True$ \ProseOrTypeError;
    \item $\vs$ is $\TBool$;
  \end{itemize}

\item All of the following apply (\textsc{neg\_error}):
\begin{itemize}
  \item $\op$ is $\NEG$;
  \item determining whether $\vt$ \typesatisfies\ $\TReal$ yields $\False$ \ProseOrTypeError;
  \item determining whether $\vt$ \typesatisfies\ $\TInt(\unconstrained)$ yields $\False$ \ProseOrTypeError;
  \item the result is a type error indicating the $\NEG$ is appropriate only for \texttt{real} and \texttt{integer} types;
\end{itemize}

\item All of the following apply (\textsc{neg\_t\_rel}):
\begin{itemize}
  \item $\op$ is $\NEG$;
  \item determining whether $\vt$ \typesatisfies\ $\TReal$ yields $\True$ \ProseOrTypeError;
  \item $\vs$ is $\TReal$;
\end{itemize}

\item All of the following apply (\textsc{neg\_t\_int\_unconstrained}):
\begin{itemize}
  \item $\op$ is $\NEG$;
  \item determining whether $\vt$ \typesatisfies\ $\unconstrainedinteger$ yields $\True$ \ProseOrTypeError;
  \item obtaining the \structure\ of $\vt$ yields $\unconstrainedinteger$ \ProseOrTypeError;
  \item $\vs$ is $\unconstrainedinteger$;
\end{itemize}

\item All of the following apply (\textsc{neg\_t\_int\_well\_constrained}):
\begin{itemize}
  \item $\op$ is $\NEG$;
  \item determining whether $\vt$ \typesatisfies\ $\unconstrainedinteger$ yields $\True$ \ProseOrTypeError;
  \item obtaining the \structure\ of $\vt$ yields the well-constrained integer type with constraints $\vcs$ \ProseOrTypeError;
  \item negating the constraints in $\vcs$ (see $\negateconstraint$) yields $\vcsnew$;
  \item $\vs$ is the well-constrained integer type with constraints $\vcsnew$, that is, \\
  $\TInt(\wellconstrained(\vcsnew))$;
\end{itemize}

\item All of the following apply (\textsc{neg\_t\_int\_underconstrained}):
\begin{itemize}
  \item $\op$ is $\NEG$;
  \item determining whether $\vt$ \typesatisfies\ $\unconstrainedinteger$ yields $\True$ \ProseOrTypeError;
  \item obtaining the \structure\ of $\vt$ yields the underconstrained integer type with parameter $\vv$, that is,
  $\TInt(\wellconstrained(\vv))$ \ProseOrTypeError;
  \item converting $\TInt(\wellconstrained(\vv))$ to a well-constrained integer with an exact constraint expression for $\vv$
  (see $\getwellconstrainedstructure$) yields $\vtone$;
  \item obtaining the appropriate type for the unary operator $\NEG$ in $\tenv$ for $\vt$ yields $\vs$.
\end{itemize}

\item All of the following apply (\textsc{not\_t\_bits}):
  \begin{itemize}
  \item $\op$ is $\NOT$;
  \item $\vt$ has the structure of a bitvector;
  \item $\vs$ is $\vt$.
  \end{itemize}
\end{itemize}

\subsection{Example}

\CodeSubsection{\CheckUnopBegin}{\CheckUnopEnd}{../Typing.ml}

\begin{emptyformal}
\subsection{Formally}
\begin{mathpar}
\inferrule[bnot\_t\_bool]{
  \checktypesat(\tenv, \vtone, \TBool) \typearrow \True \OrTypeError\\
}{
  \CheckUnop(\tenv, \BNOT, \vtone) \typearrow \TBool
}
\end{mathpar}

\hypertarget{def-negateconstraint}{}
We now define the helper function
\[
  \negateconstraint(\intconstraint) \aslto \intconstraint
\]
which takes an integer constraint and returns the constraint that corresponds to the negation of all
the values it represents:

\begin{mathpar}
\inferrule{}
{
  \negateconstraint(\ConstraintExact(\ve)) \typearrow \ConstraintExact(\EUnop(\MINUS, \ve))
}
\and
\inferrule{}
{
  \negateconstraint(\ConstraintRange(\vvtop, \vbot)) \typearrow \\
  \ConstraintRange(\EUnop(\MINUS, \vbot), \EUnop(\MINUS, \vvtop))
}
\end{mathpar}

\begin{mathpar}
\inferrule[neg\_error]{
  \typesat(\tenv, \vtone, \TInt(\unconstrained)) \typearrow \False \OrTypeError\\
  \typesat(\tenv, \vtone, \TReal) \typearrow \False \OrTypeError\\
}{
  \CheckUnop(\tenv, \NEG, \vt) \typearrow \TypeErrorVal{InappropriateTypeForNeg}
}
\end{mathpar}

\begin{mathpar}
\inferrule[neg\_t\_real]{
  \typesat(\tenv, \vtone, \TReal) \typearrow \True \OrTypeError
}{
  \CheckUnop(\tenv, \NEG, \vt) \typearrow \TReal
}
\end{mathpar}

\begin{mathpar}
\inferrule[neg\_t\_int\_unconstrained]{
  \typesat(\tenv, \vtone, \unconstrainedinteger) \typearrow \vbone \OrTypeError\\
  \tstruct(\tenv, \vtone) \typearrow \TInt(\unconstrained)\\
}{
  \CheckUnop(\tenv, \NEG, \vt) \typearrow \TInt(\unconstrained)
}
\end{mathpar}

\begin{mathpar}
\inferrule[neg\_t\_int\_well\_constrained]{
  \typesat(\tenv, \vt, \unconstrainedinteger) \typearrow \True\\
  \tstruct(\tenv, \vtone) \typearrow \TInt(\wellconstrained(\vcs))\\
  \vc \in \vcs: \negateconstraint(\vc) \typearrow \vneg_\vc\\
  \vcsnew \eqdef [\vc \in \vcs: \vneg_\vc]
}{
  \CheckUnop(\tenv, \NEG, \vt) \typearrow \TInt(\wellconstrained(\vcsnew))
}
\end{mathpar}

\begin{mathpar}
\inferrule[neg\_t\_int\_underconstrained]{
  \typesat(\tenv, \vt, \unconstrainedinteger) \typearrow \True\\
  \tstruct(\tenv, \vtone) \typearrow \TInt(\underconstrained(\vv))\\
  \getwellconstrainedstructure(\tenv, \vt) \typearrow \vtone\\
  \CheckUnop(\tenv, \NEG, \vtone) \typearrow \vs
}{
  \CheckUnop(\tenv, \NEG, \vt) \typearrow \vs
}
\end{mathpar}

\begin{mathpar}
\inferrule[not\_t\_bits]{
  \checkstructurelabel(\tenv, \vt, \TBits) \typearrow \True \OrTypeError
}{
  \CheckUnop(\tenv, \NOT, \vt) \typearrow \vt
}
\end{mathpar}
\end{emptyformal}

\isempty{\subsection{Comments}}

\section{TypingRule.CheckBinop \label{sec:TypingRule.CheckBinop}}
\hypertarget{def-checkbinop}{}
The function
\[
  \CheckBinop(\overname{\staticenvs}{\tenv} \aslsep \overname{\binop}{\op} \aslsep \overname{\ty}{\vtone}
  \aslsep \overname{\ty}{\vttwo})
  \aslto \overname{\ty}{\vt}
\]
determines the result type $\vt$ of applying the binary operator $\op$
to operands of type $\vtone$ and $\vttwo$ in the static environment $\tenv$,
returning a type error, if one is detected.

\subsection{Prose}
One of the following applies:
\begin{itemize}
  \item All of the following apply (\textsc{boolean}):
  \begin{itemize}
    \item $\op$ is $\AND$, $\OR$, $\EQOP$ or $\IMPL$;
    \item determining whether $\vtone$ \typesatisfies\ $\TBool$ in $\tenv$ yields $\True$ \ProseOrTypeError;
    \item determining whether $\vttwo$ \typesatisfies\ $\TBool$ in $\tenv$ yields $\True$ \ProseOrTypeError;
    \item $\vt$ is $\TBool$.
  \end{itemize}

  \item All of the following apply (\textsc{bits\_bool}):
  \begin{itemize}
    \item $\op$ is $\AND$, $\OR$, or $\EOR$;
    \item checking whether $\vtone$ and $\vttwo$ have the \structure\ of bitvector types
          of the same width in $\tenv$ yields $\True$ \ProseOrTypeError;
    \item the bitvector width of $\vtone$ in $\tenv$ is $\vw$;
    \item $\vt$ is the bitvector type of width $\vw$ and empty list of bitfields, that is, \\ $\TBits(\vw, \emptylist)$.
  \end{itemize}

  \item All of the following apply (\textsc{plus\_minus\_error}):
  \begin{itemize}
    \item $\op$ is $\PLUS$ or $\MINUS$;
    \item obtaining the \structure\ of $\vtone$ in $\tenv$ is $\vtonestruct$ \ProseOrTypeError;
    \item $\vtonestruct$ is neither a bitvector type nor an integer type;
    \item the result is a type error indicating that the type of $\vtone$ is inappropriate for $\op$.
  \end{itemize}

  \item All of the following apply (\textsc{plus\_minus\_bits\_int}):
  \begin{itemize}
    \item $\op$ is $\PLUS$ or $\MINUS$;
    \item obtaining the \structure\ of $\vtone$ in $\tenv$ is $\vtonestruct$ \ProseOrTypeError;
    \item $\vtonestruct$ is a bitvector type;
    \item $\vttwo$ \typesatisfies\ the unconstrained integer type in $\tenv$;
    \item obtaining the bitwidth of $\vtone$ in $\tenv$ yields $\vw$.
    \item $\vt$ is the bitvector type of width $\vw$ and empty list of bitfields, that is, \\ $\TBits(\vw, \emptylist)$.
  \end{itemize}

  \item All of the following apply (\textsc{plus\_minus\_bits\_bits}):
  \begin{itemize}
    \item $\op$ is $\PLUS$ or $\MINUS$;
    \item obtaining the \structure\ of $\vtone$ in $\tenv$ is a bitvector of width $\veone$, that is,\\ $\TBits(\veone, \Ignore)$;
    \item $\vttwo$ does not \typesatisfy\ the unconstrained integer type in $\tenv$
    \item obtaining the \structure\ of $\vttwo$ in $\tenv$ is $\vttwostruct$ \ProseOrTypeError;
    \item determining whether $\vttwostruct$ has a bitvector type yields $\True$ \ProseOrTypeError;
    \item $\vttwostruct$ is a bitvector of width $\vwtwo$, that is, $\TBits(\vwtwo, \Ignore)$;
    \item determining whether $\vwone$ and $\vwtwo$ are equal bitwidths yields $\vb$;
    \item $\vb$ is $\True$ \ProseOrTypeError;
    \item $\vt$ is the bitvector type of width $\vwone$ and empty list of bitfields, that is, \\ $\TBits(\vwone, \emptylist)$.
  \end{itemize}

  \item All of the following apply (\textsc{eq\_neq\_error}):
  \begin{itemize}
    \item $\op$ is either $\EQOP$ or $\NEQ$;
    \item the \underlyingtype\ of $\vtone$ in $\tenv$ is $\vtoneanon$ \ProseOrTypeError;
    \item the \underlyingtype\ of $\vttwo$ in $\tenv$ is $\vttwoanon$ \ProseOrTypeError;
    \item the AST labels of $\vtoneanon$ and $\vttwoanon$ are different or one of them is not in
          $\{\TInt, \TReal, \TBool, \TBits, \TEnum\}$;
    \item the result is a type error indicating that the types of $\vtone$ and $\vttwo$ are inappropriate for $\op$.
  \end{itemize}

  \item All of the following apply (\textsc{eq\_neq\_bits}):
  \begin{itemize}
    \item $\op$ is either $\EQOP$ or $\NEQ$;
    \item the \underlyingtype\ of $\vtone$ in $\tenv$ is $\vtoneanon$ \ProseOrTypeError;
    \item $\vtoneanon$ is a bitvector type;
    \item the \underlyingtype\ of $\vttwo$ in $\tenv$ is $\vttwoanon$ \ProseOrTypeError;
    \item checking whether the bitwidth of $\vtoneanon$ and $\vttwoanon$ is the same yields $\True$ \ProseOrTypeError;
    \item $\vt$ is $\TBool$.
  \end{itemize}

  \item All of the following apply (\textsc{eq\_neq\_bool}):
  \begin{itemize}
    \item $\op$ is either $\EQOP$ or $\NEQ$;
    \item the \underlyingtype\ of $\vtone$ in $\tenv$ is $\TBool$ \ProseOrTypeError;
    \item the \underlyingtype\ of $\vttwo$ in $\tenv$ is $\TBool$ \ProseOrTypeError;
    \item checking whether $\vtoneanon$ \typesatisfies\ $\TBool$ yields $\True$ \ProseOrTypeError;
    \item checking whether $\vttwoanon$ \typesatisfies\ $\TBool$ yields $\True$ \ProseOrTypeError;
    \item $\vt$ is $\TBool$.
  \end{itemize}

  \item All of the following apply (\textsc{eq\_neq\_real}):
  \begin{itemize}
    \item $\op$ is either $\EQOP$ or $\NEQ$;
    \item the \underlyingtype\ of $\vtone$ in $\tenv$ is $\TReal$ \ProseOrTypeError;
    \item the \underlyingtype\ of $\vttwo$ in $\tenv$ is $\TReal$ \ProseOrTypeError;
    \item checking whether $\vtoneanon$ \typesatisfies\ $\TBool$ yields $\True$ \ProseOrTypeError;
    \item checking whether $\vttwoanon$ \typesatisfies\ $\TBool$ yields $\True$ \ProseOrTypeError;
    \item $\vt$ is $\TBool$.
  \end{itemize}

  \item All of the following apply (\textsc{eq\_neq\_string}):
  \begin{itemize}
    \item $\op$ is either $\EQOP$ or $\NEQ$;
    \item the \underlyingtype\ of $\vtone$ in $\tenv$ is $\TString$ \ProseOrTypeError;
    \item the \underlyingtype\ of $\vttwo$ in $\tenv$ is $\TString$ \ProseOrTypeError;
    \item checking whether $\vtoneanon$ \typesatisfies\ $\TBool$ yields $\True$ \ProseOrTypeError;
    \item checking whether $\vttwoanon$ \typesatisfies\ $\TBool$ yields $\True$ \ProseOrTypeError;
    \item $\vt$ is $\TBool$.
  \end{itemize}

  \item All of the following apply (\textsc{eq\_neq\_enum}):
  \begin{itemize}
    \item $\op$ is either $\EQOP$ or $\NEQ$;
    \item the \underlyingtype\ of $\vtone$ in $\tenv$ is $\TEnum(\vlione)$ \ProseOrTypeError;
    \item the \underlyingtype\ of $\vttwo$ in $\tenv$ is $\\TEnum(\vlitwo)$ \ProseOrTypeError;
    \item checking whether $\vlione$ is equal to $\vlitwo$ yields $\True$ \ProseOrTypeError;
    \item $\vt$ is $\TBool$.
  \end{itemize}

  \item All of the following apply (\textsc{relational}):
  \begin{itemize}
    \item $\op$ is one of $\LT$, $\LEQ$, $\GT$, and $\GEQ$;
    \item determining whether both $\vtone$ and $\vttwo$ \typesatisfy\ the unconstrained integer type in $\tenv$
          or both $\vtone$ and $\vttwo$ \typesatisfy\ the \texttt{real} type in $\tenv$ yields $\True$ \ProseOrTypeError;
    \item $\vt$ is $\TBool$.
  \end{itemize}

  \item All of the following apply (\textsc{arith\_error}):
  \begin{itemize}
    \item obtaining the \structure\ of $\vtone$ in $\tenv$ yields $\vtonestruct$ \ProseOrTypeError;
    \item obtaining the \structure\ of $\vttwo$ in $\tenv$ yields $\vttwostruct$ \ProseOrTypeError;
    \item the operator and the AST labels of $\vtonestruct$ and $\vttwostruct$ do not match any of the rows in the following table:\\
    \begin{center}
    \begin{tabular}{lll}
      \op    & $\astlabel(\vtonestruct)$ & $\astlabel(\vttwostruct)$\\
      \hline
      \MUL   & \TInt  & \TInt\\
      \DIV   & \TInt  & \TInt\\
      \DIVRM & \TInt  & \TInt\\
      \MOD   & \TInt  & \TInt\\
      \SHL   & \TInt  & \TInt\\
      \SHR   & \TInt  & \TInt\\
      \POW   & \TInt  & \TInt\\
      \PLUS  & \TInt  & \TInt\\
      \MINUS & \TInt  & \TInt\\
      \PLUS  & \TReal & \TReal\\
      \MINUS & \TReal & \TReal\\
      \MUL   & \TReal & \TReal\\
      \RDIV  & \TReal & \TReal\\
      \POW   & \TReal & \TInt\\
    \end{tabular}
  \end{center}
    \item the result is a type error indicating that the types of $\vtone$ and $\vttwo$ are inappropriate for $\op$.
  \end{itemize}

  \item All of the following apply (\textsc{arith\_t\_int\_unconstrained1}, \\
                                    \textsc{arith\_t\_int\_unconstrained2}):
  \begin{itemize}
    \item $\op$ is one of $\{\MUL, \DIV, \DIVRM, \MOD, \SHL,  \SHR, \POW, \PLUS, \MINUS\}$;
    \item the \wellconstrainedstructure\ of either $\vtone$ or $\vttwo$ in $\tenv$ is that of the unconstrained integer type;
    \item $\vt$ is the unconstrained integer type;
  \end{itemize}

  \item All of the following apply (\textsc{arith\_t\_int\_wellconstrained1}):
  \begin{itemize}
    \item $\op$ is one of $\{\MUL, \POW, \PLUS, \MINUS\}$;
    \item the \wellconstrainedstructure\ of either $\vtone$ in $\tenv$ is that of a well-constrained integer type with
          constraints $\vcsone$;
          \item the \wellconstrainedstructure\ of either $\vttwo$ in $\tenv$ is that of a well-constrained integer type with
          constraints $\vcstwo$;
    \item applying $\op$ to $\vcsone$ and $\vcstwo$ in $\tenv$ yields $\vcs$;
    \item $\vt$ is the well-constrained integer type with constraints $\vcs$;
  \end{itemize}

  \item All of the following apply (\textsc{arith\_t\_int\_wellconstrained2}):
  \begin{itemize}
    \item $\op$ is one of $\{\DIVRM, \DIV, \MOD\}$;
    \item the \wellconstrainedstructure\ of either $\vtone$ in $\tenv$ is that of a well-constrained integer type with
          constraints $\vcsone$;
          \item the \wellconstrainedstructure\ of either $\vttwo$ in $\tenv$ is that of a well-constrained integer type with
          constraints $\vcstwo$;
    \item checking whether $\vcstwo$ represents strictly-positive integers yields $\True$ \ProseOrTypeError;
    \item applying $\op$ to $\vcsone$ and $\vcstwo$ in $\tenv$ yields $\vcs$;
    \item $\vt$ is the well-constrained integer type with constraints $\vcs$;
  \end{itemize}

  \item All of the following apply (\textsc{arith\_t\_int\_wellconstrained3}):
  \begin{itemize}
    \item $\op$ is one of $\{\DIVRM, \DIV, \MOD\}$;
    \item the \wellconstrainedstructure\ of either $\vtone$ in $\tenv$ is that of a well-constrained integer type with
          constraints $\vcsone$;
          \item the \wellconstrainedstructure\ of either $\vttwo$ in $\tenv$ is that of a well-constrained integer type with
          constraints $\vcstwo$;
    \item checking whether $\vcstwo$ represents non-negative integers yields $\True$ \ProseOrTypeError;
    \item applying $\op$ to $\vcsone$ and $\vcstwo$ in $\tenv$ yields $\vcs$;
    \item $\vt$ is the well-constrained integer type with constraints $\vcs$;
  \end{itemize}

  \item All of the following apply (\textsc{plus\_minus\_mul\_real}):
  \begin{itemize}
    \item $\op$ is one of $\{\PLUS, \MINUS, \MUL\}$;
    \item obtaining the \structure\ of $\vtone$ in $\tenv$ yields $\TReal$;
    \item obtaining the \structure\ of $\vttwo$ in $\tenv$ yields $\TReal$;
    \item $\vt$ is $\TReal$.
  \end{itemize}

  \item All of the following apply (\textsc{pow\_real\_int}):
  \begin{itemize}
    \item $\op$ is one of $\{\PLUS, \MINUS, \MUL\}$;
    \item obtaining the \structure\ of $\vtone$ in $\tenv$ yields $\TReal$;
    \item obtaining the \structure\ of $\vttwo$ in $\tenv$ yields an integer type;
    \item $\vt$ is $\TReal$.
  \end{itemize}

  \item All of the following apply (\textsc{rdiv}):
  \begin{itemize}
    \item $\op$ is one of $\{\RDIV\}$;
    \item determining whether $\vtone$ \typesatisfies\ $\TReal$ yields $\True$ \ProseOrTypeError;
    \item determining whether $\vttwo$ \typesatisfies\ $\TReal$ yields $\True$ \ProseOrTypeError;
    \item $\vt$ is $\TReal$.
  \end{itemize}
\end{itemize}

\subsection{Example}

\CodeSubsection{\CheckBinopBegin}{\CheckBinopEnd}{../Typing.ml}

\begin{emptyformal}
\subsection{Formally}
\begin{mathpar}
\inferrule[boolean]{
  \op \in  \{\BAND, \BOR, \IMPL, \EQOP\}\\
  \checktypesat(\tenv, \vt1, \TBool) \typearrow \True \OrTypeError\\
  \checktypesat(\tenv, \vttwo, \TBool) \typearrow \True \OrTypeError
}{
  \CheckBinop(\tenv, \op, \vtone, \vttwo) \typearrow \TBool}
\end{mathpar}

\begin{mathpar}
\inferrule[bits\_bool]{
  \op \in  \{\AND, \OR, \EOR\}\\
  \checkbitsequalwidth(\tenv, \vtone, \vttwo) \typearrow \True \OrTypeError\\
  \getbitvectorwidth(\tenv, \vtone) \typearrow \vw
}{
  \CheckBinop(\tenv, \op, \vtone, \vttwo) \typearrow \TBits(\vw, \emptylist)
}
\end{mathpar}

\begin{mathpar}
\inferrule[plus\_minus\_error]{
  \op \in  \{\PLUS, \MINUS\}\\
  \tstruct(\tenv, \vtone) \typearrow \vtonestruct \OrTypeError\\\\
  \astlabel(\vtonestruct) \not\in \{\TBits,\TInt\}\\
}{
  \CheckBinop(\tenv, \op, \vtone, \vttwo) \typearrow \TypeErrorVal{InappropriateTypeForPlusMinus}
}
\and
\inferrule[plus\_minus\_bits\_int]{
  \op \in  \{\PLUS, \MINUS\}\\
  \tstruct(\tenv, \vtone) \typearrow \vtonestruct \OrTypeError\\\\
  \astlabel(\vtonestruct) = \TBits\\
  \typesat(\tenv, \vttwo, \unconstrainedinteger) \typearrow \True\\
  \getbitvectorwidth(\tenv, \vtone) \typearrow \vw\\
}{
  \CheckBinop(\tenv, \op, \vtone, \vttwo) \typearrow \TBits(\vw, \emptylist)
}
\and
\inferrule[plus\_minus\_bits\_bits]{
  \op \in  \{\PLUS, \MINUS\}\\
  \tstruct(\tenv, \vtone) \typearrow \TBits(\vwone, \Ignore) \\
  \typesat(\tenv, \vttwo, \unconstrainedinteger) \typearrow \False\\
  \tstruct(\tenv, \vttwo) \typearrow \vttwostruct \OrTypeError\\\\
  \checktrans{\astlabel(\vttwostruct)=\TBits}{ExpectedBitvectorType} \checktransarrow \True \OrTypeError\\
  \vttwostruct \eqname \TBits(\vwtwo, \Ignore)\\
  \bitwidthequal(\tenv, \vwone, \vwtwo) \typearrow \vb\\
  \checktrans{\vb}{DifferentBitwidths} \checktransarrow \True \OrTypeError
}{
  \CheckBinop(\tenv, \op, \vtone, \vttwo) \typearrow \TBits(\vwone, \emptylist)
}
\end{mathpar}

\begin{mathpar}
\inferrule[eq\_neq\_error]
{
  \op \in  \{\EQOP, \NEQ\}\\
  \makeanonymous(\tenv, \vtone) \typearrow \vtoneanon \OrTypeError\\
  \makeanonymous(\tenv, \vttwo) \typearrow \vttwoanon \OrTypeError\\
  (\astlabel(\vtoneanon) \neq \astlabel(\vttwoanon)) \lor
  \astlabel(\vtone) \not\in \{\TInt, \TReal, \TBool, \TBits, \TEnum\}
}{
  \CheckBinop(\tenv, \op, \vtone, \vttwo) \typearrow \TypeErrorVal{InappropriateTypeForEQ}
}
\and
\inferrule[eq\_neq\_bits]{
  \op \in  \{\EQOP, \NEQ\}\\
  \makeanonymous(\tenv, \vtone) \typearrow \vtoneanon \OrTypeError\\
  \astlabel(\vtoneanon) = \TBits\\
  \makeanonymous(\tenv, \vttwo) \typearrow \vttwoanon \OrTypeError\\
  \checkbitsequalwidth(\tenv, \vtoneanon, \vttwoanon) \typearrow \True \OrTypeError
}{
  \CheckBinop(\tenv, \op, \vtone, \vttwo) \typearrow \TBool
}
\and
\inferrule[eq\_neq\_bool]{
  \op \in  \{\EQOP, \NEQ\}\\
  \makeanonymous(\tenv, \vtone) \typearrow \TBool\\
  \makeanonymous(\tenv, \vttwo) \typearrow \TBool\\
  \checktypesat(\tenv, \vtoneanon, \TBool) \typearrow \True \OrTypeError\\
  \checktypesat(\tenv, \vttwoanon, \TBool) \typearrow \True \OrTypeError
}{
  \CheckBinop(\tenv, \op, \vtone, \vttwo) \typearrow \TBool
}
\and
\inferrule[eq\_neq\_real]{
  \op \in  \{\EQOP, \NEQ\}\\
  \makeanonymous(\tenv, \vtone) \typearrow \TReal\\
  \makeanonymous(\tenv, \vttwo) \typearrow \TReal\\
  \checktypesat(\tenv, \vtoneanon, \TReal) \typearrow \True \OrTypeError\\
  \checktypesat(\tenv, \vttwoanon, \TReal) \typearrow \True \OrTypeError
}{
  \CheckBinop(\tenv, \op, \vtone, \vttwo) \typearrow \TBool
}
\and
\inferrule[eq\_neq\_string]{
  \op \in  \{\EQOP, \NEQ\}\\
  \makeanonymous(\tenv, \vtone) \typearrow \TString\\
  \makeanonymous(\tenv, \vttwo) \typearrow \TString\\
  \checktypesat(\tenv, \vtoneanon, \TString) \typearrow \True \OrTypeError\\
  \checktypesat(\tenv, \vttwoanon, \TString) \typearrow \True \OrTypeError
}{
  \CheckBinop(\tenv, \op, \vtone, \vttwo) \typearrow \TBool
}
\and
\inferrule[eq\_neq\_enum]{
  \op \in  \{\EQOP, \NEQ\}\\
  \makeanonymous(\tenv, \vtone) \typearrow \TEnum(\vlione)\\
  \makeanonymous(\tenv, \vttwo) \typearrow \TEnum(\vlitwo)\\
  \checktrans{\vlione = \vlitwo}{DifferentEnumLabels} \checktransarrow \True \OrTypeError
}{
  \CheckBinop(\tenv, \op, \vtone, \vttwo) \typearrow \TBool
}
\end{mathpar}

\begin{mathpar}
\inferrule[relational]{
  \op \in  \{\LT, \LEQ, \GT, \GEQ\}\\
  \typesat(\tenv, \vtone, \unconstrainedinteger) \typearrow \vbone \OrTypeError\\
  \typesat(\tenv, \vttwo, \unconstrainedinteger) \typearrow \vbtwo \OrTypeError\\
  \typesat(\tenv, \vtone, \TReal) \typearrow \vbthree \OrTypeError\\
  \typesat(\tenv, \vttwo, \TReal) \typearrow \vbfour \OrTypeError\\
  \checktrans{\vbone \land \vbtwo \lor \vbthree \land \vbfour}{InappropriateTypeForRel} \checktransarrow \True \OrTypeError
}{
  \CheckBinop(\tenv, \op, \vtone, \vttwo) \typearrow \TBool
}
\end{mathpar}

\begin{mathpar}
\inferrule[arith\_error]{
  \tstruct(\tenv, \vtone) \typearrow \vtonestruct \OrTypeError\\
  \tstruct(\tenv, \vttwo) \typearrow \vttwostruct \OrTypeError\\
  (\op, \astlabel(\vtonestruct), \astlabel(\vttwostruct)) \not\in
  {
    \left\{
    \begin{array}{lclcl}
      (\MUL   &,& \TInt  &,& \TInt)\\
      (\DIV   &,& \TInt  &,& \TInt)\\
      (\DIVRM &,& \TInt  &,& \TInt)\\
      (\MOD   &,& \TInt  &,& \TInt)\\
      (\SHL   &,& \TInt  &,& \TInt)\\
      (\SHR   &,& \TInt  &,& \TInt)\\
      (\POW   &,& \TInt  &,& \TInt)\\
      (\PLUS  &,& \TInt  &,& \TInt)\\
      (\MINUS &,& \TInt  &,& \TInt)\\
      (\PLUS  &,& \TReal &,& \TReal)\\
      (\MINUS &,& \TReal &,& \TReal)\\
      (\MUL   &,& \TReal &,& \TReal)\\
      (\RDIV  &,& \TReal &,& \TReal)\\
      (\POW   &,& \TReal &,& \TInt)\\
    \end{array}
    \right\}
  }
}{
  \CheckBinop(\tenv, \op, \vtone, \vttwo) \typearrow \TypeErrorVal{InappropriateTypeForBinop}
}
\end{mathpar}

The following two rules are not mutually exclusive, but both yield the same result when they are both active.
\begin{mathpar}
\inferrule[arith\_t\_int\_unconstrained1]{
  \op \in  \{\MUL, \DIV, \DIVRM, \MOD, \SHL,  \SHR, \POW, \PLUS, \MINUS\}\\
  \getwellconstrainedstructure(\tenv, \vtone) \typearrow \unconstrainedinteger\\
  \getwellconstrainedstructure(\tenv, \vttwo) \typearrow \TInt(\Ignore)\\
}{
  \CheckBinop(\tenv, \op, \vtone, \vttwo) \typearrow \unconstrainedinteger
}
\and
\inferrule[arith\_t\_int\_unconstrained2]{
  \op \in  \{\MUL, \DIV, \DIVRM, \MOD, \SHL,  \SHR, \POW, \PLUS, \MINUS\}\\
  \getwellconstrainedstructure(\tenv, \vtone) \typearrow \TInt(\Ignore)\\
  \getwellconstrainedstructure(\tenv, \vttwo) \typearrow \unconstrainedinteger\\
}{
  \CheckBinop(\tenv, \op, \vtone, \vttwo) \typearrow \unconstrainedinteger
}
\end{mathpar}

\begin{mathpar}
\inferrule[arith\_t\_int\_wellconstrained1]{
  \op \in  \{\MUL, \POW, \PLUS, \MINUS\}\\
  \getwellconstrainedstructure(\tenv, \vtone) \typearrow \TInt(\wellconstrained(\vcsone))\\
  \getwellconstrainedstructure(\tenv, \vttwo) \typearrow \TInt(\wellconstrained(\vcstwo))\\
  \tododefine{constraints\_binop}(\tenv, \op, \vcsone, \vcstwo) \typearrow \vcs
}{
  \CheckBinop(\tenv, \op, \vtone, \vttwo) \typearrow \TInt(\wellconstrained(\vcs))
}
\and
\inferrule[arith\_t\_int\_wellconstrained2]{
  \op \in  \{\DIVRM, \DIV, \MOD\}\\
  \getwellconstrainedstructure(\tenv, \vtone) \typearrow \TInt(\wellconstrained(\vcsone))\\
  \getwellconstrainedstructure(\tenv, \vttwo) \typearrow \TInt(\wellconstrained(\vcstwo))\\
  \tododefine{constraints\_is\_strict\_positive}(\vcstwo) \typearrow \vb\\
  \checktrans{\vb}{DenominatorMustBePositive} \checktransarrow \True \OrTypeError\\
  \tododefine{constraints\_binop}(\tenv, \op, \vcsone, \vcstwo) \typearrow \vcs
}{
  \CheckBinop(\tenv, \op, \vtone, \vttwo) \typearrow \TInt(\wellconstrained(\vcs))
}
\and
\inferrule[arith\_t\_int\_wellconstrained3]{
  \op \in  \{\SHL, \SHR\}\\
  \getwellconstrainedstructure(\tenv, \vtone) \typearrow \TInt(\wellconstrained(\vcsone))\\
  \getwellconstrainedstructure(\tenv, \vttwo) \typearrow \TInt(\wellconstrained(\vcstwo))\\
  \tododefine{constraints\_is\_non\_negative}(\vcstwo) \typearrow \vb\\
  \checktrans{\vb}{ShifterMustBeNonNegative} \checktransarrow \True \OrTypeError\\
  \tododefine{constraints\_binop}(\tenv, \op, \vcsone, \vcstwo) \typearrow \vcs
}{
  \CheckBinop(\tenv, \op, \vtone, \vttwo) \typearrow \TInt(\wellconstrained(\vcs))
}
\end{mathpar}

\begin{mathpar}
\inferrule[plus\_minus\_mul\_real]{
  \op \in  \{\PLUS, \MINUS, \MUL\}\\
  \tstruct(\tenv, \vtone) \typearrow \TReal\\
  \tstruct(\tenv, \vttwo) \typearrow \TReal
}{
  \CheckBinop(\tenv, \op, \vtone, \vttwo) \typearrow \TReal
}
\end{mathpar}

\begin{mathpar}
\inferrule[pow\_real\_int]{
  \tstruct(\tenv, \vtone) \typearrow \TReal\\
  \astlabel(\tstruct(\tenv, \vttwo)) \typearrow \TInt
}{
  \CheckBinop(\tenv, \POW, \vtone, \vttwo) \typearrow \TReal
}
\end{mathpar}

\begin{mathpar}
\inferrule[rdiv]{
  \checktypesat(\tenv, \vtone, \TReal) \typearrow \True \OrTypeError\\
  \checktypesat(\tenv, \vttwo, \TReal) \typearrow \True \OrTypeError\\
}{
  \CheckBinop(\tenv, \RDIV, \vtone, \vttwo) \typearrow \TReal
}
\end{mathpar}
\end{emptyformal}

\subsection{Comments}
  This is related to \identr{BKNT}, \identr{ZYWY}, \identr{BZKW},
  \identr{KFYS}, \identr{KXMR}, \identr{SQXN}, \identr{MRHT}, \identr{JGWF},
  \identr{TTGQ}, \identi{YHML}, \identi{YHRP}, \identi{VMZF}, \identi{YXSY},
  \identi{LGHJ}, \identi{RXLG}.

\section{TypingRule.FindNamedLCA \label{sec:Typing.FindNamedLCA}}
\hypertarget{def-namedlowestcommonancestor}{}
The function
\[
  \namedlca(\overname{\staticenvs}{\tenv} \aslsep \overname{\ty}{\vt} \aslsep \overname{\ty}{\vs})
  \aslto \overname{\ty}{\tty} \cup \overname{\TTypeError}{\TypeErrorConfig}
\]
returns the lowest common named super type --- $\tty$ --- of the types $\vt$ and $\vs$ in $\tenv$.

\newcommand\supers[0]{\texttt{supers}}
The helper function
\[
  \supers(\overname{\staticenvs}{\tenv} \aslsep \overname{\ty}{\vt})
  \aslto \pow{\ty}
\]
returns the set of \emph{named supertypes} given via the $\subtypes$ function of the global environment:
\[
  \supers(\tenv, \vt) \triangleq
  \begin{cases}
    \{\vt\} \cup (\vs) & \text{ if }G^\tenv.\subtypes(\vt) = \vs\\
    \{\vt\}  & \text{ otherwise } (G^\tenv.\subtypes(\vt) = \bot)\\
  \end{cases}
\]

\subsection{Prose}
One of the following holds:
\begin{itemize}
  \item $\vtsupers$ is in the set of named supertypes of $\vt$;
  \item All of the following hold (\textsc{found}):
  \begin{itemize}
    \item $\vs$ is in $\vtsupers$;
    \item $\tty$ is $\vs$;
  \end{itemize}

  \item All of the following hold (\textsc{super}):
  \begin{itemize}
    \item $\vs$ is not in $\vtsupers$;
    \item $\vs$ has a named super type in $\tenv$ --- $\vsp$;
    \item $\tty$ is the lowest common named supertype of $\vt$ and $\vsp$ in $\tenv$.
  \end{itemize}

  \item All of the following hold (\textsc{none}):
  \begin{itemize}
    \item $\vs$ is not in $\vtsupers$;
    \item $\vs$ has no named super type in $\tenv$;
    \item $\tty$ is $\None$.
  \end{itemize}
\end{itemize}

\begin{emptyformal}
\subsection{Formally}
\begin{mathpar}
\inferrule[found]{
  \supers(\tenv, \vt) \typearrow \vtsupers\\
  \vs \in \vtsupers
}{
  \namedlca(\tenv, \vt, \vs) \typearrow \vs
}
\and
\inferrule[super]{
  \supers(\tenv, \vt) \typearrow \vtsupers\\
  \vs \not\in \vtsupers\\
  G^\tenv.\subtypes(\vs) = \vsp\\
  \namedlca(\tenv, \vt, \vsp) \typearrow \tty
}{
  \namedlca(\tenv, \vt, \vs) \typearrow \tty
}
\and
\inferrule[none]{
  \supers(\tenv, \vt) \typearrow \vtsupers\\
  \vs \not\in \vtsupers\\
  G^\tenv.\subtypes(\vs) = \None
}{
  \namedlca(\tenv, \vt, \vs) \typearrow \None
}
\end{mathpar}
\end{emptyformal}

%%%%%%%%%%%%%%%%%%%%%%%%%%%%%%%%%%%%%%%%%%%%%%%%%%%%%%%%%%%%%%%%%%%%%%%%%%%%%%%%
\chapter{Typing of types \label{chap:typingoftypes}}
%%%%%%%%%%%%%%%%%%%%%%%%%%%%%%%%%%%%%%%%%%%%%%%%%%%%%%%%%%%%%%%%%%%%%%%%%%%%%%%%
\hypertarget{def-annotatetype}{}
The function
\[
  \annotatetype{\overname{\Bool}{\decl} \aslsep \overname{\staticenvs}{\tenv} \aslsep \overname{\ty}{\tty}}
  \aslto \overname{\ty}{\newty} \cup \overname{\TTypeError}{\TypeErrorConfig}
\]
annotates a type $\tty$ in an environment $\tenv$, resulting in a
annotated type $\newty$.
The flag $\decl$ indicates whether $\tty$ is a tyoe currently being declared.
The result is a type error, if one is detected.

One of the following applies:
\begin{itemize}
  \item TypingRule.TString (see \secref{TypingRule.TString});
  \item TypingRule.TReal (see \secref{TypingRule.TReal});
  \item TypingRule.TBool (see \secref{TypingRule.TBool});
  \item TypingRule.TNamed (see \secref{TypingRule.TNamed});
  \item TypingRule.TInt (see \secref{TypingRule.TInt});
  \item TypingRule.TBits (see \secref{TypingRule.TBits});
  \item TypingRule.TTuple (see \secref{TypingRule.TTuple});
  \item TypingRule.TArray (see \secref{TypingRule.TArray});
  \item TypingRule.TEnumDecl (see \secref{TypingRule.TEnumDecl});
  \item TypingRule.TRecordExceptionDecl (see
    \secref{TypingRule.TRecordExceptionDecl});
  \item TypingRule.TNonDecl (see \secref{TypingRule.TNonDecl});
\end{itemize}
A type error is returned, if one is detected.

\section{TypingRule.TString \label{sec:TypingRule.TString}}

\subsection{Prose}
All of the following apply:
\begin{itemize}
  \item $\tty$ is the string type \TString.
  \item $\newty$ is the string type \TString.
\end{itemize}

\subsection{Example}
In the following example, all the uses of \texttt{string} are valid:
\VerbatimInput{../tests/ASLTypingReference.t/TypingRule.TString.asl}

\CodeSubsection{\TStringBegin}{\TStringEnd}{../Typing.ml}

\begin{emptyformal}
\subsection{Formally}
\begin{mathpar}
\inferrule{}
{
  \annotatetype{\Ignore, \tenv, \TString} \typearrow \TString
}
\end{mathpar}
\end{emptyformal}

\isempty{\subsection{Comments}}

\section{TypingRule.TReal \label{sec:TypingRule.TReal}}

\subsection{Prose}
All of the following apply:
\begin{itemize}
  \item $\tty$ is the real type \TReal.
  \item $\newty$ is the real type \TReal.
\end{itemize}

\subsection{Example}
In the following example, all the uses of \texttt{real} are valid:
\VerbatimInput{../tests/ASLTypingReference.t/TypingRule.TReal.asl}

\begin{emptyformal}
\subsection{Formally}
\begin{mathpar}
\inferrule{}
{
  \annotatetype{\Ignore, \tenv, \TReal} \typearrow \TReal
}
\end{mathpar}
\end{emptyformal}

\CodeSubsection{\TRealBegin}{\TRealEnd}{../Typing.ml}

\isempty{\subsection{Comments}}

\section{TypingRule.TBool \label{sec:TypingRule.TBool}}

\subsection{Prose}
All of the following apply:
\begin{itemize}
  \item $\tty$ is the boolean type, \TBool;
  \item $\newty$ is the boolean type, \TBool.
\end{itemize}

\subsection{Example}
In the following example, all the uses of \texttt{boolean} are valid:
\VerbatimInput{../tests/ASLTypingReference.t/TypingRule.TBool.asl}

\CodeSubsection{\TBoolBegin}{\TBoolEnd}{../Typing.ml}

\begin{emptyformal}
\subsection{Formally}
\begin{mathpar}
\inferrule{}
{
  \annotatetype{\Ignore, \tenv, \TBool} \typearrow \TBool
}
\end{mathpar}
\end{emptyformal}

\isempty{\subsection{Comments}}

\section{TypingRule.TNamed \label{sec:TypingRule.TNamed}}

\subsection{Prose}
All of the following apply:
\begin{itemize}
  \item $\tty$ is the named type $\vx$, that is $\TNamed(\vx)$;
  \item retrieving the type associated with $\vx$ from the static environment $\tenv$ \\ (via $\declaredtype$) yields $\vt$ \ProseOrTypeError;
  \item $\newty$ is $\tty$.
\end{itemize}

\subsection{Example}
In the following example, all the uses of \texttt{MyType} are valid:
\VerbatimInput{../tests/ASLTypingReference.t/TypingRule.TNamed.asl}

\CodeSubsection{\TNamedBegin}{\TNamedEnd}{../Typing.ml}

\begin{emptyformal}
\subsection{Formally}
\begin{mathpar}
\inferrule{
  \declaredtype(\tenv, \vx) \typearrow \vt \OrTypeError
}{
  \annotatetype{\Ignore, \tenv, \TNamed(\vx)} \typearrow \TNamed(\vx)
}
\end{mathpar}
\end{emptyformal}

\isempty{\subsection{Comments}}

\section{TypingRule.TInt \label{sec:TypingRule.TInt}}

\subsection{Prose}
One of the following applies:
\begin{itemize}
  \item All of the following apply:
    \begin{itemize}
      \item $\tty$ is the unconstrained integer type;
      \item $\newty$ is the unconstrained integer type.
    \end{itemize}
  \item All of the following apply:
    \begin{itemize}
      \item $\tty$ is a under-constrained integer type;
      \item $\newty$ is the under-constrained integer type $\tty$.
    \end{itemize}
  \item All of the following apply:
    \begin{itemize}
      \item $\tty$ is the well-constrained integer type constrained by
        constraints $\vc_i$, for $u=1..k$;
      \item annotating each constraint $\vc_i$, for $i=1..k$,
      yields $\newc_i$ \ProseOrTypeError;
      \item $\newconstraints$ is the list of annotated constraints $\newc_i$,
      for $i=1..k$;
      \item $\newty$ is the well-constrained integer type constrained
        by $\newconstraints$.
    \end{itemize}
\end{itemize}

\subsection{Example}

In the following examples, all the uses of integer types are valid:
\VerbatimInput{../tests/ASLTypingReference.t/TypingRule.TIntUnConstrained.asl}
\VerbatimInput{../tests/ASLTypingReference.t/TypingRule.TIntWellConstrained.asl}
\VerbatimInput{../tests/ASLTypingReference.t/TypingRule.TIntUnderConstrained.asl}

\CodeSubsection{\TIntBegin}{\TIntEnd}{../Typing.ml}

\begin{emptyformal}
\subsection{Formally}
\begin{mathpar}
\inferrule[unconstrained]{
  \tty \eqname \TInt(\unconstrained)\\
}{
  \annotatetype{\Ignore, \tenv, \tty} \typearrow \tty
}
\and
\inferrule[underconstrained]{
  \tty \eqname \TInt(\underconstrained(\Ignore))\\
}{
  \annotatetype{\Ignore, \tenv, \tty} \typearrow \tty
}
\and
\inferrule[well\_constrained]{
  \constraints \eqname \vc_{1..k}\\
  i=1..k: \tododefine{annotate\_constraint}(\vc_i) \typearrow\newc_i \OrTypeError\\\\
  \newconstraints \eqdef \newc_{1..k}
}{
  \annotatetype{\Ignore, \tenv, \TInt(\wellconstrained(\constraints))} \typearrow \\ \TInt(\wellconstrained(\newconstraints))
}
\end{mathpar}
\end{emptyformal}

\isempty{\subsection{Comments}}

\section{TypingRule.TBits \label{sec:TypingRule.TBits}}

\subsection{Prose}
All of the following apply:
\begin{itemize}
  \item $\tty$ is the bit-vector type with width given by the expression
    $\ewidth$ and the bitfields given by $\bitfields$, that is, $\TBits(\ewidth, \bitfields)$;
  \item annotating the \staticallyevaluable\  integer expression $\ewidth$ yields $\ewidthp$ \ProseOrTypeError;
  \item annotating the bitfields $\bitfields$ yields $\bitfieldsp$ \ProseOrTypeError;
  \item $\newty$ is the bit-vector type with width given by the expression
    $\ewidthp$ and the bitfields given by $\bitfieldsp$, that is, $\TBits(\ewidthp, \bitfieldsp)$
\end{itemize}

\subsection{Example}
\VerbatimInput{../tests/ASLTypingReference.t/TypingRule.TBits.asl}

In the following example, all the uses of bitvector types are valid:
\CodeSubsection{\TBitsBegin}{\TBitsEnd}{../Typing.ml}

\begin{emptyformal}
\subsection{Formally}
\begin{mathpar}
\inferrule{
  \annotatestaticconstrainedinteger(\tenv, \ewidth) \typearrow \ewidthp \OrTypeError\\
  \annotatebitfields(\tenv, \ewidthp, \bitfields) \typearrow \bitfieldsp \OrTypeError
}{
  \annotatetype{\Ignore, \tenv, \TBits(\ewidth, \bitfields)} \typearrow \TBits(\ewidthp, \bitfieldsp)
}
\end{mathpar}
\end{emptyformal}

\isempty{\subsection{Comments}}

\section{TypingRule.TTuple \label{sec:TypingRule.TTuple}}

\subsection{Prose}
All of the following apply:
\begin{itemize}
  \item $\tty$ is the tuple type with member types $\tys$, that is, $\TTuple(\tys)$;
  \item $\tys$ is the list $\tty_i$, for $i=1..k$;
  \item annotating each type $\tty_i$ in $\tenv$, for $i=1..k$,
  yields $\ttyp_i$ \ProseOrTypeError;
  \item $\newty$ is the tuple type with member types $\ttyp$, for $i=1..k$.
\end{itemize}

\subsection{Example}
In the following example, all the uses of tuple types are valid:
\VerbatimInput{../tests/ASLTypingReference.t/TypingRule.TTuple.asl}

\CodeSubsection{\TTupleBegin}{\TTupleEnd}{../Typing.ml}

\begin{emptyformal}
\subsection{Formally}
\begin{mathpar}
\inferrule{
  k \geq 2\\
  \tys \eqname \tty_{1..k}\\
  i=1..k: \annotatetype{\False, \tenv, \tty_i} \typearrow \ttyp_i \OrTypeError
}{
  \annotatetype{\Ignore, \tenv, \TTuple(\tys)} \typearrow \TTuple(\tysp)
}
\end{mathpar}
\end{emptyformal}

\isempty{\subsection{Comments}}

\section{TypingRule.TArray \label{sec:TypingRule.TArray}}

\subsection{Prose}
All of the following apply:
\begin{itemize}
  \item $\tty$ is the array type indexed by integer bounded by the
    expression $\ve$ and of elements of type $\vt$;
  \item Annotating the type $\vt$ in $\tenv$ yields $\vtp$ \ProseOrTypeError;
  \item One of the following applies:
  \begin{itemize}
    \item All of the following apply (\textsc{expr\_is\_enum}):
    \begin{itemize}
      \item determining whether $\ve$ corresponds to an enumeration in $\tenv$
      via \\ $\tododefine{get\_variable\_enum}$ yields the enumeration variable
      name $\vs$ of size $\vi$, that is, $\langle \vs, \vi \rangle$;
      \item $\newty$ is the array type indexed by an enumeration type
      named $\vs$ of length $\vi$ and of elements of type $\vtp$, that is, $\TArray(\ArrayLengthEnum(\vs, \vi), \vtp)$.
    \end{itemize}

    \item All of the following apply (\textsc{expr\_not\_enum}):
    \begin{itemize}
      \item determining whether $\ve$ corresponds to an enumeration in $\tenv$
      via \\ $\tododefine{get\_variable\_enum}$ yields $\None$ (meaning it does not
      correspond to an enumeration);
      \item annotating the statically evaluable integer expression $\ve$ yields
      $\vep$ \ProseOrTypeError;
      \item $\newty$ the array type indexed by integer bounded by
      the expression $\vep$ and of elements of type $\vtp$, that is,
      $\TArray(\ArrayLengthExpr(\vep), \vtp)$.
    \end{itemize}

    \item All of the following apply (\textsc{index\_enum}):
    \begin{itemize}
      \item $\ve$ is an enumeration type index with variable $\vs$ and size $\vi$,
      that is, \\ $\ArrayLengthEnum(\vs, \vi)$;
      \item let $\tty$ be the named type defined for $\vs$, that is, $\TNamed(\vs)$;
      \item determining the \underlyingtype\ of $\tty$ yields $\vt$ \ProseOrTypeError;
      \item checking whether $\vt$ is an enumeration type yields $\True$ \ProseOrTypeError;
      \item $\vt$ is an enumeration with labels $\vli$;
      \item checking whether $\vli$ has the same length as $\vi$ yields $\True$ \ProseOrTypeError;
      \item $\newty$ is the array type indexed by an enumeration type
      named $\vs$ of length $\vi$ and of elements of type $\vtp$, that is, $\TArray(\ArrayLengthEnum(\vs, \vi), \vtp)$.
    \end{itemize}
  \end{itemize}
\end{itemize}

\subsection{Example}
In the following example, all the uses of array types are valid:
\VerbatimInput{../tests/ASLTypingReference.t/TypingRule.TArray.asl}

\CodeSubsection{\TArrayBegin}{\TArrayEnd}{../Typing.ml}

\begin{emptyformal}
\subsection{Formally}
\begin{mathpar}
\inferrule[expr\_is\_enum]{
  \annotatetype{\False, \tenv, \vt} \typearrow \vtp \OrTypeError\\\\
  \tododefine{get\_variable\_enum}(\tenv, \ve) \typearrow \langle \vs, \vi \rangle
}{
  \annotatetype{\Ignore, \tenv, \TArray(\ArrayLengthExpr(\ve), \vt)} \typearrow \TArray(\ArrayLengthEnum(\vs, \vi), \vtp)
}
\and
\inferrule[expr\_not\_enum]{
  \annotatetype{\tenv, \vt} \typearrow \vtp \OrTypeError\\\\
  \tododefine{get\_variable\_enum}(\tenv, \ve) \typearrow \None\\
  \tododefine{annotate\_static\_integer}(\tenv, \ve) \typearrow \vep \OrTypeError
}{
  \annotatetype{\Ignore, \tenv, \TArray(\ArrayLengthExpr(\ve), \vt)} \typearrow \TArray(\ArrayLengthExpr(\vep), \vtp)
}
\and
\inferrule[index\_enum]{
  \annotatetype{\False, \tenv, \vt} \typearrow \vtp \OrTypeError\\\\
  \tsubs \eqdef \TNamed(\vs)\\
  \makeanonymous(\tenv, \tsubs) \typearrow \vt \OrTypeError\\\\
  \checktrans{\astlabel(\vt) = \TEnum}{ExpectedEnumeration} \checktransarrow \True \OrTypeError\\\\
  \vt \eqname \TEnum((\vli))\\
  \checktrans{\equallength(\vli, \vi)}{TypeConflict} \checktransarrow \True \OrTypeError
}{
  \annotatetype{\Ignore, \tenv, \TArray(\ArrayLengthEnum(\vs, \vi), \vt)} \typearrow \\
  \TArray(\ArrayLengthEnum(\vs, \vi), \vtp)
}
\end{mathpar}
\end{emptyformal}

\isempty{\subsection{Comments}}

\section{TypingRule.TEnumDecl \label{sec:TypingRule.TEnumDecl}}

\subsection{Prose}
All of the following apply:
\begin{itemize}
  \item $\tty$ is the enumeration type with enumeration literals
    $\vli$, that is, $\TEnum(\vli)$;
  \item $\decl$ is $\True$, indicating that $\tty$ should be considered in the context of a declaration;
  \item determining that $\vli$ does not contain duplicates yields $\True$ \ProseOrTypeError;
  \item determining that none of the labels in $\vli$ is declared in the global environment
  yields $\True$ \ProseOrTypeError;
  \item $\newty$ is the enumeration type $\tty$.
\end{itemize}

\subsection{Example}
In the following example, all the uses of enumeration types are valid:
\VerbatimInput{../tests/ASLTypingReference.t/TypingRule.TEnumDecl.asl}

\CodeSubsection{\TEnumDeclBegin}{\TEnumDeclEnd}{../Typing.ml}

\begin{emptyformal}
\subsection{Formally}
\begin{mathpar}
\inferrule{
  \checknoduplicates(\vli) \typearrow \True \OrTypeError\\
  \vl \in \vli: \tododefine{check\_var\_not\_in\_genv}(\tenv, \vl) \typearrow \True \OrTypeError
}{
  \annotatetype{\True, \tenv, \TEnum(\vli)} \typearrow \TEnum(\vli)
}
\end{mathpar}
\end{emptyformal}

\subsection{Comments}
This is related to \identd{YZBQ}, \identr{DWSP}, \identi{MZXL}.

\section{TypingRule.TRecordExceptionDecl \label{sec:TypingRule.TRecordExceptionDecl}}

\subsection{Prose}
All of the following apply:
\begin{itemize}
  \item $\tty$ is either a record type or an exception type, corresponding to its AST label $L$;
  \item the list of fields of $\tty$ is $\fields$;
  \item $\decl$ is $\True$, indicating that $\tty$ should be considered in the context of a declaration;
  \item $\fields$ is a list of pairs where the first element is an identifier and the second is a type --- $(\vx_i, \vt_i)$, for $i=1..k$;
  \item checking that the list of field identifiers $\vx_{1..k}$ does not contain duplicates
  yields $\True$ \ProseOrTypeError;
  \item annotating each field type $\vt_i$, for $i=1..k$, yields an annotated type $\vtp_i$
  \ProseOrTypeError;
  \item $\fieldsp$ is the list with $(\vx_i, \vtp_i)$, for $i=1..k$;
  \item $\newty$ is the AST node with AST label $L$ (either record type or exception type,
  corresponding to the type $\tty$) and fields $\fieldsp$.
\end{itemize}

\subsection{Example}
In the following example, all the uses of record or exception types are valid:
\VerbatimInput{../tests/ASLTypingReference.t/TypingRule.TRecordExceptionDecl.asl}

\CodeSubsection{\TRecordExceptionDeclBegin}{\TRecordExceptionDeclEnd}{../Typing.ml}

\begin{emptyformal}
\subsection{Formally}
\begin{mathpar}
\inferrule{
  \fields \eqname [i=1..k: (\vx_i, \vt_i)]\\
  \checknoduplicates(\vx_{1..k}) \typearrow \True \OrTypeError\\
  i=1..k: \annotatetype{\False, \tenv, \vt_i} \typearrow \vtp_i \OrTypeError\\
  \fieldsp \eqdef [i=1..k: (\vx_i, \vtp_i)]
}{
  \annotatetype{\True, \tenv, L(\fields)} \typearrow L(\fieldsp)
}
\end{mathpar}
\end{emptyformal}

\isempty{\subsection{Comments}}


\section{TypingRule.TNonDecl \label{sec:TypingRule.TNonDecl}}

\subsection{Prose}
All of the following apply:
\begin{itemize}
  \item $\tty$ is either a record type, an exception type, or an enumeration type;
  \item $\decl$ is $\False$, indicating that $\tty$ should be considered to be outside the context of a declaration
  of $\tty$;
  \item a type error is returned, indicating that the use of anonymous form of enumerations, record,
  and exceptions types is not allowed here.
\end{itemize}

\subsection{Example}

In the following example, the use of a record type outside of a declaration is erroneous:
\VerbatimInput{../tests/ASLTypingReference.t/TypingRule.TNonDecl.asl}

\CodeSubsection{\TNonDeclBegin}{\TNonDeclEnd}{../Typing.ml}

\begin{emptyformal}
\subsection{Formally}
\begin{mathpar}
\inferrule{
  \astlabel(\tty) \in \{\TEnum, \TRecord, \TException\}
}{
  \annotatetype{\False, \tenv, \tty} \typearrow \TypeErrorVal{AnnonymousFormNotAllowedHere}
}
\end{mathpar}
\end{emptyformal}

\isempty{\subsection{Comments}}

%%%%%%%%%%%%%%%%%%%%%%%%%%%%%%%%%%%%%%%%%%%%%%%%%%%%%%%%%%%%%%%%%%%%%%%%%%%%%%%%%%%%
\chapter{Typing of Bitfields}
%%%%%%%%%%%%%%%%%%%%%%%%%%%%%%%%%%%%%%%%%%%%%%%%%%%%%%%%%%%%%%%%%%%%%%%%%%%%%%%%%%%%

We define rules for annotating a single bitfield and a list of bitfields:
\begin{itemize}
  \item TypingRule.TBitField (see \secref{TypingRule.TBitField});
  \item TypingRule.TBitFields (see \secref{TypingRule.TBitFields});
\end{itemize}

\section{TypingRule.TBitField \label{sec:TypingRule.TBitField}}
\hypertarget{def-annotatebitfield}{}
The function
\[
  \annotatebitfield(\overname{\staticenvs}{\tenv} \aslsep \overname{\Z}{\width} \aslsep \overname{\bitfield}{\field})
  \aslto \overname{\bitfield}{\newfield} \cup \overname{\TTypeError}{\TypeErrorConfig}
\]
annotates a bitfield --- $\field$ --- with an integer --- $\width$ --- indicating the number of bits in
the bitvector type that contains $\field$,
in an environment $\tenv$, resulting in an
annotated bitfield --- $\newfield$ --- or a type error, if one is detected.

\subsection{Prose}
\begin{itemize}
  \item Annotating the slices $\slices$ yields $\slicesone$ \ProseOrTypeError;
  One of the following applies:
  \begin{itemize}
    \item All of the following apply (\textsc{simple}):
    \begin{itemize}
      \item checking whether the range of positions in $\slicesone$ fit inside $0..\width$ yields $\True$ \ProseOrTypeError;
      \item $\newfield$ is a bitfield named $\name$ with list of slices $\slicesone$, that is, $\BitFieldSimple(\name, \sliceone)$.
    \end{itemize}

    \item All of the following apply (\textsc{nested}):
    \begin{itemize}
      \item converting the $\slicesone$ into a list of positions with $\width$ and static environment $\tenv$
      yields $\positions$ \ProseOrTypeError;
      \item checking that all positions in $\positions$ fit inside $0..\width$ yields $\True$ \ProseOrTypeError;
      \item let $\widthp$ be the length of the list $\positions$;
      \item annotating the bitfields $\bitfieldsp$ with $\widthp$ in static environment $\tenv$ yields $\bitfieldspp$ \ProseOrTypeError;
      \item $\newfields$ is the nested bitfield with $\slicesone$ and bitfields $\bitfieldspp$, that is, $\BitFieldNested(\slicesone, \bitfieldspp)$.
    \end{itemize}

    \item All of the following apply (\textsc{type}):
    \begin{itemize}
      \item Annotating the type $\vt$ yields $\vtp$ \ProseOrTypeError;
      \item checking whether the range of positions in $\slicesone$ fit inside $0..\width$ yields $\True$ \ProseOrTypeError;
      \item converting the list of slices $\slicesone$ into a list of positions in $\tenv$ yields $\positions$ \ProseOrTypeError;
      \item checking that all positions in $\positions$ fit inside $0..\width$ yields $\True$ \ProseOrTypeError;
      \item let $\widthp$ be the length of the list $\positions$;
      \item checking whether the $\vt$ and the bitvector with $\widthp$ bits have the same width yields $\True$ \ProseOrTypeError
      \item $\newfield$ is the typed-bitfield with name $\name$, list of slices $\slicesone$ and type $\vtp$, that is, \BitFieldType(\name, \slicesone, \vtp).
    \end{itemize}
  \end{itemize}
\end{itemize}

\subsection{Example}
In the following example, all the uses of bitvector types with bitfields are valid:
\VerbatimInput{../tests/ASLTypingReference.t/TypingRule.TBitField.asl}

\CodeSubsection{\TBitFieldBegin}{\TBitFieldEnd}{../Typing.ml}

\begin{emptyformal}
\subsection{Formally}
\begin{mathpar}
\inferrule[simple]{
  \annotateslices(\tenv, \slices) \typearrow \slicesone \OrTypeError\\
  \tododefine{check\_slices\_in\_width}(\tenv, \width, \slicesone) \typearrow \True \OrTypeError
}{
  \annotatebitfield(\tenv, \width, \BitFieldSimple(\name, \slices)) \typearrow \\
  \BitFieldSimple(\name, \sliceone)
}
\end{mathpar}

\begin{mathpar}
\inferrule[nested]{
  \annotateslices(\tenv, \slices) \typearrow \slicesone \OrTypeError\\
  \tododefine{disjoint\_slices\_to\_positions}(\tenv, \width, \slicesone) \typearrow \positions \OrTypeError\\\\
  \tododefine{check\_positions\_in\_width}(\tenv, \width, \positions) \typearrow \True \OrTypeError\\
  \widthp \eqdef |\positions|\\
  \annotatebitfields(\tenv, \widthp, \bitfieldsp) \typearrow \bitfieldspp \OrTypeError\\
}{
  \annotatebitfield(\tenv, \width, \BitFieldNested(\name, \slices, \bitfieldsp)) \typearrow \\
  \BitFieldNested(\slicesone, \bitfieldspp)
}
\end{mathpar}

\begin{mathpar}
\inferrule[type]{
  \annotateslices(\tenv, \slices) \typearrow \slicesone \OrTypeError\\
  \annotatetype{\tenv, \vt} \typearrow \vtp \OrTypeError\\
  \tododefine{check\_slices\_in\_width}(\tenv, \width, \slicesone) \typearrow \True \OrTypeError\\
  \tododefine{disjoint\_slices\_to\_positions}(\tenv, \slicesone) \typearrow \positions \OrTypeError\\\\
  \tododefine{check\_positions\_in\_width}(\tenv, \slicesone, \width, \positions) \typearrow \True \OrTypeError\\
  \widthp \eqdef |\positions|\\
  \checkbitsequalwidth(\TBits(\widthp, \emptylist), \vt) \typearrow \True \OrTypeError
}{
  \annotatebitfield(\tenv, \width, \BitFieldType(\name, \slices, \vt)) \typearrow \\
  \BitFieldType(\name, \slicesone, \vtp)
}
\end{mathpar}
\end{emptyformal}

\isempty{\subsection{Comments}}

\section{TypingRule.TBitFields \label{sec:TypingRule.TBitFields}}
\hypertarget{def-annotatebitfields}{}
The function
\[
  \annotatebitfields(\overname{\staticenvs}{\tenv} \aslsep \overname{\expr}{\ewidth} \aslsep \overname{\bitfields}{\fields})
  \aslto \overname{\bitfields}{\newfields} \cup \overname{\TTypeError}{\TypeErrorConfig}
\]
annotates a list of bitfields --- $\fields$ --- with an expression denoting the overall number of bits in the containing
bitvector type --- $\ewidth$,
in an environment $\tenv$, resulting in an
annotated list of bitfields --- $\newfields$ or a type error, if one is detected.

\subsection{Prose}
All of the following apply:
\begin{itemize}
  \item checking that the list of bitfield names in $\bitfields$ does not contain duplicates yields $\True$ \ProseOrTypeError;
  \item symbolically reducing $\ewidth$ in $\tenv$ yields the literal integer for $\vi$ \ProseOrTypeError;
  \item annotating each bitfield $\field$ in $\fields$ with width $\width$ in $\tenv$ yields the corresponding annotated
  bitfield $\newfield$ \ProseOrTypeError;
  \item $\newfields$ is the list of annotated bitfields.
\end{itemize}

\isempty{\subsection{Example}}

\CodeSubsection{\TBitFieldsBegin}{\TBitFieldsEnd}{../Typing.ml}

\begin{emptyformal}
\subsection{Formally}
\begin{mathpar}
\inferrule{
  \names \eqdef [\field\in\fields: \bitfieldgetname(\field)]\\
  \checknoduplicates(\names) \typearrow \True \OrTypeError\\
  \reduceconstants{\tenv, \ewidth} \typearrow \lint(\vi) \OrTypeError\\
  \field\in\fields: \annotatebitfield(\tenv, \width, \field) \typearrow \newfield \OrTypeError\\
  \newfields \eqdef [\field\in\fields: \newfield]
}
{
  \annotatebitfields(\tenv, \ewidth, \fields) \typearrow \newfields
}
\end{mathpar}
\end{emptyformal}

\isempty{\subsection{Comments}}

%%%%%%%%%%%%%%%%%%%%%%%%%%%%%%%%%%%%%%%%%%%%%%%%%%%%%%%%%%%%%%%%%%%%%%%%%%%%%%%%%%%%
\chapter{Typing of Expressions}
%%%%%%%%%%%%%%%%%%%%%%%%%%%%%%%%%%%%%%%%%%%%%%%%%%%%%%%%%%%%%%%%%%%%%%%%%%%%%%%%%%%%
\hypertarget{def-annotateexpr}{}
The function
\[
  \annotateexpr{\overname{\staticenvs}{\tenv} \aslsep \overname{\expr}{\ve}}
  \aslto (\overname{\ty}{\vt} \times \overname{\expr}{\newe})
  \cup \overname{\TTypeError}{\TypeErrorConfig}
\]
specifies how to annotate an expression $\ve$ in
an environment \tenv.  Formally, the result of annotating the expression
$\ve$ in \tenv\ is either the pair $(\vt, \newe)$, where $\vt$ is a type and
$\newe$ is an annotated expression, or a type error, and one of the following applies:
\begin{itemize}
\item TypingRule.ELit (see \secref{TypingRule.ELit});
\item TypingRule.ELocalVarConstant (see \secref{TypingRule.ELocalVarConstant})
\item TypingRule.ELocalVar (see \secref{TypingRule.ELocalVar})
\item TypingRule.EGlobalVarConstant (see \secref{TypingRule.EGlobalVarConstant})
\item TypingRule.EGlobalVarConstantNoVal (see \secref{TypingRule.EGlobalVarConstantNoVal})
\item TypingRule.EGlobalVar (see \secref{TypingRule.EGlobalVar})
\item TypingRule.EUndefIdent (see \secref{TypingRule.EUndefIdent})
\item TypingRule.Binop (see \secref{TypingRule.Binop})
\item TypingRule.Unop (see \secref{TypingRule.Unop})
\item TypingRule.ECond (see \secref{TypingRule.ECond})
\item TypingRule.ESlice (see \secref{TypingRule.ESlice})
\item TypingRule.ESetter (see \secref{TypingRule.ESetter})
\item TypingRule.ECall (see \secref{TypingRule.ECall})
\item TypingRule.EGetArray (see \secref{TypingRule.EGetArray})
\item TypingRule.ESliceOrEGetArrayError (see \secref{TypingRule.ESliceOrEGetArrayError})
\item TypingRule.EStructuredNotStructured (see \secref{TypingRule.EStructuredNotStructured})
\item TypingRule.EStructuredMissingField (see \secref{TypingRule.EStructuredMissingField})
\item TypingRule.ERecord (see \secref{TypingRule.ERecord})
\item TypingRule.EGetRecordField (see \secref{TypingRule.EGetRecordField})
\item TypingRule.EGetBadRecordField (see \secref{TypingRule.EGetBadRecordField})
\item TypingRule.EGetBadBitField (see \secref{TypingRule.EGetBadBitField})
\item TypingRule.EGetBitField (see \secref{TypingRule.EGetBitField})
\item TypingRule.EGetBitFieldNested (see \secref{TypingRule.EGetBitFieldNested})
\item TypingRule.EGetBitFieldTyped (see \secref{TypingRule.EGetBitFieldTyped})
\item TypingRule.EGetTupleItem (see \secref{TypingRule.EGetTupleItem})
\item TypingRule.EGetBadField (see \secref{TypingRule.EGetBadField})
\item TypingRule.EConcat (see \secref{TypingRule.EConcat})
\item TypingRule.ETuple (see \secref{TypingRule.ETuple})
\item TypingRule.EUnknown (see \secref{TypingRule.EUnknown})
\item TypingRule.EPattern (see \secref{TypingRule.EPattern})
\item TypingRule.ATC (see \secref{TypingRule.ATC})
\end{itemize}

The annotation rewrites the input expression in the following cases, making the annotation of statements simpler:
\begin{itemize}
  \item Variables with constant values are substituted by their constant values.
  \item Slicing expressions that correspond to calling a getter are replaced with respective call expressions.
  \item Slicing expressions that correspond to array accesses are replaced with array access expressions.
\end{itemize}

We also define the following helper rules:
\begin{itemize}
  \item TypingRule.Lit (\secref{TypingRule.Lit})
  \item TypingRule.ExpressionList (\secref{TypingRule.ExpressionList})
  \item TypingRule.TypingRule.ReduceSlicesToCall (\secref{TypingRule.ReduceSlicesToCall})
  \item TypingRule.StaticConstrainedInteger (\secref{TypingRule.StaticConstrainedInteger})
\end{itemize}

\section{TypingRule.ELit \label{sec:TypingRule.ELit}}
\subsection{Prose}
All of the following apply:
\begin{itemize}
\item $\ve$ is a literal expression $\vv$;
\item $\vt$ is the type of the literal $\vv$;
\item $\newe$ is $\ve$.
\end{itemize}

\subsection{Example}

\CodeSubsection{\ELitBegin}{\ELitEnd}{../Typing.ml}
\begin{emptyformal}
\subsection{Formally}
\begin{mathpar}
\inferrule{\annotateliteral{\vv} \typearrow \vt}
{\annotateexpr{\tenv, \ELiteral(\vv)} \typearrow (\vt, \ELiteral(\vv))}
\end{mathpar}
\end{emptyformal}

\section{TypingRule.ELocalVarConstant \label{sec:TypingRule.ELocalVarConstant}}

\subsection{Prose}
All of the following apply:
\begin{itemize}
\item $\ve$ denotes a local variable $\vx$;
\item $\vx$ is bound to a local constant $\vv$ of type $\tty$ in the local environment given by $\tenv$;
\item $\vt$ is $\tty$;
\item $\newe$ is the Literal $\vv$.
\end{itemize}

\subsection{Example}

\CodeSubsection{\ELocalVarConstantBegin}{\ELocalVarConstantEnd}{../Typing.ml}

\begin{emptyformal}
\subsection{Formally}
\begin{mathpar}
\inferrule{
  L^\tenv.\constantvalues(\vx) = \vv\\
  L^\tenv.\localstoragetypes(\vx) = (\tty, \LDKConstant)
  }
{\annotateexpr{\tenv, \EVar(\vx)} \typearrow (\tty, \eliteral{\vv})}
\end{mathpar}
\end{emptyformal}

\isempty{\subsection{Comments}}

\section{TypingRule.ELocalVar \label{sec:TypingRule.ELocalVar}}

\subsection{Prose}
All of the following apply:
\begin{itemize}
\item $\ve$ denotes a local variable $\vx$;
\item $\vx$ is not bound to a constant in the local environment given by $\tenv$;
\item $\vx$ has type $\tty$ in the local environment given by $\tenv$;
\item $\vt$ is $\tty$;
\item $\newe$ is $\ve$.
\end{itemize}

\subsection{Example}

\CodeSubsection{\ELocalVarBegin}{\ELocalVarEnd}{../Typing.ml}

\begin{emptyformal}
\subsection{Formally}
\begin{mathpar}
\inferrule{
  L^\tenv.\constantvalues(\vx) = \bot\\
  L^\tenv.\localstoragetypes(\vx) =  (\tty, k) \\
  k \in \{\LDKVar, \LDKLet\}
  }
{\annotateexpr{\tenv, \EVar(\vx)} \typearrow (\tty, \EVar(\vx))}
\end{mathpar}
\end{emptyformal}

\isempty{\subsection{Comments}}

\section{TypingRule.EGlobalVarConstantVal \label{sec:TypingRule.EGlobalVarConstant}}

\subsection{Prose}
All of the following apply:
\begin{itemize}
\item $\ve$ denotes a global variable $\vx$;
\item $\vx$ is bound to a constant $\vv$ of type $\tty$ in the global environment given by $\tenv$;
\item $\vt$ is $\tty$;
\item $\newe$ is the Literal $\vv$.
\end{itemize}

\subsection{Example}

\CodeSubsection{\EGlobalVarConstantBegin}{\EGlobalVarConstantEnd}{../Typing.ml}

\begin{emptyformal}
  \subsection{Formally}
\begin{mathpar}
\inferrule{
  G^\tenv.\globalstoragetypes(\vx) = (\tty, \GDKConstant)\\
  G^\tenv.\constantvalues(\vx) = \vv
  }
{\annotateexpr{\tenv, \EVar(\vx)} \typearrow (\tty, \eliteral{\vv})}
\end{mathpar}
\end{emptyformal}

\isempty{\subsection{Comments}}

\section{TypingRule.EGlobalVarConstantNoVal \label{sec:TypingRule.EGlobalVarConstantNoVal}}
Our type system does not currently address assignments of non-constant expressions (for example,
function calls) to global constant variables. This section can be seen as a place holder until
the right details are filled in.

\subsection{Prose}
All of the following apply:
\begin{itemize}
\item $\ve$ denotes a global variable $\vx$;
\item $\vx$ is not bound to constant in the global environment given by $\tenv$;
\item $\vt$ is $\tty$;
\item $\newe$ is $\ve$.
\end{itemize}

\subsection{Example}

\CodeSubsection{\EGlobalVarConstantNoValBegin}{\EGlobalVarConstantNoValEnd}{../Typing.ml}

\begin{emptyformal}
  \subsection{Formally}
\begin{mathpar}
\inferrule{
  G^\tenv.\globalstoragetypes(\vx) = (\tty, \GDKConstant)\\
  G^\tenv.\constantvalues(\vx) = \bot
  }
{\annotateexpr{\tenv, \EVar(\vx)} \typearrow (\tty, \EVar(\vx))}
\end{mathpar}
\end{emptyformal}

\isempty{\subsection{Comments}}

\section{TypingRule.EGlobalVar \label{sec:TypingRule.EGlobalVar}}

\subsection{Prose}
All of the following apply:
\begin{itemize}
\item $\ve$ denotes a global variable $\vx$;
\item $\vx$ is not bound to a global constant;
\item $\vx$ has type $\tty$ in the global environment given by $\tenv$;
\item $\vt$ is $\tty$;
\item $\newe$ is $\ve$.
\end{itemize}

\subsection{Example}

\CodeSubsection{\EGlobalVarBegin}{\EGlobalVarEnd}{../Typing.ml}

\begin{emptyformal}
  \subsection{Formally}
\begin{mathpar}
\inferrule{
  G^\tenv.\constantvalues(\vx) = \bot\\
  G^\tenv.\globalstoragetypes(\vx) = (\tty, k)\\
  k \neq \GDKConstant
  }
{\annotateexpr{\tenv, \EVar(\vx)} \typearrow (\tty, \EVar(\vx))}
\end{mathpar}
\end{emptyformal}

\isempty{\subsection{Comments}}

\section{TypingRule.EUndefIdent \label{sec:TypingRule.EUndefIdent}}

\subsection{Prose}
All of the following apply:
\begin{itemize}
\item $\ve$ is a variable $\vx$;
\item $\vx$ is not bound in $\tenv$;
\item the result is a type error indicating that $\vx$ is an undefined identifier.
\end{itemize}

\subsection{Example}

\CodeSubsection{\EUndefIdentBegin}{\EUndefIdentEnd}{../Typing.ml}

\begin{emptyformal}
\subsection{Formally}
\begin{mathpar}
\inferrule{
  G^\tenv.\globalstoragetypes(\vx) = \bot\\
  L^\tenv.\globalstoragetypes(\vx) = \bot\\
}{
  \annotateexpr{\tenv, \EVar(\vx)} \typearrow \TypeErrorVal{UndefinedIdentifier}
}
\end{mathpar}
\end{emptyformal}

\isempty{\subsection{Comments}}

\section{TypingRule.Binop \label{sec:TypingRule.Binop}}

\subsection{Prose}
All of the following apply:
\begin{itemize}
  \item $\ve$ denotes a binary operation $\op$ over two expressions $\veone$ and $\vetwo$, that is, \\ $\EBinop(\op, \veone, \vetwo)$;
  \item the result of annotating $\veone$ in $\tenv$ is $(\vtone, \veonep)$ \ProseOrTypeError;
  \item the result of annotating $\vetwo$ in $\tenv$ is $(\vttwo, \vetwop)$ \ProseOrTypeError;
  \item the result of checking compatibility of $\op$ with $\vtone$ and $\vttwo$ as per \secref{TypingRule.CheckBinop}
  is $\vt$ \ProseOrTypeError;
  \item $\newenv$ denotes $\op$ over $\veonep$ and $\vetwop$.
\end{itemize}

\subsection{Example}

\CodeSubsection{\BinopBegin}{\BinopEnd}{../Typing.ml}

\begin{emptyformal}
\subsection{Formally}
\begin{mathpar}
\inferrule{
  \annotateexpr{\tenv, \veone} \typearrow (\vtone, \veone') \OrTypeError\\\\
  \annotateexpr{\tenv, \vetwo} \typearrow (\vttwo, \vetwo') \OrTypeError\\\\
  \CheckBinop(\tenv, \op, \vtone, \vttwo) \typearrow \vt \OrTypeError
}{
  \annotateexpr{\tenv, \EBinop(\op, \veone, \vetwo)} \typearrow (\vt, \EBinop(\op, \veone', \vetwo'))
}
\end{mathpar}
\end{emptyformal}

\isempty{\subsection{Comments}}

\section{TypingRule.Unop \label{sec:TypingRule.Unop}}

\subsection{Prose}
All of the following apply:
\begin{itemize}
  \item $\ve$ denotes a unary operation $\op$ over an expression $\vep$, that is $\EUnop(\op, \vep)$;
  \item annotating $\vep$ in $\tenv$ yields $(\vtpp, \vepp)$ \ProseOrTypeError;
  \item checking compatibility of $\op$ with $\vtpp$ as per \secref{TypingRule.CheckUnop} yields $\vt$ \ProseOrTypeError;
  \item $\newe$ denotes $\op$ over $\vepp$, that is, $\EUnop(\op, \vepp)$.
\end{itemize}

\subsection{Example}

\CodeSubsection{\UnopBegin}{\UnopEnd}{../Typing.ml}

\begin{emptyformal}
\subsection{Formally}
\begin{mathpar}
\inferrule{
  \annotateexpr{\tenv, \vep} \typearrow (\vtpp, \vepp) \OrTypeError\\
  \CheckUnop(\tenv, \op, \vtpp) \typearrow \vt \OrTypeError
}{
  \annotateexpr{\tenv, \EUnop(\op, \vep)} \typearrow (\vt, \EUnop(\op, \vepp))
}
\end{mathpar}
\end{emptyformal}

\isempty{\subsection{Comments}}

\section{TypingRule.ECond \label{sec:TypingRule.ECond}}

\subsection{Prose}
All of the following apply:
\begin{itemize}
  \item $\ve$ denotes a conditional expression with condition $\econd$ with two options $\etrue$ and $\efalse$;
  \item annotating $\econd$ in $\tenv$ results in $(\tcond, \econdp)$ \ProseOrTypeError;
  \item annotating $\etrue$ in $\tenv$ results in $(\ttrue, \etruep)$ \ProseOrTypeError;
  \item annotating $\efalse$ in $\tenv$ results in $(\tfalse, \efalsep)$;
  \item obtaining the lowest common ancestor of $\ttrue$ and $\tfalse$ results in $\vt$ \ProseOrTypeError;
  \item $\newe$ is the condition $\econdp$ with two options $\etruep$ and $\efalsep$, that is, $\ECond(\econdp, \etruep, \efalsep)$.
\end{itemize}

\subsection{Example}

\CodeSubsection{\ECondBegin}{\ECondEnd}{../Typing.ml}

\begin{emptyformal}
\subsection{Formally}
\begin{mathpar}
\inferrule{
  \annotateexpr{\tenv, \econd} \typearrow (\tcond, \econd') \OrTypeError\\
  \annotateexpr{\tenv, \etrue} \typearrow (\ttrue, \etrue') \OrTypeError\\
  \annotateexpr{\tenv, \efalse} \typearrow (\tfalse, \efalse') \OrTypeError\\
  \lca(\ttrue, \tfalse) \typearrow \vt \OrTypeError
}{
  \annotateexpr{\ECond(\econd, \etrue, \efalse)} \typearrow (\vt, \ECond(\econdp, \etruep, \efalsep))
}
\end{mathpar}
\end{emptyformal}

\subsection{Comments}
  This is related to \identr{XZVT}.

\section{TypingRule.ESlice \label{sec:TypingRule.ESlice}}

\subsection{Prose}
All of the following apply:
\begin{itemize}
  \item $\ve$ denotes the slicing of expression $\vep$ by the slices $\slices$, that is, \\
  $\ESlice(\vep, \slices)$;
  \item determining whether $\vep$ together with $\slices$ corresponds to a subprogram call
  in $\tenv$ via $\reduceslicestocall$ yields a negative answer --- $\None$ \ProseOrTypeError;
  \item annotating the expression $\vep$ in $\tenv$ yields $(\tep,\vepp)$ \ProseOrTypeError;
  \item obtaining the \structure\ of $\vt$ in $\tenv$ yields $\structtep$ \ProseOrTypeError;
  \item $\structtep$ is either a bitvector or an integer;
  \item obtaining the width of $\slices$ in $\tenv$ via $\sliceswidth$ yields $\vw$ \ProseOrTypeError;
  \item $\slicesp$ is the result of annotating $\slices$ in $\tenv$;
  \item $\vt$ is the bitvector type of width $\vw$, that is, $\TBits(\vw, \emptylist)$;
  \item $\newe$ is the slicing of expression $\vepp$ by the slices $\slicesp$, that is, \\
  $\ESlice(\vepp, \slicesp)$.
\end{itemize}

\subsection{Example}

\CodeSubsection{\ESliceBegin}{\ESliceEnd}{../Typing.ml}

\begin{emptyformal}
\subsection{Formally}
\begin{mathpar}
\inferrule{
  \reduceslicestocall(\tenv, \vep, \slices) \typearrow \None \OrTypeError\\\\
  \annotateexpr{\tenv, \vep} \typearrow (\tep, \vepp) \OrTypeError\\\\
  \tstruct(\tenv, \tep) \typearrow \structtep \OrTypeError\\\\
  \astlabel(\structtep) \in \{\TInt, \TBits\}\\
  \sliceswidth(\tenv, \slices) \typearrow \vw \OrTypeError\\\\
  \annotateslices(\tenv, \slices) \typearrow \slicesp \OrTypeError
}{
  \annotateexpr{\tenv, \ESlice(\vep, \slices)} \typearrow (\TBits(\vw, \emptylist), \ESlice(\vepp, \slicesp))
}
\end{mathpar}
\end{emptyformal}

\subsection{Comments}
  The width of \slices\ might be a symbolic expression if one of the
  widths references a \texttt{let} identifier with a non-compile-time-constant
  initialiser expression.

  This is related to \identi{MJWM}.

\section{TypingRule.ESetter \label{sec:TypingRule.ESetter}}

\subsection{Prose}
All of the following apply:
\begin{itemize}
  \item $\ve$ denotes the slicing of expression $\vep$ by the slices $\slices$, that is, \\
  $\ESlice(\vep, \slices)$;
  \item determining whether $\vep$ together with $\slices$ corresponds to a subprogram call
  in $\tenv$ via $\reduceslicestocall$ yields a positive answer --- $\langle (\name, \vargs)\rangle$ \ProseOrTypeError;
  \item annotating a call with $(\tenv, \name, \vargs, \emptylist, \STSetter)$
  (that is, an empty list of parameters) yields $(\nameone, \vargsone, \eqs, \langle\tty\rangle)$ \ProseOrTypeError;
  \item $vt$ is $\tty$;
  \item $\newe$ is the call expression $\ECall(\nameone, \vargsone, \eqs)$.
\end{itemize}

\subsection{Example}

\CodeSubsection{\ESetterBegin}{\ESetterEnd}{../Typing.ml}

\begin{emptyformal}
\subsection{Formally}
\begin{mathpar}
\inferrule{
  \reduceslicestocall(\tenv, \vep, \slices) \typearrow \None \OrTypeError\\\\
  \annotatecall{\tenv, \name, \vargs, \STSetter} \typearrow (\nameone, \vargsone, \eqs, \langle\tty\rangle) \OrTypeError\\
}{
  \annotateexpr{\tenv, \ESlice(\vep, \slices)} \typearrow (\tty, \ECall(\nameone, \vargsone, \eqs))
}
\end{mathpar}
\end{emptyformal}

\subsection{Comments}

\section{TypingRule.ECall \label{sec:TypingRule.ECall}}

\subsection{Prose}
All of the following apply:
\begin{itemize}
  \item $\ve$ denotes a call to a subprogram named $\name$ with arguments $\vargs$, that is, \\ $\ECall(\name, \vargs)$;
  \item annotating the call of that subprogram in $\tenv$ (annotating calls is defined in Chapter~\ref{chap:TypingSubprogramCalls})
  yields $(\namep, \vargsp, \eqsp, \langle \vt \rangle)$ \ProseOrTypeError;
  \item $\newe$ is the call to the subprogram named $\namep$ with arguments $\vargsp$
    and parameters $\eqsp$, that is, $\ECall(\name, \vargsp, \eqsp)$.
\end{itemize}

\subsection{Example}

\CodeSubsection{\ECallBegin}{\ECallEnd}{../Typing.ml}

\begin{emptyformal}
\subsection{Formally}
\begin{mathpar}
\inferrule{
  \annotatecall{\tenv, \name, \vargs, \STFunction} \typearrow (\namep, \vargsp, \eqsp, \langle \vt \rangle) \OrTypeError
}{
  \annotateexpr{\tenv, \ECall(\name, \vargs)} \typearrow (\vt, \ECall(\name, \vargsp, \eqsp))
}
\end{mathpar}
\end{emptyformal}

\subsection{Comments}
  This is related to \identd{CFYP}, \identr{BQJG}.

\section{TypingRule.EGetArray \label{sec:TypingRule.EGetArray}}

\subsection{Prose}
All of the following apply:
\begin{itemize}
  \item $\ve$ denotes the slicing of expression $\vep$ by the slices $\slices$;
  \item determining whether $\vep$ together with $\slices$ corresponds to a subprogram call
  in $\tenv$ via $\reduceslicestocall$ yields a negative answer --- $\None$ \ProseOrTypeError;
  \item $(\tep,\vepp)$ is the result of annotating the expression $\vep$ in $\tenv$;
  \item $\tep$ has the structure of an array with index $\size$ and element type $\tty$';
  \item One of the following applies:
  \begin{itemize}
    \item All of the following apply (\textsc{okay}):
    \begin{itemize}
      \item $\slices$ consists of a single slice $\SliceSingle(\eindex)$;
      \item annotating the expression $\eindex$ in $\tenv$ yields $(\tindexp, \eindexp)$ \ProseOrTypeError;
      \item determining the type of the array index for $\size$ in $\tenv$ via \\ $\typeofarraylength$
      yields $\wantedtindex$;
      \item determining whether $\tindexp$ \typesatisfies\ $\wantedtindex$ yields $\True$ \ProseOrTypeError;
      \item $\vt$ is $\ttyp$;
      \item $\newe$ is the array access expression for $\vepp$ and index $\eindexp$, that is, $\EGetArray(\vepp, \eindexp)$.
    \end{itemize}

    \item All of the following apply (\textsc{error}):
    \begin{itemize}
      \item $\slices$ consists of a single slice $\SliceSingle(\eindex)$;
      \item the result is a type error indicating that an array must be accessed with a slice corresponding to a single index
      expression.
    \end{itemize}
  \end{itemize}
\end{itemize}

\subsection{Example}

\CodeSubsection{\EGetArrayBegin}{\EGetArrayEnd}{../Typing.ml}

\begin{emptyformal}
\subsection{Formally}
\begin{mathpar}
\inferrule[okay]{
  \reduceslicestocall(\tenv, \vep, \slices) \typearrow \None \OrTypeError\\\\
  \annotateexpr{\tenv, \ve} \typearrow (\tep, \vepp) \OrTypeError\\\\
  \tstruct(\tenv, \tep) \typearrow \TArray(\size, \ttyp) \OrTypeError\\\\
  \slices = [ \SliceSingle(\eindex) ]\\
  \annotateexpr{\tenv, \eindex} \typearrow (\tindexp, \eindexp) \OrTypeError\\
  \typeofarraylength(\tenv, \size) \typearrow \wantedtindex\\
  \checktypesat(\tenv, \tindexp, \wantedtindex) \typearrow \True \OrTypeError\\
}{
  \annotateexpr{\tenv, \ESlice(\vep, \slices)} \typearrow (\ttyp, \EGetArray(\vepp, \eindexp))
}
\and
\inferrule[error]{
  \reduceslicestocall(\tenv, \vep, \slices) \typearrow \None \OrTypeError\\\\
  \annotateexpr{\tenv, \ve} \typearrow (\tep, \vepp) \OrTypeError\\\\
  \tstruct(\tenv, \tep) \typearrow \TArray(\size, \ttyp) \OrTypeError\\\\
  \slices \neq [ \SliceSingle(\Ignore) ]\\
}{
  \annotateexpr{\tenv, \ESlice(\vep, \slices)} \typearrow \TypeErrorVal{IllegalArraySlice}
}
\end{mathpar}
\end{emptyformal}

\isempty{\subsection{Comments}}

\section{TypingRule.ESliceOrEGetArrayError \label{sec:TypingRule.ESliceOrEGetArrayError}}

\subsection{Prose}
All of the following apply:
\begin{itemize}
  \item $\ve$ denotes the slicing of expression $\vep$ by the slices $\slices$;
  \item determining whether $\vep$ together with $\slices$ corresponds to a subprogram call
  in $\tenv$ via $\reduceslicestocall$ yields a negative answer --- $\None$ \ProseOrTypeError;
  \item $(\tep,\vepp)$ is the result of annotating the expression $\vep$ in $\tenv$;
  \item $\tep$ has the structure $\vtp$;
  \item $\vtp$ is neither an integer type, a bitvector type, or an array type;
  \item the result is an error indicating that the type of $\vep$ is inappropriate for slicing.
\end{itemize}

\subsection{Example}

\CodeSubsection{\ESliceOrEGetArrayErrorBegin}{\ESliceOrEGetArrayErrorEnd}{../Typing.ml}

\begin{emptyformal}
\subsection{Formally}
\begin{mathpar}
\inferrule{
  \reduceslicestocall(\tenv, \vep, \slices) \typearrow \None \OrTypeError\\\\
  \annotateexpr{\tenv, \ve} \typearrow (\tep, \vepp) \OrTypeError\\\\
  \tstruct(\tenv, \tep) \typearrow \vtp\\
  \astlabel(\vtp) \not\in \{\TInt, \TBits, \TArray\}
}{
  \annotateexpr{\tenv, \ESlice(\vep, \slices)} \typearrow \TypeErrorVal{IllegalSliceType}
}
\end{mathpar}
\end{emptyformal}

\isempty{\subsection{Comments}}

\section{TypingRule.EStructuredNotStructured \label{sec:TypingRule.EStructuredNotStructured}}

\subsection{Prose}
All of the following apply:
\begin{itemize}
\item $\ve$ denotes the record expression or an exception expression of type $\tty$ with fields $\fields$;
\item determining whether $\tty$ is a named type yields $\True$ \ProseOrTypeError;
\item determining the \structure\ of $\tty$ yields $\vtp$ \ProseOrTypeError;
\item $\vtp$ is neither a record nor an exception type;
\item the result is an error indicating that $\tty$ is not appropriate for constructing a record.
\end{itemize}

\subsection{Example}

\CodeSubsection{\EStructuredNotStructuredBegin}{\EStructuredNotStructuredEnd}{../Typing.ml}

\begin{emptyformal}
\subsection{Formally}
\begin{mathpar}
\inferrule{
  \checktrans{\astlabel(\tty) = \TNamed}{NamedTypeExpected} \checktransarrow \True \OrTypeError\\
  \tstruct{\tenv, \tty} \typearrow\vtp \OrTypeError\\\\
  \astlabel(\vtp) \not\in \{\TRecord, \TException\}
}
{
  \annotateexpr{\tenv, \ERecord(\tty, \fields)} \typearrow \TypeErrorVal{NonStructuredType}
}
\end{mathpar}
\end{emptyformal}

\isempty{\subsection{Comments}}
  This is related to \identr{WBCQ}.

\section{TypingRule.EStructuredMissingField \label{sec:TypingRule.EStructuredMissingField}}

\subsection{Prose}
A§ll of the following apply:
\begin{itemize}
  \item $\ve$ denotes the record expression or an exception expression of type $\tty$ with fields $\fields$;
  \item $\tty$ is the name of a record or exception type with fields $\fieldtypes$;
  \item there exists a field in $\fieldtypes$ that is not initialised by $\fields$;
  \item the result is an error indicating that a field is missing initialization.
\end{itemize}

\subsection{Example}

\CodeSubsection{\EStructuredMissingFieldBegin}{\EStructuredMissingFieldEnd}{../Typing.ml}

\begin{emptyformal}
\subsection{Formally}
\begin{mathpar}
\inferrule{
  \checktrans{\astlabel(\tty) = \TNamed}{NamedTypeExpected} \checktransarrow \True \OrTypeError\\
  \tstruct(\tenv, \tty) \typearrow L(\fieldtypes) \OrTypeError\\\\
  L \in \{\TRecord, \TException\}\\
  \initializedfields \eqdef \{\name \;|\; (\name, \Ignore)\in\fields\}\\
  \names \eqdef \fieldnames(\fieldtypes)\\
  \vb \eqdef \initializedfields = \names\\
  \vb = \False
}
{
  \annotateexpr{\tenv, \ERecord(\tty, \fields)} \typearrow \TypeErrorVal{MissingField}
}
\end{mathpar}
\end{emptyformal}

\subsection{Comments}
  This is related to \identr{WBCQ}.

\section{TypingRule.ERecord \label{sec:TypingRule.ERecord}}

\subsection{Prose}
All of the following apply:
\begin{itemize}
  \item $\ve$ denotes the record creation expression (which is also used for creating exceptions) of type $\tty$ with fields $\fields$;
  \item $\tty$ is the name of a record or exception type with fields $\fieldtypes$;
  \item \underline{every} field in $\fieldtypes$ is initialised by a corresponding expression in $\fields$;
  \item annotating the expressions that initialize each of the fields in $\fields$ via \\
        $\annotatefieldinit$ yields $\fieldsp$ \ProseOrTypeError;
  \item $\vt$ is $\tty$;
  \item $\newe$ is the record expression with type $\tty$ and field initializers $\fieldsp$, that is, $\ERecord(\tty, \fieldsp)$;
\end{itemize}

\subsection{Example}

\CodeSubsection{\ERecordBegin}{\ERecordEnd}{../Typing.ml}

\begin{emptyformal}
\subsection{Formally}
\begin{mathpar}
\inferrule{
  \checktrans{\astlabel(\tty) = \TNamed}{NamedTypeExpected} \checktransarrow \True \OrTypeError\\
  \tstruct(\tenv, \tty) \typearrow L(\fieldtypes) \OrTypeError\\\\
  L \in \{\TRecord, \TException\}\\
  \initializedfields \eqdef \{\name \;|\; (\name, \Ignore)\in\fields\}\\
  \names \eqdef \fieldnames(\fieldtypes)\\
  \initializedfields = \names\\
  (\name, \vep) \in \fields: \annotatefieldinit(\tenv, (\name, \vep), \fieldtypes) \typearrow (\name, \vepp) \OrTypeError\\
  \fieldsp \eqdef [(\name, \vep) \in \fields : (\name, \vepp)]
}{
  \annotateexpr{\tenv, \ERecord(\tty, \fields)} \typearrow (\tty, \ERecord(\tty, \fieldsp))
}
\end{mathpar}
\end{emptyformal}

\subsection{Comments}
  This is related to \identr{WBCQ}.

\section{TypingRule.EGetRecordField \label{sec:TypingRule.EGetRecordField}}
\subsection{Prose}
All of the following apply:
\begin{itemize}
  \item $\ve$ denotes the access of field $\fieldname$ in the value represented by the expression $\veone$, that is, $\EGetField(\veone, \fieldname)$;
  \item annotating the expression $\veone$ in $\tenv$ yields $(\vteone, \vetwo)$ \ProseOrTypeError;
  \item obtaining the \underlyingtype\ of $\vteone$ yields $\vtetwo$ \ProseOrTypeError;
  % \item checking whether the field access with expression $\vtetwo$ and $\fieldname$ represents a call yields $\None$ \ProseOrTypeError;
  \item $\vtetwo$ is either a record type or an exception type with fields $\fields$;
  \item the field $\fieldname$ is associated with the type $\vt$ in $\fields$
  \item $\newe$ is the access of field $\fieldname$ on the record or exception object $\vetwo$, that is, $\EGetField(\vetwo, \fieldname)$.
\end{itemize}

\subsection{Example}

\CodeSubsection{\EGetRecordFieldBegin}{\EGetRecordFieldEnd}{../Typing.ml}

\begin{emptyformal}
\subsection{Formally}
\begin{mathpar}
\inferrule{
  \annotateexpr{\tenv, \veone} \typearrow (\vteone, \vetwo) \OrTypeError\\\\
  \makeanonymous(\tenv, \vteone) \typearrow \vtetwo \OrTypeError\\\\
  % \tododefine{reduce\_field\_to\_call}(\tenv, \vtetwo, \fieldname) \typearrow \None \OrTypeError\\
  \vtetwo \eqname L(\fields)\\
  L \in \{\TRecord, \TException\}\\
  \assocopt(\fields, \fieldname) \typearrow \langle \vt\rangle
}{
  \annotateexpr{\tenv, \EGetField(\veone, \fieldname)} \typearrow (\vt, \EGetField(\vetwo, \fieldname))
}
\end{mathpar}
\end{emptyformal}

\isempty{\subsection{Comments}}

\section{TypingRule.EGetBadRecordField \label{sec:TypingRule.EGetBadRecordField}}

\subsection{Prose}
All of the following apply:
\begin{itemize}
  \item $\ve$ denotes the access of field $\fieldname$ in the value represented by the expression $\veone$, that is, $\EGetField(\veone, \fieldname)$;
  \item annotating the expression $\veone$ in $\tenv$ yields $(\vteone, \vetwo)$ \ProseOrTypeError;
  \item obtaining the \underlyingtype\ of $\vteone$ yields $\vtetwo$ \ProseOrTypeError;
  % \item checking whether the field access with expression $\vtetwo$ and $\fieldname$ represents a call yields $\None$ \ProseOrTypeError;
  \item $\vtetwo$ is either a record type or an exception type with fields $\fields$;
  \item the field $\fieldname$ is not associated with any type in $\fields$
  \item the result is a type error indicating the missing field.
\end{itemize}

\subsection{Example}

\CodeSubsection{\EGetBadRecordFieldBegin}{\EGetBadRecordFieldEnd}{../Typing.ml}

\begin{emptyformal}
\subsection{Formally}
\begin{mathpar}
\inferrule{
  \annotateexpr{\tenv, \veone} \typearrow (\vteone, \vetwo) \OrTypeError\\
  \makeanonymous(\tenv, \vteone) \typearrow \vtetwo \OrTypeError\\
  % \tododefine{reduce\_field\_to\_call}(\tenv, \vtetwo, \fieldname) \typearrow \None \OrTypeError\\
  \vtetwo \eqname L(\fields)\\
  L \in \{\TRecord, \TException\}\\
  \assocopt(\fields, \fieldname) \typearrow \None
}{
  \annotateexpr{\tenv, \EGetField(\veone, \fieldname)} \typearrow \TypeErrorVal{FieldDoesNotExist}
}
\end{mathpar}
\end{emptyformal}

\isempty{\subsection{Comments}}

\section{TypingRule.EGetBadBitField \label{sec:TypingRule.EGetBadBitField}}

\subsection{Prose}
All of the following apply:
\begin{itemize}
  \item $\ve$ denotes the access of field $\fieldname$ in the value represented by the expression $\veone$, that is, $\EGetField(\veone, \fieldname)$;
  \item annotating the expression $\veone$ in $\tenv$ yields $(\vteone, \vetwo)$ \ProseOrTypeError;
  \item obtaining the \underlyingtype\ of $\vteone$ yields $\vtetwo$ \ProseOrTypeError;
  % \item checking whether the field access with expression $\vtetwo$ and $\fieldname$ represents a call yields $\None$ \ProseOrTypeError;
  \item $\vtetwo$ is a bitvector type with bit fields $\bitfields$;
  \item the field $\fieldname$ is not found in $\bitfields$
  \item the result is a type error indicating the missing field.
\end{itemize}

\subsection{Example}

\CodeSubsection{\EGetBadBitFieldBegin}{\EGetBadBitFieldEnd}{../Typing.ml}

\begin{emptyformal}
\begin{mathpar}
\inferrule{
  \ve \eqname \EGetField(\veone, \fieldname)\\
  \annotateexpr{\tenv, \veone} \typearrow (\vteone, \vetwo) \OrTypeError\\
  \makeanonymous(\tenv, \vteone) \typearrow \vtetwo \OrTypeError\\
  % \tododefine{reduce\_field\_to\_call}(\tenv, \vtetwo, \fieldname) \typearrow \None \OrTypeError\\
  \vtetwo \eqname \TBits(\Ignore, \bitfields)\\
  \tododefine{find\_bitfield}(\bitfields, \fieldname) \typearrow \None
}{
  \annotateexpr{\tenv, \ve} \typearrow \TypeErrorVal{FieldDoesNotExist}
}
\end{mathpar}
\end{emptyformal}

\isempty{\subsection{Comments}}


\section{TypingRule.EGetBitField \label{sec:TypingRule.EGetBitField}}

\subsection{Prose}
All of the following apply:
\begin{itemize}
  \item $\ve$ denotes the access of field $\fieldname$ in the value represented by the expression $\veone$, that is, $\EGetField(\veone, \fieldname)$;
  \item annotating the expression $\veone$ in $\tenv$ yields $(\vteone, \vetwo)$ \ProseOrTypeError;
  \item obtaining the \underlyingtype\ of $\vteone$ yields $\vtetwo$ \ProseOrTypeError;
  % \item checking whether the field access with expression $\vtetwo$ and $\fieldname$ represents a call yields $\None$ \ProseOrTypeError;
  \item $\vtetwo$ is a bitvector type with bit fields $\bitfields$;
  \item $\fieldname$ is declared in $\bitfields$ with a slice list $\slices$, that is, \\ $\BitFieldSimple(\Ignore, \slices)$;
  \item $\vethree$ denotes the slicing of the expression \vetwo\ by the slices $\slices$, that is, \\ $\ESlice(\vetwo, \slices)$;
  \item annotating $\vethree$ in $\tenv$ yields $(\vt, \newe)$ \ProseOrTypeError.
\end{itemize}

\subsection{Example}

\CodeSubsection{\EGetBitFieldBegin}{\EGetBitFieldEnd}{../Typing.ml}

\begin{emptyformal}
\subsection{Formally}
\begin{mathpar}
\inferrule{
  \ve \eqname \EGetField(\veone, \fieldname)\\
  \annotateexpr{\tenv, \veone} \typearrow (\vteone, \vetwo) \OrTypeError\\
  \makeanonymous(\tenv, \vteone) \typearrow \vtetwo \OrTypeError\\
  % \tododefine{reduce\_field\_to\_call}(\tenv, \vtetwo, \fieldname) \typearrow \None \OrTypeError\\
  \vtetwo \eqname \TBits(\Ignore, \bitfields)\\
  \tododefine{find\_bitfield}(\bitfields, \fieldname) \typearrow \langle \BitFieldSimple(\Ignore, \slices)\rangle\\
  \vethree \eqdef \ESlice(\vetwo, \slices)\\
  \annotateexpr{\tenv, \vethree} \typearrow (\vt, \newe) \OrTypeError
}{
  \annotateexpr{\tenv, \ve} \typearrow \typearrow (\vt, \newe)
}
\end{mathpar}
\end{emptyformal}

\isempty{\subsection{Comments}}

\section{TypingRule.EGetBitFieldNested \label{sec:TypingRule.EGetBitFieldNested}}

\subsection{Prose}
All of the following apply:
\begin{itemize}
  \item $\ve$ denotes the access of field $\fieldname$ in the value represented by the expression $\veone$, that is, $\EGetField(\veone, \fieldname)$;
  \item annotating the expression $\veone$ in $\tenv$ yields $(\vteone, \vetwo)$ \ProseOrTypeError;
  \item obtaining the \underlyingtype\ of $\vteone$ yields $\vtetwo$ \ProseOrTypeError;
  % \item checking whether the field access with expression $\vtetwo$ and $\fieldname$ represents a call yields $\None$ \ProseOrTypeError;
  \item $\vtetwo$ is a bitvector type with bit fields $\bitfields$;
  \item $\fieldname$ is declared in $\bitfields$ with a slice list $\slices$ and nested bitfields $\bitfieldsp$, that is,
        $\BitFieldNested(\Ignore, \slices, \bitfieldsp)$;
  \item $\vethree$ denotes the slicing of the expression \vetwo\ by the slices $\slices$, that is, \\ $\ESlice(\vetwo, \slices)$;
  \item annotating $\vethree$ in $\tenv$ yields $(\vtefour, \newe)$ \ProseOrTypeError;
  \item $\vtefour$ is a bitvector type with length expression $\width$, that is, $\TBits(\width, \Ignore)$;
  \item $\vt$ is a bitvector type with length expression $\width$ and bitfields $\bitfieldsp$.
\end{itemize}

\subsection{Example}

\CodeSubsection{\EGetBitFieldNestedBegin}{\EGetBitFieldNestedEnd}{../Typing.ml}

\begin{emptyformal}
\subsection{Formally}
\begin{mathpar}
\inferrule{
  \ve \eqname \EGetField(\veone, \fieldname)\\
  \annotateexpr{\tenv, \veone} \typearrow (\vteone, \vetwo) \OrTypeError\\\\
  \makeanonymous(\tenv, \vteone) \typearrow \vtetwo \OrTypeError\\
  % \tododefine{reduce\_field\_to\_call}(\tenv, \vtetwo, \fieldname) \typearrow \None \OrTypeError\\\\
  \vtetwo \eqname \TBits(\Ignore, \bitfields)\\
  \tododefine{find\_bitfield}(\bitfields, \fieldname) \typearrow \langle \BitFieldNested(\Ignore, \slices, \bitfieldsp)\rangle\\
  \vethree \eqdef \ESlice(\vetwo, \slices)\\
  \annotateexpr{\tenv, \vethree} \typearrow (\vtefour, \newe) \OrTypeError\\\\
  \vtefour \eqname \TBits(\width, \Ignore)\\
  \vt \eqdef \TBits(\width, \bitfieldsp)
}{
  \annotateexpr{\tenv, \ve} \typearrow (\vt, \newe)
}
\end{mathpar}
\end{emptyformal}

\isempty{\subsection{Comments}}

\section{TypingRule.EGetBitFieldTyped \label{sec:TypingRule.EGetBitFieldTyped}}

\subsection{Prose}
All of the following apply:
\begin{itemize}
  \item $\ve$ denotes the access of field $\fieldname$ in the value represented by the expression $\veone$, that is, $\EGetField(\veone, \fieldname)$;
  \item annotating the expression $\veone$ in $\tenv$ yields $(\vteone, \vetwo)$ \ProseOrTypeError;
  \item obtaining the \underlyingtype\ of $\vteone$ yields $\vtetwo$ \ProseOrTypeError;
  % \item checking whether the field access with expression $\vtetwo$ and $\fieldname$ represents a call yields $\None$ \ProseOrTypeError;
  \item $\vtetwo$ is a bitvector type with bit fields $\bitfields$;
  \item $\fieldname$ is declared in $\bitfields$ with a slice list $\slices$ and typed-bitfield with type $vt$ that is,
        $\BitFieldType(\Ignore, \slices, \vt)$;
  \item $\vethree$ denotes the slicing of the expression \vetwo\ by the slices $\slices$, that is, \\ $\ESlice(\vetwo, \slices)$;
  \item annotating $\vethree$ in $\tenv$ yields $(\vtefour, \newe)$ \ProseOrTypeError;
  \item determining whether $\vtefour$ \typesatisfies\ $\vt$ yields $\True$ \ProseOrTypeError.
\end{itemize}

\subsection{Example}

\CodeSubsection{\EGetBitFieldTypedBegin}{\EGetBitFieldTypedEnd}{../Typing.ml}

\begin{emptyformal}
\subsection{Formally}
\begin{mathpar}
\inferrule{
  \ve \eqname \EGetField(\veone, \fieldname)\\
  \annotateexpr{\tenv, \veone} \typearrow (\vteone, \vetwo) \OrTypeError\\\\
  \makeanonymous(\tenv, \vteone) \typearrow \vtetwo \OrTypeError\\
  % \tododefine{reduce\_field\_to\_call}(\tenv, \vtetwo, \fieldname) \typearrow \None \OrTypeError\\\\
  \vtetwo \eqname \TBits(\Ignore, \bitfields)\\
  \tododefine{find\_bitfield}(\bitfields, \fieldname) \typearrow \langle \BitFieldType(\Ignore, \slices, \vt)\rangle\\
  \vethree \eqdef \ESlice(\vetwo, \slices)\\
  \annotateexpr{\tenv, \vethree} \typearrow (\vtefour, \newe) \OrTypeError\\\\
  \checktypesat(\tenv, \vtefour, \vt) \typearrow \True \OrTypeError
}{
  \annotateexpr{\tenv, \ve} \typearrow (\vt, \newe)
}
\end{mathpar}
\end{emptyformal}

\isempty{\subsection{Comments}}

\section{TypingRule.EGetTupleItem \label{sec:TypingRule.EGetTupleItem}}

\subsection{Prose}
All of the following apply:
\begin{itemize}
  \item $\ve$ denotes the access of field $\fieldname$ in the value represented by the expression $\veone$, that is, $\EGetField(\veone, \fieldname)$;
  \item annotating the expression $\veone$ in $\tenv$ yields $(\vteone, \vetwo)$ \ProseOrTypeError;
  \item obtaining the \underlyingtype\ of $\vteone$ yields $\vtetwo$ \ProseOrTypeError;
  % \item checking whether the field access with expression $\vtetwo$ and $\fieldname$ represents a call yields $\None$ \ProseOrTypeError;
  \item $\vtetwo$ is tuple type with list of types $\tys$, that is, $\TTuple(\tys)$;
  \item $\fieldname$ is an identifier with the prefix \texttt{item} and the constant $\vindex$;
  \item determining whether $\vindex$ is between $0$ and the number of types in $\tys$, inclusive, yields $\True$ \ProseOrTypeError;
  \item $\vt$ is the type at position $\vindex$ of $\tys$;
  \item $\newe$ is the expression for obtaining the item at index $\vindex$ from the expression $\vetwo$, that is, $\EGetItem(\vetwo, \vindex)$.
\end{itemize}

\subsection{Example}

\CodeSubsection{\EGetTupleItemBegin}{\EGetTupleItemEnd}{../Typing.ml}

\begin{emptyformal}
\subsection{Formally}
\begin{mathpar}
\inferrule{
  \ve \eqname \EGetField(\veone, \fieldname)\\
  \annotateexpr{\tenv, \veone} \typearrow (\vteone, \vetwo) \OrTypeError\\\\
  \makeanonymous(\tenv, \vteone) \typearrow \vtetwo \OrTypeError\\
  % \tododefine{reduce\_field\_to\_call}(\tenv, \vtetwo, \fieldname) \typearrow \None \OrTypeError\\\\
  \vtetwo \eqname \TTuple(\tys)\\
  \fieldname \eqname \texttt{"item<index>"}\\
  \checktrans{0 \leq \vindex \leq |\tys|}{IndexOutOfRange} \checktransarrow \True \OrTypeError\\\\
  \vt \eqdef \tys[\vindex]\\
  \newe \eqdef \EGetItem(\vetwo, \vindex)
}{
  \annotateexpr{\tenv, \ve} \typearrow (\vt, \newe)
}
\end{mathpar}
\end{emptyformal}

\isempty{\subsection{Comments}}

\section{TypingRule.EGetBadField \label{sec:TypingRule.EGetBadField}}

\subsection{Prose}
All of the following apply:
\begin{itemize}
  \item $\ve$ denotes the access of field $\fieldname$ in the value represented by the expression $\veone$, that is, $\EGetField(\veone, \fieldname)$;
  \item annotating the expression $\veone$ in $\tenv$ yields $(\vteone, \vetwo)$ \ProseOrTypeError;
  \item obtaining the \underlyingtype\ of $\vteone$ yields $\vtetwo$ \ProseOrTypeError;
  % \item checking whether the field access with expression $\vtetwo$ and $\fieldname$ represents a call yields $\None$ \ProseOrTypeError;
  \item $\vtetwo$ is neither one of the following types: record, exception, bitvector, or tuple;
  \item the result is an error indicating that the type of $\veone$ is inappropriate for accessing the field $\fieldname$.
\end{itemize}

 \subsection{Example}

\CodeSubsection{\EGetBadFieldBegin}{\EGetBadFieldEnd}{../Typing.ml}

\begin{emptyformal}
\subsection{Formally}
\begin{mathpar}
\inferrule{
  \ve \eqname \EGetField(\veone, \fieldname)\\
  \annotateexpr{\tenv, \veone} \typearrow (\vteone, \vetwo) \OrTypeError\\\\
  \makeanonymous(\tenv, \vteone) \typearrow \vtetwo \OrTypeError\\
  % \tododefine{reduce\_field\_to\_call}(\tenv, \vtetwo, \fieldname) \typearrow \None \OrTypeError\\\\
  \astlabel(\vtetwo) \not\in \{\TRecord, \TException, \TBits, \TTuple\}
}{
  \annotateexpr{\tenv, \ve} \typearrow \TypeErrorVal{ConflictingTypes}
}
\end{mathpar}
\end{emptyformal}

\isempty{\subsection{Comments}}

\section{TypingRule.EConcat \label{sec:TypingRule.EConcat}}

\subsection{Prose}
All of the following apply:
\begin{itemize}
  \item $\ve$ denotes the concatenation of a non-empty list of expressions $\vli$, that is, \\ $\EConcat(\vli)$;
  \item annotating each expression $\vle[i]$ in $\tenv$, for $i=1..k$, yields $(\vt_i, \ve_i$) \ProseOrTypeError;
  \item $\ves$ is the list of expressions $\ve_i$, for $i=1..k$;
  \item obtaining the bitvector width of $\vt_i$ in $\tenv$ (which also checks that $\vt_i$ is a bitvector type),
        for $i=1..k$, yields $\vw_i$ \ProseOrTypeError;
  \item to obtain the (symbolic) width of the resulting bitvector, first define $\widthsum_1$ to be $\vw_1$;
  \item then define $\widthsum_i$, for $i=2..k$, to be obtained by reducing the expression that sums \\ $\widthsum_{i-1}$ with the width $\vw_i$;
  \item $\vt$ is the bitvector of length $\widthsum_k$ and the empty bitfield list, that is, \\ $\TBits(\widthsum_k, \emptylist)$;
  \item $\newe$ is the concatenation expression for $\ves$, that is, $\EConcat(\ves)$.
\end{itemize}

\subsection{Example}

\CodeSubsection{\EConcatBegin}{\EConcatEnd}{../Typing.ml}

\begin{emptyformal}
\subsection{Formally}
\begin{mathpar}
\inferrule{
  i=1..k: \annotateexpr{\tenv, \vli[i]} \typearrow (\vt_i, \ve_i) \OrTypeError\\\\
  \vts \eqdef [i=1..k: \vt_i]\\
  \ves \eqdef [i=1..k: \ve_i]\\
  i=1..k: \getbitvectorwidth(\tenv, \vt_i) \typearrow \vw_i \OrTypeError\\\\
  \widthsum_1 \eqdef \vw_1\\
  i=2..k: \tododefine{reduce\_expr}(\tenv, \EBinop(\PLUS, \widthsum_{i-1}, \vw_i)) \typearrow \widthsum_i
}{
  \annotateexpr{\tenv, \EConcat(\vli)} \typearrow (\TBits(\widthsum_k, \emptylist), \EConcat(\ves))
}
\end{mathpar}
\end{emptyformal}

\subsection{Comments}
  This is related to \identr{NYNK} and \identr{KCZS}.

  The sum of the widths of the bitvector types~\texttt{ts} might be a symbolic
expression that is unresolvable to an integer. For example:
    \VerbatimInput{../tests/ASLTypingReference.t/TypingRule.EConcatUnresolvableToInteger.asl}

\section{TypingRule.ETuple \label{sec:TypingRule.ETuple}}

\subsection{Prose}
All of the following apply:
\begin{itemize}
  \item $\ve$ denotes a tuple expression with list of expressions $\vli$, that is, $ \ETuple(\vli)$;
  \item annotating each expression $\vle[i]$ in $\tenv$, for $i=1..k$, yields $(\vt_i, \ve_i$) \ProseOrTypeError;
  \item $\vt$ is the tuple type with list of types $\vt_i$, for $i=1..k$;
  \item $\newe$ is tuple expression over list of expressions $\ve_i$, for $i=1..k$.
\end{itemize}

\subsection{Example}

\CodeSubsection{\ETupleBegin}{\ETupleEnd}{../Typing.ml}

\begin{emptyformal}
\subsection{Formally}
\begin{mathpar}
\inferrule{
  i=1..k: \annotateexpr{\tenv, \vle[i]} \typearrow (\vt_i, \ve_i) \OrTypeError
}{
  \annotateexpr{\tenv, \ETuple(\vli)} \typearrow (\TTuple(\vt_{1..k}), \ETuple(\ve_{1..k}))
}
\end{mathpar}
\end{emptyformal}

\isempty{\subsection{Comments}}

\section{TypingRule.EUnknown \label{sec:TypingRule.EUnknown}}

\subsection{Prose}
All of the following apply:
\begin{itemize}
  \item $\ve$ denotes an expression \UNKNOWN\ of type $\tty$, that is, $\EUnknown(\tty)$;
  \item annotating the type $\tty$ in $\tenv$ yields $\ttyone$ \ProseOrTypeError;
  \item obtaining the \structure\ of $\ttyone$ in $\tenv$ yields $\ttytwo$ \ProseOrTypeError;
  \item $\vt$ is $\ttyone$;
  \item $\newe$ is an expression \UNKNOWN\ of type $\ttytwo$, that is, $\EUnknown(\ttytwo)$.
\end{itemize}

\subsection{Example}

\CodeSubsection{\EUnknownBegin}{\EUnknownEnd}{../Typing.ml}

\begin{emptyformal}
\subsection{Formally}
\begin{mathpar}
\inferrule{
  \annotatetype{\tenv, \tty} \typearrow \ttyone \OrTypeError\\
  \tstruct(\tenv, \ttyone) \typearrow \ttytwo \OrTypeError
}{
  \annotateexpr{\tenv, \EUnknown(\tty)} \typearrow (\ttyone, \EUnknown(\ttytwo))
}
\end{mathpar}
\end{emptyformal}

\isempty{\subsection{Comments}}

\section{TypingRule.EPattern \label{sec:TypingRule.EPattern}}

\subsection{Prose}
All of the following apply:
\begin{itemize}
  \item $\ve$ denotes whether the expression $\veone$ matches the pattern $\vpat$, that is, $\EPattern(\veone, \vpat)$;
  \item annotating the expression $\veone$ in $\tenv$ yields $(\vtetwo, \vetwo)$ \ProseOrTypeError;
  \item annotating the pattern $\vttwo$ in $\tenv$ yields $\vpatp$ \ProseOrTypeError;
  \item $\vt$ is $\TBool$;
  \item $\newe$ denotes whether the expression $\vetwo$ matches $\vpatp$, that is, $\EPattern(\vetwo, \vpatp)$.
\end{itemize}

\subsection{Example}

\CodeSubsection{\EPatternBegin}{\EPatternEnd}{../Typing.ml}

\begin{emptyformal}
\subsection{Formally}
\begin{mathpar}
\inferrule{
  \annotateexpr{\tenv, \veone} \typearrow (\vtetwo, \vetwo) \OrTypeError\\
  \annotatepattern(\tenv, \vtetwo) \typearrow \vpatp \OrTypeError
}{
  \annotateexpr{\tenv, \EPattern(\veone, \vpat)} \typearrow (\TBool, \EPattern(\vetwo, \vpatp))
}
\end{mathpar}
\end{emptyformal}

\isempty{\subsection{Comments}}

\section{TypingRule.ATC \label{sec:TypingRule.ATC}}

\subsection{Prose}
All of the following apply:
\begin{itemize}
  \item $\ve$ denotes an asserting type conversion with expression $\vep$ and type $\tty$, that is $\EATC(\vep, \tty)$;
  \item annotating the expression $\vep$ in $\tenv$ yields $(\vt, \vepp)$ \ProseOrTypeError;
  \item obtaining the \structure\ of $\vt$ in $\tenv$ yields $\vtstruct$ \ProseOrTypeError;
  \item annotating the type $\tty$ in $\tenv$ yields $\tty'$ \ProseOrTypeError;
  \item obtaining the \structure\ of $\tty'$ in $\tenv$ yields $\vtystruct$ \ProseOrTypeError;
  \item One of the following applies:
  \begin{itemize}
  \item All of the following apply (\textsc{type\_equal}):
  \begin{itemize}
    \item determining whether $\vtstruct$ is equivalent to $\vtystruct$ in $\tenv$ \\ yields $\True$;
    \item $\vt$ is $\tty'$ and $\newe$ is $\vepp$.
  \end{itemize}
  \item All of the following apply (\textsc{dynamic}):
    \begin{itemize}
      \item determining whether $\vtstruct$ is equivalent to $\vtystruct$ in $\tenv$ \\ yields $\False$,
      meaning that an execution-time check that the expression $\vep$ evaluates to a value in the
      dynamic domain of $\tty$ is required;
      \item both $\vtstruct$ and $\vtystruct$ are bitvector types or integer types.
      \item $\vt$ is $\vtp$ and $\newe$ is the asserting type conversion expression over $\vepp$ \\ and
      $\vtystruct$, that is, $\EATC(\vepp, \vtystruct)$.
    \end{itemize}
  \item All of the following apply:
    \begin{itemize}
    \item determining whether $\vtstruct$ is equivalent to $\vtystruct$ in $\tenv$ \\ yields $\False$;
    \item $\vtstruct$ and $\vtystruct$ are not both bitvector types or integer types.
    \item a type error indicating the conflicting types is returned.
    \end{itemize}
  \end{itemize}
\end{itemize}

\subsection{Example}

\CodeSubsection{\ATCBegin}{\ATCEnd}{../Typing.ml}

\begin{emptyformal}
\subsection{Formally}
\begin{mathpar}
\inferrule[type\_equal]{
  \annotateexpr{\tenv, \vep} \typearrow (\vt, \vepp) \OrTypeError\\
  \tstruct(\tenv, \vt) \typearrow \vtstruct \OrTypeError\\
  \annotatetype{\tenv, \tty} \typearrow \tty' \OrTypeError\\
  \tstruct(\tenv, \tty') \typearrow \vtystruct \OrTypeError\\
  \typeequal(\tenv, \vtstruct, \vtystruct) \typearrow \True
}{
  \annotateexpr{\tenv, \EATC(\vep, \tty)} \typearrow (\tty', \vepp)
}
\end{mathpar}

\begin{mathpar}
\inferrule[dynamic]{
  \annotateexpr{\tenv, \vep} \typearrow (\vt, \vepp) \OrTypeError\\
  \tstruct(\tenv, \vt) \typearrow \vtstruct \OrTypeError\\
  \annotatetype{\tenv, \tty} \typearrow \tty' \OrTypeError\\
  \tstruct(\tenv, \tty') \typearrow \vtystruct \OrTypeError\\
  \typeequal(\tenv, \vtstruct, \vtystruct) \typearrow \False\\
  \vb \eqdef \astlabel(\vtstruct) = \astlabel(\vtystruct) \land
  \astlabel(\vtstruct) \in \{\TBits, \TInt \}\\
  \vb \eqdef \True
}{
  \annotateexpr{\tenv, \EATC(\vep, \tty)} \typearrow (\vtp, \EATC(\vepp, \vtystruct))
}
\end{mathpar}

\begin{mathpar}
\inferrule[error]{
  \annotateexpr{\tenv, \vep} \typearrow (\vt, \vepp) \OrTypeError\\
  \tstruct(\tenv, \vt) \typearrow \vtstruct \OrTypeError\\
  \annotatetype{\tenv, \tty} \typearrow \tty' \OrTypeError\\
  \tstruct(\tenv, \tty') \typearrow \vtystruct \OrTypeError\\
  \typeequal(\tenv, \vtstruct, \vtystruct) \typearrow \False\\
  \vb \eqdef \astlabel(\vtstruct) = \astlabel(\vtystruct) \land
  \astlabel(\vtstruct) \in \{\TBits, \TInt \}\\
  \vb = \False
}{
  \annotateexpr{\tenv, \EATC(\vep, \tty)} \typearrow \TypeErrorVal{ATC}
}
\end{mathpar}
\end{emptyformal}

\subsection{Comments}
  This is related to \identr{VBLL}, \identi{KRLL}, \identg{PFRQ}, \identi{XVBG},
  \identr{GYJZ}, \identi{SZVF}, \identr{PZZJ}, \identr{YCPX}, \identi{ZLBW},
  \identi{TCST}, \identi{CGRH}, \identi{YJBB}.

\hypertarget{def-annotateliteral}{}
\section{TypingRule.Lit \label{sec:TypingRule.Lit}}

Annotating literals is done via the helper function
\[
  \annotateliteral{\overname{\literal}{\vl}} \aslto \overname{\ty}{\vt}
\]
which we use in this chapter for TypingRule.ELit as well as in subsequent chapters.
\subsection{Prose}
The result of annotating a literal $\vl$ is $\vt$ and one of the following applies:
\begin{itemize}
\item $\vl$ is an integer literal $\vn$ and $\vt$ is the well-constrained integer type, constraining
its set to the single value $\vn$;
\item $\vl$ is a Boolean literal and $\vt$ is the Boolean type;
\item $\vl$ is a real literal and $\vt$ is the real type;
\item $\vl$ is a string literal and $\vt$ is the string type;
\item $\vl$ is a string literal and $\vt$ is the string type;
\item $\vl$ is a bitvector literal of length $\vn$ and $\vt$ is the bitvector type of fixed width $\vn$.
\end{itemize}

\subsection{Example}
In the following example, we show several literals and their corresponding types in comments:
\VerbatimInput{../tests/ASLTypingReference.t/TypingRule.Lit.asl}

\CodeSubsection{\LitBegin}{\LitEnd}{../Typing.ml}

\begin{emptyformal}
\subsection{Formally}
\begin{mathpar}
\inferrule{}{\annotateliteral{\lint(n)}\typearrow \TInt(\langle[\ConstraintExact(\ELInt{n})]\rangle)}
\and
\inferrule{}{\annotateliteral{\lbool(\Ignore)}\typearrow \TBool}
\and
\inferrule{}{\annotateliteral{\lreal(\Ignore)}\typearrow \TReal}
\and
\inferrule{}{\annotateliteral{\lstring(\Ignore)}\typearrow \TString}
\and
\inferrule{
  n \eqdef |\bits|
}{
  \annotateliteral{\lbitvector(\bits)}\typearrow \TBits(\ELInt{n}, \emptylist)
}
\end{mathpar}
\end{emptyformal}

\section{TypingRule.ExpressionList \label{sec:TypingRule.ExpressionList}}
\hypertarget{def-annotateexprs}{}
The function
\[
  \annotateexprlist{\overname{\staticenvs}{\tenv} \aslsep \overname{\expr^*}{\exprs}}
  \aslto \overname{(\ty \times \expr)^*}{\typedexprs}
  \cup \overname{\TTypeError}{\TypeErrorConfig}
\]
annotates a list of expressions in from left to right.

\subsection{Prose}
One of the following applies:
\begin{itemize}
  \item All of the following apply (\textsc{empty}):
  \begin{itemize}
    \item $\exprs$ is empty;
    \item $\typedexprs$ is empty.
  \end{itemize}

  \item All of the following apply (\textsc{non\_empty}):
  \begin{itemize}
    \item $\exprs$ has $\ve$ as its head expression and $\exprsone$ as its tail;
    \item annotating $\ve$ in $\tenv$ yields the pair $\typedexpr$ consisting of a type and an expression
     \ProseOrTypeError;
    \item annotating the expression list $\exprsone$ in $\tenv$ yields
    $\typedexprs$ \ProseOrTypeError;
    \item $\typedexprs$ is the list with $\typedexpr$ as its head
    and $\typedexprs$ as its tail.
  \end{itemize}
\end{itemize}

\begin{emptyformal}
\subsection{Formally}
\begin{mathpar}
\inferrule[empty]{}
{
  \annotateexprlist{\tenv, \emptylist} \typearrow \emptylist
}
\and
\inferrule[non\_empty]{
  \annotateexpr{\tenv, \ve} \typearrow \typedexpr \OrTypeError\\
  \annotateexprlist{\tenv, \exprsone} \typearrow \typedexprs \OrTypeError\\
  \typedexprs \eqdef [\typedexpr] + \typedexprsone
}
{
  \annotateexprlist{\tenv, [\ve] + \exprsone} \typearrow \typedexprs
}
\end{mathpar}
\end{emptyformal}

\section{TypingRule.ReduceSlicesToCall \label{sec:TypingRule.ReduceSlicesToCall}}
\hypertarget{def-reduceslicestocall}{}
The function
\[
  \reduceslicestocall(\overname{\staticenvs}{\tenv} \aslsep \overname{\expr}{\ve} \aslsep \overname{\slice^*}{\slices})
  \aslto
  \langle (\overname{\identifier}{\name} \times \overname{\expr^*}{\vargs})\rangle
  \cup \overname{\TTypeError}{\TypeErrorConfig}
\]
checks whether the expression $\ve$ together with the list of slices $\slices$ constitute
a call to a subprogram in $\tenv$.
If so, it returns a pair consisting of the name of the called subprogram --- $\name$ ---
and the list of actual arguments --- $\vargs$. Otherwise, it returns $\None$.
The result is a type error, if one is detected.

\subsection{Prose}
One of the following applies:
\begin{itemize}
  \item All of the following apply (\textsc{yes}):
  \begin{itemize}
    \item $\ve$ is a variable expression for $\vx$, that is, $\EVar(\vx)$;
    \item determining whether $\vx$ is a subprogram name with $\slices$ as its actual arguments
    via $\tododefine{should\_slices\_reduce\_to\_call}$
    yields a list of actual argument expressions $\vargs$ \ProseOrTypeError;
    \item $\name$ is $\vx$;
    \item the result is $\langle (\name, \vle)\rangle$.
  \end{itemize}

  \item All of the following apply (\textsc{no}):
  \begin{itemize}
    \item $\ve$ is a variable expression for $\vx$, that is, $\EVar(\vx)$;
    \item determining whether $\vx$ is a subprogram name with $\slices$ as its actual arguments
    via $\tododefine{should\_slices\_reduce\_to\_call}$
    yields $\None$;
    \item the result is $\None$.
  \end{itemize}

  \item All of the following apply (\textsc{non\_var}):
  \begin{itemize}
    \item $\ve$ is a variable expression for $\vx$, that is, $\EVar(\vx)$;
    \item the result is $\None$.
  \end{itemize}
\end{itemize}

\CodeSubsection{\ReduceSlicesToCallBegin}{\ReduceSlicesToCallEnd}{../Typing.ml}

\begin{emptyformal}
\subsection{Formally}
\begin{mathpar}
\inferrule[yes]{
  \tododefine{should\_slices\_reduce\_to\_call}(\tenv, \vx, \slices) \typearrow \langle \vargs \rangle\\
  \name \eqdef \vx
}
{
  \reduceslicestocall(\tenv, \EVar(\vx), \slices) \typearrow \langle (\name, \vargs)\rangle
}
\and
\inferrule[no]{
  \tododefine{should\_slices\_reduce\_to\_call}(\tenv, \vx, \slices) \typearrow \None
}
{
  \reduceslicestocall(\tenv, \EVar(\vx), \slices) \typearrow \None
}
\and
\inferrule[non\_var]{
  \astlabel(\ve) \neq \EVar
}
{
  \reduceslicestocall(\tenv, \ve, \slices) \typearrow \None
}
\end{mathpar}
\end{emptyformal}

\hypertarget{def-annotatestaticconstrainedinteger}{}
\section{TypingRule.StaticConstrainedInteger \label{sec:TypingRule.StaticConstrainedInteger}}

The function
\[
  \annotatestaticconstrainedinteger(\overname{\staticenvs}{\tenv} \aslsep \overname{\expr}{\ve}) \aslto
  \overname{\expr}{\vepp} \cup \overname{\TTypeError}{\TypeErrorConfig}
\]
annotates a \staticallyevaluable\  integer expression $\ve$ in the static environment $\tenv$
and returns the annotated expression $\vepp$.
A type error is returned, if one is detected.

\subsection{Prose}
All of the following apply:
\begin{itemize}
  \item annotating the expression $\ve$ in $\tenv$ yields $ (\vt, \vep)$ \ProseOrTypeError;
  \item determining whether $\vt$ is a statically \constrainedinteger\ in $\tenv$ yields $\True$ \ProseOrTypeError;
  \item determining whether $\vep$ is \staticallyevaluable\  in $\tenv$ yields $\True$ \ProseOrTypeError;
  \item symbolically reducing the expression $\vep$ yields $\vepp$.
\end{itemize}

\CodeSubsection{\StaticConstrainedIntegerBegin}{\StaticConstrainedIntegerEnd}{../Typing.ml}

\subsection{Formally}
\begin{mathpar}
\inferrule{
  \annotateexpr{\tenv, \ve} \typearrow (\vt, \vep) \OrTypeError\\
  \checkconstrainedinteger(\tenv, \vt) \typearrow \True \OrTypeError\\
  \tododefine{check\_statically\_evaluable}(\tenv, \vep) \typearrow \True \OrTypeError\\
  \tododefine{reduce\_expr}(\tenv, \vep) \typearrow \vepp
}{
  \annotatestaticconstrainedinteger(\tenv, \ve) \typearrow \vepp
}
\end{mathpar}

%%%%%%%%%%%%%%%%%%%%%%%%%%%%%%%%%%%%%%%%%%%%%%%%%%%%%%%%%%%%%%%%%%%%%%%%%%%%%%%%%%%%
\chapter{Typing of Left-Hand-Side Expressions}
%%%%%%%%%%%%%%%%%%%%%%%%%%%%%%%%%%%%%%%%%%%%%%%%%%%%%%%%%%%%%%%%%%%%%%%%%%%%%%%%%%%%
\hypertarget{def-annotatelexpr}{}
The function
\[
  \annotatelexpr{
    \overname{\staticenvs}{\tenv} \aslsep
    \overname{\lexpr}{\vle} \aslsep
    \overname{\ty}{\vte}} \aslto
    \overname{\lexpr}{\newle} \cup \TTypeError
\]
annotates a left-hand side expression $\vle$ in an environment $\tenv$, assuming $\vle$
to be the type of the corresponding right-hand-side expression,
resulting in an annotated expression $\newle$.
The result is a type error, if one is detected.
One of the following applies:
\begin{itemize}
\item TypingRule.LEDiscard (see \secref{TypingRule.LEDiscard}),
\item TypingRule.LELocalVar (see \secref{TypingRule.LELocalVar}),
\item TypingRule.LEGlobalVar (see \secref{TypingRule.LEGlobalVar}),
\item TypingRule.LEDestructuring (see \secref{TypingRule.LEDestructuring}),
\item TypingRule.LESlice (see \secref{TypingRule.LESlice}),
\item TypingRule.LESetArray (see \secref{TypingRule.LESetArray}),
\item TypingRule.LESetBadStructuredField (see \secref{TypingRule.LESetBadStructuredField}),
\item TypingRule.LESetStructuredField (see \secref{TypingRule.LESetStructuredField}),
\item TypingRule.LESetBadBitField (see \secref{TypingRule.LESetBadBitField}),
\item TypingRule.LESetBitField (see \secref{TypingRule.LESetBitField}),
\item TypingRule.LESetBitFieldNested (see \secref{TypingRule.LESetBitFieldNested}),
\item TypingRule.LESetBitFieldTyped (see \secref{TypingRule.LESetBitFieldTyped}),
\item TypingRule.LESetBadField (see \secref{TypingRule.LESetBadField}),
\item TypingRule.LEConcat (see \secref{TypingRule.LEConcat}).
\end{itemize}

We also make use of the helper tuple
TypingRule.LEBits (see \secref{TypingRule.LEBits}).

\hypertarget{def-rexpr}{}
Some of the rules require viewing left-hand-side expressions as their corresponding right-hand side expressions.
The correspondence is defined in the ASL Syntax Reference~\cite[Chapter 5]{ASLAbstractSyntaxReference}
and given by the function $\torexpr : \lexpr \rightarrow \expr$.

\section{TypingRule.LEDiscard \label{sec:TypingRule.LEDiscard}}

\subsection{Prose}
All of the following apply:
\begin{itemize}
\item $\vle$ denotes an expression that can be discarded, that is, $\LEDiscard$;
\item $\newle$ is $\vle$.
\end{itemize}

\subsection{Example}

\CodeSubsection{\LEDiscardBegin}{\LEDiscardEnd}{../Typing.ml}

\begin{emptyformal}
\subsection{Formally}
\begin{mathpar}
\inferrule{}{
  \annotatelexpr{\tenv, \LEDiscard, \vte} \typearrow \vle
}
\end{mathpar}
\end{emptyformal}

\isempty{\subsection{Comments}}

\section{TypingRule.LELocalVar \label{sec:TypingRule.LELocalVar}}

\subsection{Prose}
All of the following apply:
\begin{itemize}
  \item $\vle$ denotes a local variable $\vx$, that is, $\LEVar(\vx)$;
  \item $\vx$ is locally declared as a mutable variable of type $\tty$ in $\tenv$;
  \item determining whether $\tty$ \typesatisfies\ $\vte$ in $\tenv$ yields $\True$ \ProseOrTypeError;
  \item $\newle$ is $\vle$.
\end{itemize}

\subsection{Example}

\CodeSubsection{\LELocalVarBegin}{\LELocalVarEnd}{../Typing.ml}

\begin{emptyformal}
\subsection{Formally}
\begin{mathpar}
\inferrule{
  \vle \eqname \LEVar(\vx)\\
  L^\tenv.\localstoragetypes(\id) = (\tty, \LDKVar)\\
  \checktypesat(\tenv, \vte, \tty) \typearrow \True \OrTypeError
}{
  \annotatelexpr{\tenv, \vle, \vte} \typearrow \vle
}
\end{mathpar}

\end{emptyformal}

\subsection{Comments}
  This is related to \identr{WDGQ}, \identr{GNTS}, \identi{MMKF},
  \identi{DGWJ}, \identi{KKCC} and \identr{LXQZ}.

\section{TypingRule.LEGlobalVar \label{sec:TypingRule.LEGlobalVar}}

\subsection{Prose}
All of the following apply:
\begin{itemize}
  \item $\vle$ denotes a local variable $\vx$, that is, $\LEVar(\vx)$;
  \item $\vx$ is globally declared as a mutable variable of type $\tty$ in $\tenv$;
  \item determining whether $\tty$ \typesatisfies\ $\vte$ in $\tenv$ yields $\True$ \ProseOrTypeError;
  \item $\newle$ is $\vle$.
\end{itemize}

\subsection{Example}

\CodeSubsection{\LEGlobalVarBegin}{\LEGlobalVarEnd}{../Typing.ml}

\begin{emptyformal}
\subsection{Formally}
\begin{mathpar}
\inferrule{
  \vle \eqname \LEVar(\vx)\\
  G^\tenv.\globalstoragetypes(\vx) = (\tty, \GDKVar)\\
  \checktypesat(\tenv, \vte, \tty) \typearrow \True \OrTypeError
 }{
  \annotatelexpr{\tenv, \vle, \vte} \typearrow \newle
}
\end{mathpar}
\end{emptyformal}

\subsection{Comments}
  This is related to \identr{WDGQ}.

\section{TypingRule.LEDestructuring \label{sec:TypingRule.LEDestructuring}}

\subsection{Prose}
All of the following apply:
\begin{itemize}
  \item $\vle$ denotes a tuple of left-hand-side expressions $\les$, that is, $\LEDestructuring(\les)$;
  \item $\les$ is a list $\ve_{1..k}$;
  \item $\vte$ is a tuple type $\subtys$;
  \item determining whether $\les$ and $\subtys$ have the same length yields $\True$ \ProseOrTypeError;
  \item $\subtys$ is the list of types $\vt_{1..k}$;
  \item annotating the left-hand-side expression $\ve_i$ with the type $\vt_i$, for $i=1..k$, yields $\vep_i$ \ProseOrTypeError;
  \item the list of expressions $\lesp$ is $\vep_i$, for $i=1..k$;
  \item $\newle$  is the list of left-hand-side expressions $\lesp$, that is, $\LEDestructuring(\lesp)$.
\end{itemize}

\subsection{Example}

\CodeSubsection{\LEDestructuringBegin}{\LEDestructuringEnd}{../Typing.ml}

\begin{emptyformal}
\subsection{Formally}
\begin{mathpar}
\inferrule{
  \vle \eqname \LEDestructuring(\les)\\
  \les \eqname [\ve_{1..k}]\\
  \checktrans{\astlabel(\vle) = \TTuple}{TupleTypeExpected} \checktransarrow \True \OrTypeError\\\\
  \vte \eqname \TTuple(\subtys)\\
  \equallength(\les, \subtys) \typearrow \vb\\
  \checktrans{\vb}{DifferentLengths} \checktransarrow \True \OrTypeError\\\\
  \subtys \eqname [\vt_{1..k}]\\
  i=1..k: \annotatelexpr{\tenv, \ve_i,\vt_i} \typearrow \vep_i \OrTypeError\\
  \lesp \eqname [i=1..k: \vep_i]
}{
  \annotatelexpr{\tenv, \vle, \vte} \typearrow \LEDestructuring(\lesp)
}
\end{mathpar}
\end{emptyformal}

\isempty{\subsection{Comments}}

\section{TypingRule.LESlice \label{sec:TypingRule.LESlice}}

\subsection{Prose}
All of the following apply:
\begin{itemize}
  \item $\vle$ denotes the slicing of a left-hand-side expression $\vleone$ by the slices $\slices$, that is, $\LESlice(\vleone, \slices)$;
  \item annotating the right-hand-side expression corresponding to $\vleone$ in $\tenv$ yields \\
        $(\vtleone, \Ignore)$ \ProseOrTypeError;
  \item $\vtleone$ is a bitvector type;
  \item annotating the left-hand-side expression $\vleone$ in $\tenv$ yields $\vletwo$ \ProseOrTypeError;
  \item obtaining the width of the slices $\slices$ in $\tenv$ and simplifying them yields $\vwidth$;
  \item $\vt$ is the bitvector type of width $\width$ and empty list of bitfields;
  \item checking whether $\vte$ \typesatisfies\ $\vt$ yields $\True$ \ProseOrTypeError;
  \item annotating $\slices$ in $\tenv$ yields $\slicestwo$ \ProseOrTypeError;
  \item checking that the slices $\slicestwo$ are all disjoint yields $\True$ \ProseOrTypeError;
  \item $\newle$ is the slicing of $\vletwo$ by $\slicestwo$, that is, $\LESlice(\vletwo, \slicestwo)$.
\end{itemize}

\subsection{Example}

\CodeSubsection{\LESliceBegin}{\LESliceEnd}{../Typing.ml}

\begin{emptyformal}
\subsection{Formally}
\begin{mathpar}
\inferrule{
  \annotateexpr{\tenv, \torexpr(\vleone)} \typearrow (\vtleone, \Ignore)\\
  \tstruct(\tenv, \vtleone) \typearrow \structtleone \OrTypeError\\
  \astlabel(\structtleone) = \TBits\\
  \annotatelexpr{\tenv, \vleone, \vtleone} \typearrow \vletwo\\
  \sliceswidth(\tenv, \slices) \typearrow \widthp\\
  \tododefine{reduce\_expr}(\widthp) \typearrow \vwidth\\
  \vt \eqdef \TBits(\vwidth, \emptylist)\\
  \checktypesat(\tenv, \vte, \vt) \typearrow \True \OrTypeError\\
  \annotateslices(\tenv, \slices) \typearrow \slicestwo \OrTypeError\\
  \tododefine{check\_disjoint\_slices}(\tenv, \slicestwo) \typearrow \True \OrTypeError\\
  \newle \eqdef \LESlice(\vletwo, \slicestwo)
}{
  \annotatelexpr{\tenv, \overname{\LESlice(\vleone, \slices)}{\vle}, \vte} \typearrow \newle
}
\end{mathpar}
\end{emptyformal}

\isempty{\subsection{Comments}}

\section{TypingRule.LESetArray \label{sec:TypingRule.LESetArray}}

\subsection{Prose}
All of the following apply:
\begin{itemize}
  \item $\vle$ denotes the slicing of a left-hand-side expression $\vleone$ by the slices $\slices$, that is, $\LESlice(\vleone, \slices)$;
  \item annotating the right-hand-side expression corresponding to $\vleone$ in $\tenv$ yields \\ $(\vtleone, \Ignore)$ \ProseOrTypeError;
  \item obtaining the \structure\ of $\vtleone$ in $\tenv$ yields an array type of size $\size$ and element type $\vt$, that is, $\TArray(\size, \vt)$ \ProseOrTypeError;
  \item annotating the left-hand-side expression $\vleone$ with type $\vtleone$ in $\tenv$ yields $\vletwo$ \ProseOrTypeError;
  \item determining that $\vte$ \typesatisfies\ $\vt$ in $\tenv$ yields $\True$ \ProseOrTypeError;
  \item determining whether $\slices$ is a single slice with index expression $\eindex$ yields $\True$ \ProseOrTypeError;
  \item annotating the index expression $\eindex$ in $\tenv$ yields $(\tindexp, \eindexp)$ \ProseOrTypeError;
  \item determining the array length type of $\size$ in $\tenv$ (via $\typeofarraylength$) yields $\wantedtindex$;
  \item determining whether $\tindexp$ \typesatisfies\ $\wantedtindex$ in $\tenv$ yields $\True$ \ProseOrTypeError;
  \item $\newle$ is an access to array $\vletwo$ at index $\eindexp$, that is, \\ $\LESetArray(\vletwo, \eindexp)$.
\end{itemize}

\subsection{Example}

\CodeSubsection{\LESetArrayBegin}{\LESetArrayEnd}{../Typing.ml}

\begin{emptyformal}
\subsection{Formally}
\begin{mathpar}
\inferrule{
  \annotateexpr{\tenv, \torexpr(\vleone)} \typearrow (\vtleone, \Ignore) \OrTypeError\\
  \tstruct(\tenv, \vtleone) \typearrow \TArray(\size, \vt) \OrTypeError\\
  \annotatelexpr{\tenv, \vleone, \vtleone} \typearrow \vletwo \OrTypeError\\
  \checktypesat(\tenv, \vte, \vt) \typearrow \True \OrTypeError\\
  \checktrans{|\slices| = 1}{ArraySliceShouldBeSingleIndex} \checktransarrow \True \OrTypeError\\\\
  \slices \eqname [\vs]\\
  \checktrans{\astlabel(\vs) = \SliceSingle}{ArraySliceShouldBeSingleIndex} \checktransarrow \True \OrTypeError\\\\
  \vs \eqname \SliceSingle(\eindex)\\
  \annotateexpr{\tenv, \eindex} \typearrow (\tindexp, \eindexp) \OrTypeError\\
  \typeofarraylength(\tenv, \size) \typearrow \wantedtindex\\
  \checktypesat(\tenv, \tindexp, \wantedtindex) \typearrow \True \OrTypeError\\
  \newle \eqdef \LESetArray(\vletwo, \eindexp)
}{
  \annotatelexpr{\tenv, \overname{\LESlice(\vleone, \slices)}{\vle}, \vte} \typearrow \newle
}
\end{mathpar}
\end{emptyformal}

\isempty{\subsection{Comments}}

\section{TypingRule.LESetBadStructuredField \label{sec:TypingRule.LESetBadStructuredField}}

\subsection{Prose}
All of the following apply:
\begin{itemize}
  \item $\vle$ denotes the access to the field named $\field$ in $\vleone$, that is, \\ $\LESetField(\vleone, \field)$;
  \item annotating the right-hand-side expression corresponding to $\vleone$ in $\tenv$ yields $\vletwo$ \ProseOrTypeError;
  \item the \structure\ of $\vtleone$ in $\tenv$ is either a record type or an exception type with list of fields $\fields$ \ProseOrTypeError;
  \item $\field$ is not associated with any type in $\fields$;
  \item the result is an error indicating that the field $\field$ is missing from the type of $\vleone$.
\end{itemize}

\subsection{Example}

\CodeSubsection{\LESetBadStructuredFieldBegin}{\LESetBadStructuredFieldEnd}{../Typing.ml}

\begin{emptyformal}
\subsection{Formally}
\begin{mathpar}
\inferrule{
  \annotateexpr{\tenv, \torexpr(\vleone)} \typearrow (\vtleone, \Ignore) \OrTypeError\\
  \annotatelexpr{\tenv, \vleone, \vtleone} \typearrow \vletwo \OrTypeError\\
  \tstruct(\tenv, \vtleone) \typearrow L(\fields) \OrTypeError\\
  L \in \{\TException, \TRecord\}\\
  \assocopt(\fields, \field) \typearrow \None
}{
  \annotatelexpr{\tenv, \overname{\LESetField(\vleone, \field)}{\vle}, \vte} \typearrow \TypeErrorVal{MissingField}
}
\end{mathpar}
\end{emptyformal}

\isempty{\subsection{Comments}}

\section{TypingRule.LESetStructuredField \label{sec:TypingRule.LESetStructuredField}}

\subsection{Prose}
All of the following apply:
\begin{itemize}
  \item $\vle$ denotes the access to the field named \texttt{field} in $\vleone$;
  \item annotating the right-hand-side expression corresponding to $\vleone$ in $\tenv$ yields \\ $(\vtleone, \Ignore)$ \ProseOrTypeError;
  \item annotating the left-hand-side expression  $\vleone$ with type $\vtleone$ in $\tenv$ yields $\vletwo$ \ProseOrTypeError;
  \item obtaining the \structure\ of $\vtleone$ in $\tenv$ yields either a record type or an exception type with fields $\fields$ \ProseOrTypeError;
  \item the type associated with the field $\field$ in $\fields$ is $\vt$;
  \item determining whether $\vte$ \typesatisfies\ $\vt$ yields $\True$ \ProseOrTypeError;
  \item $\newle$ is the access to the field $\field$ in $\vletwo$, that is, $\LESetField(\vletwo, \field)$.
\end{itemize}

\subsection{Example}

\CodeSubsection{\LESetStructuredFieldBegin}{\LESetStructuredFieldEnd}{../Typing.ml}

\begin{emptyformal}
\subsection{Formally}
\begin{mathpar}
\inferrule{
  \annotateexpr{\tenv, \torexpr(\vleone)} \typearrow (\vtleone, \Ignore) \OrTypeError\\
  \annotatelexpr{\tenv, \vleone, \vtleone} \typearrow \vletwo \OrTypeError\\
  \tstruct(\tenv, \vtleone) \typearrow L(\fields) \OrTypeError\\
  L \in \{\TException, \TRecord\}\\
  \assocopt(\fields, \field) \typearrow \langle\vt\rangle\\
  \checktypesat(\tenv, \vte, \vt) \typearrow \True \OrTypeError\\\\
  \newle \eqdef \LESetField(\vletwo, \field)
}{
  \annotatelexpr{\tenv, \overname{\LESetField(\vleone, \field)}{\vle}, \vte} \typearrow \newle
}
\end{mathpar}
\end{emptyformal}

\isempty{\subsection{Comments}}

\section{TypingRule.LESetBadBitField \label{sec:TypingRule.LESetBadBitField}}

\subsection{Prose}
All of the following apply:
\begin{itemize}
  \item $\vle$ denotes the access to the field named $\field$ in $\vleone$, that is, \\ $\LESetField(\vleone, \field)$;
  \item annotating the right-hand-side expression corresponding to $\vleone$ in $\tenv$ yields $\vtleone$ \ProseOrTypeError;
  \item annotating the left-hand-side expression $\vleone$ in $\tenv$ yields $\vletwo$ \ProseOrTypeError;
  \item obtaining the \structure\ of $\vtleone$ in $\tenv$ yields a bitvector type with bitfields \\ $\bitfields$, that is,
        $\TBits(\Ignore, \bitfields)$ \ProseOrTypeError;
  \item find whether a bitfield $\field$ exists in $\bitfields$ yields $\None$;
  \item the result is a type error indicating that the field $\field$ is missing from the type of $\vleone$.
\end{itemize}

\subsection{Example}


\CodeSubsection{\LESetBadBitFieldBegin}{\LESetBadBitFieldEnd}{../Typing.ml}

\begin{emptyformal}
\subsection{Formally}
\begin{mathpar}
\inferrule{
  \annotateexpr{\tenv, \torexpr(\vleone)} \typearrow (\vtleone, \Ignore) \OrTypeError\\
  \annotatelexpr{\tenv, \vleone, \vtleone} \typearrow \vletwo \OrTypeError\\
  \tstruct(\tenv, \vtleone) \typearrow \TBits(\Ignore, \bitfields) \OrTypeError\\
  \tododefine{find\_bitfield\_opt}(\bitfields, \field) \typearrow \None
}{
  \annotatelexpr{\tenv, \overname{\LESetField(\vleone, \field)}{\vle}, \vte} \typearrow \TypeErrorVal{MissingField}
}
\end{mathpar}
\end{emptyformal}

\isempty{\subsection{Comments}}

\section{TypingRule.LESetBitField \label{sec:TypingRule.LESetBitField}}

\subsection{Prose}
All of the following apply:
\begin{itemize}
\item $\vle$ denotes the access to the field named $\field$ in $\vleone$, that is, \\ $\LESetField(\vleone, \field)$;
\item annotating the right-hand-side expression corresponding to $\vleone$ in $\tenv$ yields \\ $(\vtleone, \Ignore)$ \ProseOrTypeError;
\item annotating the left-hand-side expression $\vleone$ in $\tenv$ yields $\vletwo$ \ProseOrTypeError;
\item obtaining the \structure\ of $\vtleone$ in $\tenv$ yields a bitvector type with with bitfields $\bitfields$ \ProseOrTypeError;
\item looking for $\field$ in $\bitfields$ yields a bitfield with corresponding slices $\slices$, that is, $\BitFieldSimple(\Ignore, \slices)$;
\item $\vw$ is the width of $\slices$;
\item $\vt$ is defined as the bitvector type of width $\vw$ and empty list of bitfields, that is, $\TBits(\vw, \emptylist)$;
\item checking whether $\vt$ \typesatisfies\ $\vte$ in $\tenv$ yields $\True$ \ProseOrTypeError;
\item $\vletwo$ is defined as the slicing of $\vleone$ by $\slices$, that is, $\LESlice(\vleone, \slices)$;
\item annotating the left-hand-side expression $\vletwo$ in $\tenv$ yields $\newle$ \ProseOrTypeError.
\end{itemize}

\subsection{Example}

\CodeSubsection{\LESetBitFieldBegin}{\LESetBitFieldEnd}{../Typing.ml}

\begin{emptyformal}
\subsection{Formally}
\begin{mathpar}
\inferrule{
  \annotateexpr{\tenv, \torexpr(\vleone)} \typearrow (\vtleone, \Ignore) \OrTypeError\\
  \annotatelexpr{\tenv, \vleone, \vtleone} \typearrow \vletwo \OrTypeError\\
  \tstruct(\tenv, \vtleone) \typearrow \TBits(\Ignore, \bitfields) \OrTypeError\\
  \tododefine{find\_bitfield\_opt}(\bitfields, \field) \typearrow \langle \BitFieldSimple(\Ignore, \slices) \rangle\\
  \sliceswidth(\tenv, \vslices) \typearrow \vw\\
  \vt \eqdef \TBits(\vw, \emptylist)\\
  \checktypesat(\tenv, \vte, \vt) \typearrow \True \OrTypeError\\
  \vletwo \eqdef \LESlice(\vleone, \slices)\\
  \annotatelexpr{\tenv, \vletwo, \vte} \typearrow \newle \OrTypeError
}{
  \annotatelexpr{\tenv, \overname{\LESetField(\vleone, \field)}{\vle}, \vte} \typearrow \newle
}
\end{mathpar}
\end{emptyformal}

\isempty{\subsection{Comments}}

\section{TypingRule.LESetBitFieldNested \label{sec:TypingRule.LESetBitFieldNested}}

\subsection{Prose}
All of the following apply:
\begin{itemize}
\item $\vle$ denotes the access to the field named $\field$ in $\vleone$, that is, \\ $\LESetField(\vleone, \field)$;
\item annotating the right-hand-side expression corresponding to $\vleone$ in $\tenv$ yields \\ $(\vtleone, \Ignore)$ \ProseOrTypeError;
\item annotating the left-hand-side expression $\vleone$ in $\tenv$ yields $\vletwo$ \ProseOrTypeError;
\item obtaining the \structure\ of $\vtleone$ in $\tenv$ yields a bitvector type with with bitfields $\bitfields$ \ProseOrTypeError;
\item looking for $\field$ in $\bitfields$ yields a nested bitfield with corresponding slices $\slices$ and list of bitfields
      $\bitfieldsp$, that is, \\ $\BitFieldNested(\Ignore, \slices, \bitfieldsp)$;
\item $\vw$ is the width of $\slices$;
\item $\vt$ is defined as the bitvector type of width $\vw$ and list of bitfields $\bitfieldsp$, that is, $\TBits(\vw, \bitfieldsp)$;
\item checking whether $\vt$ \typesatisfies\ $\vte$ in $\tenv$ yields $\True$ \ProseOrTypeError;
\item $\vletwo$ is defined as the slicing of $\vleone$ by $\slices$, that is, $\LESlice(\vleone, \slices)$;
\item annotating the left-hand-side expression $\vletwo$ in $\tenv$ yields $\newle$ \ProseOrTypeError.
\end{itemize}

\subsection{Example}

\CodeSubsection{\LESetBitFieldNestedBegin}{\LESetBitFieldNestedEnd}{../Typing.ml}

\begin{emptyformal}
\subsection{Formally}
\begin{mathpar}
\inferrule{
  \annotateexpr{\tenv, \torexpr(\vleone)} \typearrow (\vtleone, \Ignore) \OrTypeError\\
  \annotatelexpr{\tenv, \vleone, \vtleone} \typearrow \vletwo \OrTypeError\\
  \tstruct(\tenv, \vtleone) \typearrow \TBits(\Ignore, \bitfields) \OrTypeError\\
  \tododefine{find\_bitfield\_opt}(\bitfields, \field) \typearrow \langle \BitFieldNested(\Ignore, \slices, \bitfieldsp) \rangle\\
  \sliceswidth(\tenv, \vslices) \typearrow \vw\\
  \vt \eqdef \TBits(\vw, \bitfieldsp)\\
  \checktypesat(\tenv, \vte, \vt) \typearrow \True \OrTypeError\\
  \vletwo \eqdef \LESlice(\vleone, \slices)\\
  \annotatelexpr{\tenv, \vletwo, \vte} \typearrow \newle \OrTypeError
}{
  \annotatelexpr{\tenv, \overname{\LESetField(\vleone, \field)}{\vle}, \vte} \typearrow \newle
}
\end{mathpar}
\end{emptyformal}

\isempty{\subsection{Comments}}

\section{TypingRule.LESetBitFieldTyped \label{sec:TypingRule.LESetBitFieldTyped}}

\subsection{Prose}
All of the following apply:
\begin{itemize}
  \item $\vle$ denotes the access to the field named $\field$ in $\vleone$, that is, \\ $\LESetField(\vleone, \field)$;
  \item annotating the right-hand-side expression corresponding to $\vleone$ in $\tenv$ yields \\ $(\vtleone, \Ignore)$ \ProseOrTypeError;
  \item annotating the left-hand-side expression $\vleone$ in $\tenv$ yields $\vletwo$ \ProseOrTypeError;
  \item obtaining the \structure\ of $\vtleone$ in $\tenv$ yields a bitvector type with with bitfields $\bitfields$ \ProseOrTypeError;
  \item looking for $\field$ in $\bitfields$ yields a typed bitfield with corresponding slices $\slices$ and a type $\vt$,
        that is, \\ $\BitFieldType(\Ignore, \vslices, \vt))$;
  \item $\vw$ is the width of $\slices$;
  \item $\vtp$ is defined as the bitvector type of width $\vw$ and an empty list of bitfields, that is, $\TBits(\vw , \emptylist)$;
  \item checking whether $\vtp$ \typesatisfies\ $\vt$ in $\tenv$ yields $\True$ \ProseOrTypeError;
  \item checking whether $\vt$ \typesatisfies\ $\vte$ in $\tenv$ yields $\True$ \ProseOrTypeError;
  \item $\vletwo$ is defined as the slicing of $\vleone$ by $\slices$, that is, $\LESlice(\vleone, \slices)$;
  \item annotating the left-hand-side expression $\vletwo$ in $\tenv$ yields $\newle$ \ProseOrTypeError.
\end{itemize}

\subsection{Example}

\CodeSubsection{\LESetBitFieldTypedBegin}{\LESetBitFieldTypedEnd}{../Typing.ml}

\begin{emptyformal}
\subsection{Formally}
\begin{mathpar}
\inferrule{
  \annotateexpr{\tenv, \torexpr(\vleone)} \typearrow (\vtleone, \Ignore) \OrTypeError\\
  \annotatelexpr{\tenv, \vleone, \vtleone} \typearrow \vletwo \OrTypeError\\
  \tstruct(\tenv, \vtleone) \typearrow \TBits(\Ignore, \bitfields) \OrTypeError\\
  \tododefine{find\_bitfield\_opt}(\bitfields, \field) \typearrow \langle \BitFieldType(\Ignore, \vslices, \vt) \rangle\\
  \sliceswidth(\tenv, \vslices) \typearrow \vw\\
  \vtp \eqdef \TBits(\vw , \emptylist)
  \checktypesat(\tenv, \vtp, \vt) \typearrow \True \OrTypeError\\
  \checktypesat(\tenv, \vte, \vt) \typearrow \True \OrTypeError\\
  \vletwo \eqdef \LESlice(\vleone, \slices)\\
  \annotatelexpr{\tenv, \vletwo, \vte} \typearrow \newle \OrTypeError
}{
  \annotatelexpr{\tenv, \overname{\LESetField(\vleone, \field)}{\vle}, \vte} \typearrow \newle
}
\end{mathpar}
\end{emptyformal}

\isempty{\subsection{Comments}}

\section{TypingRule.LESetBadField \label{sec:TypingRule.LESetBadField}}

\subsection{Prose}
All of the following apply:
\begin{itemize}
  \item $\vle$ denotes the access to the field named $\field$ in $\vleone$, that is, \\ $\LESetField(\vleone, \field)$;
  \item annotating the right-hand-side expression corresponding to $\vleone$ in $\tenv$ yields \\ $(\vtleone, \Ignore)$ \ProseOrTypeError;
  \item annotating the left-hand-side expression $\vleone$ in $\tenv$ yields $\vletwo$ \ProseOrTypeError;
  \item obtaining the \structure\ of $\vtleone$ in $\tenv$ yields a type $\vt$ \ProseOrTypeError;
  \item $\vt$ is neither a record type, an exception type, or a bitvector type;
  \item the result is an error indicating that the type of $\vle$ conflicts with the requirements of a field access expression.
\end{itemize}

\subsection{Example}

\CodeSubsection{\LESetBadFieldBegin}{\LESetBadFieldEnd}{../Typing.ml}

\begin{emptyformal}
\subsection{Formally}
\begin{mathpar}
\inferrule{
  \annotateexpr{\tenv, \torexpr(\vleone)} \typearrow (\vtleone, \Ignore) \OrTypeError\\
  \annotatelexpr{\tenv, \vleone, \vtleone} \typearrow \vletwo \OrTypeError\\
  \tstruct(\tenv, \vtleone) \typearrow \vt \OrTypeError\\
  \astlabel(\vt) \not\in \{\TException, \TRecord, \TBits\}
}{
  \annotatelexpr{\tenv, \overname{\LESetField(\vleone, \field)}{\vle}, \vte} \typearrow \TypeErrorVal{TypeConflict}
}
\end{mathpar}
\end{emptyformal}

\isempty{\subsection{Comments}}

\section{TypingRule.LEConcat \label{sec:TypingRule.LEConcat}}

\subsection{Prose}
All of the following apply:
\begin{itemize}
  \item $\vle$ denotes the concatenation of left-hand-side expressions $\les$, that is, \\ $\LEConcat(\les, \Ignore)$;
  \item annotating the right-hand-side expression corresponding to $\vle$ in $\tenv$ yields \\ $(\vteeq, \Ignore)$ \ProseOrTypeError;
  \item checking whether the bitwidth of $\vteeq$ equals the bitwidth of $\vte$ in $\tenv$ yields $\True$ \ProseOrTypeError;
  \item $\les$ is the list of left-hand-side expressions $\vle_i$, for $i=1..k$;
  \item annotating each left-hand-side expression $\vle_i$ as a bitvector-typed expression (via $\annotatelebits$)
        yields the annotated left-hand-side expression $\vleone_i$ and corresponding bitwidth $\width_i$, for $i=1..k$;
  \item $\lesone$ is defined as the list $\vleone_{1..k}$;
  \item $\widths$ is defined as the list $\width_{1..k}$;
  \item $\newle$ is the concatenation of left-hand-side expressions $\lesone$ with corresponding list of widths $\widths$.
\end{itemize}

\subsection{Example}

\CodeSubsection{\LEConcatBegin}{\LEConcatEnd}{../Typing.ml}

\begin{emptyformal}
\subsection{Formally}
\begin{mathpar}
\inferrule{
  \vle \eqname \LEConcat(\les, \Ignore)\\
  \annotateexpr{\tenv, \torexpr(\vle)} \typearrow (\vteeq, \Ignore) \OrTypeError\\
  \checkbitsequalwidth(\tenv, \vteeq, \vte) \typearrow \True \OrTypeError\\
  \les \eqname \vle_{1..k}\\
  i=1..k: \annotatelebits(\tenv, \vle_i) \typearrow (\vleone_i, \width_i) \OrTypeError\\
  \lesone \eqdef \vleone_{1..k}\\
  \widths \eqdef \width_{1..k}
}{
  \annotatelexpr{\tenv, \vle, \vte} \typearrow \overname{\LEConcat(\lesone, \widths)}{\newle}
}
\end{mathpar}
\end{emptyformal}

\isempty{\subsection{Comments}}

\section{TypingRule.LEBits \label{sec:TypingRule.LEBits}}
\hypertarget{def-annotatelebits}{}
The helper function
\[
  \annotatelebits(\overname{\staticenvs}{\tenv} \aslsep \overname{\lexpr}{\vle})
  \aslto \overname{\lexpr}{\vleone} \times \overname{\N}{\width}
\]
annotates a left-hand-side expression $\vle$, which is checked to be of bitvector type
with width $\width$,
resulting in the annotated expression and $\width$, or a type error, if one is detected.

All of the following apply:
\begin{itemize}
  \item annotating the right-hand-side expression corresponding to $\vle$ in $\tenv$ yields \\ $(\vteone, \Ignore)$ \ProseOrTypeError;
  \item obtaining the \structure\ of $\vteone$ in $\tenv$ yields $\vteonestruct$ \ProseOrTypeError;
  \item checking whether $\vteonestruct$ is a bitvector type yields $\True$ \ProseOrTypeError;
  \item $\vteonestruct$ is a bitvector type with width $\ewidth$;
  \item symbolically reducing $\ewidth$ to a literal yields $\vl$ \ProseOrTypeError;
  \item checking whether $\vl$ is an integer literal yields $\True$ \ProseOrTypeError;
  \item $\vl$ is the integer literal for the integer $\width$;
  \item $\vtetwo$ is defined as the bitvector type of width given by $\width$ and an empty list of bitfields, that is,
        $\TBits(\ELInt{\width}, \emptylist)$;
  \item annotating the left-hand-side expression $\vtetwo$ in $\tenv$ yields $\vleone$ \ProseOrTypeError.
\end{itemize}

\begin{emptyformal}
\subsection{Formally}
\begin{mathpar}
\inferrule{
  \annotateexpr{\tenv, \torexpr(\vleone)} \typearrow (\vteone, \Ignore) \OrTypeError\\
  \tstruct(\tenv, \vteone) \typearrow \vteonestruct \OrTypeError\\
  \checktrans{\astlabel(\vteonestruct) = \TBits}{BitvectorTypeExpected} \checktransarrow \True \OrTypeError\\
  \vteonestruct \eqname \TBits(\ewidth, \Ignore)\\
  \reduceconstants{\tenv, \ewidth} \typearrow \vl\\
  \checktrans{\astlabel(\vl) = \lint}{IntegerLiteralExpected} \checktransarrow \True \OrTypeError\\
  \vl \eqname \lint(\width)\\
  \vtetwo \eqdef \TBits(\ELInt{\width}, \emptylist)\\
  \annotatelexpr{\tenv, \vtetwo} \typearrow \vleone \OrTypeError
}{
  \annotatelebits(\tenv, \vle) \typearrow (\vleone, \width)
}
\end{mathpar}
\end{emptyformal}

%%%%%%%%%%%%%%%%%%%%%%%%%%%%%%%%%%%%%%%%%%%%%%%%%%%%%%%%%%%%%%%%%%%%%%%%%%%%%%%%%%%%
\chapter{Typing of Slices \label{chap:typingslices}}
%%%%%%%%%%%%%%%%%%%%%%%%%%%%%%%%%%%%%%%%%%%%%%%%%%%%%%%%%%%%%%%%%%%%%%%%%%%%%%%%%%%%
\hypertarget{def-annotateslice}{}
The function
\[
  \annotateslice(\overname{\staticenvs}{\tenv} \aslsep \overname{\slice}{\vs})
  \aslto
  \overname{\slice}{\vsp} \cup \overname{\TTypeError}{\TypeErrorConfig}
\]
annotates a slices $\vs$ in the static environment $\tenv$,
resulting in an annotated slice $\vsp$.
A type error is returned, if one is detected.

One of the following applies:
\begin{itemize}
\item TypingRule.SliceSingle (see \secref{TypingRule.SliceSingle}),
\item TypingRule.SliceLength (see \secref{TypingRule.SliceLength}),
\item TypingRule.SliceRange (see \secref{TypingRule.SliceRange}),
\item TypingRule.SliceStar (see \secref{TypingRule.SliceStar}).
\end{itemize}

We also define a rule for typing of a list of slices:
TypingRule.Slices (see \secref{TypingRule.Slices}).

\hypertarget{def-annotateslices}{}
The function
\[
  \annotateslices(\overname{\staticenvs}{\tenv} \aslsep \overname{\slice^*}{\slices})
  \aslto
  \overname{\slice^*}{\slicesp} \cup \overname{\TTypeError}{\TypeErrorConfig}
\]
annotates a list of slices $\slices$ in the static environment $\tenv$,
resulting in an annotated list of slices $\slicesp$.
A type error is returned, if one is detected.

The relevant rule is given by:
\begin{itemize}
  \item TypingRule.Slices (see \secref{TypingRule.Slices})
\end{itemize}

\section{TypingRule.SliceSingle \label{sec:TypingRule.SliceSingle}}

\subsection{Prose}
All of the following apply:
\begin{itemize}
  \item $\vs$ is a slice at index \vi, that is $\SliceSingle(\vi)$;
  \item annotating the slice at offset $\vi$ of length $1$ yields $\vsp$ \ProseOrTypeError.
\end{itemize}

\subsection{Example}

\CodeSubsection{\SliceSingleBegin}{\SliceSingleEnd}{../Typing.ml}

\begin{emptyformal}
\subsection{Formally}
\begin{mathpar}
\inferrule{
  \annotateslice(\SliceLength(\vi, \eliteral{1})) \typearrow \vsp \OrTypeError
}{
  \annotateslice(\tenv, \SliceSingle(\vi)) \typearrow \vsp
}
\end{mathpar}
\end{emptyformal}

\subsection{Comments}
  \identr{GXKG}: The notation \texttt{b[i]} is syntactic sugar for \texttt{b[i +: 1]}.

\section{TypingRule.SliceLength \label{sec:TypingRule.SliceLength}}
\subsection{Prose}
All of the following apply:
\begin{itemize}
  \item $\vs$ is a slice of length $\elength$ and offset $\eoffset$, that is, $\SliceLength(\eoffset, \elength)$;
  \item annotating the expression $\eoffset$ in $\tenv$ yields $(\toffset, \eoffsetp)$ \ProseOrTypeError;
  \item annotating the \staticallyevaluable\ \constrainedinteger\ expression $\elength$ in $\tenv$ yields
  $\elength$ \ProseOrTypeError;
  \item determining whether $\toffset$ has the \structureofinteger\ yields $\True$ \ProseOrTypeError;
  \item $\vsp$ is the slice at offset $\eoffsetp$ and length $\elength'$, that is,\\
   $\SliceLength(\eoffsetp, \elength')$.
\end{itemize}

\subsection{Example}

\CodeSubsection{\SliceLengthBegin}{\SliceLengthEnd}{../Typing.ml}

\begin{emptyformal}
\subsection{Formally}
\begin{mathpar}
\inferrule{
  \annotateexpr{\tenv, \eoffset} \typearrow (\toffset, \eoffsetp) \OrTypeError\\
  \annotatestaticconstrainedinteger(\tenv, \elength) \typearrow \elengthp \OrTypeError\\
  \checkstructureinteger(\tenv, \eoffsetp, \toffset) \typearrow \True \OrTypeError
}{
  \annotateslice(\tenv, \SliceLength(\eoffset, \elength)) \typearrow \SliceLength(\eoffsetp, \elength')
}
\end{mathpar}
\end{emptyformal}

\isempty{\subsection{Comments}}

\section{TypingRule.SliceRange \label{sec:TypingRule.SliceRange}}

\subsection{Prose}
All of the following apply:
\begin{itemize}
  \item $\vs$ is a slice for the range \texttt{(j, i)}, that is $\SliceRange(\vj, \vi)$;
  \item $\prelength$ is \texttt{i+:(j-i+1)};
  \item annotating the slice at offset $\vi$ of length $\prelength$ yields $\vsp$ \ProseOrTypeError.
\end{itemize}

\subsection{Example}

\CodeSubsection{\SliceRangeBegin}{\SliceRangeEnd}{../Typing.ml}

\begin{emptyformal}
\subsection{Formally}
\begin{mathpar}
\inferrule{
  \staticbinop(\MINUS, \vi, \vi) \typearrow \prelengthp\\
  \staticbinop(\PLUS, \prelengthp, \eliteral{1}) \typearrow \prelength\\
  \annotateslice(\SliceLength(\vi, \prelength)) \typearrow \vsp \OrTypeError
}{
  \annotateslice(\tenv, \SliceRange(\vj, \vi)) \typearrow \vsp
}
\end{mathpar}
\end{emptyformal}

\subsection{Comments}
    \identr{GXKG}: The notation \texttt{b[j:i]} is syntactic sugar for \texttt{b[i+:(j-i+1)]}.

\section{TypingRule.SliceStar \label{sec:TypingRule.SliceStar}}

\subsection{Prose}
All of the following apply:
\begin{itemize}
  \item $\vs$ is a slice \texttt{[factor *: pre\_length]}, that is, $\SliceStar(\factor, \prelength)$;
  \item $\preoffset$ is $\factor * \prelength$;
  \item annotating the slice at offset $\preoffset$ of length $\prelength$ yields $\vsp$ \ProseOrTypeError.
\end{itemize}

\subsection{Example}

\CodeSubsection{\SliceStarBegin}{\SliceStarEnd}{../Typing.ml}

\begin{emptyformal}
\subsection{Formally}
\begin{mathpar}
\inferrule{
  \staticbinop(\MUL, \factor, \prelength) \typearrow \preoffset\\
  \annotateslice(\SliceLength(\preoffset, \prelength)) \typearrow \vsp \OrTypeError
}{
  \annotateslice(\tenv, \SliceStar(\factor, \prelength)) \typearrow \vsp
}
\end{mathpar}
\end{emptyformal}

\subsection{Comments}
    \identr{GXQG}: The notation \texttt{b[i *: n]} is syntactic sugar for \texttt{b[i*n +: n]}

\section{TypingRule.Slices \label{sec:TypingRule.Slices}}
\subsection{Prose}
All of the following apply:
\begin{itemize}
  \item annotating the slices in $\slices$ from left to right yields the list of annotated slices $\slicesp$ \ProseOrTypeError.
\end{itemize}

\subsection{Example}

\begin{emptyformal}
\subsection{Formally}
\begin{mathpar}
\inferrule{
  \vs\in\slices: \annotateslice(\tenv, \vs) \typearrow \vsp \OrTypeError\\\\
  \slicesp = [\vs\in\slices: \vsp]
}{
  \annotateslices(\tenv, \slices) \typearrow \slicesp
}
\end{mathpar}
\end{emptyformal}

\subsection{Comments}

%%%%%%%%%%%%%%%%%%%%%%%%%%%%%%%%%%%%%%%%%%%%%%%%%%%%%%%%%%%%%%%%%%%%%%%%%%%%%%%%%%%%
\chapter{Typing of Patterns}
%%%%%%%%%%%%%%%%%%%%%%%%%%%%%%%%%%%%%%%%%%%%%%%%%%%%%%%%%%%%%%%%%%%%%%%%%%%%%%%%%%%%
\hypertarget{def-annotatepattern}{}
The function
\[
  \annotatepattern(
    \overname{\staticenvs}{\tenv} \aslsep
    \overname{\ty}{\vt} \aslsep
    \overname{\pattern}{\vp}) \aslto \overname{\pattern}{\newp} \cup \overname{\TTypeError}{\TypeErrorConfig}
\]
annotates a pattern $\vp$ in a static environment $\tenv$ given a type $\vt$,
resulting in a pattern $\newp$ or a type error, if one is detected, and one of the following applies:
\begin{itemize}
\item TypingRule.PAll (see \secref{TypingRule.PAll}),
\item TypingRule.PAny (see \secref{TypingRule.PAny}),
\item TypingRule.PGeq (see \secref{TypingRule.PGeq}),
\item TypingRule.PLeq (see \secref{TypingRule.PLeq}),
\item TypingRule.PNot (see \secref{TypingRule.PNot}),
\item TypingRule.PRange (see \secref{TypingRule.PRange}),
\item TypingRule.PSingle (see \secref{TypingRule.PSingle}),
\item TypingRule.PMask (see \secref{TypingRule.PMask}),
\item TypingRule.PTuple (see \secref{TypingRule.PTuple}).
\end{itemize}

\section{TypingRule.PAll \label{sec:TypingRule.PAll}}

\subsection{Prose}
All of the following apply:
\begin{itemize}
  \item $\vp$ is the pattern matching everything, that is, $\PatternAll$;
  \item $\newp$ is $\vp$.
\end{itemize}

\subsection{Example}

\CodeSubsection{\PAllBegin}{\PAllEnd}{../Typing.ml}

\begin{emptyformal}
\subsection{Formally}
\begin{mathpar}
\inferrule{}
{
  \annotatepattern(\tenv, \vt, \PatternAll) \typearrow \PatternAll
}
\end{mathpar}
\end{emptyformal}

\isempty{\subsection{Comments}}

\section{TypingRule.PAny\label{sec:TypingRule.PAny}}

\subsection{Prose}
All of the following apply:
\begin{itemize}
\item $\vp$ is the pattern which matches anything in a list $\vli$, that is, $\PatternAny(\vli)$;
\item annotating each pattern in $\vli$ yields the list of annotated pattern $\newli$ \ProseOrTypeError;
\item $\newp$ is the pattern which matches anything in $\newli$, that is, \\ $\PatternAny(\newli)$.
\end{itemize}

\subsection{Example}

\CodeSubsection{\PAnyBegin}{\PAnyEnd}{../Typing.ml}

\begin{emptyformal}
\subsection{Formally}
\begin{mathpar}
\inferrule{
  \vl\in\vli: \annotatepattern(\tenv, \vt, \vl) \typearrow \vlp \OrTypeError\\\\
  \newli \eqdef [\vl\in\vli: \vlp]
}
{
  \annotatepattern(\tenv, \vt, \PatternAny(\vli)) \typearrow \PatternAny(\newli)
}
\end{mathpar}
\end{emptyformal}

\isempty{\subsection{Comments}}

\section{TypingRule.PGeq \label{sec:TypingRule.PGeq}}

\subsection{Prose}
All of the following apply:
\begin{itemize}
\item $\vp$ is the pattern which matches anything greater than or equal to an expression $\ve$,
that is, $\PatternGeq(\ve)$;
\item annotating the expression $\ve$ in $\tenv$ yields $(\vte, \vep)$ \ProseOrTypeError;
\item determining whether $\vep$ is a \staticallyevaluable\ expression yields $\True$ \ProseOrTypeError;
\item obtaining the \structure\ of $\vt$ in $\tenv$ yields $\vtstruct$ \ProseOrTypeError;
\item obtaining the \structure\ of $\vte$ in $\tenv$ yields $\testruct$ \ProseOrTypeError;
\item $\vb$ is true if and only if $\vtstruct$ and $\testruct$ are both integer types or both real types;
\item if $\vb$ is $\False$ a type error is returned (indicating that the types of $\vt$ and $\vte$
      are inappropriate for the $\GEQ$ operator),
which short-circuits the entire evaluation;
\item $\newp$ is the pattern which matches anything greater than or equal to $\vep$.
\end{itemize}

\subsection{Example}

\CodeSubsection{\PGeqBegin}{\PGeqEnd}{../Typing.ml}

\begin{emptyformal}
\subsection{Formally}
\begin{mathpar}
\inferrule{
  \annotateexpr{\tenv, \ve} \typearrow (\vte, \vep) \OrTypeError\\\\
  \checkstaticallyevaluable(\tenv, \vep) \typearrow \True \OrTypeError\\\\
  \tstruct(\tenv, \vt) \typearrow \vtstruct \OrTypeError\\\\
  \tstruct(\tenv, \vte) \typearrow \testruct \OrTypeError\\\\
  {
    \begin{array}{rl}
      \vb \eqdef& \astlabel(\vtstruct) = \astlabel(\testruct)\ \land\\
                & \astlabel(\vtstruct) \in \{\TInt, \TReal\}
    \end{array}
  }\\
  \checktrans{\vb}{InvalidTypesForBinop} \checktransarrow \True \OrTypeError
}
{
  \annotatepattern(\tenv, \vt, \PatternGeq(\ve)) \typearrow \PatternGeq(\vep)
}
\end{mathpar}
\end{emptyformal}

\isempty{\subsection{Comments}}

\section{TypingRule.PLeq \label{sec:TypingRule.PLeq}}

\subsection{Prose}
All of the following apply:
\begin{itemize}
\item $\vp$ is the pattern which matches anything less than or equal to an expression $\ve$,
that is, $\PatternLeq(\ve)$;
\item annotating the expression $\ve$ in $\tenv$ yields $(\vte, \vep)$ \ProseOrTypeError;
\item determining whether $\vep$ is a \staticallyevaluable\ expression yields $\True$ \ProseOrTypeError;
\item obtaining the \structure\ of $\vt$ in $\tenv$ yields $\vtstruct$ \ProseOrTypeError;
\item obtaining the \structure\ of $\vte$ in $\tenv$ yields $\testruct$ \ProseOrTypeError;
\item $\vb$ is true if and only if $\vtstruct$ and $\testruct$ are both integer types or both real types;
\item if $\vb$ is $\False$ a type error is returned (indicating that the types of $\vt$ and $\vte$
      are inappropriate for the $\LEQ$ operator),
which short-circuits the entire evaluation;
\item $\newp$ is the pattern which matches anything less than or equal to $\vep$.
\end{itemize}

\subsection{Example}

\CodeSubsection{\PLeqBegin}{\PLeqEnd}{../Typing.ml}

\begin{emptyformal}
\subsection{Formally}
\begin{mathpar}
\inferrule{
  \annotateexpr{\tenv, \ve} \typearrow (\vte, \vep) \OrTypeError\\\\
  \checkstaticallyevaluable(\tenv, \vep) \typearrow \True \OrTypeError\\\\
  \tstruct(\tenv, \vt) \typearrow \vtstruct \OrTypeError\\\\
  \tstruct(\tenv, \vte) \typearrow \testruct \OrTypeError\\\\
  {
    \begin{array}{rl}
      \vb \eqdef& \astlabel(\vtstruct) = \astlabel(\testruct)\ \land\\
                & \astlabel(\vtstruct) \in \{\TInt, \TReal\}
    \end{array}
  }\\
  \checktrans{\vb}{InvalidTypesForBinop} \checktransarrow \True \OrTypeError
}
{
  \annotatepattern(\tenv, \vt, \PatternLeq(\ve)) \typearrow \PatternLeq(\vep)
}
\end{mathpar}
\end{emptyformal}

\isempty{\subsection{Comments}}

\section{TypingRule.PNot \label{sec:TypingRule.PNot}}

\subsection{Prose}
Annotating a pattern $\vt$ in an environment $\tenv$ given a type $\vt$ (\texttt{annotate\_pattern}) results in a pattern $\newp$ and all of the following apply:
\begin{itemize}
  \item $\vp$ is the pattern which matches the negation of a pattern $\vq$, that is, $\PatternNot(\vq)$;
  \item annotating $\vq$ in $\tenv$ yields $\newq$ \ProseOrTypeError;
  \item $\newp$ is pattern which matches the negation of $\newq$, that is, $\PatternLeq(\newq)$.
\end{itemize}

\subsection{Example}

\CodeSubsection{\PNotBegin}{\PNotEnd}{../Typing.ml}

\begin{emptyformal}
\subsection{Formally}
\begin{mathpar}
\inferrule{
  \annotatepattern(\tenv, \vq) \typearrow \newq \OrTypeError
}{
  \annotatepattern(\tenv, \vt, \PatternNot(\vq)) \PatternNot \PatternLeq(\newq)
}
\end{mathpar}
\end{emptyformal}

\isempty{\subsection{Comments}}

\section{TypingRule.PRange \label{sec:TypingRule.PRange}}

\subsection{Prose}
All of the following apply:
\begin{itemize}
  \item $\vp$ is the pattern which matches anything within the range given by
  expressions $\veone$ and $\vetwo$, that is, $\PatternRange(\veone, \vetwo)$;
  \item annotating the expression $\veone$ in $\tenv$ yields $(\vteone, \veonep)$ \ProseOrTypeError;
  \item annotating the expression $\vetwo$ in $\tenv$ yields $(\vtetwo, \vetwop)$ \ProseOrTypeError;
  \item determining whether both $\veonep$ and $\vetwop$ are compile-time constant expressions yields $\True$ \ProseOrTypeError;
  \item obtaining the \structure\ for $\vt$, $\vteone$, and $\vtetwo$ yields
        $\vtstruct$, $\vteonestruct$, and $\vtetwostruct$, respectively \ProseOrTypeError;
  \item a check the AST labels of $\vtstruct$, $\vteonestruct$, and $\vtetwostruct$ are all the same and are either
        $\TInt$ or $\TReal$ yields $\True$. Otherwise, the result is a type error indicating that the types of
        $\veone$, $\vetwo$ and the type $\vt$ are inappropriate for a range pattern, which short-circuits the entire evaluation.
  \item $\newp$ is a range pattern with bounds $\veonep$ and $\vetwop$, that is, $\PatternRange(\veonep, \vetwop)$.
\end{itemize}

\subsection{Example}

\CodeSubsection{\PRangeBegin}{\PRangeEnd}{../Typing.ml}

\begin{emptyformal}
\subsection{Formally}
\begin{mathpar}
\inferrule{
  \annotateexpr{\tenv, \veone} \typearrow (\vteone, \veonep) \OrTypeError\\
  \annotateexpr{\tenv, \vetwo} \typearrow (\vtetwo, \vetwop) \OrTypeError\\
  \tstruct(\tenv, \vt) \typearrow \vtstruct \OrTypeError\\
  \tstruct(\tenv, \vteone) \typearrow \vteonestruct \OrTypeError\\
  \tstruct(\tenv, \vtetwo) \typearrow \vtetwostruct \OrTypeError\\
  {
    \begin{array}{rl}
      \vb \eqdef& \astlabel(\vtstruct) = \astlabel(\vteonestruct) = \astlabel(\vtetwostruct)\ \land\\
                & \astlabel(\vtstruct) \in \{\TInt, \TReal\}
    \end{array}
  }\\
  \checktrans{\vb}{InvalidTypesForBinop} \checktransarrow \True \OrTypeError
}{
  \annotatepattern(\tenv, \vt, \PatternRange(\veone, \vetwo)) \typearrow \PatternRange(\veonep, \vetwop)
}
\end{mathpar}
\end{emptyformal}

\isempty{\subsection{Comments}}

\section{TypingRule.PSingle \label{sec:TypingRule.PSingle}}

\subsection{Prose}
All of the following apply:
\begin{itemize}
  \item $\vp$ is the pattern that matches the expression $\ve$, that is, $\PatternSingle(\ve)$;
  \item annotating the expression $\ve$ in $\tenv$ yields $(\vte, \vep)$ \ProseOrTypeError;
  \item obtaining the \structure\ of $\vt$ yields $\vtstruct$ \ProseOrTypeError;
  \item obtaining the \structure\ of $\vte$ yields $\testruct$ \ProseOrTypeError;
  \item One of the following holds:
  \begin{itemize}
    \item All of the following apply (\textsc{t\_bool, t\_real, t\_int}):
    \begin{itemize}
      \item the labels of $\vtstruct$ and $\testruct$ are both either $\TBool$, $\TReal$, or $\TInt$ \ProseOrTypeError;
    \end{itemize}

    \item All of the following apply (\textsc{t\_bits}):
    \begin{itemize}
      \item the labels of $\vtstruct$ and $\testruct$ are both $\TBits$ \ProseOrTypeError;
      \item determining whether the bitwidths of $\vtstruct$ and $\testruct$ are equal yields $\True$ \ProseOrTypeError;
    \end{itemize}

    \item All of the following apply (\textsc{t\_enum}):
    \begin{itemize}
      \item the labels of $\vtstruct$ and $\testruct$ are both $\TEnum$ \ProseOrTypeError;
      \item determining whether the lists of enumeration literals of $\vtstruct$ and $\testruct$ are equal yields $\True$ \ProseOrTypeError;
    \end{itemize}

    \item All of the following apply (\textsc{error}):
    \begin{itemize}
      \item determining whether the labels of $\vtstruct$ and $\testruct$ are the same yields $\True$ \ProseOrTypeError;
      \item the label of $\vtstruct$ is not one of $\TBool$, $\TReal$, $\TInt$, $\TBits$, or $\TEnum$;
      \item the result is a type error indicating that the types $\vt$ and $\vte$ are inappropriate for this pattern.
    \end{itemize}
  \end{itemize}
  \item $\newp$ is the the pattern that matches the expression $\vep$, that is, $\PatternSingle(\vep)$.
\end{itemize}

\subsection{Example}

\CodeSubsection{\PSingleBegin}{\PSingleEnd}{../Typing.ml}

\begin{emptyformal}
\subsection{Formally}
\begin{mathpar}
\inferrule[t\_bool, t\_real, t\_int]{
  \annotateexpr{\tenv, \ve} \typearrow (\vte, \vep) \OrTypeError\\
  \tstruct(\tenv, \vt) \typearrow \vtstruct \OrTypeError\\
  \tstruct(\tenv, \vte) \typearrow \testruct \OrTypeError\\
  \checktrans{\astlabel(\vtstruct) = \astlabel(\testruct)}{InvalidTypesForBinop} \checktransarrow \True \OrTypeError\\\\
  \astlabel(\vtstruct) \in \{\TBool, \TReal, \TInt\}
}{
  \annotatepattern(\tenv, \vt, \PatternSingle(\ve)) \typearrow \PatternSingle(\vep)
}
\end{mathpar}

\begin{mathpar}
\inferrule[t\_bits]{
  \annotateexpr{\tenv, \ve} \typearrow (\vte, \vep) \OrTypeError\\
  \tstruct(\tenv, \vt) \typearrow \vtstruct \OrTypeError\\
  \tstruct(\tenv, \vte) \typearrow \vtestruct \OrTypeError\\
  \astlabel(\vtstruct) = \astlabel(\testruct) = \TBits\\
  \bitwidthequal(\tenv, \vtstruct, \testruct) \typearrow \vb\\
  \checktrans{\vb}{BitvectorsDifferentWidths} \checktransarrow \True \OrTypeError\\
}{
  \annotatepattern(\tenv, \vt, \PatternSingle(\ve)) \typearrow \PatternSingle(\vep)
}
\end{mathpar}

\begin{mathpar}
\inferrule[t\_enum]{
  \annotateexpr{\tenv, \ve} \typearrow (\vte, \vep) \OrTypeError\\
  \tstruct(\tenv, \vt) \typearrow \vtstruct \OrTypeError\\
  \tstruct(\tenv, \vte) \typearrow \vtestruct \OrTypeError\\
  \vtstruct \eqname \TEnum(\vlione)\\
  \vtestruct \eqname \TEnum(\vlitwo)\\
  \checktrans{\vlione = \vlitwo}{EnumDifferentLabels} \checktransarrow \True \OrTypeError\\
}{
  \annotatepattern(\tenv, \vt, \PatternSingle(\ve)) \typearrow \PatternSingle(\vep)
}
\end{mathpar}

\begin{mathpar}
\inferrule[error]{
  \annotateexpr{\tenv, \ve} \typearrow (\vte, \vep) \OrTypeError\\
  \tstruct(\tenv, \vt) \typearrow \vtstruct \OrTypeError\\
  \tstruct(\tenv, \vte) \typearrow \testruct \OrTypeError\\
  \checktrans{\astlabel(\vtstruct) = \astlabel(\testruct)}{InvalidTypesForBinop} \checktransarrow \True \OrTypeError\\\\
  \astlabel(\vtstruct) \in \{\TBool, \TReal, \TInt, \TBits, \TEnum\}
}{
  \annotatepattern(\tenv, \vt, \PatternSingle(\ve)) \typearrow \TypeErrorVal{TypeConflict}
}
\end{mathpar}
\end{emptyformal}

\isempty{\subsection{Comments}}

\section{TypingRule.PMask \label{sec:TypingRule.PMask}}

\subsection{Prose}
All of the following apply:
  \begin{itemize}
  \item $\vp$ is the pattern which matches a mask $\vm$, that is, $\PatternMask(\vm)$;
  \item determining whether $\vt$ has the structure of a bitvector type yields $\True$ \ProseOrTypeError;
  \item $\vn$ is the length of mask $\vm$;
  \item determining whether $\vt$ \typesatisfies\ the bitvector type of length $\vn$ \\
        (that is, $\TBits(\vn, \emptylist)$), yields $\True$ \ProseOrTypeError;
  \item $\newp$ is $\vp$.
\end{itemize}

\subsection{Example}

\CodeSubsection{\PMaskBegin}{\PMaskEnd}{../Typing.ml}

\begin{emptyformal}
\subsection{Formally}
\begin{mathpar}
\inferrule{
  \checkstructurelabel(\tenv, \vt, \TBits) \typearrow \True \OrTypeError\\
  \vn \eqdef |\vm|\\
  \checktypesat(\tenv, \vt, \TBits(\vn, \emptylist)) \typearrow \True \OrTypeError
}{
  \annotatepattern(\tenv, \vt, \PatternMask(\vm)) \typearrow \PatternMask(\vm)
}
\end{mathpar}
\end{emptyformal}

\subsection{Comments}
  This is related to \identi{VMKF}.

\section{TypingRule.PTuple \label{sec:TypingRule.PTuple}}

\subsection{Prose}
All of the following apply:
  \begin{itemize}
  \item $\vp$ is the pattern which matches a tuple $\vli$, that is, $\PatternTuple(\vli)$;
  \item obtaining the \structure\ of $\vt$ yields $\vtstruct$ \ProseOrTypeError;
  \item determining whether $\vtstruct$ is a tuple type yields $\True$ \ProseOrTypeError;
  \item $\vtstruct$ is a tuple type with list of tuple $\vts$;
  \item determining whether $\vts$ is a list of the same size as $\vli$ yields $\True$ \ProseOrTypeError;
  \item annotating each pattern in $\vli$ with the corresponding type in $\vts$ at each position $\vi$
        yields a pattern $\vlip[\vi]$ \ProseOrTypeError;
  \item $\newli$ is the list of annotated patterns $\vlip[\vi]$ at the same positions those of $\vli$;
  \item $\newp$ is the pattern which matches the tuple $\newli$, that is, $\PatternTuple(\newli)$.
  \end{itemize}

\subsection{Example}

\CodeSubsection{\PTupleBegin}{\PTupleEnd}{../Typing.ml}

\begin{emptyformal}
\subsection{Formally}
\begin{mathpar}
\inferrule{
  \tstruct(\tenv, \vt) \typearrow \vtstruct \OrTypeError\\
  \checktrans{\astlabel(\vtstruct) = \TTuple}{TypeConflict} \checktransarrow \True \OrTypeError\\
  \vtstruct \eqname \TTuple(\vts)\\
  \checktrans{\equallength(\vli, \vts)}{InvalidArity} \checktransarrow \True \OrTypeError\\
  \vi\in\listrange(\vli): \annotatepattern(\tenv, \vts[\vi], \vli[\vi]) \typearrow \vlip[i] \OrTypeError\\
  \newli \eqdef \vi\in\listrange(\vli): \vlip[\vi]
}{
  \annotatepattern(\tenv, \vt, \PatternTuple(\vli)) \typearrow \PatternTuple(\newli)
}
\end{mathpar}
\end{emptyformal}

\isempty{\subsection{Comments}}

%%%%%%%%%%%%%%%%%%%%%%%%%%%%%%%%%%%%%%%%%%%%%%%%%%%%%%%%%%%%%%%%%%%%%%%%%%%%%%%%%%%%
\chapter{Typing of Local Declarations}
%%%%%%%%%%%%%%%%%%%%%%%%%%%%%%%%%%%%%%%%%%%%%%%%%%%%%%%%%%%%%%%%%%%%%%%%%%%%%%%%%%%%
\hypertarget{def-annotatelocaldeclitem}{}
The function
\[
  \begin{array}{c}
  \annotatelocaldeclitem{
    \overname{\ty}{\tty} \aslsep
    \overname{\staticenvs}{\tenv} \aslsep
    \overname{\localdeclkeyword}{\ldk} \aslsep
    \overname{\localdeclitem}{\ldi}
   } \aslto\\
  (\overname{\staticenvs}{\newtenv} \aslsep \overname{\localdeclitem}{\newldi})
  \cup \overname{\TTypeError}{\TypeErrorConfig}
  \end{array}
\]
annotates a local declaration item $\ldi$ with a local declaration keyword $\ldk$, given a type $\tty$,
in a static environment $\tenv$ results in $(\newenv, \newldi)$ where $\newenv$ is the modified
static environment and $\newldi$ is the annotated local declaration item.
A type error is returned, if one is detected.

One of the following applies:
\begin{itemize}
\item TypingRule.LDDiscard (see \secref{TypingRule.LDDiscard}),
\item TypingRule.LDVar (see \secref{TypingRule.LDVar}),
\item TypingRule.LDTyped (see \secref{TypingRule.LDTyped}),
\item TypingRule.LDTuple (see \secref{TypingRule.LDTuple}).
\end{itemize}

This is related to \identr{YSPM}.

\section{TypingRule.LDDiscard \label{sec:TypingRule.LDDiscard}}

\subsection{Prose}
All of the following apply:
\begin{itemize}
  \item $\ldi$ is a local declaration which can be discarded, that is, $\LDIDiscard(\None)$;
  \item $\newenv$ is $\tenv$;
  \item $\newldi$ is $\ldi$.
\end{itemize}

\subsection{Example}
\VerbatimInput{../tests/ASLTypingReference.t/TypingRule.LDDiscard.asl}

\CodeSubsection{\LDDiscardBegin}{\LDDiscardEnd}{../Typing.ml}

\begin{emptyformal}
\subsection{Formally}
\begin{mathpar}
\inferrule{}{
  \annotatelocaldeclitem{\tenv, \tty, \LDIDiscard(\None), \ldk} \typearrow (\tenv, \ldi)
}
\end{mathpar}
\end{emptyformal}

\isempty{\subsection{Comments}}

\section{TypingRule.LDVar \label{sec:TypingRule.LDVar}}

\subsection{Prose}
All of the following apply:
\begin{itemize}
  \item $\ldi$ denotes a variable $\vx$, that is, $\LDIVar(\vx)$;
  \item determining whether $\vx$ is not declared in $\tenv$ yields $\True$ \ProseOrTypeError;
  \item $\newenv$ is $\tenv$ modified so that $\vx$ is locally declared to have type $\tty$;
  \item $\newldi$ is the declaration of variable $\vx$.
\end{itemize}

\subsection{Example}
\VerbatimInput{../tests/ASLTypingReference.t/TypingRule.LDVar.asl}

\CodeSubsection{\LDVarBegin}{\LDVarEnd}{../Typing.ml}

\begin{emptyformal}
\subsection{Formally}
\begin{mathpar}
\inferrule{
  \checkvarnotinenv{\tenv, \vx} \typearrow \True \OrTypeError\\
  \addlocal(\tenv, \vx, \tty, \ldk) \typearrow \newtenv
}{
  \annotatelocaldeclitem{\tenv, \tty, \LDIVar(\vx), \ldk} \typearrow (\newtenv, \LDIVar(\vx))
}
\end{mathpar}
\end{emptyformal}

\isempty{\subsection{Comments}}
This is related to \identr{YSPM}, \identd{FXST}.

\section{TypingRule.LDTyped\label{sec:TypingRule.LDTyped}}

\subsection{Prose}
All of the following apply:
\begin{itemize}
  \item $\ldi$ denotes a local declaration item $\ldip$ with local declaration keyword $\ldk$
  and a type $\vt$, that is $\LDITyped(\ldip, \vt)$;
  \item annotating the type $\vt$ in $\tenv$ yields $\vtp$ \ProseOrTypeError;
  \item determining whether $\vtp$ can be initialized with $\tty$ in $\tenv$ yields $\True$ \ProseOrTypeError;
  \item annotating the local declaration item $\ldip$ with the local declaration keyword $\ldk$, given
  the type $\vt$, in the environment $\tenv$, yields $(\newtenv,\newldip)$;
  \item $\newldi$ is the local declaration denoting $\newldip$ and the type $\vtp$, that is, \\
  $\LDITyped(\newldip, \vtp)$.
\end{itemize}

\subsection{Example}
\VerbatimInput{../tests/ASLTypingReference.t/TypingRule.LDTyped.asl}

\CodeSubsection{\LDTypedBegin}{\LDTypedEnd}{../Typing.ml}

\begin{emptyformal}
\subsection{Formally}
\begin{mathpar}
\inferrule{
  \annotatetype{\tenv, \vt} \typearrow \vtp \OrTypeError\\
  \canbeinitializedwith(\tenv, \vtp, \tty) \typearrow \True \OrTypeError\\
  \annotatelocaldeclitem{\tenv, \vtp, \ldip, \ldk} \typearrow (\newtenv, \newldip) \OrTypeError
}
{
  \annotatelocaldeclitem{\tenv, \tty, \LDITyped(\ldip, \vt), \ldk} \typearrow \\
  (\newtenv, \LDITyped(\newldip, \vtp))
}
\end{mathpar}
\end{emptyformal}

\isempty{\subsection{Comments}}

\section{TypingRule.LDTuple\label{sec:TypingRule.LDTuple}}

\subsection{Prose}
All of the following apply:
\begin{itemize}
  \item $\ldi$ denotes a tuple of local declaration items $\ldi_{1..k}$, that is, $\LDITuple(\ldi_{1..k})$;
  \item determining the \structure\ of $\tty$ in $\tenv$ yields $\vtp$ \ProseOrTypeError;
  \item determining whether $\vtp$ is a tuple type yields $\True$ \ProseOrTypeError;
  \item determining whether $\vtp$ the number of elements of $vtp$ is $k$ yields $\True$ \ProseOrTypeError;
  \item annotating the local declaration items in $\ldis$ from right to left with their corresponding
        (that is, with the same index) types $t_{1..k}$ in $\tenv$,
        propagating static environments from one annotation to the next,
        yields the local declaration items $\ldip_{1..k}$ \ProseOrTypeError;
  \item $\newtenv$ is the static environment yielded by annotating $\ldi_1$;
  \item $\newldi$ is a tuple of local declaration items with $\ldip_{1..k}$, that is, \\
        $\LDITuple(\ldip_{1..k})$.
\end{itemize}

\subsection{Example}
\VerbatimInput{../tests/ASLTypingReference.t/TypingRule.LDTuple.asl}

\CodeSubsection{\LDTupleBegin}{\LDTupleEnd}{../Typing.ml}

\begin{emptyformal}
\subsection{Formally}
\begin{mathpar}
\inferrule{
  \tstruct(\tenv, \tty) \typearrow \vtp \OrTypeError\\\\
  \checktrans{\astlabel(\vtp) = \TTuple}{TupleTypeExpected} \checktransarrow \True \OrTypeError\\\\
  \vtp \eqname \TTuple([\vt_{1..n}])\\\\
  \checktrans{k = n}{InvalidArity} \checktransarrow \True \OrTypeError\\\\
  \newtenv_k = \tenv\\
  i=k..1:
  \annotatelocaldeclitem{\newtenv_{i}, \vt_{i}, \ldi_{i}, \ldk} \typearrow (\newtenv_{i-1}, \ldip_i) \OrTypeError\\\\
  \newtenv = \newtenv_0
}{
  \annotatelocaldeclitem{\tenv, \tty, \LDITuple(\ldi_{1..k}), \ldk} \typearrow \\
  (\newtenv, \LDITuple(\ldip_{1..k}))
}
\end{mathpar}
\end{emptyformal}

\isempty{\subsection{Comments}}

%%%%%%%%%%%%%%%%%%%%%%%%%%%%%%%%%%%%%%%%%%%%%%%%%%%%%%%%%%%%%%%%%%%%%%%%%%%%%%%%%%%%
\chapter{Typing of Statements}
%%%%%%%%%%%%%%%%%%%%%%%%%%%%%%%%%%%%%%%%%%%%%%%%%%%%%%%%%%%%%%%%%%%%%%%%%%%%%%%%%%%%

The function
\[
  \annotatestmt{\overname{\staticenvs}{\tenv} \aslsep \overname{\stmt}{\vs}} \aslto
  (\overname{\stmt}{\news}\aslsep \overname{\staticenvs}{\newenv})
  \cup \overname{\TTypeError}{\TypeErrorConfig}
\]
annotates a statement $\vs$ in an environment $\tenv$, resulting in the annotated statement
$\news$ and a modified environment $\newenv$. The result is a type error, if one is detected.

One of the following applies:
\begin{itemize}
  \item TypingRule.SPass (see \secref{TypingRule.SPass}),
  \item TypingRule.SAssign (see \secref{TypingRule.SAssign}),
  \item TypingRule.SReturnNone (see \secref{TypingRule.SReturnNone}),
  \item TypingRule.SReturnOne (see \secref{TypingRule.SReturnOne}),
  \item TypingRule.SReturnSome (see \secref{TypingRule.SReturnSome}),
  \item TypingRule.SSeq (see \secref{TypingRule.SSeq}),
  \item TypingRule.SCall (see \secref{TypingRule.SCall}),
  \item TypingRule.SCond (see \secref{TypingRule.SCond}),
  \item TypingRule.SCase (see \secref{TypingRule.SCase}),
  \item TypingRule.SAssert (see \secref{TypingRule.SAssert}),
  \item TypingRule.SWhile (see \secref{TypingRule.SWhile}),
  \item TypingRule.SRepeat (see \secref{TypingRule.SRepeat}),
  \item TypingRule.SFor (see \secref{TypingRule.SFor}),
  \item TypingRule.SThrowNone (see \secref{TypingRule.SThrowNone}),
  \item TypingRule.SThrowSome (see \secref{TypingRule.SThrowSome}),
  \item TypingRule.STry (see \secref{TypingRule.STry}).
  \item TypingRule.SDeclSome (see \secref{TypingRule.SDeclSome}),
  \item TypingRule.SDeclNone (see \secref{TypingRule.SDeclNone}).
\end{itemize}

We also define the following helper functions:
\begin{itemize}
  \item TypingRule.CaseAlt (see \secref{TypingRule.CaseAlt}),
  \item TypingRule.SForConstraints (see \secref{TypingRule.SForConstraints}),
  \item TypingRule.MinConstraints (see \secref{TypingRule.MinConstraints}),
  \item TypingRule.MaxConstraints (see \secref{TypingRule.MaxConstraints}),
  \item TypingRule.MinConstraint (see \secref{TypingRule.MinConstraint}),
  \item TypingRule.MaxConstraint (see \secref{TypingRule.MaxConstraint}).
\end{itemize}

\section{TypingRule.SPass \label{sec:TypingRule.SPass}}

\subsection{Prose}
All of the following apply:
\begin{itemize}
  \item $\vs$ is a pass statement, that is, $\SPass$;
  \item $\news$ is $\vs$;
  \item $\newenv$ is $\tenv$.
\end{itemize}

\subsection{Example}

\CodeSubsection{\SPassBegin}{\SPassEnd}{../Typing.ml}

\begin{emptyformal}
\subsection{Formally}
\begin{mathpar}
\inferrule{}{\annotatestmt{\tenv, \SPass} \typearrow (\SPass,\tenv)}
\end{mathpar}
\end{emptyformal}

\isempty{\subsection{Comments}}

\section{TypingRule.SAssign \label{sec:TypingRule.SAssign}}

\subsection{Prose}
All of the following apply:
\begin{itemize}
  \item $\vs$ is an assignment \texttt{le = re}, that is, $\SAssign(\vle, \vre)$;
  \item One of the following applies:
  \begin{itemize}
    \item All of the following apply (\textsc{setter}):
    \begin{itemize}
      \item reducing $(\tenv, \vle, \vre)$ to a setter call via \\ $\inlinesetter{}$ yields the statement $\news$
      (indicating that the assignment corresponds to setter) \ProseOrTypeError;
      \item $\newenv$ is $\tenv$.
    \end{itemize}

    \item All of the following apply (\textsc{non\_setter}):
    \begin{itemize}
      \item reducing $(\tenv, \vle, \vre)$ to a setter call via \\ $\inlinesetter{}$ yields $\None$
            (indicating the assignment does not correspond to a setter);
      \item annotating the right-hand-side expression $\vre$ in $\tenv$ yields $(\vtre, \vreone)$ \ProseOrTypeError;
      \item annotating the left-hand-side expression $\vle$ with the type $\vtre$ in $\tenv$ yields $\vleone$ \ProseOrTypeError;
      \item $\news$ is the assignment \texttt{le1 = re1}, that is, $\SAssign(\vleone, \vreone)$;
      \item $\newenv$ is $\tenv$.
    \end{itemize}

  \end{itemize}
\end{itemize}

\subsection{Example}

\CodeSubsection{\SAssignBegin}{\SAssignEnd}{../Typing.ml}

\begin{emptyformal}
\subsection{Formally}
\begin{mathpar}
\inferrule[setter]{
  \inlinesetter{\tenv, \vle, \vre} \typearrow \langle \news \rangle \OrTypeError\\
}{
  \annotatestmt{\tenv, \overname{\SAssign(\vle, \vre)}{\vs}} \typearrow (\news,\overname{\tenv}{\newtenv})
}
\and
\inferrule[non\_setter]{
  \inlinesetter{\tenv, \vle, \vre} \typearrow \langle \rangle \OrTypeError\\
  \annotateexpr{\tenv, \vre} \typearrow (\vtre, \vreone) \OrTypeError\\\\
  \annotatelexpr{\tenv, \vle, \vtre} \typearrow \vleone \OrTypeError
}{
  \annotatestmt{\tenv, \overname{\SAssign(\vle, \vre)}{\vs}} \typearrow (\overname{\SAssign(\vleone, \vreone)}{\news},\overname{\tenv}{\newtenv})
}
\end{mathpar}
\end{emptyformal}

\isempty{\subsection{Comments}}

\section{TypingRule.SReturnNone \label{sec:TypingRule.SReturnNone}}

\subsection{Prose}
All of the following apply:
\begin{itemize}
  \item $\vs$ is a \texttt{return} statement with no expression, that is, $\SReturn(\None)$;
  \item the enclosing subprogram does not have a \texttt{return} type (it is either a setter
        or a procedure);
  \item $\news$ is a \texttt{return} statement with no expression, that is, $\SReturn(\None)$;
  \item $\newenv$ is $\tenv$.
\end{itemize}

\subsection{Example}

\CodeSubsection{\SReturnNoneBegin}{\SReturnNoneEnd}{../Typing.ml}

\begin{emptyformal}
\subsection{Formally}
\begin{mathpar}
\inferrule{
  L^\tenv.\returntype = \None
}{
  \annotatestmt{\tenv, \overname{\SReturn(\None)}{\vs}} \typearrow (\overname{\SReturn(\None)}{\news}, \overname{\tenv}{\newtenv})
}
\end{mathpar}
\end{emptyformal}

\subsection{Comments}
  This is related to \identr{FTPK}.

\section{TypingRule.SReturnOne \label{sec:TypingRule.SReturnOne}}

\subsection{Prose}
All of the following apply:
\begin{itemize}
\item One of the following applies:
  \begin{itemize}
    \item All of the following apply (\textsc{return\_expr\_no\_return\_type}):
    \begin{itemize}
      \item $\vs$ is a \texttt{return} statement with some expression;
      \item the enclosing subprogram does not have a return type;
    \end{itemize}
    \item All of the following apply (\textsc{return\_type\_no\_return\_expr}):
    \begin{itemize}
      \item $\vs$ is a \texttt{return} statement with no expression;
      \item the enclosing subprogram has a returned type;
    \end{itemize}
  \end{itemize}
  \item the result is an error indicating the mismatch between the declared (existence of the) return type
        and the (existence of the) return expression.
\end{itemize}

\subsection{Example}

\CodeSubsection{\SReturnOneBegin}{\SReturnOneEnd}{../Typing.ml}

\begin{emptyformal}
\subsection{Formally}
\begin{mathpar}
\inferrule[return\_expr\_no\_return\_type]{
  L^\tenv.\returntype = \None
}{
  \annotatestmt{\tenv, \overname{\SReturn(\langle\Ignore\rangle)}{\vs}} \typearrow \TypeErrorVal{InvalidReturnStmt}
}
\and
\inferrule[return\_type\_no\_return\_expr]{
  L^\tenv.\returntype = \langle\Ignore\rangle
}{
  \annotatestmt{\tenv, \overname{\SReturn(\None)}{\vs}} \typearrow \TypeErrorVal{InvalidReturnStmt}
}
\end{mathpar}
\end{emptyformal}

\subsection{Comments}
  This is related to \identr{FTPK}.

\section{TypingRule.SReturnSome \label{sec:TypingRule.SReturnSome}}

\subsection{Prose}
All of the following apply:
\begin{itemize}
  \item $\vs$ is a \texttt{return} statement with an expression $\ve$, that is, $\SReturn(\langle \vep \rangle)$;
  \item the enclosing subprogram has a return type $\vt$;
  \item annotating the righ-hand-side expression $\ve$ in $\tenv$ yields $(\tep,\vep)$ \ProseOrTypeError;
  \item checking whether $\vtep$ \typesatisfies\ $\vt$ in $\tenv$ yields $\True$ \ProseOrTypeError;
  \item $\news$ is a \texttt{return} statement with value $\vep$, that is, $\SReturn(\langle \vep \rangle)$;
  \item $\newenv$ is $\tenv$.
\end{itemize}

\subsection{Example}

\CodeSubsection{\SReturnSomeBegin}{\SReturnSomeEnd}{../Typing.ml}

\begin{emptyformal}
\subsection{Formally}
\begin{mathpar}
\inferrule{
  L^\tenv.\returntype = \langle \vt \rangle\\
  \annotateexpr{\tenv, \ve} \typearrow (\vtep, \vep) \OrTypeError\\\\
  \checktypesat(\tenv, \vtep, \vt) \typearrow \True \OrTypeError
}{
  \annotatestmt{\tenv, \overname{\SReturn(\langle \ve \rangle)}{\vs}} \typearrow
  (\overname{\SReturn(\langle \vep \rangle)}{\news}, \overname{\tenv}{\newtenv})
}
\end{mathpar}
\end{emptyformal}

\subsection{Comments}
This is related to \identr{FTPK}.

\section{TypingRule.SSeq \label{sec:TypingRule.SSeq}}

\subsection{Prose}
All of the following apply:
\begin{itemize}
  \item $\vs$ is the AST node for the sequence of statements $\vsone$ and $\vstwo$, that is, $\SSeq(\vsone, \vstwo)$;
  \item annotating $\vsone$ in $\tenv$ yields $(\newsone, \tenvone)$ \ProseOrTypeError;
  \item annotating $\vstwo$ in $\tenvone$ yields $(\newstwo, \newtenv)$ \ProseOrTypeError;
  \item $\news$ is the AST node for the sequence of statements $\newsone$ and $\newstwo$, that is, $\SSeq(\newsone, \newstwo)$.
\end{itemize}

\subsection{Example}


\CodeSubsection{\SSeqBegin}{\SSeqEnd}{../Typing.ml}

\begin{emptyformal}
\subsection{Formally}
\begin{mathpar}
\inferrule{
  \annotatestmt{\tenv, \vs1} \typearrow (\newsone, \tenvone) \OrTypeError\\\\
  \annotatestmt{\tenvone, \vs2} \typearrow (\newstwo, \newtenv) \OrTypeError
}{
  \annotatestmt{\tenv, \overname{\SSeq(\vsone, \vstwo)}{\vs}} \typearrow (\overname{\SSeq(\newsone, \newstwo)}{\news}, \newtenv)
}
\end{mathpar}
\end{emptyformal}

\isempty{\subsection{Comments}}

\section{TypingRule.SCall \label{sec:TypingRule.SCall}}

\subsection{Prose}
All of the following apply:
\begin{itemize}
  \item $\vs$ is a call to a subprogram named $\name$ with arguments $\vargs$;
  \item annotating the call to $\name$ with arguments $\vargs$, as a procedure (that is, with $\STProcedure$),
        as per \chapref{TypingSubprogramCalls} (which makes sure that the call does not have a return type),
        yields $(\newname, \newargs, \neweqs)$ \ProseOrTypeError;
  \item $\news$ is the call to a subprogram named $\newname$ with arguments
        $\newargs$ and parameter assignments $\neweqs$;
  \item $\newtenv$ is $\tenv$.
\end{itemize}

\subsection{Example}

\CodeSubsection{\SCallBegin}{\SCallEnd}{../Typing.ml}

\begin{emptyformal}
\subsection{Formally}
\begin{mathpar}
\inferrule{
  \annotatecall{\tenv, \name, \vargs, \STProcedure} \typearrow (\newname, \newargs, \neweqs)
}{
  \annotatestmt{\tenv, \overname{\SCall(\name, \vargs)}{\vs}} \typearrow (\overname{\SCall(\newname, \newargs, \neweqs)}{\news}, \tenv)
}
\end{mathpar}
\end{emptyformal}

\subsection{Comments}
  This is related to \identd{VXKM}.

\section{TypingRule.SCond \label{sec:TypingRule.SCond}}

\subsection{Prose}
All of the following apply:
\begin{itemize}
  \item $\vs$ is a condition $\ve$ with the statements $\vsone$ and $\vstwo$, that is, $\SCond(\ve, \vsone, \vstwo)$;
  \item annotating the right-hand-side expression $\ve$ in $\tenv$ yields $(\tcond, \econd)$ \ProseOrTypeError;
  \item checking that $\tcond$ \typesatisfies\ $\TBool$ yields $\True$ \ProseOrTypeError;
  \item annotating the statement $\vsone$ in $\tenv$ yields $\vsonep$ \ProseOrTypeError;
  \item annotating the statement $\vstwo$ in $\tenv$ yields $\vstwop$ \ProseOrTypeError;
  \item $\news$ is the condition $\econd$ with the statements $\vsonep$ and $\vstwop$, that is, $\SCond(\econd, \vsonep, \vstwop)$;
  \item $\newenv$ is $\tenv$.
\end{itemize}

\subsection{Example}

\CodeSubsection{\SCondBegin}{\SCondEnd}{../Typing.ml}

\begin{emptyformal}
\subsection{Formally}
\begin{mathpar}
\inferrule{
  \annotateexpr{\tenv, \ve} \typearrow (\tcond, \econd) \OrTypeError\\\\
  \checktypesat(\tenv, \tcond, \TBool) \typearrow \True \OrTypeError\\\\
  \annotateblock{\tenv, \vsone} \typearrow \vsonep \OrTypeError\\\\
  \annotateblock{\tenv, \vstwo} \typearrow \vstwop \OrTypeError
}{
  \annotatestmt{\tenv, \overname{\SCond(\ve, \vsone, \vstwo)}{\vs}} \typearrow
  (\overname{\SCond(\econd, \vsonep, \vstwop)}{\news}, \overname{\tenv}{\newtenv})
}
\end{mathpar}
\end{emptyformal}

\subsection{Comments}
  This is related to \identr{NBDJ}.

\section{TypingRule.SCase \label{sec:TypingRule.SCase}}

\subsection{Prose}
All of the following apply:
\begin{itemize}
  \item $\vs$ is a case statement with expression $\ve$ and case clauses $\vcases$, that is, \\
        $\SCase(\veone, \vcasesone)$;
  \item annotating the right-hand-side expression $\ve$ in $\tenv$ yields $(\vte, \veone)$ \ProseOrTypeError;
  \item annotating each case clause as per \secref{TypingRule.CaseAlt} in $\vcases$ yields the annotated list of clauses $\vcasesone$ \ProseOrTypeError;
  \item $\news$ is a case statement with expression $\veone$ and case clauses $\vcasesone$;
  \item $\newenv$ is $\tenvone$.
\end{itemize}

\subsection{Example}

\CodeSubsection{\SCaseBegin}{\SCaseEnd}{../Typing.ml}

\begin{emptyformal}
\subsection{Formally}
\begin{mathpar}
\inferrule{
  \annotateexpr{\tenv, \ve} \typearrow (\vte, \veone) \OrTypeError\\\\
  \vi\in\listrange(\vcases): \annotatecase{\tenv, \vcases[\vi]} \typearrow \vcase_\vi \OrTypeError\\\\
  \vcasesone \eqdef [\vi\in\listrange(\vcases): \vcase_\vi]
}{
  \annotatestmt{\tenv, \overname{\SCase(\ve, \vcases)}{\vs}} \typearrow
  (\overname{\SCase(\veone, \vcasesone)}{\news}, \overname{\tenv}{\newtenv})
}
\end{mathpar}
\end{emptyformal}

\subsection{Comments}
  This is related to \identr{WGSY}.

\section{TypingRule.SAssert \label{sec:TypingRule.SAssert}}
\subsection{Prose}
All of the following apply:
\begin{itemize}
  \item $\vs$ is an assert statement with expression $\ve$, that is, $\SAssert(\ve)$;
  \item annotating the right-hand-side expression $\ve$ in $\tenv$ yields $(\tep,\vep)$ \ProseOrTypeError;
  \item checking that $\vtep$ \typesatisfies\ $\TBool$ in $\tenv$ yields $\True$ \ProseOrTypeError;
  \item $\news$ is an assert statement with expression $\vep$, that is, $\SAssert(\vep)$;
  \item $\newenv$ is $\tenv$.
\end{itemize}

\subsection{Example}

\CodeSubsection{\SAssertBegin}{\SAssertEnd}{../Typing.ml}

\begin{emptyformal}
\subsection{Formally}
\begin{mathpar}
\inferrule{
  \annotateexpr{\tenv, \ve} \typearrow (\vtep, \vep) \OrTypeError\\
  \checktypesat(\tenv, \vtep, \TBool) \typearrow \True \OrTypeError
}{
  \annotatestmt{\tenv, \overname{\SAssert(\ve)}{\vs}} \typearrow (\overname{\SAssert(\vep)}{\news}, \overname{\tenv}{\newtenv})
}
\end{mathpar}
\end{emptyformal}

\subsection{Comments}
  This is related to \identr{JQYF}.

\section{TypingRule.SWhile \label{sec:TypingRule.SWhile}}
\subsection{Prose}
All of the following apply:
\begin{itemize}
\item $\vs$ is a \texttt{while} statement with expression $\veone$ and statement block $\vsone$, that is, \\
      $\SWhile(\veone, \vsone)$;
\item annotating the right-hand-side expression $\veone$ in $\tenv$ yields $(\vt, \vetwo)$ \ProseOrTypeError;
\item checking that $\vt$ \typesatisfies\ $\TBool$ in $\tenv$ yields $\True$ \ProseOrTypeError;
\item $\news$ is a \texttt{while} statement with expression $\vetwo$ and statement block $\vstwo$, that is, $\SWhile(\vetwo, \vstwo)$;
\item $\newenv$ is $\tenv$.
\end{itemize}

\subsection{Example}

\CodeSubsection{\SWhileBegin}{\SWhileEnd}{../Typing.ml}

\begin{emptyformal}
\subsection{Formally}
\begin{mathpar}
\inferrule{
  \annotateexpr{\tenv, \veone} \typearrow (\vt, \vetwo) \OrTypeError\\\\
  \checktypesat(\tenv, \vt, \TBool) \typearrow \True \OrTypeError\\
  \annotateblock{\tenv, \vsone} \typearrow \vstwo \OrTypeError
}{
  \annotatestmt{\tenv, \overname{\SWhile(\veone, \vsone)}{\vs}} \typearrow (\overname{\SWhile(\vetwo, \vstwo)}{\news}, \overname{\tenv}{\newtenv})
}
\end{mathpar}
\end{emptyformal}

\subsection{Comments}
  This is related to \identr{FTVN}.

\section{TypingRule.SRepeat \label{sec:TypingRule.SRepeat}}

\subsection{Prose}
All of the following apply:
\begin{itemize}
  \item $\vs$ is a \texttt{repeat} statement with expression $\veone$ and statement block $\vsone$, that is, \\
        $\SRepeat(\vetwo, \vstwo)$;
  \item annotating $\vsone$ as a block statement in $\tenv$ yields $\vstwo$ \ProseOrTypeError;
  \item annotating the right-hand-side expression $\veone$ in $\tenv$ yields $(\vt, \vetwo)$ \ProseOrTypeError;
  \item checking that $\vt$ \typesatisfies\ $\TBool$ in $\tenv$ yields $\True$ \ProseOrTypeError;
  \item $\news$ is a \texttt{repeat} statement with expression $\vetwo$ and statement block $\vstwo$, that is, $\SRepeat(\vetwo, \vstwo)$;
  \item $\newenv$ is $\tenv$.
\end{itemize}

\subsection{Example}

\CodeSubsection{\SRepeatBegin}{\SRepeatEnd}{../Typing.ml}

\begin{emptyformal}
\subsection{Formally}
\begin{mathpar}
\inferrule{
  \annotateblock{\tenv, \vsone} \typearrow \vstwo \OrTypeError\\\\
  \annotateexpr{\tenv, \veone} \typearrow (\vt, \vetwo) \OrTypeError\\\\
  \checktypesat(\tenv, \vt, \TBool) \typearrow \True \OrTypeError
}{
  \annotatestmt{\tenv, \overname{\SRepeat(\veone, \vsone)}{\vs}} \typearrow
  (\overname{\SRepeat(\vetwo, \vstwo)}{\news}, \overname{\tenv}{\newtenv})
}
\end{mathpar}
\end{emptyformal}

\subsection{Comments}
  This is related to \identr{FTVN}.

\section{TypingRule.SFor \label{sec:TypingRule.SFor}}
\subsection{Prose}
All of the following apply:
\begin{itemize}
  \item $\vs$ is a \texttt{for} statement with index $\id$, direction $\dir$, start expression
        $\veone$, end expression $\vetwo$, and a body statement (block) $\vsp$, that is, $\SFor(\id, \dir, \veone, \vetwo, \vsp)$;
  \item annotating the right-hand-side expression $\veone$ in $\tenv$ yields $(\vtone, \veonep)$ \ProseOrTypeError;
  \item annotating the right-hand-side expression $\vetwo$ in $\tenv$ yields $(\vttwo, \vetwop)$ \ProseOrTypeError;
  \item obtaining the \wellconstrainedversion\ of $\vtone$ in $\tenv$ yields $\structone$;
  \item obtaining the \wellconstrainedversion\ of $\vttwo$ in $\tenv$ yields $\structtwo$;
  \item obtaining the constraints on the loop index $\id$ from $\structone$, $\structtwo$, and $\dir$ in $\tenv$
        via $\getforconstraints$ yields $\vcs$ \ProseOrTypeError;
  \item $\tty$ is the integer type with constraints $\vcs$;
  \item checking that $\id$ is not already declared in $\tenv$ yields $\True$ \ProseOrTypeError;
  \item adding $\id$ as a local immutable variable with type $\tty$ to $\tenv$ yields $\tenvp$;
  \item annotating $\vsp$ as a block statement in $\tenvp$ yields $\vspp$ \ProseOrTypeError;
  \item $\news$ is the \texttt{for} statement with index $\id$, direction $\dir$, start expression $\veonep$,
        end expression $\vetwop$, and the body statement (block) $\vspp$, that is, \\
        $\SFor(\id, \veonep, \dir, \vetwop, \vspp)$;
  \item $\newtenv$ is $\tenv$ (notice that this means $\id$ is only declared for annotating $\vsp$ but then goes
        out of scope).
\end{itemize}

% \begin{itemize}
% \item $\vs$ is a \texttt{for} statement with index \texttt{id}, direction \texttt{dir}, two expressions
%   $\veone$ and \vetwo and a statement block \texttt{s'};
% \item \texttt{t1,e1'} is the result of annotating $\veone$ in $\tenv$;
% \item \texttt{t2,e2'} is the result of annotating \vetwo in $\tenv$;
% \item an error is returned: ``\texttt{ASL Typing Error : A subtype of integer was expected, t1 was provided}'' or $\vtone$ has the structure of an integer type and all of the following apply:
% \item an error is returned: ``\texttt{ASL Typing Error : A subtype of integer was expected, t2 was provided}'' or $\vttwo$ has the structure of an integer type and all of the following apply:
% \item One of the following applies:
%   \begin{itemize}
%     \item All of the following apply:
%       \begin{itemize}
%         \item $\vtone$ has the structure of an unconstrained integer type;
%         \item $\tty$ is the unconstrained integer type;
%       \end{itemize}
%     \item All of the following apply:
%       \begin{itemize}
%         \item $\vttwo$ has the structure of an unconstrained integer type;
%         \item $\tty$ is the unconstrained integer type;
%       \end{itemize}
%     \item All of the following apply:
%       \begin{itemize}
%         \item $\vtone$ has the structure of a constrained integer type with constraint \texttt{cs1};
%         \item $\vttwo$ has the structure of a constrained integer type with constraint \texttt{cs2};
%         \item One of the following applies:
%           \begin{itemize}
%             \item All of the following apply:
%               \begin{itemize}
%                 \item \texttt{dir} is \texttt{to};
%                 \item \texttt{bot\_cs} is \texttt{cs1};
%                 \item \texttt{top\_cs} is \texttt{cs2};
%               \end{itemize}
%             \item All of the following apply:
%               \begin{itemize}
%                 \item \texttt{dir} is \texttt{down to};
%                 \item \texttt{bot\_cs} is \texttt{cs2};
%                 \item \texttt{top\_cs} is \texttt{cs1};
%               \end{itemize}
%           \end{itemize}
%         \item One of the following applies:
%           \begin{itemize}
%             \item All of the following apply:
%               \begin{itemize}
%                 \item \texttt{bot\_cs} contains a an expression that is not evaluable at compile-time;
%                 \item \texttt{cs} is the empty constraint;
%               \end{itemize}
%             \item All of the following apply:
%               \begin{itemize}
%                 \item \texttt{top\_cs} contains a an expression that is not evaluable at compile-time;
%                 \item \texttt{cs} is the empty constraint;
%               \end{itemize}
%             \item All of the following apply:
%               \begin{itemize}
%                 \item \texttt{bot} is the minimum of the constraints \texttt{bot\_cs};
%                 \item \texttt{top} is the maximum of the constraints \texttt{top\_cs};
%                 \item \texttt{bot} is less or equal than \texttt{top};
%                 \item \texttt{cs} is the constraint \texttt{bot .. top};
%               \end{itemize}
%             \item All of the following apply:
%               \begin{itemize}
%                 \item \texttt{bot} is the minimum of the constraints \texttt{bot\_cs};
%                 \item \texttt{top} is the maximum of the constraints \texttt{top\_cs};
%                 \item \texttt{top} is strictly less than \texttt{bot}
%                 \item \texttt{cs} is \texttt{cs1};
%               \end{itemize}
%           \end{itemize}
%         \item $\tty$ is the constrained integer type with constraint \texttt{cs};
%       \end{itemize}
%   \end{itemize}
% \item an error is returned ``\texttt{ASL Typing Error: cannot declare already \\ declared element "id".}'' or \texttt{id} is not bound in $\tenv$ and all of the following apply:
% \item $\tenvp$ is $\tenv$ modified so that \texttt{id} is locally declared of type $\tty$;
% \item \texttt{s''} is the result of annotating \texttt{s'} in $\tenvp$;
% \item $\news$ is a for statement with index \texttt{id}, direction \texttt{dir}, two expressions $\veonep$ and $\vetwop$ and statement \texttt{s''};
% \item $\newenv$ is $\tenv$.
% \end{itemize}

\subsection{Example}

\CodeSubsection{\SForBegin}{\SForEnd}{../Typing.ml}

\begin{emptyformal}
\subsection{Formally}
\begin{mathpar}
\inferrule{
  \annotateexpr{\tenv, \veone} \typearrow (\vtone, \veonep) \OrTypeError\\\\
  \annotateexpr{\tenv, \vetwo} \typearrow (\vttwo, \vetwop) \OrTypeError\\\\
  \getwellconstrainedstructure(\tenv, \vtone) \typearrow \structone\\
  \getwellconstrainedstructure(\tenv, \vttwo) \typearrow \structtwo\\
  \getforconstraints(\tenv, \structone, \structtwo, \dir) \typearrow \vcs \OrTypeError\\\\
  \tty \eqdef \TInt(\vcs)\\
  \checkvarnotinenv{\tenv, \id} \typearrow \True \OrTypeError\\\\
  \addlocal(\tenv, \tty, \id, \LDKLet) \typearrow \tenvp\\
  \annotateblock{\tenvp, \vsp} \typearrow \vspp \OrTypeError\\\\
  \news \eqdef \SFor(\id, \veonep, \dir, \vetwop, \vspp)
}{
  \annotatestmt{\tenv, \overname{\SFor(\id, \dir, \veone, \vetwo, \vsp)}{\vs}} \typearrow (\news, \overname{\tenv}{\newtenv})
}
\end{mathpar}
\end{emptyformal}

\subsection{Comments}
  This is related to \identr{SSBD}, \identr{ZSND}, \identr{VTJW}.

\section{TypingRule.SThrowNone \label{sec:TypingRule.SThrowNone}}

\subsection{Prose}
All of the following apply:
\begin{itemize}
  \item $\vs$ is a throw statement with no expression, that is, $\SThrow(\None)$;
  \item $\news$ is $\vs$;
  \item $\newenv$ is $\tenv$.
\end{itemize}

\subsection{Example}

\CodeSubsection{\SThrowNoneBegin}{\SThrowNoneEnd}{../Typing.ml}

\begin{emptyformal}
\subsection{Formally}
\begin{mathpar}
\inferrule{}{
  \annotatestmt{\tenv, \overname{\SThrow(\None)}{\vs}} \typearrow (\overname{\SThrow(\None)}{\news}, \overname{\tenv}{\newtenv})
}
\end{mathpar}
\end{emptyformal}

\subsection{Comments}
  Note that \identr{BRCJ} is done in~\cite[SemanticsRule.TopLevel]{ASLSemanticsReference}.

\section{TypingRule.SThrowSome \label{sec:TypingRule.SThrowSome}}

\subsection{Prose}
All of the following apply:
\begin{itemize}
  \item $\vs$ is a throw statement with expression $\ve$, that is, $\SThrow(\langle (\ve, \Ignore) \rangle)$;
  \item annotating the right-hand-side expression $\ve$ in $\tenv$ yields $(\vte, \vep)$ \ProseOrTypeError;
  \item checking that $\vte$ has the structure of an exception type yields $\True$ \ProseOrTypeError;
  \item $\news$ is a throw statement with expression $\vep$ and type $\vte$, that is, \\
        $\SThrow(\langle (\vep, \langle\vte\rangle) \rangle)$;
  \item $\newenv$ is $\tenv$.
\end{itemize}

\subsection{Example}

\CodeSubsection{\SThrowSomeBegin}{\SThrowSomeEnd}{../Typing.ml}

\begin{emptyformal}
\subsection{Formally}
\begin{mathpar}
\inferrule{
  \annotateexpr{\tenv, \ve} \typearrow (\vte, \vep) \OrTypeError\\\\
  \checkstructurelabel(\tenv, \vte, \TException) \typearrow \True \OrTypeError
}{
  \annotatestmt{\tenv, \overname{\SThrow(\langle (\ve, \Ignore) \rangle)}{\vs}} \typearrow
  (\overname{\SThrow(\langle (\vep, \langle\vte\rangle) \rangle)}{\news}, \overname{\tenv}{\newtenv})
}
\end{mathpar}
\end{emptyformal}

\subsection{Comments}
  This is related to \identr{NXRC}.

\section{TypingRule.STry \label{sec:TypingRule.STry}}

\subsection{Prose}
All of the following apply:
\begin{itemize}
  \item $\vs$ is a try statement with statement $\vsp$, list of catchers $\catchers$ and an optional \texttt{otherwise} block;
  \item annotating the statement $\vsp$ as a block statement yields $\vspp$ \ProseOrTypeError;
  \item annotating each catcher $\vc$ in $\catchers$ in $\tenv$ yields $\vcp$ \ProseOrTypeError;
  \item $\catchersp$ is the list of annotated catchers $\vcp$ for each $\vc\in\catchers$;
  \item One of the following applies:
  \begin{itemize}
    \item All of the following apply (\textsc{no\_otherwise}):
    \begin{itemize}
      \item there is no \texttt{otherwise} statement;
      \item $\news$ is a try statement with statement $\vspp$, list catchers $\catchersp$ and no \texttt{otherwise} statement,
            that is \\
            $\STry(\vspp, \catchersp, \None)$;
    \end{itemize}

    \item All of the following apply (\textsc{otherwise}):
    \begin{itemize}
      \item there is an \texttt{otherwise} statement $\otherwise$;
      \item annotating the statement $\otherwise$ as a block statement in $\tenv$ yields $\otherwisep$ \ProseOrTypeError;
      \item $\news$ is a try statement with statement $\vspp$, list catchers $\catchersp$ and \texttt{otherwise} statement
            $\otherwisep$, that is \\
            $\STry(\vspp, \catchersp, \langle\otherwisep\rangle)$;
    \end{itemize}
  \end{itemize}
  \item $\newenv$ is $\tenv$.
\end{itemize}

\subsection{Example}

\CodeSubsection{\STryBegin}{\STryEnd}{../Typing.ml}

\begin{emptyformal}
\subsection{Formally}
\begin{mathpar}
\inferrule[no\_otherwise]{
  \annotateblock{\tenv, \vsp} \typearrow \vspp \OrTypeError\\\\
  \vc \in \catchers: \annotatecatcher{\tenv, \vc} \typearrow \vcp \OrTypeError\\\\
  \catchersp \eqdef [\vc \in \catchers : \vcp]\\
  \news \eqdef \STry(\vspp, \catchersp, \None)
}{
  \annotatestmt{\tenv, \overname{\STry(\vsp, \catchers, \None)}{\vs}} \typearrow (\news, \overname{\tenv}{\newtenv})
}
\and
\inferrule[otherwise]{
  \annotateblock{\tenv, \vsp} \typearrow \vspp \OrTypeError\\\\
  \annotateblock{\tenv, \otherwise} \typearrow \otherwisep \OrTypeError\\\\
  \vc \in \catchers: \annotatecatcher{\tenv, \vc} \typearrow \vcp \OrTypeError\\\\
  \catchersp \eqdef [\vc \in \catchers : \vcp]\\
  \news \eqdef \STry(\vspp, \catchersp, \otherwise')
}{
  \annotatestmt{\tenv, \overname{\STry(\vsp, \catchers, \langle\otherwise\rangle)}{\vs}} \typearrow (\news, \overname{\tenv}{\newtenv})
}
\end{mathpar}
\end{emptyformal}

\subsection{Comments}
  This is related to \identr{WVXS}.

\section{TypingRule.SDeclSome \label{sec:TypingRule.SDeclSome}}

\subsection{Prose}
All of the following apply:
\begin{itemize}
  \item $\vs$ is a declaration with local declaration keyword $\ldk$, local identifiers $\ldi$, and an expression $\ve$,
        that is, $\SDecl(\ldk, \ldi, \langle\ve\rangle)$;
  \item annotating the right-hand-side expression $\ve$ in $\tenv$ yields $(\vte,\vep)$ \ProseOrTypeError;
  \item One of the following applies:
  \begin{itemize}
    \item All of the following apply (\textsc{constant}):
    \begin{itemize}
      \item $\ldk$ indicates a local constant declaration, that is, $\LDKConstant$;
      \item symbolically simplifying $\ve$ in $\tenv$ yields the literal $\vv$ \ProseOrTypeError;
      \item declaring a local constant of type $\vte$, literal $\vv$ and identifier $\ldi$ in $\tenv$ yields $(\newtenv, \ldip)$;
      \item $\news$ is a declaration with $\ldk$, $\ldip$ and an expression $\vep$.
    \end{itemize}

    \item All of the following apply (\textsc{non\_constant}):
    \begin{itemize}
      \item $\ldk$ indicates that this is not a local constant declaration, that is, $\ldk\neq\LDKConstant$;
      \item declaring the local identifiers $\ldi$ of type $\vte$ with local declaration keyword $\ldk$ in $\tenv$
            yields $(\newtenv, \ldip)$;
      \item $\news$ is a declaration with $\ldk$, $\ldip$ and an expression $\vep$.
    \end{itemize}
  \end{itemize}
\end{itemize}

\subsection{Example}

\CodeSubsection{\SDeclSomeBegin}{\SDeclSomeEnd}{../Typing.ml}

\begin{emptyformal}
\subsection{Formally}
\begin{mathpar}
\inferrule[constant]{
  \annotateexpr{\tenv, \ve} \typearrow (\vte, \vep) \OrTypeError\\\\
  \ldk =\LDKConstant\\
  \reduceconstants{\tenv, \ve} \typearrow \vv \OrTypeError\\
  \declarelocalconstant{\tenv, \vte, \vv, \ldi} \typearrow (\newtenv, \ldip)\\
  \news \eqdef \SDecl(\LDKConstant, \ldip, \langle\vep\rangle)
}{
  \annotatestmt{\tenv, \overname{\SDecl(\ldk, \ldi, \langle\ve\rangle)}{\vs}} \typearrow (\news, \newtenv)
}
\and
\inferrule[non\_constant]{
  \annotateexpr{\tenv, \ve} \typearrow (\vte, \vep) \OrTypeError\\\\
  \ldk \neq \LDKConstant\\
  \annotatelocaldeclitem{\tenv, \vte, \ldk, \ldi} \typearrow (\newtenv, \ldip)\\
  \news \eqdef \SDecl(\LDKConstant, \ldip, \langle\vep\rangle)
}{
  \annotatestmt{\tenv, \overname{\SDecl(\ldk, \ldi, \langle\ve\rangle)}{\vs}} \typearrow (\news, \newtenv)
}
\end{mathpar}
\end{emptyformal}

\isempty{\subsection{Comments}}
  This is related to \identr{YSPM}.

\section{TypingRule.SDeclNone \label{sec:TypingRule.SDeclNone}}

\subsection{Prose}
All of the following apply:
\begin{itemize}
\item $\vs$ is a local declaration statement with a variable keyword and local identifiers $\ldi$, and no initial expression,
      that is, $\SDecl(\LDKVar, \ldi, \None)$ (local declarations of \texttt{let} variables and constants require
      an initializing expression, otherwise they are rejected by an ASL parser);
\item annotating the uninitialised local declarations $\ldi$ in $\tenv$ yields $(\newtenv, \ldip)$;
\item $\news$ is a local declaration statement with variable keyword, local identifiers $\ldip$, and no initial expression,
      that is, $\SDecl(\LDKVar, \ldip, \None)$.
\end{itemize}

\subsection{Example}

\CodeSubsection{\SDeclNoneBegin}{\SDeclNoneEnd}{../Typing.ml}

\begin{emptyformal}
\subsection{Formally}
\begin{mathpar}
\inferrule{
  \annotatelocaldeclitemuninit{\tenv, \ldi} \typearrow (\newtenv, \ldip) \OrTypeError\\\\
  \news \eqdef \SDecl(\LDKVar, \ldip, \None)
}{
  \annotatestmt{\tenv, \overname{\SDecl(\LDKVar, \ldi, \None)}{\vs}} \typearrow (\news, \newtenv)
}
\end{mathpar}
\end{emptyformal}

\isempty{\subsection{Comments}}

\section{TypingRule.CaseAlt \label{sec:TypingRule.CaseAlt}}

\hypertarget{def-annotatecase}{}
The helper function
\[
  \annotatecase{
    \overname{\staticenvs}{\tenv} \aslsep
    \overname{\ty}{\vte} \aslsep
    \overname{\casealt}{\vcase}
  } \aslto
  \overname{\casealt}{\vcaseone} \cup \overname{\TTypeError}{\TypeErrorConfig}
\]
annotates the case clause $\vcase$ for matching an expression of type $\vte$ in $\tenv$,
resulting in the annotated case clause $\vcaseone$, or a type error, if one is detected.

\subsection{Prose}
All of the following apply:
\begin{itemize}
  \item $\vcase$ is a case clause with pattern $\vpzero$, optional \texttt{where} expression $\vwzero$,
        and \texttt{otherwise} statement $\vszero$, that is,
        $\{ \text{pattern} : \vpzero, \text{where} : \vwzero, \text{stmt} : \vszero \}$;
  \item annotating the pattern $\vpzero$ with type $\vte$ in $\tenv$ yields $\vpone$ \ProseOrTypeError;
  \item annotating the statement $\vszero$ as a block statement in $\tenv$ yields $\vsone$ \ProseOrTypeError;
  \item One of the following applies:
  \begin{itemize}
    \item All of the following apply (\textsc{no\_where}):
    \begin{itemize}
      \item $\vwzero$ is $\None$ (that is, no \texttt{where} expression);
      \item $\vcaseone$ is $\{ \text{pattern} : \vpone, \text{where} : \None, \text{stmt} : \vsone \}$.
    \end{itemize}

    \item All of the following apply (\textsc{where}):
    \begin{itemize}
      \item $\vwzero$ is the expression $\vewzero$;
      \item annotating the expression $\vewzero$ in $\tenv$ yields $(\vtwe, \vewone)$ \ProseOrTypeError;
      \item checking whether the structure of $\vtwe$ in $\tenv$ is that of the \texttt{boolean} type yields $\True$ \ProseOrTypeError;
      \item $\vcaseone$ is $\{ \text{pattern} : \vpone, \text{where} : \langle\vewone\rangle, \text{stmt} : \vsone \}$.
    \end{itemize}
  \end{itemize}
\end{itemize}

\begin{emptyformal}
\subsection{Formally}
\begin{mathpar}
\inferrule[no\_where]{
  \vcase \eqname \{ \text{pattern} : \vpzero, \text{where} : \None, \text{stmt} : \vszero \}\\
  \annotatepattern(\tenv, \vte, \vpzero) \typearrow \vpone \OrTypeError\\\\
  \annotateblock{\tenv, \vszero} \typearrow \vsone \OrTypeError\\
}{
  \annotatecase{\tenv, \vcase} \typearrow \{ \text{pattern} : \vpone, \text{where} : \None, \text{stmt} : \vsone \}
}
\and
\inferrule[where]{
  \vcase \eqname \{ \text{pattern} : \vpzero, \text{where} : \langle\vewzero\rangle, \text{stmt} : \vszero \}\\
  \annotatepattern(\tenv, \vte, \vpzero) \typearrow \vpone \OrTypeError\\\\
  \annotateblock{\tenv, \vszero} \typearrow \vsone \OrTypeError\\\\
  \annotateexpr{\tenv, \vewzero} \typearrow (\vtwe, \vewone) \OrTypeError\\\\
  \checkstructurelabel(\tenv, \vtwe, \TBool) \typearrow \True \OrTypeError
}{
  \annotatecase{\tenv, \vcase} \typearrow \{ \text{pattern} : \vpone, \text{where} : \langle\vewone\rangle, \text{stmt} : \vsone \}
}
\end{mathpar}
\end{emptyformal}

\section{TypingRule.SForConstraints \label{sec:TypingRule.SForConstraints}}
\hypertarget{def-getforconstraints}{}
\newcommand\minmaxtop[0]{\texttt{MinMaxTop}}
The function
\[
  \getforconstraints(
    \overname{\staticenvs}{\tenv} \aslsep
    \overname{\ty}{\structone} \aslsep
    \overname{\ty}{\structtwo} \aslsep
    \overname{\dir}{\dir}
  ) \aslto
  \begin{array}{c}
    \overname{\left\{
      \begin{array}{c}
        \unconstrained,\\
        \wellconstrained(\intconstraint^+),\\
        \underconstrained(\identifier)
      \end{array}
    \right\}}{\vcs} \\
    \cup\ \overname{\TTypeError}{\TypeErrorConfig}
    \end{array}
\]
infers the constraints --- $\vcs$ --- for a \texttt{for} loop index variable from the
\wellconstrainedversion\ of the type of the start expression --- $\structone$ ---
and the \wellconstrainedversion\ of the type of the end expression --- $\structtwo$.
A type error is returned, if one is detected.

\subsection{Prose}
One of the following applies:
\begin{itemize}
  \item All of the following apply (\textsc{non\_integer\_error}):
  \begin{itemize}
    \item at least one of $\structone$ and $\structtwo$ is not an integer type;
    \item the result is a type error indicating that an integer-typed expression is expected.
  \end{itemize}

  \item All of the following apply (\textsc{unconstrained}):
  \begin{itemize}
    \item both $\structone$ and $\structtwo$ are integer types;
    \item either one of $\structone$ and $\structtwo$ is an unconstrained integer;
    \item $\vc$ is $\unconstrained$.
  \end{itemize}

  \item All of the following apply (\textsc{well\_constrained\_int\_range}):
  \begin{itemize}
    \item $\structone$ is an integer type with constraint $\vcone$;
    \item $\structtwo$ is an integer type with constraint $\vctwo$;
    \item the pair $(\botcs, \topcs)$ is defined to be $(\csone, \cstwo)$ if $\dir$ is $\UP$ and $(\cstwo, \csone)$, otherwise;
    \item obtaining the minimum of $\botcs$ via $\minconstraints$ in $\tenv$ yields an integer $\vbot$;
    \item obtaining the maximum of $\topcs$ via $\maxconstraints$ in $\tenv$ yields an integer $\vvtop$;
    \item $\vbot$ is less than or equal to $\vvtop$;
    \item $\vc$ is the well-constrained range between between the literal integer expression for $\vbot$ and
          the literal integer expression for $\vvtop$, that is, \\
          $\wellconstrained([\ConstraintRange(\ELInt{\vbot}, \ELInt{\vvtop})])$;
  \end{itemize}

  \item All of the following apply (\textsc{well\_constrained\_int\_reverse\_range}):
  \begin{itemize}
    \item $\structone$ is an integer type with constraint $\vcone$;
    \item $\structtwo$ is an integer type with constraint $\vctwo$;
    \item the pair $(\botcs, \topcs)$ is defined to be $(\csone, \cstwo)$ if $\dir$ is $\UP$ and $(\cstwo, \csone)$, otherwise;
    \item obtaining the minimum of $\botcs$ via $\minconstraints$ in $\tenv$ yields an integer $\vbot$;
    \item obtaining the maximum of $\topcs$ via $\maxconstraints$ in $\tenv$ yields an integer $\vvtop$;
    \item $\vbot$ is greater than $\vvtop$;
    \item $\vc$ is the well-constrained constraint for $\csone$, that is, \\
          $\wellconstrained(\csone)$;
  \end{itemize}

  \item All of the following apply (\textsc{well\_constrained\_int\_symbolic\_range}):
  \begin{itemize}
    \item $\structone$ is a well-constrained integer with constraints $\csone$;
    \item $\structtwo$ is a well-constrained integer with constraints $\cstwo$;
    \item the pair $(\botcs, \topcs)$ is defined to be $(\csone, \cstwo)$ if $\dir$ is $\UP$ and $(\cstwo, \csone)$, otherwise;
    \item obtaining the minimum of $\botcs$ via $\minconstraints$ in $\tenv$ yields $\vbot$;
    \item obtaining the maximum of $\topcs$ via $\maxconstraints$ in $\tenv$ yields $\vvtop$;
    \item at least one of $\vbot$ and $\vvtop$ is not an integer;
    \item $\botcs$ is a single exact constraint with expression $\ebot$;
    \item $\topcs$ is a single exact constraint with expression $\etop$;
    \item $\vc$ is the single (well-constrained) range constraint for from $\ebot$ to $\etop$, that is, \\
          $\wellconstrained([\ConstraintRange(\ebot, \etop)])$
  \end{itemize}

  \item All of the following apply (\textsc{well\_constrained\_else}):
  \begin{itemize}
    \item $\structone$ is a well-constrained integer with constraints $\csone$;
    \item $\structtwo$ is a well-constrained integer with constraints $\cstwo$;
    \item the pair $(\botcs, \topcs)$ is defined to be $(\csone, \cstwo)$ if $\dir$ is $\UP$ and $(\cstwo, \csone)$, otherwise;
    \item obtaining the minimum of $\botcs$ via $\minconstraints$ in $\tenv$ yields $\vbot$;
    \item obtaining the maximum of $\topcs$ via $\maxconstraints$ in $\tenv$ yields $\vvtop$;
    \item at least one of $\vbot$ and $\vvtop$ is not an integer;
    \item at least one of $\botcs$ and $\topcs$ is not a single exact constraint;
    \item $\vc$ is $\unconstrained$
  \end{itemize}
\end{itemize}

\begin{emptyformal}
\subsection{Formally}
\begin{mathpar}
\inferrule[non\_integer\_error]{
    \astlabel(\structone) \neq \TInt \land \astlabel(\structtwo) \neq \TInt
}{
  \getforconstraints(\tenv, \structone, \structtwo, \dir) \typearrow \\
  \TypeErrorVal{IntegerTypeExpected}
}
\end{mathpar}

\begin{mathpar}
\inferrule[unconstrained]{
  \astlabel(\vcone) = \unconstrained \lor \astlabel(\vctwo) = \unconstrained
}{
  \getforconstraints(\tenv, \overname{\TInt(\vcone)}{\structone}, \overname{\TInt(\vctwo)}{\structtwo}, \dir) \typearrow
  \overname{\unconstrained}{\vc}
}
\end{mathpar}

\begin{mathpar}
\inferrule[well\_constrained\_int\_range]{
    \astlabel(\vcone) = \wellconstrained(\csone) \land \astlabel(\vctwo) = \wellconstrained(\cstwo)\\
    (\botcs, \topcs) \eqdef \choice{\dir = \UP}{(\csone, \cstwo)}{(\cstwo, \csone)}\\
    \minconstraints(\tenv, \botcs) \typearrow \vbot \in \Z\\
    \maxconstraints(\tenv, \topcs) \typearrow \vvtop \in \Z\\
    \vbot \leq \vvtop\\
    \vc \eqdef \wellconstrained([\ConstraintRange(\ELInt{\vbot}, \ELInt{\vvtop})])
}{
  \getforconstraints(\tenv, \overname{\TInt(\vcone)}{\structone}, \overname{\TInt(\vctwo)}{\structtwo}, \dir) \typearrow \vc
}
\end{mathpar}

\begin{mathpar}
\inferrule[well\_constrained\_int\_reverse\_range]{
    \astlabel(\vcone) = \wellconstrained(\csone) \land \astlabel(\vctwo) = \wellconstrained(\cstwo)\\
    (\botcs, \topcs) \eqdef \choice{\dir = \UP}{(\csone, \cstwo)}{(\cstwo, \csone)}\\
    \minconstraints(\tenv, \botcs) \typearrow \vbot \in \Z\\
    \maxconstraints(\tenv, \topcs) \typearrow \vvtop \in \Z\\
    \vbot > \vvtop
}{
  \getforconstraints(\tenv, \overname{\TInt(\vcone)}{\structone}, \overname{\TInt(\vctwo)}{\structtwo}, \dir) \typearrow
  \overname{\wellconstrained(\csone)}{\vc}
}
\end{mathpar}

\begin{mathpar}
\inferrule[well\_constrained\_int\_symbolic\_range]{
    \astlabel(\vcone) = \wellconstrained(\csone) \land \astlabel(\vctwo) = \wellconstrained(\cstwo)\\
    (\botcs, \topcs) \eqdef \choice{\dir = \UP}{(\csone, \cstwo)}{(\cstwo, \csone)}\\
    \minconstraints(\tenv, \botcs) \typearrow \vbot\\
    \maxconstraints(\tenv, \topcs) \typearrow \vvtop\\
    (\vbot \not\in \Z \lor \vvtop \not\in \Z)\\
    \botcs = [\ConstraintExact(\ebot)]\\
    \topcs = [\ConstraintExact(\etop)]\\
    \vc \eqdef \wellconstrained([\ConstraintRange(\ebot, \etop)])
}{
  \getforconstraints(\tenv, \overname{\TInt(\vcone)}{\structone}, \overname{\TInt(\vctwo)}{\structtwo}, \dir) \typearrow \vc
}
\end{mathpar}

\begin{mathpar}
\inferrule[well\_constrained\_else]{
  \astlabel(\vcone) = \wellconstrained(\csone) \land \astlabel(\vctwo) = \wellconstrained(\cstwo)\\
  (\botcs, \topcs) \eqdef \choice{\dir = \UP}{(\csone, \cstwo)}{(\cstwo, \csone)}\\
  \minconstraints(\tenv, \botcs) \typearrow \vbot\\
  \maxconstraints(\tenv, \topcs) \typearrow \vvtop\\
  (\vbot \not\in \Z \lor \vvtop \not\in \Z)\\
  (\botcs \neq [\ConstraintExact(\Ignore)] \lor \botcs \neq [\ConstraintExact(\Ignore)])
}{
  \getforconstraints(\tenv, \overname{\TInt(\vcone)}{\structone}, \overname{\TInt(\vctwo)}{\structtwo}, \dir) \typearrow \overname{\unconstrained}{\vc}
}
\end{mathpar}
\end{emptyformal}

\section{TypingRule.MinConstraints \label{sec:TypingRule.MinConstraints}}
The function
\hypertarget{def-minconstraints}{}
\[
  \minconstraints(\overname{\staticenvs}{\tenv} \aslsep \overname{\intconstraint^*}{\vcs}) \aslto
  \overname{\overname{\Z}{\vi} \cup \{\minmaxtop\}}{r}
\]
operate over a list of constraints $\vcs$ to find whether there is an integer $\vi$ that represents the
minimal value of all the constraints. Otherwise, the result is the special value $\minmaxtop$.

\subsection{Prose}
One of the following applies:
\begin{itemize}
  \item All of the following apply (\textsc{single}):
  \begin{itemize}
    \item $\vcs$ consists of a single constraint $\vc$;
    \item $r$ is the result of finding the minimal constraint in $\vc$ via $\minconstraint$.
  \end{itemize}

  \item All of the following apply (\textsc{head\_tail}):
  \begin{itemize}
    \item $\vcs$ is a list consisting of a constraint $\vc_1$ followed by a non-empty list of constraints $\vc_{2..k}$;
    \item finding the minimal constraint in $\vc_1$ via $\minconstraint$ results in the integer $a$ or $\minmaxtop$,
          which short-circuits the entire evaluation;
    \item finding the minimal constraints in $\vc_{2..k}$ via $\minconstraints$ results in the integer $b$ or $\minmaxtop$,
          which short-circuits the entire evaluation;
    \item $r$ is the minimum of $a$ and $b$.
  \end{itemize}
\end{itemize}

\begin{emptyformal}
\subsection{Formally}
\begin{mathpar}
\inferrule[single]{
  \minconstraint(\tenv, \vc) \typearrow r
}{
  \minconstraints(\tenv, [\vc]) \typearrow r
}
\and
\inferrule[head\_tail]{
  k > 1\\
  \minconstraint(\tenv, \vc_1) \typearrow a \terminateas \minmaxtop\\\\
  \minconstraints(\tenv, \vc_{2..k}) \typearrow b \terminateas \minmaxtop
}{
  \minconstraints(\tenv, [\vc_{1..k}]) \typearrow \overname{\min\{a,b\}}{r}
}
\end{mathpar}
\end{emptyformal}

\section{TypingRule.MaxConstraints \label{sec:TypingRule.MaxConstraints}}
The function
\hypertarget{def-maxconstraints}{}
\[
  \maxconstraints(\overname{\staticenvs}{\tenv} \aslsep \overname{\intconstraint^*}{\vcs}) \aslto
  \overname{\overname{\Z}{\vi} \cup \{\minmaxtop\}}{r}
\]
operate over a list of constraints $\vcs$ to find whether there is an integer $\vi$ that represents the
maximal value of all the constraints. Otherwise, the result is the special value $\minmaxtop$.

\subsection{Prose}
One of the following applies:
\begin{itemize}
  \item All of the following apply (\textsc{single}):
  \begin{itemize}
    \item $\vcs$ consists of a single constraint $\vc$;
    \item $r$ is the result of finding the maximal constraint in $\vc$ via $\maxconstraint$.
  \end{itemize}

  \item All of the following apply (\textsc{head\_tail}):
  \begin{itemize}
    \item $\vcs$ is a list consisting of a constraint $\vc_1$ followed by a non-empty list of constraints $\vc_{2..k}$;
    \item finding the maximal constraint in $\vc_1$ via $\maxconstraint$ results in the integer $a$ or $\minmaxtop$,
          which short-circuits the entire evaluation;
    \item finding the maximal constraints in $\vc_{2..k}$ via $\maxconstraints$ results in the integer $b$ or $\minmaxtop$,
          which short-circuits the entire evaluation;
    \item $r$ is the maximum of $a$ and $b$.
  \end{itemize}
\end{itemize}

\begin{emptyformal}
\subsection{Formally}
\begin{mathpar}
\inferrule[single]{
  \maxconstraint(\tenv, \vc) \typearrow r
}{
  \maxconstraints(\tenv, [\vc]) \typearrow r
}
\and
\inferrule[head\_tail]{
  k > 1\\
  \maxconstraint(\tenv, \vc_1) \typearrow a \terminateas \minmaxtop\\\\
  \maxconstraints(\tenv, \vc_{2..k}) \typearrow b \terminateas \minmaxtop
}{
  \maxconstraints(\tenv, [\vc_{1..k}]) \typearrow \overname{\max\{a,b\}}{r}
}
\end{mathpar}
\end{emptyformal}

\section{TypingRule.MinConstraint \label{sec:TypingRule.MinConstraint}}
\hypertarget{def-minconstraint}{}
The function
\[
  \minconstraint(\overname{\staticenvs}{\tenv} \aslsep \overname{\intconstraint}{\vc}) \aslto \overname{\Z \cup \{\minmaxtop\}}{r}\\
\]
infers an integer representing the minimal value of the constraint $\vc$ if it can be symbolically
simplified into an integer literal. Otherwise, the result is $\minmaxtop$.

\subsection{Prose}
One of the following applies:
\begin{itemize}
  \item All of the following apply (\textsc{min\_exact\_literal}):
  \begin{itemize}
    \item $\vc$ is a single value constraint with expression $\ve$, that is, $\ConstraintExact(\ve)$;
    \item symbolically simplifying $\ve$ via $\reduceexpr$ yields a literal integer expression for $\vi$;
    \item $r$ is $\vi$.
  \end{itemize}

  \item All of the following apply (\textsc{min\_exact\_top}):
  \begin{itemize}
    \item $\vc$ is a single value constraint with expression $\ve$, that is, $\ConstraintExact(\ve)$;
    \item symbolically simplifying $\ve$ via $\reduceexpr$ does not yield a literal integer expression for $\vi$;
    \item $r$ is $\minmaxtop$.
  \end{itemize}

  \item All of the following apply (\textsc{min\_range\_literal}):
  \begin{itemize}
    \item $\vc$ is a range constraint with start expression $\ve$, that is, $\ConstraintRange(\ve, \Ignore)$;
    \item symbolically simplifying $\ve$ via $\reduceexpr$ yields a literal integer expression for $\vi$;
    \item $r$ is $\vi$.
  \end{itemize}

  \item All of the following apply (\textsc{min\_range\_top}):
  \begin{itemize}
    \item $\vc$ is a range constraint with start expression $\ve$, that is, $\ConstraintRange(\ve, \Ignore)$;
    \item symbolically simplifying $\ve$ via $\reduceexpr$ does not yield a literal integer expression for $\vi$;
    \item $r$ is $\minmaxtop$.
  \end{itemize}
\end{itemize}

\begin{emptyformal}
\subsection{Formally}
\begin{mathpar}
\inferrule[min\_exact\_literal]{
  \reduceexpr(\tenv, \ve) \typearrow \ELInt{\vi}
}{
  \minconstraint(\tenv, \overname{\ConstraintExact(\ve)}{\vc}) \typearrow \vi
}
\and
\inferrule[min\_exact\_top]{
  \reduceexpr(\tenv, \ve) \typearrow \vep\\
  \astlabel(\vep) \neq \ELiteral
}{
  \minconstraint(\tenv, \overname{\ConstraintExact(\ve)}{\vc}) \typearrow \minmaxtop
}
\and
\inferrule[min\_range\_literal]{
  \reduceexpr(\tenv, \ve) \typearrow \ELInt{\vi}
}{
  \minconstraint(\tenv, \overname{\ConstraintRange(\ve, \Ignore)}{\vc}) \typearrow \vi
}
\and
\inferrule[min\_range\_top]{
  \reduceexpr(\tenv, \ve) \typearrow \vep\\
  \astlabel(\vep) \neq \ELiteral
}{
  \minconstraint(\tenv, \overname{\ConstraintRange(\ve, \Ignore)}{\vc}) \typearrow \minmaxtop
}
\end{mathpar}
\end{emptyformal}

\section{TypingRule.MaxConstraint \label{sec:TypingRule.MaxConstraint}}
\hypertarget{def-maxconstraint}{}
\[
  \maxconstraint(\overname{\staticenvs}{\tenv} \aslsep \overname{\intconstraint}{\vc}) \aslto \Z \cup \{\minmaxtop\}\\
\]
infers an integer representing the maximal value of the constraint $\vc$ if it can be symbolically
simplified into an integer literal. Otherwise, the result is $\minmaxtop$.

\subsection{Prose}
One of the following applies:
\begin{itemize}
  \item All of the following apply (\textsc{max\_exact\_literal}):
  \begin{itemize}
    \item $\vc$ is a single value constraint with expression $\ve$, that is, $\ConstraintExact(\ve)$;
    \item symbolically simplifying $\ve$ via $\reduceexpr$ yields a literal integer expression for $\vi$;
    \item $r$ is $\vi$.
  \end{itemize}

  \item All of the following apply (\textsc{max\_exact\_top}):
  \begin{itemize}
    \item $\vc$ is a single value constraint with expression $\ve$, that is, $\ConstraintExact(\ve)$;
    \item symbolically simplifying $\ve$ via $\reduceexpr$ does not yield a literal integer expression for $\vi$;
    \item $r$ is $\minmaxtop$.
  \end{itemize}

  \item All of the following apply (\textsc{max\_range\_literal}):
  \begin{itemize}
    \item $\vc$ is a range constraint with an end expression $\ve$, that is, $\ConstraintRange(\Ignore, \ve)$;
    \item symbolically simplifying $\ve$ via $\reduceexpr$ yields a literal integer expression for $\vi$;
    \item $r$ is $\vi$.
  \end{itemize}

  \item All of the following apply (\textsc{max\_range\_top}):
  \begin{itemize}
    \item $\vc$ is a range constraint with an end expression $\ve$, that is, $\ConstraintRange(\Ignore, \ve)$;
    \item symbolically simplifying $\ve$ via $\reduceexpr$ does not yield a literal integer expression for $\vi$;
    \item $r$ is $\minmaxtop$.
  \end{itemize}
\end{itemize}

\begin{mathpar}
\inferrule[max\_exact\_literal]{
  \reduceexpr(\tenv, \ve) \typearrow \ELInt{\vi}
}{
  \maxconstraint(\tenv, \overname{\ConstraintExact(\ve)}{\vc}) \typearrow \vi
}
\and
\inferrule[max\_exact\_top]{
  \reduceexpr(\tenv, \ve) \typearrow \vep\\
  \astlabel(\vep) \neq \ELiteral
}{
  \maxconstraint(\tenv, \overname{\ConstraintExact(\ve)}{\vc}) \typearrow \minmaxtop
}
\and
\inferrule[max\_range\_literal]{
  \reduceexpr(\tenv, \ve) \typearrow \ELInt{\vi}
}{
  \maxconstraint(\tenv, \overname{\ConstraintRange(\Ignore, \ve)}{\vc}) \typearrow \vi
}
\and
\inferrule[max\_range\_top]{
  \reduceexpr(\tenv, \ve) \typearrow \vep\\
  \astlabel(\vep) \neq \ELiteral
}{
  \maxconstraint(\tenv, \overname{\ConstraintRange(\Ignore, \ve)}{\vc}) \typearrow \minmaxtop
}
\end{mathpar}

%%%%%%%%%%%%%%%%%%%%%%%%%%%%%%%%%%%%%%%%%%%%%%%%%%%%%%%%%%%%%%%%%%%%%%%%%%%%%%%%%%%%
\chapter{Typing of Blocks}
%%%%%%%%%%%%%%%%%%%%%%%%%%%%%%%%%%%%%%%%%%%%%%%%%%%%%%%%%%%%%%%%%%%%%%%%%%%%%%%%%%%%
\hypertarget{def-annotateblock}{}
The function
\[
  \annotateblock{\overname{\staticenvs}{\tenv} \aslsep \overname{\stmt}{\vs}} \aslto
  \overname{\stmt}{\newstmt} \cup \overname{\TTypeError}{\TypeErrorConfig}
\]
annotates a block statement $\vs$ in static environment $\tenv$ and returns the annotated
statement $\newstmt$ or a type error, if one is detected.

\section{TypingRule.Block \label{sec:TypingRule.Block}}

\subsection{Prose}
All of the following apply:
\begin{itemize}
  \item annotating the statement $\vs$ in $\tenv$ yields $(\newstmt, \newtenv)$ \ProseOrTypeError;
  \item the modified environment $\newtenv$ is dropped.
\end{itemize}

\subsection{Example}
\VerbatimInput{../tests/ASLTypingReference.t/TypingRule.Block0.asl}

\CodeSubsection{\BlockBegin}{\BlockEnd}{../Typing.ml}

\begin{emptyformal}
\subsection{Formally}
\begin{mathpar}
\inferrule{
  \annotatestmt{\tenv, \vs} \typearrow (\newstmt, \Ignore) \OrTypeError
}{
  \annotateblock{\tenv, \vs} \typearrow \newstmt
}
\end{mathpar}
\end{emptyformal}

\subsection{Comments}
A local identifier declared in a block statement (with \texttt{var}, \texttt{let}, or \texttt{constant})
is in scope from the point immediately after its declaration until the end of the
immediately enclosing block. This means, we can discard the environment at the end of
an enclosing block, which has the effect of dropping bindings of the identifiers declared inside the block.

  This is related to \identr{JBXQ}.

%%%%%%%%%%%%%%%%%%%%%%%%%%%%%%%%%%%%%%%%%%%%%%%%%%%%%%%%%%%%%%%%%%%%%%%%%%%%%%%%%%%%
\chapter{Typing of Catchers}
%%%%%%%%%%%%%%%%%%%%%%%%%%%%%%%%%%%%%%%%%%%%%%%%%%%%%%%%%%%%%%%%%%%%%%%%%%%%%%%%%%%%
\hypertarget{def-annotatecatcher}{}
The function
\[
  \annotatecatcher{
    \overname{\staticenvs}{\tenv} \aslsep
    (\overname{\langle\identifier\rangle}{\nameopt} \times \overname{\ty}{\tty} \times \overname{\stmt}{\vstmt})
  } \aslto
  (\overname{\langle\identifier\rangle}{\nameopt} \times \overname{\ty}{\ttyp} \times \overname{\stmt}{\newstmt})
  \cup \overname{\TTypeError}{\TypeErrorConfig}
\]
annotates a catcher given by the optional name of the matched exception --- $\nameopt$ ---
the exception type --- $\tty$ --- and the statement to execute upon catching the exception --- $\vstmt$.
The result is the catcher with the same optional name --- $\nameopt$, an annotated type $\ttyp$, and annotated statement $\newstmt$.
The result is a type error, if one is detected.

One of the following applies:
\begin{itemize}
\item TypingRule.CatcherNone (see \secref{TypingRule.CatcherNone}),
\item TypingRule.CatcherSome (see \secref{TypingRule.CatcherSome}).
\end{itemize}

\section{TypingRule.CatcherNone \label{sec:TypingRule.CatcherNone}}

\subsection{Prose}
All of the following apply:
\begin{itemize}
  \item the catcher has no named identifier, that is, $(\None, \tty, \vstmt)$;
  \item annotating the type $\tty$ in $\tenv$ yields $\ttyp$ \ProseOrTypeError;
  \item determining whether $\ttyp$ has the \structure\ of an exception type yields $\True$ \ProseOrTypeError;
  \item annotating the block $\vstmt$ in $\tenv$ yields $\newstmt$.
\end{itemize}

\subsection{Example}

\CodeSubsection{\CatcherNoneBegin}{\CatcherNoneEnd}{../Typing.ml}

\begin{emptyformal}
\subsection{Formally}
\begin{mathpar}
\inferrule{
  \annotatetype{\tenv, \vt} \typearrow \ttyp \OrTypeError\\
  \checkstructurelabel(\tenv, \ttyp, \TException) \typearrow \True \OrTypeError\\
  \annotateblock{\tenv, \vstmt} \typearrow \newstmt \OrTypeError
}{
  \annotatecatcher{\tenv, (\overname{\None}{\nameopt}, \tty, \vstmt)} \typearrow (\overname{\None}{\nameopt}, \ttyp, \newstmt)
}
\end{mathpar}
\end{emptyformal}

\subsection{Comments}
  This is related to \identr{SDJK}.

\section{TypingRule.CatcherSome \label{sec:TypingRule.CatcherSome}}

\subsection{Prose}
All of the following apply:
\begin{itemize}
  \item the catcher has a named identifier, that is, $(\langle\name\rangle, \tty, \vstmt)$;
  \item annotating the type $\tty$ in $\tenv$ yields $\ttyp$ \ProseOrTypeError;
  \item determining whether $\ttyp$ has the \structure\ of an exception type yields $\True$ \ProseOrTypeError;
  \item the identifier $\name$ is not bound in $\tenv$;
  \item binding $\name$ in the local environment of $\tenv$ with the type $\ttyp$ as an immutable variable
        (that is, with the local declaration keyword $\LDKLet$), yields the static environment $\tenvp$;
  \item annotating the block $\vstmt$ in $\tenvp$ yields $\newstmt$.
\end{itemize}

\subsection{Example}

\CodeSubsection{\CatcherSomeBegin}{\CatcherSomeEnd}{../Typing.ml}

\begin{emptyformal}
\subsection{Formally}
\begin{mathpar}
\inferrule{
  \annotatetype{\tenv, \vt} \typearrow \ttyp \OrTypeError\\
  \checkstructurelabel(\tenv, \ttyp, \TException) \typearrow \True \OrTypeError\\\\
  \checkvarnotinenv{\tenv, \name} \typearrow \True \OrTypeError\\\\
  \addlocal(\tenv, \name, \ttyp, \LDKLet) \typearrow \tenvp\\
  \annotateblock{\tenvp, \vstmt} \typearrow \newstmt \OrTypeError
}{
  \annotatecatcher{\tenv, (\overname{\langle\name\rangle}{\nameopt}, \tty, \vstmt)} \typearrow
  (\overname{\langle\name\rangle}{\nameopt}, \ttyp, \newstmt)
}
\end{mathpar}
\end{emptyformal}

\subsection{Comments}
  This is related to \identr{SDJK}, \identr{WVXS}, \identi{FCGK}.

%%%%%%%%%%%%%%%%%%%%%%%%%%%%%%%%%%%%%%%%%%%%%%%%%%%%%%%%%%%%%%%%%%%%%%%%%%%%%%%%%%%%
\chapter{Typing of Subprogram Calls \label{chap:TypingSubprogramCalls}}
%%%%%%%%%%%%%%%%%%%%%%%%%%%%%%%%%%%%%%%%%%%%%%%%%%%%%%%%%%%%%%%%%%%%%%%%%%%%%%%%%%%%
\hypertarget{def-annotatecall}
The function
\[
  \begin{array}{rl}
  \annotatecall{ &
    \overname{\staticenvs}{\tenv} \aslsep
    \overname{\identifier}{\name} \aslsep
    \overname{\expr^*}{\vargs} \aslsep
    \overname{\subprogramtype}{\calltype}
  } \aslto \\ &
  (\overname{\identifier}{\nameone} \aslsep
  \overname{\expr^*}{\vargsone} \aslsep
  \overname{(\identifier\times\expr)^*}{\eqsone} \aslsep
  \overname{\langle \ty \rangle}{\rettyone})
\end{array}
\]
annotates the call to subprogram $\name$ with arguments $\vargs$,
parameters $\eqs$, and call type $\calltype$, resulting in values for an annotated call ---
$\nameone$, $\vargsone$, $\eqstwo$, $\rettyone$ --- or a type error if one is detected.

The rule TypingRule.FCall (see \secref{TypingRule.FCall}) applies.

We also define helper functions via respective rules:
\begin{itemize}
  \item TypingRule.FindCheckDeduce (see \secref{TypingRule.FindCheckDeduce}),
  which finds a subprogram that matches the call, checks for type errors, and infers expressions
  for parameters.
  % \item TypingRule.FindSubprogram (see \secref{TypingRule.FindSubprogram}),
  % which finds a subprogram that clashes with the subprogram
  % \item TypingRule.SubprogramFinder (see \secref{TypingRule.SubprogramFinder}),
  % \item TypingRule.HasArgClash (see \secref{TypingRule.HasArgClash}),
  % \item TypingRule.DeduceEquations (see \secref{TypingRule.DeduceEquations}),
  % \item TypingRule.RenameTypeEquations (see \secref{TypingRule.RenameTypeEquations}),
\end{itemize}

\section{TypingRule.FCall \label{sec:TypingRule.FCall}}
\subsection{Prose}
One of the following applies:
\begin{itemize}
  \item All of the following apply (\textsc{function\_or\_getter}):
  \begin{itemize}
    \item Applying $\findcheckdeduce$ to $\tenv$, $\name$, $\exprs$, $\eqs$, and $\calltype$
    yields $(\nameone, \vargsone, \eqsone, \langle \tty \rangle)$ \ProseOrTypeError;
    \item $\calltype$ is either a function $\STFunction$ or a getter $\STGetter$;
    \item substituting the variables appearing in $\tty$ by the corresponding expressions,
    according to $\eqsone$ yields $\retty$;
    \item $\rettyone$ is $\langle\retty\rangle$.
  \end{itemize}

  \item All of the following apply (\textsc{procedure\_or\_setter}):
  \begin{itemize}
    \item Applying $\findcheckdeduce$ to $\tenv$, $\name$, $\exprs$, $\eqs$, and $\calltype$
    yields $(\nameone, \vargsone, \eqsone, \None)$;
    \item $\calltype$ is either a procedure $\STProcedure$ or a setter $\STSetter$;
    \item $\rettyone$ is $\None$.
  \end{itemize}

  \item All of the following apply (\textsc{mismatch}):
  \begin{itemize}
    \item Applying $\findcheckdeduce$ to $\tenv$, $\name$, $\exprs$, $\eqs$, and $\calltype$
    yields $(\nameone, \vargsone, \eqsone, \retty)$;
    \item the following condition does not hold:
    $\retty$ is $\None$ if and only if $\calltype$ is one of $\STProcedure$ or $\STSetter$;
    \item a type error is returned indicating that the existence of a return value does not match
    the kind of subprogram.
  \end{itemize}
\end{itemize}

\subsection{Example}

\CodeSubsection{\FCallBegin}{\FCallEnd}{../Typing.ml}

\begin{emptyformal}
\subsection{Formally}
\begin{mathpar}
\inferrule[function\_or\_getter]{
  \findcheckdeduce(\tenv, \name, \exprs, \calltype) \typearrow \\
  (\nameone, \vargsone, \eqsone, \langle \tty \rangle) \OrTypeError\\\\
  \calltype \in \{\STFunction, \STGetter, \STEmptyGetter\}\\
  \tododefine{rename\_ty\_eqs}(\tenv, \eqsone, \tty) \typearrow \retty
}
{
  \annotatecall{\tenv, \name, \exprs, \calltype} \typearrow (\nameone, \vargsone, \eqsone, \langle\retty\rangle)
}
\and
\inferrule[procedure\_or\_setter]{
  \findcheckdeduce(\tenv, \name, \exprs,\calltype) \typearrow \\
  (\nameone, \vargsone, \eqsone, \None) \\\\
  \calltype \in \{\STProcedure, \STSetter, \STEmptySetter\}
}
{
  \annotatecall{\tenv, \name, \exprs, \calltype} \typearrow (\nameone, \vargsone, \eqsone, \None)
}
\and
\inferrule[mismatch]{
  \findcheckdeduce(\tenv, \name, \exprs, \calltype) \typearrow \\
  (\nameone, \vargsone, \eqsone, \retty) \\\\
  \neg(\calltype \in \{\STProcedure, \STSetter\} \leftrightarrow \retty=\None)
}
{
  \annotatecall{\tenv, \name, \exprs, \calltype} \typearrow \TypeErrorVal{CallMismatch}
}
\end{mathpar}
\end{emptyformal}

\subsection{Comments}
  This is related to \identi{VFDP}, \identd{TRFW}, \identr{KMDB},
  \identi{YMHX}, \identr{CCVD}, \identr{QYBH}, \identr{PFWQ}, \identr{ZLWD},
  \identi{FLKF}, \identd{PMBL}, \identr{MWBN}, \identr{TZSP}, \identr{SBWR},
  \identi{CMLP}, \identr{BQJG}, \identr{RTCF}.

\section{TypingRule.FindCheckDeduce \label{sec:TypingRule.FindCheckDeduce}}
\hypertarget{def-findcheckdeduce}{}
The function
\[
  \begin{array}{rl}
    \findcheckdeduce( &
      \overname{\staticenvs}{\tenv} \aslsep
      \overname{\identifier}{\name} \aslsep
      \overname{\expr^*}{\vargs} \aslsep
      \overname{(\identifier\times\expr)^*}{\eqs} \aslsep
      \overname{\subprogramtype}{\calltype}
     ) \aslto \\ &
    \overname{\identifier}{\nameone} \aslsep
    \overname{\expr^*}{\vargsone} \aslsep
    \overname{(\identifier\times\expr)^*}{\eqsone} \aslsep
    \overname{\subprogramtype}{\retty}
  \end{array}
\]
performs three tasks:
\begin{itemize}
  \item Resolving the correct subprogram from the rule arguments. That is, finding a subprogram
  that with the same name and argument types that \typeclash\ with the actual argument types;
  \item Deducing from the actual arguments and formal parameters equations ---
  the expressions associated with parameter names;
  \item Checking that the call expression matches the subprogram declaration.
\end{itemize}
The function takes a static environment $\tenv$, a subprogram $\name$,
a list of actual arguments $\vargs$,
parameters $\eqs$ arguments, and call type $\calltype$.
It returns a unique name $\nameone$, annotated actual arguments $\vargsone$,
parameter arguments $\eqsfour$, and an optional return type $\retty$.
A type error is returned, if one is detected.

\subsection{Prose}
All of the following apply:
\begin{itemize}
  \item Annotating the list of actual argument expressions $\vargs$ in $\tenv$ yields $\callerargtyped$ \ProseOrTypeError;
  \item $\callerargtyped$ is a list of pairs where the first element is a type and the second is an expression;
  \item splitting $\callerargtyped$ yields a list of types --- $\callerargtypes$ --- and a list of expressions ---
  $\vargsone$, respectively;
  \item finding the subprogram matching the name $\name$ and argument types $\callerargtypes$ in $\tenv$
  yields a tuple $(\extranargs, \nameone, \calleeargtypes, \retty, \calleeparams)$ \ProseOrTypeError;
  \item One of the following applies:
  \begin{itemize}
    \item All of the following apply (\textsc{bad\_arity}):
    \begin{itemize}
      \item the length of the actual list of arguments is not the same as the list of arguments of the matched
      subprogram;
      \item a type error is returned.
    \end{itemize}

    \item All of the following apply (\textsc{okay}):
    \begin{itemize}
      \item The list of parameter arguments $\eqs$ is reversed and prepended to the list of parameter arguments
      i$\extranargs$, yielding $\eqstwo$;
      \item annotating static constrained integers in $\eqstwo$ in $\tenv$ yields $\eqsthree$ \ProseOrTypeError;
      \item annotating the callee parameters using $\calleeargtypes$ and $\callerargtypes$ in $\tenv$
      yields $\eqsfour'$;
      \item $\eqsfour$ is $\eqsfour'$ concatenated with $\eqsthree$;
      \item checking that $\callerargtypes$ \typesatisfies\ $\calleeargtypes$ in $\tenv$ yields
      $\True$ \ProseOrTypeError;
      \item checking that each callee parameter in $\calleeparams$ is defined by the call
      parameters $\eqsfour$ yields $\True$ \ProseOrTypeError.
    \end{itemize}
  \end{itemize}
\end{itemize}

\CodeSubsection{\FindCheckDeduceBegin}{\FindCheckDeduceEnd}{../Typing.ml}

\begin{emptyformal}
\subsection{Formally}
\begin{mathpar}
\inferrule[bad\_arity]{
  \annotateexprlist{\tenv, \vargs} \typearrow \callerargtyped \OrTypeError\\
  \splitlist(\callerargtyped) = (\callerargtypes, \vargsone)\\
  \findsubprogram(\tenv, \name, \callerargtypes) \typearrow \\
  (\Ignore, \Ignore, \calleeargtypes, \Ignore, \Ignore) \OrTypeError\\\\
  \equallength(\calleeargtypes, \vargsone) \typearrow \False
}
{
  \findcheckdeduce(\tenv, \name, \vargs, \eqs, \calltype) \typearrow \\
  \TypeErrorVal{CallBadArity}
}
\and
\inferrule[okay]{
  \annotateexprlist{\tenv, \vargs} \typearrow \callerargtyped \OrTypeError\\
  \splitlist(\callerargtyped) = (\callerargtypes, \defpoint{\vargsone})\\
  \findsubprogram(\tenv, \name, \callerargtypes) \typearrow \\
  (\extranargs, \defpoint{\nameone}, \calleeargtypes, \defpoint{\retty}, \calleeparams) \OrTypeError\\\\
  \equallength(\calleeargtypes, \vargsone) \typearrow \True\\
  \eqs \eqname \veq_{1..k}\\
  \eqstwo \eqdef [i=k..1: \veq_i] + \extranargs\\
  % \tododefine{apply\_eqs}(\tenv, \calleeargtypes, \callerargtyped) \typearrow \eqstwo' \OrTypeError\\
  % \eqstwo \eqdef \eqstwo' + \eqsone\\
  \tododefine{annotate\_static\_constrained\_integers}(\tenv, \eqstwo) \typearrow \eqsthree \OrTypeError\\
  \tododefine{annotate\_callee\_params}(\tenv, \calleeargtypes, \callerargtypes) \typearrow \eqsfour' \OrTypeError\\
  \defpoint{\eqsfour} \eqdef \eqsfour' + \eqsthree\\
  \tododefine{check\_args\_typesat}(\tenv, \calleeargtypes, \callerargtypes) \typearrow \True \OrTypeError\\
  \tododefine{check\_callee\_params}(\tenv, \calleeparams, \eqsfour) \typearrow \True \OrTypeError\\
}
{
  \findcheckdeduce(\tenv, \name, \vargs, \eqs, \calltype) \typearrow
  (\nameone, \vargsone, \eqsfour, \retty)
}
\end{mathpar}
\end{emptyformal}

\isempty{\subsection{Comments}}

\section{TypingRule.FindSubprogram \label{sec:TypingRule.FindSubprogram}}
\hypertarget{def-findsubprogram}{}
The function
\[
  \begin{array}{rl}
    \findsubprogram( &
      \overname{\staticenvs}{\tenv} \aslsep
      \overname{\identifier}{\name} \aslsep
      \overname{\ty^*}{\callerargtypes}
     ) \aslto \\ &
    (
      \overname{(\identifier\times\expr)^*}{\extranargs} \aslsep
      \overname{\identifier}{\nameone} \aslsep
      \overname{(\identifier\times\ty)^*}{\calleeargtypes} \aslsep
      \overname{\langle\ty\rangle}{\retty} \aslsep
      \overname{(\identifier \times \langle\ty\rangle)}{\calleeparams}
    )
  \end{array}
\]
takes the name of a subprogram --- $\name$ --- and a list argument types --- $\calleeargtypes$,
and determines whether:
\begin{itemize}
  \item there is no declared subprogram that matches $\name$ and $\calleeargtypes$;
  \item there is exactly one subprogram that matches $\name$ and $\calleeargtypes$;
  \item there is more than one subprogram that matches ($\name$ and $\calleeargtypes$);
\end{itemize}
The first and last cases indicate a type error.
If the second case holds, the function returns:
\begin{itemize}
\item $\extranargs$ --- a list matching parameters to their caller expressions based on bitvector parameters;
\item $\identifier$ --- an identifier that uniquely matches this subprogram;
\item $\calleeargtypes$ --- the list of formal argument types of the matched subprogram;
\item $\retty$ --- the type of the returned value, in case of a function or a getter;
\item $\calleeparams$ --- parameters declared in the matched subprogram.
\end{itemize}
A type error is returned, if one is detected.

\begin{emptyformal}
\subsection{Formally}
% \begin{mathpar}
% \inferrule{
% }
% {
%   \findsubprogram(\tenv, \name, \callerargtypes) \typearrow \\
%   (\extranargs, \nameone, \calleeargtypes, \retty, \calleeparams)
% }
% \end{mathpar}
\end{emptyformal}

%%%%%%%%%%%%%%%%%%%%%%%%%%%%%%%%%%%%%%%%%%%%%%%%%%%%%%%%%%%%%%%%%%%%%%%%%%%%%%%%%%%%
\chapter{Typing of Subprograms}
%%%%%%%%%%%%%%%%%%%%%%%%%%%%%%%%%%%%%%%%%%%%%%%%%%%%%%%%%%%%%%%%%%%%%%%%%%%%%%%%%%%%
\hypertarget{def-annotatesubprogram}{}
The function
\[
  \annotatesubprogram{\overname{\staticenvs}{\tenv} \aslsep \overname{\func}{\vf}} \aslto \overname{\func}{\vfp}
\]
annotates a subprogram $\vf$ in an environment $\tenv$, resulting in an annotated subprogram $\vfp$,
or a type error, if one is detected.

\section{TypingRule.Subprogram \label{sec:TypingRule.Subprogram}}

\subsection{Prose}
All of the following apply:
\begin{itemize}
  \item $\vf$ is a $\func$ AST node subprogram body $\body$;
  \item annotating the block $\body$ in $\tenv$ as per \secref{TypingRule.Block} yields $\newbody$ \ProseOrTypeError;
  \item $\vfp$ is $\vf$ with the subprogram body substituted with $\newbody$.
\end{itemize}

\subsection{Example}

\CodeSubsection{\SubprogramBegin}{\SubprogramEnd}{../Typing.ml}

\begin{emptyformal}
\subsection{Formally}
\begin{mathpar}
\inferrule{
  {
    \begin{array}{rrcl}
  \vf \eqname \{ & \funcname            &:& \id,\\
                 & \funcparameters      &:& \vp,\\
                 & \funcargs            &:& \vargs,\\
                 & \funcbody            &:& \SBASL(\body),\\
                 & \funcreturntype      &:& \vt,\\
                 & \funcsubprogramtype  &:& \subprogramtype \\
              \} &&&
    \end{array}
  }\\
  \annotateblock{\tenv, \body} \typearrow \newbody \OrTypeError\\
  {
    \begin{array}{rrcl}
  \vfp \eqdef \{ & \funcname            &:& \id,\\
                 & \funcparameters      &:& \vp,\\
                 & \funcargs            &:& \vargs,\\
                 & \funcbody            &:& \SBASL(\newbody),\\
                 & \funcreturntype      &:& \vt,\\
                 & \funcsubprogramtype  &:& \subprogramtype \\
              \} &&&
    \end{array}
  }\\
}{
  \annotatesubprogram{\tenv, \vf} \typearrow \vfp
}
\end{mathpar}
\end{emptyformal}

\subsection{Comments}
This is related to \identi{GHGK}, \identr{HWTV}, \identr{SCHV}, \identr{VDPC},
\identr{TJKQ}, \identi{LFJZ}, \identi{BZVB}, \identi{RQQB}.

%%%%%%%%%%%%%%%%%%%%%%%%%%%%%%%%%%%%%%%%%%%%%%%%%%%%%%%%%%%%%%%%%%%%%%%%%%%%%%%%%%%%
\chapter{Typing of Global Declarations}
%%%%%%%%%%%%%%%%%%%%%%%%%%%%%%%%%%%%%%%%%%%%%%%%%%%%%%%%%%%%%%%%%%%%%%%%%%%%%%%%%%%%
\hypertarget{def-declaredecl}{}
\hypertarget{def-annotatedecl}{}
The function
\[
  \annotatedecl{\overname{\staticenvs}{\tenv} \aslsep \overname{\decl}{\vd}}
  \aslto \overname{\staticenvs}{\newtenv}
\]
annotates a global declaration $\vd$ in a static environment $\tenv$,
resulting in an annotated declaration $\newd$ and a new environment $\newtenv$,
which contains the declared element.
A type error is returned, if one is detected.

The function
\[
  \declaredecl{\overname{\staticenvs}{\tenv} \aslsep \overname{\decl}{\vd}}
  \aslto \overname{\staticenvs}{\newtenv}
\]
adds a global declaration $\vd$ in a static environment $\tenv$,
resulting in a new environment $\newtenv$, which contains the declared element.
A type error is returned, if one is detected.

One of the following applies:
\begin{itemize}
  \item TypingRule.DeclareOneFunc (see \secref{TypingRule.DeclareOneFunc}).
  \item TypingRule.DeclareGlobalStorage (see \secref{TypingRule.DeclareGlobalStorage}).
  \item TypingRule.DeclareType (see \secref{TypingRule.DeclareType}).
\end{itemize}

\section{TypingRule.DeclareOneFunc \label{sec:TypingRule.DeclareOneFunc}}
\subsection{Prose}
Declaring a subprogram $\funcsig$ in a given environment $\tenv$ results
in $\newenv$ and all of the following apply:
\begin{itemize}
  \item $\name$ is the identifier associated with the subprogram declaration;
  \item $\tenv$ does not contain another subprogram declaration for $\name$ that clashes with $\funcsig$;
  \item $\newenv$ is $\tenv$ where
  $\funcsig$ has been added to the set of subprograms declared with $\name$ (\subprogramrenamings)
  and $\name$ is associated with $\funcsig$.
\end{itemize}

\subsection{Example}

\CodeSubsection{\DeclareOneFuncBegin}{\DeclareOneFuncEnd}{../Typing.ml}
\begin{emptyformal}
\subsection{Formally}
\begin{mathpar}
\inferrule{
  {
  \begin{array}{rrcl}
    \funcsig \eqdef \{ & \funcname            &:& \name,\\
                   & \funcparameters      &:& \vp,\\
                   & \funcargs            &:& \vargs,\\
                   & \funcbody            &:& \SBASL(\newbody),\\
                   & \funcreturntype      &:& \vt,\\
                   & \funcsubprogramtype  &:& \subprogramtype \\
                \} &&&
      \end{array}
    }\\
  \tododefine{add\_new\_func}(\tenv, \name, \vargs, \vp) \typearrow
  (\tenvone, \namep) \OrTypeError\\
  \checkvarnotinenv{\tenvone, \namep} \typearrow \True \OrTypeError\\\\
  \tododefine{check\_setter\_has\_getter}(\tenvone, \funcsig) \typearrow \True \OrTypeError\\\\
  {
  \begin{array}{rrcl}
    \funcsig \eqdef \{ & \funcname            &:& \namep,\\
                   & \funcparameters      &:& \vp,\\
                   & \funcargs            &:& \vargs,\\
                   & \funcbody            &:& \SBASL(\newbody),\\
                   & \funcreturntype      &:& \vt,\\
                   & \funcsubprogramtype  &:& \subprogramtype \\
                \} &&&
      \end{array}
    }\\
    \tododefine{add\_subprogram}(\namep, \funcsigone, \tenvone) \typearrow \newtenv \OrTypeError
}
{
  \declaredecl{\tenv, \funcsig} \typearrow (\newtenv, \funcsigone)
}
\end{mathpar}

\begin{mathpar}
\inferrule{
  \vf \eqname \{\funcbody: \SBASL(\Ignore)\}\\
  \tododefine{annotate\_and\_declare\_func}(\tenv, \vf) \typearrow (\newtenv, \vfone) \OrTypeError\\\\
  \annotatesubprogram{\tenv, \vfone} \typearrow \vftwo \OrTypeError\\\\
  \newd \eqdef \DFunc(\vftwo)
}
{
  \annotatedecl{\tenv, \DFunc(\vf)} \typearrow (\newd, \newenv)
}
\end{mathpar}

% We define the helper rules \hassubprogramtypeclash, \argsclash, and subprogramclash\ to determined whether
% two subprogram clash in terms of their subprogram types and their lists of argument types,
% and then use them to annotate a subprogram declaration.

% \begin{mathpar}
% \inferrule{}{\hassubprogramtypeclash(\tenv, \STFunction, \Ignore)}
% \and
% \inferrule{}{\hassubprogramtypeclash(\tenv, \Ignore, \STFunction)}
% \and
% \inferrule{}{\hassubprogramtypeclash(\tenv, \STProcedure, \Ignore)}
% \and
% \inferrule{}{\hassubprogramtypeclash(\tenv, \Ignore, \STProcedure)}
% \and
% \inferrule{}{\hassubprogramtypeclash(\tenv, \STGetter, \STGetter)}
% \and
% \inferrule{}{\hassubprogramtypeclash(\tenv, \STSetter, \STSetter)}
% \and
% \inferrule{
%   \texttt{t\_args1} = [i=1..k: (\Ignore, \vt_i)]\\
%   \texttt{s\_args1} = [i=1..k: (\Ignore, \vs_i)]\\
%   j \in 1..k\\
%   \typeclashes(\tenv, \vt_j, \vs_j)
% }
% {\argsclash(\tenv, \texttt{t\_args}, \texttt{s\_args})}
% \and
% \inferrule{
%   \argsclash(\vf.\subprogramtype, \texttt{g}.\subprogramtype)\\
%   \argsclash(\vf.\vargs, \texttt{g}.\vargs)\\
% }
% { \subprogramclash(\vf, \texttt{g}) }
% \and
% \inferrule{
%   \name = \funcsig.\name\\
%   \texttt{same\_named} = G^\tenv.\subprogramrenamings(\name)\\
%   \texttt{fo} \in \texttt{same\_named}: \neg\subprogramclash(\tenv, \funcsig, \texttt{fo})\\
%   G' = G^\tenv.\subprogramrenamings[\name \mapsto \texttt{same\_named} \cup \{\funcsig\}]\\
%   G'' = G'.\subprograms[\name \mapsto \funcsig]\\
%   \newenv = (G'', L^\tenv)
% }
% {
%   \declaredecl{\tenv, \funcsig} \typearrow G''
% }
% \end{mathpar}
\end{emptyformal}
\subsection{Comments}

This relates to \identi{HJRD}, \identd{BTBR}, \identi{FSFQ}, \identi{PFGQ}, and \identr{PGFC}.

\section{TypingRule.DeclareGlobalStorage \label{sec:TypingRule.DeclareGlobalStorage}}
\subsection{Prose}
Annotating a global storage declaration $\vd$ in a given environment $\tenv$ results
in $\newenv$ and one of the following applies:
\begin{itemize}
  \item All of the following apply:
  \begin{itemize}
    \item $\vd$ declares a global constant named \name, with initial value expression $\ve$, and without a type annotation.
    \item $\vv$ is the literal computed in \tenv\ by evaluating $\ve$.
    \item $\vt$ is the type inferred for $\vv$
    \item $\newenv$ is \tenv\ extended with a declaration of the constant \name, with initial value $\vv$ and type $\vt$.
  \end{itemize}

  \item All of the following apply:
  \begin{itemize}
    \item $\vd$ declares a global constant named \name, with initial value expression $\ve$, and type annotation $\tty$.
    \item $\vv$ is the literal computed in \tenv\ by evaluating $\ve$.
    \item $\vt$ is the type inferred for $\vv$.
    \item $\vt$ \typesatisfies\  $\tty$ in \tenv.
    \item $\newenv$ is \tenv\ extended with a declaration of the constant \name, with initial value $\vv$ and type $\tty$.
  \end{itemize}

  \item All of the following apply:
  \begin{itemize}
    \item $\vd$ declares a global constant or global let with no initial value expression.
    \item An error ``\texttt{Constants or let-bindings must be initialized}'' is returned.
  \end{itemize}

  \item All of the following apply:
  \begin{itemize}
    \item $\vd$ declares a global variable or config named \name, with no initial value expression, and type annotation $\tty$.
    \item One of the following applies:
    \begin{itemize}
      \item An error ``\texttt{identifier already declared}'' is returned and all of the following apply:
      \begin{itemize}
        \item $\name$ is not yet declared in the global environment.
        \item $\newenv$ is \tenv\ extended with a declaration of the global storage element named \name, and type $\tty$.
      \end{itemize}
    \end{itemize}
  \end{itemize}

  \item All of the following apply:
  \begin{itemize}
    \item $\vd$ declares a global storage element with named \name, with initial value expression $\ve$, and no type.
    \item $\vt$ is the type resulting from annotating the expression $\ve$ in \tenv.
    \item $\newenv$ is \tenv\ extended with a declaration of the global storage element named \name\ and type $\tty$.
  \end{itemize}

  \item All of the following apply:
  \begin{itemize}
    \item $\vd$ declares a global storage element with named \name, with initial value expression $\ve$, and type annotation $\tty$.
    \item \texttt{t, \vep} is the result of annotating the expression $\ve$ in \tenv.
    \item One of the following applies:
    \begin{itemize}
      \item $\vt$ does not type-satisfy $\tty$ in \tenv.
      \item A ``Conflicting types'' error is returned.
    \end{itemize}
    \item $\newenv$ is \tenv\ extended with a declaration of the global storage element named \name\ and type $\tty$.
  \end{itemize}
\end{itemize}

\subsection{Example}

\CodeSubsection{\DeclareGlobalStorageBegin}{\DeclareGlobalStorageEnd}{../Typing.ml}
\begin{emptyformal}
  \subsection{Formally}
\begin{mathpar}
  \inferrule[Case 1]{
    \gsd = \{\textsf{keyword} : \GDKConstant, \textsf{initial\_value} : \langle \ve \rangle, \textsf{ty} : \None, \textsf{name}:\name \}\\
    \texttt{reduce\_constants}(\tenv, \ve) \typearrow \vv\\
    \annotateliteral{\vv} \typearrow \vt\\
    \texttt{declare\_const}(\tenv, \name, \vt, \vv) \typearrow \newtenv
  }
  { \declaredecl{\tenv, \gsd} \typearrow \newtenv }
  \and
  \inferrule[Case 2]{
    \gsd = \{\textsf{keyword} : \GDKConstant, \textsf{initial\_value} : \langle \ve \rangle, \textsf{ty} : \langle\tty\rangle, \textsf{name}:\name \}\\
    \texttt{reduce\_constants}(\tenv, \ve) \typearrow \vv\\
    \annotateliteral{\vv} \typearrow \vt\\
    \typesat(\tenv, \vt, \tty)\\
    \texttt{declare\_const}(\tenv, \name, \tty, \vv) \typearrow \newtenv
  }
  { \declaredecl{\tenv, \gsd} \typearrow \newtenv }
  \and
  \inferrule[Case 4]{
    \gsd = \{\textsf{keyword} : \vk, \textsf{initial\_value} : \langle \ve \rangle, \textsf{ty} : \langle \tty \rangle, \textsf{name}:\name \}\\
    \vk \in \{\GDKConfig, \GDKVar\}\\
    G^\tenv.\globalstoragetypes(\name) = \bot\\
    \newtenv = (G^\tenv.\globalstoragetypes[\name \mapsto (\tty, \vk)], L^\tenv)
  }
  { \declaredecl{\tenv, \gsd} \typearrow \newtenv }
  \and
  \inferrule[Case 5]{
    \gsd = \{\textsf{keyword} : \vk, \textsf{initial\_value} : \langle \ve \rangle, \textsf{ty} : \None, \textsf{name}:\name \}\\
    G^\tenv.\globalstoragetypes(\name) = \bot\\
    \annotateexpr{\tenv, \ve} \typearrow (\vt, \Ignore)\\
    \newtenv = (G^\tenv.\globalstoragetypes[\name \mapsto (\vt, \vk)], L^\tenv)
  }
  { \declaredecl{\tenv, \gsd} \typearrow \newtenv }
  \and
  \inferrule[Case 6]{
    \gsd = \{\textsf{keyword} : \vk, \textsf{initial\_value} : \langle \ve \rangle, \textsf{ty} : \langle\tty\rangle, \textsf{name}:\name \}\\
    G^\tenv.\globalstoragetypes(\name) = \bot\\
    \annotateexpr{\tenv, \ve} \typearrow (\vt, \vep)\\
    \typesat(\tenv, \vt, \tty)\\
    \newtenv = (G^\tenv.\globalstoragetypes[\name \mapsto (\tty, \vk)], L^\tenv)
  }
  { \declaredecl{\tenv, \gsd} \typearrow \newtenv }
\end{mathpar}
\end{emptyformal}
\subsection{Comments}
This relates to \identr{YSPM} and \identr{FWQM}.

\section{TypingRule.DeclareType \label{sec:TypingRule.DeclareType}}
\subsection{Prose}
Declaring a type named $\name$ with a type specification $\tty$,
optionally a supertype $\sup$ and extra fields $\fields$, in a given environment $\tenv$ results
in $\tenvp$ and one of the following applies:
\begin{enumerate}
  \item All of the following apply:
  \begin{itemize}
    \item \name\ is already declared in \tenv.
    \item An error ``identifier already declared'' is returned.
  \end{itemize}
  \item All of the following apply:
  \begin{itemize}
    \item \name\ is not declared in \tenv.
    \item  \tenv',\tty' are the result of attempting to add $\name$ as a subtype of $\sup$ and
    constructing the type \tty' as \tty\ with the added fields $\fields$ in \tenv.
    \item \tty' is a valid type.
    \item \tenv'' is \tenv' extended with \tty'.
    \item If \tty' has the structure of an enumeration then \newtenv\ is \tenv'' extended with the declarations of constants for each identifier,
    and otherwise \newtenv\ is \tenv''.
  \end{itemize}
\end{enumerate}

\subsection{Example}

\CodeSubsection{\DeclareTypeBegin}{\DeclareTypeEnd}{../Typing.ml}
\begin{emptyformal}
  \subsection{Formally}
\newcommand\attemptaddsubtype[0]{\texttt{attempt\_add\_subtype}}
\newcommand\attemptaddenum[0]{\texttt{attempt\_add\_enum}}
\newcommand\checkisvalidtype[0]{\texttt{check\_is\_valid\_type}}

\begin{mathpar}
  % \inferrule{ \astlabel(\tty) \not\in \{\TRecord, \TException, \TBits\} }
  % { \checkisvalidtype(\tenv, \tty) }
  % \and
  \inferrule{
    \checkvarnotinenv{\tenv, \name}\\
    \attemptaddsubtype(\tenv, \tty, \vs) \typearrow (\tenv', \tty')\\
    \checkisvalidtype(\tenv, \tty')\\
    \tenv'' = (G^{\tenv'}.\globalstoragetypes[\name\mapsto \tty'], L^{\tenv'})\\
    \attemptaddenum(\tenv'', \tty') \typearrow \newtenv
  }
  { \declaredecl{\name, \tty, \vs} \typearrow \newtenv }
\end{mathpar}
\end{emptyformal}
\subsection{Comments}
This is related to \identr{DHRC}, \identd{YZBQ}, \identr{DWSP}, \identi{MZXL}, \identr{MDZD}, \identr{CHKR}.

%%%%%%%%%%%%%%%%%%%%%%%%%%%%%%%%%%%%%%%%%%%%%%%%%%%%%%%%%%%%%%%%%%%%%%%%%%%%%%%%%%%%
\chapter{Typing of Specifications}
%%%%%%%%%%%%%%%%%%%%%%%%%%%%%%%%%%%%%%%%%%%%%%%%%%%%%%%%%%%%%%%%%%%%%%%%%%%%%%%%%%%%

An ASL specification consists of a list of declarations.
They type system does not take the order of declarations into consideration.
More precisely, the type system operates over the declarations after they have been
ordered based on their dependencies.
Type checking the specification is considered successful if all declarations can be successfully annotated.

\section{TypingRule.Specification \label{sec:TypingRule.Specification}}
\subsection{Prose}
Annotating an ASL specification $\decls$ in an environment $\tenv$ results in an annotated specification $\declsp$
and a new environment $\newenv$ and all of the following apply:
\begin{itemize}
  \item $\ordereddecls$ is the result of topologically ordering $\decls$ according to their
  mutual dependencies.
  \item \newtenv\ is the result of declaring all global declarations in $\ordereddecls$ in \tenv.
  \item $\declsp$ is the result of annotating every declaration in $\decls$ in the environment \newtenv.
\end{itemize}

\subsection{Example}

\CodeSubsection{\SpecificationBegin}{\SpecificationEnd}{../Typing.ml}
\begin{emptyformal}
  \subsection{Formally}
\begin{mathpar}
  \inferrule{
    \texttt{order\_topologically}(\tenv, \decls) \typearrow \ordereddecls\\
    \ordereddecls = [i=1..k: \vd_i]\\
    \tenv_0 = \tenv\\
    i=1..k: \declaredecl{\tenv_{i-1}, \vd_{i-1}} \typearrow \tenv_{i}\\
    \newtenv = \tenv_{k}\\
    \decls' = [i=1..k: \texttt{annotate\_decl}(\newtenv, \vd_i) ]
  }
  { \annotatespec{\tenv, \decls} \typearrow (\decls', \newtenv) }
\end{mathpar}
\end{emptyformal}
\subsection{Comments}
This relates to \identi{LWQQ}.

%%%%%%%%%%%%%%%%%%%%%%%%%%%%%%%%%%%%%%%%%%%%%%%%%%%%%%%%%%%%%%%%%%%%%%%%%%%%%%%%%%%%
\chapter{Static Evaluation}
\label{chap:staticevaluation}
%%%%%%%%%%%%%%%%%%%%%%%%%%%%%%%%%%%%%%%%%%%%%%%%%%%%%%%%%%%%%%%%%%%%%%%%%%%%%%%%%%%%

\hypertarget{def-staticeval}{}
\section{TypingRule.StaticEval}
The partial function
\[
  \staticeval(\overname{\staticenvs}{\tenv} \aslsep \overname{\expr}{\ve}) \;\aslto\;
  \overname{\literals}{\vv} \cup \overname{\TTypeError}{\TypeErrorConfig}
\]
evaluates an expression $\ve$, from a subset of the set of all expressions, in environment $\tenv$, returning a literal $\vv$.

\begin{emptyformal}
\subsection{Formally}
\begin{mathpar}
\inferrule[e\_literal]{}
{
  \staticeval(\tenv, \ELiteral(\vv)) \typearrow \vv
}
\and
\inferrule[e\_var]{
  \lookupconstant(\tenv, \vx) \typearrow \vv
}{
  \staticeval(\tenv, \EVar(\vx)) \typearrow \vv
}
\end{mathpar}

\begin{mathpar}
\inferrule[e\_binop]{
  \staticeval(\tenv, \veone) \typearrow \vvone\\
  \staticeval(\tenv, \vetwo) \typearrow \vvtwo\\
  \staticbinop(\op, \vvone, \vvtwo) \typearrow \vv
}{
  \staticeval(\tenv, \EBinop(\op, \veone, \vetwo)) \typearrow \vv
}
\end{mathpar}

\begin{mathpar}
\inferrule[e\_unop]{
  \staticeval(\tenv, \veone) \typearrow \vvone\\
  \unop(\op, \vvone) \typearrow \vv
}{
  \staticeval(\tenv, \EUnop(\op, \veone)) \typearrow \vv
}
\end{mathpar}

\begin{mathpar}
\inferrule[e\_slice\_int]{
  \tododefine{slices\_to\_positions}(\tenv, \slices) \typearrow \positions\\
  \staticeval(\tenv, \veone) \typearrow \vvone\\
  \posmax \eqdef \max(\positions)\\
  \bv \eqdef \tododefine{bitvector\_of\_z}(\posmax + 1, \vi)\\
  \vv \eqdef \lbitvector(\tododefine{extract\_slice}(\bv, \positions))
}{
  \staticeval(\tenv, \ESlice(\lint(\vi), \slices)) \typearrow \vv
}
\end{mathpar}

\begin{mathpar}
\inferrule[e\_slice\_bitvector]{
  \tododefine{slices\_to\_positions}(\tenv, \slices) \typearrow \positions\\
  \staticeval(\tenv, \veone) \typearrow \vvone\\
  \tododefine{bitvector\_length}(\bv) > \posmax\\
  \vv \eqdef \lbitvector(\tododefine{extract\_slice}(\bv, \positions))
}{
  \staticeval(\tenv, \ESlice(\lbitvector(\bv), \slices)) \typearrow \vv
}
\end{mathpar}

\begin{mathpar}
\inferrule[e\_slice\_type\_error]{
  \astlabel(\veone) \not\in \{\lint, \lbitvector\}
}{
  \staticeval(\tenv, \ESlice(\veone, \slices)) \typearrow \TypeError(\texttt{TypeMismatch}(\veone, [\TInt, \TBits]))
}
\end{mathpar}

\begin{mathpar}
\inferrule[e\_cond]{
  \staticeval(\tenv, \econd) \typearrow \vcond\\
  \vcond \eqname \lbool(\vb)\\
  \vep \eqdef \choice{\vb}{\veone}{\vetwo}\\
  \staticeval(\tenv, \vep) \typearrow \vv
}{
  \staticeval(\tenv, \ECond(\econd, \veone, \vetwo)) \typearrow \vv
}
\end{mathpar}
\end{emptyformal}

%%%%%%%%%%%%%%%%%%%%%%%%%%%%%%%%%%%%%%%%%%%%%%%%%%%%%%%%%%%%%%%%%%%%%%%%%%%%%%%%%%%%
\chapter{Symbolic Subsumption Testing \label{chap:symbolicsubsumptiontesting}}
%%%%%%%%%%%%%%%%%%%%%%%%%%%%%%%%%%%%%%%%%%%%%%%%%%%%%%%%%%%%%%%%%%%%%%%%%%%%%%%%%%%%
\hypertarget{def-symbolicdomain}{}
The symbolic reasoning operates by first transforming types into expressions in a \emph{symbolic domain} AST
(defined next, reusing $\intconstraint$ from the parsed AST) over which it then operates:
\hypertarget{def-symdom}{}
\hypertarget{def-dbool}{}
\[
  \begin{array}{rcl}
    \symdom &::=& \DBool                    \hypertarget{def-dstring}{}\\
            &|  & \DString                  \hypertarget{def-dreal}{}\\
            &|  & \DReal                    \hypertarget{def-dsymbols}{}\\
            &|  & \DSymbols(\identifier^+)  \hypertarget{def-dint}{}\\
            &|  & \DInt(\intset)            \hypertarget{def-dbits}{}\\
            &|  & \DBits(\intset)           \hypertarget{def-intset}{} \hypertarget{def-finite}{}\\
    \intset &::=& \Finite(\Z^+)             \hypertarget{def-top}{}\\
            &|  & \Top                      \hypertarget{def-fromsymtax}{}\\
            &|  & \FromSyntax(\syntax)      \hypertarget{def-syntax}{}\\
    \syntax &::=& \intconstraint^*
  \end{array}
\]

\section{TypingRule.SymSubsumes}
\hypertarget{def-symsubsumes}{}
The predicate
\[
  \symsubsumes(\overname{\staticenvs}{\tenv} \aslsep \overname{\ty}{\vt} \aslsep \overname{\ty}{\vs})
  \aslto \overname{\Bool}{\vb}
\]
soundly approximates $\subsumes(\tenv, \vt, \vs)$.

\begin{emptyformal}
\subsection{Formally}
\begin{mathpar}
  \inferrule{
    \symdomoftype(\tenv, \vt) \typearrow \dt\\
    \symdomoftype(\tenv, \vs) \typearrow \ds\\
    \symdomissubset(\tenv, \dt, \ds) \typearrow \vb
  }
  {
    \symsubsumes(\tenv, \vt, \vs) \typearrow \vb
  }
\end{mathpar}
\end{emptyformal}

\section{TypingRule.SymDomOfType}
\hypertarget{def-symdomoftype}{}
The partial function
\[
  \symdomoftype(\overname{\staticenvs}{\tenv} \aslsep \overname{\ty}{\vt}) \aslto \overname{\symdom}{\vd}
\]
transforms a type $\vt$ in a static environment $\tenv$ into a symbolic domain $\vd$.

% | T_Bits (width, _) -> (
%     try
%       match of_expr env width with
%       | D_Int (Finite int_set as d) ->
%           if Z.equal (IntSet.cardinal int_set) Z.one then D_Bits d
%           else raise StaticEvaluationTop
%       | D_Int (FromSyntax [ Constraint_Exact _ ] as d) -> D_Bits d
%       | _ -> raise StaticEvaluationTop
%     with StaticEvaluationTop ->
%       D_Bits (FromSyntax [ Constraint_Exact width ]))
% | T_Array _ | T_Exception _ | T_Record _ | T_Tuple _ ->
%     failwith "Unimplemented: domain of a non singular type."
% | T_Named _ -> assert false (* make anonymous *)

\begin{emptyformal}
\subsection{Formally}
\begin{mathpar}
  \inferrule{}{ \symdomoftype(\tenv, \vt, \TBool) \typearrow \DBool }
  \and
  \inferrule{}{ \symdomoftype(\tenv, \vt, \TString) \typearrow \DString }
  \and
  \inferrule{}{ \symdomoftype(\tenv, \vt, \TReal) \typearrow \DReal }
  \and
  \inferrule{}{ \symdomoftype(\tenv, \vt, \TEnum(\vli)) \typearrow \DSymbols(\vli) }
  \and
  \inferrule{}{ \symdomoftype(\tenv, \vt, \TInt(\unconstrained)) \typearrow \DInt(\Top) }
  \and
  \inferrule{}{ \symdomoftype(\tenv, \vt, \TInt(\underconstrained(\id))) \typearrow \\
  \DInt(\FromSyntax([\ConstraintExact(\EVar(\id))])) }
  \and
  \inferrule{
    \intsetofintconstraints(\tenv, \vcs) \typearrow \vis
  }{ \symdomoftype(\tenv, \vt, \TInt(\wellconstrained(\vcs))) \typearrow \DInt(\vis)}
\end{mathpar}

\begin{mathpar}
  \inferrule{
    \symdomofexpr(\tenv, \width) \typearrow \DInt(\Finite([n]))
  }{
    \symdomoftype(\tenv, \vt, \TBits(\width)) \typearrow \DBits(\Finite([n]))
  }
  \and
  \inferrule{
    \symdomofexpr(\tenv, \width) \typearrow \DInt(\Finite([n_{1..k}]))\\
    k > 1
  }{
    \symdomoftype(\tenv, \vt, \TBits(\width)) \typearrow \\ \DBits(\FromSyntax([\ConstraintExact(\EVar(\width))]))
  }
  \and
  \inferrule{
    \symdomofexpr(\tenv, \width) \typearrow \DInt(\FromSyntax([\ConstraintExact(\vv)]))
  }{
    \symdomoftype(\tenv, \vt, \TBits(\width)) \typearrow \\ \DBits(\FromSyntax([\ConstraintExact(\vv)]))
  }
  \and
  \inferrule{
    \symdomofexpr(\tenv, \width) \typearrow \DInt(\FromSyntax(\vc))\\
    \vc \neq [\ConstraintExact(\vv)]
  }{
    \symdomoftype(\tenv, \vt, \TBits(\width)) \typearrow \\ \DBits(\FromSyntax([\ConstraintExact(\EVar(\width))]))
  }
  \and
  \inferrule{
    \symdomofexpr(\tenv, \width) \typearrow \DInt(\Top)
  }{
    \symdomoftype(\tenv, \vt, \TBits(\width)) \typearrow \\ \DBits(\FromSyntax([\ConstraintExact(\EVar(\width))]))
  }
\end{mathpar}
\end{emptyformal}

\section{TypingRule.SymDomOfExpr}
\hypertarget{def-symdomofexpr}{}

\section{TypingRule.SymDomOfLiteral}
\hypertarget{def-symdomofliteral}{}

\section{TypingRule.SymIntSetOfConstraints}
\hypertarget{def-intsetofintconstraintse}{}

\section{TypingRule.SymDomIsSubset}
\hypertarget{def-symdomissubset}{}

% | D_Bool, D_Bool | D_String, D_String | D_Real, D_Real -> true
% | D_Symbols s1, D_Symbols s2 -> ISet.subset s1 s2
% | D_Bits is1, D_Bits is2 | D_Int is1, D_Int is2 ->
%     int_set_is_subset env is1 is2

\begin{mathpar}
  \inferrule[bool]{}{ \symdomissubset(\tenv, \DBool, \DBool) \typearrow \True }
  \and
  \inferrule[string]{}{ \symdomissubset(\tenv, \DString, \DString) \typearrow \True }
  \and
  \inferrule[real]{}{ \symdomissubset(\tenv, \DReal, \DReal) \typearrow \True }
  \and
  \inferrule[symbols]{
    \vb \eqdef \{\vsone\} = \{\vstwo\}
  }{ \symdomissubset(\tenv, \DSymbols(\vsone), \DSymbols(\vstwo)) \typearrow \vb }
  \and
  \inferrule[bits]{
    \symintsetsubset(\tenv, \isone, \istwo) \typearrow \vb
  }{ \symdomissubset(\tenv, \DBits(\isone), \DBits(\istwo)) \typearrow \vb }
  \and
  \inferrule[int]{
    \symintsetsubset(\tenv, \isone, \istwo) \typearrow \vb
  }{ \symdomissubset(\tenv, \DInt(\isone), \DInt(\istwo)) \typearrow \vb }
  \and
  \inferrule[different\_labels]{
    \astlabel(\dt) \neq \astlabel(\ds)
  }{ \symdomissubset(\tenv, \dt, \ds) \typearrow \False }
\end{mathpar}

\section{TypingRule.SymIntSetSubset}
\hypertarget{def-symintsetsubset}{}

\section{TypingRule.SymSyntaxSubset}
\hypertarget{def-symsyntaxsubset}{}

%%%%%%%%%%%%%%%%%%%%%%%%%%%%%%%%%%%%%%%%%%%%%%%%%%%%%%%%%%%%%%%%%%%%%%%%%%%%%%%%%%%%
\chapter{Symbolic Reduction and Equivalence Testing \label{chap:symbolicequivalencetesting}}
%%%%%%%%%%%%%%%%%%%%%%%%%%%%%%%%%%%%%%%%%%%%%%%%%%%%%%%%%%%%%%%%%%%%%%%%%%%%%%%%%%%%

In this chapter, we define two forms of symbolic reasoning ---
\emph{symbolic reduction} and \emph{symbolic equivalence testing}.
Symbolic reduction simplifies expressions into \emph{equivalent} expressions
that are simpler to reason about.
In out context, equivalence means that we can substitute one expression for another without
affecting the semantics of the overall specification.
%
Symbolic equivalence is a \emph{conservative} test.
By conservative, we mean that if a test for equivalence returns $\True$ then the expressions
being compared are indeed equivalent, but if the test returns $\False$ then
there are two possibilities:
\begin{itemize}
  \item the expressions are not equivalent;
  \item the expressions are equivalent, but the reasoning power of our rules
  is not enough to prove it, and so we conservatively answer negatively.
\end{itemize}
In proof-theoretic terms, we can say that our equivalence tests are \emph{sound} but \emph{incomplete}.

Notice that for a conservative test, it is always correct to return $\False$.

We first define symbolic expressions and operations over symbolic expressions (\secref{symbolicexpressions})
and then we define the following rules:
\begin{itemize}
  \item TypingRule.ToIR (\secref{TypingRule.ToIR})
  \item TypingRule.ToIRCase (\secref{TypingRule.ToIRCase})
  \item TypingRule.ExprEqualNorm (\secref{TypingRule.ExprEqualNorm})
  \item TypingRule.ExprEqual (\secref{TypingRule.ExprEqual})
  \item TypingRule.ExprEqualCase (\secref{TypingRule.ExprEqualCase})
  \item TypingRule.TypeEqual (\secref{TypingRule.TypeEqual})
  \item TypingRule.BitwidthEqual (\secref{TypingRule.BitwidthEqual})
  \item TypingRule.BitFieldsEqual (\secref{TypingRule.BitFieldsEqual})
  \item TypingRule.BitFieldEqual (\secref{TypingRule.BitFieldEqual})
  \item TypingRule.ConstraintsEqual (\secref{TypingRule.ConstraintsEqual})
  \item TypingRule.ConstraintEqual (\secref{TypingRule.ConstraintEqual})
  \item TypingRule.SlicesEqual (\secref{TypingRule.SlicesEqual})
  \item TypingRule.SliceEqual (\secref{TypingRule.SliceEqual})
  \item TypingRule.ArrayLengthEqual (\secref{TypingRule.ArrayLengthEqual})
  \item TypingRule.LiteralEqual (\secref{TypingRule.LiteralEqual})
\end{itemize}

\section{Symbolic Expressions \label{sec:symbolicexpressions}}
Our symbolic reduction and equivalence testing rules use \emph{symbolic expressions}, defined below:
% \newcommand\symexpr[0]{\hyperlink{def-symexpr}{\texttt{sym\_expr}}}
% \hypertarget{def-symexpr}{}
\[
  \hypertarget{def-polynomial}{}\hypertarget{def-sum}{}
  \begin{array}{rcl}
    % \symexpr    &::=& \polynomial                                     \hypertarget{def-polynomial}{}\hypertarget{def-sum}{}\\
    \polynomial &\triangleq& \Sum(\monomial \partialto \Q)                  \hypertarget{def-monomial}{}\hypertarget{def-prod}{}\\
    \monomial   &\triangleq& \Prod(\Identifiers \partialto \Npos)\\
  \end{array}
\]

We now explain each component of a symbolic expression and how it can be interpreted as a mathematical formula
via the interpretation function $\alpha$.
We also define operations over symbolic expressions.

\begin{definition}[Monomial]
\emph{Monomials} are partial functions from variables to positive integer.
%
A non-empty monomial, $\Prod(\vm)\in\monomial$ where $\vm \neq \emptyfunc$, can be interpreted as follows:
\[
  \alpha(\Prod(\vm)) \triangleq \prod_{\vx \in \dom(\vm)} \vx^{\vm(\vx)} \enspace.
\]

An empty monomial is interpreted as the constant $1$:
\[
  \alpha(\Prod(\emptyfunc)) \triangleq 1 \enspace.
\]
\end{definition}
For example,
\[
  \alpha(\ \Prod(\{\vx\mapsto 3, \vy\mapsto 1, \vz\mapsto2\})\ ) = x^3 \cdot y \cdot z^2 \enspace.
\]

\hypertarget{def-mulmonomials}{}
The function
\[
  \mulmonomials(\overname{\monomial}{\vmone} \aslsep \overname{\monomial}{\vmtwo}) \rightarrow \overname{\monomial}{\vm}
\]
multiplies two monomials and returns a monomial
\begin{mathpar}
  \inferrule{
    {
      \vf \eqdef \lambda \vx\in\identifier.\
      \left\{
      \begin{array}{ll}
        \vfone(\vx) & \text{if } \vx \in \dom(\vfone) \setminus \dom(\vftwo)\\
        \vfone(\vx) & \text{if } \vx \in \dom(\vftwo) \setminus \dom(\vfone)\\
        \vfone(\vx)+\vftwo(\vx) & \text{else } \vx \in \dom(\vfone) \cap \dom(\vftwo)\\
      \end{array}
      \right.
    }
  }
  {
    \mulmonomials(\overname{\Prod(\vfone)}{\vmone}, \overname{\Prod(\vftwo)}{\vmtwo}) \typearrow \overname{\Prod(\vf)}{\vm}
  }
\end{mathpar}
For example,
\[
  \begin{array}{ll}
  \mulmonomials( & \Prod(\{\vx\mapsto 3, \vy\mapsto 1, \vz\mapsto2\}), \Prod(\{\vx\mapsto 1, \vw\mapsto 2\})\ ) =\\
                 & \Prod(\{\vx\mapsto 4, \vy\mapsto 1, \vz\mapsto2, \vw\mapsto2\})
  \end{array}
\]

\begin{definition}[Polynomial]
  \emph{Polynomials} are partial functions from monomials to rationals.
  Intuitively, each monomial is mapped to its factor.
  A polynomial $\Sum(\vp)$ can be interpreted as follows:
  %
\[
  \alpha(\Sum(\vp)) \triangleq \sum_{\vm \in \dom(\vp)} \vp(\vm)\cdot\alpha(\vm)
\]
\end{definition}
For example,
\[
  \Sum\left(\left\{
    \begin{array}{lcl}
      \Prod(\{\vx\mapsto 3, \vy\mapsto 1, \vz\mapsto2\}) &\mapsto& -1,\\
      \Prod(\{\vx\mapsto 2, \vy\mapsto 1\}) &\mapsto& \frac{3}{4}
    \end{array} \right\}\right) =
    -1\cdot x^3 \cdot y \cdot z^2 + \frac{3}{4} \cdot \vx^2\cdot \vy \enspace.
\]

The function
\[
  \addpolynomials : \polynomial \times \polynomial \rightarrow \polynomial
\]
adds two polynomials:
\begin{mathpar}
\inferrule{
  {
    \vf \eqdef \lambda \vm\in\monomial.\
    \left\{
    \begin{array}{ll}
      \vfone(\vm) & \text{if } \vm \in \dom(\vfone) \setminus \dom(\vftwo)\\
      \vfone(\vm) & \text{if } \vm \in \dom(\vftwo) \setminus \dom(\vfone)\\
      \vfone(\vm)+\vftwo(\vm) & \text{else } \vm \in \dom(\vfone) \cap \dom(\vftwo)\\
    \end{array}
    \right.
  }
}{
  \addpolynomials(\overname{\Sum(\vfone)}{\vpone}, \overname{\Sum(\vftwo)}{\vptwo}) \typearrow \overname{\Sum(\vf)}{\vp}
}
\end{mathpar}

The overloaded function
\[
  \addpolynomials : \polynomial^* \rightarrow \polynomial
\]
adds a list of polynomials:
\begin{mathpar}
\inferrule[empty]{}{ \addpolynomials(\emptylist) \typearrow \Prod(\emptyfunc) }
\and
\inferrule[one]{}{ \addpolynomials([ \vp ]) \typearrow \vp }
\and
\inferrule[two\_or\_more]{
  \addpolynomials(\vp_{2..k}) \typearrow \vpp\\
  \addpolynomials(\vp_1, \vpp) \typearrow \vp
}{
  \addpolynomials(\vp_{1..k}) \typearrow \vp
}
\end{mathpar}

The function
\[
  \mulpolynomials : \overname{\polynomial}{\vpone} \times \overname{\polynomial}{\vptwo} \rightarrow \overname{\polynomial}{\vp}
\]
multiplies two polynomials.
\begin{mathpar}
\inferrule{
  {
    \vps \eqdef \{ \Sum(\{\mulmonomials(\vmone, \vmtwo) \mapsto \vcone\times\vctwo\})
      \;|\; \vfone(\vmone)=\vcone, \vftwo(\vmtwo)=\vctwo\}
  }\\
  \vps \eqname \{ i=1..k: \vp[i] \}\\
  \addpolynomials(i=1..k: \vp[i]) \typearrow \vp\\
}{
  \mulpolynomials(\overname{\Sum(\vfone)}{\vpone}, \overname{\Sum(\vftwo)}{\vptwo}) \typearrow \vp
}
\end{mathpar}

\section{TypingRule.ToIR \label{sec:TypingRule.ToIR}}
\hypertarget{def-toir}{}
The function
\[
  \toir(\overname{\staticenvs}{\tenv} \aslsep \overname{\expr}{\ve}) \aslto
  \overname{\polynomial}{\vp}\ \cup\ \{\bot\}\ \cup\ \overname{\TTypeError}{\TypeErrorConfig}
\]
transforms a subset of ASL expressions into symbolic expressions. If an ASL expression cannot be represented
by a symbolic expression (because, for example, it contains operations that are not available in symbolic expressions),
the special value $\bot$ is returned.

\subsection{Prose}
Intuitively, $\toir$ first conducts a case analysis to determine whether the ASL expression corresponds to a polynomial.
If that fails, it proceeds to chek whether the expression is a compile time constant.

One of the following applies:
\begin{itemize}
  \item All of the following apply (\textsc{case\_success}):
  \begin{itemize}
    \item the expression $\ve$ can be transformed into a symbolic expression $\vp$,
          that is, $\vp \neq \bot$;
    \item the result is $\vp$;
  \end{itemize}

  \item All of the following apply (\textsc{static\_eval\_literal}):
  \begin{itemize}
    \item the expression $\ve$ cannot be transformed into a symbolic expression;
    \item the expression $\ve$ can be statically evaluated to yield an integer value $\vv$;
    \item $\vp$ is the polynomial representing the value $\vv$.
  \end{itemize}

  \item All of the following apply (\textsc{static\_eval\_non\_literal}):
  \begin{itemize}
    \item the expression $\ve$ cannot be transformed into a symbolic expression;
    \item the expression $\ve$ evaluates to a result that is not an integer value;
    \item $\vp$ is $\bot$.
  \end{itemize}
\end{itemize}

\begin{emptyformal}
\subsection{Formally}
\begin{mathpar}
\inferrule[case\_success]{
  \toircase(\tenv, \ve) \typearrow \vp\\
  \vp \neq \bot
}{
  \toir(\tenv, \ve) \typearrow \vp
}
\and
\inferrule[static\_eval\_literal]{
  \toircase(\tenv, \ve) \typearrow \bot\\
  \staticeval(\tenv, \ve) \typearrow \lint(\vv)\\
  \vp \eqdef \Sum( \{ \Prod(\emptyfunc)\mapsto \vv \} )
}{
  \toir(\tenv, \ve) \typearrow \vp
}
\and
\inferrule[static\_eval\_non\_literal]{
  \toircase(\tenv, \ve) \typearrow \bot\\
  \staticeval(\tenv, \ve) \typearrow \vv\\
  \astlabel(\vv) \neq \lint
}{
  \toir(\tenv, \ve) \typearrow \bot
}
\end{mathpar}
\end{emptyformal}

\section{TypingRule.ToIRCase \label{sec:TypingRule.ToIRCase}}
\hypertarget{def-toircase}{}
The function
\[
  \toircase(\overname{\staticenvs}{\tenv} \aslsep \overname{\expr}{\ve}) \aslto \overname{\polynomial}{\vp} \cup \{\bot\}
\]
transforms a subset of ASL expressions into symbolic expressions. If an expression cannot be represented
by a symbolic expression, the special value $\bot$ is returned.

\subsection{Prose}
Intuitively, $\toir$ first conducts a case analysis to determine whether the ASL expression corresponds to a polynomial.
If that fails, it proceeds to chek whether the expression is a compile time constant.

\newcommand\ProseOrTypeErrorOrBot[0]{or a type error or $\bot$, which short-circuits the entire evaluation}

One of the following applies:
\begin{itemize}
  \item All of the following apply (\textsc{literal\_int}):
  \begin{itemize}
    \item $\ve$ is an integer literal expression for $\vi$, that is, $\ELiteral(\lint(\vi))$;
    \item $\vp$ is the symbolic expression for $\vi$.
  \end{itemize}

  \item All of the following apply (\textsc{literal\_other}):
  \begin{itemize}
    \item $\ve$ is a variable expression other than an integer literal;
    \item $\vp$ is $\bot$.
  \end{itemize}

  \item All of the following apply (\textsc{int\_constant}):
  \begin{itemize}
    \item $\ve$ is a variable expression with identifier $\vs$, that is, $\EVar(\vs)$;
    \item looking up the constant associated with $\vs$ in $\tenv$ yields the literal expression for $\vv$, that is, $\ELiteral(\vv)$;
    \item checking whether $\vv$ is an integer literal yields $\True$ \ProseOrTypeError;
    \item $\vv$ is an integer literal for $\vi$;
    \item $\vp$ is the symbolic expression for $\vi$, that is, $\Sum( \{ \Prod(\emptyfunc)\mapsto \vi \} )$.
  \end{itemize}

  \item All of the following apply (\textsc{int\_exact\_constant}):
  \begin{itemize}
    \item $\ve$ is a variable expression with identifier $\vs$, that is, $\EVar(\vs)$;
    \item looking up the constant associated with $\vs$ in $\tenv$ yields $\bot$;
    \item determining the type of $\vs$ yields $\vt$ \ProseOrTypeError;
    \item the \underlyingtype\ of $\vt$ is $\ttyone$ \ProseOrTypeError;
    \item checking whether $\ttyone$ is an integer type yields $\True$ \ProseOrTypeError;
    \item $\ttyone$ is a well-constrained integer with the exact constraint $\ve$, that is, \\ $\TInt(\wellconstrained([\ConstraintExact(\ve)]))$;
    \item converting $\ve$ to a symbolic expression yields $\vp$ (which may possibly be $\bot$).
  \end{itemize}

  \item All of the following apply (\textsc{int\_var}):
  \begin{itemize}
    \item $\ve$ is a variable expression with identifier $\vs$, that is, $\EVar(\vs)$;
    \item looking up the constant associated with $\vs$ in $\tenv$ yields $\bot$;
    \item determining the type of $\vs$ yields $\vt$ \ProseOrTypeError;
    \item the \underlyingtype\ of $\vt$ is $\ttyone$ \ProseOrTypeError;
    \item checking whether $\ttyone$ is an integer type yields $\True$ \ProseOrTypeError;
    \item $\ttyone$ is not a well-constrained integer with a single exact constraint;
    \item $\vp$ is the symbolic expression for the variable $\vs$, that is, $\Sum( \{ \Prod(\{\vs\mapsto 1\})\mapsto 1 \} )$.
  \end{itemize}

  \item All of the following apply (\textsc{ebinop\_plus}):
  \begin{itemize}
    \item $\ve$ is a binary addition expression with operands $\veone$ and $\vetwo$, that is, \\ $\EBinop(\PLUS, \veone, \vetwo)$;
    \item converting $\veone$ to a symbolic expression in $\tenv$ yields $\irone$ \ProseOrTypeErrorOrBot;
    \item converting $\vetwo$ to a symbolic expression in $\tenv$ yields $\irtwo$ \ProseOrTypeErrorOrBot;
    \item $\vp$ is the symbolic expression adding up $\irone$ and $\irtwo$.
  \end{itemize}

  \item All of the following apply (\textsc{ebinop\_minus}):
  \begin{itemize}
    \item $\ve$ is a binary Substraction expression with operands $\veone$ and $\vetwo$, that is, \\ $\EBinop(\MINUS, \veone, \vetwo)$;
    \item $\vep$ is the addition expression with operands $\veone$ and the negation of $\vetwo$, that is, \\ $\EBinop(\PLUS, \veone, \EBinop(\MINUS, \vetwo))$;
    \item converting $\vpp$ into a symbolic expression in $\tenv$ yields $\vp$ \ProseOrTypeErrorOrBot.
  \end{itemize}

  \item All of the following apply (\textsc{ebinop\_mul}):
  \begin{itemize}
    \item $\ve$ is a binary multiplication expression with operands $\veone$ and $\vetwo$, that is, \\ $\EBinop(\MUL, \veone, \vetwo)$;
    \item converting $\veone$ to a symbolic expression in $\tenv$ yields $\irone$ \ProseOrTypeErrorOrBot;
    \item converting $\vetwo$ to a symbolic expression in $\tenv$ yields $\irtwo$ \ProseOrTypeErrorOrBot;
    \item $\vp$ is the symbolic expression multiplying $\irone$ and $\irtwo$.
  \end{itemize}

  \item All of the following apply (\textsc{ebinop\_div\_non\_int\_denominator}):
  \begin{itemize}
    \item $\ve$ is a binary division expression with operands $\veone$ and $\vetwo$, that is, \\ $\EBinop(\DIV, \veone, \vetwo)$;
    \item $\vetwo$ is not an integer literal expression;
    \item $\vp$ is $\bot$.
  \end{itemize}

  \item All of the following apply (\textsc{ebinop\_div\_int\_denominator}):
  \begin{itemize}
    \item $\ve$ is a binary division expression with operands $\veone$ and $\vetwo$, that is, \\ $\EBinop(\DIV, \veone, \vetwo)$;
    \item $\vetwo$ is an integer literal expression for $\vitwo$;
    \item converting $\veone$ to a symbolic expression in $\tenv$ yields $\irone$ \ProseOrTypeErrorOrBot;
    \item $\vftwo$ is $\frac{1}{\vitwo}$;
    \item $\vp$ is the polynomial $\irone$ with each monomial multiplied by $\vftwo$.
  \end{itemize}

  \item All of the following apply (\textsc{ebinop\_shl\_non\_lint\_exponent}):
  \begin{itemize}
    \item $\ve$ is a binary shift-left expression with operands $\veone$ and $\vetwo$, that is, \\ $\EBinop(\SHL, \veone, \vetwo)$;
    \item $\vetwo$ is not an integer literal expression;
    \item $\vp$ is $\bot$.
  \end{itemize}

  \item All of the following apply (\textsc{ebinop\_shl\_non\_neg\_shift}):
  \begin{itemize}
    \item $\ve$ is a binary shift-left expression with operands $\veone$ and $\vetwo$, that is, \\ $\EBinop(\SHL, \veone, \vetwo)$;
    \item $\vetwo$ is an integer literal expression for $\vitwo$;
    \item $\vitwo$ is negative;
    \item $\vp$ is $\bot$.
  \end{itemize}

  \item All of the following apply (\textsc{ebinop\_shl\_okay}):
  \begin{itemize}
    \item $\ve$ is a binary shift-left expression with operands $\veone$ and $\vetwo$, that is, \\ $\EBinop(\SHL, \veone, \vetwo)$;
    \item $\vetwo$ is an integer literal expression for $\vitwo$;
    \item converting $\veone$ to a symbolic expression in $\tenv$ yields $\irone$ \ProseOrTypeErrorOrBot;
    \item $\vitwo$ is non-negative;
    \item $\vftwo$ is $2^{\vitwo}$;
    \item $\vp$ is the polynomial $\irone$ with each monomial multiplied by $\vftwo$.
  \end{itemize}

  \item All of the following apply (\textsc{ebinop\_other\_non\_literals}):
  \begin{itemize}
    \item $\ve$ is a binary expression with an operator $\op$ that is other than $\PLUS$, $\MINUS$, $\MUL$, or $\SHL$,
          applied to the operand expressions $\veone$ and $\vetwo$;
    \item at least one of $\veone$ and $\vetwo$ is not a literal expression;
    \item $\vp$ is $\bot$.
  \end{itemize}

  \item All of the following apply (\textsc{ebinop\_other\_literals\_non\_int\_result}):
  \begin{itemize}
    \item $\ve$ is a binary expression with an operator $\op$ that is other than $\PLUS$, $\MINUS$, $\MUL$, $\DIV$, or $\SHL$,
          applied to the operand expressions $\veone$ and $\vetwo$;
    \item $\veone$ is a literal expression for literal $\vlone$;
    \item $\vetwo$ is a literal expression for literal $\vltwo$;
    \item statically applying $\op$ to $\vlone$ and $\vltwo$ yields the literal $\vl$, which is not an integer literal;
    \item $\vp$ is $\bot$.
  \end{itemize}

  \item All of the following apply (\textsc{ebinop\_other\_literals\_int\_result}):
  \begin{itemize}
    \item $\ve$ is a binary expression with an operator $\op$ that is other than $\PLUS$, $\MINUS$, $\MUL$, or $\SHL$,
          applied to the operand expressions $\veone$ and $\vetwo$;
    \item $\veone$ is a literal expression for literal $\vlone$;
    \item $\vetwo$ is a literal expression for literal $\vltwo$;
    \item statically applying $\op$ to $\vlone$ and $\vltwo$ yields the integer literal for $k$;
    \item $\vp$ is the symbolic expression for the integer $k$, that is, $\Sum( \{ \Prod(\emptyfunc)\mapsto k \} )$.
  \end{itemize}

  \item All of the following apply (\textsc{eunop\_neg}):
  \begin{itemize}
    \item $\ve$ is a unary expression with the negation operator $\NEG$ and operand $\veone$;
    \item converting the binary expression with operator $\MUL$ and left-hand-side operand for the integer literal $-1$ and
    right-hand-side operand $\veone$ in $\tenv$ yields $\vp$ \ProseOrTypeErrorOrBot.
  \end{itemize}

  \item All of the following apply (\textsc{eunop\_other}):
  \begin{itemize}
    \item $\ve$ is a unary expression with an operator other than $\NEG$;
    \item $\vp$ is bottom
  \end{itemize}

  \item All of the following apply (\textsc{other}):
  \begin{itemize}
    \item $\ve$ is an expression with a label other than $\ELiteral$, $\EVar$, $\EBinop$, and $\EUnop$;
    \item $\vp$ is bottom
  \end{itemize}
\end{itemize}

\begin{emptyformal}
\subsection{Formally}
\begin{mathpar}
\inferrule[literal\_int]{}
{
  \toircase(\tenv, \overname{\ELiteral(\lint(\vi))}{\ve}) \typearrow \overname{\Sum( \{ \Prod(\emptyfunc)\mapsto \vi \} )}{\vp}
}
\and
\inferrule[literal\_other]{
  \astlabel(\vv) \neq \lint
}{
  \toircase(\tenv, \overname{\ELiteral(\vv)}{\ve}) \typearrow \bot
}
\and
\inferrule[int\_constant]{
  \lookupconstant(\tenv, \vs) \typearrow \ELiteral(\vv)\\
  \checktrans{\astlabel(\vv) = \lint}{ExpectedIntegerLiteral} \typearrow \True \OrTypeError\\
  \vv \eqname \lint(\vi)
}{
  \toircase(\tenv, \overname{\EVar(\vs)}{\ve}) \typearrow \overname{\Sum( \{ \Prod(\emptyfunc)\mapsto \vi \} )}{\vp}
}
\and
\inferrule[int\_exact\_constraint]{
  \lookupconstant(\tenv, \vs) \typearrow \bot\\
  \typeof(\vs) \typearrow \vt \OrTypeError\\\\
  \makeanonymous(\vt) \typearrow \ttyone \OrTypeError\\\\
  \checktrans{\astlabel(\ttyone) = \TInt}{ExpectedIntegerType} \typearrow \True \OrTypeError\\\\
  \ttyone = \TInt(\wellconstrained([\ConstraintExact(\ve)]))\\
  \toir(\ve) \typearrow \vp
}{
  \toircase(\tenv, \overname{\EVar(\vs)}{\ve}) \typearrow \vp
}
\and
\inferrule[int\_var]{
  \lookupconstant(\tenv, \vs) \typearrow \bot\\
  \typeof(\vs) \typearrow \vt\\
  \makeanonymous(\vt) \typearrow \ttyone\\
  \checktrans{\astlabel(\ttyone) = \TInt}{ExpectedIntegerType} \typearrow \True\\
  \ttyone \neq \TInt(\wellconstrained([\ConstraintExact(\Ignore)]))
}{
  \toircase(\tenv, \overname{\EVar(\vs)}{\ve}) \typearrow \overname{\Sum( \{ \Prod(\{\vs\mapsto 1\})\mapsto 1 \} )}{\vp}
}
\end{mathpar}

\begin{mathpar}
\inferrule[ebinop\_plus]{
  \toir(\tenv, \veone) \typearrow \irone \OrTypeError, \bot\\\\
  \toir(\tenv, \vetwo) \typearrow \irtwo \OrTypeError, \bot\\\\
  \vp \eqdef \addpolynomials(\irone, \irtwo)
}{
  \toircase(\tenv, \overname{\EBinop(\PLUS, \veone, \vetwo)}{\ve}) \typearrow \vp
}
\end{mathpar}

\begin{mathpar}
\inferrule[ebinop\_minus]{
  \vep \eqdef \EBinop(\PLUS, \veone, \EBinop(\MINUS, \vetwo))\\
  \toir(\tenv, \vpp) \typearrow \vp \OrTypeError, \bot\\\\
}{
  \toircase(\tenv, \overname{\EBinop(\MINUS, \veone, \vetwo)}{\ve}) \typearrow \vp
}
\end{mathpar}

\begin{mathpar}
\inferrule[ebinop\_mul]{
  \toir(\tenv, \veone) \typearrow \irone \OrTypeError, \bot\\\\
  \toir(\tenv, \vetwo) \typearrow \irtwo \OrTypeError, \bot\\\\
  \vp \eqdef \mulpolynomials(\irone, \irtwo)
}{
  \toircase(\tenv, \overname{\EBinop(\MUL, \veone, \vetwo)}{\ve}) \typearrow \vp
}
\end{mathpar}

\begin{mathpar}
\inferrule[ebinop\_div\_non\_int\_denominator]{
  \vetwo \neq \ELiteral(\lint(\Ignore))
}{
  \toircase(\tenv, \overname{\EBinop(\DIV, \veone, \vetwo)}{\ve}) \typearrow \bot
}
\and
\inferrule[ebinop\_div\_int\_denominator]{
  \toir(\tenv, \veone) \typearrow \irone \OrTypeError, \bot\\\\
  \vftwo \eqdef \frac{1}{\vitwo}\\
  \irone \eqname \Sum [i=1..k: \vm_\vi \mapsto \vc_\vi]\\
  \vp \eqdef \Sum [i=1..k: \vm_\vi \mapsto \vc_\vi \times \vftwo]\\
}{
  \toircase(\tenv, \overname{\EBinop(\DIV, \veone, \overname{\ELiteral(\lint(\vitwo))}{\vetwo})}{\ve}) \typearrow \vp
}
\end{mathpar}

\begin{mathpar}
\inferrule[ebinop\_shl\_non\_lint\_exponent]{
    \vetwo \neq \ELiteral(\lint(\Ignore))
}{
  \toircase(\tenv, \overname{\EBinop(\SHL, \Ignore, \vetwo)}{\ve}) \typearrow \bot
}
\end{mathpar}

\begin{mathpar}
\inferrule[ebinop\_shl\_neg\_shift]{
  \vitwo < 0
}{
  \toircase(\tenv, \overname{\EBinop(\SHL, \veone, \ELiteral(\lint(\vitwo)))}{\ve}) \typearrow \bot
}
\and
  \inferrule[ebinop\_shl\_okay]{
    \toir(\tenv, \veone) \typearrow \irone \OrTypeError, \bot\\\\
    \vitwo \geq 0\\
    \vftwo \eqdef 2^{\vitwo}\\
    \irone \eqname \Sum [i=1..k: \vm_\vi \mapsto \vc_\vi]\\
    \vp \eqdef \Sum [i=1..k: \vm_\vi \mapsto \vc_\vi \times \vftwo]\\
}{
  \toircase(\tenv, \overname{\EBinop(\SHL, \veone, \ELiteral(\lint(\vitwo)))}{\ve}) \typearrow \vp
}
\end{mathpar}

\begin{mathpar}
\inferrule[ebinop\_other\_non\_literals]{
  \op \not\in \{\PLUS, \MINUS, \MUL, \DIV, \SHL\}\\
  (\veone \neq \ELiteral(\Ignore) \lor \vetwo \neq \ELiteral(\Ignore))
}{
  \toircase(\tenv, \overname{\EBinop(\op, \veone, \vetwo)}{\ve}) \typearrow \bot
}
\and
\inferrule[ebinop\_other\_literals\_non\_int\_result]{
  \op \not\in \{\PLUS, \MINUS, \MUL, \SHL\}\\
  \staticbinop(\op, \vlone, \vltwo) \typearrow \vl\\
  \vl \neq \lint(\Ignore)
}{
  \toircase(\tenv, \overname{\EBinop(\op, \ELiteral(\vlone), \ELiteral(\vltwo))}{\ve}) \typearrow \bot
}
\and
\inferrule[ebinop\_other\_literals\_int\_result]{
  \op \not\in \{\PLUS, \MINUS, \MUL, \SHL\}\\
  \staticbinop(\op, \vlone, \vltwo) \typearrow \lint(k)\\
  \vp \eqdef \Sum( \{ \Prod(\emptyfunc)\mapsto k \} )
}{
  \toircase(\tenv, \overname{\EBinop(\op, \ELiteral(\vlone), \ELiteral(\vltwo))}{\ve}) \typearrow \vp
}
\end{mathpar}

\begin{mathpar}
\inferrule[eunop\_neg]{
  \toir(\tenv, \EBinop(\MUL, \ELiteral(\lint(-1)),\veone )) \typearrow \vp \OrTypeError, \bot\\\\
}{
  \toircase(\tenv, \overname{\EUnop(\NEG, \veone)}{\ve}) \typearrow \vp
}
\and
\inferrule[eunop\_other]{
  \op \neq \NEG
}{
  \toircase(\tenv, \overname{\EUnop(\op, \Ignore)}{\ve}) \typearrow \bot
}
\end{mathpar}

\begin{mathpar}
\inferrule[other]{
  \astlabel(\ve) \not\in \{\ELiteral, \EVar, \EBinop, \EUnop\}
}{
  \toircase(\tenv, \ve) \typearrow \bot
}
\end{mathpar}
\end{emptyformal}

\section{TypingRule.ExprEqualNorm \label{sec:TypingRule.ExprEqualNorm}}
\hypertarget{def-exprequalnorm}{}
The function
\[
  \exprequalnorm(\overname{\staticenvs}{\tenv} \aslsep \overname{\expr}{\veone} \aslsep \overname{\expr}{\vetwo})
  \aslto \overname{\{\True, \False\}}{\vb} \cup \overname{\TTypeError}{\TypeErrorConfig}
\]
conservatively tests whether the expression $\veone$ is equivalent to the expression $\vetwo$ in environment $\tenv$
by attempting to transform both expressions to their symbolic expression form
and, if successful, comparing the resulting normal forms for equality.
The result is given in $\vb$ or a type error, if one is detected.

\subsection{Prose}
One of the following applies:
\begin{itemize}
  \item All of the following apply (\textsc{all\_supported}):
  \begin{itemize}
    \item transforming $\veone$ into a symbolic expression in $\tenv$ yields $\irone$ \ProseOrTypeError;
    \item transforming $\vetwo$ into a symbolic expression in $\tenv$ yields $\irtwo$ \ProseOrTypeError;
    \item $\vb$ is the result of equating $\irone$ and $\irtwo$.
  \end{itemize}

  \item All of the following apply (\textsc{unsupported1}):
  \begin{itemize}
    \item transforming $\veone$ into a symbolic expression in $\tenv$ yields $\bot$;
    \item $\vb$ is $\False$;
  \end{itemize}

  \item All of the following apply (\textsc{unsupported2}):
  \begin{itemize}
    \item transforming $\veone$ into a symbolic expression in $\tenv$ yields $\irone$;
    \item transforming $\vetwo$ into a symbolic expression in $\tenv$ yields $\bot$;
    \item $\vb$ is $\False$;
  \end{itemize}
\end{itemize}

\begin{emptyformal}
\subsection{Formally}
\begin{mathpar}
\inferrule[all\_supported]{
  \toir(\veone) \typearrow \irone \OrTypeError\\\\
  \toir(\vetwo) \typearrow \irtwo \OrTypeError\\\\
  \vb \eqdef \irone = \irtwo
}{
  \exprequalnorm(\tenv, \veone, \vetwo) \typearrow \vb
}
\and
\inferrule[unsupported1]{
  \toir(\veone) \typearrow \bot
}{
  \exprequalnorm(\tenv, \veone, \vetwo) \typearrow \False
}
\and
\inferrule[unsupported2]{
  \toir(\veone) \typearrow \irone\\
  \toir(\vetwo) \typearrow \bot
}{
  \exprequalnorm(\tenv, \veone, \vetwo) \typearrow \False
}
\end{mathpar}
\end{emptyformal}

\section{TypingRule.ExprEqual \label{sec:TypingRule.ExprEqual}}
\hypertarget{def-exprequal}{}
The function
\[
  \exprequal(\overname{\staticenvs}{\tenv} \aslsep \overname{\expr}{\veone} \aslsep \overname{\expr}{\vetwo}) \aslto
  \overname{\{\True, \False\}}{\vb} \cup \overname{\TTypeError}{\TypeErrorConfig}
\]
conservatively checks whether the expression $\veone$ is equivalent to the expression $\vetwo$ in environment $\tenv$.
The result is given in $\vb$ or a type error, if one is detected.

\subsection{Prose}
One of the following applies:
\begin{itemize}
  \item All of the following apply (\textsc{norm\_true}):
  \begin{itemize}
    \item comparing $\veone$ to $\vetwo$ in $\tenv$ via $\exprequalnorm$ yields $\True$ \ProseOrTypeError;
    \item $\vb$ is $\True$.
  \end{itemize}

  \item All of the following apply (\textsc{norm\_false}):
  \begin{itemize}
    \item comparing $\veone$ to $\vetwo$ in $\tenv$ via $\exprequalnorm$ yields $\False$;
    \item comparing $\veone$ to $\vetwo$ by case analysis via $\exprequalcase$ yields $\vb$ \ProseOrTypeError.
  \end{itemize}
\end{itemize}

\begin{emptyformal}
\subsection{Formally}
\begin{mathpar}
\inferrule[norm\_true]{
  \exprequalnorm(\tenv, \veone, \vetwo) \typearrow \True \OrTypeError
}{
  \exprequal(\tenv, \veone, \vetwo) \typearrow \True
}
\and
\inferrule[norm\_false]{
  \exprequalnorm(\tenv, \veone, \vetwo) \typearrow \False\\
  \exprequalcase(\tenv, \veone, \vetwo) \typearrow \vb \OrTypeError
}{
  \exprequal(\tenv, \veone, \vetwo) \typearrow \vb
}
\end{mathpar}
\end{emptyformal}

\section{TypingRule.ExprEqualCase \label{sec:TypingRule.ExprEqualCase}}
\hypertarget{def-exprequalcase}{}
The function
\[
  \exprequalcase(\overname{\staticenvs}{\tenv} \aslsep \overname{\expr}{\veone} \aslsep \overname{\expr}{\vetwo})
  \aslto \overname{\{\True, \False\}}{\vb} \cup \overname{\TTypeError}{\TypeErrorConfig}
\]
specializes the equivalence test for expressions $\veone$ and $\vetwo$ in $\tenv$
for the different types of expressions.
The result is given in $\vb$ or a type error, if one is detected.

\subsection{Prose}
One of the following applies:
\begin{itemize}
  \item All of the following apply (\textsc{different\_labels}):
  \begin{itemize}
    \item the AST labels of $\veone$ and $\vetwo$ are different;
    \item $\vb$ is $\False$.
  \end{itemize}

  \item All of the following apply (\textsc{e\_binop}):
  \begin{itemize}
    \item $\veone$ is a binary expression with operator $\opone$ and operands $\veoneone$ and $\veonetwo$,
          that is, $\EBinop(\opone, \veoneone, \veonetwo)$;
    \item $\vetwo$ is a binary expression with operator $\optwo$ and operands $\vetwoone$ and $\vetwotwo$,
          that is, $\EBinop(\optwo, \vetwoone, \vetwotwo)$;
    \item testing the equivalence of $\veoneone$ and $\vetwoone$ in $\tenv$ yields $\vbone$ \ProseOrTypeError;
    \item testing the equivalence of $\veonetwo$ and $\vetwotwo$ in $\tenv$ yields $\vbtwo$ \ProseOrTypeError;
    \item $\vb$ is $\True$ if and only if $\opone$ is equal to $\optwo$ and both $\vbone$ and $\vbtwo$ are $\True$.
  \end{itemize}

  \item All of the following apply (\textsc{e\_call}):
  \begin{itemize}
    \item $\veone$ is a call expression with subprogram name $\nameone$ and list of arguments $vargsone$,
          that is, $\ECall(\nameone, \vargsone, \Ignore)$;
    \item $\vetwo$ is a call expression with subprogram name $\nametwo$ and list of arguments $vargstwo$,
          that is, $\ECall(\nametwo, \vargstwo, \Ignore)$;
    \item checking whether $\nameone$ is equal to $\nametwo$ either yields $\True$ or $\False$, which short-circuits the entire evaluation;
    \item checking whether the lists of arguments $\vargsone$ and $\vargstwo$ have the same lengths yields
          $\True$ or $\False$, which short-circuits the entire evaluation;
    \item for each index $i$ in the list of indices for $\vargsone$, testing whether $\vargsone[i]$ is equivalent to $\vargstwo[i]$
          in $\tenv$ yields $\vb_i$ \ProseOrTypeError;
    \item $\vb$ is $\True$ if and only if $\vb_i$ is $\True$ for each index $i$ in the list of indices for $\vargsone$.
  \end{itemize}

  \item All of the following apply (\textsc{e\_concat}):
  \begin{itemize}
    \item $\veone$ is a concatenation expression with $\vlone$, that is, $\EConcat(\vlone)$;
    \item $\vetwo$ is a concatenation expression with $\vltwo$, that is, $\EConcat(\vltwo)$;
    \item checking whether the lists of expressions $\vlone$ and $\vltwo$ have the same lengths yields
          $\True$ or $\False$, which short-circuits the entire evaluation;
    \item for each index $i$ in the list of indices for $\vlone$, testing whether $\vlone[i]$ is equivalent to $\vltwo[i]$
          in $\tenv$ yields $\vb_i$ \ProseOrTypeError;
    \item $\vb$ is $\True$ if and only if $\vb_i$ is $\True$ for each index $i$ in the list of indices for $\vlone$.
  \end{itemize}

  \item All of the following apply (\textsc{e\_cond}):
  \begin{itemize}
    \item $\veone$ is a conditional expression with expressions $\veoneone$, $\veonetwo$, and $\veonethree$,
          that is, $\ECond(\veoneone, \veonetwo, \veonethree)$;
    \item $\vetwo$ is a conditional expression with expressions $\vetwoone$, $\vetwotwo$, and $\vetwothree$,
          that is, $\ECond(\vetwoone, \vetwotwo, \vetwothree)$;
    \item testing whether $\veoneone$ is equivalent to $\vetwoone$ yields $\vbone$ \ProseOrTypeError;
    \item testing whether $\veonetwo$ is equivalent to $\vetwotwo$ yields $\vbtwo$ \ProseOrTypeError;
    \item testing whether $\veonethree$ is equivalent to $\vetwothree$ yields $\vbthree$ \ProseOrTypeError;
    \item $\vb$ is $\True$ if and only if all of $\vbone$, $\vbtwo$, and $\vbthree$ are $\True$.
  \end{itemize}

  \item All of the following apply (\textsc{e\_slice}):
  \begin{itemize}
    \item $\veone$ is a slicing expression with expression $\veoneone$ and list of slices $\slicesone$,
          that is, $\ESlice(\veoneone, \slicesone)$;
    \item $\veone$ is a slicing expression with expression $\vetwoone$ and list of slices $\slicestwo$,
          that is, $\ESlice(\vetwoone, \slicestwo)$;
    \item testing whether $\veoneone$ is equivalent to $\vetwoone$ yields $\vbone$ \ProseOrTypeError;
    \item testing whether the lists of slices $\slicesone$ and $\slicestwo$ are equivalent in $\tenv$ yields $\vbtwo$ \ProseOrTypeError;
    \item $\vb$ is $\True$ if and only both $\vbone$ and $\vbtwo$ are $\True$.
  \end{itemize}

  \item All of the following apply (\textsc{e\_getarray}):
  \begin{itemize}
    \item $\veone$ is an array access expression with array expression $\veoneone$ and position expression $\veonetwo$,
          that is, $\EGetArray(\veoneone, \veonetwo)$;
    \item $\vetwo$ is an array access expression with array expression $\vetwoone$ and position expression $\vetwotwo$,
          that is, $\EGetArray(\vetwoone, \vetwotwo)$;
    \item testing whether $\veoneone$ is equivalent to $\vetwoone$ yields $\vbone$ \ProseOrTypeError;
    \item testing whether $\veonetwo$ is equivalent to $\vetwotwo$ yields $\vbtwo$ \ProseOrTypeError;
    \item $\vb$ is $\True$ if and only both $\vbone$ and $\vbtwo$ are $\True$.
  \end{itemize}

  \item All of the following apply (\textsc{e\_getfield}):
  \begin{itemize}
    \item $\veone$ is a field access expression with sub-expression $\veoneone$ and field name $\vfieldone$,
          that is, $\EGetField(\veoneone, \vfieldone)$;
    \item $\vetwo$ is a field access expression with sub-expression $\vetwoone$ and field name $\vfieldtwo$,
          that is, $\EGetField(\vetwoone, \vfieldtwo)$;
    \item $\vbone$ is $\True$ if and only if $\vfieldone$ is equal to $\vfieldtwo$;
    \item testing whether $\veoneone$ is equivalent to $\vetwoone$ yields $\vbtwo$ \ProseOrTypeError;
    \item $\vb$ is $\True$ if and only both $\vbone$ and $\vbtwo$ are $\True$.
  \end{itemize}

  \item All of the following apply (\textsc{e\_getfields}):
  \begin{itemize}
    \item $\veone$ is a fields access expression with sub-expression $\veoneone$ and list of field names $\vfieldsone$,
          that is, $\EGetFields(\veoneone, \vfieldsone)$;
    \item $\vetwo$ is a fields access expression with sub-expression $\vetwoone$ and list of field names $\vfieldstwo$,
          that is, $\EGetFields(\vetwoone, \vfieldstwo)$;
    \item $\vbone$ is $\True$ if and only if $\vfieldsone$ is equal to $\vfieldstwo$;
    \item testing whether $\veoneone$ is equivalent to $\vetwoone$ yields $\vbtwo$ \ProseOrTypeError;
    \item $\vb$ is $\True$ if and only both $\vbone$ and $\vbtwo$ are $\True$.
  \end{itemize}

  \item All of the following apply (\textsc{e\_getitem}):
  \begin{itemize}
    \item $\veone$ is a tuple access expression with sub-expression $\veoneone$ and position $\vione$,
          that is, $\EGetItem(\veoneone, \vione)$;
    \item $\vetwo$ is a tuple access expression with sub-expression $\vetwoone$ and position $\vitwo$,
          that is, $\EGetItem(\vetwoone, \vitwo)$;
    \item $\vbone$ is $\True$ if and only if $\vione$ is equal to $\vitwo$;
    \item testing whether $\veoneone$ is equivalent to $\vetwoone$ yields $\vbtwo$ \ProseOrTypeError;
    \item $\vb$ is $\True$ if and only both $\vbone$ and $\vbtwo$ are $\True$.
  \end{itemize}

  \item All of the following apply (\textsc{e\_literal}):
  \begin{itemize}
    \item $\veone$ is a literal expression with literal $\vvone$;
    \item $\vetwo$ is a literal expression with literal $\vvtwo$;
    \item $\vb$ is $\True$ if and only if $\vvone$ is equivalent to $\vvtwo$ in $\tenv$.
  \end{itemize}

  \item All of the following apply (\textsc{e\_pattern}):
  \begin{itemize}
    \item both $\veone$ and $\vetwo$ are pattern expressions;
    \item $\vb$ is $\False$.
  \end{itemize}

  \item All of the following apply (\textsc{e\_record}):
  \begin{itemize}
    \item both $\veone$ and $\vetwo$ are record expressions;
    \item $\vb$ is $\False$.
  \end{itemize}

  \item All of the following apply (\textsc{e\_tuple}):
  \begin{itemize}
    \item $\veone$ is a tuple expression with sub-expressions list $\vlone$,
          that is, $\ETuple(\vlone)$;
    \item $\vetwo$ is a tuple expression with sub-expressions list $\vltwo$,
          that is, $\ETuple(\vltwo)$;
    \item checking whether the lengths of $\vlone$ and $\vltwo$ are equal yields either $\True$ or $\False$, which short-circuits
          the entire evaluation;
    \item for each index $i$ in the list of indices for $\vlone$, testing whether $\vlone[i]$ is equivalent to $\vltwo[i]$
          in $\tenv$ yields $\vb_i$ \ProseOrTypeError;
    \item $\vb$ is $\True$ if and only if $\vb_i$ is $\True$ for each index $i$ in the list of indices for $\vlone$.
  \end{itemize}

  \item All of the following apply (\textsc{e\_unop}):
  \begin{itemize}
    \item $\veone$ is a unary operator expression with operator $\opone$ and operand expressions $\veoneone$,
          that is, $\EUnop(\opone, \veoneone)$;
    \item $\vetwo$ is a unary operator expression with operator $\optwo$ and operand expressions $\vetwoone$,
          that is, $\EUnop(\optwo, \vetwoone)$;
    \item testing whether $\veoneone$ is equivalent to $\vetwoone$ in $\tenv$ yields $\vbone$;
    \item $\vb$ is $\True$ if and only if $\opone$ is equal to $\optwo$ and $\vbone$ is $\True$.
  \end{itemize}

  \item All of the following apply (\textsc{e\_unknown}):
  \begin{itemize}
    \item both $\veone$ and $\vetwo$ are $\UNKNOWN$ expressions;
    \item $\vb$ is $\False$.
  \end{itemize}

  \item All of the following apply (\textsc{e\_atc}):
  \begin{itemize}
    \item $\veone$ is a type assertion with sub-expression with operator $\veoneone$ and type $\vtone$,
          that is, $\EATC(\veoneone, \vtone)$;
    \item $\vetwo$ is a type assertion with sub-expression with operator $\vetwoone$ and type $\vttwo$,
          that is, $\EATC(\vetwoone, \vttwo)$;
    \item testing whether $\veoneone$ is equivalent to $\vetwoone$ in $\tenv$ yields $\vbone$;
    \item testing whether $\vtone$ is equivalent to $\vttwo$ in $\tenv$ yields $\vbtwo$;
    \item $\vb$ is $\True$ if and only if both $\vbone$ and $\vbtwo$ are $\True$.
  \end{itemize}

  \item All of the following apply (\textsc{e\_var}):
  \begin{itemize}
    \item $\veone$ is a variable expression with identifier $\nameone$, that is, $\EVar(\nameone)$;
    \item $\vetwo$ is a variable expression with identifier $\nametwo$, that is, $\EVar(\nametwo)$;
    \item $\vb$ is $\True$ if and only if both $\nameone$ is equal to $\nametwo$.
  \end{itemize}
\end{itemize}

\begin{emptyformal}
\subsection{Formally}
\begin{mathpar}
\inferrule[different\_labels]{
  \astlabel(\veone) \neq \astlabel(\vetwo)
}{
  \exprequalcase(\tenv, \veone, \vetwo) \typearrow \False
}
\end{mathpar}

\begin{mathpar}
\inferrule[e\_binop]{
  \veone \eqname \EBinop(\opone, \veoneone, \veonetwo)\\
  \vetwo \eqname \EBinop(\optwo, \vetwoone, \vetwotwo)\\
  \exprequal(\veoneone, \vetwoone) \typearrow \vbone \OrTypeError\\\\
  \exprequal(\veonetwo, \vetwotwo) \typearrow \vbtwo \OrTypeError\\\\
  \vb \eqdef (\opone = \optwo) \land \vbone \land \vbtwo
}{
  \exprequalcase(\tenv, \veone, \vetwo) \typearrow \vb
}
\end{mathpar}

(Recall that a conjunction over an empty set equals $\True$.)
\begin{mathpar}
\inferrule[e\_call]{
  \veone \eqname \ECall(\nameone, \vargsone, \Ignore)\\
  \vetwo \eqname \ECall(\nametwo, \vargstwo, \Ignore)\\\\
  \booltrans{\nameone = \nametwo} \booltransarrow \True \terminateas \False\\\\
  \equallength(\vargsone, \vargstwo) \typearrow \True \terminateas \False\\\\
  i \in \listrange(\vargsone): \exprequal(\tenv, \vargsone[i], \vargstwo[i]) \typearrow \vb_i \OrTypeError\\\\
  \vb \eqdef \bigwedge_{i \in \listrange(\vargsone)} \vb_i
}{
  \exprequalcase(\tenv, \veone, \vetwo) \typearrow \vb
}
\end{mathpar}

\begin{mathpar}
  \inferrule[e\_concat]{
  \veone \eqname \EConcat(\vlone)\\
  \vetwo \eqname \EConcat(\vltwo)\\\\
  \equallength(\vlone, \vltwo) \typearrow \True \terminateas \False\\\\
  i \in \listrange(\vlone): \exprequal(\tenv, \vlone[i], \vltwo[i]) \typearrow \vb_i \OrTypeError\\\\
  \vb \eqdef \bigwedge_{i \in \listrange(\vlone)} \vb_i
}{
  \exprequalcase(\tenv, \veone, \vetwo) \typearrow \vb
}
\end{mathpar}

\begin{mathpar}
\inferrule[e\_cond]{
  \veone \eqname \ECond(\veoneone, \veonetwo, \veonethree)\\
  \vetwo \eqname \ECond(\vetwoone, \vetwotwo, \vetwothree)\\\\
  \exprequal(\tenv, \veoneone, \vetwoone) \typearrow \vbone \OrTypeError\\\\
  \exprequal(\tenv, \veonetwo, \vetwotwo) \typearrow \vbtwo \OrTypeError\\\\
  \exprequal(\tenv, \veonethree, \vetwothree) \typearrow \vbthree \OrTypeError\\\\
  \vb \eqdef \vbone \land \vbtwo \land \vbthree
}{
  \exprequalcase(\tenv, \veone, \vetwo) \typearrow \True
}
\end{mathpar}

\begin{mathpar}
  \inferrule[e\_slice]{
  \veone \eqname \ESlice(\veoneone, \slicesone)\\
  \vetwo \eqname \ESlice(\vetwoone, \slicestwo)\\\\
  \exprequal(\tenv, \veoneone, \vetwoone) \typearrow \vbone \OrTypeError\\\\
  \slicesequal(\tenv, \slicesone, \slicestwo) \typearrow \vbtwo \OrTypeError\\\\
  \vb \eqdef \vbone \land \vbtwo
}{
  \exprequalcase(\tenv, \veone, \vetwo) \typearrow \vb
}
\end{mathpar}

\begin{mathpar}
  \inferrule[e\_getarray]{
  \veone \eqname \EGetArray(\veoneone, \veonetwo)\\
  \vetwo \eqname \EGetArray(\vetwoone, \vetwotwo)\\\\
  \exprequal(\tenv, \veoneone, \vetwoone) \typearrow \vbone \OrTypeError\\\\
  \exprequal(\tenv, \veonetwo, \vetwotwo) \typearrow \vbtwo \OrTypeError\\\\
  \vb \eqdef \vbone \land \vbtwo
}{
  \exprequalcase(\tenv, \veone, \vetwo) \typearrow \vb
}
\end{mathpar}

\begin{mathpar}
  \inferrule[e\_getfield]{
  \veone \eqname \EGetField(\veoneone, \vfieldone)\\
  \vetwo \eqname \EGetField(\vetwoone, \vfieldtwo)\\\\
  \vbone \eqdef \vfieldone = \vfieldtwo\\
  \exprequal(\tenv, \veoneone, \vetwoone) \typearrow \vbtwo \OrTypeError\\\\
  \vb \eqdef \vbone \land \vbtwo
}{
  \exprequalcase(\tenv, \veone, \vetwo) \typearrow \vb
}
\end{mathpar}

\begin{mathpar}
  \inferrule[e\_getfields]{
  \veone \eqname \EGetFields(\veoneone, \vfieldsone)\\
  \vetwo \eqname \EGetFields(\vetwoone, \vfieldstwo)\\\\
  \vbone \eqdef \vfieldsone = \vfieldstwo\\
  \exprequal(\tenv, \veoneone, \vetwoone) \typearrow \vbtwo \OrTypeError\\\\
  \vb \eqdef \vbone \land \vbtwo
}{
  \exprequalcase(\tenv, \veone, \vetwo) \typearrow \vb
}
\end{mathpar}

\begin{mathpar}
  \inferrule[e\_getitem]{
  \veone \eqname \EGetItem(\veoneone, \vione)\\
  \vetwo \eqname \EGetItem(\vetwoone, \vitwo)\\\\
  \vbone \eqdef \vione = \vitwo\\
  \exprequal(\tenv, \veoneone, \vetwoone) \typearrow \vbtwo \OrTypeError\\\\
  \vb \eqdef \vbone \land \vbtwo
}{
  \exprequalcase(\tenv, \veone, \vetwo) \typearrow \vb
}
\end{mathpar}

\begin{mathpar}
  \inferrule[e\_literal]{
  \veone \eqname \ELiteral(\vvone)\\
  \vetwo \eqname \ELiteral(\vvtwo)\\\\
  \literalequal(\vvone, \vvtwo) \typearrow \vb
}{
  \exprequalcase(\tenv, \veone, \vetwo) \typearrow \vb
}
\end{mathpar}

\begin{mathpar}
\inferrule[e\_pattern]{
  \astlabel(\veone) = \EPattern \land \astlabel(\vetwo) = \EPattern
}{
  \exprequalcase(\tenv, \veone, \vetwo) \typearrow \False
}
\end{mathpar}

\begin{mathpar}
\inferrule[e\_record]{
  \astlabel(\veone) = \ERecord \land \astlabel(\vetwo) = \ERecord
}{
  \exprequalcase(\tenv, \veone, \vetwo) \typearrow \False
}
\end{mathpar}

\begin{mathpar}
\inferrule[e\_tuple]{
  \veone \eqname \ETuple(\vlone)\\
  \vetwo \eqname \ETuple(\vltwo)\\
  \equallength(\vlone, \vltwo) \typearrow \True \terminateas \False\\\\
  i \in \listrange(\vlone): \exprequal(\tenv, \vlone[i], \vltwo[i]) \typearrow \vb_i \OrTypeError\\\\
  \vb \eqdef \bigwedge_{i \in \listrange(\vlone)} \vb_i
}{
  \exprequalcase(\tenv, \veone, \vetwo) \typearrow \vb
}
\end{mathpar}

\begin{mathpar}
\inferrule[e\_unop]{
  \veone \eqname \EUnop(\opone, \veoneone)\\
  \vetwo \eqname \EUnop(\optwo, \vetwoone)\\\\
  \exprequal(\veoneone, \vetwoone) \typearrow \vbone \OrTypeError\\\\
  \vb \eqdef (\opone = \optwo) \land \vbone
}{
  \exprequalcase(\tenv, \veone, \vetwo) \typearrow \vb
}
\end{mathpar}

\begin{mathpar}
\inferrule[e\_unknown]{
  (\astlabel(\veone) = \EUnknown \land \astlabel(\vetwo) = \EUnknown)
}{
  \exprequalcase(\tenv, \veone, \vetwo) \typearrow \False
}
\end{mathpar}

\begin{mathpar}
\inferrule[e\_atc]{
  \veone \eqname \EATC(\veoneone, \vtone)\\
  \vetwo \eqname \EATC(\vetwoone, \vttwo)\\
  \exprequal(\tenv, \veoneone, \vetwoone) \typearrow \vbone \OrTypeError\\\\
  \typeequal(\tenv, \vtone, \vttwo) \typearrow \vbtwo \OrTypeError\\\\
  \vb \eqdef \vbone \land \vbtwo
}{
  \exprequalcase(\tenv, \veone, \vetwo) \typearrow \vb
}
\end{mathpar}

\begin{mathpar}
\inferrule[e\_var]{
  \veone \eqname \EVar(\nameone)\\
  \vetwo \eqname \EVar(\nametwo)\\\\
  \vb \eqdef \nameone = \nametwo
}{
  \exprequalcase(\tenv, \veone, \vetwo) \typearrow \vb
}
\end{mathpar}
\end{emptyformal}
%   | E_Pattern _, _ | E_Record _, _ -> assert false

\section{TypingRule.TypeEqual \label{sec:TypingRule.TypeEqual}}
\hypertarget{def-typeequal}{}
The function
\[
  \typeequal(\overname{\ty}{\vtone} \aslsep \overname{\ty}{\vttwo}) \aslto
   \overname{\{\True, \False\}}{\vb} \cup \overname{\TTypeError}{\TypeErrorConfig}
\]
conservatively tests whether the type $\vtone$ is equivalent to the type $\vttwo$ in environment $\tenv$
and yields the result in $\vb$.  The result is a type error, if one is detected.

\subsection{Prose}
One of the following applies:
\begin{itemize}
  \item All of the following apply (\textsc{different\_labels}):
  \begin{itemize}
    \item the AST labels of $\vtone$ and $\vttwo$ are different;
    \item $\vb$ is $\False$.
  \end{itemize}

  \item All of the following apply (\textsc{tbool\_treal\_tstring}):
  \begin{itemize}
    \item both $\vtone$ and $\vttwo$ are both either $\TBool$, $\TReal$, or $\TString$;
    \item $\vb$ is $\True$.
  \end{itemize}

  \item All of the following apply (\textsc{tint\_unconstrained}):
  \begin{itemize}
    \item both $\vtone$ and $\vttwo$ are the unconstrained integer type $\TInt(\unconstrained)$;
    \item $\vb$ is $\True$.
  \end{itemize}

  \item All of the following apply (\textsc{tint\_underconstrained}):
  \begin{itemize}
    \item $\vtone$ is the underconstrained integer type with identifier $\vione$, that is, \\ $\TInt(\underconstrained(\vione))$;
    \item $\vttwo$ is the underconstrained integer type with identifier $\vitwo$, that is, \\ $\TInt(\underconstrained(\vitwo))$;
    \item $\vb$ is $\True$ if and only if $\vione$ is equal to $\vitwo$.
  \end{itemize}

  \item All of the following apply (\textsc{tint\_wellconstrained}):
  \begin{itemize}
    \item $\vtone$ is the well-constrained integer type with list of constraints $\vcone$, that is, \\ $\TInt(\wellconstrained(\vcone))$;
    \item $\vttwo$ is the well-constrained integer type with list of constraints $\vctwo$, that is, \\ $\TInt(\wellconstrained(\vctwo))$;
    \item testing whether $\vcone$ and $\vctwo$ are equivalent in $\tenv$ yields $\vb$ \ProseOrTypeError.
  \end{itemize}

  \item All of the following apply (\textsc{tbits}):
  \begin{itemize}
    \item $\vtone$ is the bitvector type with width expression $\vwone$ and list of bitfields $\bfone$, that is, $\TBits(\vwone, \bfone)$;
    \item $\vttwo$ is the bitvector type with width expression $\vwtwo$ and list of bitfields $\bftwo$, that is, $\TBits(\vwtwo, \bftwo)$;
    \item testing whether $\vwone$ and $\vwtwo$ are equivalent bitwidths in $\tenv$ yields $\vbone$ \ProseOrTypeError;
    \item testing whether $\bfone$ and $\bftwo$ are equivalent lists of bitfields in $\tenv$ yields $\vbtwo$ \ProseOrTypeError;
    \item $\vb$ is $\True$ if and only if both $\vbone$ and $\vbtwo$ are $\True$.
  \end{itemize}

  \item All of the following apply (\textsc{tarray}):
  \begin{itemize}
    \item $\vtone$ is an array type with index $\vlone$ and element type $\vtone$, that is, $\TArray(\vlone, \vtone)$;
    \item $\vttwo$ is an array type with index $\vltwo$ and element type $\vttwo$, that is, $\TArray(\vltwo, \vttwo)$;
    \item testing whether $\vlone$ is equivalent to $\vltwo$ in $\tenv$ yields $\vbone$ \ProseOrTypeError;
    \item testing whether $\vtone$ is equivalent to $\vttwo$ in $\tenv$ yields $\vbtwo$ \ProseOrTypeError;
    \item $\vb$ is $\True$ if and only if both $\vbone$ and $\vbtwo$ are $\True$.
  \end{itemize}

  \item All of the following apply (\textsc{tnamed}):
  \begin{itemize}
    \item $\vtone$ is a named type with identifier $\vsone$, that is $\TNamed(\vsone)$;
    \item $\vttwo$ is a named type with identifier $\vstwo$, that is $\TNamed(\vstwo)$;
    \item $\vb$ is $\True$ if and only if $\vsone$ is equal to $\vstwo$.
  \end{itemize}

  \item All of the following apply (\textsc{tenum}):
  \begin{itemize}
    \item $\vtone$ is an enumeration type with identifier $\vlone$, that is $\TEnum(\vlone)$;
    \item $\vttwo$ is an enumeration type with identifier $\vltwo$, that is $\TEnum(\vltwo)$;
    \item $\vb$ is $\True$ if and only if $\vlone$ is equal to $\vltwo$.
  \end{itemize}

  \item All of the following apply (\textsc{tstructured}):
  \begin{itemize}
    \item $L$ is either $\TRecord$ or $\TException$;
    \item $\vtone$ is either a record type or an exception type with list of fields $\vfieldsone$, that is $L(\vfieldsone)$;
    \item $\vttwo$ is either a record type or an exception type with list of fields $\vfieldstwo$, that is $L(\vfieldstwo)$;
    \item checking whether the set of field names in $\vfieldsone$ is equal to the set of field names in $\vfieldstwo$
          yields $\True$ or $\False$, which short-circuits the entire evaluation;
    \item for each field $\vf$ in the set of fields of $\vfieldsone$, testing whether the type associated with
          $\vf$ in $\vfieldsone$ is equivalent to the type associated with
          $\vf$ in $\vfieldstwo$ in $\tenv$ yields $\vb_\vf$ \ProseOrTypeError;
    \item $\vb$ is $\True$ if and only if $\vb_\vf$ is $\True$ for each field $\vf$ in the set of fields of $\vfieldsone$.
  \end{itemize}

  \item All of the following apply (\textsc{ttuple}):
  \begin{itemize}
    \item $\vtone$ is a tuple type with list of types $\vtsone$, that is $\TTuple(\vtsone)$;
    \item $\vttwo$ is a tuple type with list of types $\vtstwo$, that is $\TTuple(\vtstwo)$;
    \item checking whether the list of types $\vtsone$ has the same length as the list of types $\vtstwo$ yields $\True$
          or $\False$, which short-circuits the entire evaluation;
    \item for each index $i$ in the list $\vtsone$, testing whether $\vtsone[i]$ is equivalent to $\vtstwo[i]$ in $\tenv$
          yields $\vb_i$ \ProseOrTypeError;
    \item $\vb$ is $\True$ if and only if $\vb_i$ is $\True$ for each index $i$ in the list $\vtsone$.
  \end{itemize}
\end{itemize}

\begin{emptyformal}
\subsection{Formally}
\begin{mathpar}
\inferrule[different\_labels]{
  \astlabel(\vtone) \neq \astlabel(\vttwo)
}{
  \typeequal(\tenv, \vtone, \vttwo) \typearrow \False
}
\end{mathpar}

\begin{mathpar}
\inferrule[TBool\_TReal\_TString]{
  \astlabel(\vtone) = \astlabel(\vttwo)\\
  \astlabel(\vtone) \in \{\TBool, \TReal, \TString\}
}{
  \typeequal(\tenv, \vtone, \vttwo) \typearrow \True
}
\and
\end{mathpar}

\begin{mathpar}
\inferrule[tint\_unconstrained]{}
{
  \typeequal(\tenv, \TInt(\unconstrained), \TInt(\unconstrained)) \typearrow \True
}
\and
\inferrule[tint\_underconstrained]{
  \vb \eqdef \vione = \vitwo
}{
  \typeequal(\tenv, \TInt(\underconstrained(\vione)), \TInt(\underconstrained(\vitwo))) \typearrow \vb
}
\and
\inferrule[tint\_wellconstrained]{
  \constraintsequal(\tenv, \vcone, \vctwo) \typearrow \vb \OrTypeError
}{
  \typeequal(\tenv, \TInt(\wellconstrained(\vcone)), \TInt(\wellconstrained(\vctwo))) \typearrow \vb
}
\end{mathpar}

\begin{mathpar}
\inferrule[tbits]{
  \bitwidthequal(\tenv, \vwone, \vwtwo) \typearrow \vbone \OrTypeError\\\\
  \bitfieldsequal(\tenv, \bfone, \bftwo) \typearrow \vbtwo \OrTypeError\\\\
  \vb \eqdef \vbone \land \vbtwo
}{
  \typeequal(\tenv, \TBits(\vwone, \bfone), \TBits(\vwtwo, \bftwo)) \typearrow \vb
}
\end{mathpar}

\begin{mathpar}
\inferrule[tarray]{
  \exprequal(\tenv, \vlone, \vltwo) \typearrow \vbone \OrTypeError\\\\
  \typeequal(\tenv, \vtone, \vttwo) \typearrow \vbtwo \OrTypeError\\\\
  \vb \eqdef \vbone \land \vbtwo
}{
  \typeequal(\tenv, \TArray(\vlone, \vtone), \TArray(\vltwo, \vttwo)) \typearrow \vb
}
\end{mathpar}

\begin{mathpar}
\inferrule[tnamed]{
  \vb \eqdef \vsone = \vstwo
}{
  \typeequal(\tenv, \TNamed(\vsone), \TNamed(\vstwo)) \typearrow \vb
}
\end{mathpar}

\begin{mathpar}
\inferrule[tenum]{
  \vb \eqdef \vlone = \vltwo
}{
  \typeequal(\tenv, \TEnum(\vlone), \TEnum(\vltwo)) \typearrow \vb
}
\end{mathpar}

\begin{mathpar}
\inferrule[tstructured]{
  L \in \{\TRecord, \TException\}\\
  \booltrans{\fieldnames(\vfieldsone) = \fieldnames(\vfieldstwo)} \booltransarrow \True \terminateas \False\\\\
  {
    \begin{array}{l}
  \vf \in \fieldnames(\vfieldsone): \\ \typeequal(\tenv, \fieldtype(\vfieldsone, \vf), \fieldtype(\vfieldstwo, \vf)) \typearrow \vb_\vf \OrTypeError
    \end{array}
  }\\\\
  \vb \eqdef \bigwedge_{\vf \in \fieldnames(\vfieldsone)} \vb_\vf
}{
  \typeequal(\tenv, L(\vfieldsone), L(\vfieldstwo)) \typearrow \vb
}
\end{mathpar}

\begin{mathpar}
\inferrule[ttuple]{
  \equallength(\vtsone, \vtstwo) \typearrow \True \terminateas \False\\
  i \in \listrange(\vtsone): \typeequal(\tenv, \vtsone[i], \vtstwo[i]) \typearrow \vb_i \OrTypeError\\\\
  \vb \eqdef \bigwedge_{i \in \listrange(\vtsone)} \vb_i
}{
  \typeequal(\tenv, \TTuple(\vtsone), \TTuple(\vtstwo)) \typearrow \vb
}
\end{mathpar}
\end{emptyformal}

\section{TypingRule.BitwidthEqual \label{sec:TypingRule.BitwidthEqual}}
\hypertarget{def-bitwidthequal}{}
The function
\[
  \bitwidthequal(\overname{\staticenvs}{\tenv} \aslsep \overname{\expr}{\vwone} \aslsep \overname{\expr}{\vwtwo})
  \aslto \overname{\{\True, \False\}}{\vb} \cup \overname{\TTypeError}{\TypeErrorConfig}
\]
conservatively tests whether the bitwidth expression $\vwone$ is equivalent to the bitwidth expression $\vwtwo$
in environment $\tenv$ and yields the result in $\vb$.  The result is a type error, if one is detected.

\subsection{Prose}
Testing whether the expressions $\vwone$ and $\vwtwo$ are equivalent in $\tenv$ yields $\vb$ \ProseOrTypeError.

\begin{emptyformal}
\subsection{Formally}
\begin{mathpar}
\inferrule{
  \exprequal(\tenv, \vwone, \vwtwo) \typearrow \vb \OrTypeError
}{
  \bitwidthequal(\tenv, \vwone, \vwtwo) \typearrow \vb
}
\end{mathpar}
\end{emptyformal}

\section{TypingRule.BitFieldsEqual \label{sec:TypingRule.BitFieldsEqual}}
\hypertarget{def-bitfieldsequal}{}
The function
\[
  \bitfieldsequal(\overname{\staticenvs}{\tenv} \aslsep \overname{\bitfield^*}{\bfone} \aslsep \overname{\bitfield^*}{\bftwo})
  \aslto \overname{\{\True, \False\}}{\vb} \cup \overname{\TTypeError}{\TypeErrorConfig}
\]
conservatively tests whether the list of bitfields $\bfone$ is equivalent to the list of bitfields $\bftwo$
in environment $\tenv$ and yields the result in $\vb$.  The result is a type error, if one is detected.

\subsection{Prose}
One of the following applies:
\begin{itemize}
  \item All of the following apply (\textsc{different\_lengths}):
  \begin{itemize}
    \item the number of bitfields in $\bfone$ is different to the number of bitfields in $\bftwo$;
    \item $\vb$ is $\False$.
  \end{itemize}

  \item All of the following apply (\textsc{same\_lengths}):
  \begin{itemize}
    \item the number of bitfields in $\bfone$ is the same as the number of bitfields in $\bftwo$;
    \item testing whether the bitfield $\bfone[i]$ is equivalent to $\bftwo[i]$ in $\tenv$ for every index
          of $\bfone$ yields $\vb_i$ \ProseOrTypeError;
    \item $\vb$ is $\True$ if and only if $\vb_i$ is $\True$ for every index of $\bfone$.
  \end{itemize}
\end{itemize}

\begin{emptyformal}
\subsection{Formally}
\begin{mathpar}
\inferrule[different\_lengths]{
  \equallength(\bfone, \bftwo) \typearrow \False
}{
  \bitfieldsequal(\tenv, \bfone, \bftwo) \typearrow \False
}
\and
\inferrule[same\_lengths]{
  \equallength(\bfone, \bftwo) \typearrow \True\\
  i\in\listrange(\bfone): \bitfieldequal(\tenv, \bfone[i], \bftwo[i]) \typearrow \vb_i\\\\
  \vb \eqdef \bigwedge_{i\in\listrange(\bfone)} \vb_i
}{
  \bitfieldsequal(\tenv, \bfone, \bftwo) \typearrow \vb
}
\end{mathpar}
\end{emptyformal}

\section{TypingRule.BitFieldEqual \label{sec:TypingRule.BitFieldEqual}}
The function
\[
  \bitfieldequal(\overname{\staticenvs}{\tenv} \aslsep \overname{\bitfield}{\bfone} \aslsep \overname{\bitfield}{\bftwo})
  \aslto \overname{\{\True, \False\}}{\vb} \cup \overname{\TTypeError}{\TypeErrorConfig}
\]
conservatively tests whether the bitfield $\bfone$ is equivalent to the bitfield $\bftwo$ in environment $\tenv$
and yields the result in $\vb$.  The result is a type error, if one is detected.

One of the following applies:
\begin{itemize}
  \item All of the following apply (\textsc{different\_labels}):
  \begin{itemize}
    \item the AST labels of $\bfone$ and $\bftwo$ are different;
    \item $\vb$ is $\False$.
  \end{itemize}

  \item All of the following apply (\textsc{bitfield\_simple}):
  \begin{itemize}
    \item $\bfone$ is a simple bitfield with name $\nameone$ and slices $\slicesone$, that is, \\ $\BitFieldSimple(\nameone, \slicesone)$;
    \item $\bftwo$ is a simple bitfield with name $\nametwo$ and slices $\slicestwo$, that is, \\ $\BitFieldSimple(\nametwo, \slicestwo)$;
    \item checking whether $\nameone$ is equal to $\nametwo$ yields $\vbone$;
    \item testing whether $\slicesone$ and $\slicestwo$ are equivalent in $\tenv$ yields $\vbtwo$ \ProseOrTypeError;
    \item $\vb$ is $\True$ if and only if both $\vbone$ and $\vbtwo$ are $\True$.
  \end{itemize}

  \item All of the following apply (\textsc{bitfield\_nested}):
  \begin{itemize}
    \item $\bfone$ is a nested bitfield with name $\nameone$, slices $\slicesone$, and nested bitfields $\bfoneone$, that is,
          $\BitFieldNested(\nameone, \slicesone, \bfoneone)$;
    \item $\bftwo$ is a nested bitfield with name $\nametwo$, slices $\slicestwo$, and nested bitfields $\bftwoone$, that is,
          $\BitFieldNested(\nametwo, \slicestwo, \bftwoone)$;
    \item checking whether $\nameone$ is equal to $\nametwo$ yields $\vbone$;
    \item testing whether $\slicesone$ and $\slicestwo$ are equivalent in $\tenv$ yields $\vbtwo$ \ProseOrTypeError;
    \item testing whether the bitfields $\bfoneone$ and $\bftwoone$ are equivalent in $\tenv$ yields $\vbtwo$ \ProseOrTypeError;
    \item $\vb$ is $\True$ if and only if both $\vbone$ and $\vbtwo$ are $\True$.
  \end{itemize}

  \item All of the following apply (\textsc{bitfield\_typed}):
  \begin{itemize}
    \item $\bfone$ is a typed bitfield with name $\nameone$, slices $\slicesone$, and type $\vtone$, that is,
          $\BitFieldType(\nameone, \slicesone, \vtone)$;
    \item $\bftwo$ is a typed bitfield with name $\nametwo$, slices $\slicestwo$, and type $\vttwo$, that is,
          $\BitFieldType(\nametwo, \slicestwo, \vttwo)$;
    \item checking whether $\nameone$ is equal to $\nametwo$ yields $\vbone$;
    \item testing whether $\slicesone$ and $\slicestwo$ are equivalent in $\tenv$ yields $\vbtwo$ \ProseOrTypeError;
    \item testing whether the types $\vtone$ and $\vttwo$ are equivalent in $\tenv$ yields $\vbtwo$ \ProseOrTypeError;
    \item $\vb$ is $\True$ if and only if both $\vbone$ and $\vbtwo$ are $\True$.
  \end{itemize}
\end{itemize}

\begin{emptyformal}
\subsection{Formally}
\begin{mathpar}
\inferrule[different\_labels]{
  \astlabel(\bfone) \neq \astlabel(\bftwo)
}{
  \bitfieldequal(\tenv, \bfone, \bftwo) \typearrow \False
}
\and
\inferrule[bitfield\_simple]{
  \bfone \eqname \BitFieldSimple(\nameone, \slicesone)\\
  \bftwo \eqname \BitFieldSimple(\nametwo, \slicestwo)\\
  \booltrans{\nameone = \nametwo} \booltransarrow \vbone\\
  \slicesequal(\tenv, \slicesone, \slicestwo) \typearrow \vbtwo \OrTypeError\\\\
  \vb \eqdef \vbone \land \vbtwo
}{
  \bitfieldequal(\tenv, \bfone, \bftwo) \typearrow \vb
}
\and
\inferrule[bitfield\_nested]{
  \bfone \eqname \BitFieldNested(\nameone, \slicesone, \bfoneone)\\
  \bftwo \eqname \BitFieldNested(\nametwo, \slicestwo, \bftwoone)\\
  \booltrans{\nameone = \nametwo} \booltransarrow \True \terminateas \False\\\\
  \slicesequal(\tenv, \slicesone, \slicestwo) \typearrow \vbone \terminateas \TypeErrorConfig,\\\\
  \bitfieldsequal(\tenv, \bfoneone, \bftwoone) \typearrow \vbtwo
}{
  \bitfieldequal(\tenv, \bfone, \bftwo) \typearrow \vb
}
\and
\inferrule[bitfield\_typed]{
  \bfone \eqname \BitFieldType(\nameone, \slicesone, \vtone)\\
  \bftwo \eqname \BitFieldType(\nametwo, \slicestwo, \vttwo)\\
  \booltrans{\nameone = \nametwo} \booltransarrow \True \terminateas \False \\\\
  \slicesequal(\tenv, \slicesone, \slicestwo) \typearrow \vbone \OrTypeError\\\\
  \typeequal(\tenv, \vtone, \vttwo) \typearrow \vbtwo \OrTypeError\\\\
  \vb \eqdef \vbone \land \vbtwo
}{
  \bitfieldequal(\tenv, \bfone, \bftwo) \typearrow \vb
}
\end{mathpar}
\end{emptyformal}

\section{TypingRule.ConstraintsEqual \label{sec:TypingRule.ConstraintsEqual}}
\hypertarget{def-constraintsequal}{}
The function
\[
  \constraintsequal(\overname{\staticenvs}{\tenv} \aslsep \overname{\intconstraints}{\csone} \aslsep \overname{\intconstraints}{\cstwo})
  \aslto \overname{\{\True, \False\}}{\vb} \cup \overname{\TTypeError}{\TypeErrorConfig}
\]
conservatively tests whether the constraint list $\csone$ is equivalent to the constraint list $\cstwo$ in environment $\tenv$
and yields the result in $\vb$.  The result is a type error, if one is detected.

\subsection{Prose}
One of the following applies:
\begin{itemize}
  \item All of the following apply (\textsc{different\_lengths}):
  \begin{itemize}
    \item the number of constraints in $\csone$ is different to the number of constraints in $\cstwo$;
    \item $\vb$ is $\False$.
  \end{itemize}

  \item All of the following apply (\textsc{same\_lengths}):
  \begin{itemize}
    \item the number of constraints in $\csone$ is the same as the number of constraints in $\cstwo$;
    \item testing whether the constraint $\csone[i]$ is equivalent to the constraint $\cstwo[i]$ in $\tenv$
          yields $\vb_i$ for each index in the index in the indices for $\csone$ ($i\in\listrange(\csone)$) \ProseOrTypeError;
    \item $\vb$ is $\True$ if and only if all $\vb_i$ are $\True$ for each index in the indices for $\csone$.
  \end{itemize}
\end{itemize}

\begin{emptyformal}
\subsection{Formally}
\begin{mathpar}
\inferrule[different\_lengths]{
  \equallength(\csone, \cstwo) \typearrow \False
}{
  \constraintsequal(\tenv, \csone, \cstwo) \typearrow \False
}
\and
\inferrule[same\_lengths]{
  \equallength(\csone, \cstwo) \typearrow \True\\
  i\in\listrange(\csone): \constraintequal(\tenv, \csone[i], \cstwo[i]) \typearrow \vb_i\\
  \vb \eqdef \bigwedge_{i\in\listrange(\csone)} \vb_i
}{
  \constraintsequal(\tenv, \csone, \cstwo) \typearrow \vb
}
\end{mathpar}
\end{emptyformal}

\section{TypingRule.ConstraintEqual \label{sec:TypingRule.ConstraintEqual}}
\hypertarget{def-constraintequal}{}
The function
\[
  \constraintequal(\overname{\staticenvs}{\tenv} \aslsep \overname{\intconstraint}{\vcone} \aslsep \overname{\intconstraint}{\vstwo})
  \aslto \overname{\{\True, \False\}}{\vb} \cup \overname{\TTypeError}{\TypeErrorConfig}
\]
conservatively tests whether the constraint $\vcone$ is equivalent to the constraint $\vctwo$ in environment $\tenv$
and yields the result in $\vb$.  The result is a type error, if one is detected.

\subsection{Prose}
One of the following applies:
\begin{itemize}
  \item All of the following apply (\textsc{different\_labels}):
  \begin{itemize}
    \item the AST labels of $\vcone$ and $\vctwo$ are different;
    \item $\vb$ is $\False$.
  \end{itemize}

  \item All of the following apply (\textsc{constraint\_exact}):
  \begin{itemize}
    \item $\vcone$ is an exact constraint with sub-expression $\veone$, that is, $\ConstraintExact(\veone)$;
    \item $\vctwo$ is an exact constraint with sub-expression $\vetwo$, that is, $\ConstraintExact(\vetwo)$;
    \item testing whether $\veone$ is equivalent to $\vetwo$ yields $\vb$ \ProseOrTypeError.
  \end{itemize}

  \item All of the following apply (\textsc{constraint\_range}):
  \begin{itemize}
    \item $\vcone$ is range constraint with sub-expressions $\veoneone$ and $\veonetwo$, that is, \\ $\ConstraintRange(\veoneone, \veonetwo)$;
    \item $\vctwo$ is range constraint with sub-expressions $\vetwoone$ and $\vetwotwo$, that is, \\ $\ConstraintRange(\vetwoone, \vetwotwo)$;
    \item testing whether $\veoneone$ is equivalent to $\vetwoone$ yields $\vbone$ \ProseOrTypeError;
    \item testing whether $\veonetwo$ is equivalent to $\vetwotwo$ yields $\vbtwo$ \ProseOrTypeError;
    \item $\vb$ is $\True$ if and only if both $\vbone$ and $\vbtwo$ are $\True$.
  \end{itemize}
\end{itemize}

\begin{emptyformal}
\subsection{Formally}
\begin{mathpar}
\inferrule[different\_labels]{
  \astlabel(\vcone) \neq \astlabel(\vctwo)
}{
  \constraintequal(\tenv, \vcone, \vctwo) \typearrow \False
}
\and
\inferrule[constraint\_exact]{
  \vcone \eqname \ConstraintExact(\veone)\\
  \vctwo \eqname \ConstraintExact(\vetwo)\\
  \exprequal(\tenv, \veone, \vetwo) \typearrow \vb \OrTypeError
}{
  \constraintequal(\tenv, \vcone, \vctwo) \typearrow \vb
}
\and
\inferrule[constraint\_range]{
  \bfone \eqname \ConstraintRange(\veoneone, \veonetwo)\\
  \bftwo \eqname \ConstraintRange(\vetwoone, \vetwotwo)\\
  \exprequal(\tenv, \veoneone, \vetwoone) \typearrow \vbone \OrTypeError\\\\
  \exprequal(\tenv, \veonetwo, \vetwotwo) \typearrow \vbtwo \OrTypeError\\\\
  \vb \eqdef \vbone \land \vbtwo
}{
  \constraintequal(\tenv, \bfone, \bftwo) \typearrow \vb
}
\end{mathpar}
\end{emptyformal}

\section{TypingRule.SlicesEqual \label{sec:TypingRule.SlicesEqual}}
\hypertarget{def-slicesequal}{}
The function
\[
  \slicesequal(\overname{\staticenvs}{\tenv} \aslsep \overname{\slice^*}{\slicesone} \aslsep \overname{\slice^*}{\slicestwo})
  \aslto \overname{\{\True, \False\}}{\vb} \cup \overname{\TTypeError}{\TypeErrorConfig}
\]
conservatively tests whether the list of slices $\slicesone$ is equivalent to the list of slices $\slicestwo$
in environment $\tenv$ and yields the result in $\vb$.  The result is a type error, if one is detected.

\subsection{Formally}
One of the following applies:
\begin{itemize}
  \item All of the following apply (\textsc{different\_lengths}):
  \begin{itemize}
    \item checking whether the number of slices in $\slicesone$ is equal to the number of slice in $\slicestwo$ yields $\False$;
    \item $\vb$ is $\False$.
  \end{itemize}

  \item All of the following apply (\textsc{same\_lengths}):
  \begin{itemize}
    \item checking whether the number of slices in $\slicesone$ is equal to the number of slice in $\slicestwo$ yields $\True$;
    \item determining whether the expression $\slicesone[i]$ is equivalent to $\slicestwo[i]$ in $\tenv$
          for each index in the indices for $\slicesone$ ($i \in \listrange(\slicesone$) yields $\vb_i$ \ProseOrTypeError;
    \item $\vb$ is $\True$ if and only if all $\vb_i$ are $\True$ for each index in the indices for $\slicesone$.
  \end{itemize}
\end{itemize}

\begin{emptyformal}
\subsection{Formally}
\begin{mathpar}
\inferrule[different\_lengths]{
  \equallength(\slicesone, \slicestwo) \typearrow \False
}{
  \slicesequal(\tenv, \slicesone, \slicestwo) \typearrow \False
}
\and
\inferrule[same\_lengths]{
  \equallength(\slicesone, \slicestwo) \typearrow \True\\
  i\in\listrange(\slicesone): \sliceequal(\tenv, \slicesone[i], \slicestwo[i]) \typearrow \vb_i \OrTypeError\\\\
  \vb \eqdef \bigwedge_{i\in\listrange(\slicesone)} \vb_i
}{
  \slicesequal(\tenv, \slicesone, \slicestwo) \typearrow \vb
}
\end{mathpar}
\end{emptyformal}

\section{TypingRule.SliceEqual \label{sec:TypingRule.SliceEqual}}
\hypertarget{def-sliceequal}{}
The function
\[
  \sliceequal(\overname{\staticenvs}{\tenv} \aslsep \overname{\slice}{\sliceone} \aslsep \overname{\slice}{\slicetwo})
  \aslto \overname{\{\True, \False\}}{\vb} \cup \overname{\TTypeError}{\TypeErrorConfig}
\]
conservatively tests whether the slice $\sliceone$ is equivalent to the slice $\slicetwo$
in environment $\tenv$ and yields the result in $\vb$. The result is a type error, if one is detected.

\subsection{Prose}
One of the following applies:
\begin{itemize}
  \item All of the following apply (\textsc{different\_labels}):
  \begin{itemize}
    \item $\sliceone$ and $\slicetwo$ have different AST labels;
    \item $\vb$ is $\False$.
  \end{itemize}

  \item All of the following apply (\textsc{slice\_single}):
  \begin{itemize}
    \item $\sliceone$ is a slice for a single position, given by the expression $\veone$, that is, $\SliceSingle(\veone)$;
    \item $\slicetwo$ is a slice for a single position, given by the expression $\vetwo$, that is, $\SliceSingle(\vetwo)$;
    \item testing $\veone$ and $\vetwo$ for equivalence yields $\vb$ \ProseOrTypeError.
  \end{itemize}

  \item All of the following apply (\textsc{slice\_range}):
  \begin{itemize}
    \item $\sliceone$ is a slice for a range of positions, given by the expressions $\veoneone$ and $\veonetwo$, that is, $\SliceRange(\veoneone, \veonetwo)$;
    \item $\slicetwo$ is a slice for a range of positions, given by the expressions $\vetwoone$ and $\vetwotwo$, that is, $\SliceRange(\vetwoone, \vetwotwo)$;
    \item testing $\veoneone$ and $\vetwoone$ for equivalence yields $\vbone$ \ProseOrTypeError;
    \item testing $\veonetwo$ and $\vetwotwo$ for equivalence yields $\vbtwo$ \ProseOrTypeError;
    \item $\vb$ is $\True$ if and only if both $\vbone$ and $\vbtwo$ are $\True$.
  \end{itemize}

  \item All of the following apply (\textsc{slice\_length}):
  \begin{itemize}
    \item $\sliceone$ is a slice for a range of positions, given by the start expression $\veoneone$ and length expression $\veonetwo$, that is, $\SliceLength(\veoneone, \veonetwo)$;
    \item $\slicetwo$ is a slice for a range of positions, given by the start expression $\vetwoone$ and length expression $\vetwotwo$, that is, $\SliceLength(\vetwoone, \vetwotwo)$;
    \item testing $\veoneone$ and $\vetwoone$ for equivalence yields $\vbone$ \ProseOrTypeError;
    \item testing $\veonetwo$ and $\vetwotwo$ for equivalence yields $\vbtwo$ \ProseOrTypeError;
    \item $\vb$ is $\True$ if and only if both $\vbone$ and $\vbtwo$ are $\True$.
  \end{itemize}
\end{itemize}

\begin{emptyformal}
\subsection{Formally}
\begin{mathpar}
\inferrule[different\_label]{
  \astlabel(\sliceone) \neq \astlabel(\slicetwo)
}{
  \slicesequal(\tenv, \sliceone, \slicetwo) \typearrow \False
}
\and
\inferrule[slice\_single]{
  \exprequal(\tenv, \veone, \vetwo) \typearrow \vb \OrTypeError
}{
  \slicesequal(\tenv, \SliceSingle(\veone), \SliceSingle(\vetwo)) \typearrow \vb
}
\and
\inferrule[slice\_range]{
  \exprequal(\tenv, \veoneone, \vetwoone) \typearrow \vbone \OrTypeError\\\\
  \exprequal(\tenv, \vetwoone, \vetwotwo) \typearrow \vbtwo \OrTypeError\\\\
  \vb \eqdef \vbone \land \vbtwo
}{
  \slicesequal(\tenv, \SliceRange(\veoneone, \veonetwo), \SliceRange(\vetwoone, \vetwotwo)) \typearrow \vb
}
\and
\inferrule[slice\_length]{
  \exprequal(\tenv, \veoneone, \vetwoone) \typearrow \vbone \OrTypeError\\\\
  \exprequal(\tenv, \vetwoone, \vetwotwo) \typearrow \vbtwo \OrTypeError\\\\
  \vb \eqdef \vbone \land \vbtwo
}{
  \slicesequal(\tenv, \SliceLength(\veoneone, \veonetwo), \SliceLength(\vetwoone, \vetwotwo)) \typearrow \vb
}
\end{mathpar}
\end{emptyformal}

\section{TypingRule.ArrayLengthEqual \label{sec:TypingRule.ArrayLengthEqual}}
\hypertarget{def-arraylengthequal}{}
The function
\[
  \arraylengthequal(\overname{\arrayindex}{\vlone} \aslsep \overname{\arrayindex}{\vltwo})
  \aslto \overname{\{\True, \False\}}{\vb} \cup \overname{\TTypeError}{\TypeErrorConfig}
\]
tests whether the array lengths $\vlone$ and $\vltwo$ are equivalent and yields the result
in $\vb$. The result is a type error, if one is detected.

\subsection{Prose}
One of the following applies:
\begin{itemize}
  \item All of the following apply (\textsc{different\_labels}):
  \begin{itemize}
    \item $\vlone$ and $\vltwo$ have different AST labels;
    \item $\vb$ is $\False$.
  \end{itemize}

  \item All of the following apply (\textsc{expr\_expr}):
  \begin{itemize}
    \item $\vlone$ is an integer type length expression with sub-expression $\veoneone$, that is, $\ArrayLengthExpr(\veoneone)$;
    \item $\vltwo$ is an integer type length expression with sub-expression $\vetwoone$, that is, $\ArrayLengthExpr(\vetwoone)$;
    \item testing whether $\veoneone$ and $\vetwoone$ are equivalent in $\tenv$ yields $\vb$ \ProseOrTypeError.
  \end{itemize}

  \item All of the following apply (\textsc{enum\_enum}):
  \begin{itemize}
    \item $\vlone$ is an enumeration type length expression over the enumeration $\vsone$, that is, $\ArrayLengthEnum(\vsone, \Ignore)$;
    \item $\vltwo$ is an enumeration type length expression over the enumeration $\vstwo$, that is, $\ArrayLengthEnum(\vstwo, \Ignore)$;
    \item $\vb$ is $\True$ if and only if $\vsone$ is equal to $\vstwo$.
  \end{itemize}
\end{itemize}

\begin{emptyformal}
\subsection{Formally}
\begin{mathpar}
\inferrule[different\_labels]{
  \astlabel(\vlone) \neq \astlabel(\vltwo)
}{
  \arraylengthequal(\vlone, \vltwo) \typearrow \False
}
\and
  \inferrule[expr\_expr]{
  \exprequal(\veoneone, \vetwoone) \typearrow \vb \OrTypeError
}{
  \arraylengthequal(\ArrayLengthExpr(\veoneone), \ArrayLengthExpr(\vetwoone)) \typearrow \vb
}
\and
\inferrule[enum\_enum]{
  \vb \eqdef \vsone = \vstwo
}{
  \arraylengthequal(\ArrayLengthEnum(\vsone, \Ignore), \ArrayLengthEnum(\vstwo, \Ignore)) \typearrow \vb
}
\end{mathpar}
\end{emptyformal}

\section{TypingRule.LiteralEqual \label{sec:TypingRule.LiteralEqual}}
\hypertarget{def-literalequal}{}
The function
\[
  \literalequal(\overname{\literal}{\vvone} \aslsep \overname{\literal}{\vvtwo}) \rightarrow \overname{\{\True, \False\}}{\vb}
\]
tests whether literal $\vvone$ is $\vvtwo$ by equating them.

\subsection{Prose}
$\vb$ is $\True$ if and only if $\vvone$ is equal to $\vvtwo$.

\begin{emptyformal}
\subsection{Formally}
\begin{mathpar}
  \inferrule{
    \vb \eqdef \vvone = \vvtwo
  }
  {
    \literalequal(\vvone, \vvtwo) \typearrow \vb
  }
\end{mathpar}
\end{emptyformal}

%%%%%%%%%%%%%%%%%%%%%%%%%%%%%%%%%%%%%%%%%%%%%%%%%%%%%%%%%%%%%%%%%%%%%%%%%%%%%%%%%%%%
\chapter{Utility Rules}
%%%%%%%%%%%%%%%%%%%%%%%%%%%%%%%%%%%%%%%%%%%%%%%%%%%%%%%%%%%%%%%%%%%%%%%%%%%%%%%%%%%%

\section{Checked Transitions}
\hypertarget{def-checktrans}{}
We define the following rules to allow us asserting that a condition holds,
returning a type error otherwise:
\begin{mathpar}
  \inferrule[check\_trans\_true]{}{ \checktrans{\True}{<message>} \checktransarrow \True }
  \and
  \inferrule[check\_trans\_false]{}{ \checktrans{\False}{<message>} \checktransarrow \TypeErrorVal{\texttt{<message>}} }
\end{mathpar}

\hypertarget{def-pairstomap}{}
\section{Converting a List of Pairs to a Map \label{sec:PairsToMap}}
The parametric function
\[
  \pairstomap(\overname{(\identifier\times T)^*}{\pairs}) \aslto \overname{(\identifier\partialto T)}{f} \cup \TTypeError
\]
converts a list of pairs --- $\pairs$ --- where each pair consists of an identifier and a value
of type $T$ into a function mapping each identifier to its respective value in the list.
If a duplicate identifier exists in $\pairs$ then a type error is returned.

\subsection{Prose}
One of the following applies:
\begin{itemize}
  \item All of the following apply (\textsc{empty}):
  \begin{itemize}
    \item $\pairs$ is empty;
    \item $f$ is the empty function.
  \end{itemize}

  \item All of the following apply (\textsc{error}):
  \begin{itemize}
    \item there exist two different positions in the list where the identifier is the same;
    \item the result is a type error indicating the existence of a duplicate identifier.
  \end{itemize}

  \item All of the following apply (\textsc{okay}):
  \begin{itemize}
    \item all identifiers occurring in the list are unique;
    \item $f$ is a function that associates to each identifier the value appearing with it in $\pairs$.
  \end{itemize}
\end{itemize}

\begin{emptyformal}
\begin{mathpar}
\inferrule[empty]{}{ \pairstomap(\emptylist) \typearrow \emptyfunc }
\and
\inferrule[error]{
  i,j \in 1..k\\
  i \neq j\\
  \id_i = \id_j
}
{
  \pairstomap([i=1..k: (\id_i,t_i)]) \typearrow \TypeErrorVal{DuplicateIdentifier}
}
\and
\inferrule[okay]{
  \forall i,j \in 1..k. \id_i \neq \id_j\\
  {
  f \eqdef \lambda \id.\ \begin{cases}
    t_i & \text{ if }i\in1..k \land \id = \id_i\\
    \bot & \text{ otherwise}
  \end{cases}
  }
}
{
  \pairstomap([i=1..k: (\id_i,t_i)]) \typearrow f
}
\end{mathpar}
\end{emptyformal}

\hypertarget{def-checknoduplicates}{}
\section{TypingRule.CheckNoDuplicates \label{sec:TypingRule.CheckNoDuplicates}}
The function
\[
  \checknoduplicates(\overname{\identifier^*}{\id_{1..k}}) \aslto \True \cup \TTypeError
\]
checks whether a non-empty list of identifiers contains a duplicate identifier. If it does not, the result
is $\True$ and otherwise the result is a type error.

\subsection{Prose}
One of the following applies:
\begin{itemize}
  \item All of the following apply (\textsc{okay}):
  \begin{itemize}
    \item the set containing all identifiers in the list has the same cardinality as the length of the list;
    \item the result is $\True$.
  \end{itemize}

  \item All of the following apply (\textsc{error}):
  \begin{itemize}
    \item there exist two different positions in the list where the identifier is the same;
    \item the result is a type error indicating the existence of a duplicate identifier.
  \end{itemize}
\end{itemize}

\begin{emptyformal}
\begin{mathpar}
  \inferrule[okay]{
    |\{\id_{1..k}\}| = k
  }
  {
    \checknoduplicates(\id_{1..k}) \typearrow \True
  }
\and
\inferrule[error]{
  i,j \in 1..k\\
  i \neq j\\
  \id_i = \id_j
}
{
  \checknoduplicates(\id_{1..k}) \typearrow \TypeErrorVal{DuplicateIdentifier}
}
\end{mathpar}
\end{emptyformal}

\hypertarget{def-annotatefieldinit}{}
\section{Annotating Field Initializers}
The function
\[
  \annotatefieldinit(
    \overname{\staticenvs}{\tenv} \aslsep
    \overname{(\identifier\times\expr)}{(\name, \vep)} \aslsep
    \overname{\field^*}{\fieldtypes}
  ) \aslto
  \overname{(\identifier\times\expr)}{(\name, \vepp)}
\]
annotates a field initializers $(\name, \vep)$ in a record expression
with list of fields $\fieldtypes$ and returns the annotated field initializer
$(\name, \vepp)$. A type error is returned, if one is detected.

\subsection{Prose}
All of the following apply:
\begin{itemize}
  \item annotating the expression $\vep$ in $\tenv$ yields $(\vtp, \vepp)$ \ProseOrTypeError\\
  \item One of the following applies:
  \begin{itemize}
    \item All of the following apply (\textsc{okay}):
    \begin{itemize}
      \item the unique type associated with $\name$ in $\fieldtypes$ is $\tspecp$;
      \item determining whether $\vtp$ \typesatisfies\ $\tspecp$ in $\tenv$ yields $\True$ \ProseOrTypeError;
    \end{itemize}

    \item All of the following apply (\textsc{error}):
    \begin{itemize}
      \item there is no type associated with $\name$ in $\fieldtypes$;
      \item the result is a type error indicating that the field $\name$ does not exist.
    \end{itemize}
  \end{itemize}
\end{itemize}

\begin{emptyformal}
\subsection{Formally}
\begin{mathpar}
\inferrule[okay]{
  \annotateexpr{\tenv, \vep} \typearrow (\vtp, \vepp) \OrTypeError\\\\
  \fieldtype(\fieldtypes, \name) = \tspecp\\
  \checktypesat(\tenv, \vtp, \tspecp) \typearrow \True \OrTypeError
}{
  \annotatefieldinit(\tenv, (\name, \vep), \fieldtypes) \typearrow (\name, \vepp)
}
\and
\inferrule[error]{
  \annotateexpr{\tenv, \vep} \typearrow (\vtp, \vepp) \OrTypeError\\\\
  \fieldtype(\fieldtypes, \name) = \bot
}{
  \annotatefieldinit(\tenv, (\name, \vep), \fieldtypes) \typearrow \\
  \TypeErrorVal{FieldDoesNotExist}
}
\end{mathpar}
\end{emptyformal}

\section{TypingRule.BitFieldGetName \label{sec:TypingRule.BitFieldGetName}}
\hypertarget{def-bitfieldgetname}{}
The function
\[
  \bitfieldgetname : \overname{\bitfield}{\vbf} \aslto \overname{\identifier}{\name}
\]
returns the name of a bitfield --- $\name$, given a bitfield $\vbf$.

\subsection{Prose}
One of the following applies:
\begin{itemize}
  \item $\vb$ is a simple bitfield with name $\name$, that is, $\BitFieldSimple(\name, \Ignore)$;
  \item $\vb$ is a nested bitfield with name $\name$, that is, $\BitFieldNested(\name, \Ignore, \Ignore)$;
  \item $\vb$ is a typed bitfield with name $\name$, that is, $\BitFieldType(\name, \Ignore, \Ignore)$.
\end{itemize}

\begin{emptyformal}
\subsection{Formally}
\begin{mathpar}
  \inferrule[simple]{}{
    \bitfieldgetname(\BitFieldSimple(\name, \Ignore)) \typearrow \name
  }
  \and
  \inferrule[nested]{}{
    \bitfieldgetname(\BitFieldNested(\name, \Ignore, \Ignore)) \typearrow \name
  }
  \and
  \inferrule[type]{}{
    \bitfieldgetname(\BitFieldType(\name, \Ignore, \Ignore)) \typearrow \name
  }
\end{mathpar}
\end{emptyformal}

\hypertarget{def-checkvarnotinenv}{}
\hypertarget{def-varinenv}{}
\section{TypingRule.CheckVarNotInEnv}
The function
\[
  \varinenv{\overname{\staticenvs}{\tenv} \aslsep \overname{\identifier}{\id}}
  \aslto \overname{\Bool}{\vb}
\]
determines whether an identifier $\id$ is declared in the static environment $\tenv$.

The function
\[
  \checkvarnotinenv{\overname{\staticenvs}{\tenv} \aslsep \overname{\identifier}{\id}}
  \aslto \True \cup \TTypeError
\]
checks whether $\id$ is declared in $\tenv$. If it is, the result is a type error,
and otherwise the result is $\True$.

\subsection{Prose}
$\varinenv{\tenv, \id}$ is true if and only if one of the following applies:
\begin{itemize}
  \item $\id$ is declared as a local identifier in $\tenv$;
  \item $\id$ is declared as a global identifier in $\tenv$;
  \item $\id$ is declared as a subprogram in $\tenv$;
  \item $\id$ is declared as a type in $\tenv$.
\end{itemize}

\begin{emptyformal}
\subsection{Formally}
\begin{mathpar}
\inferrule{
  {
    \begin{array}{rl}
  \vb \eqdef & L^\tenv.\localstoragetypes(\id) \neq \bot\ \lor\\
             & G^\tenv.\globalstoragetypes(\id) \neq \bot \lor\\
             & G^\tenv.\subprograms(\id) \neq \bot\ \lor\\
             & G^\tenv.\declaredtypes(\id) \neq \bot
  \end{array}
}
}{
  \varinenv{\tenv, \id} \typearrow \vb
}
\end{mathpar}

\begin{mathpar}
\inferrule[okay]{
  \varinenv{\tenv, \id} \typearrow \False
}{
  \checkvarnotinenv{\tenv, \id} \typearrow \True
}
\and
  \inferrule[error]{
  \varinenv{\tenv, \id} \typearrow \True
}{
  \checkvarnotinenv{\tenv, \id} \typearrow \TypeErrorVal{AlreadyDeclared}
}
\end{mathpar}
\end{emptyformal}

\hypertarget{def-addlocal}{}
\section{TypingRule.AddLocal \label{sec:TypingRule.AddLocal}}
The function
\[
  \addlocal(
    \overname{\staticenvs}{\tenv} \aslsep
    \overname{\identifier}{\id} \aslsep
    \overname{\ty}{\tty} \aslsep
    \overname{\localdeclkeyword}{\ldk})
  \aslto
  \overname{\staticenvs}{\newtenv}
\]
adds the identifier $\id$ as a local storage element with type $\tty$ and local declaration keyword $\ldk$
to the local environment of $\tenv$, resulting in the static environment $\newtenv$.

\subsection{Prose}
All of the following apply:
\begin{itemize}
  \item the map $\newlocalstoragetypes$ is defined by updating the map \\
        $\localstoragetypes$ of $\tenv$
        with the binding $\id$ to the type $\tty$ and local declaration keyword $\ldk$, that is, $(\tty,\ldk)$;
  \item $\newtenv$ is defined by updating the local environment with the binding of \\
        $\localstoragetypes$ to $\newlocalstoragetypes$.
\end{itemize}

\begin{emptyformal}
\subsection{Formally}
\begin{mathpar}
\inferrule{
  \newlocalstoragetypes \eqdef L^\tenv.\localstoragetypes[\id \mapsto (\tty, \ldk)]\\
  \newtenv \eqdef (G^\tenv, L^\tenv[\localstoragetypes \mapsto \newlocalstoragetypes])
}
{
  \addlocal(\tenv, \id, \tty, \ldk) \typearrow \newtenv
}
\end{mathpar}
\end{emptyformal}

\hypertarget{def-declaredtype}{}
\section{TypingRule.DeclaredType \label{sec:TypingRule.DeclaredType}}

The function
\[
  \declaredtype(\overname{\staticenvs}{\tenv} \aslsep \overname{\identifier}{\id}) \aslto \overname{\ty}{\vt} \cup \TTypeError
\]
retrieves the type associated with the identifier $\id$ in the static environment $\tenv$.
If the identifier is not associated with a declared type, a type error is returned.

\subsection{Prose}
One of the following applies:
\begin{itemize}
  \item All of the following apply (\textsc{exists}):
  \begin{itemize}
    \item $\id$ is bound in the global environment to the type $\vt$.
  \end{itemize}

  \item All of the following apply (\textsc{type\_not\_declared}):
  \begin{itemize}
    \item $\id$ is not bound in the global environment to any type;
    \item the result is a type error indicating the lack of a type declaration for $\id$.
  \end{itemize}
\end{itemize}

\subsection{Formally}
\begin{mathpar}
\inferrule[exists]{
  G^\tenv.\declaredtypes(\id) = \vt
}
{
  \declaredtype(\tenv, \id) \typearrow \vt
}
\and
\inferrule[type\_not\_declared]{
  G^\tenv.\declaredtypes(\id) = \bot
}
{
  \declaredtype(\tenv, \id) \typearrow \TypeErrorVal{TypeNotDeclared}
}
\end{mathpar}

\section{TypingRule.FindBitfieldOpt}
\hypertarget{def-findbitfieldopt}{}
The function
\[
  \findbitfieldopt(\overname{\identifier}{\name} \aslsep \overname{\bitfield^*}{\bitfields})
  \aslto \overname{\langle\bitfield\rangle}{\vr}
\]
returns the bitfield associated with the name $\name$ in the list of bitfields $\bitfields$,
if there is one. Otherwise, the result is $\None$.

\subsection{Prose}
One of the following applies:
\begin{itemize}
  \item All of the following apply (\textsc{match}):
  \begin{itemize}
    \item $\bitfields$ starts with a bitfield $\vbf$;
    \item obtaining the name associated with $\vbf$ yields $\name$;
    \item the result if $\vbf$.
  \end{itemize}

  \item All of the following apply (\textsc{tail}):
  \begin{itemize}
    \item $\bitfields$ starts with a bitfield $\vbf$ and continues with the tail list $\bitfieldsp$;
    \item obtaining the name associated with $\vbf$ yields $\namep$, which is different than $\name$;
    \item finding the bitfield associated with $\name$ in $\bitfieldsp$ yields the result $\vr$.
  \end{itemize}

  \item All of the following apply (\textsc{empty}):
  \begin{itemize}
    \item $\bitfields$ is an empty list;
    \item the result is $\None$.
  \end{itemize}
\end{itemize}

\begin{emptyformal}
\begin{mathpar}
\inferrule[match]{
  \bitfieldgetname(\vbf) \typearrow \name
}{
  \findbitfieldopt(\name, \overname{\vbf + \bitfieldsp}{\bitfields}) \typearrow \overname{\langle\vbf\rangle}{\vr}
}
\and
\inferrule[tail]{
  \bitfieldgetname(\vbf) \typearrow \namep\\
  \name \neq \namep\\
  \findbitfieldopt(\name, \bitfieldsp) \typearrow \vr
}{
  \findbitfieldopt(\name, \overname{\vbf + \bitfieldsp}{\bitfields}) \typearrow \vr
}
\and
\inferrule[empty]{}{
  \findbitfieldopt(\name, \overname{\emptylist}{\bitfields}) \typearrow \None
}
\end{mathpar}
\end{emptyformal}

\section{TypingRule.MemBfs}
\hypertarget{def-membfs}{}
The function
\[
  \membfs(\overname{\staticenvs}{\tenv} \aslsep \overname{\bitfield^+}{\bfstwo} \aslsep \overname{\bitfield}{\vbfone})
  \aslto \overname{\Bool}{\vb}
\]
checks whether the bitfield $\vbf$ exists in $\bfstwo$ in the context of $\tenv$, returning the result in $\vb$.

\subsection{Prose}
One of the following applies:
\begin{itemize}
  \item All of the following apply (\textsc{none}):
  \begin{itemize}
    \item the name associated with the bitfield $\vbfone$ is $\name$;
    \item finding the bitfield associated with $\name$ in $\bfstwo$ yields $\None$;
    \item $\vb$ is $\False$.
  \end{itemize}

  \item All of the following apply (\textsc{simple\_any}):
  \begin{itemize}
    \item the name associated with the bitfield $\vbfone$ is $\name$;
    \item finding the bitfield associated with $\name$ in $\bfstwo$ yields $\vbftwo$;
    \item $\vbftwo$ is a simple bitfield;
    \item symbolically checking whether $\vbfone$ is equivalent to $\vbftwo$ in $\tenv$ yields $\vb$.
  \end{itemize}

  \item All of the following apply (\textsc{nested\_simple}):
  \begin{itemize}
    \item the name associated with the bitfield $\vbfone$ is $\name$;
    \item finding the bitfield associated with $\name$ in $\bfstwo$ yields $\vbftwo$;
    \item $\vbftwo$ is a nested bitfield with name $\nametwo$, slices $\slicestwo$, and bitfields $\bfstwop$;
    \item $\vbfone$ is a simple bitfield;
    \item symbolically checking whether $\vbfone$ is equivalent to $\vbftwo$ in $\tenv$ yields $\vb$.
  \end{itemize}

  \item All of the following apply (\textsc{nested\_nested}):
  \begin{itemize}
    \item the name associated with the bitfield $\vbfone$ is $\name$;
    \item finding the bitfield associated with $\name$ in $\bfstwo$ yields $\vbftwo$;
    \item $\vbftwo$ is a nested bitfield with name $\nametwo$, slices $\slicestwo$, and bitfields $\bfstwop$;
    \item $\vbfone$ is a nested bitfield with name $\nameone$, slices $\sliceone$, and $\bfsone$;
    \item $\vbone$ is true if and only if $\nameone$ is equal to $\nametwo$;
    \item symbolically equating the slices $\slicesone$ and $\slicestwo$ in $\tenv$ yields $\vbtwo$;
    \item checking $\bfsone$ is included in $\bfstwop$ in the context of $\tenv$ yields $\vbthree$;
    \item $\vb$ is defined as the conjunction of $\vbone$, $\vbtwo$, and $\vbthree$.
  \end{itemize}

  \item All of the following apply (\textsc{nested\_typed}):
  \begin{itemize}
    \item the name associated with the bitfield $\vbfone$ is $\name$;
    \item finding the bitfield associated with $\name$ in $\bfstwo$ yields $\vbftwo$;
    \item $\vbftwo$ is a nested bitfield with name $\nametwo$, slices $\slicestwo$, and bitfields $\bfstwop$;
    \item $\vbfone$ is a typed bitfield;
    \item $\vb$ is $\False$.
  \end{itemize}

  \item All of the following apply (\textsc{typed\_simple}):
  \begin{itemize}
    \item the name associated with the bitfield $\vbfone$ is $\name$;
    \item finding the bitfield associated with $\name$ in $\bfstwo$ yields $\vbftwo$;
    \item $\vbftwo$ is a typed bitfield with name $\nametwo$, slices $\slicestwo$, and type $\ttytwo$;
    \item $\vbfone$ is a simple bitfield;
    \item symbolically checking whether $\vbfone$ is equivalent to $\vbftwo$ in $\tenv$ yields $\vb$.
  \end{itemize}

  \item All of the following apply (\textsc{typed\_nested}):
  \begin{itemize}
    \item the name associated with the bitfield $\vbfone$ is $\name$;
    \item finding the bitfield associated with $\name$ in $\bfstwo$ yields $\vbftwo$;
    \item $\vbftwo$ is a typed bitfield with name $\nametwo$, slices $\slicestwo$, and type $\ttytwo$;
    \item $\vbfone$ is a nested bitfield;
    \item $\vb$ is $\False$.
  \end{itemize}

  \item All of the following apply (\textsc{typed\_typed}):
  \begin{itemize}
    \item the name associated with the bitfield $\vbfone$ is $\name$;
    \item finding the bitfield associated with $\name$ in $\bfstwo$ yields $\vbftwo$;
    \item $\vbftwo$ is a typed bitfield with name $\nametwo$, slices $\slicestwo$, and type $\ttytwo$;
    \item $\vbfone$ is a typed bitfield with name $\nameone$, slices $\slicesone$, and type $\ttyone$;
    \item $\vbone$ is true if and only if $\nameone$ is equal to $\nametwo$;
    \item symbolically equating the slices $\slicesone$ and $\slicestwo$ in $\tenv$ yields $\vbtwo$;
    \item checking whether $\ttyone$ structurally subtypes $\ttytwo$ in $\tenv$ yields $\vbthree$;
    \item $\vb$ is defined as the conjunction of $\vbone$, $\vbtwo$, and $\vbthree$.
  \end{itemize}
\end{itemize}

\begin{emptyformal}
\begin{mathpar}
\inferrule[none]{
  \bitfieldgetname(\vbfone) \typearrow \name\\
  \findbitfieldopt(\name, \bfstwo) \typearrow \None
}{
  \membfs(\tenv, \bfstwo, \vbfone) \typearrow \False
}
\and
\inferrule[simple\_any]{
  \bitfieldgetname(\vbf) \typearrow \name\\
  \findbitfieldopt(\name, \bfstwo) \typearrow \langle \vbftwo \rangle\\
  \astlabel(\vbftwo) = \BitFieldSimple\\
  \bitfieldsequal(\tenv, \vbfone, \vbftwo) \typearrow \vb
}{
  \membfs(\tenv, \bfstwo, \vbfone) \typearrow \vb
}
\end{mathpar}

\begin{mathpar}
\inferrule[nested\_simple]{
  \bitfieldgetname(\vbf) \typearrow \name\\
  \findbitfieldopt(\name, \bfstwo) \typearrow \langle \vbftwo \rangle\\
  \vbftwo = \BitFieldNested(\nametwo, \slicestwo, \bfstwop)\\
  \vbfone = \BitFieldSimple(\Ignore)\\
  \bitfieldsequal(\tenv, \vbfone, \vbftwo) \typearrow \vb
}{
  \membfs(\tenv, \bfstwo, \vbfone) \typearrow \overname{\False}{\vb}
}
\and
\inferrule[nested\_nested]{
  \bitfieldgetname(\vbf) \typearrow \name\\
  \findbitfieldopt(\name, \bfstwo) \typearrow \langle \vbftwo \rangle\\
  \vbftwo = \BitFieldNested(\nametwo, \slicestwo, \bfstwop)\\
  \vbfone = \BitFieldNested(\nameone, \slicesone, \bfsone)\\
  \vbone \eqdef \nameone = \nametwo\\
  \slicesequal(\tenv, \slicesone, \slicestwo) \typearrow \vbtwo\\
  \bitfieldsincluded(\tenv, \bfsone, \bfstwop) \typearrow \vbthree\\
  \vb \eqdef \vbone \land \vbtwo \land \vbthree
}{
  \membfs(\tenv, \bfstwo, \vbfone) \typearrow \vb
}
\and
\inferrule[nested\_typed]{
  \bitfieldgetname(\vbf) \typearrow \name\\
  \findbitfieldopt(\name, \bfstwo) \typearrow \langle \vbftwo \rangle\\
  \vbftwo = \BitFieldNested(\nametwo, \slicestwo, \bfstwop)\\
  \astlabel(\vbfone) = \BitFieldType
}{
  \membfs(\tenv, \bfstwo, \vbfone) \typearrow \overname{\False}{\vb}
}
\end{mathpar}

\begin{mathpar}
\inferrule[typed\_simple]{
  \bitfieldgetname(\vbf) \typearrow \name\\
  \findbitfieldopt(\name, \bfstwo) \typearrow \langle \vbftwo \rangle\\
  \vbftwo = \BitFieldType(\nametwo, \slicestwo, \ttytwo)\\
  \vbfone = \BitFieldSimple(\Ignore)\\
  \bitfieldsequal(\tenv, \vbfone, \vbftwo) \typearrow \vb
}{
  \membfs(\tenv, \bfstwo, \vbfone) \typearrow \vb
}
\and
\inferrule[typed\_nested]{
  \bitfieldgetname(\vbf) \typearrow \name\\
  \findbitfieldopt(\name, \bfstwo) \typearrow \langle \vbftwo \rangle\\
  \vbftwo = \BitFieldType(\nametwo, \slicestwo, \ttytwo)\\
  \astlabel(\vbfone) = \BitFieldNested
}{
  \membfs(\tenv, \bfstwo, \vbfone) \typearrow \overname{\False}{\vb}
}
\and
\inferrule[typed\_typed]{
  \bitfieldgetname(\vbf) \typearrow \name\\
  \findbitfieldopt(\name, \bfstwo) \typearrow \langle \vbftwo \rangle\\
  \vbftwo = \BitFieldType(\nametwo, \slicestwo, \ttytwo)\\
  \vbfone = \BitFieldType(\nameone, \slicesone, \ttyone)\\
  \vbone \eqdef \nameone = \nametwo\\
  \slicesequal(\tenv, \slicesone, \slicestwo) \typearrow \vbtwo\\
  \structsubtypesat(\tenv, \ttyone, \ttytwo) \typearrow \vbthree \OrTypeError\\\\
  \vb \eqdef \vbone \land \vbtwo \land \vbthree
}{
  \membfs(\tenv, \bfstwo, \vbfone) \typearrow \vb
}
\end{mathpar}
\end{emptyformal}

\section{TypingRule.BitFieldsIncluded}
\hypertarget{def-bitfieldsincluded}{}
The predicate
\[
  \bitfieldsincluded(\overname{\staticenvs}{\tenv}, \overname{\bitfield^*}{\bfsone} \aslsep \overname{\bitfield^*}{\bfstwo})
  \aslto \overname{\Bool}{\vb} \cup \overname{\TTypeError}{\TypeErrorConfig}
\]
tests whether the set of bit fields $\bfsone$ is included in the set of bit fields $\bfstwo$ in environment $\tenv$,
returning a type error, if one is detected.

\subsection{Prose}
All of the following apply:
\begin{itemize}
  \item checking whether each field $\vbf$ in $\bfsone$ exists in $\bfstwo$ via $\membfs$ yields $\vb_\vbf$ \ProseOrTypeError;
  \item the result --- $\vb$ --- is the conjunction of $\vb_\vbf$ for all bitfields $\vbf$ in $\bfsone$.
\end{itemize}

\begin{emptyformal}
\begin{mathpar}
\inferrule{
  \vbf \in \bfsone: \membfs(\bfstwo, \vbf) \typearrow \vb_\vbf \OrTypeError\\\\
  \vbf \eqdef \bigwedge_{\bf \in \bfsone} \vb_\vbf
}{
  \bitfieldsincluded(\tenv, \bfsone, \bfstwo) \typearrow \vb
}
\end{mathpar}
\end{emptyformal}

\hypertarget{def-typeofarraylength}{}
\section{TypingRule.TypeOfArrayLength \label{sec:TypingRule.TypeOfArrayLength}}
The function
\[
  \typeofarraylength(\overname{\staticenvs}{\tenv} \aslsep \overname{\arrayindex}{\size}) \aslto
  \overname{\ty}{\vt}
\]
returns the type for the array index $\size$ in the static environment $\tenv$.

\subsection{Prose}
One of the following applies:
\begin{itemize}
  \item All of the following apply (\textsc{enum}):
  \begin{itemize}
    \item $\size$ is an enumeration index over the enumeration $\vs$, that is, \\ $\ArrayLengthEnum(\vs, \Ignore)$;
    \item $\vt$ is the named type for $\vs$, that is, $\TNamed(\vs)$.
  \end{itemize}

  \item All of the following apply (\textsc{expr}):
  \begin{itemize}
    \item $\size$ is an expression index for $\ve$, that is, $\ArrayLengthExpr(\ve)$;
    \item symbolically reducing the expression corresponding to $\ve - 1$ via \tododefine{reduce\_expr} in $\tenv$
    yields the expression $\vm$;
    \item $\vc$ is the range constraint for $0..\vm$, that is, $\ConstraintRange(\eliteral{0}, \vm)$;
    \item $\vt$ is the well-constrained integer with the single constraint $\vc$.
  \end{itemize}
\end{itemize}

\CodeSubsection{\TypeOfArrayLengthBegin}{\TypeOfArrayLengthEnd}{../types.ml}
\subsection{Formally}
\begin{mathpar}
\inferrule[enum]{}
{
  \typeofarraylength(\tenv, \ArrayLengthEnum(\vs, \Ignore)) \typearrow \TNamed(\vs)
}
\and
\inferrule[expr]{
  \tododefine{reduce\_expr}(\EBinop(\MINUS, \ve, \eliteral{1})) \typearrow \vm\\
  \vc \eqdef \ConstraintRange(\eliteral{0}, \vm)
}
{
  \typeofarraylength(\tenv, \ArrayLengthExpr(\ve)) \typearrow \TInt(\wellconstrained([\vc]))
}
\end{mathpar}

\hypertarget{def-checkstructureinteger}{}
\section{TypingRule.CheckStructureInteger \label{sec:TypingRule.CheckStructureInteger}}
The function
\[
  \checkstructureinteger(\overname{\staticenvs}{\tenv} \aslsep \overname{\ty}{\vt}) \aslto
  \True \cup \TTypeError
\]
returns $\True$ is $\vt$ is has the \structure\ an integer type and a type error otherwise.

\subsection{Prose}
One of the following applies:
\begin{itemize}
  \item All of the following apply (\textsc{okay}):
  \begin{itemize}
    \item determining the \structure\ of $\vt$ yields $\vtp$ \ProseOrTypeError;
    \item $\vtp$ is an integer type;
    \item the result is $\True$;
  \end{itemize}

  \item All of the following apply (\textsc{error}):
  \begin{itemize}
    \item determining the \structure\ of $\vt$ yields $\vtp$ \ProseOrTypeError;
    \item $\vtp$ is not an integer type;
    \item the result is a type error indicating that $\vt$ was expected to have the \structure\ of an integer.
  \end{itemize}
\end{itemize}

\CodeSubsection{\CheckStructureIntegerBegin}{\CheckStructureIntegerEnd}{../Typing.ml}

\subsection{Formally}
\begin{mathpar}
\inferrule[okay]{
  \tstruct(\vt) \typearrow \vtp \OrTypeError\\\\
  \astlabel(\vtp) = \TInt
}
{
  \checkstructureinteger(\tenv, \vt) \typearrow \True
}
\and
\inferrule[error]{
  \tstruct(\vt) \typearrow \vtp\\
  \astlabel(\vtp) \neq \TInt
}
{
  \checkstructureinteger(\tenv, \vt) \typearrow \TypeErrorVal{ExpectedIntegerStructure}
}
\end{mathpar}

\hypertarget{def-checkstructurelabel}{}
\section{TypingRule.CheckStructure \label{sec:TypingRule.CheckStructure}}
The function
\[
  \checkstructurelabel(\overname{\staticenvs}{\tenv} \aslsep \overname{\ty}{\vt} \aslsep \overname{\astlabels}{\vl}) \aslto
  \True \cup \TTypeError
\]
returns $\True$ is $\vt$ is has the \structure\ a of type corresponding to the AST label $\vl$ and a type error otherwise.

\subsection{Prose}
One of the following applies:
\begin{itemize}
  \item All of the following apply (\textsc{okay}):
  \begin{itemize}
    \item determining the \structure\ of $\vt$ yields $\vtp$ \ProseOrTypeError;
    \item $\vtp$ has the label $\vl$;
    \item the result is $\True$;
  \end{itemize}

  \item All of the following apply (\textsc{error}):
  \begin{itemize}
    \item determining the \structure\ of $\vt$ yields $\vtp$ \ProseOrTypeError;
    \item $\vtp$ does not have the label $\vl$;
    \item the result is a type error indicating that $\vt$ was expected to have the \structure\ of a type with the AST label $\vl$.
  \end{itemize}
\end{itemize}

\subsection{Formally}
\begin{mathpar}
\inferrule[okay]{
  \tstruct(\vt) \typearrow \vtp \OrTypeError\\\\
  \astlabel(\vtp) = \vl
}
{
  \checkstructurelabel(\tenv, \vt, \vl) \typearrow \True
}
\and
\inferrule[error]{
  \tstruct(\vt) \typearrow \vtp\\
  \astlabel(\vtp) \neq \vl
}
{
  \checkstructurelabel(\tenv, \vt, \vl) \typearrow \TypeErrorVal{UnexpectedTypeStructure}
}
\end{mathpar}

\hypertarget{def-storageispure}{}
\section{TypingRule.StorageIsPure \label{sec:TypingRule.StorageIsPure}}
The function
\[
  \storageispure(\overname{\staticenvs}{\tenv} \aslsep \overname{\identifier}{\vs}) \aslto
  \overname{\Bool}{\vb} \cup \TTypeError
\]
$\vb$ is true if and only if the identifier $\vs$ corresponds to a \pure\ storage element
in the static environment $\tenv$.

\subsection{Prose}
One of the following applies:
\begin{itemize}
  \item All of the following apply (\textsc{local}):
  \begin{itemize}
    \item $\vs$ is a locally declared storage element;
    \item $\vb$ is true if and only if $\vs$ is declared as a constant or as an immutable variable (\texttt{let}).
  \end{itemize}

  \item All of the following apply (\textsc{global}):
  \begin{itemize}
    \item $\vs$ is a globally declared storage element;
    \item $\vb$ is true if and only if $\vs$ is declared as a constant, a configuration variable, or an immutable variable.
  \end{itemize}

  \item All of the following apply (\textsc{error}):
  \begin{itemize}
    \item $\vs$ is not defined in the environment as a storage element;
    \item the result is a type error indicating that $\vs$ is not defined as a storage element.
  \end{itemize}
\end{itemize}

\CodeSubsection{\StorageIsPureBegin}{\StorageIsPureEnd}{../Typing.ml}

\subsection{Formally}
\begin{mathpar}
\inferrule[local]{
  L^\tenv.\localstoragetypes(\vs) = (\Ignore, \ldk)\\
  \vb \eqdef \ldk \in \{\LDKConstant, \LDKLet\}
}
{
  \storageispure(\tenv, \vs) \typearrow \vb
}
\and
\inferrule[global]{
  L^\tenv.\localstoragetypes(\vs) = \bot\\
  G^\tenv.\globalstoragetypes(\vs) = (\Ignore, \gdk)\\
  \vb \eqdef \gdk \in \{\GDKConstant, \GDKConfig, \GDKLet\}
}
{
  \storageispure(\tenv, \vs) \typearrow \vb
}
\and
\inferrule[error]{
  L^\tenv.\localstoragetypes(\vs) = \bot\\
  G^\tenv.\globalstoragetypes(\vs) = \bot
}
{
  \storageispure(\tenv, \vs) \typearrow \TypeErrorVal{UndefinedIdentifier}
}
\end{mathpar}

\hypertarget{def-checkstaticallyevaluable}{}
\section{TypingRule.CheckStaticallyEvaluable \label{sec:TypingRule.CheckStaticallyEvaluable}}
The function
\[
  \checkstaticallyevaluable(\overname{\staticenvs}{\tenv} \aslsep \overname{\expr}{\ve}) \aslto
  \True \cup \TTypeError
\]
returns $\True$ if $\ve$ is a \staticallyevaluable\ expression in the static environment $\tenv$ and a type error otherwise.

\subsection{Prose}
All of the following applies:
\begin{itemize}
  \item symbolically reducing $\ve$ in $\tenv$ yields $\veone$;
  \item determining the set of used identifiers in $\veone$ yields $\useset$;
  \item $\vb$ is true if and only if every identifier in $\useset$ is pure;
  \item the result is $\True$ is $\vb$ is $\True$, otherwise it is a type error indicating that the expression
  is not statically evaluable.
\end{itemize}

\CodeSubsection{\CheckStaticallyEvaluableBegin}{\CheckStaticallyEvaluableEnd}{../Typing.ml}

\subsection{Formally}
\begin{mathpar}
\inferrule{
  \tododefine{reduce\_expr}(\tenv, \ve) \typearrow \veone\\
  \tododefine{use\_e}(\veone) \typearrow \useset\\
  \id\in\useset: \storageispure(\tenv, \id) \typearrow \vb_\id\\
  \vb \eqdef \bigwedge_{\id\in\useset} \vb_\id\\
  \checktrans{\vb}{NotStaticallyEvaluable} \checktransarrow \True \OrTypeError
}
{
  \checkstaticallyevaluable(\tenv, \ve) \typearrow \True
}
\end{mathpar}

\section{TypingRule.ToWellConstrained}
\hypertarget{def-towellconstrained}{}
The function
\[
  \towellconstrained(\overname{\ty}{\vt}) \aslto \overname{\ty}{\vtp}
\]
returns the \wellconstrainedversion\ of a type $\vt$ --- $\vtp$, which is defined as follows.

One of the following applies:
\begin{itemize}
  \item All of the following apply (\textsc{t\_int\_underconstrained}):
  \begin{itemize}
    \item $\vt$ is an underconstrained integer for the variable $\vv$;
    \item $\vtp$ is the well-constrained integer constrained by the variable expression for $\vv$,
    that is, $\TInt(\wellconstrained(\constraintexact(\EVar(\vv))))$.
  \end{itemize}

  \item All of the following apply (\textsc{t\_int\_other, other}):
  \begin{itemize}
    \item $\vt$ is not an underconstrained integer for the variable $\vv$;
    \item $\vtp$ is $\vt$.
  \end{itemize}
\end{itemize}

\begin{emptyformal}
\subsection{Formally}
\begin{mathpar}
\inferrule[t\_int\_underconstrained]{}
{
  \towellconstrained(\TInt(\underconstrained(\vv))) \typearrow\\ \TInt(\wellconstrained(\constraintexact(\EVar(\vv))))
}
\and
\inferrule[t\_int\_other]{
  \astlabel(\vi) \neq \underconstrained
}
{
  \towellconstrained(\TInt(\vi)) \typearrow \vt
}
\and
\inferrule[other]{
  \astlabel(\vt) \neq \TInt
}
{
  \towellconstrained(\vt) \typearrow \vt
}
\end{mathpar}
\end{emptyformal}

\section{TypingRule.GetWellConstrainedStructure}
\hypertarget{def-getwellconstrainedstructure}{}
The function
\[
  \getwellconstrainedstructure(\overname{\staticenvs}{\tenv} \aslsep \overname{\ty}{\vt})
  \aslto \overname{\ty}{\vtp}
\]
returns the \wellconstrainedstructure\ of a type $\vt$ in a static environment $\tenv$ --- $\vtp$, which is defined as follows.

\subsection{Prose}
All of the following apply:
\begin{itemize}
  \item the \structure\ of $\vt$ in $\tenv$ is $\vtone$ \ProseOrTypeError;
  \item the well-constrained version of $\vtone$ is $\vtp$.
\end{itemize}

\begin{emptyformal}
\subsection{Formally}
\begin{mathpar}
\inferrule{
  \tstruct(\tenv, \vt) \typearrow \vtone \OrTypeError\\\\
  \towellconstrained(\vtone) \typearrow \vtp
}
{
  \getwellconstrainedstructure(\tenv, \vt) \typearrow \vtp
}
\end{mathpar}
\end{emptyformal}

\section{TypingRule.GetBitvectorWidth}
\hypertarget{def-getbitvectorwidth}{}
The function
\[
  \getbitvectorwidth(\overname{\staticenvs}{\tenv} \aslsep \overname{\ty}{\vt}) \aslto
  \overname{\expr}{\ve} \cup \TTypeError
\]
returns the expression $\ve$, which represents the width of the bitvector type $\vt$,
or a type error if $\vt$ is not a bitvector type or another type error is detected.

\subsection{Prose}
One of the following applies:
\begin{itemize}
  \item All of the following apply (\textsc{okay}):
  \begin{itemize}
    \item obtaining the \structure\ of $\vt$ in $\tenv$ yields a bitvector type with width expression $\ve$,
          that is, $\TBits(\ve, \Ignore)$ \ProseOrTypeError;
    \item the result is $\ve$.
  \end{itemize}

  \item All of the following apply (\textsc{error}):
  \begin{itemize}
    \item obtaining the \structure\ of $\vt$ in $\tenv$ yields a type that is not a bitvector type;
    \item the result is a type error indicating that a bitvector type was expected.
  \end{itemize}
\end{itemize}

\begin{emptyformal}
\subsection{Formally}
\begin{mathpar}
\inferrule[okay]{
  \tstruct(\tenv, \vt) \typearrow \TBits(\ve, \Ignore) \OrTypeError
}{
  \getbitvectorwidth(\tenv, \vt) \typearrow \ve
}
\and
\inferrule[error]{
  \tstruct(\tenv, \vt) \typearrow \vtp\\
  \astlabel(\vtp) \neq \TBits
}{
  \getbitvectorwidth(\tenv, \vt) \typearrow \TypeErrorVal{ExpectedBitvectorType}
}
\end{mathpar}
\end{emptyformal}

\section{TypingRule.CheckBitsEqualWidth}
\hypertarget{def-checkbitsequalwidth}{}
The function
\[
  \checkbitsequalwidth(
    \overname{\staticenvs}{\tenv} \aslsep
    \overname{\ty}{\vtone}) \aslsep
    \overname{\ty}{\vttwo})\aslto
  \True \cup \TTypeError
\]
tests whether the types $\vtone$ and $\vttwo$ are bitvector types of the same width.
If the answer is positive, the result is $\True$. Otherwise, the result is a type error.

\subsection{Prose}
All of the following apply:
\begin{itemize}
  \item obtaining the width of $\vtone$ in $\tenv$ (via $\getbitvectorwidth$) yields the expression $\vn$ \ProseOrTypeError;
  \item obtaining the width of $\vttwo$ in $\tenv$ (via $\getbitvectorwidth$) yields the expression $\vm$ \ProseOrTypeError;
  \item One of the following applies:
  \begin{itemize}
    \item All of the following apply (\textsc{true}):
    \begin{itemize}
      \item symbolically checking whether the bitwidth expressions $\vn$ and $\vm$ are equal (via $\bitwidthequal$) yields $\True$;
      \item the result is $\True$.
    \end{itemize}

    \item All of the following apply (\textsc{error}):
    \begin{itemize}
      \item symbolically checking whether the bitwidth expressions $\vn$ and $\vm$ are equal (via $\bitwidthequal$) yields $\False$;
      \item the result is a type error indicating that the bitwidths are different.
    \end{itemize}
  \end{itemize}
\end{itemize}

\begin{emptyformal}
\subsection{Formally}
\begin{mathpar}
\inferrule[true]{
  \getbitvectorwidth(\tenv, \vtone) \typearrow \vn \OrTypeError\\
  \getbitvectorwidth(\tenv, \vttwo) \typearrow \vm \OrTypeError\\
  \bitwidthequal(\tenv, \vn, \vm) \typearrow \True
}{
  \checkbitsequalwidth(\tenv, \vtone, \vttwo) \typearrow \True
}
\and
\inferrule[error]{
  \getbitvectorwidth(\tenv, \vtone) \typearrow \vn \OrTypeError\\
  \getbitvectorwidth(\tenv, \vttwo) \typearrow \vm \OrTypeError\\
  \bitwidthequal(\tenv, \vn, \vm) \typearrow \False
}{
  \checkbitsequalwidth(\tenv, \vtone, \vttwo) \typearrow \TypeErrorVal{DifferentBitwidths}
}
\end{mathpar}
\end{emptyformal}

\section{AssocOpt}
\hypertarget{def-assocopt}{}
The function
\[
  \assocopt(\overname{(\identifier\times T)^*}{\vli} \aslsep \overname{\identifier}{\id}) \typearrow \langle \overname{T}{\vv} \rangle
\]
returns the value $\vv$ associated with the identifier $\id$ in the list of pairs $\vli$ or $\None$, if no such association exists.

\subsection{Prose}
One of the following applies:
\begin{itemize}
  \item All of the following apply (\textsc{member}):
  \begin{itemize}
    \item a pair $(\id,\vv)$ exists in the list $\vli$;
    \item the result is $\langle\vv\rangle$.
  \end{itemize}

  \item All of the following apply (\textsc{not\_member}):
  \begin{itemize}
    \item every pair $(\vx,\Ignore)$ in the list $\vli$ has $\vx\neq\id$;
    \item the result is $\None$.
  \end{itemize}
\end{itemize}

\begin{emptyformal}
\subsection{Formally}
\begin{mathpar}
\inferrule[not\_member]{
  (\vx, \vv) \in \vli: \vx \neq \id
}{
  \assocopt(\vli, \id) \typearrow \None
}
\and
\inferrule[member]{
  (\id, \vv) \in \vli
}{
  \assocopt(\vli, \id) \typearrow \langle \vv \rangle
}
\end{mathpar}
\end{emptyformal}

\section{LookupConstant}
\hypertarget{def-lookupconstant}{}
The function
\[
  \lookupconstant(\overname{\staticenvs}{\tenv} \aslsep \overname{\identifier}{\vs})
  \;\aslto\; \overname{\literal}{\vv}\ \cup\ \{\bot\}
\]
looks up the environment $\tenv$ for a constant $\vv$ associated with an identifier
$\vs$. The result is $\bot$ if $\vs$ is not associated with any constant.

\subsection{Prose}
One of the following applies:
\begin{itemize}
  \item All of the following apply (\textsc{local}):
  \begin{itemize}
    \item $\vs$ is associated with a constant $\vv$ in the local environment of $\tenv$;
  \end{itemize}

  \item All of the following apply (\textsc{global}):
  \begin{itemize}
    \item $\vs$ is not associated with a constant in the local environment of $\tenv$;
    \item $\vs$ is associated with a constant $\vv$ in the global environment of $\tenv$;
  \end{itemize}

  \item All of the following apply (\textsc{global}):
  \begin{itemize}
    \item $\vs$ is not associated with a constant in the local environment of $\tenv$;
    \item $\vs$ is not associated with a constant in the global environment of $\tenv$;
    \item the result is $\bot$.
  \end{itemize}
\end{itemize}

\begin{emptyformal}
\subsection{Formally}
\begin{mathpar}
\inferrule[local]{
  L^\tenv.\constantvalues(\vs) = \vv
}{
  \lookupconstant(\tenv, \vs) \typearrow \vv
}
\and
\inferrule[global]{
  L^\tenv.\constantvalues(\vs) = \bot\\
  G^\tenv.\constantvalues(\vs) = \vv
}{
  \lookupconstant(\tenv, \vs) \typearrow \vv
}
\and
\inferrule[none]{
  L^\tenv.\constantvalues(\vs) = \bot\\
  G^\tenv.\constantvalues(\vs) = \bot
}{
  \lookupconstant(\tenv, \vs) \typearrow \bot
}
\end{mathpar}
\end{emptyformal}

\section{TypeOf}
\hypertarget{def-typeof}{}
The function
\[
  \typeof(\overname{\staticenvs}{\tenv} \aslsep \overname{\identifier}{\vs})
  \;\aslto\; \overname{\ty}{\tty}\ \cup\ \overname{\TTypeError}{\TypeErrorConfig}
\]
looks up the environment $\tenv$ for a type $\tty$ associated with an identifier
$\vs$. The result is type error if $\vs$ is not associated with any type.

\subsection{Prose}
One of the following applies:
\begin{itemize}
  \item All of the following apply (\textsc{local}):
  \begin{itemize}
    \item $\vs$ is associated with a type $\tty$ in the local environment of $\tenv$;
  \end{itemize}

  \item All of the following apply (\textsc{global}):
  \begin{itemize}
    \item $\vs$ is not associated with a type in the local environment of $\tenv$;
    \item $\vs$ is associated with a type $\tty$ in the global environment of $\tenv$;
  \end{itemize}

  \item All of the following apply (\textsc{error}):
  \begin{itemize}
    \item $\vs$ is not associated with a type in the local environment of $\tenv$;
    \item $\vs$ is not associated with a type in the global environment of $\tenv$;
    \item the result is a type error indicating that $\vs$ was expected to be associated
          with a type.
  \end{itemize}
\end{itemize}

\begin{emptyformal}
\subsection{Formally}
\begin{mathpar}
\inferrule[local]{
  L^\tenv.\localstoragetypes(\vs) = \tty
}{
  \typeof(\tenv, \vs) \typearrow \tty
}
\and
\inferrule[global]{
  L^\tenv.\localstoragetypes(\vs) = \bot\\
  G^\tenv.\globalstoragetypes(\vs) = \tty
}{
  \typeof(\tenv, \vs) \typearrow \tty
}
\and
\inferrule[none]{
  L^\tenv.\localstoragetypes(\vs) = \bot\\
  G^\tenv.\globalstoragetypes(\vs) = \bot
}{
  \typeof(\tenv, \vs) \typearrow \TypeErrorVal{UndefinedIdentifier}
}
\end{mathpar}
\end{emptyformal}
%%%%%%%%%%%%%%%%%%%%%%%%%%%%%%%%%%%%%%%%%%%%%%%%%%%%%%%%%%%%%%%%%%%%%%%%%%%%%%%%%%%%
\bibliographystyle{plain}
\bibliography{ASL}

\end{document}
