\chapter{Introduction\label{chap:Introduction}}

This reference defines Arm’s Architecture Specification Language (ASL), which is the language
used in Arm’s architecture reference manuals to describe the Arm architecture.

ASL is designed and used to specify architectures. As a specification language, it is designed to be accessible,
understandable, and unambiguous to programmers, hardware
engineers, and hardware verification engineers, who collectively have quite a small intersection of languages they
all understand. It can intentionally under specify behaviors in the architecture being described.

ASL is:
\begin{itemize}
    \item a first-order language with strong static type-checking.
    \item whitespace-insensitive.
    \item imperative.
\end{itemize}

ASL has support for:
\begin{itemize}
    \item bitvectors:
    \begin{itemize}
        \item as a type.
        \item as a literal constant.
        \item bitvector concatenation.
        \item bitvector constants with wildcards.
        \item bitslices.
        \item dependent types to support function overloading using bitvector lengths.
        \item dependent types to reason about lengths of bitvectors.
    \end{itemize}
    \item unbounded arithmetic types “integer” and “real”.
    \item exceptions.
    \item enumerations.
    \item arrays.
    \item records.
    \item call-by-value.
    \item type inference.
\end{itemize}

ASL does not have support for:
\begin{itemize}
    \item references or pointers.
    \item macros.
    \item templates.
    \item virtual functions.
\end{itemize}

A \emph{specification} consists of a self-contained collection of ASL code.
\lrmcomment{\identd{GVBK}}
More specifically, a specification is the set of declarations written in ASL code which describe an architecture.

\section{Example Specification 1}
\figref{spec1} shows a small example of a specification written in ASL. It consists of the following declarations:
\begin{itemize}
    \item Global bitvectors \texttt{R0}, \texttt{R1}, and \texttt{R2} representing the state of the system.
    \item A function \texttt{MyOR} demonstrating a simple bit-wise OR function of 2 bitvectors.
    \item Initialization of \texttt{R0} and \texttt{R1} bitvectors.
    \item Assignment of bitvector \texttt{R2} with the result of a function call.
\end{itemize}

\begin{center}
\lstinputlisting[caption=Example specification 1\label{fi:spec1}]{\definitiontests/spec1.asl}
\end{center}

\section{Example Specification 2}
\figref{spec2} shows a small example of a specification written in ASL. It consists of the following declarations:
\begin{itemize}
\item A global variable \texttt{COUNT} representing the state of the system.
\item A procedure \texttt{ColdReset} to initialize the state of the system when power is applied and the system is reset.
    This interpretation of the function is a convention used in this particular specification. It is up to each
    specification to decide the role of each function.
\item A procedure \texttt{Step} to advance the state of the system. That is, it defines the \emph{transition relation} of the system.
    Again, this interpretation is a convention used in this particular specification, not part of the ASL language
    itself.
\end{itemize}

\begin{center}
\lstinputlisting[caption=Example specification 2\label{fi:spec2}]{\definitiontests/spec2.asl}
\end{center}

\section{Example Specification 3}
\figref{spec3} shows a small example of a specification in ASL. It consists of the following declarations:
\begin{itemize}
    \item A function \texttt{Dot8} which operates on 2 bitvectors a byte at a time.
    \item A global variable \texttt{COUNT} to indicate the number of calls to the \texttt{Fib} function.
    \item A function \texttt{Fib} demonstrating recursion with a bound of 1000 on its depth.
    \item Assignment of a global bitvector \texttt{X} with a call to the \texttt{Dot8} function.
    \item Assignment of a variable from the result of a call to the recursive function \texttt{Fib}.
    \item A function \texttt{main}.
\end{itemize}

\begin{center}
\lstinputlisting[caption=Example specification 3\label{fi:spec3}]{\definitiontests/spec3.asl}
\end{center}

The ASL type system and its semantics are defined in terms of
the ASL abstract syntax (\chapref{AbstractSyntax}).
Familiarity with the AST is \underline{required} to understand both.
The mathematical background needed to understand the formalization
of the ASL type system and ASL semantics appears in \chapref{FormalSystem},
\chapref{TypeChecking}, and \chapref{Semantics}.

% \section{The Language Pipeline}
% \begin{description}
% \item[Lexical Analysis] The text is first transformed into a list of \emph{tokens}.
%         This stage is defined in \secref{lexicalanalysis};
% \item[Parsing] The list of tokens is transformed into a \emph{parse tree}.
%         This stage is defined in \secref{parsing};
% \item[Abstraction] The parse tree is transformed into an abstract syntax tree. This is a conceptual stage. In actuality,
%         the parsing stage transforms the list of tokens directly into an abstract syntax tree. However, it is useful to
%         distinguish between the parsing state and the abstraction stage.
%         ASL abstract syntax trees are defined in \chapref{ASLAbstractSyntax}.
%         This stage is defined in \chapref{BuildingAbstractSyntaxTrees}.
% \end{description}
