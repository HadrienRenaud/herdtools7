\chapter{Assignable Expressions\label{chap:AssignableExpressions}}

\hypertarget{def-assignableexpression}{}
\hypertarget{def-assignableexpressions}{}
We refer to expressions that may appear on the left hand side of an assignment statement as \assignableexpressions.
An \assignableexpression\ is grammatically derived from $\Nlexpr$ and is represented as an AST
by $\lexpr$.

We show the syntax relevant to \assignableexpressions\ in \secref{AssignableExpressionsSyntax} and
the rules need to build the AST for \assignableexpressions\ in \secref{AssignableExpressionsAbstractSyntaxBuilders}.
We then define the abstract syntax, typing, and semantics of the different kinds of \assignableexpressions:
\begin{itemize}
\item Discarding assignment expressions (see \secref{DiscardingAssignmentExpressions})
\item Variable assignment expressions (see \secref{VariableAssignmentExpressions})
\item Multi-assignment expressions (see \secref{MultiAssignmentExpressions})
\item Array assignment expressions (see \secref{ArrayAssignmentExpressions})
\item Bitvector slice assignment expressions (see \secref{BitvectorSliceAssignmentExpressions})
\item Bitfield assignment expressions (see \secref{BitfieldAssignmentExpressions})
\item Structured type field assignment Expressions (see \secref{StructuredTypeFieldAssignmentExpressions})
\item Multi-slice assignment expressions (see \secref{MultiSliceAssignmentExpressions})
\end{itemize}

\hypertarget{def-annotatelexpr}{}
The function
\[
  \annotatelexpr{
    \overname{\staticenvs}{\tenv} \aslsep
    \overname{\lexpr}{\vle} \aslsep
    \overname{\ty}{\vte}} \aslto
    \overname{\lexpr}{\newle} \cup \TTypeError
\]
annotates a left-hand side expression $\vle$ in an environment $\tenv$, assuming $\vte$
to be the type of the corresponding right-hand-side expression,
resulting in an annotated expression $\newle$.
\ProseOtherwiseTypeError

The relation
\hypertarget{def-evallexpr}{}
\[
  \evallexpr{\overname{\envs}{\env} \aslsep \overname{\lexpr}{\vle} \aslsep (\overname{\vals}{\vv} \times \overname{\XGraphs}{\vg})} \;\aslrel\;
    \Normal(\overname{\XGraphs}{\newg},\overname{\envs}{\newenv}) \cup
    \overname{\TThrowing}{\ThrowingConfig} \cup \overname{\TDynError}{\DynErrorConfig}
\]
evaluates the assignment of a value $\vv$
to the left-hand-side expression $\vle$ in an environment $\env$,
resulting in either a configuration $\Normal(\newg, \env)$ or an abnormal configuration.

\paragraph{Semantics Rules Naming Convention:}
In this chapter, variables containing $\vm$ range over $\vals\times\XGraphs$
while variables where the $\vm$ is replaced with $\vv$ correspond to their value component.
For example, $\rmarray\eqname(\rvarray,\vgtwo)$ and $\mindex\eqname(\vindex, \vgone)$.

\paragraph{Viewing Assignable Expressions as Right-hand-side Expressions:}
Some of the typing rules and semantics rules require viewing \assignableexpressions\
as \rhsexpressions.
The correspondence is given by the function $\torexpr : \lexpr \rightarrow \expr$., defined in \secref{LeftToRight}.
%
For example, \hyperlink{SemanticsRule.LESetField}{SemanticsRule.LESetField}
needs to evaluate the record subexpression $\rerecord$, which is an \assignableexpression.
To achieve this, $\torexpr(\record)$ is used to obtain an \rhsexpression, which then allows
using $\texttt{eval\_expr}$ to evaluate it.

\section{Syntax\label{sec:AssignableExpressionsSyntax}}
\begin{flalign*}
\Nlexpr \derives\ & \Nlexpratom &\\
|\ & \Tminus &\\
|\ & \Tlpar \parsesep \Clisttwo{\Nlexpr} \parsesep \Trpar &\\
\Nlexpratom \derives\ & \Tidentifier &\\
|\ & \Nlexpratom \parsesep \Nslices &\\
|\ & \Nlexpratom \parsesep \Tdot \parsesep \Tidentifier{\field} &\\
|\ & \Nlexpratom \parsesep \Tdot \parsesep \Tlbracket \parsesep \Clist{{\Tidentifier}} \parsesep \Trbracket &\\
|\ & \Tlbracket \parsesep \NClist{{\Nlexpratom}} \parsesep \Trbracket &
\end{flalign*}

\subsection{Abstract Syntax Builders\label{sec:AssignableExpressionsAbstractSyntaxBuilders}}
\subsubsection{ASTRule.LExpr \label{sec:ASTRule.LExpr}}
\hypertarget{build-lexpr}{}
The function
\[
  \buildlexpr(\overname{\parsenode{\Nlexpr}}{\vparsednode}) \;\aslto\; \overname{\lexpr}{\vastnode}
\]
transforms a parse node $\vparsednode$ into an AST node $\vastnode$.

\begin{mathpar}
\inferrule[lexpr\_atom]{}{
  \buildlexpr(\Nlexpr(\punnode{\Nlexpratom})) \astarrow \overname{\astof{\vlexpratom}}{\vastnode}
}
\end{mathpar}

\begin{mathpar}
\inferrule[discard]{}{
  \buildlexpr(\Nlexpr(\Tminus)) \astarrow \overname{\LEDiscard}{\vastnode}
}
\end{mathpar}

\begin{mathpar}
\inferrule[multi\_lexpr]{
  \buildclist[\Nlexpr](\vlexprs) \astarrow \vlexprasts
}{
  \buildlexpr(\Nlexpr(\Tlpar, \namednode{\vlexprs}{\Clisttwo{\Nlexpr}}, \Trpar)) \astarrow
  \overname{\LEDestructuring(\vlexprasts)}{\vastnode}
}
\end{mathpar}

\subsubsection{ASTRule.LExprAtom \label{sec:ASTRule.LExprAtom}}
\hypertarget{build-lexpratom}{}
The function
\[
  \buildlexpratom(\overname{\parsenode{\Nlexpratom}}{\vparsednode}) \;\aslto\; \overname{\lexpr}{\vastnode}
\]
transforms a parse node $\vparsednode$ into an AST node $\vastnode$.

\begin{mathpar}
\inferrule[var]{}{
  \buildlexpratom(\Nlexpr(\Tidentifier(\id))) \astarrow
  \overname{\LEVar(\id)}{\vastnode}
}
\end{mathpar}

\begin{mathpar}
\inferrule[slice]{}{
  \buildlexpratom(\Nlexpr(\punnode{\Nlexpratom}, \punnode{\Nslices})) \astarrow
  \overname{\LESlice(\astof{\vlexpratom}, \astof{\vslices})}{\vastnode}
}
\end{mathpar}

\begin{mathpar}
\inferrule[set\_field]{}{
  \buildlexpratom(\Nlexpr(\punnode{\Nlexpratom}, \Tdot, \Tidentifier(\id))) \astarrow
  \overname{\LESetField(\astof{\vlexpratom}, \id)}{\vastnode}
}
\end{mathpar}

\begin{mathpar}
\inferrule[set\_fields]{
  \buildclist[\buildidentity](\vfields) \astarrow \vfieldasts
}{
  {
  \begin{array}{r}
  \buildlexpratom(\Nlexpr(\punnode{\Nlexpratom}, \Tdot, \Tlbracket, \namednode{\vfields}{\Clist{\Tidentifier}}, \Trbracket)) \astarrow\\
  \overname{\LESetFields(\astof{\vlexpratom}, \vfieldasts)}{\vastnode}
  \end{array}
  }
}
\end{mathpar}

\begin{mathpar}
\inferrule[concat]{
  \buildclist[\buildlexpratom](\vlexprs) \astarrow \vlexprasts
}{
  {
    \begin{array}{r}
  \buildlexpratom(\Nlexpr(\Tlbracket, \namednode{\vlexprs}{\NClist{{\Nlexpratom}}}, \Trbracket)) \astarrow\\
  \overname{\LEConcat(\vlexprasts)}{\vastnode}
    \end{array}
  }
}
\end{mathpar}

\section{Discarding Assignment Expressions\label{sec:DiscardingAssignmentExpressions}}
\subsection{Abstract Syntax}
\begin{flalign*}
\lexpr \derives\ & \overtext{\LEDiscard}{\texttt{"-"}} &
\end{flalign*}

\subsection{Typing}
\subsubsection{TypingRule.LEDiscard\label{sec:TypingRule.LEDiscard}}
\subsubsection{Prose}
All of the following apply:
\begin{itemize}
  \item $\vle$ denotes an expression that can be discarded, that is, $\LEDiscard$;
  \item $\newle$ is $\vle$.
\end{itemize}\CodeSubsection{\LEDiscardBegin}{\LEDiscardEnd}{../Typing.ml}
\subsubsection{Formally}
\begin{mathpar}
\inferrule{}{
  \annotatelexpr{\tenv, \overname{\LEDiscard}{\vle}, \vte} \typearrow \overname{\LEDiscard}{\newle}
}
\end{mathpar}

\subsection{Semantics}
\subsubsection{SemanticsRule.LEDiscard\label{sec:SemanticsRule.LEDiscard}}
\subsubsection{Example}
In the specification:
\VerbatimInput{../tests/ASLSemanticsReference.t/SemanticsRule.LEDiscard.asl}
\texttt{- = 42;} does not affect the environment.

\subsubsection{Prose}
All of the following apply:
\begin{itemize}
  \item $\vle$ is a discarding expression, $\LEDiscard$;
  \item $\newg$ is $\vg$;
  \item $\newenv$ is $\env$.
\end{itemize}
\subsubsection{Formally}
\begin{mathpar}
\inferrule{
  \newg\eqdef\vg\\
  \newenv\eqdef\env
}{
  \evallexpr{\env, \LEDiscard, (\vv, \vg)} \evalarrow \Normal(\newg, \newenv)
}
\end{mathpar}
\CodeSubsection{\EvalLEDiscardBegin}{\EvalLEDiscardEnd}{../Interpreter.ml}

\section{Variable Assignment Expressions\label{sec:VariableAssignmentExpressions}}
\subsection{Abstract Syntax}
\begin{flalign*}
\lexpr \derives\ & \LEVar(\identifier) &
\end{flalign*}

\subsection{Typing}
\subsubsection{TypingRule.LEVar\label{sec:TypingRule.LEVar}}
\subsubsection{Prose}
All of the following apply:
\begin{itemize}
  \item $\vle$ denotes a left-hand-side variable expression for $\vx$, that is, $\LEVar(\vx)$;
  \item One of the following applies (\textsc{local}):
  \begin{itemize}
    \item $\vx$ is declared in $\tenv$ as a local storage element with type $\tty$ and local declaration keyword $k$;
    \item checking that $k$ corresponds to a mutable variable, that is, $\LDKVar$, yields $\True$\ProseTerminateAs{\AssignmentToImmutable};
    \item determining whether $\tty$ \typesatisfies\ $\vte$ in $\tenv$ yields $\True$\ProseOrTypeError;
    \item $\newle$ is $\vle$.
  \end{itemize}

  \item One of the following applies (\textsc{global}):
  \begin{itemize}
    \item $\vx$ is declared in $\tenv$ as a global storage element with type $\tty$ and global declaration keyword $k$;
    \item checking that $k$ corresponds to a mutable variable, that is, $\GDKVar$, yields $\True$\ProseTerminateAs{\AssignmentToImmutable};
    \item determining whether $\tty$ \typesatisfies\ $\vte$ in $\tenv$ yields $\True$\ProseOrTypeError;
    \item $\newle$ is $\vle$.
  \end{itemize}

  \item One of the following applies (\textsc{error\_undefined}):
  \begin{itemize}
    \item $\vx$ is not declared in $\tenv$ as a local storage element nor as a global storage element;
    \item the result is a type error $\UndefinedIdentifier$.
  \end{itemize}
\end{itemize}
\subsubsection{Formally}
\begin{mathpar}
\inferrule[local]{
  L^\tenv.\localstoragetypes(\id) = (\tty, k)\\
  \checktrans{k = \LDKVar}{\AssignmentToImmutable} \checktransarrow \True \OrTypeError\\\\
  \checktypesat(\tenv, \vte, \tty) \typearrow \True \OrTypeError
}{
  \annotatelexpr{\tenv, \overname{\LEVar(\vx)}{\vle}, \vte} \typearrow \overname{\vle}{\newle}
}
\end{mathpar}

\begin{mathpar}
\inferrule[global]{
  L^\tenv.\globalstoragetypes(\id) = (\tty, k)\\
  \checktrans{k = \GDKVar}{AssignToImmutable} \checktransarrow \True \OrTypeError\\\\
  \checktypesat(\tenv, \vte, \tty) \typearrow \True \OrTypeError
}{
  \annotatelexpr{\tenv, \overname{\LEVar(\vx)}{\vle}, \vte} \typearrow \overname{\vle}{\newle}
}
\end{mathpar}

\begin{mathpar}
\inferrule[error\_undefined]{
  L^\tenv.\localstoragetypes(\id) = \bot\\
  L^\tenv.\globalstoragetypes(\id) = \bot
}{
  \annotatelexpr{\tenv, \overname{\LEVar(\vx)}{\vle}, \vte} \typearrow \TypeErrorVal{\UndefinedIdentifier}
}
\end{mathpar}\lrmcomment{
  This is related to \identr{WDGQ}, \identr{GNTS}, \identi{MMKF},
  \identi{DGWJ}, \identi{KKCC} and \identr{LXQZ}.
}
\CodeSubsection{\LEVarBegin}{\LEVarEnd}{../Typing.ml}

\subsection{Semantics}
\subsubsection{SemanticsRule.LEVar\label{sec:SemanticsRule.LEVar}}
\subsubsection{Example (Local Variable)}
In the specification:
\VerbatimInput{../tests/ASLSemanticsReference.t/SemanticsRule.LELocalVar.asl}
SemanticsRule.LELocalVar is (only) used to assign the value $42$ to the left-hand-side expression
\texttt{x} within \texttt{x = 42;}.

\subsubsection{Example (Global Variable)}
In the specification:
\VerbatimInput{../tests/ASLSemanticsReference.t/SemanticsRule.LEGlobalVar.asl}
SemanticsRule.LEGlobalVar is (only) used to assign the value $42$ to the left-hand-side expression
\texttt{x} within \texttt{x = 42;}.

\subsubsection{Prose}
All of the following apply:
\begin{itemize}
    \item $\vle$ denotes a variable, $\LEVar(\vx)$;
    \item One of the following applies:
    \begin{itemize}
        \item All of the following apply (\textsc{local}):
        \begin{itemize}
            \item $\vx$ is locally bound in $\env$;
            \item $\newenv$ is $\env$ where $\vx$ is bound to $\vv$ in the local component of the environment.
        \end{itemize}

        \item All of the following apply (\textsc{global}):
        \begin{itemize}
            \item $\vx$ is globally bound in $\env$;
            \item $\newenv$ is $\env$ where $\vx$ is bound to $\vv$ in the global component of the environment.
        \end{itemize}
    \end{itemize}
    \item $\newg$ is the ordered composition of $\vg$ and a Write Effect for $\vx$ with the $\asldata$ edge;
\end{itemize}

\subsubsection{Formally}
\begin{mathpar}
\inferrule[local]{
  \env \eqname (\tenv, \denv)\\
  \vx \in \dom(L^\denv)\\
  \newenv \eqdef (\tenv, (G^\denv, L^\denv[\vx \mapsto \vv]))\\
  \newg \eqdef \ordered{\vg}{\asldata}{\WriteEffect(\vx)}
}{
  \evallexpr{\env, \LEVar(\vx), (\vv, \vg)} \evalarrow \Normal(\newg,\newenv)
}
\end{mathpar}

\begin{mathpar}
\inferrule[global]{
  \env \eqname (\tenv, \denv)\\
  \vx \in \dom(G^\denv)\\
  \newenv \eqdef(\tenv, (G^\denv[\vx \mapsto \vv], L^\denv))\\
  \newg \eqdef \ordered{\vg}{\asldata}{\WriteEffect(\vx)}
}{
  \evallexpr{\env, \LEVar(\vx), (\vv, \vg)} \evalarrow \Normal(\newg,\newenv)
}
\end{mathpar}
\CodeSubsection{\EvalLEVarBegin}{\EvalLEVarEnd}{../Interpreter.ml}

\section{Multi-assignment Expressions\label{sec:MultiAssignmentExpressions}}
\subsection{Abstract Syntax}
\begin{flalign*}
\lexpr \derives\ & \LEDestructuring(\lexpr^*) &
\end{flalign*}

\subsection{Typing}
\subsubsection{TypingRule.LEDestructuring\label{sec:TypingRule.LEDestructuring}}
\subsubsection{Prose}
All of the following apply:
\begin{itemize}
  \item $\vle$ denotes a tuple of left-hand-side expressions $\les$, that is, $\LEDestructuring(\les)$;
  \item $\les$ is a list $\ve_{1..k}$;
  \item checking whether $\vte$ is a tuple type yields $\True$\ProseTerminateAs{\ExpectedTupleType};
  \item $\vte$ is a tuple type over the list of types $\tys$, that is, $\TTuple(\tys)$;
  \item determining whether $\les$ and $\subtys$ have the same length yields $\True$\ProseTerminateAs{\LengthsMismatch};
  \item $\subtys$ is the list of types $\vt_{1..k}$;
  \item annotating the left-hand-side expression $\ve_i$ with the type $\vt_i$, for $i=1..k$, yields $\vep_i$\ProseOrTypeError;
  \item the list of expressions $\lesp$ is $\vep_i$, for $i=1..k$;
  \item $\newle$ is the list of left-hand-side expressions $\lesp$, that is, $\LEDestructuring(\lesp)$.
\end{itemize}
\subsubsection{Formally}
\begin{mathpar}
\inferrule{
  \les \eqname [\ve_{1..k}]\\
  \checktrans{\astlabel(\vte) = \TTuple}{\ExpectedTupleType} \checktransarrow \True \OrTypeError\\\\
  \vte \eqname \TTuple(\tys)\\
  \equallength(\les, \tys) \typearrow \vb\\
  \checktrans{\vb}{\LengthsMismatch} \checktransarrow \True \OrTypeError\\\\
  \tys \eqname [\vt_{1..k}]\\
  i=1..k: \annotatelexpr{\tenv, \ve_i,\vt_i} \typearrow \vep_i \OrTypeError\\\\
  \lesp \eqname [i=1..k: \vep_i]
}{
  \annotatelexpr{\tenv, \overname{\LEDestructuring(\les)}{\vle}, \vte} \typearrow \overname{\LEDestructuring(\lesp)}{\newle}
}
\end{mathpar}
\CodeSubsection{\LEDestructuringBegin}{\LEDestructuringEnd}{../Typing.ml}

\subsection{Semantics}
\subsubsection{SemanticsRule.LEDestructuring\label{sec:SemanticsRule.LEDestructuring}}
\subsubsection{Example}
In the specification:
\VerbatimInput{../tests/ASLSemanticsReference.t/SemanticsRule.LEDestructuring.asl}
\texttt{(x, y) = (3, 42)} binds \texttt{x} to $\nvint(3)$ and \texttt{y} to
$\nvint(42)$ in the environment where \texttt{x} is bound to $\nvint(42)$ and \texttt{y} is bound to $\nvint(3)$.

\subsubsection{Prose}
All of the following apply:
\begin{itemize}
  \item $\vle$ denotes a list of left-hand-side expressions, $\LEDestructuring(\vlelist)$;
  \item $\vlelist$ is the list of expressions $\vle_{1..n}$;
  \item getting the values from the native vector $\vv$ at each index $i=1..n$
  results in $\vv_{i=1..n}$;
  \item $\nmonads$ is the list of pairs consisting of $\vv_i$ and $\vg$ for $i=1..n$;
  \item evaluating the multi-assignment between $\vlelist$ and the list $\nmonads$
  in $\env$ achieves the effects of assigning each value to the respective
  subexpressions, resulting in the output configuration $C$.
\end{itemize}
\subsubsection{Formally}
\begin{mathpar}
\inferrule{
  \vlelist \eqname [\vle_{1..n}]\\
  i=1..n: \getindex(i, \vv) \evalarrow \vv_i\\
  \nmonads \eqdef [i=1..n: (\vv_i, \vg)]\\
  \evalmultiassignment(\env, \vlelist, \nmonads) \evalarrow C
}{
  \evallexpr{\env, \LEDestructuring(\vlelist), (\vv, \vg)} \evalarrow C
}
\end{mathpar}
\CodeSubsection{\EvalLEDestructuringBegin}{\EvalLEDestructuringEnd}{../Interpreter.ml}

\subsubsection{SemanticsRule.LEMultiAssign\label{sec:SemanticsRuleLEMultiAssign}}
\subsubsection{Prose}
The helper relation
\hypertarget{def-evalmultiassign}{}
\[
  \evalmultiassignment(\overname{\envs}{\env} \aslsep \overname{\expr^*}{\vlelist} \aslsep \overname{(\vals \times \XGraphs)^*}{\vmlist}) \;\aslrel\;
  \Normal(\overname{\XGraphs}{\newg} \aslsep \overname{\envs}{\newenv}) \cup
  \overname{\TThrowing}{\ThrowingConfig} \cup \overname{\TDynError}{\DynErrorConfig}
\]
evaluates multi-assignments.
That is, the simultaneous assignment of the list of value-execution graph pairs $\vmlist$
to the corresponding list of left-hand side expressions $\vlelist$, in the environment $\env$.
The result is either the execution graph $\vg$ and new environment $\newenv$ or an abnormal configuration.

\subsubsection{Formally}
\begin{mathpar}
\inferrule[empty]{}
{
  \evalmultiassignment(\env, \emptylist, \emptylist) \evalarrow \Normal(\emptygraph, \env)
}
\and
\inferrule[nonempty]{
  \vlelist \eqname [\vle] \concat \vlelistone\\
  \vmlist \eqname [\vm] \concat \vmlistone\\
  \evallexpr{\env, \vle, \vm} \evalarrow \Normal(\envone, \vgone) \OrAbnormal\\
  \evalmultiassignment(\envone, \vlelistone, \vmlistone) \evalarrow \Normal(\newenv, \vgtwo) \OrAbnormal\\
  \newg \eqdef \ordered{\vgone}{\aslpo}{\vgtwo}
}{
  \evalmultiassignment(\env, \vlelist, \vmlist) \evalarrow \Normal(\newg, \newenv)
}
\end{mathpar}
Notice that this rule is only defined when the lists $\vlelist$ and $\vmlist$ have the same length.
To see this, notice that
to form a derivation tree, we must employ the \textsc{nonempty} case, which ensures both lists
have at least one element and shortens the lengths of both lists by one,
until both lists become empty
which is when the \textsc{empty} axiom case is used.

\section{Array Assignment Expressions\label{sec:ArrayAssignmentExpressions}}
\subsection{Abstract Syntax}
\begin{flalign*}
\lexpr \derives\ & \LESetArray(\lexpr, \expr) &
\end{flalign*}

\subsection{Typing}
\subsubsection{TypingRule.LESetArray\label{sec:TypingRule.LESetArray}}
\subsubsection{Prose}
All of the following apply:
\begin{itemize}
  \item $\vle$ denotes the slicing of a left-hand-side expression $\vleone$ by the slices $\slices$, that is, $\LESlice(\vleone, \slices)$;
  \item annotating the right-hand-side expression corresponding to $\vleone$ in $\tenv$ yields \\ $(\vtleone, \Ignore)$\ProseOrTypeError;
  \item obtaining the \structure\ of $\vtleone$ in $\tenv$ yields an array type of size $\size$ and element type $\vt$, that is, $\TArray(\size, \vt)$\ProseOrTypeError;
  \item annotating the left-hand-side expression $\vleone$ with type $\vtleone$ in $\tenv$ yields $\vletwo$\ProseOrTypeError;
  \item determining that $\vte$ \typesatisfies\ $\vt$ in $\tenv$ yields $\True$\ProseOrTypeError;
  \item determining whether $\slices$ is a single slice with index expression $\eindex$ yields $\True$\ProseOrTypeError;
  \item annotating the index expression $\eindex$ in $\tenv$ yields $(\tindexp, \eindexp)$\ProseOrTypeError;
  \item determining the array length type of $\size$ in $\tenv$ (via $\typeofarraylength$) yields $\wantedtindex$;
  \item determining whether $\tindexp$ \typesatisfies\ $\wantedtindex$ in $\tenv$ yields \\
        $\True$\ProseOrTypeError;
  \item $\newle$ is an access to array $\vletwo$ at index $\eindexp$, that is, \\ $\LESetArray(\vletwo, \eindexp)$.
\end{itemize}
\subsubsection{Formally}
\begin{mathpar}
\inferrule{
  \annotateexpr{\tenv, \torexpr(\vleone)} \typearrow (\vtleone, \Ignore) \OrTypeError\\\\
  \tstruct(\tenv, \vtleone) \typearrow \TArray(\size, \vt) \OrTypeError\\\\
  \annotatelexpr{\tenv, \vleone, \vtleone} \typearrow \vletwo \OrTypeError\\\\
  \checktypesat(\tenv, \vte, \vt) \typearrow \True \OrTypeError\\\\
  \checktrans{\listlen{\slices} = 1}{ArraySliceShouldBeSingleIndex} \checktransarrow \True \OrTypeError\\\\
  \slices \eqname [\vs]\\
  \checktrans{\astlabel(\vs) = \SliceSingle}{ArraySliceShouldBeSingleIndex} \checktransarrow \True \OrTypeError\\\\
  \vs \eqname \SliceSingle(\eindex)\\
  \annotateexpr{\tenv, \eindex} \typearrow (\tindexp, \eindexp) \OrTypeError\\\\
  \typeofarraylength(\tenv, \size) \typearrow \wantedtindex\\
  \checktypesat(\tenv, \tindexp, \wantedtindex) \typearrow \True \OrTypeError\\\\
  \newle \eqdef \LESetArray(\vletwo, \eindexp)
}{
  \annotatelexpr{\tenv, \overname{\LESlice(\vleone, \slices)}{\vle}, \vte} \typearrow \newle
}
\end{mathpar}
\CodeSubsection{\LESetArrayBegin}{\LESetArrayEnd}{../Typing.ml}

\subsection{Semantics}
\subsubsection{SemanticsRule.LESetArray\label{sec:SemanticsRule.LESetArray}}
\subsubsection{Example}
The specification:
\VerbatimInput{../tests/ASLSemanticsReference.t/SemanticsRule.LESetArray.asl}
binds the third element of \texttt{my\_array} to the value \texttt{53}.

\subsubsection{Prose}
All of the following apply:
\begin{itemize}
  \item $\vle$ denotes an array update expression, $\LESetArray(\rearray, \eindex)$;
  \item evaluating the right-hand-side expression corresponding to $\rearray$ in $\env$
  is \Normal(\rmarray, \envone)\ProseOrAbnormal;
  \item evaluating $\eindex$ in $\envone$ is \Normal(\mindex, \envtwo)\ProseOrAbnormal;
  \item $\mindex$ consists of the native integer $\vindex$ and the execution graph $\vgone$;
  \item $\vindex$ is the native integer for $\vi$;
  \item $\rmarray$ consists of the native vector $\rvarray$ and the execution graph $\vgtwo$;
  \item setting the value $\vv$ at index $\vi$ of $\rvarray$ is the native vector $\vvone$;
  \item $\vmone$ is the pair consisting of $\vvone$ and the parallel composition of $\vgone$ and $\vgtwo$;
  \item the steps so far computed the updated array, but have not assigned it to the variable
  bound to the array given by $\rearray$, which is achieved next.
  Evaluating the left-hand-side expression $\rearray$ in an environment $\envtwo$ with $\vmone$
  is the output configuration $C$.
\end{itemize}
\subsubsection{Formally}
\begin{mathpar}
\inferrule{
  \evalexpr{\env, \torexpr(\rearray)} \evalarrow \Normal(\rmarray, \envone) \OrAbnormal\\
  \evalexpr{\envone, \eindex} \evalarrow \Normal(\mindex, \envtwo) \OrAbnormal\\
  \mindex \eqname (\vindex, \vgone)\\
  \vindex \eqname \nvint(\vi)\\
  \rmarray \eqname (\rvarray, \vgtwo)\\
  \setindex(\vi, \vv, \rvarray) \evalarrow \vvone\\
  \vmone \eqdef (\vvone, \vgone \parallelcomp \vgtwo)\\
  \evallexpr{\envtwo, \rearray, \vmone} \evalarrow C
}{
  \evallexpr{\env, \LESetArray(\rearray, \eindex), (\vv, \vg)} \evalarrow C
}
\end{mathpar}
\CodeSubsection{\EvalLESetArrayBegin}{\EvalLESetArrayEnd}{../Interpreter.ml}

\subsubsection{Comments}
\lrmcomment{This is related to \identr{WHRS}:}
If the declared type of the \rhsexpression\ of a setter has the structure of a
bitvector or a type with fields, then if a bitslice or field selection is
applied to a setter invocation, then the assignment to that bitslice is
implemented using the following Read-Modify-Write (RMW) behavior:
\begin{itemize}
\item invoking the getter of the same name as the setter, with the same actual
arguments as the setter invocation
\item performing the assignment to the bitslice or field of the result of the
getter invocation
\item invoking the setter to assign the resulting value
\end{itemize}

We note that the index is guaranteed by the type-checker to be within the array bounds
via \secref{TypingRule.LESetArray}.

\section{Bitvector Slice Assignment Expressions\label{sec:BitvectorSliceAssignmentExpressions}}
\subsection{Abstract Syntax}
\begin{flalign*}
\lexpr \derives\ & \LESlice(\lexpr, \slice^*) &
\end{flalign*}

\subsection{Typing}
\subsubsection{TypingRule.LESlice\label{sec:TypingRule.LESlice}}
\subsubsection{Prose}
All of the following apply:
\begin{itemize}
  \item $\vle$ denotes the slicing of a left-hand-side expression $\vleone$ by the slices $\slices$, that is, $\LESlice(\vleone, \slices)$;
  \item annotating the right-hand-side expression corresponding to $\vleone$ in $\tenv$ yields \\
        $(\vtleone, \Ignore)$\ProseOrTypeError;
  \item $\vtleone$ is a bitvector type;
  \item annotating the left-hand-side expression $\vleone$ in $\tenv$ yields $\vletwo$\ProseOrTypeError;
  \item obtaining the width of the slices $\slices$ in $\tenv$ and simplifying them yields $\vwidth$;
  \item $\vt$ is the bitvector type of width $\width$ and empty list of bitfields;
  \item checking whether $\vte$ \typesatisfies\ $\vt$ yields $\True$\ProseOrTypeError;
  \item annotating $\slices$ in $\tenv$ yields $\slicestwo$\ProseOrTypeError;
  \item checking that the slices $\slicestwo$ are all disjoint yields $\True$\ProseOrTypeError;
  \item checking that $\slices$ is not empty yields $\True$\ProseTerminateAs{\EmptySlice};
  \item $\newle$ is the slicing of $\vletwo$ by $\slicestwo$, that is, $\LESlice(\vletwo, \slicestwo)$.
\end{itemize}
\subsubsection{Formally}
\begin{mathpar}
\inferrule{
  \annotateexpr{\tenv, \torexpr(\vleone)} \typearrow (\vtleone, \Ignore)\\
  \tstruct(\tenv, \vtleone) \typearrow \structtleone \OrTypeError\\\\
  \astlabel(\structtleone) = \TBits\\
  \annotatelexpr{\tenv, \vleone, \vtleone} \typearrow \vletwo\\
  \sliceswidth(\tenv, \slices) \typearrow \widthp\\
  \normalize(\tenv, \widthp) \typearrow \vwidth\\
  \vt \eqdef \TBits(\vwidth, \emptylist)\\
  \checktypesat(\tenv, \vte, \vt) \typearrow \True \OrTypeError\\\\
  \annotateslices(\tenv, \slices) \typearrow \slicestwo \OrTypeError\\\\
  \checkdisjointslices(\tenv, \slicestwo) \typearrow \True \OrTypeError\\\\
  \checktrans{\slices \neq \emptylist}{\EmptySlice} \typearrow \True \OrTypeError\\\\
  \newle \eqdef \LESlice(\vletwo, \slicestwo)
}{
  \annotatelexpr{\tenv, \overname{\LESlice(\vleone, \slices)}{\vle}, \vte} \typearrow \newle
}
\end{mathpar}
\CodeSubsection{\LESliceBegin}{\LESliceEnd}{../Typing.ml}

\subsubsection{TypingRule.CheckDisjointSlices\label{sec:}TypingRule.CheckDisjointSlices}
\hypertarget{def-checkdisjointslices}{}
The function
\[
\checkdisjointslices(\overname{\staticenvs}{\tenv} \aslsep \overname{\slice^*}{\slices})
\aslto \True \cup \overname{\TTypeError}{\TypeErrorConfig}
\]
tests whether the list of slices $\slices$ do not overlap in $\tenv$, yielding $\True$.
\ProseOtherwiseTypeError

\subsubsection{Prose}
All of the following apply:
\begin{itemize}
  \item applying $\disjointslicestopositions$ to $\slices$ in $\tenv$ yields a set of positions\ProseOrTypeError.
  \item the result is $\True$.
\end{itemize}
\subsubsection{Formally}
\begin{mathpar}
\inferrule{
  \disjointslicestopositions(\tenv, \slices) \typearrow \positions \OrTypeError
}{
  \checkdisjointslices(\tenv, \slices) \typearrow \True
}
\end{mathpar}
\CodeSubsection{\CheckDisjointSlicesBegin}{\CheckDisjointSlicesEnd}{../Typing.ml}

\subsection{Semantics}
\subsubsection{SemanticsRule.LESlice\label{sec:SemanticsRule.LESlice}}
\subsubsection{Example}
In the specification:
\VerbatimInput{../tests/ASLSemanticsReference.t/SemanticsRule.LESlice.asl}
The assignment \texttt{x[3:0] = '0000'} binds \texttt{x} to $\nvbitvector(11110000)$
via the rule SemanticsRule.LESlice.asl
in the environment where \texttt{x} is bound to $\nvbitvector(11111111)$.

\subsubsection{Prose}
All of the following apply:
\begin{itemize}
  \item $\vle$ denotes a left-hand-side slicing expression, $\LESlice(\ebv, \slices)$;
  \item evaluating the right-hand-side expression that corresponds to $\ebv$
  (given by applying $\torexpr$ to $\ebv$) in $\env$
    is $\Normal(\mbv,\envone)$\ProseOrAbnormal;
  \item evaluating $\slices$ in $\envone$ is $\Normal(\msliceranges, \envtwo)$\ProseOrAbnormal;
  \item $\msliceranges$ consists of the execution graph $\vgone$ and the list of indices\\
         $\sliceranges$;
  \item applying $\checknonoverlappingslices$ to $\sliceranges$ yields $\True$\ProseOrError;
  \item $\mbv$ consists of the native bitvector $\vbv$ and the execution graph $\vgtwo$;
  \item writing to the bitvector $\vbv$ at indices $\sliceranges$ using the values from $\vv$
  results in the updated native bitvector $\vvone$\ProseOrError;
  \item $\vgthree$ is the parallel composition of $\vg$, $\vgone$, and $\vgtwo$;
  \item $\newmbv$ is a pair consisting of $\vvone$ and the execution graph $\vgthree$;
  \item the steps so far computed the updated bitvector, but have not assigned it to the
  variable bound to the bitvector given by $\ebv$, which is achieved next.
  Evaluating the left-hand-side expression $\ebv$ with
  $\newmbv$ in an environment $\envtwo$ is the output configuration $C$,
\end{itemize}
\subsubsection{Formally}
\begin{mathpar}
\inferrule{
  \evalexpr{\env, \torexpr(\ebv)} \evalarrow \Normal(\mbv,\envone) \OrAbnormal\\
  \evalslices(\envone, \slices) \evalarrow \Normal(\msliceranges, \envtwo) \OrAbnormal\\
  \msliceranges \eqname (\sliceranges, \vgone)\\
  \mbv \eqname (\vbv, \vgtwo)\\\\
  \checknonoverlappingslices(\sliceranges) \evalarrow \True \OrDynError\\\\
  \writetobitvector(\sliceranges, \vv, \vbv) \evalarrow \vvone \OrDynError\\\\
  \vgthree \eqdef \vg \parallelcomp \vgone \parallelcomp \vgtwo\\
  \newmbv \eqdef (\vvone, \vgthree)\\
  \evallexpr{\envtwo, \ebv, \newmbv} \evalarrow C
}{
  \evallexpr{\envtwo, \LESlice(\ebv, \slices), (\vv, \vg)} \evalarrow C
}
\end{mathpar}
\CodeSubsection{\EvalLESliceBegin}{\EvalLESliceEnd}{../Interpreter.ml}
\subsubsection{Comments}
\lrmcomment{This is related to \identr{WHRS}:}

If the declared type of the \rhsexpression\ of a setter has the structure of a
bitvector or a type with fields, then if a bitslice or field selection is
applied to a setter invocation, then the assignment to that bitslice is
implemented using the following Read-Modify-Write (RMW) behavior:
\begin{itemize}
\item invoking the getter of the same name as the setter, with the same actual
arguments as the setter invocation
\item performing the assignment to the bitslice or field of the result of the
getter invocation
\item invoking the setter to assign the resulting value
\end{itemize}

\subsubsection{SemanticsRule.CheckNonOverlappingSlices\label{sec:SemanticsRule.CheckNonOverlappingSlices}}
\hypertarget{def-checknonoverlappingslices}{}
The helper function
\[
  \checknonoverlappingslices(
    \overname{(\tint\times\tint)^*}{\valueranges}
  ) \evalarrow \{\True\} \cup \overname{\TDynError}{\DynErrorConfig}
\]
checks whether the sets of integers represented by the list of ranges $\valueranges$
overlap, yielding $\True$.
\ProseOrError

\subsubsection{Prose}
All of the following apply:
\begin{itemize}
  \item view $\valueranges$ as the list $\vrange_{1..k}$;
  \item for every pair of indices $\vi$ and $\vj$ such that $1 \leq \vi < \vj \leq k$,
        applying \\
        $\checktworangesnonoverlapping$ to $\vrange_\vi$ and $\vrange_\vj$
        yields $\True$\ProseOrError;
  \item the result is $\True$.
\end{itemize}
\subsubsection{Formally}
\begin{mathpar}
\inferrule{
  {
    \begin{array}{r}
 1 \leq \vi < \vj \leq k:
 \checktworangesnonoverlapping(\vrange_\vi, \vrange_\vj) \evalarrow \\
 \True \OrDynError
    \end{array}
  }
}{
  \checknonoverlappingslices(\overname{\vrange_{1,,k}}{\valueranges}) \evalarrow \True
}
\end{mathpar}

\subsubsection{SemanticsRule.CheckTwoRangesNonOverlapping\label{sec:SemanticsRule.CheckTwoRangesNonOverlapping}}
\hypertarget{def-checktworangesnonoverlapping}{}
The helper function
\[
\begin{array}{r}
  \checktworangesnonoverlapping(
    (\overname{\tint}{\vsone}\times\overname{\tint}{\vlone}) \aslsep
    (\overname{\tint}{\vstwo}\times\overname{\tint}{\vltwo}) \aslsep
  ) \evalarrow \\
  \{\True\} \cup \overname{\TDynError}{\DynErrorConfig}
\end{array}
\]
checks whether two sets of integers represented by the
ranges $(\vsone, \vlone)$ and $(\vstwo, \vltwo)$ do not intersect, yielding $\True$.
\ProseOrError

\subsubsection{Prose}
All of the following apply:
\begin{itemize}
  \item evaluating $\PLUS$ for $\vsone$ and $\vlone$ via $\binoprel$ yields $\vsonelone$;
  \item evaluating $\LEQ$ for $\vsonelone$ and $\vstwo$ yields $\vsonelonestwo$;
  \item evaluating $\PLUS$ for $\vstwo$ and $\vltwo$ yields $\vstwoltwo$;
  \item evaluating $\LEQ$ for $\vstwoltwo$ and $\vsone$ yields $\vstwoltwosone$;
  \item evaluating $\BOR$ for $\vsonelonestwo$ and $\vstwoltwosone$ yields $\nvbool(\vb)$;
  \item checking whether $\vb$ is $\True$ yields $\True$\ProseTerminateAs{OverlappingSlices};
  \item the result is $\True$.
\end{itemize}

\subsubsection{Formally}
\begin{mathpar}
\inferrule{
  \binoprel(\PLUS, \vsone, \vlone) \evalarrow \vsonelone\\
  \binoprel(\LEQ, \vsonelone, \vstwo) \evalarrow \vsonelonestwo\\
  \binoprel(\PLUS, \vstwo, \vltwo) \evalarrow \vstwoltwo\\
  \binoprel(\LEQ, \vstwoltwo, \vsone) \evalarrow \vstwoltwosone\\
  \binoprel(\BOR, \vsonelonestwo, \vstwoltwosone) \evalarrow \nvbool(\vb)\\
  \checktrans{\vb}{OverlappingSlices} \checktransarrow \True\OrDynError
}{
  \checktworangesnonoverlapping((\vsone, \vlone), (\vstwo, \vltwo)) \evalarrow \True
}
\end{mathpar}

\section{Structured Type Field Assignment Expressions\label{sec:StructuredTypeFieldAssignmentExpressions}}
\subsection{Abstract Syntax}
\begin{flalign*}
\lexpr \derives\ & \LESetField(\lexpr, \identifier) & \\
    |\ & \LESetFields(\lexpr, \identifier^*) &
\end{flalign*}

\subsection{Typing}
\subsubsection{TypingRule.LESetBadField\label{sec:TypingRule.LESetBadField}}
\subsubsection{Prose}
All of the following apply:
\begin{itemize}
  \item $\vle$ denotes the access to the field named $\field$ in $\vleone$, that is, \\ $\LESetField(\vleone, \field)$;
  \item annotating the right-hand-side expression corresponding to $\vleone$ in $\tenv$ yields \\ $(\vtleone, \Ignore)$\ProseOrTypeError;
  \item annotating the left-hand-side expression $\vleone$ in $\tenv$ yields $\vletwo$\ProseOrTypeError;
  \item obtaining the \structure\ of $\vtleone$ in $\tenv$ yields a type $\vt$\ProseOrTypeError;
  \item $\vt$ is neither a \structuredtype\ nor a bitvector type;
  \item the result is an error indicating that the type of $\vle$ conflicts with the requirements of a field access expression.
\end{itemize}
\subsubsection{Formally}
\begin{mathpar}
\inferrule{
  \annotateexpr{\tenv, \torexpr(\vleone)} \typearrow (\vtleone, \Ignore) \OrTypeError\\\\
  \annotatelexpr{\tenv, \vleone, \vtleone} \typearrow \vletwo \OrTypeError\\\\
  \tstruct(\tenv, \vtleone) \typearrow \vt \OrTypeError\\\\
  \astlabel(\vt) \not\in \{\TException, \TRecord, \TBits\}
}{
  \annotatelexpr{\tenv, \overname{\LESetField(\vleone, \field)}{\vle}, \vte} \typearrow \TypeErrorVal{TypeConflict}
}
\end{mathpar}
\CodeSubsection{\LESetBadFieldBegin}{\LESetBadFieldEnd}{../Typing.ml}

\section{Bitfield Assignment Expressions\label{sec:BitfieldAssignmentExpressions}}
\subsection{Abstract Syntax}
\begin{flalign*}
\lexpr \derives\ & \LESlice(\lexpr, \slice^*) &
\end{flalign*}

\subsection{Typing}
\subsubsection{TypingRule.LESetBitField\label{sec:TypingRule.LESetBitField}}
\subsubsection{Prose}
All of the following apply:
\begin{itemize}
  \item $\vle$ denotes the access to the field named $\field$ in $\vleone$, that is, \\ $\LESetField(\vleone, \field)$;
  \item annotating the right-hand-side expression corresponding to $\vleone$ in $\tenv$ yields \\ $(\vtleone, \Ignore)$\ProseOrTypeError;
  \item annotating the left-hand-side expression $\vleone$ in $\tenv$ yields $\vletwo$\ProseOrTypeError;
  \item obtaining the \structure\ of $\vtleone$ in $\tenv$ yields a bitvector type with with bitfields $\bitfields$\ProseOrTypeError;
  \item One of the following applies:
  \begin{itemize}
    \item All of the following applies (\textsc{error\_missing\_field}):
    \begin{itemize}
      \item applying $\findbitfieldopt$ to $\bitfields$ and $\vfield$ yields $\None$, meaning the field is not declared
            in $\vtleone$;
      \item the result is a type error $\MissingField$.
    \end{itemize}

    \item All of the following applies (\textsc{field\_simple}):
    \begin{itemize}
      \item applying $\findbitfieldopt$ to $\bitfields$ and $\vfield$ yields a bitfield with corresponding slices $\slices$,
            that is, $\BitFieldSimple(\Ignore, \slices)$;
      \item $\vw$ is the width of $\slices$;
      \item $\vt$ is defined as the bitvector type of width $\vw$ and empty list of bitfields, that is, $\TBits(\vw, \emptylist)$;
      \item checking whether$\vte$ \typesatisfies\ $\vt$ in $\tenv$ yields $\True$\ProseOrTypeError;
      \item $\vletwo$ is defined as the slicing of $\vleone$ by $\slices$, that is, \\ $\LESlice(\vleone, \slices)$;
      \item annotating the left-hand-side expression $\vletwo$ in $\tenv$ yields $\newle$\ProseOrTypeError.
    \end{itemize}

    \item All of the following applies (\textsc{field\_nested}):
    \begin{itemize}
      \item applying $\findbitfieldopt$ to $\bitfields$ and $\vfield$ yields a nested bitfield with corresponding
            slices $\slices$ and list of bitfields $\bitfieldsp$, that is, \\ $\BitFieldNested(\Ignore, \slices, \bitfieldsp)$;
      \item $\vw$ is the width of $\slices$;
      \item $\vt$ is defined as the bitvector type of width $\vw$ and list of bitfields $\bitfieldsp$, that is, $\TBits(\vw, \bitfieldsp)$;
      \item checking whether$\vte$ \typesatisfies\ $\vt$ in $\tenv$ yields $\True$\ProseOrTypeError;
      \item $\vlethree$ is defined as the slicing of $\vleone$ by $\slices$, that is, \\ $\LESlice(\vleone, \slices)$;
      \item annotating the left-hand-side expression $\vlethree$ in $\tenv$ yields $\newle$\ProseOrTypeError.
    \end{itemize}

    \item All of the following applies (\textsc{field\_typed}):
    \begin{itemize}
      \item applying $\findbitfieldopt$ to $\bitfields$ and $\vfield$ yields a typed bitfield with corresponding
            slices $\slices$ and a type $\vt$, that is, \\ $\BitFieldType(\Ignore, \vslices, \vt)$;
      \item $\vw$ is the width of $\slices$;
      \item $\vtp$ is defined as the bitvector type of width $\vw$ and an empty list of bitfields, that is, $\TBits(\vw , \emptylist)$;
      \item checking whether $\vtp$ \typesatisfies\ $\vt$ in $\tenv$ yields $\True$\ProseOrTypeError;
      \item checking whether$\vte$ \typesatisfies\ $\vt$ in $\tenv$ yields $\True$\ProseOrTypeError;
      \item $\vletwo$ is defined as the slicing of $\vleone$ by $\slices$, that is, \\ $\LESlice(\vleone, \slices)$;
      \item annotating the left-hand-side expression $\vletwo$ in $\tenv$ yields $\newle$\ProseOrTypeError.
    \end{itemize}
  \end{itemize}
\end{itemize}
\subsubsection{Formally}
\begin{mathpar}
\inferrule[error\_missing\_field]{
  \annotateexpr{\tenv, \torexpr(\vleone)} \typearrow (\vtleone, \Ignore) \OrTypeError\\\\
  \annotatelexpr{\tenv, \vleone, \vtleone} \typearrow \vletwo \OrTypeError\\\\
  \tstruct(\tenv, \vtleone) \typearrow \TBits(\Ignore, \bitfields) \OrTypeError\\\\
  \commonprefixline\\\\
  \findbitfieldopt(\bitfields, \field) \typearrow \None\
}{
  \annotatelexpr{\tenv, \overname{\LESetField(\vleone, \field)}{\vle}, \vte} \typearrow \TypeErrorVal{\MissingField}
}
\end{mathpar}

\begin{mathpar}
\inferrule[field\_simple]{
  \annotateexpr{\tenv, \torexpr(\vleone)} \typearrow (\vtleone, \Ignore) \OrTypeError\\\\
  \annotatelexpr{\tenv, \vleone, \vtleone} \typearrow \vletwo \OrTypeError\\\\
  \tstruct(\tenv, \vtleone) \typearrow \TBits(\Ignore, \bitfields) \OrTypeError\\\\
  \commonprefixline\\\\
  \findbitfieldopt(\bitfields, \field) \typearrow \langle \BitFieldSimple(\Ignore, \slices) \rangle\\
  \sliceswidth(\tenv, \vslices) \typearrow \vw\\
  \vt \eqdef \TBits(\vw, \emptylist)\\\\
  \commonsuffixline\\\\
  \checktypesat(\tenv, \vte, \vt) \typearrow \True \OrTypeError\\\\
  \vletwo \eqdef \LESlice(\vleone, \slices)\\
  \annotatelexpr{\tenv, \vletwo, \vte} \typearrow \newle \OrTypeError
}{
  \annotatelexpr{\tenv, \overname{\LESetField(\vleone, \field)}{\vle}, \vte} \typearrow \newle
}
\end{mathpar}

\begin{mathpar}
\inferrule[field\_nested]{
  \annotateexpr{\tenv, \torexpr(\vleone)} \typearrow (\vtleone, \Ignore) \OrTypeError\\\\
  \annotatelexpr{\tenv, \vleone, \vtleone} \typearrow \vletwo \OrTypeError\\\\
  \tstruct(\tenv, \vtleone) \typearrow \TBits(\Ignore, \bitfields) \OrTypeError\\\\
  \commonprefixline\\\\
  \findbitfieldopt(\bitfields, \field) \typearrow \langle \BitFieldNested(\Ignore, \slices, \bitfieldsp) \rangle\\
  \sliceswidth(\tenv, \vslices) \typearrow \vw\\
  \vt \eqdef \TBits(\vw, \bitfieldsp)\\\\
  \commonsuffixline\\\\
  \checktypesat(\tenv, \vte, \vt) \typearrow \True \OrTypeError\\\\
  \vletwo \eqdef \LESlice(\vleone, \slices)\\
  \annotatelexpr{\tenv, \vletwo, \vte} \typearrow \newle \OrTypeError
}{
  \annotatelexpr{\tenv, \overname{\LESetField(\vleone, \field)}{\vle}, \vte} \typearrow \newle
}
\end{mathpar}

\begin{mathpar}
\inferrule[field\_typed]{
  \annotateexpr{\tenv, \torexpr(\vleone)} \typearrow (\vtleone, \Ignore) \OrTypeError\\\\
  \annotatelexpr{\tenv, \vleone, \vtleone} \typearrow \vletwo \OrTypeError\\\\
  \tstruct(\tenv, \vtleone) \typearrow \TBits(\Ignore, \bitfields) \OrTypeError\\\\
  \commonprefixline\\\\
  \findbitfieldopt(\bitfields, \field) \typearrow \langle \BitFieldType(\Ignore, \vslices, \vt) \rangle\\
  \sliceswidth(\tenv, \vslices) \typearrow \vw\\
  \vtp \eqdef \TBits(\vw , \emptylist)\\
  \checktypesat(\tenv, \vtp, \vt) \typearrow \True \OrTypeError\\\\
  \commonsuffixline\\\\
  \checktypesat(\tenv, \vte, \vt) \typearrow \True \OrTypeError\\\\
  \vletwo \eqdef \LESlice(\vleone, \slices)\\
  \annotatelexpr{\tenv, \vletwo, \vte} \typearrow \newle \OrTypeError
}{
  \annotatelexpr{\tenv, \overname{\LESetField(\vleone, \field)}{\vle}, \vte} \typearrow \newle
}
\end{mathpar}

\subsubsection{Semantics}
The semantics for assigning to individual bitvector bitfields is covered by \nameref{sec:SemanticsRule.LESlice}
as the type system transforms the \untypedast\ for assigning to an individual bitfield into an $\LESlice$ \typedast.

\subsubsection{TypingRule.LESetStructuredField\label{sec:TypingRule.LESetStructuredField}}
\subsubsection{Prose}
All of the following apply:
\begin{itemize}
  \item $\vle$ denotes the access to the field named \texttt{field} in $\vleone$;
  \item annotating the right-hand-side expression corresponding to $\vleone$ in $\tenv$ yields \\ $(\vtleone, \Ignore)$\ProseOrTypeError;
  \item annotating the left-hand-side expression  $\vleone$ with type $\vtleone$ in $\tenv$ yields $\vletwo$\ProseOrTypeError;
  \item obtaining the \structure\ of $\vtleone$ in $\tenv$ yields a \structuredtype\ with fields \\
        $\fields$\ProseOrTypeError;
  \item checking that there exists a type associated with the field $\field$ in $\fields$ $\True$ \ProseTerminateAs{\MissingField};
  \item the type associated with the field $\field$ in $\fields$ is $\vt$;
  \item determining whether $\vte$ \typesatisfies\ $\vt$ yields $\True$\ProseOrTypeError;
  \item $\newle$ is the access to the field $\field$ in $\vletwo$, that is, $\LESetField(\vletwo, \field)$.
\end{itemize}
\subsubsection{Formally}
\begin{mathpar}
\inferrule{
  \annotateexpr{\tenv, \torexpr(\vleone)} \typearrow (\vtleone, \Ignore) \OrTypeError\\\\
  \annotatelexpr{\tenv, \vleone, \vtleone} \typearrow \vletwo \OrTypeError\\\\
  \tstruct(\tenv, \vtleone) \typearrow L(\fields) \OrTypeError\\\\
  L \in \{\TException, \TRecord\}\\
  \assocopt(\fields, \field) \typearrow \tyopt\\
  \checktrans{\tyopt \neq \None}{\MissingField} \checktransarrow \True \OrTypeError\\\\
  \tyopt \eqname \langle\vt\rangle\\
  \checktypesat(\tenv, \vte, \vt) \typearrow \True \OrTypeError\\\\
  \newle \eqdef \LESetField(\vletwo, \field)
}{
  \annotatelexpr{\tenv, \overname{\LESetField(\vleone, \field)}{\vle}, \vte} \typearrow \newle
}
\end{mathpar}
\CodeSubsection{\LESetStructuredFieldBegin}{\LESetStructuredFieldEnd}{../Typing.ml}

\subsection{Semantics}
\hypertarget{SemanticsRule.LESetField}{}
\subsubsection{SemanticsRule.LESetField\label{sec:SemanticsRule.LESetField}}
\subsubsection{Example}
In the specification:
\VerbatimInput{../tests/ASLSemanticsReference.t/SemanticsRule.LESetField.asl}
\texttt{my\_record.a = 42;} binds \texttt{my\_record} to \texttt{\{a: 42, b: 100\}} in the environment where \texttt{my\_record} is bound to \texttt{\{a: 3, b: 100\}}.

\subsubsection{Prose}
All of the following apply:
\begin{itemize}
  \item $\vle$ denotes a field update expression, $\LESetField(\rerecord, \fieldname)$;
  \item evaluating the right-hand-side expression corresponding to $\rerecord$
  in $\env$ is $\Normal(\rmrecord, \envone)$\ProseOrAbnormal;
  \item $\rmrecord$ is a pair consisting of the native record $\rvrecord$ and
  the execution graph $\vgone$;
  \item setting the field $\fieldname$ in the native record $\rvrecord$ to $\vv$
  is the updated native record $\vvone$;
  \item $\vmone$ is the pair consisting of the native vector $\vvone$ and the
  execution graph that is, the parallel composition of $\vg$ and $\vgone$;
  \item the steps so far computed the updated record, but have not assigned it to
  the variable holding the record given by $\record$, which is achieved next.
  Evaluating the left-hand-side expression $\rerecord$ in an environment $\envone$ with $\vmone$
  is the output configuration $C$.
\end{itemize}
\subsubsection{Formally}
\begin{mathpar}
\inferrule{
    \evalexpr{\env, \torexpr(\rerecord)} \evalarrow \Normal(\rmrecord, \envone) \OrAbnormal\\
    \rmrecord \eqname (\rvrecord, \vgone)\\
    \setfield(\fieldname, \vv, \rvrecord) \evalarrow \vvone\\
    \vmone \eqdef (\vvone, \vg \parallelcomp \vgone)\\
    \evallexpr{\env1, \rerecord, \vmone} \evalarrow C
}{
    \evallexpr{\env, \LESetField(\rerecord, \fieldname), (\vv, \vg)} \evalarrow C
}
\end{mathpar}
\CodeSubsection{\EvalLESetFieldBegin}{\EvalLESetFieldEnd}{../Interpreter.ml}

\subsubsection{Comments}
We note that the type-checker guarantees that $\fieldname$ exists in the record given by $\record$
via \nameref{sec:TypingRule.LESetStructuredField}.

\lrmcomment{This is related to \identr{WHRS}:}
If the declared type of the \rhsexpression\ of a setter has the structure of a
bitvector or a type with fields, then if a bitslice or field selection is
applied to a setter invocation, then the assignment to that bitslice is
implemented using the following Read-Modify-Write (RMW) behavior:
\begin{itemize}
    \item invoking the getter of the same name as the setter, with the same actual
    arguments as the setter invocation
    \item performing the assignment to the bitslice or field of the result of the
    getter invocation
    \item invoking the setter to assign the resulting value
\end{itemize}

\section{Multi-slice Assignment Expressions\label{sec:MultiSliceAssignmentExpressions}}
\subsection{Abstract Syntax}
\begin{flalign*}
\lexpr \derives\ & \LEConcat(\lexpr^+) &
\end{flalign*}

\subsection{Typing}
\subsubsection{TypingRule.LEConcat\label{sec:TypingRule.LEConcat}}
\subsubsection{Prose}
All of the following apply:
\begin{itemize}
  \item $\vle$ denotes the concatenation of left-hand-side expressions $\les$, that is, \\ $\LEConcat(\les, \Ignore)$;
  \item annotating the right-hand-side expression corresponding to $\vle$ in $\tenv$ yields \\ $(\vteeq, \Ignore)$\ProseOrTypeError;
  \item checking whether the bitwidth of $\vteeq$ equals the bitwidth of $\vte$ in $\tenv$ yields $\True$\ProseOrTypeError;
  \item $\les$ is the list of left-hand-side expressions $\vle_i$, for $i=1..k$;
  \item annotating each left-hand-side expression $\vle_i$ as a bitvector-typed expression (via $\annotatelebitslice$)
        yields the annotated left-hand-side expression $\vleone_i$ and corresponding bitwidth $\width_i$, for $i=1..k$;
  \item $\lesone$ is defined as the list $\vleone_{1..k}$;
  \item $\widths$ is defined as the list $\width_{1..k}$;
  \item $\newle$ is the concatenation of left-hand-side expressions $\lesone$ with corresponding list of widths $\widths$.
\end{itemize}
\subsubsection{Formally}
\begin{mathpar}
\inferrule{
  \vle \eqname \LEConcat(\les, \Ignore)\\
  \annotateexpr{\tenv, \torexpr(\vle)} \typearrow (\vteeq, \Ignore) \OrTypeError\\\\
  \checkbitsequalwidth(\tenv, \vteeq, \vte) \typearrow \True \OrTypeError\\\\
  \les \eqname \vle_{1..k}\\
  i=1..k: \annotatelebitslice(\tenv, \vle_i) \typearrow (\vleone_i, \width_i) \OrTypeError\\\\
  \lesone \eqdef \vleone_{1..k}\\
  \widths \eqdef \width_{1..k}
}{
  \annotatelexpr{\tenv, \vle, \vte} \typearrow \overname{\LEConcat(\lesone, \widths)}{\newle}
}
\end{mathpar}
\CodeSubsection{\LEConcatBegin}{\LEConcatEnd}{../Typing.ml}

\subsubsection{TypingRule.LEBitSlice\label{sec:TypingRule.LEBitSlice}}
\hypertarget{def-annotatelebitsice}{}
The helper function
\[
  \annotatelebitslice(\overname{\staticenvs}{\tenv} \aslsep \overname{\lexpr}{\vle})
  \aslto \overname{\lexpr}{\vleone} \times \overname{\N}{\width}
\]
annotates a left-hand-side expression $\vle$, which is checked to be of bitvector type
with width $\width$,
resulting in the annotated expression and $\width$, or a type error, if one is detected.

All of the following apply:
\begin{itemize}
  \item annotating the right-hand-side expression corresponding to $\vle$ in $\tenv$ yields \\ $(\vteone, \Ignore)$\ProseOrTypeError;
  \item applying $\getbitvectorwidth$ to $\vteone$ in $\tenv$ yields $\vwidth$\ProseOrTypeError;
  \item define $\vtetwo$ as the bitvector type of width given by $\width$ and an empty list of bitfields, that is,
        $\TBits(\ELInt{\width}, \emptylist)$;
  \item annotating the left-hand-side expression $\vtetwo$ in $\tenv$ yields $\vleone$\ProseOrTypeError.
\end{itemize}
\subsubsection{Formally}
\begin{mathpar}
\inferrule{
  \annotateexpr{\tenv, \torexpr(\vleone)} \typearrow (\vteone, \Ignore) \OrTypeError\\\\
  \getbitvectorwidth(\tenv, \vteone) \typearrow \vwidth \OrTypeError\\\\
  \vtetwo \eqdef \TBits(\ELInt{\width}, \emptylist)\\
  \annotatelexpr{\tenv, \vle, \vtetwo} \typearrow \vleone \OrTypeError
}{
  \annotatelebitslice(\tenv, \vle) \typearrow (\vleone, \width)
}
\end{mathpar}

\subsection{Semantics}
\subsubsection{SemanticsRule.LEConcat\label{sec:emanticsRule.LEConcat}}
\newcommand\extractslices[0]{\hyperlink{def-extractslices}{\textfunc{extract\_slices}}}
\subsubsection{Prose}
All of the following apply:
\begin{itemize}
    \item applying $\extractslices$ to $\vwidths$, the literal expression for $0$, and $(\vv, \vg)$, in $\env$ to
        extract a list of native bitvector values that correspond to the slices in $\vv$
        yields $(\ms, \Ignore)$\ProseOrError;
    \item applying $\evalmultiassignment$ to $\vles$ and $\ms$ in $\env$ to assign from the slices in
          $\ms$ to the \assignableexpressions\ in $\vles$ yields the resulting configuration $C$.
\end{itemize}
\subsubsection{Formally}
\begin{mathpar}
\inferrule{
    \extractslices(\env, \vwidths, \elint{0}, (\vv, \vg)) \evalarrow (\ms, \Ignore) \OrDynError\\\\
    \evalmultiassignment(\env, \vles, \ms) \evalarrow C
}{
    \evallexpr{\env, \overname{\LEConcat(\vles, \vwidths)}{\vle}, (\vv, \vg)} \evalarrow C
}
\end{mathpar}

\subsubsection{SemanticsRule.ExtractSlices\label{sec:emanticsRule.ExtractSlices}}
The helper relation
\[
    \extractslices(\overname{\dynamicenvs}{\env}, \overname{\expr^*}{\vwidths}, \overname{((\vals \times \XGraphs)^*)^*}{\ms}, \overname{(\vals \times \XGraphs)}{\vm})
    \;\aslrel\;
    \overname{((\vals \times \XGraphs)^*)^*}{\msout}
    \cup \overname{\TDynError}{\DynErrorConfig}
\]

\subsubsection{Prose}
\subsubsection{Formally}

% | LE_Concat (les, Some widths) ->
% let extract_one e_width ms =
%   let m' =
%     let* v = m
%     and* width = eval_expr_sef env e_width
%     and* start = eval_expr_sef env e_start in
%     B.read_from_bitvector [ (start, width) ] v
%   in
%   m' :: ms
% in
% let ms = List.fold_right extract_one widths [] in
% multi_assign V1 env les ms
