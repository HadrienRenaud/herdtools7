\chapter{Local Storage Declarations\label{chap:LocalStorageDeclarations}}

Local storage declarations are similar to \assignableexpressions, except that they introduce new variables or constants
into the local static environment.

\hypertarget{def-localdeclarationkeyword}{}
\hypertarget{def-localdeclarationitem}{}
A \localdeclarationkeyword\ is one of \texttt{var}, \texttt{let}, and \texttt{constant}.
A \localdeclarationitem\ is an element derived from $\Ndeclitem$.
A \localdeclaration\ consists of a \localdeclarationitem\ and a \localdeclarationkeyword.

We show the syntax relevant to local declarations in \secref{LocalStorageDeclarationsSyntax} and
the AST rule and rules need to build the AST for \assignableexpressions\ in \secref{LocalStorageDeclarationsAbstractSyntax}.
We then define the typing and semantics of the different kinds of local declarations:
\begin{itemize}
\item Discarding declarations (see \secref{DiscardingDeclarations})
\item Un-annotated variable declarations (see \secref{UnAnnotatedVariableDeclarations})
\item Type-annotated variable declarations (see \secref{TypeAnnotatedVariableDeclarations})
\item Tuple declarations (see \secref{TupleDeclarations})
\end{itemize}

\hypertarget{def-annotatelocaldeclitem}{}
\paragraph{Typing:} The function
\[
  \begin{array}{c}
  \annotatelocaldeclitem{
    \overname{\staticenvs}{\tenv} \aslsep
    \overname{\ty}{\tty} \aslsep
    \overname{\localdeclkeyword}{\ldk} \aslsep
    \overname{\langle\expr\rangle}{\veopt} \aslsep
    \overname{\localdeclitem}{\ldi}
   } \aslto\\
  (\overname{\staticenvs}{\newtenv} \aslsep \overname{\localdeclitem}{\newldi})
  \cup \overname{\TTypeError}{\TypeErrorConfig}
  \end{array}
\]
annotates a \localdeclarationitem\ $\ldi$ with a \localdeclarationkeyword\ $\ldk$, given a type $\tty$,
and optional initializing expression $\veopt$,
in a static environment $\tenv$ results in $(\newenv, \newldi)$ where $\newenv$ is the modified
static environment and $\newldi$ is the annotated local declaration item.
\ProseOtherwiseTypeError

\paragraph{Semantics:} The relation
\hypertarget{def-evallocaldecl}{}
\[
  \evallocaldecl{
    \overname{\envs}{\env} \aslsep
    \overname{\localdeclitem}{\ldi} \aslsep
    \overname{\langle\overname{\vals}{\vv}\times\overname{\XGraphs}{\vgone}\rangle}{\minitopt}
    } \;\aslrel\;
    \Normal(\overname{\XGraphs}{\newg}, \overname{\envs}{\newenv})
\]
evaluates a \localdeclarationitem\ $\ldi$ in an environment
$\env$ with an optional initialization value $\minitopt$.
That is, the right-hand side of the declaration, if it exists,
has already been evaluated, yielding $\minitopt$ (see, for example, \nameref{sec:SemanticsRule.SDeclSome}).
Evaluation of the local variables $\ldi$
in an environment $\env$ is either $\Normal(\vg, \newenv)$
or an abnormal configuration.

While there are three different categories of local storage elements ---
constants, mutable variables (declared via \texttt{var}), and immutable variables (declared via \texttt{let}) ---
from the perspective of the semantics of local storage elements, they are all treated the same way.

\section{Syntax\label{sec:LocalStorageDeclarationsSyntax}}
Declaring a local storage element is done via the following grammar rules:
\begin{flalign*}
\Nstmt \derivesinline\ & \Nlocaldeclkeyword \parsesep \Ndeclitem \parsesep \Teq \parsesep \Nexpr \parsesep \Tsemicolon &\\
|\ & \Tvar \parsesep \Ndeclitem \parsesep \option{\Teq \parsesep \Nexpr} \parsesep \Tsemicolon &\\
|\ & \Tvar \parsesep \Clisttwo{\Tidentifier} \parsesep \Tcolon \parsesep \Nty \parsesep \Tsemicolon &\\
\end{flalign*}

\begin{flalign*}
\Nlocaldeclkeyword \derivesinline\ & \Tlet \;|\; \Tconstant&\\
\Ndeclitem \derives\ & \Nuntypeddeclitem \parsesep \Nasty&\\
|\ & \Nuntypeddeclitem  &\\
\Nuntypeddeclitem \derivesinline\ & \Tidentifier &\\
|\ & \Tminus &\\
|\ & \Plisttwo{\Ndeclitem} &
\end{flalign*}

\section{Abstract Syntax\label{sec:LocalStorageDeclarationsAbstractSyntax}}
\begin{flalign*}
\localdeclkeyword \derives\ & \LDKVar \;|\; \LDKConstant \;|\; \LDKLet &\\
\localdeclitem \derives\ & \LDIDiscard &\\
    |\ & \LDIVar(\identifier) & \\
    |\ & \LDITuple(\localdeclitem^*) &\\
    |\ & \LDITyped(\localdeclitem, \ty) &
\end{flalign*}

\subsubsection{ASTRule.LocalDeclKeyword\label{sec:ASTRule.LocalDeclKeyword}}
\hypertarget{build-localdeclkeyword}{}
The function
\[
\buildlocaldeclkeyword(\overname{\parsenode{\Nlocaldeclkeyword}}{\vparsednode}) \;\aslto\;
  \overname{\localdeclkeyword}{\vastnode}
\]
transforms a parse node $\vparsednode$ into an AST node $\vastnode$.

\begin{mathpar}
\inferrule[let]{}{
  \buildlocaldeclkeyword(\overname{\Nlocaldeclkeyword(\Tlet)}{\vparsednode}) \astarrow \overname{\LDKLet}{\vastnode}
}
\end{mathpar}

\begin{mathpar}
\inferrule[constant]{}{
  \buildlocaldeclkeyword(\overname{\Nlocaldeclkeyword(\Tconstant)}{\vparsednode}) \astarrow \overname{\LDKConstant}{\vastnode}
}
\end{mathpar}

\subsubsection{ASTRule.DeclItem\label{sec:ASTRule.DeclItem}}
\hypertarget{build-declitem}{}
The function
\[
  \builddeclitem(\overname{\parsenode{\Ndeclitem}}{\vparsednode}) \;\aslto\; \overname{\localdeclitem}{\vastnode}
\]
transforms a parse node $\vparsednode$ into an AST node $\vastnode$.

\begin{mathpar}
\inferrule[typed]{}{
  {
    \begin{array}{r}
  \builddeclitem(\Ndeclitem(\punnode{\Nuntypeddeclitem}, \punnode{\Nasty})) \astarrow \\
  \overname{\LDITyped(\astof{\vuntypedlocaldeclitem}, \astof{\vasty})}{\vastnode}
    \end{array}
  }
}
\end{mathpar}

\begin{mathpar}
\inferrule[untyped]{}{
  \builddeclitem(\Ndeclitem(\punnode{\Nuntypeddeclitem})) \astarrow
  \overname{\astof{\vuntypedlocaldeclitem}}{\vastnode}
}
\end{mathpar}

\subsubsection{ASTRule.UntypedDeclItem\label{sec:ASTRule.UntypedDeclItem}}
\hypertarget{build-untypeddeclitem}{}
The function
\[
  \builduntypeddeclitem(\overname{\parsenode{\Nuntypeddeclitem}}{\vparsednode}) \;\aslto\; \overname{\localdeclitem}{\vastnode}
\]
transforms a parse node $\vparsednode$ into an AST node $\vastnode$.

\begin{mathpar}
\inferrule[var]{}{
  \builduntypeddeclitem(\Nuntypeddeclitem(\Tidentifier(\id))) \astarrow
  \overname{\LDIVar(\id)}{\vastnode}
}
\end{mathpar}

\begin{mathpar}
\inferrule[discard]{}{
  \builduntypeddeclitem(\Nuntypeddeclitem(\Tminus)) \astarrow
  \overname{\LDIDiscard}{\vastnode}
}
\end{mathpar}

\begin{mathpar}
\inferrule[tuple]{
  \buildclist[\builddeclitem](\vdeclitems) \astarrow \vdeclitemasts
}{
  {
    \begin{array}{r}
  \builduntypeddeclitem(\Nuntypeddeclitem(\namednode{\vdeclitems}{\Plisttwo{\Ndeclitem}})) \astarrow \\
  \overname{\LDITuple(\vdeclitemasts)}{\vastnode}
    \end{array}
  }
}
\end{mathpar}

\section{Discarding Declarations\label{sec:DiscardingDeclarations}}
\subsection{Typing}
\subsubsection{TypingRule.LDDiscard \label{sec:TypingRule.LDDiscard}}
\subsubsection{Example}
\VerbatimInput{../tests/ASLTypingReference.t/TypingRule.LDDiscard.asl}

\subsubsection{Prose}
All of the following apply:
\begin{itemize}
  \item $\ldi$ is a local declaration which can be discarded, that is, $\LDIDiscard(\None)$;
  \item $\newenv$ is $\tenv$;
  \item $\newldi$ is $\ldi$.
\end{itemize}
\subsubsection{Formally}
\begin{mathpar}
\inferrule{}{
  \annotatelocaldeclitem{\tenv, \tty, \ldk, \veopt, \overname{\LDIDiscard(\None)}{\ldi}} \typearrow (\tenv, \ldi)
}
\end{mathpar}
\CodeSubsection{\LDDiscardBegin}{\LDDiscardEnd}{../Typing.ml}

\subsection{Semantics}
\subsubsection{SemanticsRule.LDDiscard \label{sec:SemanticsRule.LDDiscard}}
\subsubsection{Example}
In the specification:
\VerbatimInput{../tests/ASLSemanticsReference.t/SemanticsRule.LDDiscard.asl}
\texttt{var - : integer;} does not modify the environment.

\subsubsection{Prose}
All of the following apply:
\begin{itemize}
  \item $\ldi$ indicates that the initialization value will be discarded,
        $\LDIDiscard$;
  \item $\newg$ is the empty graph;
  \item $\newenv$ is $\env$.
\end{itemize}
\subsubsection{Formally}
\begin{mathpar}
\inferrule{}
{
  \evallocaldecl{\env, \overname{\LDIDiscard}{\ldi}, \overname{\color{white}{\texttt{xx}\Ignore}\texttt{xx}}{\minitopt}}
  \evalarrow \Normal(\overname{\emptygraph}{\newg}, \overname{\env}{\newenv})
}
\end{mathpar}
\CodeSubsection{\EvalLDDiscardBegin}{\EvalLDDiscardEnd}{../Interpreter.ml}

\section{Un-annotated Variable Declarations\label{sec:UnAnnotatedVariableDeclarations}}
\subsection{Typing}
\subsubsection{TypingRule.LDVar \label{sec:TypingRule.LDVar}}
\subsubsection{Example}
\VerbatimInput{../tests/ASLTypingReference.t/TypingRule.LDVar.asl}

\subsubsection{Prose}
All of the following apply:
\begin{itemize}
  \item $\ldi$ denotes a variable $\vx$, that is, $\LDIVar(\vx)$;
  \item determining whether $\vx$ is not declared in $\tenv$ yields $\True$\ProseOrTypeError;
  \item $\tenvtwo$ is $\tenv$ modified so that $\vx$ is locally declared to have type $\tty$;
  \item applying $\addimmutableexpr$ to $\ldk$, $\veopt$, and $\vx$ in $\tenv$ (to conditionally
        update $\tenvtwo$) yields $\newtenv$;
  \item $\newldi$ is the declaration of variable $\vx$.
\end{itemize}
\subsubsection{Formally}
\begin{mathpar}
\inferrule{
  \checkvarnotinenv{\tenv, \vx} \typearrow \True \OrTypeError\\\\
  \addlocal(\tenv, \vx, \tty, \ldk) \typearrow \tenvtwo\\
  \addimmutableexpr(\tenvtwo, \ldk, \veopt, \vx) \typearrow \newtenv
}{
  \annotatelocaldeclitem{\tenv, \tty, \ldk, \veopt, \overname{\LDIVar(\vx)}{\ldi}} \typearrow (\newtenv, \LDIVar(\vx))
}
\end{mathpar}
\CodeSubsection{\LDVarBegin}{\LDVarEnd}{../Typing.ml}
\lrmcomment{This is related to \identr{YSPM}, \identd{FXST}.}

\subsection{Semantics}
\subsubsection{SemanticsRule.LDVar \label{sec:SemanticsRule.LDVar}}
    \subsubsection{Prose}
    All of the following apply:
    \begin{itemize}
    \item $\ldi$ is a variable declaration, $\LDIVar(\vx)$;
    \item $\minitopt$ is $\vm$;
    \item $\vm$ is a pair consisting of the value $\vv$ and execution graph $\vgone$;
    \item declaring $\vx$ in $\env$ is $(\newenv, \vgtwo)$;
    \item $\newg$ is the ordered composition of $\vgone$ and $\vgtwo$ with the $\asldata$ edge.
    \end{itemize}

\subsubsection{Example}
In the specification:
\VerbatimInput{../tests/ASLSemanticsReference.t/SemanticsRule.LDVar0.asl}
\texttt{var x = 3;} binds \texttt{x} to the evaluation of \texttt{3} in $\env$.

\subsubsection{Example}
In the specification:
\VerbatimInput{../tests/ASLSemanticsReference.t/SemanticsRule.LDVar1.asl}
\texttt{var x : integer = 3;} binds \texttt{x} to the evaluation of
\texttt{3} in $\env$, without type consideration at runtime.

\subsubsection{Formally}
\begin{mathpar}
\inferrule{
  \vm \eqname (\vv, \vgone)\\
  \declarelocalidentifier(\env, \vx, \vv)\evalarrow(\newenv, \vgtwo)\\
  \newg \eqdef \ordered{\vgone}{\asldata}{\vgtwo}
}{
  \evallocaldecl{\env, \LDIVar(\vx), \langle \vm\rangle} \evalarrow \Normal(\newg, \newenv)
}
\end{mathpar}
\CodeSubsection{\EvalLDVarBegin}{\EvalLDVarEnd}{../Interpreter.ml}

\section{Type-annotated Variable Declarations\label{sec:TypeAnnotatedVariableDeclarations}}
\subsection{Typing}
\subsubsection{TypingRule.LDTyped\label{sec:TypingRule.LDTyped}}
\subsubsection{Example}
\VerbatimInput{../tests/ASLTypingReference.t/TypingRule.LDTyped.asl}

\subsubsection{Prose}
All of the following apply:
\begin{itemize}
  \item $\ldi$ denotes a local declaration item $\ldip$ with local declaration keyword $\ldk$
  and a type $\vt$, that is $\LDITyped(\ldip, \vt)$;
  \item determining the \structure{} of right-hand side type $\tty$ in $\tenv$ yields $\ttyp$;
  \item propagating integer constraints from $\ttyp$ to $\vt$ using $\inheritintegerconstraints$ yields $\vtone$;
  \item annotating the type $\vtone$ in $\tenv$ yields $\vttwo$\ProseOrTypeError;
  \item determining whether $\vttwo$ can be initialized with $\tty$ in $\tenv$ yields $\True$\ProseOrTypeError;
  \item annotating the local declaration item $\ldip$ with the local declaration keyword $\ldk$, given
  the type $\vttwo$, in the environment $\tenv$, yields $(\newtenv,\newldip)$;
  \item $\newldi$ is the local declaration denoting $\newldip$ and the type $\vtp$, that is, \\
  $\LDITyped(\newldip, \vttwo)$.
\end{itemize}
\subsubsection{Formally}
\begin{mathpar}
\inferrule{
  \tstruct(\tenv, \tty) \typearrow \ttyp \\
  \inheritintegerconstraints(\vt, \ttyp) \typearrow \vtone \OrTypeError \\
  \annotatetype{\tenv, \vtone} \typearrow \vttwo \OrTypeError\\\\
  \checkcanbeinitializedwith(\tenv, \vttwo, \tty) \typearrow \True \OrTypeError\\\\
  \annotatelocaldeclitem{\tenv, \vttwo, \ldk, \veopt, \ldip} \typearrow (\newtenv, \newldip) \OrTypeError
}{
  \annotatelocaldeclitem{\tenv, \tty, \ldk, \veopt, \overname{\LDITyped(\ldip, \vt)}{\ldi}} \typearrow \\
  (\newtenv, \LDITyped(\newldip, \vttwo))
}
\end{mathpar}
\CodeSubsection{\LDTypedBegin}{\LDTypedEnd}{../Typing.ml}

\subsubsection{TypingRule.InheritIntegerConstraints\label{sec:TypingRule.InheritIntegerConstraints}}
\hypertarget{def-inheritintegerconstraints}{}
The helper function
\[
\inheritintegerconstraints(\overname{\ty}{\lhs} \aslsep \overname{\ty}{\rhs})
\typearrow \lhsp \cup\ \overname{\TTypeError}{\TypeErrorConfig}
\]
propagates integer constraints from the right-hand side type $\rhs$ to the left-hand side type annotation $\lhs$.
It can fail with a type error.

\subsubsection{Prose}
One of the following applies:
\begin{itemize}
  \item All of the following apply (\textsc{int}):
  \begin{itemize}
    \item $\lhs$ is a \wellconstrainedintegertype{} with no constraints;
    \item $\rhs$ is a \wellconstrainedintegertype{}\ProseOrTypeError;
    \item $\lhsp$ is $\rhs$.
  \end{itemize}

  \item All of the following apply (\textsc{tuple}):
  \begin{itemize}
    \item $\lhs$ is a tuple of types \texttt{lhs\_tys};
    \item $\rhs$ is a tuple of types \texttt{rhs\_tys};
    \item the lengths of \texttt{lhs\_tys} and \texttt{rhs\_tys} are equal\ProseOrTypeError;
    \item define \texttt{lhs\_tys'} by applying $\inheritintegerconstraints$ to each element of \texttt{lhs\_tys} and \texttt{rhs\_tys}\ProseOrTypeError;
    \item $\lhsp$ is $\TTuple(\texttt{lhs\_tys'})$.
  \end{itemize}

  \item All of the following apply (\textsc{other}):
  \begin{itemize}
    \item $\lhs$ is not a \wellconstrainedintegertype{} with no constraints, or one of $\lhs$ and $\rhs$ is not a tuple type;
    \item $\lhsp$ is $\lhs$.
  \end{itemize}
\end{itemize}

\subsubsection{Formally}
\begin{mathpar}
\inferrule[int]{
  \rhs \eqname \TInt(\wellconstrained(\Ignore)) \OrTypeError
}{
  \inheritintegerconstraints(\overname{\TInt(\wellconstrained(\emptylist))}{\lhs}, \rhs) \typearrow \overname{\rhs}{\lhsp}
}
\end{mathpar}

\begin{mathpar}
\inferrule[tuple]{
  \lhs \eqname \TTuple(\texttt{lhs\_tys}) \\
  \rhs \eqname \TTuple(\texttt{rhs\_tys}) \\\\
  \equallength(\texttt{lhs\_tys}, \texttt{rhs\_tys}) \typearrow \True \OrTypeError \\
  \vi \in \listrange(\lhs): \inheritintegerconstraints(\texttt{lhs\_tys}_i, \texttt{rhs\_tys}_i) \typearrow \texttt{lhs\_tys'}_i \OrTypeError \\
}{
  \inheritintegerconstraints(\lhs, \rhs) \typearrow \overname{\TTuple(\texttt{lhs\_tys'})}{\lhsp}
}
\end{mathpar}

\begin{mathpar}
\inferrule[other]{
  \lhs \neq \TInt(\wellconstrained(\Ignore)) \;\lor\; \astlabel(\lhs) \neq \TTuple \;\lor\; \astlabel(\rhs) \neq \TTuple
}{
  \inheritintegerconstraints(\lhs, \rhs) \typearrow \overname{\lhs}{\lhsp}
}
\end{mathpar}

\subsubsection{TypingRule.CheckCanBeInitializedWith\label{sec:TypingRule.CheckCanBeInitializedWith}}
\hypertarget{def-checkcanbeinitializedwith}{}
The helper function
\[
\checkcanbeinitializedwith(\overname{\staticenvs}{\tenv} \aslsep \overname{\ty}{\vs} \aslsep \overname{\ty}{\vt})
\typearrow \{\True\} \cup \overname{\TTypeError}{\TypeErrorConfig}
\]
checks whether an expression of type $\vs$ can be used to initialize a storage element of type $\vt$ in the static environment
$\tenv$.
If the answer if $\False$, the result is a type error.

\subsubsection{Prose}
One of the following applies:
\begin{itemize}
  \item All of the following apply (\textsc{okay}):
  \begin{itemize}
    \item testing whether $\vt$ \typesatisfies\ $\vs$ in $\tenv$ yields $\True$;
    \item the result is $\True$.
  \end{itemize}

  \item All of the following apply (\textsc{error}):
  \begin{itemize}
    \item testing whether $\vt$ \typesatisfies\ $\vs$ in $\tenv$ yields $\False$;
    \item the result is a type error indicating that an expression of type $\vs$ cannot
          be used to initialize a storage element of type $\vt$.
  \end{itemize}
\end{itemize}

\subsubsection{Formally}
\begin{mathpar}
\inferrule[okay]{
  \typesat(\tenv, \vt, \vs) \typearrow \True
}{
  \checkcanbeinitializedwith(\tenv, \vs, \vt) \typearrow \True
}
\end{mathpar}

\begin{mathpar}
\inferrule[error]{
  \typesat(\tenv, \vt, \vs) \typearrow \False
}{
  \checkcanbeinitializedwith(\tenv, \vs, \vt) \typearrow \TypeErrorVal{CannotBeInitializedWith}
}
\end{mathpar}
\lrmcomment{This is related to \identr{ZCVD} and \identr{LXQZ}.}

\subsection{Semantics}
\subsubsection{SemanticsRule.LDTyped\label{sec:SemanticsRule.LDTyped}}
\subsubsection{Example (Initialized)}
In the specification:
\VerbatimInput{../tests/ASLSemanticsReference.t/SemanticsRule.LDTyped.asl}
\texttt{var x : integer = 42;} binds \texttt{x} in $\env$ to $\nvint(42)$.

\subsubsection{Example (Uninitialized)}
In the specification:
\VerbatimInput{../tests/ASLSemanticsReference.t/SemanticsRule.LDUninitialisedTyped.asl}
\verb|var x : integer{3..43};| binds \texttt{x} in $\env$ to the base value of \verb|integer{3..43}|,
which is $\nvint(3)$.

\subsubsection{Prose}
One of the following applies:
\begin{itemize}
    \item All of the following apply (\textsc{initialized}):
    \begin{itemize}
        \item $\ldi$ is a typed declaration, $\LDITyped(\ldione, \vt)$;
        \item $\minitopt$ corresponds to the initializing value $\vm$;
        \item the resulting configuration is obtained via the evaluation
        of the local declaration $\ldione$ in $\env$ with $\minitopt$ as $\vm$,
        that is, \\ $\evallocaldecl{\env, \ldi1, \langle \vm\rangle}$.
    \end{itemize}

    \item All of the following apply (\textsc{uninitialized}):
    \begin{itemize}
      \item $\ldi$ gives a local declaration with a type, but no initial value, \\
            $\LDITyped(\ldione, \vt)$;
      \item $\minitopt$ is $\None$;
      \item the base value of $\vt$ is $\vm$\ProseOrError;
      \item evaluating the local declaration $\ldione$ with $\vm$
            as the $\minitopt$ component yields the output configuration.
    \end{itemize}
\end{itemize}
\subsubsection{Formally}
\begin{mathpar}
\inferrule[initialized]{
  \evallocaldecl{\env, \ldi1, \langle \vm\rangle} \evalarrow C
}{
  \evallocaldecl{\env, \LDITyped(\ldi1, \Ignore), \langle \vm\rangle} \evalarrow C
}
\end{mathpar}

\begin{mathpar}
\inferrule[uninitialized]{
    \basevalue(\env, \vt) \evalarrow \vm \OrDynError\\\\
    \evallocaldecl{\env, \ldione, \langle \vm \rangle} \evalarrow C
}{
    \evallocaldecl{\env, \LDITyped(\ldione, \vt), \None} \evalarrow C
}
\end{mathpar}
\CodeSubsection{\EvalLDTypedBegin}{\EvalLDTypedEnd}{../Interpreter.ml}

\section{Tuple Declarations\label{sec:TupleDeclarations}}
\subsection{Typing}
\subsubsection{TypingRule.LDTuple\label{sec:TypingRule.LDTuple}}
\subsubsection{Example}
\VerbatimInput{../tests/ASLTypingReference.t/TypingRule.LDTuple.asl}

\subsubsection{Prose}
All of the following apply:
\begin{itemize}
  \item $\ldi$ denotes a tuple of local declaration items $\ldi_{1..k}$, that is, $\LDITuple(\ldi_{1..k})$;
  \item determining the \structure\ of $\tty$ in $\tenv$ yields $\vtp$\ProseOrTypeError;
  \item determining whether $\vtp$ is a tuple type yields $\True$\ProseOrTypeError;
  \item determining whether $\vtp$ the number of elements of $\vtp$ is $k$ yields $\True$\ProseOrTypeError;
  \item annotating the local declaration items in $\ldis$ from right to left with their corresponding
        (that is, with the same index) types $t_{1..k}$ in $\tenv$,
        propagating static environments from one annotation to the next,
        yields the local declaration items $\ldip_{1..k}$\ProseOrTypeError;
  \item $\newtenv$ is the static environment yielded by annotating $\ldi_1$;
  \item $\newldi$ is a tuple of local declaration items with $\ldip_{1..k}$, that is, \\
        $\LDITuple(\ldip_{1..k})$.
\end{itemize}
\subsubsection{Formally}
\begin{mathpar}
\inferrule{
  \tstruct(\tenv, \tty) \typearrow \vtp \OrTypeError\\\\
  \checktrans{\astlabel(\vtp) = \TTuple}{TupleTypeExpected} \checktransarrow \True \OrTypeError\\\\
  \vtp \eqname \TTuple([\vt_{1..n}])\\\\
  \checktrans{k = n}{InvalidArity} \checktransarrow \True \OrTypeError\\\\
  \newtenv_k = \tenv\\
  {
    \begin{array}{r}
  i=k..1:
  \annotatelocaldeclitem{\newtenv_{i}, \vt_{i}, \ldk, \None, \ldi_{i}} \typearrow \\
  (\newtenv_{i-1}, \ldip_i) \OrTypeError
    \end{array}
  }\\
  \newtenv = \newtenv_0
}{
  \annotatelocaldeclitem{\tenv, \tty, \ldk, \veopt, \overname{\LDITuple(\ldi_{1..k})}{\ldi}} \typearrow \\
  (\newtenv, \LDITuple(\ldip_{1..k}))
}
\end{mathpar}
\CodeSubsection{\LDTupleBegin}{\LDTupleEnd}{../Typing.ml}

\subsection{Semantics}
\subsubsection{SemanticsRule.LDTuple\label{sec:SemanticsRule.LDTuple}}
\subsubsection{Example}
In the specification:
\VerbatimInput{../tests/ASLSemanticsReference.t/SemanticsRule.LDTuple.asl}
\texttt{var (x,y,z) = (1,2,3);} binds \texttt{x} to the evaluation of
\texttt{1}, \texttt{y} to the evaluation of \texttt{2}, and \texttt{z} to the
evaluation of \texttt{3} in $\env$.

\subsubsection{Prose}
All of the following apply:
\begin{itemize}
  \item $\ldi$ declares a list of local variables, $\LDITuple(\ldis)$;
  \item $\minitopt$ is $\vm$;
  \item $\vm$ is a pair consisting of the native vector $\vv$ and execution graph $\vg$;
  \item $\ldis$ is a list of local declaration items $\ldi_{1..k}$;
  \item the value at each index of $\vv$ is $\vv_i$, for $i=1..k$;
  \item $\liv$ is the list of pairs $(\vv_i, \vg)$, for $i=1..k$;
  \item the output configuration is obtained by declare each local declaration item $\ldi_i$
  with the corresponding value ($\minitopt$ component) $(\vv_i, \vg)$.
\end{itemize}
\subsubsection{Formally}
\hypertarget{def-ldituplefolder}{}
We first define the helper semantic relation
\[
    \ldituplefolder(\overname{\envs}{\env} \aslsep \overname{\localdeclitem^*}{\ldis} \aslsep \overname{(\vals \times \XGraphs)^*}{\liv}) \;\aslrel\;
     \Normal(\overname{\XGraphs}{\vg} \aslsep \overname{\envs}{\newenv})
\]
via the following rules:
\begin{mathpar}
\inferrule{}
{
  \ldituplefolder(\env, \emptylist, \emptylist) \evalarrow \Normal(\emptygraph, \env)
}
\and
\inferrule{
  \ldis \eqname [\ldi] \concat \ldis'\\
  \liv \eqname [\vm] \concat \liv'\\
  \vm \eqname (\vv, \vgone)\\
  \evallocaldecl{\env, \ldi, \langle\vm\rangle} \evalarrow \Normal(\vgone, \envone)\\
  \ldituplefolder(\envone, \ldis', \liv') \evalarrow \Normal(\vgtwo, \newenv)\\
  \newg \eqdef \vgone \parallelcomp \vgtwo
}{
  \ldituplefolder(\env, \ldis, \liv) \evalarrow \Normal(\newg, \newenv)
}
\end{mathpar}

We now use the helper rules to define the rule for local declaration item tuples:
\begin{mathpar}
\inferrule{
  \vm \eqname (\vv, \vg)\\
  \ldis \eqname \ldi_{1..k}\\
  i=1..k: \getindex(i, \vv) \evalarrow \vv_i\\
  \liv \eqname [i=1..k: (\vv_i, \vg)]\\
  \ldituplefolder(\env, \ldis, \liv) \evalarrow C
}{
  \evallocaldecl{\env, \LDITuple(\ldis), \langle \vm\rangle} \evalarrow C
}
\end{mathpar}
\CodeSubsection{\EvalLDTupleBegin}{\EvalLDTupleEnd}{../Interpreter.ml}