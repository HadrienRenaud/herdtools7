\chapter{Side Effects\label{chap:SideEffects}}

This chapter defines a static \emph{side effect analysis}.
The analysis aims to \emph{soundly} answer which pairs of expressions are \emph{order independent}.
By \emph{soundly}, we mean that the analysis proves that a \underline{sufficient condition}
for \emph{order independence} holds.

Intuitively, two expressions --- $\veone$ and $\vetwo$ --- are \emph{order independent}
if they can be evaluated in any order (first $\veone$ and then $\vetwo$ or first $\vetwo$ and then $\veone$),
starting from any initial environment $\env$,
and terminate in the same configuration.

To formalize this intuition, we need to consider the following:
\begin{description}
    \item[Evaluation] We employ $\evalexprlist{\cdot}$ to evaluate a list of expressions;

    \item[Non-determinism] Recall that the semantics is non-deterministic, which means evaluation of any
    expression in the same environment may terminate with different configurations. We therefore compare
    the \underline{sets} of output configurations resulting from each order of evaluation.

    \item[Order Sensitivity] evaluating two expressions yields a list consisting of two
    corresponding values. Further, evaluating two expressions in two different orders results in the values
    appearing in reverse order relative to one another. Therefore to enable comparison, we need to reverse one of
    the lists (see $\reversevals$ below).

    \item[Dynamic Errors] if both orders of evaluation terminate in dynamic errors, we consider them
    equivalent even if the error codes are different. This abstraction is carried out by $\abstractdynerror$.
    We ignore all dynamic errors that are not assertions, by abstracting them away.
\end{description}

\begin{definition}[Order Independence]
Two expressions $\veone$ and $\vetwo$ are \emph{order independent}
if the following condition holds for every
environment $\env \in \dynamicenvs$:
\[
\begin{array}{l}
\{\ \abstractdynerror(C) \;|\; \evalexprlist{\env, [\veone, \vetwo]} \evalarrow C\ \} \setminus \{\bot\} = \\
\{\ \abstractdynerror(C) \;|\; \reversevals(\evalexprlist{\env, [\vetwo, \veone]}) \evalarrow C\ \} \setminus \{\bot\} \enspace.
\end{array}
\]
\hypertarget{def-reversevals}{}
where $\reversevals$ reverses the order of values resulting from evaluating $\vetwo$ first
and $\veone$ second into configurations resulting from evaluating $\veone$ first
and $\vetwo$ second, leaving abnormal configurations unchanged:
\[
    \reversevals(C) \triangleq \begin{cases}
        \Normal(([\vvone, \vvtwo], \vg), \env) & \text{if }C=\Normal(([\vvtwo, \vvone], \vg), \env)\\
        C & \text{else}
    \end{cases}
\]
\hypertarget{def-abstractdynerror}{}
and $\abstractdynerror$ abstracts assertion errors into $\top$,
all other dynamic errors into $\bot$ (which are ignored)
and leaves all other configurations unchanged:
\[
\abstractdynerror(C) \triangleq \begin{cases}
    C       & \text{if }\configdomain{C} \in \{\Normal, \Throwing\}\\
    \top    & \text{if }C \in \{ \DynamicErrorVal{\DynamicAssertionFailure}, \DynamicErrorVal{\DynamicTypeAssertionFailure} \}\\
    \bot    & \text{else}\\
\end{cases}
\]
That is, evaluating the expressions results in the same set of configurations,
up to the order of values, and ignoring dynamic errors that are not due to assertions.
\end{definition}

Along the way, we also define the concept of \emph{pure expressions} and \staticallyevaluable\ expressions.

%%%%%%%%%%%%%%%%%%%%%%%%%%%%%%%%%%%%%%%%%%%%%%%%%%%%%%%%%%%%%%%%%%%%%%%%%%%%%%%%
\section{Time Frames\label{sec:TimeFrames}}
%%%%%%%%%%%%%%%%%%%%%%%%%%%%%%%%%%%%%%%%%%%%%%%%%%%%%%%%%%%%%%%%%%%%%%%%%%%%%%%%
\hypertarget{def-timeframe}{}
We divide side effects by \emph{\timeframesterm}, which indicate the phase where a side effect occurs:
\begin{description}
    \item[Constant] Contains effects that take place during static evaluation (see \chapref{staticevaluation}). That is, during type-checking.
    \item[Configuration] Contains effects that take place while rewriting the initializing value of a \texttt{config} storage element.
    \item[Execution] Contains effects that take place during semantic evaluation.
\end{description}

Formally, \timeframesterm\ are totally ordered via $\timeframeless$ as follows:
\hypertarget{def-timeframetype}{}
\hypertarget{def-timeframeless}{}
\hypertarget{def-timeframeconstant}{}
\hypertarget{def-timeframeexecution}{}
\hypertarget{def-timeframeconfig}{}
\[
\TTimeFrame \triangleq \{ \timeframeconstant \timeframeless \timeframeconfig \timeframeless \timeframeexecution \}
\]
Additionally, we define the less-than-or-equal ordering as follows:
\hypertarget{def-timeframeleq}{}
\[
f \timeframeleq f' \triangleq f \timeframeless f' \lor f = f' \enspace.
\]

We now define some helper functions for constructing time frames.

\hypertarget{def-timeframemax}{}
We define the maximum of a set of time frames $\timeframemax : \pow{\TTimeFrame} \rightarrow \TTimeFrame$
as follows:
\[
    \timeframemax(T) \triangleq t \text{ such that } t\in T \land \forall t'\in T.\ t' \timeframeleq t \enspace.
\]

\subsubsection{TypingRule.TimeFrameLDK\label{sec:TypingRule.TimeFrameLDK}}
\hypertarget{def-timeframeofldk}{}
The function
\[
    \timeframeofldk(\overname{\localdeclkeyword}{\ldk}) \aslto \overname{\TTimeFrame}{\vt}
\]
constructs a \timeframeterm\ $\vt$ from a local declaration keyword $\ldk$.

\subsubsection{Formally}
\begin{mathpar}
\inferrule{}{
    {
        \timeframeofldk(\ldk) \typearrow
    \begin{cases}
        \timeframeconstant & \text{if }\ldk = \LDKConstant\\
        \timeframeexecution & \text{if }\ldk = \LDKLet\\
        \timeframeexecution & \text{if }\ldk = \LDKVar\\
    \end{cases}
    }
}
\end{mathpar}

\subsubsection{TypingRule.TimeFrameGDK\label{sec:TypingRule.TimeFrameGDK}}
\hypertarget{def-timeframeofgdk}{}
The function
\[
    \timeframeofgdk(\overname{\globaldeclkeyword}{\gdk}) \aslto \TTimeFrame
\]
constructs a \timeframeterm\ $\vt$ from a global declaration keyword $\gdk$.

\subsubsection{Formally}
\begin{mathpar}
\inferrule{}{
    {
        \timeframeofgdk(\gdk) \typearrow
    \begin{cases}
        \timeframeconstant  & \text{if }\gdk = \GDKConstant\\
        \timeframeconfig & \text{if }\gdk = \GDKConfig\\
        \timeframeexecution & \text{if }\gdk = \GDKLet\\
        \timeframeexecution & \text{if }\gdk = \GDKVar\\
    \end{cases}
    }
}
\end{mathpar}

%%%%%%%%%%%%%%%%%%%%%%%%%%%%%%%%%%%%%%%%%%%%%%%%%%%%%%%%%%%%%%%%%%%%%%%%%%%%%%%%
\section{Side Effect Descriptors\label{sec:SideEffectDescriptors}}
%%%%%%%%%%%%%%%%%%%%%%%%%%%%%%%%%%%%%%%%%%%%%%%%%%%%%%%%%%%%%%%%%%%%%%%%%%%%%%%%

\hypertarget{def-sideeffectdescriptorterm}{}
We now define \sideeffectdescriptorsterm,
which are configurations used to describe side effects, as explained below:
\hypertarget{def-tsideeffect}{}
\hypertarget{def-readlocal}{}
\[
\TSideEffect \triangleq \left\lbrace
\begin{array}{ll}
    \ReadLocal(\overname{\identifier}{\vx}, \overname{\TTimeFrame}{\vt}, \overname{\Bool}{\vimmutable})     & \cup
    \hypertarget{def-writelocal}{}\\
    \WriteLocal(\overname{\identifier}{\vx})                                            & \cup
    \hypertarget{def-readglobal}{}\\
    \ReadGlobal(\overname{\identifier}{\vx}, \overname{\TTimeFrame}{\vt}, \overname{\Bool}{\vimmutable})    & \cup
    \hypertarget{def-writeglobal}{}\\
    \WriteGlobal(\overname{\identifier}{\vx})                                           & \cup
    \hypertarget{def-throwexception}{}\\
    \ThrowException(\overname{\identifier}{\vx})                                        & \cup
    \hypertarget{def-recursivecall}{}\\
    \RecursiveCall(\overname{\identifier}{\vf})                                         & \cup
    \hypertarget{def-performsassertions}{}\\
    \PerformsAssertions                                                 & \cup
    \hypertarget{def-nondeterministic}{}\\
    \NonDeterministic                                                   &
\end{array} \right.
\]
\hypertarget{def-readlocalterm}{}
\begin{description}
    \item[$\ReadLocal$] a \ReadLocalTerm\ describes an evaluation of a construct that leads to reading the value of the local storage element
        $\vx$ at the \timeframeterm\ $\vt$ where $\vimmutable$ is $\True$ if and only if $\vx$
        was declared as an immutable local storage element (that is, \texttt{constant} or \texttt{let});
    \hypertarget{def-writelocalterm}{}
    \item[$\WriteLocal$] a \WriteLocalTerm\ describes an evaluation of a construct that leads to modifying the value of the local storage element
        $\vx$;
    \hypertarget{def-readglobalterm}{}
    \item[$\ReadGlobal$] a \ReadGlobalTerm\ describes an evaluation of a construct that leads to reading the value of the global storage element
        $\vx$ at the \timeframeterm\ $\vt$ where $\vimmutable$ is $\True$ if and only if $\vx$
        was declared as an immutable local storage element (that is, \texttt{constant}, \texttt{config}, or \texttt{let});
    \hypertarget{def-writeglobalterm}{}
    \item[$\WriteGlobal$] a \WriteGlobalTerm\ describes an evaluation of a construct that leads to modifying the value of the global storage element
        $\vx$;
    \hypertarget{def-throwexceptionterm}{}
    \item[$\ThrowException$] an \ThrowExceptionTerm\ describes an evaluation of a construct that leads to raising an exception whose type
        is named $\vx$;
    \hypertarget{def-recursivecallterm}{}
    \item[$\RecursiveCall$] a \RecursiveCallTerm\ describes an evaluation of a construct that leads to calling the recursive function $\vf$;
    \hypertarget{def-performsassertionsterm}{}
    \item[$\PerformsAssertions$] a \PerformsAssertionsTerm\ describes an evaluation of a construct that leads to evaluating an \texttt{assert} statement;
    \hypertarget{def-nondeterministicterm}{}
    \item[$\NonDeterministic$] a \NonDeterministicTerm\ describes an evaluation of a construct that leads to evaluating a non-deterministic
        expression (either \ARBITRARY\ or a library function call known to be non-deterministic).
\end{description}

We now define a few helper functions over \timeframesterm.

\subsubsection{TypingRule.TimeFrame\label{sec:TypingRule.TimeFrame}}
\hypertarget{def-sideeffecttimeframe}{}
The function
\[
    \timeframe(\overname{\TSideEffect}{\vs}) \aslto \overname{\TTimeFrame}{\vt}
\]
retrieves the \timeframeterm\ $\vt$ from a \sideeffectdescriptorterm\ $\vs$.

\subsubsection{Formally}
\begin{mathpar}
\inferrule{}{
    \timeframe(\overname{\ReadLocal(\Ignore, \vt, \Ignore)}{\vs}) \typearrow \vt
}
\and
\inferrule{}{
    \timeframe(\overname{\ReadGlobal(\Ignore, \vt, \Ignore)}{\vs}) \typearrow \vt
}
\end{mathpar}

\begin{mathpar}
\inferrule{
    {
    \configdomain{\vs} \in \left\{
        \begin{array}{c}
            \WriteLocal, \\
            \WriteGlobal, \\
            \NonDeterministic, \\
            \RecursiveCall, \\
            \ThrowException
        \end{array}
    \right\}
    }
}{
    \timeframe(\vs) \typearrow \overname{\timeframeexecution}{\vt}
}
\end{mathpar}

\begin{mathpar}
\inferrule{}{
    \timeframe(\overname{\PerformsAssertions}{\vs}) \typearrow \overname{\timeframeconstant}{\vt}
}
\end{mathpar}

\subsubsection{TypingRule.SideEffectIsPure\label{sec:TypingRule.SideEffectIsPure}}
\hypertarget{def-sideeffectispure}{}
\[
    \sideeffectispure(\overname{\TSideEffect}{\vs}) \aslto \overname{\Bool}{\vb}
\]
defines whether a \sideeffectdescriptorsterm\ $\vs$ is considered \emph{pure},
yielding the result in $\vb$.
Intuitively, a \emph{pure} \sideeffectdescriptorterm\ helps to establish that
an expression evaluates without modifying values of storage elements.

\subsubsection{Prose}
Define $\vb$ as $\True$ if and only if $\vt$ is either a \ReadLocalTerm, a \ReadGlobalTerm,
a \NonDeterministicTerm, or a \PerformsAssertionsTerm.

\subsubsection{Formally}
\begin{mathpar}
\inferrule{
    \vb \eqdef \configdomain{\vs} \in \{\ReadLocal, \ReadGlobal, \NonDeterministic, \PerformsAssertions\}
}{
    \sideeffectispure(\vt) \typearrow \vb
}
\end{mathpar}

\subsubsection{TypingRule.SideEffectIsStaticallyEvaluable\label{sec:TypingRule.SideEffectIsStaticallyEvaluable}}
\hypertarget{def-sideeffectisstaticallyevaluable}{}
\[
    \sideeffectisstaticallyevaluable(\overname{\TSideEffect}{\vs}) \aslto \overname{\Bool}{\vb}
\]
defines whether a \sideeffectdescriptorsterm\ $\vs$ is considered \emph{\staticallyevaluable},
yielding the result in $\vb$.
Intuitively, a \emph{statically evaluable} \sideeffectdescriptorterm\ helps establish that
an expression evaluates without failing assertions, without modifying any storage element,
and always yielding the same result, that is, deterministically.

\subsubsection{Prose}
Define $\vb$ as $\True$ if and only if $\vs$ is either
a \ReadLocalTerm\ associated with an immutable storage element, or
a \ReadGlobalTerm\ associated with an immutable storage element.

\subsubsection{Formally}
\begin{mathpar}
\inferrule{
    \vb \eqdef \vs = \ReadLocal(\Ignore, \Ignore, \True) \lor \vs = \ReadGlobal(\Ignore, \Ignore, \True)
}{
    \sideeffectisstaticallyevaluable(\vs) \typearrow \vb
}
\end{mathpar}

\subsubsection{TypingRule.SideEffectConflict\label{sec:TypingRule.SideEffectConflict}}
% Transliteration note: this function is not reflected in the implementation as it checks conflicts directly on sets of side effects.
\hypertarget{def-sideeffectconflict}{}
\hypertarget{def-sideeffectconflictterm}{}
The function
\[
\sideeffectconflict(\overname{\TSideEffect}{\vsone}, \overname{\TSideEffect}{\vstwo}) \aslto \overname{\Bool}{\vb}
\]
defines whether there exists a \emph{\sideeffectconflictterm} between the \sideeffectdescriptorterm\ $\vsone$
and the \sideeffectdescriptorterm\ $\vstwo$, yielding the result in $\vb$.

\subsubsection{Prose}
One of the following applies:
\begin{itemize}
    \item All of the following apply (\textsc{globalread}):
    \begin{itemize}
        \item $\vsone$ is a \ReadGlobalTerm\ for a storage element named $\id$;
        \item define $\vb$ as $\True$ if and only if $\vstwo$ is either
              a \WriteGlobalTerm\ for a storage element named $\id$ or a \RecursiveCallTerm.
    \end{itemize}

    \item All of the following apply (\textsc{globalwrite}):
    \begin{itemize}
        \item $\vsone$ is a \WriteGlobalTerm\ for a storage element named $\id$;
        \item define $\vb$ as $\True$ if and only if $\vstwo$ is either
              a \WriteGlobalTerm\ for a storage element named $\id$,
              a \ReadGlobalTerm\ for a storage element named $\id$,
              an \ThrowExceptionTerm, or
              a \RecursiveCallTerm.
    \end{itemize}

    \item All of the following apply (\textsc{exception}):
    \begin{itemize}
        \item $\vsone$ is an \ThrowExceptionTerm;
        \item define $\vb$ as $\True$ if and only if $\vstwo$ is either
                an \ThrowExceptionTerm,
                a \WriteLocalTerm,
                a \PerformsAssertionsTerm, or a
                \RecursiveCallTerm.
    \end{itemize}

    \item All of the following apply (\textsc{localread}):
    \begin{itemize}
        \item $\vsone$ is a \ReadLocalTerm\ for a storage element named $\id$;
        \item define $\vb$ as $\True$ if and only if $\vstwo$ is either
                a \WriteLocalTerm\ for a storage element named $\id$, or a
                \RecursiveCallTerm.
    \end{itemize}

    \item All of the following apply (\textsc{localwrite}):
    \begin{itemize}
        \item $\vsone$ is a \WriteLocalTerm\ for a storage element named $\id$;
        \item define $\vb$ as $\True$ if and only if $\vstwo$ is either
              a \ReadLocalTerm\ for a storage element named $\id$,
              a \WriteLocalTerm\ for a storage element named $\id$, or a
                \RecursiveCallTerm.
    \end{itemize}

    \item All of the following apply (\textsc{assertion}):
    \begin{itemize}
        \item $\vsone$ is a \PerformsAssertionsTerm;
        \item define $\vb$ as $\True$ if and only if $\vstwo$ is either
              a \PerformsAssertionsTerm\, or a
            \RecursiveCallTerm.
    \end{itemize}

    \item All of the following apply (\textsc{nondeterminism}):
    \begin{itemize}
        \item $\vsone$ is a \NonDeterministicTerm;
        \item define $\vb$ as $\False$.
    \end{itemize}

    \item All of the following apply (\textsc{recursion}):
    \begin{itemize}
        \item $\vsone$ is a \RecursiveCallTerm;
        \item define $\vb$ as $\True$ if and only if $\vstwo$ is not a \NonDeterministicTerm.
    \end{itemize}
\end{itemize}

\subsubsection{Formally}
\begin{mathpar}
\inferrule[globalread]{
    \vb \eqdef \vstwo = \WriteGlobal(\id, \Ignore, \Ignore) \lor \configdomain{\vstwo} = \RecursiveCall
}{
    \sideeffectconflict(\overname{\ReadGlobal(\id, \Ignore, \Ignore)}{\vsone}, \vstwo) \typearrow \vb
}
\end{mathpar}

\begin{mathpar}
\inferrule[globalwrite]{
    {
        \vb \eqdef \left\lbrace
        \begin{array}{ll}
            \vstwo = \WriteGlobal(\id, \Ignore, \Ignore) & \lor\\
            \vstwo = \ReadGlobal(\id, \Ignore, \Ignore)  & \lor\\
            \configdomain{\vstwo} \in \{\ThrowException, \RecursiveCall\} &
        \end{array}\right.
    }
}{
    \sideeffectconflict(\overname{\WriteGlobal(\id, \Ignore, \Ignore)}{\vsone}, \vstwo) \typearrow \vb
}
\end{mathpar}

\begin{mathpar}
\inferrule[exception]{
    \vb \eqdef \configdomain{\vstwo} \in \{\ThrowException, \WriteLocal, \PerformsAssertions, \RecursiveCall\}
}{
    \sideeffectconflict(\overname{\ThrowException}{\vsone}, \vstwo) \typearrow \vb
}
\end{mathpar}

\begin{mathpar}
\inferrule[localread]{
    \vb \eqdef \vstwo = \WriteLocal(\id, \Ignore, \Ignore) \lor \configdomain{\vstwo} = \RecursiveCall
}{
    \sideeffectconflict(\overname{\ReadLocal(\id, \Ignore, \Ignore)}{\vsone}, \vstwo) \typearrow \vb
}
\end{mathpar}

\begin{mathpar}
\inferrule[localwrite]{
    {
    \vb \eqdef \left\lbrace\begin{array}{ll}
        \vstwo = \ReadLocal(\id, \Ignore, \Ignore)  & \lor\\
        \vstwo = \WriteLocal(\id, \Ignore, \Ignore) & \lor \\
        \configdomain{\vstwo} = \RecursiveCall      &
    \end{array}\right.
    }
}{
    \sideeffectconflict(\overname{\WriteLocal(\id, \Ignore, \Ignore)}{\vsone}, \vstwo) \typearrow \vb
}
\end{mathpar}

\begin{mathpar}
\inferrule[assertion]{
    \vb \eqdef \configdomain{\vstwo} \in \{\PerformsAssertions, \RecursiveCall\}
}{
    \sideeffectconflict(\overname{\PerformsAssertions}{\vsone}, \vstwo) \typearrow \vb
}
\end{mathpar}

\begin{mathpar}
\inferrule[nondeterminism]{}{
    \sideeffectconflict(\overname{\NonDeterministic}{\vsone}, \vstwo) \typearrow \overname{\False}{\vb}
}
\end{mathpar}

\begin{mathpar}
\inferrule[recursion]{}{
    \sideeffectconflict(\overname{\RecursiveCall}{\vsone}, \vstwo) \typearrow \overname{\vstwo \neq \NonDeterministic}{\vb}
}
\end{mathpar}

\subsubsection{TypingRule.LDKIsImmutable\label{sec:TypingRule.LDKIsImmutable}}
\hypertarget{def-ldkisimmutable}{}
The function
\[
\ldkisimmutable(\overname{\localdeclkeyword}{\ldk}) \typearrow \overname{\True}{\vb}
\]
tests whether the local declaration keyword $\ldk$ relates to an immutable storage element,
yielding the result in $\vb$.

\subsubsection{Prose}
Define $\vb$ as $\True$ if and only if $\ldk$ corresponds to either the keyword \texttt{constant} or
the keyword \texttt{let}.

\subsubsection{Formally}
\begin{mathpar}
\inferrule{}{
  \ldkisimmutable(\ldk) \typearrow \overname{\ldk \in \{\LDKConstant, \LDKLet\}}{}
}
\end{mathpar}

\subsubsection{TypingRule.GDKIsImmutable\label{sec:TypingRule.GDKIsImmutable}}
\hypertarget{def-gdkisimmutable}{}
The function
\[
\gdkisimmutable(\overname{\globaldeclkeyword}{\gdk}) \typearrow \overname{\True}{\vb}
\]
tests whether the global declaration keyword $\gdk$ relates to an immutable storage element,
yielding the result in $\vb$.

\subsubsection{Prose}
Define $\vb$ as $\True$ if and only if $\gdk$ corresponds to either the keyword \texttt{constant},
the keyword \texttt{config}, or the keyword \texttt{let}.

\subsubsection{Formally}
\begin{mathpar}
\inferrule{}{
  \gdkisimmutable(\gdk) \typearrow \overname{\gdk \in \{\GDKConstant, \GDKConfig, \GDKLet\}}{}
}
\end{mathpar}

%%%%%%%%%%%%%%%%%%%%%%%%%%%%%%%%%%%%%%%%%%%%%%%%%%%%%%%%%%%%%%%%%%%%%%%%%%%%%%%%
\section{Side Effect Sets\label{sec:SideEffectSets}}
%%%%%%%%%%%%%%%%%%%%%%%%%%%%%%%%%%%%%%%%%%%%%%%%%%%%%%%%%%%%%%%%%%%%%%%%%%%%%%%%

\subsubsection{TypingRule.AreNonConflicting\label{sec:TypingRule.AreNonConflicting}}
\hypertarget{def-arenonconflicting}{}
The function
\[
    \arenonconflicting(\overname{\TSideEffectSet}{\vsesone}, \overname{\TSideEffectSet}{\vsestwo})
    \aslto \overname{\Bool}{\vb}
\]
tests whether there are no \sideeffectconflictterm\ between the set of
\sideeffectdescriptorsterm\ $\vsesone$ and the set of \sideeffectdescriptorsterm\ $\vsestwo$,
yielding the result in $\vb$.

\subsubsection{Prose}
Define $\vb$ as $\True$ if for every \sideeffectdescriptorterm\ $\vsone$ in $\vsesone$ and
every \sideeffectdescriptorterm\ $\vstwo$ in $\vsestwo$,
testing $\sideeffectconflict$ for $\vsone$ and $\vstwo$ yields $\False$.

\subsubsection{Formally}
\begin{mathpar}
\inferrule{
    \vbp \eqdef \bigvee_{\vsone \in \vsesone, \vstwo \in \vsestwo} \sideeffectconflict(\vsone, \vstwo)
}{
    \arenonconflicting(\vsesone, \vsestwo) \typearrow \overname{\neg \vbp}{\vb}
}
\end{mathpar}

\subsubsection{TypingRule.NonConflictingUnion\label{sec:TypingRule.NonConflictingUnion}}
\hypertarget{def-nonconflictingunion}{}
The function
\[
    \nonconflictingunion(\overname{\TSideEffectSet}{\vsesone} \aslsep \overname{\TSideEffectSet}{\vsestwo})
    \aslto \overname{\TSideEffectSet}{\vses} \cup \overname{\TTypeError}{\TypeErrorConfig}
\]
yields the union of the two sets of \sideeffectdescriptorsterm\ $\vsesone$ and $\vsestwo$ in $\vses$
if they are \hyperlink{def-sideeffectconflictterm}{non-conflicting}. \ProseOtherwiseTypeError

\subsubsection{Prose}
All of the following apply:
\begin{itemize}
    \item checking whether $\arenonconflicting$ yields $\True$ for $\vsesone$ and $\vsestwo$ yields\\
         $\True$\ProseTerminateAs{\ConflictingSideEffects};
    \item define $\vses$ as the union of $\vsesone$ and $\vsestwo$.
\end{itemize}

\subsubsection{Formally}
\begin{mathpar}
\inferrule{
    \checktrans{\arenonconflicting(\vsesone, \vsestwo)}{\ConflictingSideEffects} \typearrow \True \OrTypeError
}{
    \nonconflictingunion(\vsesone, \vsestwo) \typearrow \overname{\vsesone \cup \vsestwo}{\vses}
}
\end{mathpar}

\subsubsection{TypingRule.MaxTimeFrame\label{sec:TypingRule.MaxTimeFrame}}
\hypertarget{def-maxtimeframe}{}
The function
\[
    \maxtimeframe(\overname{\TSideEffectSet}{\vses}) \aslto \overname{\TTimeFrame}{\vf}
\]
defines the maximal \timeframeterm\ for a \sideeffectsetterm\ $\vses$, which is returned
in $\vf$.
(If $\vses$ is empty, the result is \timeframeconstant.)

\subsubsection{Prose}
One of the following applies:
\begin{itemize}
    \item All of the following apply (\textsc{execution}):
    \begin{itemize}
        \item there exists a \sideeffectdescriptorterm\ in $\vses$ that is either
            a \WriteLocalTerm, a \WriteGlobalTerm, an \ThrowExceptionTerm, a \RecursiveCallTerm, or
            a \NonDeterministicTerm;
        \item define $\vf$ as \timeframeexecution.
    \end{itemize}

    \item All of the following apply (\textsc{reads}):
    \begin{itemize}
        \item define $\vreads$ as the subset of $\vses$ that contains only
            \sideeffectdescriptorsterm\ that are either \ReadLocalTerm\ or \ReadGlobalTerm;
        \item $\vreads$ is equal to $\vses$;
        \item define $\vtimeframes$ as the \timeframesterm\ appearing in the \sideeffectdescriptorsterm\
            in $\vreads$;
        \item define $\vf$ as the greatest time frame in the union of $\vtimeframes$ and the singleton set for \timeframeconstant,
              where $\timeframeleq$ is used to compare any two \timeframesterm.
    \end{itemize}
\end{itemize}

\subsubsection{Formally}
\begin{mathpar}
\inferrule[execution]{
    {
    \exists \vs \in \vses.\ \configdomain{\vs} \in \left\{
        \begin{array}{c}
            \WriteLocal, \\
            \WriteGlobal, \\
            \ThrowException, \\
            \RecursiveCall, \\
            \NonDeterministic
        \end{array}
        \right\}
    }
}{
    \maxtimeframe(\vses) \typearrow \overname{\timeframeexecution}{\vf}
}
\end{mathpar}

\begin{mathpar}
\inferrule[reads]{
    \vreads \eqdef \{ \vs \in \vses \;|\; \configdomain{\vs} \in \{\ReadLocal, \ReadGlobal\} \}\\
    \vses = \vreads\\
    \vtimeframes \eqdef \{ \timeframe(\vfp) \;|\; \vfp\in\vreads \} \\
    \vf \eqdef \timeframemax(\vtimeframes \cup \{\timeframeconstant\})
}{
    \maxtimeframe(\vses) \typearrow \vf
}
\end{mathpar}

\subsubsection{TypingRule.SESIsStaticallyEvaluable\label{sec:TypingRule.SESIsStaticallyEvaluable}}
\hypertarget{def-isstaticallyevaluable}{}
\hypertarget{def-staticallyevaluable}{}
The function
\[
  \isstaticallyevaluable(\overname{\TSideEffectSet}{\vses}) \aslto \overname{\Bool}{\bv}
\]
tests whether a set of \sideeffectdescriptorsterm\ $\vses$ are all \staticallyevaluable,
yielding the result in $\vb$.

\subsubsection{Prose}
Define $\vb$ as $\True$ if and only if every \sideeffectdescriptorterm\ $\vs$ in $\vses$
is \staticallyevaluable.

\subsubsection{Formally}
\begin{mathpar}
\inferrule{
  \vb \eqdef \bigwedge_{\vs\in\vses} \sideeffectisstaticallyevaluable(\vs)
}{
  \isstaticallyevaluable(\vses) \typearrow \vb
}
\end{mathpar}

\hypertarget{def-checkstaticallyevaluable}{}
\subsubsection{TypingRule.CheckStaticallyEvaluable\label{sec:TypingRule.CheckStaticallyEvaluable}}
The function
\[
  \checkstaticallyevaluable(\overname{\TSideEffectSet}{\vses}) \aslto
  \{\True\} \cup \TTypeError
\]
returns $\True$ if the set of \sideeffectdescriptorsterm\ $\vses$ is \staticallyevaluable.
\ProseOtherwiseTypeError

\subsubsection{Prose}
All of the following applies:
\begin{itemize}
  \item applying $\isstaticallyevaluable$ to $\ve$ in $\tenv$ yields $\vb$;
  \item the result is $\True$ if $\vb$ is $\True$, otherwise it is a type error indicating that the expression
  is not statically evaluable.
\end{itemize}

\subsubsection{Formally}
\begin{mathpar}
\inferrule{
  \isstaticallyevaluable(\vses) \typearrow \vb\\
  \checktrans{\vb}{NotStaticallyEvaluable} \checktransarrow \True \OrTypeError
}{
  \checkstaticallyevaluable(\vses) \typearrow \True
}
\end{mathpar}
\CodeSubsection{\CheckStaticallyEvaluableBegin}{\CheckStaticallyEvaluableEnd}{../Typing.ml}

\subsubsection{TypingRule.SESIsPure\label{sec:TypingRule.SESIsPure}}
\hypertarget{def-sesispure}{}
The function
\[
    \sesispure(\overname{\TSideEffectSet}{\vses}) \aslto \overname{\Bool}{\vb}
\]
tests whether all side effects in the set $\vses$ are pure, yielding the result in $\vb$.

\subsubsection{Prose}
Define $\vb$ as $\True$ if and only if $\sideeffectispure$ holds for
every \sideeffectdescriptorterm\ $\vs$ in $\vses$.

\subsubsection{Formally}
\begin{mathpar}
\inferrule{
    \bigwedge_{\vs\in\vses} \sideeffectispure(\vs)
}{
    \sesispure(\vses) \typearrow \True
}
\end{mathpar}

\subsubsection{TypingRule.SESIsDeterministic\label{sec:TypingRule.SESIsDeterministic}}
\hypertarget{def-sesisdeterministic}{}
The function
\[
  \sesisdeterministic(\overname{\TSideEffectSet}{\vses}) \aslto \overname{\Bool}{\bv}
\]
tests whether the \NonDeterministic\ \sideeffectdescriptorterm\ is not included in $\vses$,
yielding the result in $\vb$.

\subsubsection{Prose}
Define $\vb$ as $\True$ if and only if \NonDeterministic\ is not included in $\vses$.

\subsubsection{Formally}
\begin{mathpar}
\inferrule{}{
  \sesisdeterministic(\vses) \typearrow \overname{\NonDeterministic \not\in \vses}{\vb}
}
\end{mathpar}

\subsubsection{TypingRule.SESIsBefore\label{sec:TypingRule.SESIsBefore}}
\hypertarget{def-sesisbefore}{}
The function
\[
  \sesisbefore(\overname{\TSideEffectSet}{\vses} \aslsep \overname{\TTimeFrame}{\vt}) \aslto \overname{\Bool}{\vb}
\]
tests whether the \timeframesterm\ of \sideeffectdescriptorsterm\ in $\vses$ are all less than or equal to the \timeframeterm\
$\vt$, yielding the result in $\vb$.
% Transliteration note: this abstracts both leq_config_time and leq_constant_time

\subsubsection{Prose}
Define $\vb$ as $\True$ if and only if the maximal \timeframeterm\ of all \sideeffectdescriptorsterm\ in $\vses$
is less than or equal to $\vt$ with respect to $\timeframeleq$.

\subsubsection{Formally}
\begin{mathpar}
\inferrule{}{
    \sesisbefore(\vses, \vt) \typearrow \overname{\maxtimeframe(\vses) \timeframeleq \vt}{\vb}
}
\end{mathpar}
