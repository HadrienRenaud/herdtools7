\chapter{Types\label{chap:Types}}

Types describe the allowed values of variables, constants, function arguments, etc.
\lrmcomment{\identi{BYVL}}

This chapter first describes how types are represented formally (see \secref{FormalRepresentationofTypes}).
Next, we introduce each type available in ASL and define
how it is represented in the syntax and AST, and how it is typechecked:
\begin{itemize}
  \item Integer types (see \secref{IntegerTypes})
  \item The \realtypeterm{} (see \secref{RealType})
  \item The \stringtypeterm{} (see \secref{StringType})
  \item The Boolean type (see \secref{BooleanType})
  \item Bitvector types (see \secref{BitvectorTypes})
  \item Tuple types (see \secref{TupleTypes})
  \item Array types (see \secref{ArrayTypes})
  \item Enumeration types (see \secref{EnumerationTypes})
  \item Record types (see \secref{RecordTypes})
  \item Exception types (see \secref{ExceptionTypes})
  \item Named types (see \secref{NamedTypes})
\end{itemize}

The chapter then defines the following aspects of types:
\begin{itemize}
\item \secref{DeclaredTypes} defines \emph{declared types} and restrictions over them;
\item \secref{DomainOfValuesForTypes} defines how values are associated with each type;
\item \secref{BasicTypeAttributes} assigns basic properties to types, which are useful
in classifying them;
\item \secref{RelationsOnTypes} defines relations on types that are needed to typecheck
expressions and statements; and
\item \secref{BaseValues} defines how to produce an expression to initialize storage
      elements of a given type (for which no initializing expression is supplied).
\end{itemize}

\section{Formal Representation of Types\label{sec:FormalRepresentationofTypes}}
\Anonymoustypes\ are grammatically derived from the non-terminal $\Nty$
and types that must be declared and named are grammatically derived from the non-terminal $\Ntydecl$.
The type system represents types by their AST, which is derived from the non-terminal $\ty$.

\subsection{Abstract Syntax}
\hypertarget{build-ty}{}
The function
\[
  \buildty(\overname{\parsenode{\Nty}}{\vparsednode}) \;\aslto\; \overname{\ty}{\vastnode}
  \cup \overname{\TBuildError}{\BuildErrorConfig}
\]
transforms an anonymous type parse node $\vparsednode$ into the corresponding AST node $\vastnode$.
\ProseOtherwiseBuildError

We define $\buildty$ per type in the following sections.

\hypertarget{build-tydecl}{}
The function
\[
  \buildtydecl(\overname{\parsenode{\Ntydecl}}{\vparsednode}) \;\aslto\; \overname{\ty}{\vastnode}
  \cup \overname{\TBuildError}{\BuildErrorConfig}
\]
transforms a \namedtype\ parse node $\vparsednode$ into an AST node $\vastnode$.
\ProseOtherwiseBuildError

We define $\buildtydecl$ per the corresponding type in the following sections.

\hypertarget{build-as-ty}{}
The function
\[
  \buildasty(\overname{\parsenode{\Nasty}}{\vparsednode}) \;\aslto\; \overname{\ty}{\vastnode}
  \cup \overname{\TBuildError}{\BuildErrorConfig}
\]
transforms a type annotation parse node $\vparsednode$ into a type AST node $\vastnode$.
\ProseOtherwiseBuildError

Formally:
\begin{mathpar}
\inferrule{
  \buildty(\vt) \astarrow \astversion{\vt}
} {
  \buildasty(\Tcolon, \namednode{\vt}{\Nty}) \astarrow \astversion{\vt}
}
\end{mathpar}

\subsection{Typing}
\hypertarget{def-annotatetype}{}
The function
\[
  \annotatetype{\overname{\Bool}{\vdecl} \aslsep \overname{\staticenvs}{\tenv} \aslsep \overname{\ty}{\tty}}
  \aslto (\overname{\ty}{\newty} \times \overname{\TSideEffectSet}{\vses}) \cup \overname{\TTypeError}{\TypeErrorConfig}
\]
typechecks a type $\tty$ in a static environment $\tenv$,
resulting in a \typedast\ $\newty$ and a \sideeffectsetterm\ $\vses$.
The flag $\decl$ indicates whether $\tty$ is a type currently being declared or not,
and makes a difference only when $\tty$ is an enumeration type or a \structuredtype.
\ProseOtherwiseTypeError

\subsection{Semantics}
Types are not evaluated dynamically.
However, the dynamic semantics of types is given by their \emph{domain of values},
which is defined in \secref{DomainOfValuesForTypes}.

\hypertarget{integertypeterm}{}
\section{Integer Types\label{sec:IntegerTypes}}
The \emph{\integertypeterm{}} represents mathematical integer value.

There are four kinds of integer types, which we explain next:
\emph{unconstrained}, \emph{well-constrained},
\emph{pending constrained}, and \emph{parameterized}.

\subsection{Unconstrained Integer Types}
The type \verb|integer| represents all integer values.
\lrmcomment{\identi{HJBH}}
There is no bound on the minimum and maximum integer value that can be represented.

\ExampleDef{Unconstrained Integer Types}
\listingref{typing-unconstrained} shows examples of unconstrained integer types.
\ASLListing{Well-typed unconstrained integer types}{typing-unconstrained}{\typingtests/TypingRule.TIntUnConstrained.asl}

\subsection{Well-constrained Integer Types}
\lrmcomment{\identr{GWCP}}
The type \texttt{integer\{$c_1,\ldots,c_n$\}} represents the
union of sets of integers represented by the \emph{integer constraints} $c_1,\ldots,c_n$.
\hypertarget{def-exactconstraintterm}{}
\hypertarget{def-rangeconstraintterm}{}
A constraint can either be an \emph{\exactconstraintterm}, consisting of a single expression like \texttt{4},
or a \emph{\rangeconstraintterm}, consisting of a pair of expressions like \texttt{1..10}.

\ExampleDef{Well-constrained Integer Types}
\listingref{typing-wellconstrained} shows examples of well-constrained integer types.
\ASLListing{Well-typed well-constrained integer types}{typing-wellconstrained}{\typingtests/TypingRule.TIntWellConstrained.asl}

\subsection{Pending-constrained Integer Types}
The type \verb|integer{-}| represents a well-constrained integer type whose
constraints have yet to be determined.
These constraints are inferred by the type system based on the expression used to initialize
the storage element (see \TypingRuleRef{InheritIntegerConstraints}).

\RequirementDef{PendingConstrainedLocal}
Pending-constrained integer types may only appear on the left-hand-side
of local storage element declarations.
%
\listingref{global-pending-constrained} shows an ill-typed specification.

\ExampleDef{Well-typed pending-constrained types}
\listingref{typing-pendingconstrained} shows examples of well-typed pending-constrained
integer types.
\ASLListing{Well-typed pending-constrained integer types}{typing-pendingconstrained}{\typingtests/TypingRule.InheritIntegerConstraints.asl}

\subsection{Parameterized Integer Types}
Subprogram parameters are implicitly \emph{parameterized integer types},
which represent a singleton set for the integer passed to the parameter
at the call site.

\ExampleDef{Parameterized Integer Types}
\listingref{typing-parameterized} shows examples of well-typed parameterized
integer types.
Notice that the type of the parameter \texttt{M} of the function \texttt{bar}
is a parameterized integer type, \underline{not} an unconstrained integer type.
\ASLListing{Well-typed parameterized integer types}{typing-parameterized}{\typingtests/TypingRule.TIntParameterized.asl}

\subsection{Syntax\label{sec:IntegerTypesSyntax}}
\begin{flalign*}
\Nty \derives\ & \Tinteger \parsesep \Nconstraintkindopt &\\
\Nconstraintkindopt \derives \ & \Nconstraintkind \;|\; \emptysentence &\\
\Nconstraintkind \derives \ & \Tlbrace \parsesep \ClistOne{\Nintconstraint} \parsesep \Trbrace &\\
|\ & \Tlbrace \parsesep \Tminus \parsesep \Trbrace &\\
\Nintconstraint \derives \ & \Nexpr &\\
|\ & \Nexpr \parsesep \Tslicing \parsesep \Nexpr &
\end{flalign*}

\subsection{Abstract Syntax\label{sec:IntegerTypesAST}}
\begin{flalign*}
\ty \derives\ & \TInt(\constraintkind)\\
\constraintkind \derives\ & \unconstrained
& \\
|\ & \wellconstrained(\intconstraint^{+})
& \\
|\ & \pendingconstrained{}
& \\
|\ & \parameterized(\overtext{\identifier}{parameter}) &\\
\intconstraint \derives\ & \ConstraintExact(\expr)
& \\
|\ & \ConstraintRange(\overtext{\expr}{start}, \overtext{\expr}{end})&
\end{flalign*}

\ASTRuleDef{Ty.TInt}
\begin{mathpar}
\inferrule[integer]{}{
  \buildty(\Nty(\Tinteger, \punnode{\Nconstraintkindopt})) \astarrow
  \overname{\TInt(\astof{\vconstraintkindopt})}{\vastnode}
}
\end{mathpar}

\ASTRuleDef{IntConstraintsOpt}
\hypertarget{build-constraintkindopt}{}
The function
\[
  \buildconstraintkindopt(\overname{\parsenode{\Nconstraintkindopt}}{\vparsednode}) \;\aslto\; \overname{\constraintkind}{\vastnode}
\]
transforms a parse node $\vparsednode$ into an AST node $\vastnode$.

\begin{mathpar}
\inferrule[constrained]{}{
  {
    \begin{array}{r}
  \buildconstraintkindopt(\Nconstraintkindopt(\punnode{\Nconstraintkind})) \astarrow \\
  \overname{\astof{\vconstraintkind}}{\vastnode}
    \end{array}
  }
}
\end{mathpar}

\begin{mathpar}
\inferrule[unconstrained]{}{
  \buildconstraintkindopt(\Nconstraintkindopt(\emptysentence)) \astarrow
  \overname{\unconstrained}{\vastnode}
}
\end{mathpar}

\subsection{ASTRule.IntConstraints\label{sec:ASTRule.IntConstraints}}
\hypertarget{build-constraintkind}{}
The function
\[
  \buildconstraintkind(\overname{\parsenode{\Nconstraintkind}}{\vparsednode}) \;\aslto\; \overname{\constraintkind}{\vastnode}
\]
transforms a parse node $\vparsednode$ into an AST node $\vastnode$.

\begin{mathpar}
\inferrule[well\_constrained]{
  \buildclist[\buildintconstraint](\cs) \astarrow \vcsasts
}{
  {
    \begin{array}{r}
  \buildconstraintkind(\Nconstraintkind(\Tlbrace, \namednode{\cs}{\ClistOne{\Nintconstraint}}, \Trbrace)) \astarrow\\
  \overname{\wellconstrained(\vcsasts)}{\vastnode}
    \end{array}
  }
}
\end{mathpar}

\begin{mathpar}
\inferrule[pending\_constrained]{}{
  \buildconstraintkind(\Nconstraintkind(\Tlbrace, \Tminus, \Trbrace)) \astarrow
  \overname{\pendingconstrained}{\vastnode}
}
\end{mathpar}

\ASTRuleDef{IntConstraint}
\hypertarget{build-intconstraint}{}
The function
\[
  \buildintconstraint(\overname{\parsenode{\Nintconstraint}}{\vparsednode}) \;\aslto\; \overname{\intconstraint}{\vastnode}
\]
transforms a parse node $\vparsednode$ into an AST node $\vastnode$.

\begin{mathpar}
\inferrule[exact]{}{
  \buildintconstraint(\Nintconstraint(\punnode{\Nexpr})) \astarrow
  \overname{\ConstraintExact(\astof{\vexpr})}{\vastnode}
}
\end{mathpar}

\begin{mathpar}
\inferrule[range]{
  \buildexpr(\vfromexpr) \astarrow \astversion{\vfromexpr}\\
  \buildexpr(\vtoexpr) \astarrow \astversion{\vtoexpr}\\
}{
  {
    \begin{array}{r}
  \buildintconstraint(\Nintconstraint(\namednode{\vfromexpr}{\Nexpr}, \Tslicing, \namednode{\vtoexpr}{\Nexpr})) \astarrow\\
  \overname{\ConstraintRange(\astversion{\vfromexpr}, \astversion{\vtoexpr})}{\vastnode}
    \end{array}
  }
}
\end{mathpar}

\subsection{Typing Integer Types\label{sec:TypingIntegerTypes}}
\hypertarget{def-isunconstrainedinteger}{}
\hypertarget{def-isparameterizedinteger}{}
\hypertarget{def-iswellconstrainedinteger}{}
We use the following helper predicates to classify integer types:
\[
  \begin{array}{rcl}
  \isunconstrainedinteger(\overname{\ty}{\vt}) &\aslto& \Bool\\
  \isparameterizedinteger(\overname{\ty}{\vt}) &\aslto& \Bool\\
  \iswellconstrainedinteger(\overname{\ty}{\vt}) &\aslto& \Bool
  \end{array}
\]
Those are defined as follows:
\[
  \begin{array}{rcl}
  \isunconstrainedinteger(\vt) &\triangleq& \vt = \TInt(c) \land \astlabel(c)=\unconstrained\\
  \isparameterizedinteger(\vt) &\triangleq& \vt = \TInt(c) \land \astlabel(c)=\parameterized\\
  \iswellconstrainedinteger(\vt) &\triangleq& \vt = \TInt(c) \land \astlabel(c)=\wellconstrained\\
\end{array}
\]\lrmcomment{This is related to \identd{ZTPP}, \identr{WJYH}, \identr{HJPN}, \identr{CZTX}, \identr{TPHR}.}

\hypertarget{def-unconstrainedinteger}{}
We use the shorthand notation $\unconstrainedinteger \triangleq \TInt(\unconstrained)$
for unconstrained integers.

\TypingRuleDef{TInt}
\ExampleDef{Ill-typed pending-constrained integer type}
\listingref{global-pending-constrained}
and \listingref{rhs-pending-constrained}
correspond to \CaseName{pending\_constrained}.
\ASLListing{Ill-typed pending-constrained integer type}{global-pending-constrained}
{\typingtests/TypingRule.TInt.global_pending_constrained.bad.asl}

\ASLListing{Ill-typed pending-constrained integer type}{rhs-pending-constrained}
{\typingtests/TypingRule.TInt.rhs_pending_constrained.bad.asl}

\ProseParagraph
\OneApplies
\begin{itemize}
  \item \AllApplyCase{pending\_constrained}
    \begin{itemize}
      \item $\tty$ is a \pendingconstrainedintegertype;
      \item the result is a type error (\UnexpectedType).
    \end{itemize}
  \item \AllApplyCase{well\_constrained}
    \begin{itemize}
      \item $\tty$ is the well-constrained integer type constrained by
        constraints $\vc_i$, for $u=1..k$;
      \item annotating each constraint $\vc_i$, for $i=1..k$,
      yields $(\newc_i, \vxs_i)$\ProseOrTypeError;
      \item $\newconstraints$ is the list of annotated constraints $\newc_i$,
      for $i=1..k$;
      \item $\newty$ is the well-constrained integer type constrained
        by $\newconstraints$;
      \item define $\vses$ as the union of all $\vxs_i$, for $i=1..k$.
    \end{itemize}

    \item \AllApplyCase{parameterized}
    \begin{itemize}
      \item $\tty$ is a \parameterizedintegertype\ for $\name$;
      \item define $\vses$ as the singleton set for the singleton \sideeffectdescriptorterm,
            \ReadLocalTerm\ for $\name$, \timeframeconstant, and $\True$ for immutability.
      \item $\newty$ is the unconstrained integer type.
    \end{itemize}

    \item \AllApplyCase{unconstrained}
    \begin{itemize}
      \item $\tty$ is an \unconstrainedintegertype;
      \item $\newty$ is the unconstrained integer type;
      \item define $\vses$ as the empty set.
    \end{itemize}
  \end{itemize}

\FormallyParagraph
\begin{mathpar}
\inferrule[pending\_constrained]{}{
  {
    \begin{array}{r}
  \annotatetype{\overname{\Ignore}{\vdecl}, \tenv, \overname{\TInt(\pendingconstrained)}{\tty}} \typearrow
  \TypeErrorVal{\UnexpectedType}
    \end{array}
  }
}
\end{mathpar}

\begin{mathpar}
\inferrule[well\_constrained]{
  \constraints \eqname \vc_{1..k}\\
  i=1..k: \annotateconstraint(\vc_i) \typearrow (\newc_i, \vxs_i) \OrTypeError\\\\
  \newconstraints \eqdef \newc_{1..k}\\
  \vses \eqdef \bigcup_{i=1..k} \vxs_i
}{
  {
    \begin{array}{r}
  \annotatetype{\overname{\Ignore}{\vdecl}, \tenv, \overname{\TInt(\wellconstrained(\constraints))}{\tty}} \typearrow \\
  (\overname{\TInt(\wellconstrained(\newconstraints))}{\newty}, \vses)
    \end{array}
  }
}
\end{mathpar}

\begin{mathpar}
\inferrule[parameterized]{
  \tty \eqname \TInt(\parameterized(\name))\\
  \vses \eqdef \{\ \ReadLocal(\name, \timeframeconstant, \True)\ \}
}{
  \annotatetype{\overname{\Ignore}{\vdecl}, \tenv, \tty} \typearrow (\overname{\tty}{\newty}, \vses)
}
\end{mathpar}

\begin{mathpar}
\inferrule[unconstrained]{
  \tty \eqname \unconstrainedinteger
}{
  \annotatetype{\overname{\Ignore}{\vdecl}, \tenv, \tty} \typearrow (\overname{\tty}{\newty}, \overname{\emptyset}{\vses})
}
\end{mathpar}

\CodeSubsection{\TIntBegin}{\TIntEnd}{../Typing.ml}

\TypingRuleDef{AnnotateConstraint}
\hypertarget{def-annotateconstraint}{}
The function
\[
\begin{array}{r}
\annotateconstraint(\overname{\staticenvs}{\tenv} \aslsep \overname{\intconstraint}{\vc})
\aslto (\overname{\intconstraint}{\newc} \times \overname{\TSideEffectSet}{\vses})\ \cup \\
\overname{\TTypeError}{\TypeErrorConfig}
\end{array}
\]
annotates an integer constraint $\vc$ in the static environment $\tenv$ yielding the annotated
integer constraint $\newc$ and \sideeffectsetterm\ $\vses$.
\ProseOtherwiseTypeError

\listingref{annotate-constraint} shows examples of \wellconstrainedintegertypes{}
and the resulting annotated constraints in comments.
The annotated constraints inline the constant \texttt{N} and the right-hand-side
expressions of \texttt{let} storage elements.
\ASLListing{Annotated constraints}{annotate-constraint}{\typingtests/TypingRule.AnnotateConstraint.asl}

\RequirementDef{ConstraintSymbolicallyConstrained}
The expressions appearing in integer constraints must be both
\symbolicallyevaluable{} and \constrainedinteger{} types.
%
In \listingref{annotate-constraint-unconstrained}, the constraint
\verb|x..x+1| is ill-typed, since the type of \texttt{x} is not constrained.
\ASLListing{Ill-typed constraint}{annotate-constraint-unconstrained}{\typingtests/TypingRule.AnnotateConstraint.bad.asl}

\ProseParagraph
\OneApplies
\begin{itemize}
  \item \AllApplyCase{exact}
  \begin{itemize}
    \item $\vc$ is the exact integer constraint for the expression $\ve$, that is, \\ $\ConstraintExact(\ve)$;
    \item applying $\annotatesymbolicconstrainedinteger$ to $\ve$ in $\tenv$ yields \\
          $(\vep, \vses)$\ProseOrTypeError;
    \item define $\newc$ as the exact integer constraint for $\vep$, that is, $\ConstraintExact(\vep)$.
  \end{itemize}

  \item \AllApplyCase{range}
  \begin{itemize}
    \item $\vc$ is the range integer constraint for expressions $\veone$ and $\vetwo$, that is, \\ $\ConstraintRange(\veone, \vetwo)$;
    \item applying $\annotatesymbolicconstrainedinteger$ to $\veone$ in $\tenv$ yields\\ $(\veonep, \vsesone)$\ProseOrTypeError;
    \item applying $\annotatesymbolicconstrainedinteger$ to $\vetwo$ in $\tenv$ yields\\ $(\vetwop, \vsestwo)$\ProseOrTypeError;
    \item define $\newc$ as the range integer constraint for expressions $\veonep$ and $\vetwop$, that is, $\ConstraintRange(\veonep, \vetwop)$;
    \item define $\vses$ as the union of $\vsesone$ and $\vsestwo$.
  \end{itemize}
\end{itemize}

\FormallyParagraph
\begin{mathpar}
\inferrule[exact]{
  \annotatesymbolicconstrainedinteger(\tenv, \ve) \typearrow (\vep, \vses) \OrTypeError
}{
  \annotateconstraint(\tenv, \overname{\ConstraintExact(\ve)}{\vc}) \typearrow (\overname{\ConstraintExact(\vep)}{\newc}, \vses)
}
\and
\inferrule[range]{
  \annotatesymbolicconstrainedinteger(\tenv, \veone) \typearrow (\veonep, \vsesone) \OrTypeError\\\\
  \annotatesymbolicconstrainedinteger(\tenv, \vetwo) \typearrow (\vetwop, \vsestwo) \OrTypeError\\\\
  \vses \eqdef \vsesone \cup \vsestwo
}{
  \annotateconstraint(\tenv, \overname{\ConstraintRange(\veone, \vetwo)}{\vc}) \typearrow \overname{\ConstraintRange(\veonep, \vetwop)}{\newc}
}
\end{mathpar}

\hypertarget{realtypeterm}{}
\section{The Real Type\label{sec:RealType}}
The \emph{\realtypeterm{}} represents mathematical rational number values.
There is no bound on the minimum and maximum rational value that can be represented,
and there is no bound on their precision.
%
There is no mechanism in the language to generate an irrational value of \realtypeterm.

Conversions from an \integertypeterm{} value to a \realtypeterm{} value are performed
using the \stdlibfunc{Real}.
%
Conversions from a \realtypeterm{} value an \integertypeterm{} value to are performed
using the \stdlibfunc{RoundDown} or \stdlibfunc{RoundUp}.

\ExampleDef{Well-typed Real Types}
In \listingref{typing-treal}, all the uses of the \realtypeterm{} are well-typed.
\ASLListing{Well-typed real types}{typing-treal}{\typingtests/TypingRule.TReal.asl}

\subsection{Syntax}
\begin{flalign*}
\Nty \derives\ & \Treal &
\end{flalign*}

\subsection{Abstract Syntax}
\begin{flalign*}
\ty \derives\ & \TReal &
\end{flalign*}

\ASTRuleDef{TReal}
\begin{mathpar}
\inferrule{}{
  \buildty(\Nty(\Treal)) \astarrow
  \overname{\TReal}{\vastnode}
}
\end{mathpar}

\subsection{Typing the Real Type\label{sec:TypingRealType}}
\TypingRuleDef{TReal}
\ProseParagraph
\AllApply
\begin{itemize}
  \item $\tty$ is the \realtypeterm{}, $\TReal$.
  \item $\newty$ is the \realtypeterm{}, $\TReal$;
  \item define $\vses$ as the empty set.
\end{itemize}

\FormallyParagraph
\begin{mathpar}
\inferrule{}
{
  \annotatetype{\overname{\Ignore}{\vdecl}, \tenv, \overname{\TReal}{\tty}} \typearrow (\overname{\TReal}{\newty}, \overname{\emptyset}{\vses})
}
\end{mathpar}
\CodeSubsection{\TRealBegin}{\TRealEnd}{../Typing.ml}

\hypertarget{stringtypeterm}{}
\section{The String Type\label{sec:StringType}}
The \emph{\stringtypeterm{}} represents strings of characters.

Strings play relatively little role in specifications and the only operations
on strings are equality and inequality tests.
String are useful in \printstatementsterm{} for debugging and diagnostic purposes
on runtimes that support printing.

\ExampleDef{Well-typed String Types}
In \listingref{typing-tstring}, all the uses of the \stringtypeterm{} are well-typed.
\ASLListing{Well-typed string types}{typing-tstring}{\typingtests/TypingRule.TString.asl}

\subsection{Syntax}
\begin{flalign*}
\Nty \derives\ & \Tstring &
\end{flalign*}

\subsection{Abstract Syntax}
\begin{flalign*}
\ty \derives\ & \TString&
\end{flalign*}

\ASTRuleDef{Ty.String}
\begin{mathpar}
\inferrule{}{
  \buildty(\Nty(\Tstring)) \astarrow
  \overname{\TString}{\vastnode}
}
\end{mathpar}

\subsection{Typing the String Type\label{sec:TypingStringType}}
\TypingRuleDef{TString}
\ProseParagraph
\AllApply
\begin{itemize}
  \item $\tty$ is the \stringtypeterm{}, $\TString$.
  \item $\newty$ is the \stringtypeterm{}, $\TString$.
  \item \Proseeqdef{$\vses$}{the empty set}.
\end{itemize}

\FormallyParagraph
\begin{mathpar}
\inferrule{}
{
  \annotatetype{\overname{\Ignore}{\vdecl}, \tenv, \overname{\TString}{\tty}} \typearrow (\overname{\TString}{\newty}, \overname{\emptyset}{\vses})
}
\end{mathpar}
\CodeSubsection{\TStringBegin}{\TStringEnd}{../Typing.ml}

\hypertarget{booleantypeterm}{}
\section{The Boolean Type\label{sec:BooleanType}}
The \emph{\booleantypeterm{}} represents the algebraic Boolean type.

\subsubsection{Example}
In \listingref{typing-tbool}, all the uses of \texttt{boolean} are well-typed.
\ASLListing{Well-typed Boolean types}{typing-tbool}{\typingtests/TypingRule.TBool.asl}

\subsection{Syntax}
\begin{flalign*}
\Nty \derives\ & \Tboolean &
\end{flalign*}

\subsection{Abstract Syntax}
\begin{flalign*}
\ty \derives\ & \TBool &
\end{flalign*}

\ASTRuleDef{Ty.BoolType}
\begin{mathpar}
\inferrule{}{
  \buildty(\Nty(\Tboolean)) \astarrow
  \overname{\TBool}{\vastnode}
}
\end{mathpar}

\subsection{Typing the Boolean Type\label{sec:TypingBooleanType}}
\TypingRuleDef{TBool}
\ProseParagraph
\AllApply
\begin{itemize}
  \item $\tty$ is the boolean type, \TBool;
  \item $\newty$ is the boolean type, \TBool;
  \item define $\vses$ as the empty set.
\end{itemize}

\FormallyParagraph
\begin{mathpar}
\inferrule{}
{
  \annotatetype{\overname{\Ignore}{\vdecl}, \tenv, \overname{\TBool}{\tty}} \typearrow (\overname{\TBool}{\newty}, \overname{\emptyset}{\vses})
}
\end{mathpar}
\CodeSubsection{\TBoolBegin}{\TBoolEnd}{../Typing.ml}

\hypertarget{bitvectortypeterm}{}
\section{Bitvector Types\label{sec:BitvectorTypes}}
\subsection{Syntax}
\begin{flalign*}
\Nty \derives\ & \Tbit &\\
            |\ & \Tbits \parsesep \Tlpar \parsesep \Nexpr \parsesep \Trpar \parsesep \option{\Nbitfields} &\\
\Nbitfields \derives \ & \Tlbrace \parsesep \TClistZero{\Nbitfield} \parsesep \Trbrace &\\
\Nbitfield \derives \ & \Nslices \parsesep \Tidentifier &\\
                  |\ & \Nslices \parsesep \Tidentifier \parsesep \Nbitfields &\\
                  |\ & \Nslices \parsesep \Tidentifier \parsesep \Tcolon \parsesep \Nty &\\
\end{flalign*}

\subsection{Abstract Syntax}
\begin{flalign*}
\ty \derives\ & \TBits(\overtext{\expr}{width}, \bitfield^{*}) &
\end{flalign*}

\ASTRuleDef{Ty.TBits}
\begin{mathpar}
\inferrule[bit]{}{
  \buildty(\Nty(\Tbit)) \astarrow
  \overname{\TBits(\ELiteral(\lint(1)), \emptylist)}{\vastnode}
}
\end{mathpar}

\begin{mathpar}
\inferrule[bits]{
  \buildlist[\buildbitfield](\vbitfields) \astarrow \vbitfieldasts
}{
  {
    \begin{array}{r}
  \buildty(\Nty(\Tbits, \Tlpar, \punnode{\Nexpr}, \Trpar, \namednode{\vbitfields}{\maybeemptylist{\Nbitfields}})) \astarrow\\
  \overname{\TBits(\astof{\vexpr}, \vbitfieldasts)}{\vastnode}
    \end{array}
  }
}
\end{mathpar}

\subsection{Typing}
\TypingRuleDef{TBits}
\subsubsection{Example}
In \listingref{typing-tbits}, all the uses of bitvector types are well-typed.
\ASLListing{Well-typed Bitevector types}{typing-tbits}{\typingtests/TypingRule.TBits.asl}

\ProseParagraph
\AllApply
\begin{itemize}
  \item $\tty$ is the bit-vector type with width given by the expression
    $\ewidth$ and the bitfields given by $\bitfields$, that is, $\TBits(\ewidth, \bitfields)$;
  \item annotating the expression $\ewidth$ yields $(\twidth, \ewidthp, \seswidth)$\ProseOrTypeError;
  \item \Prosechecksymbolicallyevaluable{\seswidth};
  \item \Prosecheckconstrainedinteger{$\tenv$}{$\twidth$};
  \item annotating the bitfields $\bitfields$ yields $(\bitfieldsp, \vsesbitfields)$\ProseOrTypeError;
  \item \Prosestaticeval{$\tenv$}{$\ewidthp$}{$\lint(\vwidth)$};
  \item \Prosecheckcommonbitfieldsalign{$\tenv$}{$\bitfieldsp$}{$\vwidth$}\ProseOrTypeError;
  \item $\newty$ is the bit-vector type with width given by the expression
    $\ewidthp$ and the bitfields given by $\bitfieldsp$, that is, $\TBits(\ewidthp, \bitfieldsp)$;
  \item define $\vses$ as the union of $\seswidth$ and $\vsesbitfields$.
\end{itemize}

\FormallyParagraph
\begin{mathpar}
\inferrule{
  \annotateexpr{\tenv, \ewidth} \typearrow (\twidth, \ewidthp, \seswidth) \OrTypeError\\\\
  \checksymbolicallyevaluable(\seswidth) \typearrow \True \OrTypeError\\\\
  \checkconstrainedinteger(\tenv, \twidth) \typearrow \True \OrTypeError\\\\
  {
  \begin{array}{r}
    \annotatebitfields(\tenv, \ewidthp, \bitfields) \typearrow \\
    (\bitfieldsp, \vsesbitfields) \OrTypeError
  \end{array}
  }\\
  \staticeval(\tenv, \ewidthp) \typearrow \lint(\vwidth)\\
  \checkcommonbitfieldsalign(\tenv, \bitfieldsp, \vwidth) \typearrow \True \OrTypeError\\\\
  \vses \eqdef \seswidth \cup \vsesbitfields
}{
  {
    \begin{array}{r}
  \annotatetype{\overname{\Ignore}{\vdecl}, \tenv, \TBits(\ewidth, \bitfields)} \typearrow \\
  (\overname{\TBits(\ewidthp, \bitfieldsp)}{\newty}, \vses)
    \end{array}
  }
}
\end{mathpar}
\CodeSubsection{\TBitsBegin}{\TBitsEnd}{../Typing.ml}

\subsubsection{Comments}
The width of a bitvector type $\TBits(\ewidth, \bitfields)$, given by the expression \\
$\ewidth$,
must be non-negative.

\hypertarget{tupletypeterm}{}
\section{Tuple Types\label{sec:TupleTypes}}
\subsection{Syntax}
\begin{flalign*}
\Nty \derives\ & \PlistZero{\Nty} &
\end{flalign*}

\subsection{Abstract Syntax}
\begin{flalign*}
\ty \derives\ & \TTuple(\ty^{*}) &
\end{flalign*}

\ASTRuleDef{Ty.TTuple}
\begin{mathpar}
\inferrule{
  \buildplist[\buildty](\vtypes) \astarrow \vtypeasts
}{
  \buildty(\Nty(\namednode{\vtypes}{\PlistZero{\Nty}})) \astarrow
  \overname{\TTuple(\vtypeasts)}{\vastnode}
}
\end{mathpar}

\subsection{Typing Tuple Types\label{sec:TypingTupleTypes}}
\subsubsection{Example}
In \listingref{typing-ttuple}, all the uses of tuple types are well-typed.
\ASLListing{Well-typed tuple types}{typing-ttuple}{\typingtests/TypingRule.TTuple.asl}

\TypingRuleDef{TTuple}
\ProseParagraph
\AllApply
\begin{itemize}
  \item $\tty$ is the tuple type with member types $\tys$, that is, $\TTuple(\tys)$;
  \item $\tys$ is the list $\tty_i$, for $i=1..k$;
  \item annotating each type $\tty_i$ in $\tenv$, for $i=1..k$,
  yields $(\ttyp_i, \vxs_i)$\ProseOrTypeError;
  \item $\newty$ is the tuple type with member types $\ttyp$, for $i=1..k$;
  \item define $\vses$ as the union of all $\vxs_i$, for $i=1..k$.
\end{itemize}

\FormallyParagraph
\begin{mathpar}
\inferrule{
  k \geq 2\\
  \tys \eqname \tty_{1..k}\\
  i=1..k: \annotatetype{\False, \tenv, \tty_i} \typearrow (\ttyp_i, \vxs_i) \OrTypeError\\
  \vses \eqdef \bigcup_{i=1..k} \vxs_i
}{
  \annotatetype{\overname{\Ignore}{\vdecl}, \tenv, \TTuple(\tys)} \typearrow (\overname{\TTuple(\tysp)}{\newty}, \vses)
}
\end{mathpar}
\CodeSubsection{\TTupleBegin}{\TTupleEnd}{../Typing.ml}

\hypertarget{arraytypeterm}{}
\section{Array Types\label{sec:ArrayTypes}}
ASL offers two kinds of arrays:
\begin{description}
  \item[Integer-indexed arrays] representing a consecutive list of elements at positions $0$ to the size
      specified for the array. The array elements can be accessed via an \texttt{integer}
      type that specifies the $0$-based position of the element to read/update.
  \item[Enumeration-indexed arrays] representing a dictionary-like data type where the keys are defined
      by a given enumeration type. The array elements can be accessed via values of the \texttt{enumeration}
      type specified for the array type.
\end{description}

\subsection{Syntax}
\begin{flalign*}
\Nty \derives\ & \Tarray \parsesep \Tllbracket \parsesep \Nexpr \parsesep \Trrbracket \parsesep \Tof \parsesep \Nty &
\end{flalign*}

\subsection{Abstract Syntax}
\begin{flalign*}
\ty \derives\ & \TArray(\arrayindex, \ty) &\\
\arrayindex \derives\ &  \ArrayLengthExpr(\overtext{\expr}{array length}) &
\end{flalign*}

\ASTRuleDef{Ty.TArray}
\begin{mathpar}
\inferrule{}{
  {
  \begin{array}{r}
    \buildty(\Nty(\Tarray, \Tllbracket, \punnode{\Nexpr}, \Trrbracket, \Tof, \punnode{\Nty})) \astarrow\\
    \overname{\TArray(\ArrayLengthExpr(\astof{\vexpr}), \astof{\tty})}{\vastnode}
  \end{array}
  }
}
\end{mathpar}
\subsection{Typing Array Types\label{sec:TypingArrayTypes}}
\subsubsection{Example}
In \listingref{typing-tarray}, all the uses of array types are well-typed.
\ASLListing{Well-typed array types}{typing-tarray}{\typingtests/TypingRule.TArray.asl}

\TypingRuleDef{TArray}
\ProseParagraph
\AllApply
\begin{itemize}
  \item $\tty$ is the array type with element type $\vt$;
  \item Annotating the type $\vt$ in $\tenv$ yields $(\vtp, \vsest)$\ProseOrTypeError;
  \item \OneApplies
  \begin{itemize}
    \item \AllApplyCase{expr\_is\_enum}
    \begin{itemize}
      \item the array index is $\ve$ and determining whether $\ve$ corresponds to an enumeration in $\tenv$
      via $\getvariableenum$ yields the enumeration variable
      name $\vs$ of size $\vi$, that is, $\langle \vs, \vi \rangle$\ProseOrTypeError;
      \item $\newty$ is the array type indexed by an enumeration type
      named $\vs$ of length $\vi$ and of elements of type $\vtp$, that is, $\TArray(\ArrayLengthEnum(\vs, \vi), \vtp)$;
      \item define $\vses$ as $\vsest$.
    \end{itemize}

    \item \AllApplyCase{expr\_not\_enum}
    \begin{itemize}
      \item the array index is $\ve$ and determining whether $\ve$ corresponds to an enumeration in $\tenv$
      via $\getvariableenum$ yields $\None$ (meaning it does not
      correspond to an enumeration)\ProseOrTypeError;
      \item annotating the \symbolicallyevaluable{} integer expression $\ve$ yields\\
      $(\vep, \vsesindex)$\ProseOrTypeError;
      \item $\newty$ the array type indexed by integer bounded by
      the expression $\vep$ and of elements of type $\vtp$, that is,
      $\TArray(\ArrayLengthExpr(\vep), \vtp)$;
      \item define $\vses$ as the union of $\vsest$ and $\vsesindex$.
    \end{itemize}
  \end{itemize}
\end{itemize}
\FormallyParagraph
\begin{mathpar}
\inferrule[expr\_is\_enum]{
  \annotatetype{\False, \tenv, \vt} \typearrow (\vtp, \vsest) \OrTypeError\\\\
  \commonprefixline\\\\
  \getvariableenum(\tenv, \ve) \typearrow \langle \vs, \vlabels \rangle\OrTypeError
}{
  \annotatetype{\overname{\Ignore}{\vdecl}, \tenv, \overname{\AbbrevTArrayLengthExpr{\ve}{\vt}}{\tty}} \typearrow
  (\overname{\AbbrevTArrayLengthEnum{\ve}{\vlabels}{\vtp}}{\newty}, \overname{\emptyset}{\vsest})
}
\end{mathpar}

\begin{mathpar}
\inferrule[expr\_not\_enum]{
  \annotatetype{\False, \tenv, \vt} \typearrow (\vtp, \vsest) \OrTypeError\\\\
  \commonprefixline\\\\
  \getvariableenum(\tenv, \ve) \typearrow \None \OrTypeError\\\\
  \annotatesymbolicinteger(\tenv, \ve) \typearrow (\vep, \vsesindex) \OrTypeError\\\\
  \vses \eqdef \vsest \cup \vsesindex
}{
  \annotatetype{\overname{\Ignore}{\vdecl}, \tenv, \overname{\AbbrevTArrayLengthExpr{\ve}{\vt}}{\tty}} \typearrow
  (\overname{\AbbrevTArrayLengthExpr{\vep}{\vtp}}{\newty}, \vses)
}
\end{mathpar}
\CodeSubsection{\TArrayBegin}{\TArrayEnd}{../Typing.ml}

\TypingRuleDef{GetVariableEnum}
\hypertarget{def-getvariableenum}{}
The function
\[
\getvariableenum(\overname{\staticenvs}{\tenv} \aslsep \overname{\expr}{\ve}) \aslto
\langle (\overname{\identifier}{\vx}, \overname{\identifier^+}{\vlabels})\rangle
\]
tests whether the expression $\ve$ represents a variable of an enumeration type.
If so, the result is $\vx$ --- the name of the variable and the list of labels $\vlabels$,
declared for the enumeration type.
Otherwise, the result is $\None$.

\ProseParagraph
\OneApplies
\begin{itemize}
  \item \AllApplyCase{not\_evar}
  \begin{itemize}
    \item $\ve$ is not a variable expression;
    \item the result is $\None$.
  \end{itemize}

  \item \AllApplyCase{no\_declared\_type}
  \begin{itemize}
    \item $\ve$ is a variable expression for $\vx$, that is, $\EVar(\vx)$;
    \item $\vx$ is not associated with a type in the global environment of $\tenv$;
    \item the result is $\None$.
  \end{itemize}

  \item \AllApplyCase{declared\_enum}
  \begin{itemize}
    \item $\ve$ is a variable expression for $\vx$, that is, $\EVar(\vx)$;
    \item $\vx$ is associated with a type $\vt$ in the global environment of $\tenv$;
    \item obtaining the \underlyingtype\ of $\vt$ in $\tenv$ yields an enumeration type with labels $\vlabels$\ProseOrTypeError;
    \item the result is the pair consisting of $\vx$ and $\vlabels$.
  \end{itemize}

  \item \AllApplyCase{declared\_not\_enum}
  \begin{itemize}
    \item $\ve$ is a variable expression for $\vx$, that is, $\EVar(\vx)$;
    \item $\vx$ is associated with a type $\vt$ in the global environment of $\tenv$;
    \item obtaining the \underlyingtype\ of $\vt$ in $\tenv$ yields a type that is not an enumeration type;
    \item the result is $\None$.
  \end{itemize}
\end{itemize}

\FormallyParagraph
\begin{mathpar}
\inferrule[not\_evar]{
  \astlabel(\ve) \neq \EVar
}{
  \getvariableenum(\tenv, \ve) \typearrow \None
}
\end{mathpar}

\begin{mathpar}
\inferrule[no\_declared\_type]{
  G^\tenv.\declaredtypes(\vx) = \bot
}{
  \getvariableenum(\tenv, \overname{\EVar(\vx)}{\ve}) \typearrow \None
}
\end{mathpar}

\begin{mathpar}
\inferrule[declared\_enum]{
  G^\tenv.\declaredtypes(\vx) = (\vt, \Ignore)\\
  \makeanonymous(\tenv, \vt) \typearrow \TEnum(\vlabels) \OrTypeError
}{
  \getvariableenum(\tenv, \overname{\EVar(\vx)}{\ve}) \typearrow \langle(\vx, \vlabels)\rangle
}
\end{mathpar}

\begin{mathpar}
\inferrule[declared\_not\_enum]{
  G^\tenv.\declaredtypes(\vx) = (\vt, \Ignore)\\
  \makeanonymous(\tenv, \vt) \typearrow \vtone\\
  \astlabel(\vtone) \neq \TEnum
}{
  \getvariableenum(\tenv, \overname{\EVar(\vx)}{\ve}) \typearrow \None
}
\end{mathpar}

\TypingRuleDef{AnnotateSymbolicallyEvaluableExpr}
\hypertarget{def-annotatesymbolicallyevaluableexpr}{}
The function
\[
\begin{array}{r}
  \annotatesymbolicallyevaluableexpr(\overname{\staticenvs}{\tenv} \aslsep \overname{\expr}{\ve}) \aslto \\
  (\overname{\ty}{\vt}\times\overname{\expr}{\vep}\times\overname{\TSideEffectSet}{\vses}) \cup \overname{\TTypeError}{\TypeErrorConfig}
\end{array}
\]
annotates the expression $\ve$ in the static environment $\tenv$ and checks that it is \symbolicallyevaluable,
yielding the resulting type in $\vt$, the annotated expression in $\vep$ and the \sideeffectsetterm\ in $\vses$.
\ProseOtherwiseTypeError

\ProseParagraph
\AllApply
\begin{itemize}
  \item \Proseannotateexpr{$\tenv$}{$\ve$}{$(\vt, \vep, \vses)$};
  \item \Prosechecksymbolicallyevaluable{$\vses$}.
\end{itemize}

\FormallyParagraph
\begin{mathpar}
\inferrule{
  \annotateexpr{\tenv, \ve} \typearrow (\vt, \vep, \vses) \OrTypeError\\\\
  \checksymbolicallyevaluable(\vses) \typearrow \True \OrTypeError
}{
  \annotatesymbolicallyevaluableexpr(\tenv, \ve) \typearrow (\vt, \vep, \vses)
}
\end{mathpar}
\CodeSubsection{\AnnotateSymbolicallyEvaluableExprBegin}{\AnnotateSymbolicallyEvaluableExprEnd}{../Typing.ml}

\TypingRuleDef{AnnotateSymbolicInteger}
\hypertarget{def-annotatesymbolicinteger}{}
The function
\[
  \annotatesymbolicinteger(\overname{\staticenvs}{\tenv} \aslsep \overname{\expr}{\ve}) \aslto
  (\overname{\expr}{\vepp} \times \overname{\TSideEffectSet}{\vses}) \cup \overname{\TTypeError}{\TypeErrorConfig}
\]
annotates a \symbolicallyevaluable\ integer expression $\ve$ in the static environment $\tenv$
and returns the annotated expression $\vepp$ and \sideeffectsetterm\ $\vses$.
\ProseOtherwiseTypeError

\ProseParagraph
\AllApply
\begin{itemize}
  \item \Proseannotatesymbolicallyevaluableexpr{$\tenv$}{$\ve$}{$(\vt, \vep, \vses)$\ProseOrTypeError};
  \item determining whether $\vt$ has the structure of an integer yields $\True$\ProseOrTypeError;
  \item determining whether $\vep$ is \symbolicallyevaluable\ in $\tenv$ yields $\True$\ProseOrTypeError;
  \item applying $\normalize$ to $\vep$ in $\tenv$ yields $\vepp$.
\end{itemize}
\FormallyParagraph
\begin{mathpar}
\inferrule{
  \annotatesymbolicallyevaluableexpr(\tenv, \ve) \typearrow (\vt, \vep, \vses) \OrTypeError\\\\
  \checkstructureinteger(\tenv, \vt) \typearrow \True \OrTypeError\\\\
  \checksymbolicallyevaluable(\tenv, \vep) \typearrow \True \OrTypeError\\\\
  \normalize(\tenv, \vep) \typearrow \vepp
}{
  \annotatesymbolicinteger(\tenv, \ve) \typearrow (\vepp, \vses)
}
\end{mathpar}
\CodeSubsection{\AnnotateSymbolicIntegerBegin}{\AnnotateSymbolicIntegerEnd}{../Typing.ml}

\hypertarget{def-checkstructureinteger}{}
\TypingRuleDef{CheckStructureInteger}
The function
\[
  \checkstructureinteger(\overname{\staticenvs}{\tenv} \aslsep \overname{\ty}{\vt}) \aslto
  \{\True\} \cup \TTypeError
\]
returns $\True$ is $\vt$ is has the \structure\ an integer type and a type error otherwise.

\ProseParagraph
\OneApplies
\begin{itemize}
  \item \AllApplyCase{okay}
  \begin{itemize}
    \item determining the \structure\ of $\vt$ yields $\vtp$\ProseOrTypeError;
    \item $\vtp$ is an integer type;
    \item the result is $\True$;
  \end{itemize}

  \item \AllApplyCase{error}
  \begin{itemize}
    \item determining the \structure\ of $\vt$ yields $\vtp$\ProseOrTypeError;
    \item $\vtp$ is not an integer type;
    \item the result is a type error indicating that $\vt$ was expected to have the \structure\ of an integer.
  \end{itemize}
\end{itemize}

\CodeSubsection{\CheckStructureIntegerBegin}{\CheckStructureIntegerEnd}{../Typing.ml}

\FormallyParagraph
\begin{mathpar}
\inferrule[okay]{
  \tstruct(\vt) \typearrow \vtp \OrTypeError\\\\
  \astlabel(\vtp) = \TInt
}
{
  \checkstructureinteger(\tenv, \vt) \typearrow \True
}
\and
\inferrule[error]{
  \tstruct(\vt) \typearrow \vtp\\
  \astlabel(\vtp) \neq \TInt
}
{
  \checkstructureinteger(\tenv, \vt) \typearrow \TypeErrorVal{\UnexpectedType}
}
\end{mathpar}

\hypertarget{enumerationtypeterm}{}
\section{Enumeration Types\label{sec:EnumerationTypes}}
\subsection{Syntax}
\begin{flalign*}
\Ntydecl \derives\ & \Tenumeration \parsesep \Tlbrace \parsesep \TClistOne{\Tidentifier} \parsesep \Trbrace &
\end{flalign*}

\subsection{Abstract Syntax}
\begin{flalign*}
\ty \derives\ & \TEnum(\overtext{\identifier^{*}}{labels}) &
\end{flalign*}

\ASTRuleDef{TyDecl.TEnum}
\begin{mathpar}
\inferrule{
  \buildtclist[\buildidentity](\vids) \astarrow \vidasts
}{
  {
    \begin{array}{r}
  \buildtydecl(\Ntydecl(\Tenumeration, \Tlbrace, \namednode{\vids}{\TClistOne{\Tidentifier}}, \Trbrace)) \astarrow\\
  \overname{\TEnum(\vidasts)}{\vastnode}
\end{array}
  }
}
\end{mathpar}

\subsection{Typing Enumeration Types\label{sec:TypingEnumerationTypes}}
\TypingRuleDef{TEnumDecl}

\ProseParagraph
\AllApply
\begin{itemize}
  \item $\tty$ is the enumeration type with enumeration literals
    $\vli$, that is, $\TEnum(\vli)$;
  \item $\decl$ is $\True$, indicating that $\tty$ should be considered in the context of a declaration;
  \item determining that $\vli$ does not contain duplicates yields $\True$\ProseOrTypeError;
  \item determining that none of the labels in $\vli$ is declared in the global environment
  yields $\True$\ProseOrTypeError;
  \item $\newty$ is the enumeration type $\tty$;
  \item define $\vses$ as the empty set.
\end{itemize}
\FormallyParagraph
\begin{mathpar}
\inferrule{
  \checknoduplicates(\vli) \typearrow \True \OrTypeError\\\\
  \vl \in \vli: \checkvarnotingenv{G^\tenv, \vl} \typearrow \True \OrTypeError
}{
  \annotatetype{\True, \tenv, \TEnum(\vli)} \typearrow (\overname{\TEnum(\vli)}{\newty}, \overname{\emptyset}{\vses})
}
\CodeSubsection{\TEnumDeclBegin}{\TEnumDeclEnd}{../Typing.ml}

\end{mathpar}
\lrmcomment{This is related to \identd{YZBQ}, \identr{DWSP}, \identi{MZXL}.}
\subsubsection{Example}
\listingref{typing-tenum} shows an example of a well-typed enumeration type declaration.
\ASLListing{Well-typed enumeration type}{typing-tenum}{\typingtests/TypingRule.TEnumDecl.asl}

\hypertarget{recordtypeterm}{}
\section{Record Types\label{sec:RecordTypes}}
\subsection{Syntax}
\begin{flalign*}
\Ntydecl \derives\ & \Trecord \parsesep \Nfieldsopt &
\end{flalign*}

\subsection{Abstract Syntax}
\begin{flalign*}
\ty \derives\ & \TRecord(\Field^{*}) &
\end{flalign*}

\ASTRuleDef{TyDecl.TRecord}
\begin{mathpar}
\inferrule{}{
  \buildtydecl(\Ntydecl(\Trecord, \punnode{\Nfieldsopt})) \astarrow
  \overname{\TRecord(\astof{\vfieldsopt})}{\vastnode}
}
\end{mathpar}

\subsection{Typing Record Types\label{sec:TypingRecordTypes}}
\TypingRuleDef{TStructuredDecl}
\ProseParagraph
\AllApply
\begin{itemize}
  \item $\tty$ is a \structuredtype\ with AST label $L$;
  \item the list of fields of $\tty$ is $\fields$;
  \item $\decl$ is $\True$, indicating that $\tty$ should be considered in the context of a declaration;
  \item $\fields$ is a list of pairs where the first element is an identifier and the second is a type --- $(\vx_i, \vt_i)$, for $i=1..k$;
  \item checking that the list of field identifiers $\vx_{1..k}$ does not contain duplicates
  yields $\True$\ProseOrTypeError;
  \item annotating each field type $\vt_i$, for $i=1..k$, yields $(\vtp_i, \vxs_i)$
        \ProseOrTypeError;
  \item $\fieldsp$ is the list with $(\vx_i, \vtp_i)$, for $i=1..k$;
  \item $\newty$ is the AST node with AST label $L$ (either record type or exception type,
  corresponding to the type $\tty$) and fields $\fieldsp$;
  \item define $\vses$ as the union of all $\vxs_i$, for $i=1..k$.
\end{itemize}

\FormallyParagraph
\begin{mathpar}
\inferrule{
  L \in \{\TRecord, \TException\}\\
  \fields \eqname [i=1..k: (\vx_i, \vt_i)]\\
  \checknoduplicates(\vx_{1..k}) \typearrow \True \OrTypeError\\\\
  i=1..k: \annotatetype{\False, \tenv, \vt_i} \typearrow (\vtp_i, \vxs_i) \OrTypeError\\\\
  \fieldsp \eqdef [i=1..k: (\vx_i, \vtp_i)]\\
  \vses \eqdef \bigcup_{i=1..k} \vxs_i
}{
  \annotatetype{\True, \tenv, L(\fields)} \typearrow (\overname{L(\fieldsp)}{\newty}, \vses)
}
\end{mathpar}
\CodeSubsection{\TStructuredDeclBegin}{\TStructuredDeclEnd}{../Typing.ml}

\subsubsection{Example}
In \listingref{typing-trecord}, all the uses of record or exception types are well-typed.
\ASLListing{Well-typed structured types}{typing-trecord}{\typingtests/TypingRule.TRecordExceptionDecl.asl}

\hypertarget{exceptiontypeterm}{}
\section{Exception Types\label{sec:ExceptionTypes}}
\subsection{Syntax}
\begin{flalign*}
\Ntydecl \derives\ & \Texception \parsesep \Nfieldsopt &
\end{flalign*}

\subsection{Abstract Syntax}
\begin{flalign*}
\ty \derives\ & \TException(\Field^{*}) &
\end{flalign*}

\ASTRuleDef{TyDecl.TException}
\begin{mathpar}
\inferrule{}{
  \buildtydecl(\Ntydecl(\Texception, \punnode{\Nfieldsopt})) \astarrow
  \overname{\TException(\astof{\vfieldsopt})}{\vastnode}
}
\end{mathpar}

\subsection{Typing Exception Types}
The rule for typing exception types is \TypingRuleRef{TStructuredDecl}.

\hypertarget{namedtypeterm}{}
\section{Named Types\label{sec:NamedTypes}}
\subsection{Syntax}
\begin{flalign*}
\Nty \derives\ & \Tidentifier &
\end{flalign*}

\subsection{Abstract Syntax}
\begin{flalign*}
\ty \derives\ & \TNamed(\overtext{\identifier}{type name}) &
\end{flalign*}

\ASTRuleDef{Ty.TNamed}
\begin{mathpar}
\inferrule{}{
  \buildty(\Nty(\Tidentifier(\id))) \astarrow
  \overname{\TNamed(\id)}{\vastnode}
}
\end{mathpar}

\subsection{Typing Named Types\label{sec:TypingNamedTypes}}
\TypingRuleDef{TNamed}
\ProseParagraph
\AllApply
\begin{itemize}
  \item $\tty$ is the named type $\vx$, that is $\TNamed(\vx)$;
  \item checking whether $\vx$ is bound to any declared type in $\tenv$ yields $\True$\ProseOrTypeError;
  \item $\vx$ is bound to a type with associated \timeframeterm\ $\vtimeframe$;
  \item define $\vses$ as the singleton set for the \ReadGlobalTerm\ for $\vx$, $\vtimeframe$, and $\True$ for immutability;
  \item $\newty$ is $\tty$.
\end{itemize}
\FormallyParagraph
\begin{mathpar}
\inferrule{
  \checktrans{G^\tenv.\declaredtypes(\vx) \neq \bot}{\UndefinedIdentifier} \typearrow \True \OrTypeError\\\\
  G^\tenv.\declaredtypes(\vx) = (\Ignore, \vtimeframe)\\
  \vses \eqdef \{\ \ReadGlobal(\vx, \vtimeframe, \True)\ \}
}{
  \annotatetype{\overname{\Ignore}{\vdecl}, \tenv, \overname{\TNamed(\vx)}{\tty}} \typearrow (\overname{\TNamed(\vx)}{\newty}, \vses)
}
\end{mathpar}
\CodeSubsection{\TNamedBegin}{\TNamedEnd}{../Typing.ml}

\subsubsection{Example}
In \listingref{typing-tnamed}, all the uses of \texttt{MyType} are well-typed.
\ASLListing{Well-typed named types}{typing-tnamed}{\typingtests/TypingRule.TNamed.asl}

\section{Declared Types\label{sec:DeclaredTypes}}
A declared type can be an enumeration type, a record type, an exception type, or an \anonymoustype.
\subsection{Syntax}
\begin{flalign*}
\Ntydecl \derives\ & \Nty &
\end{flalign*}

\subsection{Abstract Syntax}
\ASTRuleDef{TyDecl}
\begin{mathpar}
\inferrule[ty]{}{
  \buildtydecl(\Ntydecl(\punnode{\Nty})) \astarrow
  \overname{\astof{\tty}}{\vastnode}
}
\end{mathpar}

\subsection{Typing Declared Types}
\lrmcomment{\identr{RGRVJ}}
\RequirementDef{RestrictionsOnAnonymousTypes}
A declared type for an enumeration, a record type, or an exception type
are only permitted in named type declarations. This is enforced by \TypingRuleRef{TNonDecl}.
%
See \ExampleRef{Ill-typed pending-constrained integer type}.

\TypingRuleDef{TNonDecl}
\subsubsection{Example}
In \listingref{typing-trecorderror}, the use of a record type outside of a declaration is erroneous.
\ASLListing{An erroneous use of a record type}{typing-trecorderror}{\typingtests/TypingRule.TNonDecl.asl}

\ProseParagraph
\AllApply
\begin{itemize}
  \item $\tty$ is a \structuredtype\ or an enumeration type;
  \item $\decl$ is $\False$, indicating that $\tty$ should be considered to be outside the context of a declaration
  of $\tty$;
  \item a type error is returned, indicating that the use of anonymous form of enumerations, record,
  and exceptions types is not allowed here.
\end{itemize}

\FormallyParagraph
\begin{mathpar}
\inferrule{
  \astlabel(\tty) \in \{\TEnum, \TRecord, \TException\}
}{
  \annotatetype{\False, \tenv, \tty} \typearrow \TypeErrorVal{\UnexpectedType}
}
\end{mathpar}
\CodeSubsection{\TNonDeclBegin}{\TNonDeclEnd}{../Typing.ml}

%%%%%%%%%%%%%%%%%%%%%%%%%%%%%%%%%%%%%%%%%%%%%%%%%%%%%%%%%%%%%%%%%%%%%%%%%%%%%%%%%%%%
\section{Domain of Values for Types\label{sec:DomainOfValuesForTypes}}
%%%%%%%%%%%%%%%%%%%%%%%%%%%%%%%%%%%%%%%%%%%%%%%%%%%%%%%%%%%%%%%%%%%%%%%%%%%%%%%%%%%%
This section formalizes the concept of the set of values for a given type.
The formalism is given in the form of rules.
%
The section also defines the concept of checking whether the set of values
for one type is included in the set of values for another type.

\subsection{Dynamic Domain of a Type\label{sec:DynDomain}}
\hypertarget{def-dyndomain}{}

We now define the concept of a \emph{dynamic domain} of a type
and the \emph{static domain} of a type.
Intuitively, domains assign potentially infinite sets of \nativevalues\ to types.
Dynamic domains are used by the semantics to evaluate expressions of the form \texttt{ARBITRARY: t}
by choosing a single value from the dynamic domain of $\vt$.
Static domains are used to define subtype satisfaction in \secref{TypingRule.SubtypeSatisfaction}.

Formally, the partial function
\[
  \dynamicdomain : \overname{\envs}{\env} \times \overname{\ty}{\vt}
  \partialto \overname{\pow{\vals}}{\vd}
\]
assigns the set of values that a type $\vt$ can hold in a given environment $\env$.
%
We say that $\dynamicdomain(\env, \vt)$ is the \emph{dynamic domain} of $\vt$
in the environment $\env$.
%
The \emph{static domain} of a type is the set of values which storage elements of that type may hold
\underline{across all possible dynamic environments}.
%
The reason for this distinction is that the sets of values
of integer types, bitvector types, and array types can depend on the dynamic values of variables.

Types that do not refer to variables whose values are only known dynamically have
a static domain that is equal to any of their dynamic domains.
In those cases, we simply refer to their \emph{domain}.

Associating a set of values to a type is done by evaluating any expression appearing
in the type definitions.
%
Expressions appearing in types are guaranteed to be side-effect-free by the
function $\annotatetype$.
%
Evaluation is defined by the relation $\evalexprsef\empty$.
which evaluates side-effect-free expressions and either returns
a configuration of the form $\Normal(\vv,\vg)$ or a dynamic error configuration $\DynErrorConfig$.
In the first case, $\vv$ is a \nativevalue\ and $\vg$
is an \emph{execution graph}. Execution graphs are related to the concurrent semantics
and can be ignored in the context of defining dynamic domains.
In the latter case (which can occur if, for example, an expression attempts to divide
\texttt{8} by \texttt{0}), a dynamic error configuration, for which we use the notation
$\DynErrorConfig$, is returned.
%
The dynamic domain is empty in cases where evaluating side-effect-free expressions
results in a dynamic error.
%
The dynamic domain is undefined if the type $\vt$ is not well-typed in $\tenv$.
That is, if $\annotatetype{\tenv, \vt} \typearrow \TypeErrorConfig$.

As part of the definition, we also associate dynamic domains to integer constraints
by overloading $\dynamicdomain$:
\[
  \dynamicdomain : \overname{\envs}{\env} \times \overname{\intconstraint}{\vc}
  \partialto \overname{\pow{\vals}}{\vd}
\]

\subsubsection{Prose}
For an environment $\env \in \envs$ and a type $\vt$, the domain is $\vd$ and one of the following applies:
\begin{itemize}
  \item All of the following apply (\textsc{t\_bool}):
  \begin{itemize}
    \item $\vt$ is the Boolean type, $\TBool$;
    \item $\vd$ is the set of native Boolean values, $\tbool$.
  \end{itemize}

  \item All of the following apply (\textsc{t\_string}):
  \begin{itemize}
    \item $\vt$ is the string type, $\TString$;
    \item $\vd$ is the set of all native string values, $\tstring$.
  \end{itemize}

  \item All of the following apply (\textsc{t\_real}):
  \begin{itemize}
    \item $\vt$ is the real type, $\TReal$;
    \item $\vd$ is the set of all native real values, $\treal$.
  \end{itemize}

  \item All of the following apply (\textsc{t\_enumeration}):
  \begin{itemize}
    \item $\vt$ is the enumeration type with labels $\id_{1..k}$, that is $\TEnum(\id_{1..k})$;
    \item $\vd$ is the set of all native enumeration labels $\llabel(\id_\vi, \vi)$, for $\vi = 1..k$.
  \end{itemize}

  \item All of the following apply (\textsc{t\_int\_unconstrained}):
  \begin{itemize}
    \item $\vt$ is the unconstrained integer type, $\unconstrainedinteger$;
    \item $\vd$ is the set of all native integer values, $\tint$.
  \end{itemize}

  \item All of the following apply (\textsc{t\_int\_well\_constrained}):
  \begin{itemize}
    \item $\vt$ is the well-constrained integer type $\TInt(\wellconstrained(\vc_{1..k}))$;
    \item $\vd$ is the union of the dynamic domains of each of the constraints $\vc_{1..k}$ in $\env$.
  \end{itemize}

  \item All of the following apply (\textsc{constraint\_exact\_okay}):
  \begin{itemize}
    \item $\vc$ is a constraint consisting of a single side-effect-free expression $\ve$, that is, $\ConstraintExact(\ve)$;
    \item evaluating $\ve$ in $\env$, results in a configuration with the native integer for $n$;
    \item $\vd$ is the set containing the single native integer value for $n$.
  \end{itemize}

  \item All of the following apply (\textsc{constraint\_exact\_dynamic\_error}):
  \begin{itemize}
    \item $\vc$ is a constraint consisting of a single side-effect-free expression $\ve$, that is, $\ConstraintExact(\ve)$;
    \item evaluating $\ve$ in $\env$, results in a dynamic error configuration;
    \item $\vd$ is the empty set.
  \end{itemize}

  \item All of the following apply (\textsc{constraint\_range\_okay}):
  \begin{itemize}
    \item $\vc$ is a range constraint consisting of a two side-effect-free expressions $\veone$ and $\vetwo$, that is, $\ConstraintRange(\veone, \vetwo)$;
    \item evaluating $\veone$ in $\env$, results in a configuration with the native integer for $a$;
    \item evaluating $\vetwo$ in $\env$, results in a configuration with the native integer for $b$;
    \item $\vd$ is the set containing all native integer values for integers greater or equal to $a$ and less than or equal to $b$.
  \end{itemize}

  \item All of the following apply (\textsc{constraint\_range\_dynamic\_error1}):
  \begin{itemize}
    \item $\vc$ is a range constraint consisting of a two side-effect-free expressions $\veone$ and $\vetwo$, that is, $\ConstraintRange(\veone, \vetwo)$;
    \item evaluating $\veone$ in $\env$, results in a dynamic error configuration;
    \item $\vd$ is the empty set.
  \end{itemize}

  \item All of the following apply (\textsc{constraint\_range\_dynamic\_error2}):
  \begin{itemize}
    \item $\vc$ is a range constraint consisting of a two side-effect-free expressions $\veone$ and $\vetwo$, that is, $\ConstraintRange(\veone, \vetwo)$;
    \item evaluating $\veone$ in $\env$, results in a configuration with the native integer for $a$;
    \item evaluating $\vetwo$ in $\env$, results in a dynamic error configuration;
    \item $\vd$ is the empty set.
  \end{itemize}

  \item All of the following apply (\textsc{t\_int\_parameterized}):
  \begin{itemize}
    \item $\vt$ is a \parameterizedintegertype\ for parameter $\id$, \\ $\TInt(\parameterized(\id))$;
    \item the \nativevalue\ associated with $\id$ in the local dynamic environment is the native integer value for $n$;
    \item $\vd$ is the set containing the single integer value for $n$.
  \end{itemize}

  \item All of the following apply (\textsc{t\_bits\_dynamic\_error}):
  \begin{itemize}
    \item $\vt$ is a bitvector type with size expression $\ve$, $\TBits(\ve, \Ignore)$;
    \item evaluating $\ve$ in $\env$, results in a dynamic error configuration;
    \item $\vd$ is the empty set.
  \end{itemize}

  \item All of the following apply (\textsc{t\_bits\_negative\_width\_error}):
  \begin{itemize}
    \item $\vt$ is a bitvector type with size expression $\ve$, $\TBits(\ve, \Ignore)$;
    \item evaluating $\ve$ in $\env$, results in a configuration with the native integer for $k$;
    \item $k$ is negative;
    \item $\vd$ is the empty set.
  \end{itemize}

  \item All of the following apply (\textsc{t\_bits\_empty}):
  \begin{itemize}
    \item $\vt$ is a bitvector type with size expression $\ve$, $\TBits(\ve, \Ignore)$;
    \item evaluating $\ve$ in $\env$, results in a configuration with the native integer for $0$;
    \item $\vd$ is the set containing the single \nativevalue\ for an empty bitvector.
  \end{itemize}

  \item All of the following apply (\textsc{t\_bits\_non\_empty}):
  \begin{itemize}
    \item $\vt$ is a bitvector type with size expression $\ve$, $\TBits(\ve, \Ignore)$;
    \item evaluating $\ve$ in $\env$, results in a configuration with the native integer for $k$;
    \item $k$ is greater than $0$;
    \item $\vd$ is the set containing all \nativevalues\ for bitvectors of size exactly $k$.
  \end{itemize}

  \item All of the following apply (\textsc{t\_tuple}):
  \begin{itemize}
    \item $\vt$ is a tuple type over types $\vt_i$, for $i=1..k$, $\TTuple(\vt_{1..k})$;
    \item the domain of each element $\vt_i$ is $D_i$, for $i=1..k$;
    \item evaluating $\ve$ in $\env$, results in a configuration with the native integer for $k$;
    \item $\vd$ is the set containing all native vectors of $k$ values, where the value at position $i$
    is from $D_i$.
  \end{itemize}

  \item All of the following apply (\textsc{t\_array\_dynamic\_error}):
  \begin{itemize}
    \item $\vt$ is an array type with length expression $\ve$ and element type $\vt_i$, for $i=1..k$, $\TArray(\ve, \vtone)$;
    \item evaluating $\ve$ in $\env$, results in a dynamic error configuration;
    \item $\vd$ is the empty set.
  \end{itemize}

  \item All of the following apply (\textsc{t\_array\_negative\_length\_error}):
  \begin{itemize}
    \item $\vt$ is an array type with length expression $\ve$ and element type $\vt_i$, for $i=1..k$, $\TArray(\ve, \vtone)$;
    \item evaluating $\ve$ in $\env$, results in a configuration with the native integer for $k$;
    \item $k$ is negative;
    \item $\vd$ is the empty set.
  \end{itemize}

  \item All of the following apply (\textsc{t\_array\_okay}):
  \begin{itemize}
    \item $\vt$ is an array type with length expression $\ve$ and element type $\vt_i$, for $i=1..k$, $\TArray(\ve, \vtone)$;
    \item evaluating $\ve$ in $\env$, results in a configuration with the native integer for $k$;
    \item $k$ is greater than or equal to $0$;
    \item the domain of $\vtone$ is $D_\vtone$;
    \item $\vd$ is the set containing all native vectors of $k$ values taken from $D_\vtone$.
  \end{itemize}

  \item All of the following apply (\textsc{t\_structured}):
  \begin{itemize}
    \item $\vt$ is a \structuredtype\ with typed fields $(\id_i, \vt_i$, for $i=1..k$, that is $L([i=1..k: (\id_i,\vt_i))]$
    where $L\in\{\TRecord, \TException\}$;
    \item the domain of each type $\vt_i$ is $D_i$, for $i=1..k$;
    \item $\vd$ is the set containing all native records where $\id_i$ is mapped to a value taken from $D_i$.
  \end{itemize}

  \item All of the following apply (\textsc{t\_named}):
  \begin{itemize}
    \item $\vt$ is a named type with name $\id$, $\TNamed(\id)$;
    \item the type associated with $\id$ in $\tenv$ is $\tty$;
    \item $\vd$ is the domain of $\tty$ in $\env$.
  \end{itemize}
\end{itemize}

\subsubsection{Formally}

\begin{mathpar}
\inferrule[t\_bool]{}{ \dynamicdomain(\env, \overname{\TBool}{\vt}) = \overname{\tbool}{\vd} }
\and
\inferrule[t\_string]{}{ \dynamicdomain(\env, \overname{\TString}{\vt}) = \overname{\tstring}{\vd} }
\and
\inferrule[t\_real]{}{ \dynamicdomain(\env, \overname{\TReal}{\vt}) = \overname{\treal}{\vd} }
\and
\inferrule[t\_enumeration]{}{
  \dynamicdomain(\env, \overname{\TEnum(\id_{1..k})}{\vt}) = \overname{\{ \vi = 1..k: \nvliteral{\llabel(\id_\vi, \vi)} \}}{\vd}
}
\end{mathpar}

\begin{mathpar}
  \inferrule[t\_int\_unconstrained]{}{
  \dynamicdomain(\env, \overname{\unconstrainedinteger}{\vt}) = \overname{\tint}{\vd}
}
\end{mathpar}

\begin{mathpar}
\inferrule[t\_int\_well\_constrained]{}{
  \dynamicdomain(\env, \overname{\TInt(\wellconstrained(\vc_{1..k}))}{\vt}) = \overname{\bigcup_{i=1}^k \dynamicdomain(\env, \vc_i)}{\vd}
}
\end{mathpar}

\begin{mathpar}
\inferrule[constraint\_exact\_okay]{
  \evalexprsef{\env, \ve} \evalarrow \Normal(\nvint(n), \Ignore)
}{
  \dynamicdomain(\env, \overname{\ConstraintExact(\ve)}{\vc}) = \overname{\{ \nvint(n) \}}{\vd}
}
\and
\inferrule[constraint\_exact\_dynamic\_error]{
  \evalexprsef{\env, \ve} \evalarrow \DynErrorConfig
}{
  \dynamicdomain(\env, \overname{\ConstraintExact(\ve)}{\vc}) = \overname{\emptyset}{\vd}
}
\end{mathpar}

\begin{mathpar}
\inferrule[constraint\_range\_okay]{
  \evalexprsef{\env, \veone} \evalarrow \Normal(\nvint(a), \Ignore)\\
  \evalexprsef{\env, \vetwo} \evalarrow \Normal(\nvint(b), \Ignore)
}{
  \dynamicdomain(\env, \overname{\ConstraintRange(\veone, \vetwo)}{\vc}) = \overname{\{ \nvint(n) \;|\;  a \leq n \land n \leq b\}}{\vd}
}
\and
\inferrule[constraint\_range\_dynamic\_error1]{
  \evalexprsef{\env, \veone} \evalarrow \DynErrorConfig
}{
  \dynamicdomain(\env, \overname{\ConstraintRange(\veone, \vetwo)}{\vc}) = \overname{\emptyset}{\vd}
}
\and
\inferrule[constraint\_range\_dynamic\_error2]{
  \evalexprsef{\env, \veone} \evalarrow \Normal(\Ignore, \Ignore)\\
  \evalexprsef{\env, \vetwo} \evalarrow \DynErrorConfig
}{
  \dynamicdomain(\env, \overname{\ConstraintRange(\veone, \vetwo)}{\vc}) = \overname{\emptyset}{\vd}
}
\end{mathpar}

The notation $L^\denv(\id)$ denotes the \nativevalue\ associated with the identifier $\id$
in the \emph{local dynamic environment} of $\denv$.
\begin{mathpar}
  \inferrule[t\_int\_parameterized]{
  L^\denv(\id) = \nvint(n)
}{
  \dynamicdomain(\env, \overname{\TInt(\parameterized(\id))}{\vt}) = \overname{\{ \nvint(n) \}}{\vd}
}
\end{mathpar}

\begin{mathpar}
\inferrule[t\_bits\_dynamic\_error]{
  \evalexprsef{\env, \ve} \evalarrow \DynErrorConfig
}{
  \dynamicdomain(\env, \overname{\TBits(\ve, \Ignore)}{\vt}) = \overname{\emptyset}{\vd}
}
\and
\inferrule[t\_bits\_negative\_width\_error]{
  \evalexprsef{\env, \ve} \evalarrow \Normal(\nvint(k), \Ignore)\\
  k < 0
}{
  \dynamicdomain(\env, \overname{\TBits(\ve, \Ignore)}{\vt}) = \overname{\emptyset}{\vd}
}
\and
\inferrule[t\_bits\_empty]{
  \evalexprsef{\env, \ve} \evalarrow \Normal(\nvint(0), \Ignore)
}{
  \dynamicdomain(\env, \overname{\TBits(\ve, \Ignore)}{\vt}) = \overname{\{ \nvbitvector(\emptylist) \}}{\vd}
}
\and
\inferrule[t\_bits\_non\_empty]{
  \evalexprsef{\env, \ve} \evalarrow \Normal(\nvint(k), \Ignore)\\
  k > 0
}{
  \dynamicdomain(\env, \overname{\TBits(\ve, \Ignore)}{\vt}) = \overname{\{ \nvbitvector(\vb_{1..k}) \;|\; \vb_1,\ldots,\vb_k \in \{0,1\} \}}{\vd}
}
\end{mathpar}

\begin{mathpar}
\inferrule[t\_tuple]{
  i=1..k: \dynamicdomain(\env, \vt_i) = D_i
}{
  \dynamicdomain(\env, \overname{\TTuple(\vt_{1..k})}{\vt}) =
  \overname{\{ \nvvector{\vv_{1..k}} \;|\; \vv_i \in D_i \}}{\vd}
}
\end{mathpar}

\begin{mathpar}
\inferrule[t\_array\_dynamic\_error]{
  \evalexprsef{\env, \ve} \evalarrow \DynErrorConfig
}{
  \dynamicdomain(\env, \overname{\TArray(\ve, \vtone)}{\vt}) = \overname{\emptyset}{\vd}
}
\and
\inferrule[t\_array\_negative\_length\_error]{
  \evalexprsef{\env, \ve} \evalarrow \Normal(\nvint(k), \Ignore)\\
  k < 0
}{
  \dynamicdomain(\env, \overname{\TArray(\ve, \vtone)}{\vt}) = \overname{\emptyset}{\vd}
}
\and
\inferrule[t\_array\_okay]{
  \evalexprsef{\env, \ve} \evalarrow \Normal(\nvint(k), \Ignore)\\
  k \geq 0\\
  \dynamicdomain(\env, \vtone) = D_\vtone
}{
  \dynamicdomain(\env, \overname{\TArray(\ve, \vtone)}{\vt}) =
  \overname{\{ \nvvector{\vv_{1..k}} \;|\; \vv_{1..k} \in D_{\vtone} \}}{\vd}
}
\end{mathpar}

\begin{mathpar}
\inferrule[structured]{
  L \in \{\TRecord, \TException\}\\
  i=1..k: \dynamicdomain(\env, \vt_i) = D_i
}{
  \dynamicdomain(\env, \overname{L([i=1..k: (\id_i,\vt_i))]}{\vt}) = \\
  \overname{\{ \nvrecord{\{i=1..k: \id_i\mapsto \vv_i\}} \;|\; \vv_i \in D_i \}}{\vd}
}
\end{mathpar}

\begin{mathpar}
\inferrule[t\_named]{
  G^\tenv.\declaredtypes(\id)=\tty
}{
  \dynamicdomain(\env, \overname{\TNamed(\id)}{\vt}) = \overname{\dynamicdomain(\env, \tty)}{\vd}
}
\end{mathpar}

\subsubsection{Examples}
The domain of \texttt{integer} is the infinite set of all integers.

The domain of \verb|integer {2,16}| is the set $\{\nvint(2), \nvint(16)\}$.

The domain of \verb|integer{1..3}| is the set $\{\nvint(1), \nvint(2), \nvint(3)\}$.

The domain of \verb|integer{10..1}| is the empty set as there are no integers that are
both greater than $10$ and smaller than $1$.

The domain of \texttt{bits(2)} is the set $\{\nvbitvector(00)$, $\nvbitvector(01),$
$\nvbitvector(10)$, $\nvbitvector(11)\}$.

The domain of \verb|enumeration {GREEN, ORANGE, RED}| is the set \\
$\{\nvint(1), \nvint(2), \nvint(3)\}$ and so is the domain
of \\
\verb|type TrafficLights of enumeration {GREEN, ORANGE, RED}|.

The domain of \texttt{bits({2,16})} is the set containing native bitvectors of all 2-bit and all 16-bit binary sequences.

The domain of \texttt{(integer, integer)} is the set containing all pairs of native integer values.

The domain of \verb|record {a: integer;  b: boolean}| contains all native records
that map \texttt{a} to a native integer value and \texttt{b} to a native Boolean value.

The dynamic domain of a subprogram parameter \texttt{N: integer} is the (singleton) set containing
the native integer value $c$,
which is assigned to \texttt{N} by a given dynamic environment. The static domain of that parameter
is the infinite set of all native integer values.

\lrmcomment{
This is related to \identd{BMGM}, \identr{PHRL}, \identr{PZNR},
\identr{RLQP}, \identr{LYDS}, \identr{SVDJ}, \identi{WLPJ}, \identr{FWMM},
\identi{WPWL}, \identi{CDVY}, \identi{KFCR}, \identi{BBQR}, \identr{ZWGH},
\identr{DKGQ}, \identr{DHZT}, \identi{HSWR}, \identd{YZBQ}.
}

\subsection{Subsumption Testing\label{sec:subsumptiontesting}}
Whether an assignment statement is well-typed depends on whether the dynamic domain of the
right hand side type is contained in the dynamic domain of the left hand side type,
for any given dynamic environment
(see \secref{TypingRule.SubtypeSatisfaction} where this is checked).

\begin{definition}[Subsumption]
For any given types $\vt$ and $\vs$ and static environment $\tenv$,
we say that $\vt$ \emph{subsumes} $\vs$ in $\tenv$,
if the following condition holds:
\hypertarget{def-subsumes}{}
\begin{equation}
  \subsumes(\tenv, \vt, \vs) \triangleq \forall \denv\in\dynamicenvs.\ \dynamicdomain((\tenv, \denv), \vt) \supseteq \dynamicdomain((\tenv, \denv), \vs) \enspace.
\end{equation}
\end{definition}

For example, consider the assignment
\begin{center}
\verb|var x : integer{1,2,3} = ARBITRARY : integer{1,2};|
\end{center}

It is legal, since (in any static environment), the domain of \verb|integer{1,2,3}|
is \\
$\{\nvint(1), \nvint(2), \nvint(3)\}$, which subsumes
the domain of \verb|integer{1,2}|, which is \\ $\{\nvint(1), \nvint(2)\}$.

Since dynamic domains are potentially infinite, this requires \emph{symbolic reasoning}.
Furthermore, since any (statically evaluable) expressions may appear inside integer and bitvector
types, testing subsumption is undecidable.
We therefore approximate subsumption testing \emph{conservatively} via the predicate $\symsubsumes(\tenv, \vt, \vs)$.

\hypertarget{def-soundsubsumptiontest}{}
\begin{definition}[Sound Subsumption Test]
A predicate
\[
  \symsubsumes(\overname{\staticenvs}{\tenv} \aslsep \overname{\ty}{\vt} \aslsep \overname{\ty}{\vs}) \aslto \Bool
\]
is \emph{sound} if the following condition holds:
\begin{equation}
  \begin{array}{l}
  \forall \vt,\vs\in\ty.\ \tenv\in\staticenvs. \\
  \;\;\;\; \symsubsumes(\tenv, \vt, \vs) \typearrow \True \;\Longrightarrow\; \subsumes(\tenv, \vt, \vs)  \enspace.
  \end{array}
\end{equation}
\end{definition}

That is, if a sound subsumption test returns a positive answer, it means that
$\vt$ definitely \emph{subsumes} $\vs$ in the static environment $\tenv$.
This is referred to as a \emph{true positive}.
However, a negative answer means one of two things:
\begin{description}
  \item[True Negative:] indeed, $\vt$ does not subsume $\vs$ in the static environment $\tenv$; or
  \item[False Negative:] the symbolic reasoning is unable to decide.
\end{description}

In other words, $\symsubsumes(\tenv, \vt, \vs)$ errs on the \emph{safe side} ---
it never answers $\True$ when the real answer is $\False$, which would (undesirably)
determine the following statement as well-typed:
\begin{center}
  \verb|var x : integer{1,2} = ARBITRARY: integer;|
\end{center}

A sound but trivial subsumption test is one that always returns $\False$.
However, that would make all assignments be considered as not well-typed.
Indeed, it has the maximal set of false negatives.
Reducing the set of false negatives requires stronger symbolic reasoning algorithms,
which inevitably leads to higher computational complexity.
%
The symbolic subsumption test in \chapref{SymbolicSubsumptionTesting}
attempts to accept a large enough set of true positives, based on empirical trial and error,
while maintaining the computational complexity of the symbolic reasoning relatively low.
%
In particular, it serves as the definitive subsumption test that must be utilized
by any implementation of the ASL type system.

%%%%%%%%%%%%%%%%%%%%%%%%%%%%%%%%%%%%%%%%%%%%%%%%%%%%%%%%%%%%%%%%%%%%%%%%%%%%%%%%%%%%
\section{Basic Type Attributes\label{sec:BasicTypeAttributes}}
%%%%%%%%%%%%%%%%%%%%%%%%%%%%%%%%%%%%%%%%%%%%%%%%%%%%%%%%%%%%%%%%%%%%%%%%%%%%%%%%%%%%

This section defines some basic predicates for classifying types as well as
functions that inspect the structure of types:
\begin{itemize}
  \item Builtin singular types (\TypingRuleRef{BuiltinSingularType})
  \item Builtin aggregate types (\TypingRuleRef{BuiltinAggregateType})
  \item Builtin types (\TypingRuleRef{BuiltinSingularOrAggregate})
  \item Named types (\TypingRuleRef{NamedType})
  \item Anonymous types (\TypingRuleRef{AnonymousType})
  \item Singular types (\TypingRuleRef{SingularType})
  \item Aggregate types (\TypingRuleRef{AggregateType})
  \item Structured types (\TypingRuleRef{StructuredType})
  \item Non-primitive types (\TypingRuleRef{NonPrimitiveType})
  \item Primitive types (\TypingRuleRef{PrimitiveType})
  \item The structure of a type (\TypingRuleRef{Structure})
  \item The underlying type of a type (\TypingRuleRef{MakeAnonymous})
  \item Checked constrained integers (\TypingRuleRef{CheckConstrainedInteger})
\end{itemize}

Finally, constrained types are defined in \secref{ConstrainedTypes}.

\TypingRuleDef{BuiltinSingularType}
\hypertarget{def-isbuiltinsingular}{}
The predicate
\[
  \isbuiltinsingular(\overname{\ty}{\tty}) \;\aslto\; \Bool
\]
tests whether the type $\tty$ is a \emph{builtin singular type}.

\subsubsection{Prose}
The \emph{builtin singular types} are:
\begin{itemize}
\item \texttt{integer};
\item \texttt{real};
\item \texttt{string};
\item \texttt{boolean};
\item \texttt{bits} (which also represents \texttt{bit}, as a special case);
\item \texttt{enumeration}.
\end{itemize}

\subsubsection{Example}
\listingref{typing-builtinsingulartype} defines variables of builtin singular types
\texttt{integer}, \texttt{real},
\texttt{boolean}, \texttt{bits(4)}, and~\texttt{bits(2)}
\ASLListing{Examples of builtin singular types}{typing-builtinsingulartype}{\typingtests/TypingRule.BuiltinSingularTypes.asl}

\subsubsection{Example}
In \listingref{typing-builtinenumerationtype},
the builtin singular type \texttt{Color} consists in two constants:
\texttt{RED} and~\texttt{BLACK}.
\ASLListing{An enumeration type}{typing-builtinenumerationtype}{\typingtests/TypingRule.EnumerationType.asl}

\subsubsection{Formally}
\begin{mathpar}
\inferrule{
  \vb \eqdef \astlabel(\tty) \in \{\TReal, \TString, \TBool, \TBits, \TEnum, \TInt\}
}{
  \isbuiltinsingular(\tty) \typearrow \vb
}
\end{mathpar}
\CodeSubsection{\BuiltinSingularBegin}{\BuiltinSingularEnd}{../types.ml}

\isempty{\subsubsection{Comments}}
\lrmcomment{This is related to \identd{PQCK} and \identd{NZWT}.}

\TypingRuleDef{BuiltinAggregateType}
\hypertarget{def-isbuiltinaggregate}{}
The predicate
\[
  \isbuiltinaggregate(\overname{\ty}{\tty}) \;\aslto\; \Bool
\]
tests whether the type $\tty$ is a \emph{builtin aggregate type}.

\subsubsection{Prose}
The builtin aggregate types are:
\begin{itemize}
\item tuple;
\item \texttt{array};
\item \texttt{record};
\item \texttt{exception}.
\end{itemize}

\subsubsection{Example}
\listingref{typing-builtinaggregatetypes} provides examples of some builtin aggregate types.
\ASLListing{Builtin aggregate types}{typing-builtinaggregatetypes}{\typingtests/TypingRule.BuiltinAggregateTypes.asl}

Type \texttt{Pair} is the type of integer and boolean pairs.

Arrays are declared with indices that are either integer-typed
or enumeration-typed.  In the example above, \texttt{T} is
declared as an array with an integer-typed index (as indicated
by the used of the integer-typed constant \texttt{3}) whereas
\texttt{PointArray} is declared with the index of
\texttt{Coord}, which is an enumeration type.

Arrays declared with integer-typed indices can be accessed only by integers ranging from $0$ to
the size of the array minus $1$. In the example above, $\texttt{T}$ can be accessed with
one of $0$, $1$, and $2$.

Arrays declared with an enumeration-typed index can only be accessed with labels from the corresponding
enumeration. In the example above, \texttt{PointArray} can only be accessed with one of the labels
\texttt{CX}, \texttt{CY}, and \texttt{CZ}.

The (builtin aggregate) type \verb|{ x : real, y : real, z : real }| is a record type with three fields
\texttt{x}, \texttt{y} and \texttt{z}.

\subsubsection{Example}
\listingref{typing-builtinexceptiontype} defines two (builtin aggregate) exception types:
\begin{itemize}
\item \verb|exception{}| (for \texttt{Not\_found}), which carries no value; and
\item \verb|exception { message:string }| (for \texttt{SyntaxException}), which carries a message.
\end{itemize}
Notice the similarity with record types and that the empty field list \verb|{}| can be
omitted in type declarations, as is the case for \texttt{Not\_found}.

\ASLListing{Exception types}{typing-builtinexceptiontype}{\typingtests/TypingRule.BuiltinExceptionType.asl}

\subsubsection{Formally}
\begin{mathpar}
\inferrule{ \vb \eqdef \astlabel(\tty) \in \{\TTuple, \TArray, \TRecord, \TException\} }
{ \isbuiltinaggregate(\tty) \typearrow \vb }
\end{mathpar}
\CodeSubsection{\BuiltinAggregateBegin}{\BuiltinAggregateEnd}{../types.ml}

\isempty{\subsubsection{Comments}}
\lrmcomment{This is related to \identd{PQCK} and \identd{KNBD}.}

\TypingRuleDef{BuiltinSingularOrAggregate}
\hypertarget{def-isbuiltin}{}
The predicate
\[
  \isbuiltin(\overname{\ty}{\tty}) \;\aslto\; \Bool
\]
tests whether the type $\tty$ is a \emph{builtin type}.

\subsubsection{Prose}
$\tty$ is a builtin type and one of the following applies:
\begin{itemize}
\item $\tty$ is singular;
\item $\tty$ is builtin aggregate.
\end{itemize}

\subsubsection{Example}
In the specification
\begin{verbatim}
  type ticks of integer;
\end{verbatim}
the type \texttt{integer} is a builtin type but the type of \texttt{ticks} is not.

\CodeSubsection{\BuiltinSingularOrAggregateBegin}{\BuiltinSingularOrAggregateEnd}{../types.ml}

\subsubsection{Formally}
\begin{mathpar}
  \inferrule{
    \isbuiltinsingular(\tty) \typearrow \vbone\\
    \isbuiltinaggregate(\tty) \typearrow \vbtwo
  }{
    \isbuiltin(\tty) \typearrow \vbone \lor \vbtwo
  }
\end{mathpar}

\TypingRuleDef{NamedType}
\hypertarget{def-isnamed}{}
The predicate
\[
  \isnamed(\overname{\ty}{\tty}) \;\aslto\; \Bool
\]
tests whether the type $\tty$ is a \emph{named type}.

Enumeration types, record types, and exception types must be declared
and associated with a named type.

\subsubsection{Prose}
A named type is a type that is declared by using the \texttt{type of} syntax.

\subsubsection{Example}
In the specification
\begin{verbatim}
  type ticks of integer;
\end{verbatim}
\texttt{ticks} is a named type.

\CodeSubsection{\NamedBegin}{\NamedEnd}{../types.ml}

\subsubsection{Formally}
\begin{mathpar}
\inferrule{
  \vb \eqdef \astlabel(\tty) = \TNamed
}{
  \isnamed(\tty) \typearrow \vb
}
\end{mathpar}

\isempty{\subsubsection{Comments}}
\lrmcomment{This is related to \identd{vmzx}.}

\TypingRuleDef{AnonymousType}
\hypertarget{def-isanonymous}{}
The predicate
\[
  \isanonymous(\overname{\ty}{\tty}) \;\aslto\; \Bool
\]
tests whether the type $\tty$ is an \anonymoustype.

\subsubsection{Prose}
\Anonymoustypes\ are types that are not declared using the type syntax:
integer types, the real type, the string type, the Boolean type,
bitvector types, tuple types, and array types.

\subsubsection{Example}
The tuple type \texttt{(integer, integer)} is an \anonymoustype.

\CodeSubsection{\AnonymousBegin}{\AnonymousEnd}{../types.ml}

\subsubsection{Formally}
\begin{mathpar}
\inferrule{ \vb \eqdef \astlabel(\tty) \neq \TNamed
}
{
  \isanonymous(\tty) \typearrow \vb
}
\end{mathpar}

\isempty{\subsubsection{Comments}}
\lrmcomment{This is related to \identd{VMZX}.}

\TypingRuleDef{SingularType}
\hypertarget{def-issingular}{}
The predicate
\[
  \issingular(\overname{\staticenvs}{\tenv} \aslsep \overname{\ty}{\tty}) \;\aslto\;
  \overname{\Bool}{\vb} \cup \overname{\TTypeError}{\TypeErrorConfig}
\]
tests whether the type $\tty$ is a \emph{singular type} in the static environment $\tenv$.

\subsubsection{Prose}
A type $\tty$ is singular if and only if all of the following apply:
\begin{itemize}
  \item obtaining the \underlyingtype\ of $\tty$ in the environment $\tenv$ yields $\vtone$\ProseOrTypeError;
  \item $\vtone$ is a builtin singular type.
\end{itemize}

\subsubsection{Example}
In the following example, the types \texttt{A}, \texttt{B}, and \texttt{C} are all singular types:
\begin{verbatim}
type A of integer;
type B of A;
type C of B;
\end{verbatim}

\CodeSubsection{\SingularBegin}{\SingularEnd}{../types.ml}

\subsubsection{Formally}
\begin{mathpar}
\inferrule{
  \makeanonymous(\tenv, \tty) \typearrow \vtone \OrTypeError\\\\
  \isbuiltinsingular(\vtone) \typearrow \vb
}{
\issingular(\tenv, \tty) \typearrow \vb
}
\end{mathpar}

\isempty{\subsubsection{Comments}}
\lrmcomment{This is related to \identr{GVZK}.}

\TypingRuleDef{AggregateType}
\hypertarget{def-isaggregate}{}
The predicate
\[
  \isaggregate(\overname{\staticenvs}{\tenv} \aslsep \overname{\ty}{\tty}) \;\aslto\;
  \overname{\Bool}{\vb} \cup \overname{\TTypeError}{\TypeErrorConfig}
\]
tests whether the type $\tty$ is an \emph{aggregate type} in the static environment $\tenv$.

\subsubsection{Prose}
A type $\tty$ is aggregate in an environment $\tenv$ if and only if all of the following apply:
\begin{itemize}
  \item obtaining the \underlyingtype\ of $\tty$ in the environment $\tenv$ yields $\vtone$\ProseOrTypeError;
  \item $\vtone$ is a builtin aggregate.
\end{itemize}

\subsubsection{Example}
In the following example, the types \texttt{A}, \texttt{B}, and \texttt{C} are all aggregate types:
\begin{verbatim}
type A of (integer, integer);
type B of A;
type C of B;
\end{verbatim}

\CodeSubsection{\AggregateBegin}{\AggregateEnd}{../types.ml}

\subsubsection{Formally}
\begin{mathpar}
\inferrule{
  \makeanonymous(\tenv, \tty) \typearrow \vtone \OrTypeError\\\\
  \isbuiltinaggregate(\vtone) \typearrow \vb
}{
  \isaggregate(\tenv, \tty) \typearrow \vb
}
\end{mathpar}

\isempty{\subsubsection{Comments}}
\lrmcomment{This is related to \identr{GVZK}.}

\TypingRuleDef{StructuredType}
\hypertarget{def-isstructured}{}
\hypertarget{def-structuredtype}{}
A \emph{\structuredtype} is any type that consists of a list of field identifiers
that denote individual storage elements. In ASL there are two such types --- record types and exception types.

The predicate
\[
  \isstructured(\overname{\ty}{\tty}) \;\aslto\; \overname{\Bool}{\vb}
\]
tests whether the type $\tty$ is a \structuredtype\ and yields the result in $\vb$.

\subsubsection{Prose}
The result $\vb$ is $\True$ if and only if $\tty$ is either a record type or an exception type,
which is determined via the AST label of $\tty$.

\subsubsection{Example}
In the following example, the types \texttt{SyntaxException} and \texttt{PointRecord}
are each an example of a \structuredtype:
\begin{verbatim}
type SyntaxException of exception {message: string };
type PointRecord of Record {x : real, y: real, z: real};
\end{verbatim}

\subsubsection{Formally}
\begin{mathpar}
\inferrule{}{
  \isstructured(\tty) \typearrow \overname{\astlabel(\tty) \in \{\TRecord, \TException\}}{\vb}
}
\end{mathpar}

\isempty{\subsubsection{Comments}}
\lrmcomment{This is related to \identd{WGQS}, \identd{QXYC}.}

\TypingRuleDef{NonPrimitiveType}
\hypertarget{def-isnonprimitive}{}
The predicate
\[
  \isnonprimitive(\overname{\ty}{\tty}) \;\aslto\; \overname{\Bool}{\vb}
\]
tests whether the type $\tty$ is a \emph{non-primitive type}.

\subsubsection{Prose}
One of the following applies:
\begin{itemize}
  \item All of the following apply (\textsc{singular}):
  \begin{itemize}
  \item $\tty$ is a builtin singular type;
  \item $\vb$ is $\False$.
  \end{itemize}
  \item All of the following apply (\textsc{named}):
  \begin{itemize}
    \item $\tty$ is a named type;
    \item $\vb$ is $\True$.
  \end{itemize}
  \item All of the following apply (\textsc{tuple}):
  \begin{itemize}
    \item $\tty$ is a tuple type $\vli$;
    \item $\vb$ is $\True$ if and only if there exists a non-primitive type in $\vli$.
  \end{itemize}
  \item All of the following apply (\textsc{array}):
    \begin{itemize}
    \item $\tty$ is an array of type $\tty'$
    \item $\vb$ is $\True$ if and only if $\tty'$ is non-primitive.
    \end{itemize}
  \item All of the following apply (\textsc{structured}):
    \begin{itemize}
    \item $\tty$ is a \structuredtype\ with fields $\fields$;
    \item $\vb$ is $\True$ if and only if there exists a non-primitive type in $\fields$.
    \end{itemize}
\end{itemize}

\subsubsection{Example}
The following types are non-primitive:

\begin{tabular}{ll}
\textbf{Type definition} & \textbf{Reason for being non-primitive}\\
\hline
\texttt{type A of integer}  & Named types are non-primitive\\
\texttt{(integer, A)}       & The second component, \texttt{A}, has non-primitive type\\
\texttt{array[6] of A}      & Element type \texttt{A} has a non-primitive type\\
\verb|record { a : A }|     & The field \texttt{a} has a non-primitive type
\end{tabular}

\CodeSubsection{\NonPrimitiveBegin}{\NonPrimitiveEnd}{../types.ml}

\subsubsection{Formally}
The cases \textsc{tuple} and \textsc{structured} below, use the notation $\vb_\vt$ to name
Boolean variables by using the types denoted by $\vt$ as a subscript.
\begin{mathpar}
\inferrule[singular]{
  \astlabel(\tty) \in \{\TReal, \TString, \TBool, \TBits, \TEnum, \TInt\}
}{
  \isnonprimitive(\tty) \typearrow \False
}
\end{mathpar}

\begin{mathpar}
\inferrule[named]{\astlabel(\tty) = \TNamed}{\isnonprimitive(\tty) \typearrow \True}
\end{mathpar}

\begin{mathpar}
\inferrule[tuple]{
  \vt \in \tys: \isnonprimitive(\vt) \typearrow \vb_{\vt}\\
  \vb \eqdef \bigvee_{\vt \in \tys} \vb_{\vt}
}{
  \isnonprimitive(\overname{\TTuple(\tys)}{\tty}) \typearrow \vb
}
\end{mathpar}

\begin{mathpar}
\inferrule[array]{
  \isnonprimitive(\tty') \typearrow \vb
}{
  \isnonprimitive(\overname{\TArray(\Ignore, \tty')}{\tty}) \typearrow \vb
}
\end{mathpar}

\begin{mathpar}
\inferrule[structured]{
  L \in \{\TRecord, \TException\}\\
  (\Ignore,\vt) \in \fields : \isnonprimitive(\vt) \typearrow \vb_\vt\\
  \vb \eqdef \bigvee_{\vt \in \vli} \vb_{\vt}
}{
  \isnonprimitive(\overname{L(\fields)}{\tty}) \typearrow \vb
}
\end{mathpar}

\isempty{\subsubsection{Comments}}
\lrmcomment{This is related to \identd{GWXK}.}

\TypingRuleDef{PrimitiveType}
\hypertarget{def-isprimitive}{}
The predicate
\[
  \isprimitive(\overname{\ty}{\tty}) \;\aslto\; \Bool
\]
tests whether the type $\tty$ is a \emph{primitive type}.

\subsubsection{Prose}
A type $\tty$ is primitive if it is not non-primitive.

\subsubsection{Example}
The following types are primitive:

\begin{tabular}{ll}
\textbf{Type definition} & \textbf{Reason for being primitive}\\
\hline
\texttt{integer} & Integers are primitive\\
\texttt{(integer, integer)} & All tuple elements are primitive\\
\texttt{array[5] of integer} & The array element type is primitive\\
\verb|record {ticks : integer}| & The single field \texttt{ticks} has a primitive type
\end{tabular}

\CodeSubsection{\PrimitiveBegin}{\PrimitiveEnd}{../types.ml}

\subsubsection{Formally}
\begin{mathpar}
\inferrule{
  \isnonprimitive(\tty) \typearrow \vb
}{
  \isprimitive(\tty) \typearrow \neg\vb
}
\end{mathpar}

\isempty{\subsubsection{Comments}}
\lrmcomment{This is related to \identd{GWXK}.}

\TypingRuleDef{Structure}
\hypertarget{def-structure}{}
The function
\[
  \tstruct(\overname{\staticenvs}{\tenv} \aslsep \overname{\ty}{\tty}) \aslto \overname{\ty}{\vt} \cup \overname{\TTypeError}{\TypeErrorConfig}
\]
assigns a type to its \hypertarget{def-tstruct}{\emph{\structure}}, which is the type formed by
recursively replacing named types by their type definition in the static environment $\tenv$.
If a named type is not associated with a declared type in $\tenv$, a type error is returned.

\TypingRuleRef{TypeCheckAST} ensures the absence of circular type definitions,
which ensures that \TypingRuleRef{Structure} terminates\footnote{In mathematical terms,
this ensures that \TypingRuleRef{Structure} is a proper \emph{structural induction.}}.

\subsubsection{Prose}
One of the following applies:
\begin{itemize}
\item All of the following apply (\textsc{named}):
  \begin{itemize}
  \item $\tty$ is a named type $\vx$;
  \item obtaining the declared type associated with $\vx$ in the static environment $\tenv$ yields $\vtone$\ProseOrTypeError;
  \item obtaining the structure of $\vtone$ static environment $\tenv$ yields $\vt$\ProseOrTypeError;
  \end{itemize}
\item All of the following apply (\textsc{builtin\_singular}):
  \begin{itemize}
  \item $\tty$ is a builtin singular type;
  \item $\vt$ is $\tty$.
  \end{itemize}
\item All of the following apply (\textsc{tuple}):
  \begin{itemize}
  \item $\tty$ is a tuple type with list of types $\tys$;
  \item the types in $\tys$ are indexed as $\vt_i$, for $i=1..k$;
  \item obtaining the structure of each type $\vt_i$, for $i=1..k$, in $\tys$ in the static environment $\tenv$,
  yields $\vtp_i$\ProseOrTypeError;
  \item $\vt$ is a tuple type with the list of types $\vtp_i$, for $i=1..k$.
  \end{itemize}
\item All of the following apply (\textsc{array}):
  \begin{itemize}
    \item $\tty$ is an array type of length $\ve$ with element type $\vt$;
    \item obtaining the structure of $\vt$ yields $\vtone$\ProseOrTypeError;
    \item $\vt$ is an array type with of length $\ve$ with element type $\vtone$.
  \end{itemize}
\item All of the following apply (\textsc{structured}):
  \begin{itemize}
  \item $\tty$ is a \structuredtype\ with fields $\fields$;
  \item obtaining the structure for each type $\vt$ associated with field $\id$ yields a type $\vt_\id$\ProseOrTypeError;
  \item $\vt$ is a record or an exception, in correspondence to $\tty$, with the list of pairs $(\id, \vt\_\id)$;
  \end{itemize}
\end{itemize}

\subsubsection{Example}
In this example:
\texttt{type T1 of integer;} is the named type \texttt{T1}
whose structure is \texttt{integer}.

In this example:
\texttt{type T2 of (integer, T1);}
is the named type \texttt{T2} whose structure is (integer, integer). In this
example, \texttt{(integer, T1)} is non-primitive since it uses \texttt{T1}, which is builtin aggregate.

In this example:
\texttt{var x: T1;}
the type of $\vx$ is the named (hence non-primitive) type \texttt{T1}, whose structure
is \texttt{integer}.

In this example:
\texttt{var y: integer;}
the type of \texttt{y} is the anonymous primitive type \texttt{integer}.

In this example:
\texttt{var z: (integer, T1);}
the type of \texttt{z} is the anonymous non-primitive type
\texttt{(integer, T1)} whose structure is \texttt{(integer, integer)}.

\CodeSubsection{\StructureBegin}{\StructureEnd}{../types.ml}

\subsubsection{Formally}
\begin{mathpar}
\inferrule[named]{
  \declaredtype(\tenv, \vx) \typearrow \vtone \OrTypeError\\\\
  \tstruct(\tenv, \vtone)\typearrow\vt \OrTypeError
}{
  \tstruct(\tenv, \TNamed(\vx)) \typearrow \vt
}
\and
\inferrule[builtin\_singular]{
  \isbuiltinsingular(\tty) \typearrow \True
}{
  \tstruct(\tenv, \tty) \typearrow \tty
}
\and
\inferrule[tuple]{
  \tys \eqname \vt_{1..k}\\
  i=1..k: \tstruct(\tenv, \vt_i) \typearrow \vtp_i \OrTypeError
}{
  \tstruct(\tenv, \TTuple(\tys)) \typearrow  \TTuple(i=1..k: \vtp_i)
}
\and
\inferrule[array]{
  \tstruct(\tenv, \vt) \typearrow \vtone \OrTypeError
}{
  \tstruct(\tenv, \TArray(\ve, \vt)) \typearrow \TArray(\ve, \vtone)
}
\and
\inferrule[structured]{
  L \in \{\TRecord, \TException\}\\\\
  (\id,\vt) \in \fields : \tstruct(\tenv, \vt) \typearrow \vt_\id \OrTypeError
}{
  \tstruct(\tenv, L(\fields)) \typearrow
 L([ (\id,\vt) \in \fields : (\id,\vt_\id) ])
}
\end{mathpar}

\isempty{\subsubsection{Comments}}
\lrmcomment{This is related to \identd{FXQV}.}

\TypingRuleDef{MakeAnonymous}
\hypertarget{def-makeanonymous}{}
\hypertarget{def-underlyingtype}{}
The function
\[
  \makeanonymous(\overname{\staticenvs}{\tenv} \aslsep \overname{\ty}{\tty}) \aslto \overname{\ty}{\vt} \cup \overname{\TTypeError}{\TypeErrorConfig}
\]
returns the \emph{\underlyingtype} --- $\vt$ --- of the type $\tty$ in the static environment $\tenv$ or a type error.
Intuitively, $\tty$ is the first non-named type that is used to define $\tty$. Unlike $\tstruct$,
$\makeanonymous$ replaces named types by their definition until the first non-named type is found but
does not recurse further.

\subsubsection{Example}
Consider the following example:
\begin{verbatim}
type T1 of integer;
type T2 of T1;
type T3 of (integer, T2);
\end{verbatim}

The underlying types of \texttt{integer}, \texttt{T1}, and \texttt{T2} is \texttt{integer}.

The underlying type of \texttt{(integer, T2)} and \texttt{T3} is
\texttt{(integer, T2)}.  Notice how the underlying type does not replace
\texttt{T2} with its own underlying type, in contrast to the structure of
\texttt{T2}, which is \texttt{(integer, integer)}.

\subsubsection{Prose}
One of the following applies:
\begin{itemize}
  \item All of the following apply (\textsc{named}):
  \begin{itemize}
    \item $\tty$ is a named type $\vx$;
    \item obtaining the type declared for $\vx$ yields $\vtone$\ProseOrTypeError;
    \item the \underlyingtype\ of $\vtone$ is $\vt$.
  \end{itemize}

  \item All of the following apply (\textsc{non-named}):
  \begin{itemize}
    \item $\tty$ is not a named type $\vx$;
    \item $\vt$ is $\tty$.
  \end{itemize}
\end{itemize}

\subsubsection{Formally}
\begin{mathpar}
\inferrule[named]{
  \tty \eqname \TNamed(\vx) \\
  \declaredtype(\tenv, \vx) \typearrow \vtone \OrTypeError \\\\
  \makeanonymous(\tenv, \vtone) \typearrow \vt
}{
  \makeanonymous(\tenv, \tty) \typearrow \vt
}
\and
\inferrule[non-named]{
  \astlabel(\tty) \neq \TNamed
}{
  \makeanonymous(\tenv, \tty) \typearrow \tty
}
\end{mathpar}
\CodeSubsection{\MakeAnonymousBegin}{\MakeAnonymousEnd}{../types.ml}

\TypingRuleDef{CheckConstrainedInteger}
\hypertarget{def-checkconstrainedinteger}{}
The function
\[
  \checkconstrainedinteger(\overname{\staticenvs}{\tenv} \aslsep \overname{\ty}{\tty}) \aslto \{\True\} \cup \overname{\TTypeError}{\TypeErrorConfig}
\]
checks whether the type $\vt$ is a \constrainedinteger. If so, the result is $\True$, otherwise a type error is returned.

\subsubsection{Prose}
One of the following applies:
\begin{itemize}
  \item All of the following apply (\textsc{well-constrained}):
  \begin{itemize}
    \item $\vt$ is a well-constrained integer;
    \item the result is $\True$.
  \end{itemize}

  \item All of the following apply (\textsc{parameterized}):
  \begin{itemize}
    \item $\vt$ is a \parameterizedintegertype;
    \item the result is $\True$.
  \end{itemize}

  \item All of the following apply (\textsc{unconstrained}):
  \begin{itemize}
    \item $\vt$ is an unconstrained integer or pending constrained integer;
    \item the result is a type error indicating that a constrained integer type is expected.
  \end{itemize}

  \item All of the following apply (\textsc{conflicting\_type}):
  \begin{itemize}
    \item $\vt$ is not an integer type;
    \item the result is a type error indicating the type conflict.
  \end{itemize}
\end{itemize}

\isempty{\subsubsection{Example}}

\CodeSubsection{\CheckConstrainedIntegerBegin}{\CheckConstrainedIntegerEnd}{../Typing.ml}

\subsubsection{Formally}
\begin{mathpar}
\inferrule[well-constrained]{}
{
  \checkconstrainedinteger(\tenv, \TInt(\wellconstrained(\Ignore))) \typearrow \True
}
\and
\inferrule[parameterized]{}
{
  \checkconstrainedinteger(\tenv, \TInt(\parameterized(\Ignore))) \typearrow \True
}
\and
\inferrule[unconstrained]{
  \astlabel(\vc) = \unconstrained \;\lor\; \astlabel(\vc) = \pendingconstrained
}
{
  \checkconstrainedinteger(\tenv, \TInt(\vc)) \typearrow \\
  \TypeErrorVal{\UnexpectedType}
}
\and
\inferrule[conflicting\_type]{
  \astlabel(\vt) \neq \TInt
}{
  \checkconstrainedinteger(\tenv, \vt) \typearrow \TypeErrorVal{\UnexpectedType}
}
\end{mathpar}

\section{Constrained Types\label{sec:ConstrainedTypes}}
\begin{itemize}
  \item A \emph{constrained type} is a type whose definition is parameterized by an expression.
        In ASL only integer types and bitvector types can be constrained.
        An integer type with a non-empty list of constrained is referred to as a
        \hypertarget{def-wellconstrainedintegertype}{\wellconstrainedintegertype}.
  \item A type which is not constrained is \emph{unconstrained}.
        Specifically, the \hypertarget{def-unconstrainedintegertype}{\unconstrainedintegertype}.
  \item A constrained type with a non-empty constraint is \emph{well-constrained}.
  \hypertarget{def-parameterizedintegertype}
  \item A \hypertarget{def-pendingconstrainedintegertype}{\emph{\pendingconstrainedintegertype}} is an integer type whose constraints will be inferred during type checking.
  \item A \emph{\parameterizedintegertype} is an implicit type of a subprogram parameter.
  \end{itemize}
The widths of bitvector storage elements are constrained integers.

\hypertarget{def-isunconstrainedinteger}{}
\hypertarget{def-isparameterizedinteger}{}
\hypertarget{def-iswellconstrainedinteger}{}
We use the following helper predicates to classify integer types:
\[
  \begin{array}{rcl}
  \isunconstrainedinteger(\overname{\ty}{\vt}) &\aslto& \Bool\\
  \isparameterizedinteger(\overname{\ty}{\vt}) &\aslto& \Bool\\
  \iswellconstrainedinteger(\overname{\ty}{\vt}) &\aslto& \Bool
  \end{array}
\]
Those are defined as follows:
\[
  \begin{array}{rcl}
  \isunconstrainedinteger(\vt) &\triangleq& \vt = \TInt(c) \land \astlabel(c)=\unconstrained\\
  \isparameterizedinteger(\vt) &\triangleq& \vt = \TInt(c) \land \astlabel(c)=\parameterized\\
  \iswellconstrainedinteger(\vt) &\triangleq& \vt = \TInt(c) \land \astlabel(c)=\wellconstrained\\
\end{array}
\]\lrmcomment{This is related to \identd{ZTPP}, \identr{WJYH}, \identr{HJPN}, \identr{CZTX}, \identr{TPHR}.}

\paragraph{Shorthand Notations:}

\hypertarget{def-unconstrainedinteger}{}
We use the shorthand notation $\unconstrainedinteger$ to denote the unconstrained integer type: $\TInt(\unconstrained)$.

\section{Relations Over Types\label{sec:RelationsOnTypes}}
This section defines the following relations over types and operators:
\begin{itemize}
  \item Subtype (\TypingRuleRef{Subtype})
  \item Subtype Satisfaction (\TypingRuleRef{SubtypeSatisfaction})
  \item Type Satisfaction (\TypingRuleRef{TypeSatisfaction})
  \item Type Clash (\TypingRuleRef{TypeClash})
  \item The Lowest Common Ancestor of two types (\TypingRuleRef{LowestCommonAncestor})
  \item Applying a unary operator to a type (\TypingRuleRef{ApplyUnopType})
  \item Applying a binary operator to a pair of types (\TypingRuleRef{ApplyBinopTypes})
\end{itemize}

\TypingRuleDef{Subtype}
\hypertarget{def-supertypeterm}{}
The \emph{subtype} relation is a partial order over \underline{named types}.
The \emph{\supertypeterm} is the inverse relation. That is, \tty\ is a supertype of \tsy\ if and only if \tsy\ is a subtype of \tty.

\hypertarget{def-subtypesrel}{}
The predicate
\[
  \subtypesrel(\overname{\staticenvs}{\tenv} \aslsep \overname{\ty}{\vtone} \aslsep \overname{\ty}{\vttwo})
  \aslto \overname{\Bool}{\vb}
\]
defines whether the type $\vtone$ subtypes the type $\vttwo$ in the static environment $\tenv$,
yielding the result in $\vb$.

\ProseParagraph
One of the following applies:
\begin{itemize}
  \item all of the following apply (\textsc{reflexive}):
  \begin{itemize}
    \item $\vtone$ and $\vttwo$ are both the same named type;
    \item $\vb$ is $\True$.
  \end{itemize}

  \item all of the following apply (\textsc{transitive}):
  \begin{itemize}
    \item $\vtone$ is a named type with name $\idone$, that is $\TNamed(\idone)$;
    \item $\vttwo$ is a named type with name $\idtwo$, that is $\TNamed(\idtwo)$, such that $\idone$ is different from $\idtwo$;
    \item the global static environment maintains that $\idone$ is a subtype of $\idthree$;
    \item testing whether the type named $\idthree$ is a subtype of $\vttwo$ in the static environment $\tenv$
    gives $\vb$.
  \end{itemize}

  \item all of the following apply (\textsc{no\_supertype}):
  \begin{itemize}
    \item $\vtone$ is a named type with name $\idone$, that is $\TNamed(\idone)$;
    \item $\vttwo$ is a named type with name $\idtwo$, that is $\TNamed(\idtwo)$, such that $\idone$ is different from $\idtwo$;
    \item the global static environment maintains that $\idone$ does subtype any named type;
    \item $\vb$ is $\False$.
  \end{itemize}

  \item all of the following apply (\textsc{not\_named}):
  \begin{itemize}
    \item at least one of $\vtone$ and $\vttwo$ is not a named type;
    \item $\vb$ is $\False$.
  \end{itemize}
\end{itemize}
\subsubsection{Example}
In the following example \texttt{subInt} is a subtype of itself and of \texttt{superInt}:
\begin{lstlisting}
type superInt of integer;
type subInt of integer subtypes superInt;
\end{lstlisting}

\CodeSubsection{\SubtypeBegin}{\SubtypeEnd}{../types.ml}

\FormallyParagraph
\begin{mathpar}
  \inferrule[reflexive]{}{
    \subtypesrel(\tenv, \TNamed(\id), \TNamed(\id)) \typearrow \True
  }
  \and
  \inferrule[transitive]{
    \idone \neq \idtwo\\
    G^\tenv.\subtypes(\idone) = \idthree\\
    \subtypesrel(\tenv, \TNamed(\idthree), \vttwo) \typearrow \vb
  }{
    \subtypesrel(\tenv, \TNamed(\idone), \TNamed(\idtwo)) \typearrow \vb
  }
  \and
  \inferrule[no\_supertype]{
    \idone \neq \idtwo\\
    G^\tenv.\subtypes(\idone) = \bot
  }{
    \subtypesrel(\tenv, \TNamed(\idone), \TNamed(\idtwo)) \typearrow \False
  }
  \and
  \inferrule[not\_named]{
    (\astlabel(\vtone) \neq \TNamed \lor \astlabel(\vttwo) \neq \TNamed)
  }{
    \subtypesrel(\tenv, \vtone, \vttwo) \typearrow \False
  }
\end{mathpar}

\lrmcomment{This is related to \identr{NXRX}, \identi{KGKS}, \identi{MTML}, \identi{JVRM}, \identi{CHMP}.}

\TypingRuleDef{SubtypeSatisfaction}
\hypertarget{def-subtypesat}{}
The predicate
\[
  \subtypesat(\overname{\staticenvs}{\tenv} \aslsep \overname{\ty}{\vt} \aslsep \overname{\ty}{\vs})
  \aslto \overname{\Bool}{\vb} \cup \overname{\TTypeError}{\TypeErrorConfig}
\]
tests whether a type $\vt$ \emph{subtype-satisfies} a type $\vs$ in environment $\tenv$,
returning the result $\vb$ or a type error, if one is detected.
The function assumes that both $\vt$ and $\vs$ are well-typed according to \chapref{Types}.

\ProseParagraph
One of the following applies:
\begin{itemize}
\item All of the following apply (\textsc{error1}):
  \begin{itemize}
  \item obtaining the \underlyingtype\ of $\vt$ gives a type error;
  \item the rule results in a type error.
  \end{itemize}

\item All of the following apply (\textsc{error2}):
  \begin{itemize}
    \item obtaining the \underlyingtype\ of $\vt$ gives a type $\vttwo$;
    \item obtaining the \underlyingtype\ of $\vs$ gives a type error;
    \item the rule results in a type error.
    \end{itemize}

\item All of the following apply (\textsc{different\_labels}):
  \begin{itemize}
  \item the underlying types of $\vt$ and $\vs$ have different AST labels
  (for example, \texttt{integer} and \texttt{real});
  \item $\vb$ is $\False$.
  \end{itemize}

\item All of the following apply (\textsc{simple}):
  \begin{itemize}
  \item the \underlyingtype\ of $\vt$, $\vttwo$, is either \texttt{real}, \texttt{string}, or \texttt{bool};
  \item the \underlyingtype\ of $\vs$, $\vstwo$, is either \texttt{real}, \texttt{string}, or \texttt{bool};
  \item $\vb$ is $\True$ if and only if both $\vttwo$ and $\vstwo$ have the same ASL label.
  \end{itemize}

\item All of the following apply (\textsc{t\_int}):
  \begin{itemize}
  \item the \underlyingtype\ of $\vt$, $\vttwo$, is an \texttt{integer} (any kind);
  \item the \underlyingtype\ of $\vs$, $\vstwo$, is an \texttt{integer} (any kind);
  \item determining whether $\vs$ subsumes $\vt$ in $\tenv$ via symbolic reasoning results in $\vb$.
  \end{itemize}

\item All of the following apply (\textsc{t\_enum}):
  \begin{itemize}
  \item the \underlyingtype\ of $\vt$ is an enumeration type with list of labels $\vlit$, that is, $\TEnum(\vlit)$;
  \item the \underlyingtype\ of $\vs$ is an enumeration type with list of labels $\vlis$, that is, $\TEnum(\vlis)$;
  \item $\vb$ is $\True$ if and only if $\vlit$ is equal to $\vlis$.
  \end{itemize}

\item All of the following apply (\textsc{t\_bits}):
  \begin{itemize}
  \item the \underlyingtype\ of $\vs$ is a bitvector type with width $\ws$ and bit fields $\bfss$, that is $\TBits(\ws, \bfss)$;
  \item the \underlyingtype\ of $\vt$ is a bitvector type with width $\wt$ and bit fields $\bfst$, that is $\TBits(\wt, \bfst)$;
  \item determining whether the bit fields $\bfss$ are included in the bit fields $\bfst$ in $\tenv$ yields $\True$\ProseOrTypeError;
  \item determining whether the \symbolicdomain{} of bitwidth $\ws$ subsumes the \symbolicdomain{} of bitwidth $\wt$ in $\tenv$ yields $\vb$.
  \end{itemize}

\item All of the following apply (\textsc{t\_array\_expr}):
  \begin{itemize}
  \item $\vs$ has the \underlyingtype\ of an array with index $\vlengths$ and element type $\vtys$, that is $\TArray(\vlengths, \vtys)$;
  \item $\vt$ has the \underlyingtype\ of an array with index $\vlengtht$ and element type $\vtyt$, that is $\TArray(\vlengtht, \vtyt)$;
  \item determining whether $\vtys$ and $\vtyt$ are equivalent in $\tenv$ is either $\True$
  or $\False$, which short-circuits the entire rule with $\vb=\False$;
  \item either the AST labels of $\vlengths$ and $\vlengtht$ are the same or the rule short-circuits with $\vb=\False$;
  \item $\vlengths$ is an array length expression with $\vlengthexprs$, that is \\ $\ArrayLengthExpr(\vlengthexprs)$;
  \item $\vlengtht$ is an array length expression with $\vlengthexprt$, that is \\ $\ArrayLengthExpr(\vlengthexprt)$;
  \item determining whether expressions $\vlengthexprs$ and $\vlengthexprt$ are equivalent gives $\vb$.
  \end{itemize}

  \item All of the following apply (\textsc{t\_array\_enum}):
  \begin{itemize}
  \item $\vs$ has the \underlyingtype\ of an array with index $\vlengths$ and element type $\vtys$, that is $\TArray(\vlengths, \vtys)$;
  \item $\vt$ has the \underlyingtype\ of an array with index $\vlengtht$ and element type $\vtyt$, that is $\TArray(\vlengtht, \vtyt)$;
  \item determining whether $\vtys$ and $\vtyt$ are equivalent in $\tenv$ is either $\True$
  or $\False$, which short-circuits the entire rule with $\vb=\False$;
  \item either the AST labels of $\vlengths$ and $\vlengtht$ are the same or the rule short-circuits with $\vb=\False$;
  \item $\vlengths$ is an array with indices taken from the enumeration $\vnames$, that is $\ArrayLengthEnum(\vnames, \Ignore)$;
  \item $\vlengtht$ is an array with indices taken from the enumeration $\vnamet$, that is $\ArrayLengthEnum(\vnamet, \Ignore)$;
  \item $\vb$ is $\True$ if and only if $\vnames$ and $\vnamet$ are the same.
  \end{itemize}

\item All of the following apply (\textsc{t\_tuple}):
  \begin{itemize}
  \item $\vs$ has the \underlyingtype\ of a tuple with type list $\vlis$, that is $\TTuple(\vlis)$;
  \item $\vt$ has the \underlyingtype\ of a tuple with type list $\vlit$, that is $\TTuple(\vlit)$;
  \item equating the lengths of $\vlis$ and $\vlit$ is either $\True$ or $\False$, which short-circuits
  the entire rule with $\vb=\False$;
  \item checking at each index $\vi$ of the list $\vlis$ whether the type $\vlit[\vi]$ \typesatisfies\ the type $\vlis[\vi]$
  yields $\vb_\vi$\ProseOrTypeError;
  \item $\vb$ is $\True$ if and only if all $\vb_\vi$ are $\True$;
  \end{itemize}

\item All of the following apply (\textsc{structured}):
  \begin{itemize}
  \item $\vs$ has the \underlyingtype\ $L(\vfieldss)$, which is a \structuredtype;
  \item $\vt$ has the \underlyingtype\ $L(\vfieldst)$, which is a \structuredtype;
  \item since both underlying types have the same AST label they are either both record types or both exception types;
  \item $\vb$ is $\True$ if and only if for each field in $\vfieldss$ with type $\vtys$
  there exists a field in $\vfieldst$ with type $\vtyt$ such that both $\vtys$ and $\vtyt$
  are determined to be \typeequivalent\ in $\tenv$.
  \end{itemize}
\end{itemize}

\FormallyParagraph
\begin{mathpar}
  \inferrule[error1]{
    \makeanonymous(\tenv, \vt) \typearrow \TypeErrorConfig
  }
  {
    \subtypesat(\tenv, \vt, \vs) \typearrow \TypeErrorConfig
  }
  \and
  \inferrule[error2]{
    \makeanonymous(\tenv, \vt) \typearrow \vttwo\\
    \makeanonymous(\tenv, \vs) \typearrow \TypeErrorConfig
  }
  {
    \subtypesat(\tenv, \vt, \vs) \typearrow \TypeErrorConfig
  }
  \and
  \inferrule[different\_labels]{
    \makeanonymous(\tenv, \vt) \typearrow \vttwo\\
    \makeanonymous(\tenv, \vs) \typearrow \vstwo\\
    \astlabel(\vttwo) \neq \astlabel(\vstwo)
  }
  {
    \subtypesat(\tenv, \vt, \vs) \typearrow \False
  }
\end{mathpar}

\begin{mathpar}
  \inferrule[simple]{
    \makeanonymous(\tenv, \vt) \typearrow \vttwo\\
    \makeanonymous(\tenv, \vs) \typearrow \vstwo\\
    \astlabel(\vttwo) \in \{\TReal, \TString, \TBool\}\\
    \vb \eqdef \astlabel(\vstwo) = \astlabel(\vttwo)
  }{
    \subtypesat(\tenv, \vt, \vs) \typearrow \vb
  }
\end{mathpar}

\begin{mathpar}
\inferrule[t\_int]{
  \makeanonymous(\tenv, \vt) \typearrow \vttwo\\
  \makeanonymous(\tenv, \vs) \typearrow \vstwo\\
  \astlabel(\vttwo) = \astlabel(\vstwo) = \TInt\\
  \symsubsumes(\tenv, \vs, \vt) \typearrow \vb
}{
  \subtypesat(\tenv, \vt, \vs) \typearrow \vb
}
\end{mathpar}

\begin{mathpar}
\inferrule[t\_enum]{
  \makeanonymous(\tenv, \vt) \typearrow \TEnum(\vlit)\\
  \makeanonymous(\tenv, \vs) \typearrow \TEnum(\vlis)\\
  \vb \eqdef \vlit = \vlis
}{
  \subtypesat(\tenv, \vt, \vs) \typearrow \vb
}
\end{mathpar}

\begin{mathpar}
\inferrule[t\_bits]{
  \makeanonymous(\tenv, \vs) \typearrow \TBits(\ws, \bfss)\\
  \makeanonymous(\tenv, \vt) \typearrow \TBits(\wt, \bfst)\\
  \bitfieldsincluded(\tenv, \bfss, \bfst) \typearrow \True \OrTypeError \\
  \symdomofwidth(\tenv, \ws) \typearrow \ds \\
  \symdomofwidth(\tenv, \wt) \typearrow \dt \\
  \symdomissubset(\tenv, \ds, \dt) \typearrow \vb
}{
  \subtypesat(\tenv, \vt, \vs) \typearrow \vb
}
\end{mathpar}

\begin{mathpar}
\inferrule[t\_array\_expr]{
  \makeanonymous(\tenv, \vs) \typearrow \TArray(\vlengths,\vtys) \\
  \makeanonymous(\tenv, \vt) \typearrow \TArray(\vlengtht,\vtyt) \\
  \typeequal(\tenv, \vtys, \vtyt) \typearrow \True \terminateas \False\\
  \booltrans{\astlabel(\vlengths) = \astlabel(\vlengtht)} \booltransarrow \True \terminateas \False\\
  \vlengths \eqname \ArrayLengthExpr(\vlengthexprs)\\
  \vlengtht \eqname \ArrayLengthExpr(\vlengthexprt)\\
  \exprequal(\tenv, \vlengthexprs, \vlengthexprt) \typearrow \vb
}{
  \subtypesat(\tenv, \vt, \vs) \typearrow \vb
}
\and
\inferrule[t\_array\_enum]{
  \makeanonymous(\tenv, \vs) \typearrow \TArray(\vlengths,\vtys) \\
  \makeanonymous(\tenv, \vt) \typearrow \TArray(\vlengtht,\vtyt) \\
  \typeequal(\tenv, \vtys, \vtyt) \typearrow \True\\
  \booltrans{\astlabel(\vlengths) = \astlabel(\vlengtht)} \typearrow \True \terminateas \False \\
  \vlengths \eqname \ArrayLengthEnum(\vnames, \Ignore)\\
  \vlengtht \eqname \ArrayLengthEnum(\vnamet, \Ignore)\\
  \vb \eqdef \vnames = \vnamet
}{
  \subtypesat(\tenv, \vt, \vs) \typearrow \vb
}
\end{mathpar}

\begin{mathpar}
\inferrule[t\_tuple]
{ \makeanonymous(\tenv, \vs) \typearrow\TTuple(\vlis)\\
  \makeanonymous(\tenv, \vt) \typearrow\TTuple(\vlit)\\
  \equallength(\vlis, \vlit) \typearrow\True \terminateas \False\\
  \vi\in\listrange(\vlis): \typesat(\tenv, \vlit[\vi], \vlis[\vi]) \typearrow \vb_i \terminateas \TTypeError\\
  \vb \eqdef \bigwedge_{\vi=1}^k \vb_\vi
}{
  \subtypesat(\tenv, \vt, \vs) \typearrow \vb
}
\end{mathpar}

\hypertarget{def-fieldnames}{}
For a list of typed fields $\fields$, we define the set of its field identifiers as:
\[
  \fieldnames(\fields) \triangleq \{ \id \;|\; (\id, \vt) \in \fields\}
\]
\hypertarget{def-fieldtype}{}
We define the type associated with the field name $\id$ in a list of typed fields $\fields$,
if there is a unique one, as follows:
\[
  \fieldtype(\fields, \id) \triangleq
  \begin{cases}
  \vt  & \text{ if }\{ \vtp \;|\; (\id,\vtp) \in \fields\} = \{\vt\}\\
  \bot & \text{ otherwise}
  \end{cases}
\]

\begin{mathpar}
\inferrule[structured]{
  L \in \{\TRecord, \TException\}\\
  \makeanonymous(\tenv, \vs)\typearrow L(\vfieldss) \\
  \makeanonymous(\tenv, \vt)\typearrow L(\vfieldst) \\
  \vnamess \eqdef \fieldnames(\vfieldss)\\
  \vnamest \eqdef \fieldnames(\vfieldst)\\
  \booltrans{\vnamess \subseteq \vnamest} \booltransarrow \True \terminateas \False\\
  (\id,\vtys)\in\vfieldss: \typeequal(\tenv, \vtys, \fieldtype(\vfieldst, \id)) \typearrow \vb_\id\\
  \vb \eqdef \bigwedge_{\id \in \vnamess} \vb_\id
}{
  \subtypesat(\tenv, \vs, \vt) \typearrow \vb
}
\end{mathpar}

\isempty{\subsection{Comments}}
\lrmcomment{This is related to \identd{TRVR}, \identi{SJDC}, \identi{MHYB}, \identi{TWTZ}, \identi{GYSK}, \identi{KXSD}, \identi{KNXJ}.}

\isempty{\subsection{Example}}

\CodeSubsection{\SubtypeSatisfactionBegin}{\SubtypeSatisfactionEnd}{../types.ml}

\TypingRuleDef{TypeSatisfaction}
\hypertarget{def-typesatisfies}{}
The predicate
\[
  \typesat(\overname{\staticenvs}{\tenv} \aslsep \overname{\ty}{\vt} \aslsep \overname{\ty}{\vs})
  \aslto \overname{\Bool}{\vb} \cup \overname{\TTypeError}{\TypeErrorConfig}
\]
tests whether a type $\vt$ \emph{\typesatisfies} a type $\vs$ in environment $\tenv$,
returning the result $\vb$ or a type error, if one is detected.

\hypertarget{def-checktypesat}{}
We also define
\[
  \checktypesat(\overname{\staticenvs}{\tenv} \aslsep \overname{\ty}{\vt} \aslsep \overname{\ty}{\vs})
  \aslto \{\True\} \cup \overname{\TTypeError}{\TypeErrorConfig}
\]
which is the same as $\typesat$, but yields a type error when \\ $\typesat(\tenv, \vt, \vs)$ is $\False$.

These functions assume that both $\vt$ and $\vs$ are well-typed according to \secref{Types}.

\ProseParagraph
One of the following applies:
 \begin{itemize}
  \item All of the following apply (\textsc{subtypes}):
    \begin{itemize}
    \item $\vt$ subtypes $\vs$ in $\tenv$ ;
    \item $\vb$ is $\True$.
  \end{itemize}

  \item All of the following apply (\textsc{anonymous}):
  \begin{itemize}
    \item $\vt$ does not subtype $\vs$ in $\tenv$;
    \item at least one of $\vt$ and $\vs$ is an anonymous type in $\tenv$;
    \item determining whether $\vt$ \subtypesatisfies\ $\vs$ in $\tenv$ yields $\True$\ProseOrTypeError;
    \item $\vb$ is $\True$.
  \end{itemize}

  \item All of the following apply (\textsc{t\_bits}):
  \begin{itemize}
    \item $\vt$ does not subtype $\vs$ in $\tenv$;
    \item determining whether $\vt$ is anonymous yields $\vbone$;
    \item determining whether $\vs$ is anonymous yields $\vbtwo$;
    \item determining whether $\vt$ \subtypesatisfies\ $\vs$ in $\tenv$ yields $\vbthree$;
    \item $(\vbone \lor \vbtwo) \land \vbthree$ is $\False$;
    \item $\vt$ is a bitvector type with width $\widtht$ and no bitfields;
    \item obtaining the \structure\ of $\vs$ in $\tenv$ yields a bitvector type with width \\
          $\widths$\ProseOrTypeError;
    \item determining whether $\widtht$ and $\widths$ are \bitwidthequivalent\ yields $\vb$.
  \end{itemize}

  \item All of the following apply (\textsc{otherwise1}):
  \begin{itemize}
    \item $\vt$ does not subtype $\vs$ in $\tenv$;
    \item determining whether $\vt$ is anonymous yields $\vbone$;
    \item determining whether $\vs$ is anonymous yields $\vbtwo$;
    \item determining whether $\vt$ \subtypesatisfies\ $\vs$ in $\tenv$ yields $\vbthree$;
    \item $(\vbone \lor \vbtwo) \land \vbthree$ is $\False$;
    \item obtaining the \structure\ of $\vs$ in $\tenv$ yields a $\vsstruct$\ProseOrTypeError;
    \item at least one of $\vt$ and $\vsstruct$ is not a bitvector type;
  \end{itemize}

  \item All of the following apply (\textsc{otherwise2}):
  \begin{itemize}
    \item $\vt$ does not subtype $\vs$ in $\tenv$;
    \item determining whether $\vt$ is anonymous yields $\vbone$;
    \item determining whether $\vs$ is anonymous yields $\vbtwo$;
    \item determining whether $\vt$ \subtypesatisfies\ $\vs$ in $\tenv$ yields $\vbthree$;
    \item $(\vbone \lor \vbtwo) \land \vbthree$ is $\False$;
    \item obtaining the \structure\ of $\vs$ in $\tenv$ yields a $\vsstruct$\ProseOrTypeError;
    \item both $\vt$ and $\vsstruct$ are bitvector types;
    \item the bitvector type $\vt$ has a non-empty list of bitfields;
    \item $\vb$ is $\False$;
  \end{itemize}
\end{itemize}

\subsubsection{Example}
In \listingref{typing-typesat1},
\texttt{var pair: pairT = (1, dataT1)} is legal since the right-hand-side has
anonymous, non-primitive type \texttt{(integer, T1)}.
\ASLListing{Type satisfaction example}{typing-typesat1}{\typingtests/TypingRule.TypeSatisfaction1.asl}

\subsubsection{Example}
In \listingref{typing-typesat2},
\texttt{pair = (1, dataAsInt);} is legal since the right-hand-side has anonymous,
primitive type \texttt{(integer, integer)}.
\ASLListing{Type satisfaction example}{typing-typesat2}{\typingtests/TypingRule.TypeSatisfaction2.asl}

\subsubsection{Example}
In \listingref{typing-typesat3},
\texttt{pair = (1, dataT2);} is illegal since the right-hand-side has anonymous,
non-primitive type \texttt{(integer, T2)} which does not subtype-satisfy named
type \texttt{pairT}.
\ASLListing{Type satisfaction example}{typing-typesat3}{\typingtests/TypingRule.TypeSatisfaction3.asl}

\CodeSubsection{\TypeSatisfactionBegin}{\TypeSatisfactionEnd}{../types.ml}

\FormallyParagraph
\begin{mathpar}
\inferrule[subtypes]{
  \subtypesrel(\tenv, \vt, \vs) \typearrow \True
}{
  \typesat(\tenv, \vt, \vs) \typearrow \True
}
\end{mathpar}

\begin{mathpar}
\inferrule[anonymous]{
  \subtypesrel(\tenv, \vt, \vs) \typearrow \False\\
  \isanonymous(\tenv, \vt) \typearrow \vbone\\
  \isanonymous(\tenv, \vs) \typearrow \vbtwo\\
  \vbone \lor \vbtwo\\
  \subtypesat(\tenv, \vt, \vs) \typearrow \True
}{
  \typesat(\tenv, \vt, \vs) \typearrow \True
}
\end{mathpar}

\begin{mathpar}
\inferrule[t\_bits]{
  \subtypesrel(\tenv, \vt, \vs) \typearrow \False\\
  \isanonymous(\tenv, \vt) \typearrow \vbone\\
  \isanonymous(\tenv, \vs) \typearrow \vbtwo\\
  \subtypesat(\tenv, \vt, \vs) \typearrow \vbthree\\
  \neg((\vbone \lor \vbtwo) \land \vbthree)\\
  \vt \eqname \TBits(\widtht, \emptylist)\\
  \tstruct(\tenv, \vs) \typearrow \TBits(\widths, \Ignore) \OrTypeError\\\\
  \bitwidthequal(\tenv, \widtht, \widths) \typearrow \vb
}{
  \typesat(\tenv, \vt, \vs) \typearrow \vb
}
\end{mathpar}

\begin{mathpar}
\inferrule[otherwise1]{
  \subtypesrel(\tenv, \vt, \vs) \typearrow \False\\
  \isanonymous(\tenv, \vt) \typearrow \vbone\\
  \isanonymous(\tenv, \vs) \typearrow \vbtwo\\
  \subtypesat(\tenv, \vt, \vs) \typearrow \vbthree\\
  \neg((\vbone \lor \vbtwo) \land \vbthree)\\
  \tstruct(\tenv, \vs) \typearrow \vsstruct\\
  \astlabel(\vt) \neq \TBits \lor \astlabel(\vsstruct) \neq \TBits
}{
  \typesat(\tenv, \vt, \vs) \typearrow \overname{\False}{\vb}
}
\end{mathpar}

\begin{mathpar}
\inferrule[otherwise2]{
  \subtypesrel(\tenv, \vt, \vs) \typearrow \False\\
  \isanonymous(\tenv, \vt) \typearrow \vbone\\
  \isanonymous(\tenv, \vs) \typearrow \vbtwo\\
  \subtypesat(\tenv, \vt, \vs) \typearrow \vbthree\\
  \neg((\vbone \lor \vbtwo) \land \vbthree)\\
  \tstruct(\tenv, \vs) \typearrow \vsstruct\\
  \astlabel(\vt) = \TBits \land \astlabel(\vsstruct) = \TBits\\
  \vt \eqname \TBits(\widtht, \bitfields)\\
  \bitfields \neq \emptylist
}{
  \typesat(\tenv, \vt, \vs) \typearrow \overname{\False}{\vb}
}
\end{mathpar}

The rules for the checked type-satisfy predicate are:
\begin{mathpar}
\inferrule[true]{
  \typesat(\tenv, \vt, \vs) \typearrow \True \OrTypeError\\
}{
  \checktypesat(\tenv, \vt, \vs) \typearrow \True
}
\end{mathpar}

\begin{mathpar}
\inferrule[error]{
  \typesat(\tenv, \vt, \vs) \typearrow \False
}{
  \checktypesat(\tenv, \vt, \vs) \typearrow \TypeErrorVal{\TypeSatisfactionFailure}
}
\end{mathpar}

\subsubsection{Comments}
Since the subtype relation is a partial order, it is reflexive. Therefore
every type $\vt$ is a subtype of itself, and as a consequence, every type $\vt$
\typesatisfies\  itself.

\lrmcomment{This is related to \identr{FMXK} and \identi{NLFD}.}

\TypingRuleDef{TypeClash}
\hypertarget{def-typeclashes}{}
The predicate
\[
  \typeclashes(\overname{\staticenvs}{\tenv} \aslsep \overname{\ty}{\vt} \aslsep \overname{\ty}{\vs})
  \aslto \overname{\Bool}{\vb} \cup \overname{\TTypeError}{\TypeErrorConfig}
\]
tests whether a type $\vt$ \emph{type-clashes} with a type $\vs$ in environment $\tenv$,
returning the result $\vb$ or a type error, if one is detected.

\ProseParagraph
 One of the following applies:
\begin{itemize}
  \item All of the following apply (\textsc{subtype}):
  \begin{itemize}
    \item either $\vs$ subtypes $\vt$ or $\vt$ subtypes $\vs$;
    \item $\vb$ is $\True$.
  \end{itemize}

  \item All of the following apply (\textsc{simple}):
  \begin{itemize}
    \item neither $\vs$ subtypes $\vt$ nor $\vt$ subtypes $\vs$;
    \item obtaining the \structure\ of $\vt$ in $\tenv$ yields $\vtstruct$\ProseOrTypeError;
    \item obtaining the \structure\ of $\vs$ in $\tenv$ yields $\vsstruct$\ProseOrTypeError;
    \item both $\vtstruct$ and $\vsstruct$ are one of the following types: \\ \texttt{integer}, \texttt{real}, \texttt{string};
    \item $\vb$ is $\True$.
  \end{itemize}

  \item All of the following apply (\textsc{t\_enum}):
  \begin{itemize}
    \item neither $\vs$ subtypes $\vt$ nor $\vt$ subtypes $\vs$;
    \item obtaining the \structure\ of $\vt$ in $\tenv$ yields an enumeration type with labels $\vlit$;
    \item obtaining the \structure\ of $\vs$ in $\tenv$ yields an enumeration type with labels $\vlis$;
    \item $\vb$ is $\True$ if and only if $\vlis$ and $\vlit$ are equal.
  \end{itemize}

  \item All of the following apply (\textsc{t\_array}):
  \begin{itemize}
    \item neither $\vs$ subtypes $\vt$ nor $\vt$ subtypes $\vs$;
    \item obtaining the \structure\ of $\vt$ in $\tenv$ yields an array type with element type $\vtyt$;
    \item obtaining the \structure\ of $\vs$ in $\tenv$ yields an array type with element type $\vtys$;
    \item $\vb$ is $\True$ if and only if $\vtyt$ and $\vtys$ type-clash.
  \end{itemize}

  \item All of the following apply (\textsc{t\_tuple}):
  \begin{itemize}
    \item neither $\vs$ subtypes $\vt$ nor $\vt$ subtypes $\vs$;
    \item obtaining the \structure\ of $\vt$ in $\tenv$ yields a tuple type with element types $\vt_{1..k}$;
    \item obtaining the \structure\ of $\vs$ in $\tenv$ yields a tuple type with element types $\vs_{1..n}$;
    \item if $n \neq k$ the rule short-circuits with $\vb=\False$;
    \item $\vb$ is $\True$ if and only if $\vt_i$ type-clashes with $\vs_i$, for all $i=1..k$.
  \end{itemize}

  \item All of the following apply (\textsc{otherwise\_different\_labels}):
  \begin{itemize}
    \item neither $\vs$ subtypes $\vt$ nor $\vt$ subtypes $\vs$;
    \item obtaining the \structure\ of $\vt$ in $\tenv$ yields $\vtstruct$;
    \item obtaining the \structure\ of $\vs$ in $\tenv$ yields $\vsstruct$;
    \item $\vsstruct$ and $\vtstruct$ have different AST labels;
    \item $\vb$ is $\False$;
  \end{itemize}

  \item All of the following apply (\textsc{otherwise\_structured}):
  \begin{itemize}
    \item neither $\vs$ subtypes $\vt$ nor $\vt$ subtypes $\vs$;
    \item obtaining the \structure\ of $\vt$ in $\tenv$ yields $\vtstruct$;
    \item obtaining the \structure\ of $\vs$ in $\tenv$ yields $\vsstruct$;
    \item $\vsstruct$ and $\vtstruct$ have the same AST label;
    \item $\vtstruct$ (and thus $\vsstruct$) is a \structuredtype;
    \item $\vb$ is $\False$;
  \end{itemize}
\end{itemize}



\CodeSubsection{\TypeClashBegin}{\TypeClashEnd}{../types.ml}

\FormallyParagraph
\begin{mathpar}
\inferrule[subtype]{
  (\subtypesrel(\tenv, \vs, \vt) \typearrow \True) \lor (\subtypesrel(\tenv, \vt, \vs) \typearrow \True)
}{
  \typeclashes(\tenv, \vt, \vs) \typearrow \overname{\True}{\vb}
}
\end{mathpar}

\begin{mathpar}
\inferrule[simple]{
  \subtypesrel(\tenv, \vs, \vt) \typearrow \False\\
  \subtypesrel(\tenv, \vt, \vs) \typearrow \False\\\\
  \tstruct(\tenv, \vt) \typearrow \vtstruct \OrTypeError \\
  \tstruct(\tenv, \vs) \typearrow \vsstruct \OrTypeError \\
  \astlabel(\vtstruct)=\astlabel(\vsstruct)\\
  \astlabel(\vtstruct) \in \{\TInt, \TReal, \TString, \TBits\}
}{
  \typeclashes(\tenv, \vt, \vs) \typearrow \overname{\True}{\vb}
}
\end{mathpar}

\begin{mathpar}
\inferrule[t\_enum]{
  \subtypesrel(\tenv, \vs, \vt) \typearrow \False\\
  \subtypesrel(\tenv, \vt, \vs) \typearrow \False\\\\
  \tstruct(\tenv, \vt) \typearrow \TEnum(\Ignore, \vlis) \\
  \tstruct(\tenv, \vs) \typearrow \TEnum(\Ignore, \vlit)
}{
  \typeclashes(\tenv, \vt, \vs) \typearrow \overname{\vlis = \vlit}{\vb}
}
\end{mathpar}

\begin{mathpar}
\inferrule[t\_array]{
  \subtypesrel(\tenv, \vs, \vt) \typearrow \False\\
  \subtypesrel(\tenv, \vt, \vs) \typearrow \False\\\\
  \tstruct(\tenv, \vt) \typearrow \TArray(\Ignore, \vtyt) \\
  \tstruct(\tenv, \vs) \typearrow \TArray(\Ignore, \vtys) \\
  \typeclashes(\tenv, \vtyt, \vtys) \typearrow \vb
}{
  \typeclashes(\tenv, \vt, \vs) \typearrow \vb
}
\end{mathpar}

\begin{mathpar}
\inferrule[t\_tuple]{
  \subtypesrel(\tenv, \vs, \vt) \typearrow \False\\
  \subtypesrel(\tenv, \vt, \vs) \typearrow \False\\\\
  \tstruct(\tenv, \vt) \typearrow \TTuple(\vt_{1..k}) \\
  \tstruct(\tenv, \vs) \typearrow \TTuple(\vs_{1..n}) \\
  \booltrans{n = k} \booltransarrow \True \terminateas \False\\
  i=1..k: \typeclashes(\tenv, \vt_i, \vs_i) \typearrow \vb_i\\
  \vb \eqdef \bigwedge_{\vi=1}^k \vb_i
}{
  \typeclashes(\tenv, \vt, \vs) \typearrow \vb
}
\end{mathpar}

\begin{mathpar}
\inferrule[otherwise\_different\_labels]{
  \subtypesrel(\tenv, \vs, \vt) \typearrow \False\\
  \subtypesrel(\tenv, \vt, \vs) \typearrow \False\\\\
  \tstruct(\tenv, \vt) \typearrow \vtstruct \\
  \tstruct(\tenv, \vs) \typearrow \vsstruct \\
  \astlabel(\vtstruct) \neq \astlabel(\vsstruct)
}{
  \typeclashes(\tenv, \vt, \vs) \typearrow \overname{\False}{\vb}
}
\end{mathpar}

\begin{mathpar}
\inferrule[otherwise\_structured]{
  \subtypesrel(\tenv, \vs, \vt) \typearrow \False\\
  \subtypesrel(\tenv, \vt, \vs) \typearrow \False\\\\
  \tstruct(\tenv, \vt) \typearrow \vtstruct \\
  \tstruct(\tenv, \vs) \typearrow \vsstruct \\
  \astlabel(\vtstruct) = \astlabel(\vsstruct)\\
  \vb \eqdef \astlabel(\vtstruct) \in \{\TRecord, \TException\}\\
}{
  \typeclashes(\tenv, \vt, \vs) \typearrow \overname{\False}{\vb}
}
\end{mathpar}

\subsubsection{Comments}
Note that if $\vt$ subtype-satisfies $\vs$ then $\vt$ and $\vs$ type-clash, but not the other
way around.

Note that type-clashing is an equivalence relation. Therefore if $\vt$
type-clashes with \texttt{A} and \texttt{B} then it is also the case that \texttt{A} and \texttt{B} type-clash.

\lrmcomment{This is related to \identd{VPZZ}, \identi{PQCT} and \identi{WZKM}.}

\TypingRuleDef{LowestCommonAncestor}
\hypertarget{def-lowestcommonancestor}{}
Annotating a conditional expression (see \TypingRuleRef{ECond}),
requires finding a single type that can be used to annotate the results of both subexpressions.
The function
\[
  \lca(\overname{\staticenvs}{\tenv} \aslsep \overname{\ty}{\vt} \aslsep \overname{\ty}{\vs})
  \aslto \overname{\ty}{\tty} \cup \overname{\TTypeError}{\TypeErrorConfig}
\]
returns the \emph{lowest common ancestor} of types $\vt$ and $\vs$ in $\tenv$ --- $\tty$.
The result is a type error if a lowest common ancestor does not exist or a type error is detected.

\ProseParagraph
One of the following applies:
\begin{itemize}
  \item All of the following apply (\textsc{type\_equal}):
  \begin{itemize}
    \item $\vt$ is \typeequal\ to $\vs$ in $\tenv$;
    \item $\tty$ is $\vs$ (can as well be $\vt$).
  \end{itemize}

  \item All of the following apply:
  \begin{itemize}
    \item $\vt$ is not \typeequal\ to $\vs$ in $\tenv$ and one of the following applies:

    \item All of the following apply (\textsc{named\_subtype1}):
    \begin{itemize}
      \item $\vt$ is a named type with identifier $\namesubt$, that is, $\TNamed(\namesubt)$;
      \item $\vs$ is a named type with identifier $\namesubs$, that is, $\TNamed(\namesubs)$;
      \item there is no \namedlowestcommonancestor\ of $\namesubs$ and $\namesubt$ in $\tenv$;
      \item obtaining the \underlyingtype\ of $\vs$ yields $\vanons$\ProseOrTypeError;
      \item obtaining the \underlyingtype\ of $\vt$ yields $\vanont$\ProseOrTypeError;
      \item obtaining the lowest common ancestor of $\vanons$ and $\vanont$ in $\tenv$ yields $\tty$\ProseOrTypeError.
    \end{itemize}

    \item All of the following apply (\textsc{named\_subtype2}):
    \begin{itemize}
      \item $\vt$ is a named type with identifier $\namesubt$, that is, $\TNamed(\namesubt)$;
      \item $\vs$ is a named type with identifier $\namesubs$, that is, $\TNamed(\namesubs)$;
      \item the \namedlowestcommonancestor\ of $\namesubs$ and $\namesubt$ in $\tenv$ is \\
            $\name$\ProseOrTypeError;
      \item $\tty$ is the named type with identifier $\name$, that is, $\TNamed(\name)$.
    \end{itemize}

    \item All of the following apply (\textsc{one\_named1}):
    \begin{itemize}
      \item only one of $\vt$ or $\vs$ is a named type;
      \item obtaining the \underlyingtype\ of $\vs$ yields $\vanons$\ProseOrTypeError;
      \item obtaining the \underlyingtype\ of $\vt$ yields $\vanont$\ProseOrTypeError;
      \item $\vanont$ is \typeequal\ to $\vanons$;
      \item $\tty$ is $\vt$ if it is a named type (that is, $\astlabel(\vt)=\TNamed$), and $\vs$ otherwise.
    \end{itemize}

    \item All of the following apply (\textsc{one\_named2}):
    \begin{itemize}
      \item only one of $\vt$ or $\vs$ is a named type;
      \item obtaining the \underlyingtype\ of $\vs$ yields $\vanons$\ProseOrTypeError;
      \item obtaining the \underlyingtype\ of $\vt$ yields $\vanont$\ProseOrTypeError;
      \item $\vanont$ is not \typeequal\ to $\vanons$;
      \item the lowest common ancestor of $\vanont$ and $\vanons$ in $\tenv$ is $\tty$\ProseOrTypeError.
    \end{itemize}

    \item All of the following apply (\textsc{t\_int\_unconstrained}):
    \begin{itemize}
      \item both $\vt$ and $\vs$ are integer types;
      \item at least one of $\vt$ or $\vs$ is an unconstrained integer type;
      \item $\tty$ is the unconstrained integer type.
    \end{itemize}

    \item All of the following apply (\textsc{t\_int\_parameterized}):
    \begin{itemize}
      \item neither $\vt$ nor $\vs$ are the unconstrained integer type;
      \item one of $\vt$ and $\vs$ is a \parameterizedintegertype;
      \item the \wellconstrainedversion\ of $\vt$ is $\vtone$;
      \item the \wellconstrainedversion\ of $\vs$ is $\vsone$;
      \item $\tty$ the lowest common ancestor of $\vtone$ and $\vsone$ in $\tenv$ is $\tty$\ProseOrTypeError.
    \end{itemize}

    \item All of the following apply (\textsc{t\_int\_wellconstrained}):
    \begin{itemize}
      \item $\vt$ is a well-constrained integer type with constraints $\cst$;
      \item $\vs$ is a well-constrained integer type with constraints $\css$;
      \item $\tty$ is the well-constrained integer type with constraints $\cst \concat \css$.
    \end{itemize}

    \item All of the following apply (\textsc{t\_bits}):
    \begin{itemize}
      \item $\vt$ is a bitvector type with length expression $\vet$, that is, $\TBits(\vet, \Ignore)$;
      \item $\vs$ is a bitvector type with length expression $\ves$, that is, $\TBits(\ves, \Ignore)$;
      \item applying $\typeequal$ to $\vt$ and $\vs$ in $\tenv$ yields $\False$;
      \item applying $\exprequal$ to $\vet$ and $\ves$ in $\tenv$ yields $\vbequal$;
      \item checking whether $\vbequal$ is $\True$ yields $\True$\ProseTerminateAs{\NoLCA};
      \item $\tty$ is a bitvector type with length expression $\vet$ and an empty bitfield list, that is, $\TBits(\vet, \emptylist)$.
    \end{itemize}

    \item All of the following apply (\textsc{t\_array}):
    \begin{itemize}
      \item $\vt$ is an array type with width expression $\widtht$ and element type $\vtyt$;
      \item $\vs$ is an array type with width expression $\widths$ and element type $\vtys$;
      \item applying $\arraylengthequal$ to $\widtht$ and $\widths$ in $\tenv$ to equate the array lengths,
            yields $\vbequallength$\ProseOrTypeError;
      \item checking that $\vbequallength$ is $\True$ yields $\True$\ProseTerminateAs{\NoLCA};
      \item the lowest common ancestor of $\vtyt$ and $\vtys$ is $\vtone$\ProseOrTypeError;
      \item $\tty$ is an array type with width expression $\widths$ and element type $\vtone$.
    \end{itemize}

    \item All of the following apply (\textsc{t\_tuple}):
    \begin{itemize}
      \item $\vt$ is a tuple type with type list $\vlit$;
      \item $\vs$ is a tuple type with type list $\vlis$;
      \item checking whether $\vlit$ and $\vlis$ have the same number of elements yields $\True$
            or a type error, which short-circuits the entire rule (indicating that the number of elements in both tuples is expected
            to be the same and thus there is no lowest common ancestor);
      \item applying $\lca$ to $\vlit[\vi]$ and $\vlis[\vi]$ in $\tenv$, for every position of $\vlit$,
            yields $\vt_\vi$\ProseOrTypeError;
      \item define $\vli$ to be the list of types $\vt_\vi$, for every position of $\vlit$;
      \item define $\tty$ as the tuple type with list of types $\vli$, that is, $\TTuple(\vli)$.
    \end{itemize}

    \item All of the following apply (\textsc{error}):
    \begin{itemize}
      \item either the AST labels of $\vt$ and $\vs$ are different, or one of them is $\TEnum$, $\TRecord$, or $\TException$;
      \item the result is a type error indicating the lack of a lowest common ancestor.
    \end{itemize}
  \end{itemize}
\end{itemize}
\CodeSubsection{\LowestCommonAncestorBegin}{\LowestCommonAncestorEnd}{../types.ml}

\FormallyParagraph
Since we do not impose a canonical representation on types (e.g., \verb|integer {1, 2}| is equivalent to \verb|integer {1..2}|),
the lowest common ancestor is not unique.
We define $\lca(\tenv, \vt, \vs)$ to be any type $\vtp$ that is \typeequivalent\ to the lowest common ancestor of $\vt$ and $\vs$.

\begin{mathpar}
\inferrule[type\_equal]{
  \typeequal(\tenv, \vt, \vs) \typearrow \True
}{
  \lca(\tenv, \vt, \vs) \typearrow \overname{\vs}{\tty}
}
\end{mathpar}

\begin{mathpar}
\inferrule[named\_subtype1]{
  \vt = \TNamed(\namesubs)\\
  \vs = \TNamed(\namesubt)\\
  \typeequal(\tenv, \vt, \vs) \typearrow \False\\
  \namedlca(\tenv, \namesubs, \namesubt) \typearrow \None \OrTypeError\\\\
  \makeanonymous(\tenv, \vs) \typearrow \vanons \OrTypeError\\\\
  \makeanonymous(\tenv, \vt) \typearrow \vanont \OrTypeError\\\\
  \lca(\tenv, \vanont, \vanons) \typearrow \tty \OrTypeError
}{
  \lca(\tenv, \vt, \vs) \typearrow \tty
}
\end{mathpar}

\begin{mathpar}
\inferrule[named\_subtype2]{
  \vt = \TNamed(\namesubs)\\
  \vs = \TNamed(\namesubt)\\
  \typeequal(\tenv, \vt, \vs) \typearrow \False\\
  \namedlca(\tenv, \namesubs, \namesubt) \typearrow \langle\name\rangle \OrTypeError\\
}{
  \lca(\tenv, \vt, \vs) \typearrow \overname{\TNamed(\name)}{\tty}
}
\end{mathpar}

\begin{mathpar}
\inferrule[one\_named1]{
  \typeequal(\tenv, \vt, \vs) \typearrow \False\\
  (\astlabel(\vt) = \TNamed \lor \astlabel(\vs) = \TNamed)\\
  \astlabel(\vt) \neq \astlabel(\vs)\\
  \makeanonymous(\tenv, \vs) \typearrow \vanons \OrTypeError\\\\
  \makeanonymous(\tenv, \vt) \typearrow \vanont \OrTypeError\\\\
  \typeequal(\tenv, \vanont, \vanons) \typearrow \True\\
  \tty \eqdef \choice{\astlabel(\vt) = \TNamed}{\vt}{\vs}
}{
  \lca(\tenv, \vt, \vs) \typearrow \tty
}
\end{mathpar}

\begin{mathpar}
\inferrule[one\_named2]{
  \typeequal(\tenv, \vt, \vs) \typearrow \False\\
  (\astlabel(\vt) = \TNamed \lor \astlabel(\vs) = \TNamed)\\
  \astlabel(\vt) \neq \astlabel(\vs)\\
  \makeanonymous(\tenv, \vs) \typearrow \vanons \OrTypeError\\\\
  \makeanonymous(\tenv, \vt) \typearrow \vanont \OrTypeError\\\\
  \typeequal(\tenv, \vanont, \vanons) \typearrow \False\\
  \lca(\tenv, \vanont, \vanons) \typearrow \tty \OrTypeError
}{
  \lca(\tenv, \vt, \vs) \typearrow \tty
}
\end{mathpar}

\begin{mathpar}
\inferrule[t\_int\_unconstrained]{
  \typeequal(\tenv, \vt, \vs) \typearrow \False\\
  \astlabel(\vt) = \astlabel(\vs) = \TInt\\
  \isunconstrainedinteger(\vt) \lor \isunconstrainedinteger(\vs)
}{
  \lca(\tenv, \vt, \vs) \typearrow \overname{\unconstrainedinteger}{\tty}
}
\and
\inferrule[t\_int\_parameterized]{
  \typeequal(\tenv, \vt, \vs) \typearrow \False\\
  \astlabel(\vt) = \astlabel(\vs) = \TInt\\
  \neg\isunconstrainedinteger(\vt)\\
  \neg\isunconstrainedinteger(\vs)\\
  \isparameterizedinteger(\vt) \lor \isparameterizedinteger(\vs)\\
  \towellconstrained(\tenv, \vt) \typearrow \vtone\\
  \towellconstrained(\tenv, \vs) \typearrow \vsone\\
  \lca(\tenv, \vtone, \vsone) \typearrow \tty \OrTypeError
}{
  \lca(\tenv, \vt, \vs) \typearrow \tty
}
\and
\inferrule[t\_int\_wellconstrained]
{
  \typeequal(\tenv, \vt, \vs) \typearrow \False\\
  \vt \eqname \TInt(\wellconstrained(\cst))\\
  \vs \eqname \TInt(\wellconstrained(\css))
}{
  \lca(\tenv, \vt, \vs) \typearrow \overname{\TInt(\wellconstrained(\cst \concat \css))}{\tty}
}
\end{mathpar}

\begin{mathpar}
\inferrule[t\_bits]{
  \typeequal(\tenv, \vt, \vs) \typearrow \False\\
  \exprequal(\tenv, \vet, \ves) \typearrow \vbequal\\
  \checktrans{\vbequal}{\NoLCA} \checktransarrow \True \OrTypeError
}{
  \lca(\tenv, \overname{\TBits(\vet, \Ignore)}{\vt}, \overname{\TBits(\ves, \Ignore)}{\vs}) \typearrow \overname{\TBits(\vet, \emptylist)}{\tty}
}
\end{mathpar}

\begin{mathpar}
\inferrule[t\_array]{
  \typeequal(\tenv, \vt, \vs) \typearrow \False\\
  \arraylengthequal(\tenv, \widtht, \widths) \typearrow \vbequallength \OrTypeError\\\\
  \checktrans{\vbequallength}{\NoLCA} \checktransarrow \True \OrTypeError\\\\
  \lca(\tenv, \vtyt, \vtys) \typearrow \vtone \OrTypeError
}{
  {
  \begin{array}{r}
  \lca(\tenv, \overname{\TArray(\widtht, \vtyt)}{\vt}, \overname{\TArray(\widths, \vtys)}{\vs}) \typearrow \\
  \overname{\TArray(\widtht, \vtone)}{\tty}
  \end{array}
  }
}
\end{mathpar}

\begin{mathpar}
\inferrule[t\_tuple]{
  \typeequal(\tenv, \vt, \vs) \typearrow \False\\
  \equallength(\vlit, \vlis) \typearrow \vb\\
  \checktrans{\vb}{\NoLCA} \typearrow \True \OrTypeError\\\\
  {
    \begin{array}{r}
  \vi\in\listrange(\vlit): \lca(\tenv, \vlit[\vi], \vlis[\vi]) \typearrow \\
   \vt_\vi \OrTypeError
    \end{array}
  }\\
  \vli \eqdef [\vi\in\listrange(\vlit): \vt_\vi]
}{
  \lca(\tenv, \overname{\TTuple(\vlit)}{\vt}, \overname{\TTuple(\vlis)}{\vs}) \typearrow \overname{\TTuple(\vli)}{\tty}
}
\end{mathpar}

\begin{mathpar}
\inferrule[error]{
  \typeequal(\tenv, \vt, \vs) \typearrow \False\\
  (\astlabel(\vt) \neq \astlabel(\vs)) \lor
  \astlabel(\vt) \in \{\TEnum, \TRecord, \TException\}
}{
  \lca(\tenv, \vt, \vs) \typearrow \TypeErrorVal{\NoLCA}
}
\end{mathpar}

\lrmcomment{This is related to \identr{YZHM}.}

\TypingRuleDef{ApplyUnopType}
\hypertarget{def-applyunoptype}{}
The function
\[
  \applyunoptype(\overname{\staticenvs}{\tenv} \aslsep \overname{\unop}{\op} \aslsep \overname{\ty}{\vt})
  \aslto \overname{\ty}{\vs} \cup \overname{\TTypeError}{\TypeErrorConfig}
\]
determines the result type of applying a unary operator when the type of its operand is known.
Similarly, we determine the negation of integer constraints.
\ProseOtherwiseTypeError

\ProseParagraph
One of the following applies:
\begin{itemize}
\item All of the following apply (\textsc{bnot\_t\_bool}):
  \begin{itemize}
    \item $\op$ is $\BNOT$;
    \item determining whether $\vt$ \typesatisfies\ $\TBool$ yields $\True$\ProseOrTypeError;
    \item $\vs$ is $\TBool$;
  \end{itemize}

\item All of the following apply (\textsc{neg\_error}):
\begin{itemize}
  \item $\op$ is $\NEG$;
  \item determining whether $\vt$ \typesatisfies\ $\TReal$ yields $\False$\ProseOrTypeError;
  \item determining whether $\vt$ \typesatisfies\ $\unconstrainedinteger$ yields $\False$\ProseOrTypeError;
  \item the result is a type error indicating the $\NEG$ is appropriate only for \texttt{real} and \texttt{integer} types;
\end{itemize}

\item All of the following apply (\textsc{neg\_t\_real}):
\begin{itemize}
  \item $\op$ is $\NEG$;
  \item determining whether $\vt$ \typesatisfies\ $\TReal$ yields $\True$;
  \item $\vs$ is $\TReal$;
\end{itemize}

\item All of the following apply (\textsc{neg\_t\_int\_unconstrained}):
\begin{itemize}
  \item $\op$ is $\NEG$;
  \item obtaining the \wellconstrainedstructure\ of $\vt$ yields $\unconstrainedinteger$\ProseOrTypeError;
  \item $\vs$ is $\unconstrainedinteger$;
\end{itemize}

\item All of the following apply (\textsc{neg\_t\_int\_well\_constrained}):
\begin{itemize}
  \item $\op$ is $\NEG$;
  \item obtaining the \wellconstrainedstructure\ of $\vt$ yields the well-constrained integer type with constraints $\vcs$\ProseOrTypeError;
  \item negating the constraints in $\vcs$ (see $\negateconstraint$) yields $\vcsnew$;
  \item $\vs$ is the well-constrained integer type with constraints $\vcsnew$, that is, \\
  $\TInt(\wellconstrained(\vcsnew))$;
\end{itemize}

\item All of the following apply (\textsc{not\_t\_bits}):
  \begin{itemize}
  \item $\op$ is $\NOT$;
  \item $\vt$ has the structure of a bitvector;
  \item $\vs$ is $\vt$.
  \end{itemize}
\end{itemize}

\FormallyParagraph
\begin{mathpar}
\inferrule[bnot\_t\_bool]{
  \checktypesat(\tenv, \vtone, \TBool) \typearrow \True \OrTypeError\\
}{
  \applyunoptype(\tenv, \BNOT, \vtone) \typearrow \TBool
}
\end{mathpar}
\CodeSubsection{\ApplyUnopTypeBegin}{\ApplyUnopTypeEnd}{../Typing.ml}

\hypertarget{def-negateconstraint}{}
We now define the helper function
\[
  \negateconstraint(\intconstraint) \aslto \intconstraint
\]
which takes an integer constraint and returns the constraint that corresponds to the negation of all
the values it represents:

\begin{mathpar}
\inferrule{}
{
  \negateconstraint(\ConstraintExact(\ve)) \typearrow \ConstraintExact(\EUnop(\MINUS, \ve))
}
\and
\inferrule{}
{
  \negateconstraint(\ConstraintRange(\vvtop, \vbot)) \typearrow \\
  \ConstraintRange(\EUnop(\MINUS, \vbot), \EUnop(\MINUS, \vvtop))
}
\end{mathpar}

\begin{mathpar}
\inferrule[neg\_error]{
  \typesat(\tenv, \vt, \unconstrainedinteger) \typearrow \False \OrTypeError\\\\
  \typesat(\tenv, \vt, \TReal) \typearrow \False \OrTypeError\\
}{
  \applyunoptype(\tenv, \overname{\NEG}{\op}, \vt) \typearrow \TypeErrorVal{\BadOperands}
}
\end{mathpar}

\begin{mathpar}
\inferrule[neg\_t\_real]{
  \typesat(\tenv, \vt, \TReal) \typearrow \True
}{
  \applyunoptype(\tenv, \overname{\NEG}{\op}, \vt) \typearrow \overname{\TReal}{\vs}
}
\end{mathpar}

\begin{mathpar}
\inferrule[neg\_t\_int\_unconstrained]{
  \getwellconstrainedstructure(\tenv, \vt) \typearrow \unconstrainedinteger \OrTypeError
}{
  \applyunoptype(\tenv, \overname{\NEG}{\op}, \vt) \typearrow \overname{\unconstrainedinteger}{\vs}
}
\end{mathpar}

\begin{mathpar}
\inferrule[neg\_t\_int\_well\_constrained]{
  \getwellconstrainedstructure(\tenv, \vt) \typearrow \TInt(\wellconstrained(\vcs))\\
  \vc \in \vcs: \negateconstraint(\vc) \typearrow \vneg_\vc\\
  \vcsnew \eqdef [\vc \in \vcs: \vneg_\vc]
}{
  \applyunoptype(\tenv, \overname{\NEG}{\op}, \vt) \typearrow \overname{\TInt(\wellconstrained(\vcsnew))}{\vs}
}
\end{mathpar}

\begin{mathpar}
\inferrule[not\_t\_bits]{
  \checkstructurelabel(\tenv, \vt, \TBits) \typearrow \True \OrTypeError
}{
  \applyunoptype(\tenv, \overname{\NOT}{\op}, \vt) \typearrow \vt
}
\end{mathpar}

\TypingRuleDef{ApplyBinopTypes}
\hypertarget{def-applybinoptypes}{}
The function
\[
  \applybinoptypes(\overname{\staticenvs}{\tenv} \aslsep \overname{\binop}{\op} \aslsep \overname{\ty}{\vtone}
  \aslsep \overname{\ty}{\vttwo})
  \aslto \overname{\ty}{\vt} \cup \overname{\TTypeError}{\TypeErrorConfig}
\]
determines the result type $\vt$ of applying the binary operator $\op$
to operands of type $\vtone$ and $\vttwo$ in the static environment $\tenv$.
\ProseOtherwiseTypeError

\ProseParagraph
One of the following applies:
\begin{itemize}
  \item All of the following apply (\textsc{named}):
  \begin{itemize}
    \item at least one of $\vtone$ and $\vttwo$ is a \namedtype;
    \item determining the \underlyingtype\ if $\vtone$ yields $\vtoneanon$\ProseOrTypeError;
    \item determining the \underlyingtype\ if $\vttwo$ yields $\vttwoanon$\ProseOrTypeError;
    \item \Proseapplybinoptypes{$\tenv$}{$\op$}{$\vtoneanon$}{$\vttwoanon$}{$\vt$\ProseOrTypeError}.
  \end{itemize}

  \item All of the following apply (\textsc{boolean}):
  \begin{itemize}
    \item $\op$ is $\AND$, $\OR$, $\EQOP$ or $\IMPL$;
    \item both $\vtone$ and $\vttwo$ are $\TBool$;
    \item $\vt$ is $\TBool$.
  \end{itemize}

  \item All of the following apply (\textsc{bits\_arith}):
  \begin{itemize}
    \item $\op$ is one of $\AND$, $\OR$, $\XOR$, $\PLUS$, and $\MINUS$;
    \item $\vtone$ is a bitvector type with width expression $\vwone$;
    \item $\vttwo$ is a bitvector type with width expression $\vwtwo$;
    \item checking whether $\vtone$ and $\vttwo$ have the \structure\ of bitvector types
          of the same width in $\tenv$ yields $\True$\ProseOrTypeError;
    \item $\vt$ is the bitvector type of width $\vwone$ and empty list of bitfields, that is, \\ $\TBits(\vwone, \emptylist)$.
  \end{itemize}

  \item All of the following apply (\textsc{bits\_int}):
  \begin{itemize}
    \item $\op$ is either $\PLUS$ or $\MINUS$;
    \item $\vtone$ is a bitvector type with width expression $\vw$;
    \item $\vttwo$ is an integer type;
    \item $\vt$ is the bitvector type of width $\vw$ and empty list of bitfields, that is, \\ $\TBits(\vw, \emptylist)$.
  \end{itemize}

  \item All of the following apply (\textsc{bits\_concat}):
  \begin{itemize}
    \item $\op$ is $\BVCONCAT$;
    \item $\vtone$ is a bitvector type with width expression $\vwone$;
    \item $\vttwo$ is a bitvector type with width expression $\vwtwo$;
    \item define $\vw$ as the addition of $\vwone$ and $\vwtwo$;
    \item applying \normalize{} to $\vw$ in $\tenv$ yields $\vwp$;
    \item $\vt$ is the bitvector type of width $\vwp$ and empty list of bitfields, that is, \\ $\TBits(\vw, \emptylist)$.
  \end{itemize}

  \item All of the following apply (\textsc{rel}):
  \begin{itemize}
    \item the operator $\op$ and types of $\vtone$ and $\vttwo$ match one of the rows in the following table:
    \[
    \begin{array}{lll}
      \mathbf{\op} & \mathbf{\vtone} & \mathbf{\vttwo} \\
      \hline
      \LEQ  & \TInt    & \TInt\\
      \GEQ  & \TInt    & \TInt\\
      \GT   & \TInt    & \TInt\\
      \LT   & \TInt    & \TInt\\
      \LEQ  & \TReal   & \TReal\\
      \GEQ  & \TReal   & \TReal\\
      \GT   & \TReal   & \TReal\\
      \LT   & \TReal   & \TReal\\
      \EQOP & \TInt    & \TInt\\
      \NEQ  & \TInt    & \TInt\\
      \EQOP & \TBool   & \TBool\\
      \NEQ  & \TBool   & \TBool\\
      \EQOP & \TReal   & \TReal\\
      \NEQ  & \TReal   & \TReal\\
      \EQOP & \TString & \TString\\
      \NEQ  & \TString & \TString
    \end{array}
    \]
    \item $\vt$ is $\TBool$.
  \end{itemize}

  \item All of the following apply (\textsc{eq\_neq\_bits}):
  \begin{itemize}
    \item $\op$ is either $\EQOP$ or $\NEQ$;
    \item $\vtone$ is a bitvector type with width expression $\vwone$;
    \item $\vttwo$ is a bitvector type with width expression $\vwtwo$;
    \item checking whether the bitwidth of $\vtoneanon$ and $\vttwoanon$ is the same yields $\True$\ProseOrTypeError;
    \item $\vt$ is $\TBool$.
  \end{itemize}

  \item All of the following apply (\textsc{eq\_neq\_enum}):
  \begin{itemize}
    \item $\op$ is either $\EQOP$ or $\NEQ$;
    \item $\vtone$ is $\TEnum(\vlione)$;
    \item $\vttwo$ is $\TEnum(\vlitwo)$;
    \item checking whether $\vlione$ is equal to $\vlitwo$ yields $\True$\ProseOrTypeError;
    \item $\vt$ is $\TBool$.
  \end{itemize}

  \item All of the following apply (\textsc{arith\_t\_int\_unconstrained}):
  \begin{itemize}
    \item $\op$ is one of $\{\MUL, \DIV, \DIVRM, \MOD, \SHL,  \SHR, \POW, \PLUS, \MINUS\}$;
    \item both $\vtone$ and $\vttwo$ are integer types and at least one them is the unconstrained integer type;
    \item $\vt$ is the unconstrained integer type;
  \end{itemize}

  \item All of the following apply (\textsc{arith\_t\_int\_parameterized}):
  \begin{itemize}
    \item $\op$ is one of $\{\MUL, \DIV, \DIVRM, \MOD, \SHL,  \SHR, \POW, \PLUS, \MINUS\}$;
    \item both $\vtone$ and $\vttwo$ are integer types, neither is an unconstrained integer type, and at least one them is a \parameterizedintegertype;
    \item applying $\towellconstrained$ to $\vtone$ yields $\vtonewellconstrained$;
    \item applying $\towellconstrained$ to $\vttwo$ yields $\vttwowellconstrained$;
    \item \Proseapplybinoptypes{$\tenv$}{$\op$}{$\vtonewellconstrained$}{$\vttwowellconstrained$}{$\vt$}.
  \end{itemize}

  \item All of the following apply (\textsc{arith\_t\_int\_wellconstrained}):
  \begin{itemize}
    \item $\op$ is one of $\{\MUL, \POW, \PLUS, \MINUS, \DIVRM, \DIV, \MOD, \SHL, \SHR\}$;
    \item $\vtone$ is the well-constrained integer type with constraints $\csone$;
    \item $\vttwo$ is the well-constrained integer type with constraints $\cstwo$;
    \item applying $\annotateconstraintbinop$ to $\op$, $\csone$, and $\cstwo$ in $\tenv$ yields $\vc$;
    \item $\vt$ is the integer type with constraint kind $\vc$;
  \end{itemize}

  \item All of the following apply (\textsc{arith\_real}):
  \begin{itemize}
    \item the operator $\op$ and types of $\vtone$ and $\vttwo$ match one of the rows in the following table:
    \[
    \begin{array}{lll}
      \mathbf{\op} & \mathbf{\vtone} & \mathbf{\vttwo} \\
      \hline
      \PLUS  & \TReal    & \TReal\\
      \MINUS & \TReal    & \TReal\\
      \MUL   & \TReal    & \TReal\\
      \POW   & \TReal    & \TInt\\
      \RDIV  & \TReal    & \TReal
    \end{array}
    \]
    \item $\vt$ is $\TReal$.
  \end{itemize}

  \item All of the following apply (\textsc{error}):
  \begin{itemize}
    \item obtaining the \underlyingtype\ of $\vtone$ in $\tenv$ yields $\vtoneanon$\ProseOrTypeError;
    \item obtaining the \underlyingtype\ of $\vttwo$ in $\tenv$ yields $\vttwoanon$\ProseOrTypeError;
    \item the operator and the AST labels of $\vtoneanon$ and $\vttwoanon$ do not match any of the rows in the following table:
    \[
    \begin{array}{lll}
      \hline
      \mathbf{\op} & \mathbf{\astlabel(\vtoneanon)} & \mathbf{\astlabel(\vttwoanon)} \\
      \hline
      \AND     & \TBool  & \TBool\\
      \OR      & \TBool  & \TBool\\
      \EQOP    & \TBool  & \TBool\\
      \IMPL    & \TBool  & \TBool\\
      %
      \AND     & \TBits  & \TBits\\
      \OR      & \TBits  & \TBits\\
      \XOR     & \TBits  & \TBits\\
      \PLUS    & \TBits  & \TBits\\
      \MINUS   & \TBits  & \TBits\\
      \BVCONCAT & \TBits  & \TBits\\
      %
      \PLUS    & \TBits  & \TInt\\
      \MINUS   & \TBits  & \TInt\\
      %
      \LEQ     & \TInt     & \TInt\\
      \GEQ     & \TInt     & \TInt\\
      \GT      & \TInt     & \TInt\\
      \LT      & \TInt     & \TInt\\
      \LEQ     & \TReal    & \TReal\\
      \GEQ     & \TReal    & \TReal\\
      \GT      & \TReal    & \TReal\\
      \LT      & \TReal    & \TReal\\
      \EQOP    & \TInt     & \TInt\\
      \NEQ     & \TInt     & \TInt\\
      \EQOP    & \TBool    & \TBool\\
      \NEQ     & \TBool    & \TBool\\
      \EQOP    & \TReal    & \TReal\\
      \NEQ     & \TReal    & \TReal\\
      \EQOP    & \TString  & \TString\\
      \NEQ     & \TString  & \TString\\
      %
      \MUL     & \TInt  & \TInt\\
      \DIV     & \TInt  & \TInt\\
      \DIVRM   & \TInt  & \TInt\\
      \MOD     & \TInt  & \TInt\\
      \SHL     & \TInt  & \TInt\\
      \SHR     & \TInt  & \TInt\\
      \POW     & \TInt  & \TInt\\
      \PLUS    & \TInt  & \TInt\\
      \MINUS   & \TInt  & \TInt\\
      \PLUS    & \TReal & \TReal\\
      \MINUS   & \TReal & \TReal\\
      \MUL     & \TReal & \TReal\\
      \RDIV    & \TReal & \TReal\\
      \POW     & \TReal & \TInt\\
      %
      \PLUS    & \TReal & \TReal\\
      \MINUS   & \TReal & \TReal\\
      \MUL     & \TReal & \TReal\\
      \POW     & \TReal & \TInt\\
      \RDIV    & \TReal & \TReal\\
      \hline
    \end{array}
    \]
  \end{itemize}
\end{itemize}

\FormallyParagraph
\begin{mathpar}
\inferrule[named]{
  \astlabel(\vtone) = \TNamed \lor \astlabel(\vttwo) = \TNamed\\
  \makeanonymous(\tenv, \vtone) \typearrow \vtoneanon \OrTypeError\\\\
  \makeanonymous(\tenv, \vttwo) \typearrow \vttwoanon \OrTypeError\\\\
  \applybinoptypes(\tenv, \op, \vtoneanon, \vttwoanon) \typearrow \vt \OrTypeError
}{
  \applybinoptypes(\tenv, \op, \vtone, \vttwo) \typearrow \vt
}
\end{mathpar}

\begin{mathpar}
\inferrule[boolean]{
  \op \in  \{\BAND, \BOR, \IMPL, \EQOP\}
}{
  \applybinoptypes(\tenv, \op, \overname{\TBool}{\vtone}, \overname{\TBool}{\vttwo}) \typearrow \overname{\TBool}{\vt}
}
\end{mathpar}

\begin{mathpar}
\inferrule[bits\_arith]{
  \op \in  \{\AND, \OR, \XOR, \PLUS, \MINUS\}\\
  \checkbitsequalwidth(\tenv, \vtone, \vttwo) \typearrow \True \OrTypeError
}{
  \applybinoptypes(\tenv, \op, \overname{\TBits(\vwone, \Ignore)}{\vtone}, \overname{\TBits(\vwtwo, \Ignore)}{\vttwo})
  \typearrow \overname{\TBits(\vwone, \emptylist)}{\vt}
}
\end{mathpar}

\begin{mathpar}
\inferrule[bits\_int]{
  \op \in  \{\PLUS, \MINUS\}}{
  \applybinoptypes(\tenv, \op, \overname{\TBits(\vw, \Ignore)}{\vtone}, \overname{\TInt(\Ignore)}{\vttwo}) \typearrow
  \overname{\TBits(\vw, \emptylist)}{\vt}
}
\end{mathpar}

\begin{mathpar}
\inferrule[bits\_concat]{
  \vw \eqdef \EBinop(\PLUS, \vwone, \vwtwo) \\
  \normalize(\tenv, \vw) \typearrow \vwp
}{
  \applybinoptypes(\tenv, \BVCONCAT, \overname{\TBits(\vwone, \Ignore)}{\vtone}, \overname{\TBits(\vwtwo, \Ignore)}{\vttwo}) \typearrow
  \overname{\TBits(\vwp, \emptylist)}{\vt}
}
\end{mathpar}

\begin{mathpar}
\inferrule[rel]{
  {
    (\op, \vtone, \vttwo) \in \left\{
      \begin{array}{lclcl}
        (\LEQ     &,& \TInt     &,& \TInt)\\
        (\GEQ     &,& \TInt     &,& \TInt)\\
        (\GT      &,& \TInt     &,& \TInt)\\
        (\LT      &,& \TInt     &,& \TInt)\\
        (\LEQ     &,& \TReal    &,& \TReal)\\
        (\GEQ     &,& \TReal    &,& \TReal)\\
        (\GT      &,& \TReal    &,& \TReal)\\
        (\LT      &,& \TReal    &,& \TReal)\\
        (\EQOP    &,& \TInt     &,& \TInt)\\
        (\NEQ     &,& \TInt     &,& \TInt)\\
        (\EQOP    &,& \TBool    &,& \TBool)\\
        (\NEQ     &,& \TBool    &,& \TBool)\\
        (\EQOP    &,& \TReal    &,& \TReal)\\
        (\NEQ     &,& \TReal    &,& \TReal)\\
        (\EQOP    &,& \TString  &,& \TString)\\
        (\NEQ     &,& \TString  &,& \TString)\\
      \end{array}
      \right\}
  }
}{
  \applybinoptypes(\tenv, \op, \vtone, \vttwo) \typearrow \overname{\TBool}{\vt}
}
\end{mathpar}

\begin{mathpar}
\inferrule[eq\_neq\_bits]{
  \op \in  \{\EQOP, \NEQ\}\\
  \checkbitsequalwidth(\tenv, \vtoneanon, \vttwoanon) \typearrow \True \OrTypeError
}{
  \applybinoptypes(\tenv, \op, \overname{\TBits(\vwone, \Ignore)}{\vtone}, \overname{\TBits(\vwtwo, \Ignore)}{\vttwo}) \typearrow \overname{\TBool}{\vt}
}
\end{mathpar}

\begin{mathpar}
\inferrule[eq\_neq\_enum]{
  \op \in  \{\EQOP, \NEQ\}\\
  \checktrans{\vlione = \vlitwo}{DifferentEnumLabels} \checktransarrow \True \OrTypeError
}{
  \applybinoptypes(\tenv, \op, \overname{\TEnum(\vlione)}{\vtone}, \overname{\TEnum(\vlitwo)}{\vttwo}) \typearrow \overname{\TBool}{\vt}
}
\end{mathpar}

\begin{mathpar}
\inferrule[arith\_t\_int\_unconstrained]{
  \op \in  \{\MUL, \DIV, \DIVRM, \MOD, \SHL,  \SHR, \POW, \PLUS, \MINUS\}\\
  \vcone = \unconstrained \lor \vctwo = \unconstrained
}{
  \applybinoptypes(\tenv, \op, \overname{\TInt(\vcone)}{\vtone}, \overname{\TInt(\vctwo)}{\vttwo}) \typearrow \unconstrainedinteger
}
\end{mathpar}

\begin{mathpar}
\inferrule[arith\_t\_int\_parameterized]{
  \op \in  \{\MUL, \DIV, \DIVRM, \MOD, \SHL,  \SHR, \POW, \PLUS, \MINUS\}\\
  \astlabel(\vcone) = \parameterized \lor \astlabel(\vctwo) = \parameterized\\
  \astlabel(\vcone) \neq \unconstrained \land \astlabel(\vctwo) \neq \unconstrained\\
  \towellconstrained(\vtone) \typearrow \vtonewellconstrained\\
  \towellconstrained(\vttwo) \typearrow \vttwowellconstrained\\
  \applybinoptypes(\tenv, \vtonewellconstrained, \vttwowellconstrained) \typearrow \vt \OrTypeError
}{
  \applybinoptypes(\tenv, \op, \overname{\TInt(\vcone)}{\vtone}, \overname{\TInt(\vctwo)}{\vttwo}) \typearrow \vt
}
\end{mathpar}

\begin{mathpar}
\inferrule[arith\_t\_int\_wellconstrained]{
  \op \in  \{\MUL, \POW, \PLUS, \MINUS, \DIVRM, \DIV, \MOD, \SHL, \SHR\}\\
  \vcone = \wellconstrained(\cstwo)\\
  \vctwo = \wellconstrained(\csone)\\
  \annotateconstraintbinop(\tenv, \op, \vcsone, \vcstwo) \typearrow \cs \OrTypeError
}{
      \applybinoptypes(\tenv, \op, \overname{\TInt(\vcone)}{\vtone},
        \overname{\TInt(\vctwo)}{\vttwo}) \typearrow
        \overname{\TInt(\wellconstrained(\cs))}{\vt}
}
\end{mathpar}

\begin{mathpar}
\inferrule[arith\_real]{
  (\op, \vtone, \vttwo) \in
  {
    \left\{
    \begin{array}{lclcl}
      (\PLUS  &,& \TReal &,& \TReal)\\
      (\MINUS &,& \TReal &,& \TReal)\\
      (\MUL   &,& \TReal &,& \TReal)\\
      (\POW   &,& \TReal &,& \TInt)\\
      (\RDIV  &,& \TReal &,& \TReal)
    \end{array}
    \right\}
  }
}{
  \applybinoptypes(\tenv, \op, \vtone, \vttwo) \typearrow \overname{\TReal}{\vt}
}
\end{mathpar}

\begin{mathpar}
\inferrule[error]{
  \makeanonymous(\tenv, \vtone) \typearrow \vtoneanon \OrTypeError\\\\
  \makeanonymous(\tenv, \vttwo) \typearrow \vttwoanon \OrTypeError\\\\
  (\op, \astlabel(\vtoneanon), \astlabel(\vttwoanon)) \not\in
  {
    \left\{
    \begin{array}{lclcl}
      (\AND     &,& \TBool  &,& \TBool)\\
      (\OR      &,& \TBool  &,& \TBool)\\
      (\EQOP    &,& \TBool  &,& \TBool)\\
      (\IMPL    &,& \TBool  &,& \TBool)\\
      %
      (\AND     &,& \TBits  &,& \TBits)\\
      (\OR      &,& \TBits  &,& \TBits)\\
      (\XOR     &,& \TBits  &,& \TBits)\\
      (\PLUS    &,& \TBits  &,& \TBits)\\
      (\MINUS   &,& \TBits  &,& \TBits)\\
      (\BVCONCAT &,& \TBits  &,& \TBits)\\
      %
      (\PLUS    &,& \TBits  &,& \TInt)\\
      (\MINUS   &,& \TBits  &,& \TInt)\\
      %
      (\LEQ     &,& \TInt     &,& \TInt)\\
      (\GEQ     &,& \TInt     &,& \TInt)\\
      (\GT      &,& \TInt     &,& \TInt)\\
      (\LT      &,& \TInt     &,& \TInt)\\
      (\LEQ     &,& \TReal    &,& \TReal)\\
      (\GEQ     &,& \TReal    &,& \TReal)\\
      (\GT      &,& \TReal    &,& \TReal)\\
      (\LT      &,& \TReal    &,& \TReal)\\
      (\EQOP    &,& \TInt     &,& \TInt)\\
      (\NEQ     &,& \TInt     &,& \TInt)\\
      (\EQOP    &,& \TBool    &,& \TBool)\\
      (\NEQ     &,& \TBool    &,& \TBool)\\
      (\EQOP    &,& \TReal    &,& \TReal)\\
      (\NEQ     &,& \TReal    &,& \TReal)\\
      (\EQOP    &,& \TString  &,& \TString)\\
      (\NEQ     &,& \TString  &,& \TString)\\
      %
      (\MUL   &,& \TInt  &,& \TInt)\\
      (\DIV   &,& \TInt  &,& \TInt)\\
      (\DIVRM &,& \TInt  &,& \TInt)\\
      (\MOD   &,& \TInt  &,& \TInt)\\
      (\SHL   &,& \TInt  &,& \TInt)\\
      (\SHR   &,& \TInt  &,& \TInt)\\
      (\POW   &,& \TInt  &,& \TInt)\\
      (\PLUS  &,& \TInt  &,& \TInt)\\
      (\MINUS &,& \TInt  &,& \TInt)\\
      (\PLUS  &,& \TReal &,& \TReal)\\
      (\MINUS &,& \TReal &,& \TReal)\\
      (\MUL   &,& \TReal &,& \TReal)\\
      (\RDIV  &,& \TReal &,& \TReal)\\
      (\POW   &,& \TReal &,& \TInt)\\
      %
      (\PLUS  &,& \TReal &,& \TReal)\\
      (\MINUS &,& \TReal &,& \TReal)\\
      (\MUL   &,& \TReal &,& \TReal)\\
      (\POW   &,& \TReal &,& \TInt)\\
      (\RDIV  &,& \TReal &,& \TReal)
    \end{array}
    \right\}
  }
}{
  \applybinoptypes(\tenv, \op, \vtone, \vttwo) \typearrow \TypeErrorVal{\BadOperands}
}
\end{mathpar}
\CodeSubsection{\applybinoptypesBegin}{\applybinoptypesEnd}{../Typing.ml}

\lrmcomment{
  This is related to \identr{BKNT}, \identr{ZYWY}, \identr{BZKW},
  \identr{KFYS}, \identr{KXMR}, \identr{SQXN}, \identr{MRHT}, \identr{JGWF},
  \identr{TTGQ}, \identi{YHML}, \identi{YHRP}, \identi{VMZF}, \identi{YXSY},
  \identi{LGHJ}, \identi{RXLG}.
}

\TypingRuleDef{FindNamedLCA}
\hypertarget{def-namedlowestcommonancestor}{}
The function
\[
  \namedlca(\overname{\staticenvs}{\tenv} \aslsep \overname{\ty}{\vt} \aslsep \overname{\ty}{\vs})
  \aslto \overname{\ty}{\tty} \cup \overname{\TTypeError}{\TypeErrorConfig}
\]
returns the lowest common named super type --- $\tty$ --- of the types $\vt$ and $\vs$ in $\tenv$.

\newcommand\supers[0]{\texttt{supers}}
The helper function
\[
  \supers(\overname{\staticenvs}{\tenv} \aslsep \overname{\ty}{\vt})
  \aslto \pow{\ty}
\]
returns the set of \emph{named supertypes} given via the $\subtypes$ function of the global environment:
\[
  \supers(\tenv, \vt) \triangleq
  \begin{cases}
    \{\vt\} \cup \supers(\vs) & \text{ if }G^\tenv.\subtypes(\vt) = \vs\\
    \{\vt\}  & \text{ otherwise } (\text{that is, }G^\tenv.\subtypes(\vt) = \bot)\\
  \end{cases}
\]

\ProseParagraph
One of the following holds:
\begin{itemize}
  \item $\vtsupers$ is in the set of named supertypes of $\vt$;
  \item All of the following hold (\textsc{found}):
  \begin{itemize}
    \item $\vs$ is in $\vtsupers$;
    \item $\tty$ is $\vs$;
  \end{itemize}

  \item All of the following hold (\textsc{super}):
  \begin{itemize}
    \item $\vs$ is not in $\vtsupers$;
    \item $\vs$ has a named super type in $\tenv$ --- $\vsp$;
    \item $\tty$ is the lowest common named \supertypeterm{} of $\vt$ and $\vsp$ in $\tenv$.
  \end{itemize}

  \item All of the following hold (\textsc{none}):
  \begin{itemize}
    \item $\vs$ is not in $\vtsupers$;
    \item $\vs$ has no named super type in $\tenv$;
    \item $\tty$ is $\None$.
  \end{itemize}
\end{itemize}

\FormallyParagraph
\begin{mathpar}
\inferrule[found]{
  \supers(\tenv, \vt) \typearrow \vtsupers\\
  \vs \in \vtsupers
}{
  \namedlca(\tenv, \vt, \vs) \typearrow \vs
}
\and
\inferrule[super]{
  \supers(\tenv, \vt) \typearrow \vtsupers\\
  \vs \not\in \vtsupers\\
  G^\tenv.\subtypes(\vs) = \vsp\\
  \namedlca(\tenv, \vt, \vsp) \typearrow \tty
}{
  \namedlca(\tenv, \vt, \vs) \typearrow \tty
}
\and
\inferrule[none]{
  \supers(\tenv, \vt) \typearrow \vtsupers\\
  \vs \not\in \vtsupers\\
  G^\tenv.\subtypes(\vs) = \bot
}{
  \namedlca(\tenv, \vt, \vs) \typearrow \None
}
\end{mathpar}

\TypingRuleDef{AnnotateConstraintBinop}
\hypertarget{def-annotateconstraintbinop}{}
The function
\[
\begin{array}{r}
\annotateconstraintbinop(
  \overname{\staticenvs}{\tenv} \aslsep
  \overname{\binop}{\op} \aslsep
  \overname{\intconstraint^*}{\csone} \aslsep
  \overname{\intconstraint^*}{\cstwo}
) \aslto \\
\overname{\intconstraint^*}{\annotatedcs}
\cup \overname{\TTypeError}{\TypeErrorConfig}
\end{array}
\]
annotates the application of the binary operation $\op$ to the lists of integer constraints
$\csone$ and $\cstwo$, yielding a list of constraints --- $\annotatedcs$.
\ProseOtherwiseTypeError\

The operator $\op$ is assumed to be only one of the operators in the following set:
$\{\SHL, \SHR, \POW, \MOD, \DIVRM, \MINUS, \MUL, \PLUS, \DIV\}$.
The rule employs $\binopisexploding$ to decide whether range constraints can be maintained
as range constraints or have to be converted to a list of exact constraints.

For example, applying $\PLUS$ to
$\{ \AbbrevConstraintRange{\ELInt{2}}{\ELInt{4}}\}$ and
$\{\AbbrevConstraintExact{\ELInt{2}}\}$ results in
$\{\AbbrevConstraintRange{\ELInt{4}}{\ELInt{6}}\}$,
while applying $\MUL$ to the same lists of constraints results in
$\{\AbbrevConstraintExact{\ELInt{4}}, \AbbrevConstraintExact{\ELInt{6}}, \AbbrevConstraintExact{\ELInt{8}}\}$,
which is why $\binopisexploding(\MUL) \typearrow \True$.

Annotating the constraints involves applying symbolic reasoning and in particular filtering out values that
will definitely result in a dynamic error.

\ProseParagraph
\AllApply
\begin{itemize}
  \item applying $\binopfilterrhs$ to $\op$ $\cstwo$ in $\tenv$, to filter out constraints that will definitely fail dynamically, yields $\cstwof$;
  \item \OneApplies
  \begin{itemize}
    \item \AllApplyCase{exploding}
    \begin{itemize}
      \item applying $\binopisexploding$ to $\op$ yields $\True$;
      \item applying $\explodeintervals$ to $\csone$ in $\tenv$ yields $\csonee$;
      \item applying $\explodeintervals$ to $\cstwof$ in $\tenv$ yields $\cstwoe$;
      \item \Proseeqdef{$\vexpectedconstraintlength$}{the number of constraints in \\
            $\cstwoe$ if $\op$ is $\MOD$
            and the multiplication of numbers of constraints in $\csonee$ and $\cstwoe$, respectively};
      \item \Proseeqdef{$(\csonearg, \cstwoarg)$}{$(\csonee, \cstwoe)$ if \\
            $\vexpectedconstraintlength$ is
            less than $\maxconstraintsize$ and $(\csone, \cstwof)$, otherwise};
    \end{itemize}

    \item \AllApplyCase{non\_exploding}
    \begin{itemize}
      \item applying $\binopisexploding$ to $\op$ yields $\False$;
      \item \Proseeqdef{$(\csonearg, \cstwoarg)$}{$(\csone, \cstwof)$};
    \end{itemize}
  \end{itemize}
  \item applying $\constraintbinop$ to $\op$, $\csonearg$, and $\cstwoarg$ yields $\csvanilla$;
  \item applying $\refineconstraintfordiv$ to $\op$ and $\csvanilla$ yields $\refinedcs$\ProseOrTypeError;
  \item applying $\reduceconstraints$ to $\refinedcs$ in $\tenv$ yields $\annotatedcs$.
\end{itemize}

\FormallyParagraph
\begin{mathpar}
\inferrule[exploding]{
  \binopfilterrhs(\tenv, \op, \cstwo) \typearrow \cstwof\\\\
  \commonprefixline\\\\
  \binopisexploding(\op) \typearrow \True\\
  \explodeintervals(\tenv, \csone) \typearrow \csonee\\
  \explodeintervals(\tenv, \cstwof) \typearrow \cstwoe\\
  \vexpectedconstraintlength \eqdef \choice{\op = \MOD}{\listlen{\cstwoe}}{\listlen{\csonee} \times \listlen{\cstwoe}}\\
  {
  (\csonearg, \cstwoarg) \eqdef \begin{cases}
    (\csonee, \cstwoe) & \text{if }\vexpectedconstraintlength < \maxconstraintsize\\
    (\csone, \cstwof) & \text{else}
  \end{cases}
  }\\\\
  \commonsuffixline\\\\
  \constraintbinop(\op, \csonearg, \cstwoarg) \typearrow \csvanilla\\
  \refineconstraintfordiv(\op, \csvanilla) \typearrow \refinedcs \OrTypeError\\\\
  \reduceconstraints(\tenv, \refinedcs) \typearrow \annotatedcs
}{
  \annotateconstraintbinop(\tenv, \op, \csone, \cstwo) \typearrow \annotatedcs
}
\end{mathpar}

\begin{mathpar}
\inferrule[non\_exploding]{
  \binopfilterrhs(\tenv, \op, \cstwo) \typearrow \cstwof\\\\
  \commonprefixline\\\\
  \binopisexploding(\op) \typearrow \False\\
  (\csonearg, \cstwoarg) \eqdef (\csone, \cstwof)\\\\
  \commonsuffixline\\\\
  \constraintbinop(\op, \csonearg, \cstwoarg) \typearrow \csvanilla\\
  \refineconstraintfordiv(\op, \csvanilla) \typearrow \refinedcs \OrTypeError\\\\
  \reduceconstraints(\tenv, \refinedcs) \typearrow \annotatedcs
}{
  \annotateconstraintbinop(\tenv, \op, \csone, \cstwo) \typearrow \annotatedcs
}
\end{mathpar}
\CodeSubsection{\AnnotateConstraintBinopBegin}{\AnnotateConstraintBinopEnd}{../StaticOperations.ml}

\TypingRuleDef{BinopFilterRhs}
\hypertarget{def-binopfilterrhs}{}
The function
\[
\binopfilterrhs(\overname{\staticenvs}{\tenv} \aslsep \overname{\binop}{\op} \aslsep \overname{\intconstraint^*}{\cs})
\aslto \overname{\intconstraint^*}{\newcs}
\]
filters the list of constraints $\cs$ by removing values that will definitely result in a dynamic
error if found on the right-hand-side of a binary operation expression with the operator $\op$
in any environment consisting of the static environment $\tenv$.
The result is the filtered list of constraints $\newcs$.

\ProseParagraph
One of the following applies:
\begin{itemize}
  \item All of the following apply (\textsc{greater\_or\_equal}):
  \begin{itemize}
    \item $\op$ is one of $\SHL$, $\SHR$, and $\POW$;
    \item define $\vf$ as the specialization of $\refineconstraintbysign$ for the predicate
          $\lambda x.\ x \geq 0$, which is $\True$ if and only if the tested number is greater or equal to $0$;
    \item refining the list of constraints $\cs$ with $\vf$ via $\refineconstraints$ yields $\newcs$;
    \item checking whether $\newcs$ is empty yields $\True$\ProseTerminateAs{\BadOperands}.
  \end{itemize}

  \item All of the following apply (\textsc{greater\_than}):
  \begin{itemize}
    \item $\op$ is one of $\MOD$, $\DIV$, and $\DIVRM$;
    \item define $\vf$ as the specialization of $\refineconstraintbysign$ for the predicate
          $\lambda x.\ x > 0$, which is $\True$ if and only if the tested number is greater than $0$;
    \item refining the list of constraints $\cs$ with $\vf$ via $\refineconstraints$ yields $\newcs$;
    \item checking whether $\newcs$ is empty yields $\True$\ProseTerminateAs{\BadOperands}.
  \end{itemize}

  \item All of the following apply (\textsc{no\_filtering}):
  \begin{itemize}
    \item $\op$ is one of $\MINUS$, $\MUL$, and $\PLUS$;
    \item $\newcs$ is $\cs$.
  \end{itemize}
\end{itemize}

\FormallyParagraph
\begin{mathpar}
\inferrule[greater\_or\_equal]{
  \op \in \{\SHL, \SHR, \POW\}\\\\
  \vf \eqdef \refineconstraintbysign(\tenv, \lambda x.\ x \geq 0)\\
  \refineconstraints(\cs, \vf) \typearrow \newcs\\
  \checktrans{\newcs \neq \emptylist}{\BadOperands} \typearrow \True\OrTypeError
}{
  \binopfilterrhs(\tenv, \op, \cs) \typearrow \newcs
}
\end{mathpar}

\begin{mathpar}
\inferrule[greater\_than]{
  \op \in \{\MOD, \DIV, \DIVRM\}\\\\
  \vf \eqdef \refineconstraintbysign(\tenv, \lambda x.\ x > 0)\\
  \refineconstraints(\cs, \vf) \typearrow \newcs\\
  \checktrans{\newcs \neq \emptylist}{\BadOperands} \typearrow \True\OrTypeError
}{
  \binopfilterrhs(\tenv, \op, \cs) \typearrow \newcs
}
\end{mathpar}

\begin{mathpar}
\inferrule[no\_filter]{
  \op \in \{\MINUS, \MUL, \PLUS\}
}{
  \binopfilterrhs(\op, \cs) \typearrow \overname{\cs}{\newcs}
}
\end{mathpar}

\TypingRuleDef{RefineConstraintBySign}
\hypertarget{def-refineconstraintbysign}{}
The function
\[
\refineconstraintbysign(\overname{\staticenvs}{\tenv} \aslsep \overname{\Z\rightarrow \Bool}{\vp} \aslsep \overname{\intconstraint}{\vc})
\aslto \overname{\langle\intconstraint\rangle}{\vcopt}
\]
takes a predicate $\vp$ that returns $\True$ based on the sign of its input.
The function conservatively refines the constraint $\vc$ in $\tenv$ by applying symbolic reasoning to yield a new constraint
(inside an optional)
that represents the values that satisfy the $\vc$ and for which $\vp$ holds.
In this context, conservatively means that the new constraint may represent a superset of the values that a more precise
reasoning may yield.
If the set of those values is empty the result is $\None$.

\ProseParagraph
One of the following applies:
\begin{itemize}
  \item All of the following apply (\textsc{exact\_reduces\_to\_z}):
  \begin{itemize}
    \item $\vc$ is an exact constraint for the expression $\ve$, that is, $\ConstraintExact(\ve)$;
    \item applying $\reducetozopt$ to $\ve$ in $\tenv$, in order to symbolically simplify $\ve$ to an integer,
          yields $\langle\vz\rangle$;
    \item $\vcopt$ is $\langle\vc\rangle$ if $\vp$ holds for $\vz$ and $\None$ otherwise.
  \end{itemize}

  \item All of the following apply (\textsc{exact\_does\_not\_reduce\_to\_z}):
  \begin{itemize}
    \item $\vc$ is an exact constraint for the expression $\ve$, that is, $\ConstraintExact(\ve)$;
    \item applying $\reducetozopt$ to $\ve$ in $\tenv$, in order to symbolically simplify $\ve$ to an integer,
          yields $\None$;
    \item $\vcopt$ is $\langle\vc\rangle$.
  \end{itemize}

  \item All of the following apply (\textsc{range\_both\_reduce\_to\_z}):
  \begin{itemize}
    \item $\vc$ is a range constraint for the expressions $\veone$ and $\vetwo$, that is, \\
          $\ConstraintRange(\veone, \vetwo)$;
    \item applying $\reducetozopt$ to $\veone$ in $\tenv$, in order to symbolically simplify $\veone$ to an integer,
          yields $\langle\vzone\rangle$;
    \item applying $\reducetozopt$ to $\vetwo$ in $\tenv$, in order to symbolically simplify $\vetwo$ to an integer,
          yields $\langle\vztwo\rangle$;
    \item One of the following applies (defining $\vcopt$):
    \begin{itemize}
      \item if $\vp$ is $\True$ for both $\vzone$ and $\vztwo$, define $\vcopt$ as $\langle\vc\rangle$;
      \item if $\vp$ is $\False$ for $\vzone$ and $\True$ for $\vztwo$, define $\vcopt$ as the optional range constraint
            where the bottom expression is the literal expression for $0$ if $\vp$ holds for $0$ and the literal expression for $1$ otherwise,
            and the top expression is $\vetwo$;
      \item if $\vp$ is $\True$ for $\vzone$ and $\False$ for $\vztwo$, define $\vcopt$ as the optional range constraint
            where the bottom expression is $\veone$ and the top expression is the literal expression for $0$ if $\vp$ holds for $0$
            and the literal expression for $-1$ otherwise;
      \item if $\vp$ is $\False$ for both $\vzone$ and $\vztwo$, define $\vcopt$ as $\None$.
    \end{itemize}
  \end{itemize}

  \item All of the following apply (\textsc{only\_e1\_reduces\_to\_z}):
  \begin{itemize}
    \item $\vc$ is a range constraint for the expressions $\veone$ and $\vetwo$, that is, \\
          $\ConstraintRange(\veone, \vetwo)$;
    \item applying $\reducetozopt$ to $\veone$ in $\tenv$, in order to symbolically simplify $\veone$ to an integer,
          yields $\langle\vzone\rangle$;
    \item applying $\reducetozopt$ to $\vetwo$ in $\tenv$, in order to symbolically simplify $\vetwo$ to an integer,
          yields $\None$;
    \item One of the following applies (defining $\vcopt$):
    \begin{itemize}
      \item if $\vp$ is $\True$ for $\vzone$, define $\vcopt$ as $\langle\vc\rangle$;
      \item if $\vp$ is $\False$ for $\vzone$, define $\vcopt$ as the optional range constraint with the bottom expression
            as the literal expression for $0$ if $\vp$ holds for $0$ and the literal expression for $1$ otherwise,
            and the top expression $\vetwo$.
    \end{itemize}
  \end{itemize}

  \item All of the following apply (\textsc{only\_e2\_reduces\_to\_z}):
  \begin{itemize}
    \item $\vc$ is a range constraint for the expressions $\veone$ and $\vetwo$, that is, \\
          $\ConstraintRange(\veone, \vetwo)$;
    \item applying $\reducetozopt$ to $\veone$ in $\tenv$, in order to symbolically simplify $\veone$ to an integer,
          yields $\None$;
    \item applying $\reducetozopt$ to $\vetwo$ in $\tenv$, in order to symbolically simplify $\vetwo$ to an integer,
          yields $\langle\vztwo\rangle$;
    \item One of the following applies (defining $\vcopt$):
    \begin{itemize}
      \item if $\vp$ is $\True$ for $\vztwo$, define $\vcopt$ as $\langle\vc\rangle$;
      \item if $\vp$ is $\False$ for $\vztwo$, define $\vcopt$ as the optional range constraint with the bottom expression
            $\veone$ and the top expression the literal expression for $0$ if $\vp$ holds for $0$ and the literal expression for $-1$ otherwise.
    \end{itemize}
  \end{itemize}
\end{itemize}

\FormallyParagraph
\begin{mathpar}
\inferrule[exact\_reduces\_to\_z]{
  \reducetozopt(\tenv, \ve) \typearrow \langle\vz\rangle\\
  \vcopt \eqdef \choice{\vp(\vz)}{\langle\vc\rangle}{\None}
}{
  \refineconstraintbysign(\tenv, \vp, \overname{\ConstraintExact(\ve)}{\vc}) \typearrow \vcopt
}
\end{mathpar}

\begin{mathpar}
\inferrule[exact\_does\_not\_reduce\_to\_z]{
  \reducetozopt(\tenv, \ve) \typearrow \None
}{
  \refineconstraintbysign(\tenv, \vp, \overname{\ConstraintExact(\ve)}{\vc}) \typearrow \overname{\langle\vc\rangle}{\vcopt}
}
\end{mathpar}

\begin{mathpar}
\inferrule[range\_both\_reduce\_to\_z]{
  \reducetozopt(\tenv, \veone) \typearrow \langle\vzone\rangle\\
  \reducetozopt(\tenv, \vetwo) \typearrow \langle\vztwo\rangle\\
  {
    \begin{array}{c}
  \vcopt \eqdef \\ \wrappedline\ \begin{cases}
    \langle\vc\rangle& \text{if }\vp(\vzone) \land \vp(\vztwo)\\
    \langle\ConstraintRange(\choice{\vp(0)}{\ELInt{0}}{\ELInt{1}}, \vetwo)\rangle& \text{if }\neg\vp(\vzone) \land \vp(\vztwo)\\
    \langle\ConstraintRange(\veone, \choice{\vp(0)}{\ELInt{0}}{\ELInt{-1}})\rangle& \text{if }\vp(\vzone) \land \neg\vp(\vztwo)\\
    \None& \text{if }\neg\vp(\vzone) \land \neg\vp(\vztwo)\\
  \end{cases}
\end{array}
  }
}{
  \refineconstraintbysign(\tenv, \vp, \overname{\ConstraintRange(\veone, \vetwo)}{\vc}) \typearrow \vcopt
}
\end{mathpar}

\begin{mathpar}
\inferrule[only\_e1\_reduces\_to\_z]{
  \reducetozopt(\tenv, \veone) \typearrow \langle\vzone\rangle\\
  \reducetozopt(\tenv, \vetwo) \typearrow \None\\
  {
    \begin{array}{c}
  \vcopt \eqdef \\ \wrappedline\ \begin{cases}
    \langle\vc\rangle& \text{if }\vp(\vzone)\\
    \langle\ConstraintRange(\choice{\vp(0)}{\ELInt{0}}{\ELInt{1}}, \vetwo)\rangle& \text{else}\\
  \end{cases}
\end{array}
  }
}{
  \refineconstraintbysign(\tenv, \vp, \overname{\ConstraintRange(\veone, \vetwo)}{\vc}) \typearrow \vcopt
}
\end{mathpar}

\begin{mathpar}
\inferrule[only\_e2\_reduces\_to\_z]{
  \reducetozopt(\tenv, \veone) \typearrow \None\\
  \reducetozopt(\tenv, \vetwo) \typearrow \langle\vztwo\rangle\\
  {
    \begin{array}{c}
  \vcopt \eqdef \\ \wrappedline\ \begin{cases}
    \langle\vc\rangle& \text{if }\vp(\vztwo)\\
    \langle\ConstraintRange(\veone, \choice{\vp(0)}{\ELInt{0}}{\ELInt{-1}})\rangle& \text{else}\\
  \end{cases}
\end{array}
  }
}{
  \refineconstraintbysign(\tenv, \vp, \overname{\ConstraintRange(\veone, \vetwo)}{\vc}) \typearrow \vcopt
}
\end{mathpar}

\begin{mathpar}
\inferrule[none\_reduce\_to\_z]{
  \reducetozopt(\tenv, \veone) \typearrow \None\\
  \reducetozopt(\tenv, \vetwo) \typearrow \None
}{
  \refineconstraintbysign(\tenv, \vp, \overname{\ConstraintRange(\veone, \vetwo)}{\vc}) \typearrow \overname{\vc}{\vcopt}
}
\end{mathpar}

\TypingRuleDef{ReduceToZOpt}
\hypertarget{def-reducetozopt}{}
The function
\[
\reducetozopt(\overname{\staticenvs}{\tenv} \aslsep \overname{\expr}{\ve})
\aslto \overname{\langle\Z\rangle}{\vzopt}
\]
returns an integer inside an optional if $\ve$ can be symbolically simplified into an integer in $\tenv$
and $\None$ otherwise.
The expression $\ve$ is assumed to appear in a constraint for a type that has been successfully annotated,
which means that applying $\normalize$ to it should not yield a type error.

\ProseParagraph
One of the following applies:
\begin{itemize}
  \item All of the following apply (\textsc{normalizes\_to\_z}):
  \begin{itemize}
    \item symbolically simplifying $\ve$ in $\tenv$ via $\normalize$ yields a literal expression for the integer $\vz$;
    \item define $\vzopt$ as $\langle\vz\rangle$.
  \end{itemize}

  \item All of the following apply (\textsc{does\_not\_normalize\_to\_z}):
  \begin{itemize}
    \item symbolically simplifying $\ve$ in $\tenv$ via $\normalize$ yields an expression that is not an integer literal;
    \item define $\vzopt$ as $\None$.
  \end{itemize}
\end{itemize}

\FormallyParagraph
\begin{mathpar}
\inferrule[normalizes\_to\_z]{
  \normalize(\tenv, \ve) \typearrow \ELInt{\vz}
}{
  \reducetozopt(\tenv, \ve) \typearrow \overname{\langle\vz\rangle}{\vzopt}
}
\end{mathpar}

\begin{mathpar}
\inferrule[does\_not\_normalize\_to\_z]{
  \normalize(\tenv, \ve) \typearrow \vep\\
  \forall \vz\in\Z.\ \vep \neq \ELInt{\vz}
}{
  \reducetozopt(\tenv, \ve) \typearrow \overname{\None}{\vzopt}
}
\end{mathpar}

\TypingRuleDef{RefineConstraints}
\hypertarget{def-refineconstraints}{}
The function
\[
\begin{array}{r}
\refineconstraints(\overname{\staticenvs}{\tenv} \aslsep \overname{\intconstraint\rightarrow\langle\intconstraint\rangle}{\vf} \aslsep \overname{\intconstraint^*}{\cs})
\aslto \\
\overname{\intconstraint^*}{\newcs}
\end{array}
\]
refines a list of constraints $\cs$ by applying the refinement function $\vf$ to each constraint and retaining the constraints
that do not refine to $\None$. The resulting list of constraints is given in $\newcs$.

\ProseParagraph
One of the following applies:
\begin{itemize}
  \item All of the following apply (\textsc{empty}):
  \begin{itemize}
    \item $\cs$ is the empty list;
    \item $\newcs$ is the empty list.
  \end{itemize}

  \item All of the following apply (\textsc{non\_empty\_none}):
  \begin{itemize}
    \item $\cs$ is the list with $\vc$ as its \head\ and $\csone$ as its \tail;
    \item applying $\vf$ to $\vc$ yields $\None$;
    \item applying $\refineconstraints$ to $\vf$ and $\csone$ yields $\csonep$;
    \item $\newcs$ is $\csonep$.
  \end{itemize}

  \item All of the following apply (\textsc{non\_empty\_somee}):
  \begin{itemize}
    \item $\cs$ is the list with $\vc$ as its \head\ and $\csone$ as its \tail;
    \item applying $\vf$ to $\vc$ yields $\langle\vcp\rangle$;
    \item applying $\refineconstraints$ to $\vf$ and $\csone$ yields $\csonep$;
    \item $\newcs$ is the list with $\vcp$ as its \head\ and $\csonep$ as its \tail.
  \end{itemize}
\end{itemize}

\FormallyParagraph
\begin{mathpar}
\inferrule[empty]{}{
  \refineconstraints(\tenv, \vf, \overname{\emptylist}{\cs}) \typearrow \overname{\emptylist}{\newcs}
}
\end{mathpar}

\begin{mathpar}
\inferrule[non\_empty\_none]{
  \vf(\vc) \typearrow \None\\
  \refineconstraints(\vf, \csone) \typearrow \csonep\\
}{
  \refineconstraints(\vf, \overname{[\vc]\concat \csone}{\cs}) \typearrow \overname{\csonep}{\newcs}
}
\end{mathpar}

\begin{mathpar}
\inferrule[non\_empty\_some]{
  \vf(\vc) \typearrow \langle\vcp\rangle\\
  \refineconstraints(\vf, \csone) \typearrow \csonep\\
}{
  \refineconstraints(\vf, \overname{[\vc]\concat \csone}{\cs}) \typearrow \overname{[\vcp] \concat \csonep}{\newcs}
}
\end{mathpar}

\TypingRuleDef{RefineConstraintForDiv}
\hypertarget{def-refineconstraintfordiv}{}
The function
\[
\refineconstraintfordiv(\overname{\binop}{\op} \aslsep \overname{\intconstraint^*}{\cs}) \aslto \overname{\intconstraint^*}{\vres}
  \cup \overname{\TTypeError}{\TypeErrorConfig}
\]
filters the list of constraints $\cs$ for $\op$,
removing constraints that represents a division operation that will definitely fail
when $\op$ is the division operation.
\ProseOtherwiseTypeError

\ProseParagraph
One of the following applies:
\begin{itemize}
  \item All of the following apply (\textsc{div}):
  \begin{itemize}
    \item $\op$ is $\DIV$;
    \item applying $\filterreduceconstraintdiv$ to each constraint $\cs[\vi]$, for each $\vi$ in $\listrange(\cs)$,
          yields the optional constraint $\vcopt_\vi$\ProseOrTypeError;
    \item define $\vres$ as the list made of constraints $\vcp_\vi$, for each $\vi$ in $\listrange(\cs)$
          such that $\vcopt_\vi = \langle\vcp_\vi\rangle$;
    \item checking that $\vres$ is not the empty list yields $\True$\ProseTerminateAs{\BadOperands}.
  \end{itemize}

  \item All of the following apply (\textsc{non\_div}):
  \begin{itemize}
    \item $\op$ is not $\DIV$;
    \item define $\vres$ as $\cs$.
  \end{itemize}
\end{itemize}

\FormallyParagraph
\begin{mathpar}
\inferrule[div]{
  \op = \DIV\\
  \vi\in\listrange(\cs): \filterreduceconstraintdiv(\cs[\vi]) \typearrow \vcopt_\vi \OrTypeError\\\\
  \vres \eqdef [\vi\in\listrange(\cs): \choice{\vcopt_\vi = \langle\vcp_\vi\rangle}{\vcp}{\epsilon}]\\
  \checktrans{\vres \neq \emptylist}{\BadOperands} \checktransarrow \True \OrTypeError
}{
  \refineconstraintfordiv(\op, \cs) \typearrow \vres
}
\end{mathpar}

\begin{mathpar}
\inferrule[non\_div]{
  \op \neq \DIV
}{
  \refineconstraintfordiv(\op, \cs) \typearrow \overname{\cs}{\vres}
}
\end{mathpar}
\CodeSubsection{\RefineConstraintForDIVBegin}{\RefineConstraintForDIVEnd}{../types.ml}

\TypingRuleDef{FilterReduceConstraintDiv}
\hypertarget{def-filterreduceconstraintdiv}{}
The function
\[
\filterreduceconstraintdiv(\overname{\intconstraint}{\vc}) \aslto \overname{\langle\intconstraint\rangle}{\vcopt}
\]
returns $\None$ if $\vc$ is an exact constraint for a binary expression for dividing two integer literals
where the denominator does not divide the numerator and an optional containing $\vc$.
The result is returned in $\vcopt$.
This is used to conservatively test whether $\vc$ would always fail dynamically.

\ProseParagraph
One of the following applies:
\begin{itemize}
  \item All of the following apply (\textsc{exact}):
  \begin{itemize}
    \item $\vc$ is an exact constraint for the expression $\ve$, that is, $\ConstraintExact(\ve)$;
    \item applying $\getliteraldivopt$ to $\ve$ yields $\langle(\vzone, \vztwo)\rangle$\ProseTerminateAs{\None};
    \item define $\vcopt$ as follows:
    \begin{itemize}
      \item $\None$, if $\vztwo$ is positive and $\vztwo$ does not divide $\vzone$;
      \item $\langle\vc\rangle$, otherwise.
    \end{itemize}
  \end{itemize}

  \item All of the following apply (\textsc{range}):
  \begin{itemize}
    \item $\vc$ is a range constraint for $\veone$ and $\vetwo$, that is, $\ConstraintRange(\veone, \vetwo)$;
    \item applying $\getliteraldivopt$ to $\veone$ yields $\veoneopt$;
    \item define $\vzoneopt$ as follows:
    \begin{itemize}
      \item $\vzone$ divided by $\vztwo$ and rounded up, if $\veoneopt$ is $(\vzone, \vztwo)$ and $\vztwo$ is positive;
      \item $\None$, otherwise.
    \end{itemize}
    \item applying $\getliteraldivopt$ to $\vetwo$ yields $\vetwoopt$;
    \item define $\vztwoopt$ as follows:
    \begin{itemize}
      \item $\vzthree$ divided by $\vzfour$ and rounded down, if $\vetwoopt$ is $(\vzthree, \vzfour)$ and $\vzfour$ is positive;
      \item $\None$, otherwise.
    \end{itemize}
    \item define $\vcopt$ as follows:
    \begin{itemize}
      \item the exact constraint for the literal integer $\vzfive$, if $\vzoneopt$ is $\langle\vzfive\rangle$ and $\vztwoopt$ is $\langle\vzsix\rangle$ and $\vzfive$ is equal to $\vzsix$;
      \item the range constraint for the literal integer $\vzfive$ and $\vzsix$, if $\vzoneopt$ is $\langle\vzfive\rangle$ and $\vztwoopt$ is $\langle\vzsix\rangle$ and $\vzfive$ is less than $\vzsix$;
      \item $\None$, if $\vzoneopt$ is $\langle\vzfive\rangle$ and $\vztwoopt$ is $\langle\vzsix\rangle$ and $\vzfive$ is greater than $\vzsix$;
      \item the range constraint for the literal integer $\vzfive$ and $\vetwo$, if $\vzoneopt$ is $\langle\vzfive\rangle$ and $\vztwoopt$ is $\None$;
      \item the range constraint for $\veone$ and the literal integer $\vzsix$, if $\vzoneopt$ is $\None$ and $\vztwoopt$ is $\langle\vzsix\rangle$;
      \item $\vc$ if $\vzoneopt$ is $\None$ and $\vztwoopt$ is $\None$.
    \end{itemize}
  \end{itemize}
\end{itemize}

\FormallyParagraph
\begin{mathpar}
\inferrule[exact]{
  \getliteraldivopt(\ve) \typearrow \langle(\vzone, \vztwo)\rangle \terminateas \None\\\\
  {
    \vcopt \eqdef
    \begin{cases}
      \None & \text{if }\vztwo > 0 \land \frac{\vzone}{\vztwo} \not\in \Z\\
      \langle\vc\rangle & \text{else}
    \end{cases}
  }
}{
  \filterreduceconstraintdiv(\tenv, \overname{\ConstraintExact(\ve)}{\vc}) \typearrow \vcopt
}
\end{mathpar}

\begin{mathpar}
\inferrule[range]{
  \getliteraldivopt(\veone) \typearrow \veoneopt\\
  {
    \vzoneopt \eqdef
    \begin{cases}
      \left\lceil\frac{\vzone}{\vztwo}\right\rceil & \text{if }\veoneopt = \langle(\vzone, \vztwo)\rangle \land \vztwo > 0\\
      \None & \text{else}
    \end{cases}
  }\\
  \getliteraldivopt(\vetwo) \typearrow \vetwoopt\\
  {
    \vztwoopt \eqdef
    \begin{cases}
      \left\lfloor\frac{\vzthree}{\vzfour}\right\rfloor & \text{if }\vetwoopt = \langle(\vzthree, \vzfour)\rangle \land \vzfour > 0\\
      \None & \text{else}
    \end{cases}
  }\\
  {
    \vcopt \eqdef
    \begin{cases}
     \langle\AbbrevConstraintExact{\ELInt{\vzfive}}\rangle & \text{if }\vzoneopt = \langle\vzfive\rangle \land \vztwoopt = \langle\vzsix\rangle \land \vzfive=\vzsix\\
     \langle\AbbrevConstraintRange{\ELInt{\vzfive}}{\ELInt{\vzsix}}\rangle & \text{if }\vzoneopt = \langle\vzfive\rangle \land \vztwoopt = \langle\vzsix\rangle \land \vzfive<\vzsix\\
     \None & \text{if }\vzoneopt = \langle\vzfive\rangle \land \vztwoopt = \langle\vzsix\rangle \land \vzfive>\vzsix\\
     \langle\AbbrevConstraintRange{\ELInt{\vzfive}}{\vetwo}\rangle & \text{if }\vzoneopt = \langle\vzfive\rangle \land \vztwoopt = \None\\
     \langle\AbbrevConstraintRange{\veone}{\ELInt{\vzsix}}\rangle & \text{if }\vzoneopt = \None \land \vztwoopt = \langle\vzsix\rangle\\
     \langle\ConstraintRange(\veone, \vetwo)\rangle & \text{if }\vzoneopt = \None \land \vztwoopt = \None\\
    \end{cases}
  }
}{
  \filterreduceconstraintdiv(\tenv, \overname{\ConstraintRange(\veone, \vetwo)}{\vc}) \typearrow \overname{\langle\vc\rangle}{\vcopt}
}
\end{mathpar}

\TypingRuleDef{GetLiteralDivOpt}
\hypertarget{def-getliteraldivopt}{}
The function
\[
\getliteraldivopt(\overname{\expr}{\ve}) \aslto \overname{\langle\Z\cartimes\Z\rangle}{\rangeopt}
\]
matches the expression $\ve$ to a binary operation expression over the division operation and two literal integer expressions.
If $\ve$ matches this pattern the result $\rangeopt$ is an optional containing the pair of integers appearing in the operand
expressions. Otherwise, the result is $\None$.

\ProseParagraph
The value $\rangeopt$ is $\langle(\vzone, \vztwo)\rangle$ if $\ve$ is a binary operation expression over the division operation
and two literal integer expressions for the integers $\vzone$ and $\vztwo$ and $\None$ otherwise.

\FormallyParagraph
\begin{mathpar}
\inferrule{
  \rangeopt \eqdef \choice{\ve = \EBinop(\DIV, \ELInt{\vzone}, \ELInt{\vztwo})}{\langle(\vzone, \vztwo)\rangle}{\None}
}{
  \getliteraldivopt(\ve) \typearrow \rangeopt
}
\end{mathpar}

\TypingRuleDef{ExplodeIntervals}
\hypertarget{def-explodeintervals}{}
The function
\[
\explodeintervals(\overname{\staticenvs}{\tenv} \aslsep \overname{\intconstraint^*}{\cs})
\aslto \overname{\intconstraint^*}{\newcs}
\]
applies $\explodedinterval$ to each constraint of $\cs$ in $\tenv$ and concatenates the resulting
list, yielding the result in $\newcs$.

\ProseParagraph
One of the following applies:
\begin{itemize}
  \item All of the following apply (\textsc{empty}):
  \begin{itemize}
    \item $\cs$ is the empty list;
    \item $\newcs$ is the empty list.
  \end{itemize}

  \item All of the following apply (\textsc{non\_empty}):
  \begin{itemize}
    \item $\cs$ is the list with $\vc$ as its \head\ and $\csone$ as its \tail;
    \item applying $\explodeconstraint$ to $\vc$ in $\tenv$ yields $\vcp$ (a list of constraints);
    \item applying $\explodeintervals$ to $\csone$ in $\tenv$ yields $\csonep$;
    \item $\newcs$ is the concatenation of $\vcp$ and $\csonep$.
  \end{itemize}
\end{itemize}

\FormallyParagraph
\begin{mathpar}
\inferrule[empty]{}{
  \explodeintervals(\tenv, \overname{\emptylist}{\cs}) \typearrow \overname{\emptylist}{\newcs}
}
\end{mathpar}

\begin{mathpar}
\inferrule[non\_empty]{
  \explodeconstraint(\tenv, \vc) \typearrow \vcp\\
  \explodeintervals(\tenv, \csone) \typearrow \csonep\\
}{
  \explodeintervals(\tenv, \overname{[\vc] \concat \csone}{\cs}) \typearrow \overname{\vcp \concat \csonep}{\newcs}
}
\end{mathpar}

\TypingRuleDef{ExplodeConstraint}
\hypertarget{def-explodeconstraint}{}
The function
\[
\explodeconstraint(\overname{\staticenvs}{\tenv} \aslsep \overname{\intconstraint}{\vc})
\aslto \overname{\intconstraint^*}{\vcs}
\]
expands the constraint $\vc$ into the equivalent list of exact constraints if
$\vc$ matches a n ascending range constraint that is not too large in $\tenv$
and the singleton list for $\vc$ otherwise.

\ProseParagraph
One of the following applies:
\begin{itemize}
  \item All of the following apply (\textsc{exact}):
  \begin{itemize}
    \item $\vc$ is an exact constraint;
    \item $\vcs$ is the singleton list for $\vc$.
  \end{itemize}

  \item All of the following apply (\textsc{range\_reduced}):
  \begin{itemize}
    \item $\vc$ is a range constraint for the expressions $\va$ and $\vb$;
    \item applying $\reducetozopt$ to $\va$ in $\tenv$ yields $\langle\vza\rangle$;
    \item applying $\reducetozopt$ to $\vb$ in $\tenv$ yields $\langle\vzb\rangle$;
    \item define $\explodedinterval$ as the list of exact constraints for each integer literal in the range starting
          at $\vza$ and ending at $\vzb$, inclusively, which is empty if $\vzb < \vza$;
    \item applying $\intervaltoolarge$ to $\vza$ and $\vzb$ yields $\vbtoolarge$;
    \item define $\vcs$ as the singleton list for $\vc$ if $\vbtoolarge$ is $\True$ and \\
          $\explodedinterval$ otherwise.
  \end{itemize}

  \item All of the following apply (\textsc{range\_not\_reduced}):
  \begin{itemize}
    \item $\vc$ is a range constraint for the expressions $\va$ and $\vb$;
    \item applying $\reducetozopt$ to $\va$ in $\tenv$ yields $\vzaopt$;
    \item applying $\reducetozopt$ to $\vb$ in $\tenv$ yields $\vzbopt$;
    \item at least one of $\vzaopt$ and $\vzbopt$ is $\None$;
    \item $\vcs$ is the singleton list for $\vc$.
  \end{itemize}
\end{itemize}

\FormallyParagraph
\begin{mathpar}
\inferrule[exact]{
  \astlabel(\vc) = \ConstraintExact
}{
  \explodeconstraint(\tenv, \vc) \typearrow \overname{[\vc]}{\vcs}
}
\end{mathpar}

\begin{mathpar}
\inferrule[range\_reduced]{
  \vc = \ConstraintRange(\va, \vb)\\
  \reducetozopt(\tenv, \va) \typearrow \langle\vza\rangle\\
  \reducetozopt(\tenv, \vb) \typearrow \langle\vzb\rangle\\
  \explodedinterval \eqdef [\vz \in \vza..\vzb: \ConstraintExact(\ELInt{\vz})]\\
  \intervaltoolarge(\vza, \vzb) \typearrow \vbtoolarge\\
  \vcs \eqdef \choice{\vbtoolarge}{[\vc]}{\explodedinterval}
}{
  \explodeconstraint(\tenv, \vc) \typearrow \vcs
}
\end{mathpar}

\begin{mathpar}
\inferrule[range\_not\_reduced]{
  \vc = \ConstraintRange(\va, \vb)\\
  \reducetozopt(\tenv, \va) \typearrow \vzaopt\\
  \reducetozopt(\tenv, \vb) \typearrow \vzbopt\\
  \vzaopt = \None \lor \vzbopt = \None
}{
  \explodeconstraint(\tenv, \vc) \typearrow \overname{[\vc]}{\vcs}
}
\end{mathpar}

\TypingRuleDef{IntervalTooLarge}
\hypertarget{def-intervaltoolarge}{}
The function
\[
\intervaltoolarge(\overname{\Z}{\vzone} \aslsep \overname{\Z}{\vztwo}) \aslto \overname{\Bool}{\vb}
\]
tests whether the set of numbers between $\vzone$ and $\vztwo$, inclusive, contains more than $\maxexplodedintervalsize$
integers, yielding the result in $\vb$.

\ProseParagraph
The value $\vb$ is $\True$ if and only if the absolute value of $\vzone-\vztwo$ is greater than $\maxexplodedintervalsize$.

\FormallyParagraph
\begin{mathpar}
\inferrule{}{
  \intervaltoolarge(\vzone, \vztwo) \typearrow \overname{\vztwo-\vzone > \maxexplodedintervalsize}{\vb}
}
\end{mathpar}

\TypingRuleDef{BinopIsExploding}
\hypertarget{def-binopisexploding}{}
The function
\[
\binopisexploding(\overname{\binop}{\op}) \aslto \overname{\Bool}{\vb}
\]
determines whether the binary operation $\op$ should lead to applying $\explodeintervals$
when the $\op$ is applied to a pair of constraint lists.
It is assumed that $\op$ is one of $\MUL$, $\SHL$, $\POW$, $\PLUS$, $\DIV$, $\MINUS$, $\MOD$, $\SHR$,
and $\DIVRM$.

\ProseParagraph
The value $\vb$ is $\True$ if and only if $\op$ is one of $\MUL$, $\SHL$, and $\POW$.

\FormallyParagraph
\begin{mathpar}
\inferrule{}{
  \binopisexploding(\op) \typearrow \overname{\op \in \{\MUL, \SHL, \POW, \DIV, \DIVRM, \MOD, \SHR\}}{\vb}
}
\end{mathpar}

\TypingRuleDef{BitFieldsIncluded}
\hypertarget{def-bitfieldsincluded}{}
The predicate
\[
  \bitfieldsincluded(\overname{\staticenvs}{\tenv}, \overname{\bitfield^*}{\bfsone} \aslsep \overname{\bitfield^*}{\bfstwo})
  \aslto \overname{\Bool}{\vb} \cup \overname{\TTypeError}{\TypeErrorConfig}
\]
tests whether the set of bit fields $\bfsone$ is included in the set of bit fields $\bfstwo$ in environment $\tenv$,
returning a type error, if one is detected.

\ProseParagraph
All of the following apply:
\begin{itemize}
  \item checking whether each field $\vbf$ in $\bfsone$ exists in $\bfstwo$ via $\membfs$ yields $\vb_\vbf$\ProseOrTypeError;
  \item the result --- $\vb$ --- is the conjunction of $\vb_\vbf$ for all bitfields $\vbf$ in $\bfsone$.
\end{itemize}

\FormallyParagraph
\begin{mathpar}
\inferrule{
  \vbf \in \bfsone: \membfs(\bfstwo, \vbf) \typearrow \vb_\vbf \OrTypeError\\\\
  \vbf \eqdef \bigwedge_{\bf \in \bfsone} \vb_\vbf
}{
  \bitfieldsincluded(\tenv, \bfsone, \bfstwo) \typearrow \vb
}
\end{mathpar}

\TypingRuleDef{MemBfs}
\hypertarget{def-membfs}{}
The function
\[
  \membfs(\overname{\staticenvs}{\tenv} \aslsep \overname{\bitfield^+}{\bfstwo} \aslsep \overname{\bitfield}{\vbfone})
  \aslto \overname{\Bool}{\vb}
\]
checks whether the bitfield $\vbf$ exists in $\bfstwo$ in the context of $\tenv$, returning the result in $\vb$.

\ProseParagraph
One of the following applies:
\begin{itemize}
  \item All of the following apply (\textsc{none}):
  \begin{itemize}
    \item the name associated with the bitfield $\vbfone$ is $\name$;
    \item finding the bitfield associated with $\name$ in $\bfstwo$ yields $\None$;
    \item $\vb$ is $\False$.
  \end{itemize}

  \item All of the following apply (\textsc{simple\_any}):
  \begin{itemize}
    \item the name associated with the bitfield $\vbfone$ is $\name$;
    \item finding the bitfield associated with $\name$ in $\bfstwo$ yields $\vbftwo$;
    \item $\vbftwo$ is a simple bitfield;
    \item symbolically checking whether $\vbfone$ is equivalent to $\vbftwo$ in $\tenv$ yields $\vb$.
  \end{itemize}

  \item All of the following apply (\textsc{nested\_simple}):
  \begin{itemize}
    \item the name associated with the bitfield $\vbfone$ is $\name$;
    \item finding the bitfield associated with $\name$ in $\bfstwo$ yields $\vbftwo$;
    \item $\vbftwo$ is a nested bitfield with name $\nametwo$, slices $\slicestwo$, and bitfields $\bfstwop$;
    \item $\vbfone$ is a simple bitfield;
    \item symbolically checking whether $\vbfone$ is equivalent to $\vbftwo$ in $\tenv$ yields $\vb$.
  \end{itemize}

  \item All of the following apply (\textsc{nested\_nested}):
  \begin{itemize}
    \item the name associated with the bitfield $\vbfone$ is $\name$;
    \item finding the bitfield associated with $\name$ in $\bfstwo$ yields $\vbftwo$;
    \item $\vbftwo$ is a nested bitfield with name $\nametwo$, slices $\slicestwo$, and bitfields $\bfstwop$;
    \item $\vbfone$ is a nested bitfield with name $\nameone$, slices $\sliceone$, and $\bfsone$;
    \item $\vbone$ is true if and only if $\nameone$ is equal to $\nametwo$;
    \item symbolically equating the slices $\slicesone$ and $\slicestwo$ in $\tenv$ yields $\vbtwo$;
    \item checking $\bfsone$ is included in $\bfstwop$ in the context of $\tenv$ yields $\vbthree$;
    \item $\vb$ is defined as the conjunction of $\vbone$, $\vbtwo$, and $\vbthree$.
  \end{itemize}

  \item All of the following apply (\textsc{nested\_typed}):
  \begin{itemize}
    \item the name associated with the bitfield $\vbfone$ is $\name$;
    \item finding the bitfield associated with $\name$ in $\bfstwo$ yields $\vbftwo$;
    \item $\vbftwo$ is a nested bitfield with name $\nametwo$, slices $\slicestwo$, and bitfields $\bfstwop$;
    \item $\vbfone$ is a typed bitfield;
    \item $\vb$ is $\False$.
  \end{itemize}

  \item All of the following apply (\textsc{typed\_simple}):
  \begin{itemize}
    \item the name associated with the bitfield $\vbfone$ is $\name$;
    \item finding the bitfield associated with $\name$ in $\bfstwo$ yields $\vbftwo$;
    \item $\vbftwo$ is a typed bitfield with name $\nametwo$, slices $\slicestwo$, and type $\ttytwo$;
    \item $\vbfone$ is a simple bitfield;
    \item symbolically checking whether $\vbfone$ is equivalent to $\vbftwo$ in $\tenv$ yields $\vb$.
  \end{itemize}

  \item All of the following apply (\textsc{typed\_nested}):
  \begin{itemize}
    \item the name associated with the bitfield $\vbfone$ is $\name$;
    \item finding the bitfield associated with $\name$ in $\bfstwo$ yields $\vbftwo$;
    \item $\vbftwo$ is a typed bitfield with name $\nametwo$, slices $\slicestwo$, and type $\ttytwo$;
    \item $\vbfone$ is a nested bitfield;
    \item $\vb$ is $\False$.
  \end{itemize}

  \item All of the following apply (\textsc{typed\_typed}):
  \begin{itemize}
    \item the name associated with the bitfield $\vbfone$ is $\name$;
    \item finding the bitfield associated with $\name$ in $\bfstwo$ yields $\vbftwo$;
    \item $\vbftwo$ is a typed bitfield with name $\nametwo$, slices $\slicestwo$, and type $\ttytwo$;
    \item $\vbfone$ is a typed bitfield with name $\nameone$, slices $\slicesone$, and type $\ttyone$;
    \item $\vbone$ is true if and only if $\nameone$ is equal to $\nametwo$;
    \item symbolically equating the slices $\slicesone$ and $\slicestwo$ in $\tenv$ yields $\vbtwo$;
    \item checking whether $\ttyone$ subtypes $\ttytwo$ in $\tenv$ yields $\vbthree$;
    \item $\vb$ is defined as the conjunction of $\vbone$, $\vbtwo$, and $\vbthree$.
  \end{itemize}
\end{itemize}

\FormallyParagraph
\begin{mathpar}
\inferrule[none]{
  \bitfieldgetname(\vbfone) \typearrow \name\\
  \findbitfieldopt(\name, \bfstwo) \typearrow \None
}{
  \membfs(\tenv, \bfstwo, \vbfone) \typearrow \False
}
\and
\inferrule[simple\_any]{
  \bitfieldgetname(\vbf) \typearrow \name\\
  \findbitfieldopt(\name, \bfstwo) \typearrow \langle \vbftwo \rangle\\
  \astlabel(\vbftwo) = \BitFieldSimple\\
  \bitfieldsequal(\tenv, \vbfone, \vbftwo) \typearrow \vb
}{
  \membfs(\tenv, \bfstwo, \vbfone) \typearrow \vb
}
\end{mathpar}

\begin{mathpar}
\inferrule[nested\_simple]{
  \bitfieldgetname(\vbf) \typearrow \name\\
  \findbitfieldopt(\name, \bfstwo) \typearrow \langle \vbftwo \rangle\\
  \vbftwo = \BitFieldNested(\nametwo, \slicestwo, \bfstwop)\\
  \vbfone = \BitFieldSimple(\Ignore)\\
  \bitfieldsequal(\tenv, \vbfone, \vbftwo) \typearrow \vb
}{
  \membfs(\tenv, \bfstwo, \vbfone) \typearrow \overname{\False}{\vb}
}
\and
\inferrule[nested\_nested]{
  \bitfieldgetname(\vbf) \typearrow \name\\
  \findbitfieldopt(\name, \bfstwo) \typearrow \langle \vbftwo \rangle\\
  \vbftwo = \BitFieldNested(\nametwo, \slicestwo, \bfstwop)\\
  \vbfone = \BitFieldNested(\nameone, \slicesone, \bfsone)\\
  \vbone \eqdef \nameone = \nametwo\\
  \slicesequal(\tenv, \slicesone, \slicestwo) \typearrow \vbtwo\\
  \bitfieldsincluded(\tenv, \bfsone, \bfstwop) \typearrow \vbthree\\
  \vb \eqdef \vbone \land \vbtwo \land \vbthree
}{
  \membfs(\tenv, \bfstwo, \vbfone) \typearrow \vb
}
\and
\inferrule[nested\_typed]{
  \bitfieldgetname(\vbf) \typearrow \name\\
  \findbitfieldopt(\name, \bfstwo) \typearrow \langle \vbftwo \rangle\\
  \vbftwo = \BitFieldNested(\nametwo, \slicestwo, \bfstwop)\\
  \astlabel(\vbfone) = \BitFieldType
}{
  \membfs(\tenv, \bfstwo, \vbfone) \typearrow \overname{\False}{\vb}
}
\end{mathpar}

\begin{mathpar}
\inferrule[typed\_simple]{
  \bitfieldgetname(\vbf) \typearrow \name\\
  \findbitfieldopt(\name, \bfstwo) \typearrow \langle \vbftwo \rangle\\
  \vbftwo = \BitFieldType(\nametwo, \slicestwo, \ttytwo)\\
  \vbfone = \BitFieldSimple(\Ignore)\\
  \bitfieldsequal(\tenv, \vbfone, \vbftwo) \typearrow \vb
}{
  \membfs(\tenv, \bfstwo, \vbfone) \typearrow \vb
}
\and
\inferrule[typed\_nested]{
  \bitfieldgetname(\vbf) \typearrow \name\\
  \findbitfieldopt(\name, \bfstwo) \typearrow \langle \vbftwo \rangle\\
  \vbftwo = \BitFieldType(\nametwo, \slicestwo, \ttytwo)\\
  \astlabel(\vbfone) = \BitFieldNested
}{
  \membfs(\tenv, \bfstwo, \vbfone) \typearrow \overname{\False}{\vb}
}
\and
\inferrule[typed\_typed]{
  \bitfieldgetname(\vbf) \typearrow \name\\
  \findbitfieldopt(\name, \bfstwo) \typearrow \langle \vbftwo \rangle\\
  \vbftwo = \BitFieldType(\nametwo, \slicestwo, \ttytwo)\\
  \vbfone = \BitFieldType(\nameone, \slicesone, \ttyone)\\
  \vbone \eqdef \nameone = \nametwo\\
  \slicesequal(\tenv, \slicesone, \slicestwo) \typearrow \vbtwo\\
  \subtypesat(\tenv, \ttyone, \ttytwo) \typearrow \vbthree \OrTypeError\\\\
  \vb \eqdef \vbone \land \vbtwo \land \vbthree
}{
  \membfs(\tenv, \bfstwo, \vbfone) \typearrow \vb
}
\end{mathpar}

\hypertarget{def-checkstructurelabel}{}
\TypingRuleDef{CheckStructure}
The function
\[
  \checkstructurelabel(\overname{\staticenvs}{\tenv} \aslsep \overname{\ty}{\vt} \aslsep \overname{\astlabels}{\vl}) \aslto
  \{\True\} \cup \TTypeError
\]
returns $\True$ is $\vt$ is has the \structure\ a of type corresponding to the AST label $\vl$ and a type error otherwise.

\ProseParagraph
One of the following applies:
\begin{itemize}
  \item All of the following apply (\textsc{okay}):
  \begin{itemize}
    \item determining the \structure\ of $\vt$ yields $\vtp$\ProseOrTypeError;
    \item $\vtp$ has the label $\vl$;
    \item the result is $\True$;
  \end{itemize}

  \item All of the following apply (\textsc{error}):
  \begin{itemize}
    \item determining the \structure\ of $\vt$ yields $\vtp$\ProseOrTypeError;
    \item $\vtp$ does not have the label $\vl$;
    \item the result is a type error indicating that $\vt$ was expected to have the \structure\ of a type with the AST label $\vl$.
  \end{itemize}
\end{itemize}

\FormallyParagraph
\begin{mathpar}
\inferrule[okay]{
  \tstruct(\vt) \typearrow \vtp \OrTypeError\\\\
  \astlabel(\vtp) = \vl
}
{
  \checkstructurelabel(\tenv, \vt, \vl) \typearrow \True
}
\and
\inferrule[error]{
  \tstruct(\vt) \typearrow \vtp\\
  \astlabel(\vtp) \neq \vl
}
{
  \checkstructurelabel(\tenv, \vt, \vl) \typearrow \TypeErrorVal{\UnexpectedType}
}
\end{mathpar}

\TypingRuleDef{ToWellConstrained}
\hypertarget{def-towellconstrained}{}
The function
\[
  \towellconstrained(\overname{\ty}{\vt}) \aslto \overname{\ty}{\vtp}
\]
returns the \wellconstrainedversion\ of a type $\vt$ --- $\vtp$, which is defined as follows.

\ProseParagraph
One of the following applies:
\begin{itemize}
  \item All of the following apply (\textsc{t\_int\_parameterized}):
  \begin{itemize}
    \item $\vt$ is a \parameterizedintegertype\ for the variable $\vv$;
    \item $\vtp$ is the well-constrained integer constrained by the variable expression for $\vv$,
    that is, $\TInt(\wellconstrained(\constraintexact(\EVar(\vv))))$.
  \end{itemize}

  \item All of the following apply (\textsc{t\_int\_other, other}):
  \begin{itemize}
    \item $\vt$ is not a \parameterizedintegertype\ for the variable $\vv$;
    \item $\vtp$ is $\vt$.
  \end{itemize}
\end{itemize}

\FormallyParagraph
\begin{mathpar}
\inferrule[t\_int\_parameterized]{}
{
  \towellconstrained(\TInt(\parameterized(\vv))) \typearrow\\ \TInt(\wellconstrained(\constraintexact(\EVar(\vv))))
}
\and
\inferrule[t\_int\_other]{
  \astlabel(\vi) \neq \parameterized
}{
  \towellconstrained(\TInt(\vi)) \typearrow \vt
}
\and
\inferrule[other]{
  \astlabel(\vt) \neq \TInt
}{
  \towellconstrained(\vt) \typearrow \vt
}
\end{mathpar}

\TypingRuleDef{GetWellConstrainedStructure}
\hypertarget{def-getwellconstrainedstructure}{}
The function
\[
  \getwellconstrainedstructure(\overname{\staticenvs}{\tenv} \aslsep \overname{\ty}{\vt})
  \aslto \overname{\ty}{\vtp} \cup \overname{\TTypeError}{\TypeErrorConfig}
\]
returns the \wellconstrainedstructure\ of a type $\vt$ in the static environment $\tenv$ --- $\vtp$, which is defined as follows.
\ProseOtherwiseTypeError

\ProseParagraph
All of the following apply:
\begin{itemize}
  \item the \structure\ of $\vt$ in $\tenv$ is $\vtone$\ProseOrTypeError;
  \item the well-constrained version of $\vtone$ is $\vtp$.
\end{itemize}

\FormallyParagraph
\begin{mathpar}
\inferrule{
  \tstruct(\tenv, \vt) \typearrow \vtone \OrTypeError\\\\
  \towellconstrained(\vtone) \typearrow \vtp
}{
  \getwellconstrainedstructure(\tenv, \vt) \typearrow \vtp
}
\end{mathpar}

\TypingRuleDef{GetBitvectorWidth}
\hypertarget{def-getbitvectorwidth}{}
The function
\[
  \getbitvectorwidth(\overname{\staticenvs}{\tenv} \aslsep \overname{\ty}{\vt}) \aslto
  \overname{\expr}{\ve} \cup \overname{\TTypeError}{\TypeErrorConfig}
\]
returns the expression $\ve$, which represents the width of the bitvector type $\vt$
in the static environment $\tenv$.
\ProseOrTypeError

\ProseParagraph
One of the following applies:
\begin{itemize}
  \item All of the following apply (\textsc{okay}):
  \begin{itemize}
    \item obtaining the \structure\ of $\vt$ in $\tenv$ yields a bitvector type with width expression $\ve$,
          that is, $\TBits(\ve, \Ignore)$\ProseOrTypeError;
    \item the result is $\ve$.
  \end{itemize}

  \item All of the following apply (\textsc{error}):
  \begin{itemize}
    \item obtaining the \structure\ of $\vt$ in $\tenv$ yields a type that is not a bitvector type;
    \item the result is a type error indicating that a bitvector type was expected.
  \end{itemize}
\end{itemize}

\FormallyParagraph
\begin{mathpar}
\inferrule[okay]{
  \tstruct(\tenv, \vt) \typearrow \TBits(\ve, \Ignore) \OrTypeError
}{
  \getbitvectorwidth(\tenv, \vt) \typearrow \ve
}
\and
\inferrule[error]{
  \tstruct(\tenv, \vt) \typearrow \vtp\\
  \astlabel(\vtp) \neq \TBits
}{
  \getbitvectorwidth(\tenv, \vt) \typearrow \TypeErrorVal{\UnexpectedType}
}
\end{mathpar}
\CodeSubsection{\GetBitvectorWidthBegin}{\GetBitvectorWidthEnd}{../Typing.ml}

\TypingRuleDef{GetBitvectorConstWidth}
\hypertarget{def-getbitvectorconstwidth}{}
The function
\[
  \getbitvectorconstwidth(\overname{\staticenvs}{\tenv} \aslsep \overname{\ty}{\vt}) \aslto
  \overname{\N}{\vw} \cup \overname{\TTypeError}{\TypeErrorConfig}
\]
returns the natural number $\vw$, which represents the width of the bitvector type $\vt$
in the static environment $\tenv$.
\ProseOtherwiseTypeError

\ProseParagraph
All of the following apply:
\begin{itemize}
  \item applying $\getbitvectorwidth$ to $\vt$ in $\tenv$ yields $\ewidth$\ProseOrTypeError;
  \item \Prosestaticeval{$\tenv$}{$\ewidth$}{integer for $\vw$}\ProseOrTypeError.
\end{itemize}

\FormallyParagraph
\begin{mathpar}
\inferrule{
  \getbitvectorwidth(\tenv, \vt) \typearrow \ewidth \OrTypeError\\\\
  \staticeval(\tenv, \ewidth) \typearrow \lint(\vw) \OrTypeError
}{
  \getbitvectorconstwidth(\tenv, \vt) \typearrow \vw
}
\end{mathpar}
\CodeSubsection{\GetBitvectorConstWidthBegin}{\GetBitvectorConstWidthEnd}{../Typing.ml}

\TypingRuleDef{CheckBitsEqualWidth}
\hypertarget{def-checkbitsequalwidth}{}
The function
\[
  \checkbitsequalwidth(
    \overname{\staticenvs}{\tenv} \aslsep
    \overname{\ty}{\vtone} \aslsep
    \overname{\ty}{\vttwo}) \aslto
  \{\True\} \cup \TTypeError
\]
tests whether the types $\vtone$ and $\vttwo$ are bitvector types of the same width.
If the answer is positive, the result is $\True$. \ProseOtherwiseTypeError

\ProseParagraph
All of the following apply:
\begin{itemize}
  \item obtaining the width of $\vtone$ in $\tenv$ (via $\getbitvectorwidth$) yields the expression $\vn$\ProseOrTypeError;
  \item obtaining the width of $\vttwo$ in $\tenv$ (via $\getbitvectorwidth$) yields the expression $\vm$\ProseOrTypeError;
  \item One of the following applies:
  \begin{itemize}
    \item All of the following apply (\textsc{true}):
    \begin{itemize}
      \item symbolically checking whether the bitwidth expressions $\vn$ and $\vm$ are equal (via $\bitwidthequal$) yields $\True$;
      \item the result is $\True$.
    \end{itemize}

    \item All of the following apply (\textsc{error}):
    \begin{itemize}
      \item symbolically checking whether the bitwidth expressions $\vn$ and $\vm$ are equal (via $\bitwidthequal$) yields $\False$;
      \item the result is a type error indicating that the bitwidths are different.
    \end{itemize}
  \end{itemize}
\end{itemize}

\FormallyParagraph
\begin{mathpar}
\inferrule[true]{
  \getbitvectorwidth(\tenv, \vtone) \typearrow \vn \OrTypeError\\\\
  \getbitvectorwidth(\tenv, \vttwo) \typearrow \vm \OrTypeError\\\\
  \bitwidthequal(\tenv, \vn, \vm) \typearrow \True
}{
  \checkbitsequalwidth(\tenv, \vtone, \vttwo) \typearrow \True
}
\and
\inferrule[error]{
  \getbitvectorwidth(\tenv, \vtone) \typearrow \vn \OrTypeError\\\\
  \getbitvectorwidth(\tenv, \vttwo) \typearrow \vm \OrTypeError\\\\
  \bitwidthequal(\tenv, \vn, \vm) \typearrow \False
}{
  \checkbitsequalwidth(\tenv, \vtone, \vttwo) \typearrow \TypeErrorVal{\UnexpectedType}
}
\end{mathpar}

\section{Base Values\label{sec:BaseValues}}
\hypertarget{def-basevalueterm}{}
Each type, with the exceptions stated below, have a \basevalueterm,
which is used to initialize storage elements (either local of global),
if an initializer is not supplied.

\RequirementDef{NoBaseValue}
The following types do not have a \basevalueterm{}:
\begin{itemize}
    \item \parameterizedintegertypes{};
    \item \bitvectortypesterm{} whose length depends on a \parameterizedintegertype{};
    \item a \wellconstrainedintegertype{} whose list of constraints
        represents the empty set;
    \item a \bitvectortypeterm{} whose length is negative.
\end{itemize}

\lrmcomment{\identi{WVQZ}}
Subprogram parameters can be parameterized integers, and since they will be initialized by their
invocation, there is no need to have a \basevalueterm{} for them.

\subsubsection{Example: Base Values}
\listingref{base-values} shows a specification with examples of well-typed \basevalueterm{}
for various types, followed by the output to the console.
\ASLListing{Well-typed Base Values}{base-values}{\typingtests/TypingRule.BaseValue.asl}
% CONSOLE_BEGIN aslref \typingtests/TypingRule.BaseValue.asl
\begin{Verbatim}[fontsize=\footnotesize, frame=single]
global_base = 0, unconstrained_integer_base = 0, constrained_integer_base = -3
bool_base = FALSE, real_base = 0, string_base = , enumeration_base = RED
bits_base = 0x00
tuple_base = (0, -3, RED)
record_base      = {data=0x00, time=0, flag=FALSE}
record_base_init = {data=0x00, time=0, flag=FALSE}
exception_base = {msg=}
integer_array_base = [[0, 0, 0, 0]]
enumeration_array_base = [[RED=0, GREEN=0, BLUE=0]]
\end{Verbatim}
% CONSOLE_END

\subsubsection{Example: Types Without Base Value}
\listingref{base-values-bad} shows an ill-typed specification
due to the fact that \parameterizedintegertypes{} have no defined \basevalueterm{},
which is also true for a \bitvectortypeterm{} whose width is parameterized.
\ASLListing{No Base Value for Parameterized Integer Types}{base-values-bad}{\typingtests/TypingRule.BaseValue.bad_parameterized.asl}

\listingref{base-values-bad-negative-width} shows an ill-typed specification
where the width of a bitvector is negative.
\ASLListing{No Base Value for Bitvectors of Negative Width}{base-values-bad-negative-width}{\typingtests/TypingRule.BaseValue.bad_negative_width.asl}

\listingref{base-values-bad-empty-type} shows an ill-typed specification
where the constraint \verb|5..0| represents an empty set.
Therefore, the domain of values for the type \verb|integer{5..0}| is empty,
which negates the possibility of having a \basevalueterm.
\ASLListing{No Base Value for an Empty Integer Type}{base-values-bad-empty-type}{\typingtests/TypingRule.BaseValue.bad_empty.asl}

\hypertarget{def-basevalue}{}
The function
\[
\basevalue(\overname{\staticenvs}{\tenv} \aslsep \overname{\ty}{\vt}) \aslto
\overname{\expr}{\veinit} \cup \overname{\TTypeError}{\TypeErrorConfig}
\]
returns the expression $\veinit$ which can be used to initialize a storage element
of type $\vt$ in the static environment $\tenv$.
\ProseOtherwiseTypeError

\TypingRuleDef{BaseValue}
\ProseParagraph
\OneApplies
\begin{itemize}
    \item \AllApplyCase{t\_bool} \lrmcomment{\identr{CPCK}}
    \begin{itemize}
        \item $\vt$ is the Boolean type;
        \item $\veinit$ is the literal expression for $\False$.
    \end{itemize}

    \item \AllApplyCase{t\_bits} \lrmcomment{\identr{ZVPT}}
    \begin{itemize}
        \item $\vt$ is the bitvector type with width expression $\ve$;
        \item applying $\reducetozopt$ to $\ve$ in $\tenv$ yields $\vzopt$;
        \item checking that $\vzopt$ is not $\None$ yields $\True$\ProseTerminateAs{\NoBaseValue};
        \item view $\vzopt$ as the singleton integer $\length$;
        \item checking that $\length$ is greater or equal to $0$ yields $\True$\ProseTerminateAs{\NoBaseValue};
        \item $\veinit$ is the literal expression for a bitvector made of a sequence of $\length$ values of $0$.
    \end{itemize}

    \item \AllApplyCase{t\_enum} \lrmcomment{\identr{LCCN}}
    \begin{itemize}
        \item $\vt$ is the enumeration type with a list of labels where $\name$ as its \head;
        \item $\name$ is bound to the literal $\vl$ by the $\constantvalues$ in the global static environment of $\tenv$;
        \item $\veinit$ is the literal expression for $\vl$, that is, $\eliteral{\vl}$.
    \end{itemize}

    \item \AllApplyCase{t\_int\_unconstrained} \lrmcomment{\identr{NJDZ}}
    \begin{itemize}
        \item $\vt$ is the \unconstrainedintegertype;
        \item $\veinit$ is the literal expression for $0$, that is, $\ELiteral(\lint(0))$.
    \end{itemize}

    \item \AllApplyCase{t\_int\_parameterized} \lrmcomment{\identr{QGGH}}
    \begin{itemize}
        \item $\vt$ is the \parameterizedintegertype;
        \item the result is a type error indicating the lack of a statically known base value.
    \end{itemize}

    \item \AllApplyCase{t\_int\_wellconstrained} \lrmcomment{\identr{CFTD}}
    \begin{itemize}
        \item $\vt$ is the \wellconstrainedintegertype\ with a list of constraints $\cs$;
        \item define $\vzminlist$ as the concatenation of lists obtained for each
              constraint $\cs[\vi]$ in $\tenv$, for each $\vi\in\listrange(\cs)$, via $\constraintabsmin$;
        \item checking whether $\vzminlist$ is empty yields $\True$\ProseOrTypeError{\NoBaseValue};
        \item determining the minimal absolute integer in $\vzminlist$ via $\listminabs$ yields $\vzmin$;
        \item $\veinit$ is the integer literal expression for $\vzmin$.
    \end{itemize}

    \item \AllApplyCase{t\_named}
    \begin{itemize}
        \item $\vt$ is the \namedtype\ for $\id$;
        \item obtaining the \underlyingtype\ for $\id$ in $\tenv$ yields $\vtp$\ProseOrTypeError;
        \item applying $\basevalue$ to $\vtp$ in $\tenv$ yields $\veinit$\ProseOrTypeError.
    \end{itemize}

    \item \AllApplyCase{t\_real} \lrmcomment{\identr{GYCG}}
    \begin{itemize}
        \item $\vt$ is the \realtypeterm{};
        \item $\veinit$ is the real literal expression for $0$.
    \end{itemize}

    \item \AllApplyCase{structured} \lrmcomment{\identr{MBRM}, \ident{SVJB}}
    \begin{itemize}
        \item $\vt$ is a \structuredtype\ with list of fields $\fields$;
        \item applying $\basevalue$ to $\vtefield$ in $\tenv$ for each $(\name, \vtefield)$ in $\fields$
              yields $\ve_\name$\ProseOrTypeError;
        \item $\veinit$ is the record construction expression assigning each field $\name$
              where $(\name, \vtefield)$ is an element of $\fields$ to $\vtefield$, that is, \\
              $\ERecord((\name, \vtefield) \in \fields: (\name, \ve_\name))$.
    \end{itemize}

    \item \AllApplyCase{t\_string} \lrmcomment{\identr{WKCY}}
    \begin{itemize}
        \item $\vt$ is the \stringtypeterm{};
        \item $\veinit$ is the string literal expression for the empty list of characters.
    \end{itemize}

    \item \AllApplyCase{t\_tuple} \lrmcomment{\identr{QWSQ}}
    \begin{itemize}
        \item $\vt$ is the tuple type over the list of types $\vt_{1..k}$, that is, $\TTuple(\vt_{1..k})$;
        \item applying $\basevalue$ to each type $\vt_\vi$ in $\tenv$ for $\vi=1..k$; yields the list of expressions $\ve_{1..k}$;
        \item $\veinit$ is the tuple expression $\ETuple(\ve_{1..k})$.
    \end{itemize}

    \item \AllApplyCase{t\_array\_enum}
    \begin{itemize}
        \item $\vt$ is the enumerated array type over for the enumeration $\venum$ and labels $\vlabels$ and element type $\tty$,
              that is, $\TArray(\ArrayLengthEnum(\venum, \vlabels), \tty)$ ;
        \item applying $\basevalue$ to $\tty$ in $\tenv$ yields the expression $\vvalue$\ProseOrTypeError;
        \item $\veinit$ is the array construction expression for an enumerated array with labels $\vlabels$ and initial value $\vvalue$,
              that is, $\EEnumArray\{\EArrayLabels: \vlabels, \EArrayValue: \vvalue\}$.
    \end{itemize}

    \item \AllApplyCase{t\_array\_expr} \lrmcomment{\identr{WGVR}}
    \begin{itemize}
        \item $\vt$ is the array type over an integer index expression $\vlength$ and element type $\tty$, that is,
              $\TArray(\ArrayLengthExpr(\vlength), \tty)$ ;
        \item applying $\basevalue$ to $\tty$ in $\tenv$ yields the expression $\vvalue$\ProseOrTypeError;
        \item $\veinit$ is the array construction expression with length expression $\vlength$ and value expression $\vvalue$,
              that is, $\EArray\{\EArrayLength: \length, \EArrayValue: \vvalue\}$.
    \end{itemize}
\end{itemize}

\FormallyParagraph
\begin{mathpar}
\inferrule[t\_bool]{}{
    \basevalue(\tenv, \overname{\TBool}{\vt}) \typearrow \overname{\ELiteral(\lbool(\False))}{\veinit}
}
\end{mathpar}

\begin{mathpar}
\inferrule[t\_bits]{
    \reducetozopt(\tenv, \ve) \typearrow \vzopt\\
    \checktrans{\vzopt \neq \None}{\NoBaseValue} \checktransarrow \True\OrTypeError\\\\
    \vzopt \eqname \langle\length\rangle\\
    \checktrans{\length \geq 0}{\NoBaseValue} \checktransarrow \True\OrTypeError
}{
    \basevalue(\tenv, \overname{\TBits(\ve, \Ignore)}{\vt}) \typearrow \overname{\ELiteral(\lbitvector(i=1..\length: 0))}{\veinit}
}
\end{mathpar}

\begin{mathpar}
\inferrule[t\_enum]{%
    \lookupconstant(\tenv, \name) \typearrow \vl
}{%
    \basevalue(\tenv, \overname{\TEnum(\name \concat \Ignore)}{\vt}) \typearrow \overname{\ELiteral(\vl)}{\veinit}
}
\end{mathpar}

\begin{mathpar}
\inferrule[t\_int\_unconstrained]{}{
    \basevalue(\tenv, \overname{\unconstrainedinteger}{\vt}) \typearrow \overname{\ELiteral(\lint(0))}{\veinit}
}
\end{mathpar}

\begin{mathpar}
\inferrule[t\_int\_parameterized]{}{
    \basevalue(\tenv, \overname{\TInt(\parameterized(\id))}{\vt}) \typearrow \TypeErrorVal{\NoBaseValue}
}
\end{mathpar}

\begin{mathpar}
\inferrule[t\_int\_wellconstrained]{
    \cs \eqname \vc_{1..k}\\
    \vzminlist \eqdef \constraintabsmin(\tenv, \vc_1) \concat \ldots \concat \constraintabsmin(\tenv, \vc_k)\\
    \checktrans{\vzminlist \neq \emptyset}{\NoBaseValue} \typearrow \True \OrTypeError\\\\
    \listminabs(\vzminlist) \typearrow \vzmin
}{
    \basevalue(\tenv, \overname{\TInt(\wellconstrained(\cs))}{\vt}) \typearrow \overname{\ELiteral(\lint(\vzmin))}{\veinit}
}
\end{mathpar}

\begin{mathpar}
\inferrule[t\_named]{
    \makeanonymous(\tenv, \TNamed(\id)) \typearrow \vtp \OrTypeError\\\\
    \basevalue(\tenv, \vtp) \typearrow \veinit \OrTypeError
}{
    \basevalue(\tenv, \overname{\TNamed(\id)}{\vt}) \typearrow \veinit
}
\end{mathpar}

\begin{mathpar}
\inferrule[t\_real]{}{
    \basevalue(\tenv, \overname{\TReal}{\vt}) \typearrow \overname{\ELiteral(\lreal(0))}{\veinit}
}
\end{mathpar}

\begin{mathpar}
\inferrule[structured]{
    \isstructured(\vt) \typearrow \True\\
    \vt \eqname L(\fields)\\
    (\name, \vtefield) \in \fields: \basevalue(\tenv, \vtefield) \typearrow \ve_\name \OrTypeError
}{
    \basevalue(\tenv, \vt) \typearrow \overname{\ERecord((\name, \vtefield) \in \fields: (\name, \ve_\name))}{\veinit}
}
\end{mathpar}

\begin{mathpar}
\inferrule[t\_string]{}{
    \basevalue(\tenv, \overname{\TString}{\vt}) \typearrow \overname{\ELiteral(\lstring(\emptylist))}{\veinit}
}
\end{mathpar}

\begin{mathpar}
\inferrule[t\_tuple]{
    \vi=1..k: \basevalue(\tenv, \vt_\vi) \typearrow \ve_\vi \OrTypeError
}{
    \basevalue(\tenv, \overname{\TTuple}{\vt_{1..k}}) \typearrow \overname{\ETuple(\ve_{1..k})}{\veinit}
}
\end{mathpar}

\begin{mathpar}
\inferrule[t\_array\_enum]{
    \basevalue(\tenv, \tty) \typearrow \vvalue \OrTypeError
}{
    {
        \begin{array}{r}
            \basevalue(\tenv, \overname{\TArray(\ArrayLengthEnum(\venum, \vlabels), \tty)}{\vt}) \typearrow \\
            \overname{\EEnumArray\{\EArrayLabels: \vlabels, \EArrayValue: \vvalue\}}{\veinit}
        \end{array}
    }
}
\end{mathpar}

\begin{mathpar}
\inferrule[t\_array\_expr]{
    \basevalue(\tenv, \tty) \typearrow \vvalue \OrTypeError
}{
    {
        \begin{array}{r}
            \basevalue(\tenv, \overname{\TArray(\ArrayLengthExpr(\length), \tty)}{\vt}) \typearrow\\
            \overname{\EArray\{\EArrayLength: \length, \EArrayValue: \vvalue\}}{\veinit}
        \end{array}
    }
}
\end{mathpar}

\TypingRuleDef{ConstraintAbsMin}
\hypertarget{def-constraintabsmin}{}
The function
\[
    \constraintabsmin(\overname{\staticenvs}{\tenv} \aslsep \overname{\intconstraint}{\vc}) \aslto
    \overname{\Z^*}{\vzs}
    \cup \overname{\TTypeError}{\TypeErrorVal{\NoBaseValue}}
\]
returns a single element list containing the integer closest to $0$ that satisfies the constraint $\vc$ in $\tenv$, if one exists,
and an empty list if the constraint represents an empty set.
Otherwise, the result is $\TypeErrorVal{\NoBaseValue}$.

\ProseParagraph
\OneApplies
\begin{itemize}
    \item \AllApplyCase{exact}
    \begin{itemize}
        \item $\vc$ is the constraint given by the expression $\ve$, that is, $\ConstraintExact(\ve)$;
        \item applying $\reducetozopt$ to $\ve$ in $\tenv$ yields the optional integer $\vzopt$;
        \item checking that $\vzopt$ is not $\None$ yields $\True$\ProseTerminateAs{\NoBaseValue};
        \item view $\vzopt$ as the singleton set for the integer $\vz$;
        \item define $\vzs$ as the single element list containing $\vz$.
    \end{itemize}

    \item \AllApplyCase{range}
    \begin{itemize}
        \item $\vc$ is the constraint given by the expression $\veone$ and $\vetwo$, that is, \\
                $\ConstraintRange(\veone, \vetwo)$;
        \item applying $\reducetozopt$ to $\veone$ in $\tenv$ yields the optional integer $\vzoptone$;
        \item checking that $\vzoptone$ is not $\None$ yields $\True$\ProseTerminateAs{\NoBaseValue};
        \item view $\vzoptone$ as the singleton set for $\vvone$;
        \item applying $\reducetozopt$ to $\vetwo$ in $\tenv$ yields the optional integer $\vzopttwo$;
        \item checking that $\vzopttwo$ is not $\None$ yields $\True$\ProseTerminateAs{\NoBaseValue};
        \item view $\vzopttwo$ as the singleton set for $\vvtwo$;
        \item define $\vzs$ as based on the following cases for $\vvone$ and $\vvtwo$:
        \begin{itemize}
            \item the empty list, if $\vvone$ is greater than $\vvtwo$ (since there are no integers satisfying the constraint);
            \item the single element list for $\vvtwo$, if $\vvone$ is less than $\vvtwo$ and both are negative;
            \item the single element list for $0$, if $\vvone$ is negative and $\vvtwo$ is non-negative;
            \item the single element list for $\vvone$, if $\vvone$ is non-negative and $\vvtwo$ is greater or equal to $\vvone$.
        \end{itemize}
    \end{itemize}
\end{itemize}

\FormallyParagraph
\begin{mathpar}
\inferrule[exact]{
    \reducetozopt(\tenv, \ve) \typearrow \vzopt\\
    \checktrans{\vzopt \neq \None}{\NoBaseValue} \checktransarrow \True \OrTypeError\\\\
    \vzopt \eqname \langle\vz\rangle
}{
    \constraintabsmin(\overname{\ConstraintExact(\ve)}{\vc}) \typearrow \overname{[\vz]}{\vzs}
}
\end{mathpar}

\begin{mathpar}
\inferrule[range]{
    \reducetozopt(\tenv, \veone) \typearrow \vzoptone\\
    \checktrans{\vzoptone \neq \None}{\NoBaseValue} \checktransarrow \True \OrTypeError\\\\
    \vzoptone \eqname \langle\vvone\rangle\\
    \reducetozopt(\tenv, \vetwo) \typearrow \vzopttwo\\
    \checktrans{\vzopttwo \neq \None}{\NoBaseValue} \checktransarrow \True \OrTypeError\\\\
    \vzopttwo \eqname \langle\vvtwo\rangle\\
    \vzs \eqdef {
        \begin{cases}
           \emptylist & \vvone > \vvtwo\\
           [\vvtwo] & \vvone \leq \vvtwo < 0\\
           [0] & \vvone < 0 \leq \vvtwo < 0\\
           [\vvone] & 0 \leq \vvone \leq \vvtwo\\
        \end{cases}
    }
}{
    \constraintabsmin(\overname{\ConstraintRange(\veone, \vetwo)}{\vc}) \typearrow \vzs
}
\end{mathpar}

\TypingRuleDef{ListMinAbs}
\hypertarget{def-listminabs}{}
The function
\[
\listminabs(\overname{\Z^*}{\vl}) \aslto \overname{\Z}{\vz}
\]
returns $\vz$ --- the integer closest to $0$ among the list
of integers in the list $\vl$. The result is biased towards positive integers. That is,
if two integers $x$ and $y$ have the same absolute value and $x$ is positive and $y$ is negative
then $x$ is considered closer to $0$.

\subsubsection{Example: Minimal Absolute Value}
The minimal absolute value of $[9, -3]$ is $-3$,
and the minimal absolute value of $[2, -2]$ is $2$.

\ProseParagraph
\OneApplies
\begin{itemize}
    \item \AllApplyCase{one}
    \begin{itemize}
        \item $\vl$ is the single element list for $\vz$.
    \end{itemize}

    \item \AllApplyCase{more\_than\_one}
    \begin{itemize}
        \item $\vl$ is the list where $\vzone$ is its \head\ and $\vltwo$ is its \tail;
        \item $\vltwo$ is not the empty list;
        \item applying $\listminabs$ to $\vltwo$ yields $\vztwo$;
        \item define $\vz$ based on $\vzone$ and $\vztwo$ by the following cases:
        \begin{itemize}
            \item $\vzone$ if the absolute value of $\vzone$ is less than the absolute value of $\vztwo$;
            \item $\vztwo$ if the absolute value of $\vzone$ is greater than the absolute value of $\vztwo$;
            \item $\vzone$ if $\vzone$ is equal to $\vztwo$;
            \item the absolute value of $\vzone$ if the absolute value of $\vzone$ is equal to the absolute value of $\vztwo$
                    and $\vzone$ is not equal to $\vztwo$;
        \end{itemize}
    \end{itemize}
\end{itemize}

\FormallyParagraph
\begin{mathpar}
\inferrule[one]{}{
    \listminabs(\overname{[\vz]}{\vl}) \typearrow \vz
}
\end{mathpar}

\begin{mathpar}
\inferrule[more\_than\_one]{
    \vlone \neq \emptylist\\
    \listminabs(\vltwo) = \vztwo\\
    {
        \vz \eqdef \begin{cases}
            \vzone & |\vzone| < |\vztwo|\\
            \vztwo & |\vzone| > |\vztwo|\\
            \vzone & \vzone = \vztwo\\
            |\vzone| & |\vzone| = |\vztwo| \land \vzone \neq \vztwo
        \end{cases}
    }
}{
    \listminabs(\overname{[\vzone] \concat \vltwo}{\vl}) \typearrow \vz
}
\end{mathpar}

