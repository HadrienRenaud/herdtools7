\usepackage{amsmath}  % Classic math package
\usepackage{amssymb}  % Classic math package 
\usepackage{mathtools}  % Additional math package 
\usepackage{graphicx}  % For figures
\usepackage{caption}  % For figure captions
\usepackage{subcaption}  % For subfigure captions
\usepackage{url}  % Automatically escapes urls
\usepackage{hyperref}  % Insert links inside pdfs
\usepackage[inline]{enumitem}  % For inline lists
\usepackage[export]{adjustbox}  % For centering too wide figures 
\usepackage{mathpartir}  % For deduction rules and equations paragraphs
\usepackage{comment}
\usepackage{fancyvrb}
\usepackage[
  % Even pages have notes on the left, odd on the right
  twoside,
  % Notes on the right, should be less than outer
  marginparwidth=100pt,
  % margins
  top=4.5cm, bottom=4.5cm, inner=3.5cm, outer=4.5cm
  % To visualize:
  % showframe
]{geometry}
%\usepackage{stmaryrd} % for \llbracket and \rrbracket
\input{ifempty}
\input{ifformal}
\input{iftodo}
%% Should be functional
\ifempty
\newcommand{\isempty}[1]{#1}
\else
\newcommand{\isempty}[1]{}
\fi

%%%Safety net
\iffalse
\makeatletter
\newcommand{\isempty}[1]{#1}
\makeatother
\fi

\fvset{fontsize=\small}

\usepackage{enumitem}
\renewlist{itemize}{itemize}{20}
\setlist[itemize,1]{label=\textbullet}
\setlist[itemize,2]{label=\textasteriskcentered} 
\setlist[itemize,3]{label=\textendash}
\setlist[itemize,4]{label=$\triangleright$}
\setlist[itemize,5]{label=+}
\setlist[itemize,6]{label=\textbullet}
\setlist[itemize,7]{label=\textasteriskcentered}
\setlist[itemize,8]{label=\textendash}
\setlist[itemize,9]{label=$\triangleright$}
\setlist[itemize,10]{label=+}
\setlist[itemize,11]{label=\textbullet}
\setlist[itemize,12]{label=\textasteriskcentered}
\setlist[itemize,13]{label=\textendash}
\setlist[itemize,14]{label=\textendash}
\setlist[itemize,15]{label=$\triangleright$}
\setlist[itemize,16]{label=+}
\setlist[itemize,17]{label=\textbullet}
\setlist[itemize,18]{label=\textasteriskcentered}
\setlist[itemize,19]{label=\textendash}
\setlist[itemize,20]{label=\textendash}

%%%%%%%%%%%%%%%%%%%%%%%%%%%%%%%%%%%%%%%%%%%%%%%%%%
%Abstract Syntax macros

\newcommand\emptylist[0]{[\ ]}

% Non-terminal names
\newcommand\unop[0]{\textsf{unop}}
\newcommand\binop[0]{\textsf{binop}}
\newcommand\literal[0]{\textsf{literal}}
\newcommand\expr[0]{\textsf{expr}}
\newcommand\lexpr[0]{\textsf{lexpr}}
\newcommand\slice[0]{\textsf{slice}}
\newcommand\leslice[0]{\texttt{LE\_Slice}}
\newcommand\slicesingle[0]{\texttt{Slice\_Single}}
\newcommand\slicelength[0]{\texttt{Slice\_Length}}
\newcommand\slicerange[0]{\texttt{Slice\_Range}}
\newcommand\slicestar[0]{\texttt{Slice\_Star}}

\newcommand\ty[0]{\textsf{ty}}
\newcommand\pattern[0]{\textsf{pattern}}
\newcommand\intconstraint[0]{\textsf{int\_constraint}}
\newcommand\intconstraints[0]{\textsf{int\_constraints}}
\newcommand\bitsconstraint[0]{\textsf{bits\_constraint}}
\newcommand\bitfield[0]{\textsf{bitfield}}
\newcommand\version[0]{\textsf{version}}
\newcommand\program[0]{\textsf{program}}
\newcommand\typedidentifier[0]{\textsf{typed\_identifier}}
\newcommand\localdeclkeyword[0]{\textsf{local\_decl\_keyword}}
\newcommand\globaldeclkeyword[0]{\textsf{global\_decl\_keyword}}
\newcommand\localdeclitem[0]{\textsf{local\_decl\_item}}
\newcommand\globaldecl[0]{\textsf{global\_decl}}
\newcommand\fordirection[0]{\textsf{for\_direction}}
\newcommand\stmt[0]{\textsf{stmt}}
\newcommand\decl[0]{\textsf{decl}}
\newcommand\casealt[0]{\textsf{case\_alt}}
\newcommand\catcher[0]{\textsf{catcher}}
\newcommand\subprogramtype[0]{\textsf{sub\_program\_type}}
\newcommand\subprogrambody[0]{\textsf{sub\_program\_body}}
\newcommand\func[0]{\textsf{func}}
\newcommand\Field[0]{\textsf{field}}

% Expression labels
\newcommand\evar[0]{\textsf{E\_Var}}

% Literal labels
\newcommand\lint[0]{\textsf{L\_Int}}
\newcommand\lbool[0]{\textsf{L\_Bool}}
\newcommand\lreal[0]{\textsf{L\_Real}}
\newcommand\lbitvector[0]{\textsf{L\_Bitvector}}
\newcommand\lstring[0]{\textsf{L\_String}}

\newcommand\True[0]{\textsf{True}}
\newcommand\False[0]{\textsf{False}}

% Type labels
\newcommand\TInt[0]{\texttt{T\_Int}}
\newcommand\TReal[0]{\texttt{T\_Real}}
\newcommand\TString[0]{\texttt{T\_String}}
\newcommand\TBool[0]{\texttt{T\_Bool}}
\newcommand\TBits[0]{\texttt{T\_Bits}}
\newcommand\TEnum[0]{\texttt{T\_Enum}}
\newcommand\TTuple[0]{\texttt{T\_Tuple}}
\newcommand\TArray[0]{\texttt{T\_Array}}
\newcommand\TRecord[0]{\texttt{T\_Record}}
\newcommand\TException[0]{\texttt{T\_Exception}}
\newcommand\TNamed[0]{\texttt{T\_Named}}

\newcommand\identifier[0]{\textsf{identifier}}

\newcommand\id[0]{\texttt{id}}
\newcommand\op[0]{\texttt{op}}
%%%%%%%%%%%%%%%%%%%%%%%%%%%%%%%%%%%%%%%%%%%%%%%%%%

%%%%%%%%%%%%%%%%%%%%%%%%%%%%%%%%%%%%%%%%%%%%%%%%%%
% Typing macros
\newcommand\staticloc[1]{#1.\textsf{local}}
\newcommand\constantsvalues[0]{\textsf{constants\_values}}
\newcommand\storagetypes[0]{\textsf{storage\_types}}
\newcommand\returntype[0]{\textsf{return\_type}}
\newcommand\declaredtypes[0]{\textsf{declared\_types}}
\newcommand\subtypes[0]{\textsf{subtypes}}
\newcommand\subprograms[0]{\textsf{subprograms}}
\newcommand\subprogramrenamings[0]{\textsf{subprogram\_renamings}}
\newcommand\staticglob[1]{#1.\textsf{global}}

\newcommand\entails[0]{\vdash}
\newcommand\tstruct[0]{\texttt{get\_structure}}

%%%%%%%%%%%%%%%%%%%%%%%%%%%%%%%%%%%%%%%%%%%%%%%%%%

\newcommand\ie{i.\,e.}
\newcommand\eg{e.\,g.}
\newcommand\error[0]{\textsf{error}}
\newcommand\synor{\ |\ }
\newcommand\syntt[1]{\mathtt{#1}}
\newcommand\ife[3]{\text{if}\ #1\ \text{then}\ #2\ \text{else}\ #3\ \text{end}}
\newcommand\inenv[2]{\left\langle #1, #2 \right\rangle}
\newcommand\env[1]{\left\langle #1 \right\rangle}
\newcommand\reducesto{\ \to\ }
\newcommand\llbracket{[|}
\newcommand\rrbracket{|]}
\newcommand\interp[1]{\left\llbracket #1 \right\rrbracket}
\newcommand\st[0]{\ \middle|\ }
\newcommand\field[1]{.\text{#1}}
\newcommand\globals[0]{\field{globals}}
\newcommand\locals[0]{\field{locals}}
\newcommand\X[0]{\mathcal{X}}
\newcommand\N[0]{\mathbb{N}}
\newcommand\Q[0]{\mathbb{Q}}
\newcommand\Z[0]{\mathbb{Z}}
\newcommand\asldata[0]{\mathtt{asl\_data}}
\newcommand\aslctrl[0]{\mathtt{asl\_ctrl}}
\newcommand\aslpo[0]{\mathtt{asl\_po}}
\DeclareMathOperator{\dom}{dom}
\newcommand\partialto[0]{\rightharpoonup}
\newcommand\evalexpr[0]{\texttt{eval\_expr}}
\newcommand\evalconstraint[0]{\texttt{eval\_constraint}}

%%%%%%%%%%%%%%%%%%%%%%%%%%%%%%%%%%%%%%%%%%%%%%%%%%
% Ident info

\newcommand\ident[2]{\texttt{#1}\textsubscript{\texttt{\MakeUppercase{#2}}}}
\newcommand\identi[1]{\ident{I}{#1}}
\newcommand\identr[1]{\ident{R}{#1}}
\newcommand\identd[1]{\ident{D}{#1}}
\newcommand\identg[1]{\ident{G}{#1}}
