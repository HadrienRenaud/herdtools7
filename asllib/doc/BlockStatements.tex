\chapter{Block Statements\label{chap:BlockStatements}}
Block statements are statements executing in their own scope within the scope of their enclosing subprogram.

\section{Typing\label{sec:BlockStatementsTyping}}
\hypertarget{def-annotateblock}{}
The function
\[
  \annotateblock{\overname{\staticenvs}{\tenv} \aslsep \overname{\stmt}{\vs}} \aslto
  \overname{\stmt}{\newstmt} \cup \overname{\TTypeError}{\TypeErrorConfig}
\]
annotates a block statement $\vs$ in static environment $\tenv$ and returns the annotated
statement $\newstmt$ or a type error, if one is detected.

\subsubsection{TypingRule.Block\label{sec:TypingRule.Block}}
\subsubsection{Example}
\VerbatimInput{../tests/ASLTypingReference.t/TypingRule.Block0.asl}

\subsubsection{Prose}
All of the following apply:
\begin{itemize}
  \item annotating the statement $\vs$ in $\tenv$ yields $(\newstmt, \newtenv)$\ProseOrTypeError;
  \item the modified environment $\newtenv$ is dropped.
\end{itemize}
\subsubsection{Formally}
\begin{mathpar}
\inferrule{
  \annotatestmt(\tenv, \vs) \typearrow (\newstmt, \Ignore) \OrTypeError
}{
  \annotateblock{\tenv, \vs} \typearrow \newstmt
}
\end{mathpar}
\CodeSubsection{\BlockBegin}{\BlockEnd}{../Typing.ml}

\subsection{Comments}
A local identifier declared in a block statement (with \texttt{var}, \texttt{let}, or \texttt{constant})
is in scope from the point immediately after its declaration until the end of the
immediately enclosing block. This means, we can discard the environment at the end of
an enclosing block, which has the effect of dropping bindings of the identifiers declared inside the block.

\lrmcomment{This is related to \identr{JBXQ}.}

%%%%%%%%%%%%%%%%%%%%%%%%%%%%%%%%%%%%%%%%%%%%%%%%%%%%%%%%%%%%%%%%%%%%%%
\section{Semantics\label{sec:BlockStatementsSemantics}}
%%%%%%%%%%%%%%%%%%%%%%%%%%%%%%%%%%%%%%%%%%%%%%%%%%%%%%%%%%%%%%%%%%%%%%
The relation
\hypertarget{def-evalblock}{}
\[
  \evalblock{\overname{\envs}{\env} \times \overname{\stmt}{\stm}} \;\aslrel\;
  \overname{\TContinuing}{\Continuing(\newg, \newenv)} \cup
  \overname{\TReturning}{\ReturningConfig} \cup
  \overname{\TThrowing}{\ThrowingConfig} \cup
  \overname{\TError}{\ErrorConfig}
\]
evaluates a statement $\stm$ as a \emph{block}. That is, $\stm$ is evaluated in a fresh local environment,
which drops back to the original local environment of $\env$ when the evaluation terminates.

\subsubsection{SemanticsRule.Block\label{sec:SemanticsRule.Block}}
\subsubsection{Example}
In the specification:
\VerbatimInput{\semanticstests/SemanticsRule.Block.asl}
the conditional statement \texttt{if TRUE then\ldots{} end;} defines a
block structure. Thus, the scope of the declaration \texttt{let y = 2;} is
limited to its declaring block---or the binding for \texttt{y} no longer exists
once the block is exited. As a consequence, the subsequent declaration
\texttt{let y = 1} is valid.  By contrast, the assignment of the mutable
variable~\texttt{x} persists after block end. However, observe that \texttt{x}
is defined before the block and hence still exists after the block.

\subsubsection{Prose}
All of the following apply:
\begin{itemize}
    \item $\blockenv$ is the environment $\env$ modified by replacing the local component
    (of the inner dynamic environment) by an empty one;
    \item evaluating $\stm$ in $\blockenv$, as per \chapref{eval_stmt},
    is \\ $\Continuing(\newg, \blockenvone)$\ProseTerminateAs{\ReturningConfig};
    \item $\newenv$ is formed from $\blockenvone$ after restoring the
    variable bindings of $\env$ with the updated values of $\blockenv$.
    The effect is that of discarding the bindings for variables declared inside $\stm$;
    \item the result of the entire evaluation is $\Continuing(\newg, \newenv)$.
\end{itemize}
\subsubsection{Formally}
\begin{mathpar}
\inferrule{
  \env \eqname (\tenv,\denv)\\
  \blockenv \eqdef (\tenv, (G^\denv, \emptyfunc))\\
  \evalstmt{\blockenv, \stm} \evalarrow \Continuing(\newg, \blockenvone) \terminateas \ReturningConfig,\ThrowingConfig,\ErrorConfig\\\\
  \blockenvone\eqname(\tenv, \denvone)\\
  \newenv \eqdef(\tenv, (G^{\denvone}, \restrictfunc{L^{\denvone}}{{\dom(L^\denv)}}))
}{
  \evalblock{\env, \stm} \evalarrow \Continuing(\newg, \newenv)
}
\end{mathpar}
\CodeSubsection{\EvalBlockBegin}{\EvalBlockEnd}{../Interpreter.ml}

That is, evaluating a block discards the bindings for variables declared inside $\stm$.
