\chapter{Runtime Environment\label{chap:RuntimeEnvironment}}

An ASL runtime provides run time support within a hosting environment.

Examples of a hosting environment include an interactive interpreter,
an interpreter running in batch mode,
a Verilog simulator, and Linux process (native executable).

\RequirementDef{RuntimeDefaultEntry}
The default entry point is the \texttt{main} function, which has the signature
\verb|func main() => integer|.

\RequirementDef{RuntimeReturn}
When evaluation from the entry point returns (without throwing an exception)
the runtime should pass the return value to the hosting environment.
%
For example: an ASL runtime for native executables may use the return value
of \texttt{main} as the exit status of the process.
%
By convention a return value of zero indicates success and a return value of
one indicates failure.
%
An alternative (non-default) entry point may be specified by the user if
supported by the runtime. Not all runtimes may support alternative entry points.

\RequirementDef{RuntimeUncaught}
Uncaught exceptions cause termination of the application by the runtime.
If an exception is thrown from the entry point, it is an uncaught exception.
The runtime should signal an error to the hosting environment.

\RequirementDef{Printing}
Output may be generated using the \printstatementterm, which makes
best efforts to print them to diagnostic output.
