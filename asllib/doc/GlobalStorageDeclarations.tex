\chapter{Global Storage Declarations\label{chap:GlobalStorageDeclarations}}

Global storage declarations are grammatically derived from $\Ndecl$ via the subset of productions shown in
\secref{GlobalStorageDeclarationsSyntax} and represented as ASTs via the production of $\decl$
shown in \secref{GlobalStorageDeclarationsAbstractSyntax}.
%
Global storage declarations are typed by $\declareglobalstorage$, which is defined in \nameref{sec:TypingRule.DeclareGlobalStorage}.
%
The semantics of a list of global storage declarations is defined in \nameref{sec:SemanticsRule.EvalGlobals},
where the list is ordered via \nameref{sec:SemanticsRule.BuildGlobalEnv}.
The semantics of a single global storage declarations is defined in \nameref{sec:SemanticsRule.DeclareGlobal}.

%%%%%%%%%%%%%%%%%%%%%%%%%%%%%%%%%%%%%%%%%%%%%%%%%%%%%%%%%%%%%%%%%%%%%%%%%%%%%%%%
\section{Syntax\label{sec:GlobalStorageDeclarationsSyntax}}
%%%%%%%%%%%%%%%%%%%%%%%%%%%%%%%%%%%%%%%%%%%%%%%%%%%%%%%%%%%%%%%%%%%%%%%%%%%%%%%%
\begin{flalign*}
\Ndecl  \derivesinline\ & \Nstoragekeyword \parsesep \Nignoredoridentifier \parsesep \option{\Tcolon \parsesep \Nty} \parsesep \Teq \parsesep &\\
        & \wrappedline\ \Nexpr \parsesep \Tsemicolon &\\
        |\ & \productionname{globaluninitvar}{global\_uninit\_var}\ \Tvar \parsesep \Nignoredoridentifier \parsesep \Tcolon \parsesep \Nty \parsesep \Tsemicolon&\\
        |\ & \productionname{globalpragma}{global\_pragma}\ \Tpragma \parsesep \Tidentifier \parsesep \Clist{\Nexpr} \parsesep \Tsemicolon&
\end{flalign*}

\begin{flalign*}
\Nstoragekeyword \derivesinline\ & \Tlet \;|\; \Tconstant \;|\; \Tvar \;|\; \Tconfig&\\
\Nignoredoridentifier \derivesinline\ & \Tminus \;|\; \Tidentifier &
\end{flalign*}

%%%%%%%%%%%%%%%%%%%%%%%%%%%%%%%%%%%%%%%%%%%%%%%%%%%%%%%%%%%%%%%%%%%%%%%%%%%%%%%%
\section{Abstract Syntax\label{sec:GlobalStorageDeclarationsAbstractSyntax}}
%%%%%%%%%%%%%%%%%%%%%%%%%%%%%%%%%%%%%%%%%%%%%%%%%%%%%%%%%%%%%%%%%%%%%%%%%%%%%%%%
\begin{flalign*}
\decl \derives\ & \DGlobalStorage(\globaldecl) &\\
\globaldecl \derives\ &
{\left\{
  \begin{array}{rcl}
  \GDkeyword &:& \globaldeclkeyword, \\
  \GDname &:& \identifier,\\
  \GDty &:& \ty?,\\
  \GDinitialvalue &:& \expr?
  \end{array}
  \right\}
 } &\\
 \globaldeclkeyword \derives\ & \GDKConstant \;|\; \GDKConfig \;|\; \GDKLet \;|\; \GDKVar &
\end{flalign*}

\subsubsection{ASTRule.GlobalDecl}
The relation
\[
  \builddecl : \overname{\parsenode{\Ndecl}}{\vparsednode} \;\aslrel\; \overname{\decl}{\vastnode}
\]
transforms a parse node $\vparsednode$ into an AST node $\vastnode$.

\hypertarget{build-globalstorage}{}
\begin{mathpar}
\inferrule[global\_storage]{
  \buildstoragekeyword(\keyword) \astarrow \astof{\keyword}\\
  \buildoption[\buildasty](\tty) \astarrow \ttyp\\
  \buildexpr(\vinitialvalue) \typearrow \astof{\vinitialvalue}
}
{
  {
      \builddecl\left(\overname{\Ndecl\left(
      \begin{array}{r}
      \namednode{\vkeyword}{\Nstoragekeyword}, \namednode{\name}{\Nignoredoridentifier},  \\
  \wrappedline\ \namednode{\tty}{\option{\Nasty}}, \Teq, \namednode{\vinitialvalue}{\Nexpr}, \Tsemicolon
      \end{array}
  \right)}{\vparsednode}\right)
  } \astarrow \\
  {
    \overname{
  \DGlobalStorage\left(\left\{
    \begin{array}{rcl}
    \GDkeyword &:& \astof{\vkeyword},\\
    \GDname &:& \astof{\name},\\
    \GDty &:& \ttyp,\\
    \GDinitialvalue &:& \astof{\vinitialvalue}
  \end{array}
  \right\}\right)
  }{\vastnode}
  }
}
\end{mathpar}

\hypertarget{build-globaluninitvar}{}
\begin{mathpar}
\inferrule[global\_uninit\_var]{
  \buildignoredoridentifier(\cname) \astarrow \name
}
{
  {
    \begin{array}{r}
      \builddecl(\overname{\Ndecl(\Tvar, \namednode{\cname}{\Nignoredoridentifier}, \Nasty, \Tsemicolon)}{\vparsednode}) \astarrow
    \end{array}
  } \\
  \overname{\DGlobalStorage(\{\GDkeyword: \GDKVar, \GDname: \name, \GDty: \langle\astof{\Nasty}\rangle, \GDinitialvalue: \None\})}{\vastnode}
}
\end{mathpar}

\subsubsection{ASTRule.StorageKeyword \label{sec:ASTRule.StorageKeyword}}
\hypertarget{build-storagekeyword}{}
The function
\[
\buildstoragekeyword(\overname{\parsenode{\Nstoragekeyword}}{\vparsednode}) \;\aslto\;
  \overname{\globaldeclkeyword}{\vastnode}
\]
transforms a parse node $\vparsednode$ into an AST node $\vastnode$.

\begin{mathpar}
\inferrule[let]{}{
  \buildstoragekeyword(\overname{\Nstoragekeyword(\Tlet)}{\vparsednode}) \astarrow \overname{\GDKLet}{\vastnode}
}
\end{mathpar}

\begin{mathpar}
\inferrule[constant]{}{
  \buildstoragekeyword(\overname{\Nstoragekeyword(\Tconstant)}{\vparsednode}) \astarrow \overname{\GDKConstant}{\vastnode}
}
\end{mathpar}

\begin{mathpar}
\inferrule[var]{}{
  \buildstoragekeyword(\overname{\Nstoragekeyword(\Tvar)}{\vparsednode}) \astarrow \overname{\GDKVar}{\vastnode}
}
\end{mathpar}

\begin{mathpar}
\inferrule[config]{}{
  \buildstoragekeyword(\overname{\Nstoragekeyword(\Tconfig)}{\vparsednode}) \astarrow \overname{\GDKConfig}{\vastnode}
}
\end{mathpar}

\subsubsection{ASTRule.IgnoredOrIdentifier\label{sec:ASTRule.IgnoredOrIdentifier}}
\hypertarget{build-ignoredoridentifier}{}
The relation
\[
\buildfuncargs(\overname{\parsenode{\Nignoredoridentifier}}{\vparsednode}) \;\aslrel\;
  \overname{\identifier}{\vastnode}
\]
transforms a parse node $\vparsednode$ into an AST node $\vastnode$.

\begin{mathpar}
\inferrule[discard]{
  \id \in \identifier \text{ is fresh}
}{
  \buildignoredoridentifier(\overname{\Nignoredoridentifier(\Tminus)}{\vparsednode}) \astarrow
  \overname{\id}{\vastnode}
}
\end{mathpar}

\begin{mathpar}
\inferrule[id]{}{
  \buildignoredoridentifier(\overname{\Nignoredoridentifier(\Tidentifier(\id))}{\vparsednode}) \astarrow
  \overname{\id}{\vastnode}
}
\end{mathpar}

%%%%%%%%%%%%%%%%%%%%%%%%%%%%%%%%%%%%%%%%%%%%%%%%%%%%%%%%%%%%%%%%%%%%%%%%%%%%%%%%
\section{Typing}
%%%%%%%%%%%%%%%%%%%%%%%%%%%%%%%%%%%%%%%%%%%%%%%%%%%%%%%%%%%%%%%%%%%%%%%%%%%%%%%%
We also define the following helper rules:
\begin{itemize}
  \item TypingRule.DeclareGlobalStorage (see \secref{TypingRule.DeclareGlobalStorage})
  \item TypingRule.AnnotateTypeOpt (see \secref{TypingRule.AnnotateTypeOpt})
  \item TypingRule.AnnotateExprOpt (see \secref{TypingRule.AnnotateExprOpt})
  \item TypingRule.AnnotateInitType (see \secref{TypingRule.AnnotateInitType})
  \item TypingRule.AddGlobalStorage (see \secref{TypingRule.AddGlobalStorage})
\end{itemize}

\subsubsection{TypingRule.DeclareGlobalStorage \label{sec:TypingRule.DeclareGlobalStorage}}
\hypertarget{def-declareglobalstorage}{}
The function
\[
  \declareglobalstorage(\overname{\staticenvs}{\tenv} \aslsep \overname{\globaldecl}{\gsd})
  \aslto
  \overname{\staticenvs}{\newtenv} \aslsep \overname{\globaldecl}{\newgsd}
  \cup
  \overname{\TTypeError}{\TypeErrorConfig}
\]
annotates the global storage declaration $\gsd$ in the static environment $\tenv$,
yielding a modified static environment $\newtenv$ and annotated global storage declaration $\newgsd$.
\ProseOtherwiseTypeError

\subsection{Prose}
All of the following apply:
\begin{itemize}
  \item $\gsd$ is a global storage declaration with keyword $\keyword$, initial value \\ $\initialvalue$,
        \optional\ type $\tyopt$, and name $\name$;
  \item checking that $\name$ is not already declared in the global environment of $\tenv$ yields $\True$\ProseOrTypeError;
  \item annotating the \optional\ type $\tyopt$ in $\tenv$ via $\annotatetypeopt$ yields \\
        $\tyoptp$\ProseOrTypeError;
  \item annotating the \optional\ expression $\initialvalue$ in $\tenv$ via $\annotateexpropt$ yields
        $(\initialvaluetype, \initialvaluep)$\ProseOrTypeError;
  \item choosing the correct type between $\initialvaluetype$ and $\tyoptp$ in $tenv$ via $\annotateinittype$ yields
        $\declaredt$;
  \item adding a global storage element with name $\name$, global declaration keyword \\ $\keyword$ and type $\declaredt$
        to $\tenv$ via $\addglobalstorage$ yields $\tenvone$\ProseOrTypeError;
  \item One of the following applies:
  \begin{itemize}
    \item All of the following apply (\textsc{constant}):
    \begin{itemize}
      \item $\keyword$ is $\GDKConstant$ and therefore $\initialvalue$ is some expression $\ve$ (the ASL parser guarantees
            that the expression exists);
      \item symbolically simplifying $\ve$ in $\tenvone$ via $\reduceconstants$ yields the literal $\vv$\ProseOrTypeError;
      \item $\tenvtwo$ is $\tenvone$ with its $\constantvalues$ component updated by binding $\name$ to $\vv$;
      \item $\newgsd$ is $\gsd$ with its type component as $\tyoptp$;
      \item $\newtenv$ is $\tenvtwo$.
    \end{itemize}

    \item All of the following apply (\textsc{non\_constant}):
    \begin{itemize}
      \item $\keyword$ is not $\GDKConstant$;
      \item $\newtenv$ is $\tenvone$.
    \end{itemize}
  \end{itemize}
  \item $\newgsd$ is $\gsd$ with its type component as $\tyoptp$ and $\initialvalue$ component as $\initialvaluep$;
\end{itemize}
\subsection{Formally}
\begin{mathpar}
\inferrule[constant]{
  \gsd \eqname \{
    \GDkeyword : \keyword,
    \GDinitialvalue : \initialvalue,
    \GDty : \tyopt,
    \GDname: \name
  \}\\
\checkvarnotingenv{\tenv, \name} \typearrow \True \OrTypeError\\\\
\annotatetypeopt(\tenv, \tyopt) \typearrow \tyoptp \OrTypeError\\\\
{
  \begin{array}{r}
\annotateexpropt(\tenv, \initialvalue) \typearrow \\ (\initialvaluetype, \initialvaluep) \OrTypeError
  \end{array}
}\\
\annotateinittype(\tenv, \initialvaluetype, \tyoptp) \typearrow \declaredt\\
\addglobalstorage(\tenv, \name, \keyword, \declaredt) \typearrow \tenvone \OrTypeError\\\\
\commonprefixline\\\\
\keyword = \GDKConstant\\
\initialvaluep \eqname \langle \ve \rangle\\
\reduceconstants(\tenvone, \ve) \typearrow \vv \OrTypeError\\\\
\tenvtwo \eqdef (G^\tenvone.\constantvalues[\name\mapsto\vv], L^\tenvone)\\
{
\newgsd \eqdef \left\{
  \begin{array}{rcl}
  \GDkeyword &:& \keyword, \\
  \GDinitialvalue &:& \initialvalue, \\
  \GDty &:& \tyoptp, \\
  \GDname &:& \name
  \end{array}
\right\}
}
}{
  \declareglobalstorage(\tenv, \gsd) \typearrow (\overname{\tenvtwo}{\newtenv}, \newgsd)
}
\end{mathpar}

\begin{mathpar}
\inferrule[non\_constant]{
  \gsd \eqname \{
    \GDkeyword : \keyword,
    \GDinitialvalue : \initialvalue,
    \GDty : \tyopt,
    \GDname: \name
  \}\\
\checkvarnotingenv{\tenv, \name} \typearrow \True \OrTypeError\\\\
\annotatetypeopt(\tenv, \tyopt) \typearrow \tyoptp \OrTypeError\\\\
{
  \begin{array}{r}
\annotateexpropt(\tenv, \initialvalue) \typearrow \\ (\initialvaluetype, \initialvaluep) \OrTypeError
  \end{array}
}\\
\annotateinittype(\tenv, \initialvaluetype, \tyoptp) \typearrow \declaredt\\
\addglobalstorage(\tenv, \name, \keyword, \declaredt) \typearrow \tenvone \OrTypeError\\\\
\commonprefixline\\\\
\keyword \neq \GDKConstant\\
{
\newgsd \eqdef \left\{
  \begin{array}{rcl}
  \GDkeyword &:& \keyword, \\
  \GDinitialvalue &:& \initialvalue, \\
  \GDty &:& \tyoptp, \\
  \GDname &:& \name
  \end{array}
\right\}
}
}{
  \declareglobalstorage(\tenv, \gsd) \typearrow (\overname{\tenvone}{\newtenv}, \newgsd)
}
\end{mathpar}\lrmcomment{This relates to \identr{YSPM} and \identr{FWQM}.}
\CodeSubsection{\DeclareGlobalStorageBegin}{\DeclareGlobalStorageEnd}{../Typing.ml}

\subsubsection{TypingRuleAddGlobalStorage \label{sec:TypingRule.AddGlobalStorage}}
\hypertarget{def-addglobalstorage}{}
The function
\[
  \addglobalstorage(
    \overname{\staticenvs}{\tenv} \aslsep
    \overname{\identifier}{\name} \aslsep
    \overname{\GDkeyword}{\keyword} \aslsep
    \overname{\ty}{\declaredt}
  )
  \aslto
    \overname{\staticenvs}{\newtenv} \aslsep
  \cup \overname{\TTypeError}{\TypeErrorConfig}
\]
modifies the static environment $\tenv$ by adding a global storage
for the identifier $\name$ with global storage keyword $\keyword$ and type $\declaredtypes$,
resulting in the environment $\newtenv$.
\ProseOtherwiseTypeError

\subsection{Prose}
All of the following apply:
\begin{itemize}
  \item checking that $\name$ is not declared in the global environment of $\tenv$ yields $\True$\ProseOrTypeError;
  \item $\newtenv$ is $\tenv$ with its $\globalstoragetypes$ component updated by binding $\name$ to
        $(\declaredt, \keyword)$.
\end{itemize}

\subsection{Formally}
\begin{mathpar}
\inferrule{
  \checkvarnotingenv{\tenv, \name} \typearrow \True \OrTypeError\\\\
  \newtenv \eqdef (G^\tenv.\globalstoragetypes[\name \mapsto (\declaredt, \keyword)], L^\tenv)
}
{
  \addglobalstorage(\tenv, \name, \keyword, \declaredt) \typearrow \newtenv
}
\end{mathpar}

\subsubsection{TypingRule.CheckVarNotInGEnv}
\hypertarget{def-checkvarnotingenv}{}
The function
\[
  \checkvarnotingenv{\overname{\staticenvs}{\tenv} \aslsep \overname{\Strings}{\id}}
  \aslto \{\True\} \cup \overname{\TTypeError}{\TypeErrorConfig}
\]
checks whether $\id$ is already declared in the global environment of $\tenv$.
If it is, the result is a type error, and otherwise the result is $\True$.

\subsubsection{Prose}
All of the following apply:
\begin{itemize}
  \item $\vb$ is $\True$ if and only if one of the following applies:
  \begin{itemize}
    \item $\id$ is declared as a global identifier in $\tenv$;
    \item $\id$ is declared as a subprogram in $\tenv$;
    \item $\id$ is declared as a type in $\tenv$.
  \end{itemize}

  \item checking whether $\vb$ is $\False$ yields $\True$ or a type error indicating
        that $\id$ has already been declared, thereby short-circuiting the rule.
\end{itemize}

\subsubsection{Formally}
\begin{mathpar}
\inferrule{
{
  \begin{array}{rl}
  \vb \eqdef  & G^\tenv.\globalstoragetypes(\id) \neq \bot \lor\\
              & G^\tenv.\subprograms(\id) \neq \bot\ \lor\\
              & G^\tenv.\declaredtypes(\id) \neq \bot
  \end{array}
}\\
\checktrans{\neg\vb}{IdentifierAlreadyDeclared} \checktransarrow \True \OrTypeError
}{
  \checkvarnotingenv{\tenv, \id} \typearrow \True
}
\end{mathpar}

%%%%%%%%%%%%%%%%%%%%%%%%%%%%%%%%%%%%%%%%%%%%%%%%%%%%%%%%%%%%%%%%%%%%%%%%%%%%%%%%
\section{Semantics\label{sec:GlobalStorageDeclarationsSemantics}}
%%%%%%%%%%%%%%%%%%%%%%%%%%%%%%%%%%%%%%%%%%%%%%%%%%%%%%%%%%%%%%%%%%%%%%%%%%%%%%%%

We now define the following relations:
\begin{itemize}
  \item SemanticsRule.DeclareGlobal \secref{SemanticsRule.DeclareGlobal};
  \item SemanticsRule.BaseValue \secref{SemanticsRule.BaseValue};
\end{itemize}

\subsubsection{SemanticsRule.EvalGlobals \label{sec:SemanticsRule.EvalGlobals}}
The relation
\hypertarget{def-evalglobals}{}
\[
  \evalglobals(\overname{\decl^*}{\vdecls}, (\overname{\overname{\envs}{\env} \times \overname{\XGraphs}{\vgone}}{\envm}))
  \;\aslrel\; \overname{(\envs \times \XGraphs)}{C} \cup \overname{\TError}{\ErrorConfig}
\]
updates the input environment and execution graph by initializing the global storage declarations,
either from their initializing expression or from the base value defined for their type as per \secref{SemanticsRule.BaseValue}.

\subsubsection{Prose}
One of the following applies:
\begin{itemize}
  \item All of the following apply (\textsc{empty}):
  \begin{itemize}
    \item there are no declarations of global variables;
    \item the result is $\envm$.
  \end{itemize}

  \item All of the following apply (\textsc{with\_initial\_value}):
  \begin{itemize}
    \item $\vdecls$ has $\vd$ as its head and $\vdecls'$ as its tail;
    \item $d$ is the AST node for declaring a global storage element with initial value $\ve$,
    name $\name$, and type $\vt$;
    \item $\envm$ is the environment-execution graph pair $(\env, \vgone)$;
    \item evaluating the side-effect-free expression $\ve$ in $\env$ as per \secref{SemanticsRule.ESideEffectFreeExpr} \\
    is $(\vv, \vgtwo)$\ProseOrError;
    \item declaring the global $\name$ with value $\vv$ in $\env$ gives $\envtwo$;
    \item evaluating the remaining global declarations $\vdecls'$ with the environment $\envtwo$ and the execution graph
    that is the ordered composition of $\vgone$ and $\vgtwo$ with the $\aslpo$ label gives $C$;
    \item the result of the entire evaluation is $C$.
  \end{itemize}

  \item All of the following apply (\textsc{no\_initial\_value}):
  \begin{itemize}
    \item $\vdecls$ has $\vd$ as its head and $\vdecls'$ as its tail;
    \item $d$ is the AST node for declaring a global storage element with no initial value,
    name $\name$, and type $\vt$;
    \item $\envm$ is the environment-execution graph pair $(\env, \vgone)$;
    \item the base value of type $\vt$ in $\env$ is $(\vv, \vgtwo)$\ProseOrError;
    \item declaring the global $\name$ with value $\vv$ in $\env$ gives $\envtwo$;
    \item evaluating the remaining global declarations $\vdecls'$ with the environment $\envtwo$ and the execution graph
    that is the ordered composition of $\vgone$ and $\vgtwo$ with the $\aslpo$ label gives $C$;
    \item the result of the entire evaluation is $C$.
  \end{itemize}
\end{itemize}
\subsubsection{Example}

\subsubsection{Formally}
\begin{mathpar}
\inferrule[empty]{
\vdecls \eqname \emptylist
}{
\evalglobals(\vdecls, \envm) \evalarrow \envm
}
\end{mathpar}

\begin{mathpar}
\inferrule[with\_initial\_value]{
\vdecls \eqname [\vd] \concat \vdecls'\\
\vd \eqname \DGlobalStorage(\{ \text{initial\_value}=\langle\ve\rangle, \text{name}:\name, \text{ty}:\vt, \ldots \})\\
\envm \eqname (\env, \vgone)\\
\evalexprsef{\env, \ve} \evalarrow (\vv, \vgtwo) \OrDynError\\\\
\declareglobal(\name, \vv, \env) \evalarrow \envtwo\\
\evalglobals(\vdecls', (\envtwo, \ordered{\vgone}{\aslpo}{ \vgtwo })) \evalarrow C
}{
\evalglobals(\vdecls, \envm) \evalarrow C
}
\end{mathpar}

\begin{mathpar}
\inferrule[no\_initial\_value]{
\vdecls \eqname [\vd] \concat \vdecls'\\
\vd \eqname \DGlobalStorage(\{ \text{initial\_value}:\None, \text{name}:\name, \text{ty}:\vt, \ldots \})\\
\envm \eqname (\env, \vgone)\\
\basevalue(\env, \vt) \evalarrow (\vv, \vgtwo) \OrDynError\\\\
\declareglobal(\name, \vv, \env) \evalarrow \envtwo\\
\evalglobals(\vdecls', (\envtwo, \ordered{\vgone}{\aslpo}{ \vgtwo })) \evalarrow C
}{
\evalglobals(\vdecls, \envm) \evalarrow C
}
\end{mathpar}
% \CodeSubsection{\EvalGlobalsBegin}{\EvalGlobalsEnd}{../Interpreter.ml}

\subsubsection{SemanticsRule.DeclareGlobal \label{sec:SemanticsRule.DeclareGlobal}}
\subsubsection{Prose}
The relation
\hypertarget{def-declareglobal}{}
\[
  \declareglobal(\overname{\Identifiers}{\name} \aslsep \overname{\vals}{\vv} \aslsep \overname{\envs}{\env}) \;\aslrel\; \overname{\envs}{\newenv}
\]
updates the environment $\env$ by mapping $\name$ to $\vv$ as a global storage element.

\subsubsection{Formally}
\begin{mathpar}
  \inferrule{
    \env \eqname (\tenv, (G^\denv, L^\denv))\\
    \newenv \eqdef (\tenv, (G^\denv[\name\mapsto \vv], L^\denv))
  }
  { \declareglobal(\name, \vv, \env) \evalarrow \newenv  }
\end{mathpar}

\subsubsection{SemanticsRule.BaseValue \label{sec:SemanticsRule.BaseValue}}
The relation
\hypertarget{def-basevalue}{}
\[
  \basevalue(\overname{(\overname{\staticenvs}{\tenv} \times \overname{\dynamicenvs}{\denv})}{\env} \aslsep \overname{\ty}{\vt}) \;\aslrel\;
  (\overname{\vals}{\vv} \times \overname{\XGraphs}{\vg}) \cup \overname{\TError}{\ErrorConfig}
\]
returns the \emph{base value} of a type.
The result is an error configuration if a dynamic error is detected.

\hypertarget{def-tstruct}{}
\paragraph{Type Structure} To obtain the base value of a type, we first obtain its \emph{structure}, using the function
\[
  \tstruct(\overname{\staticenvs}{\tenv} \aslsep \overname{\ty}{\tty}) \aslto \overname{\ty}{\vt} \cup \overname{\TTypeError}{\TypeErrorConfig}
\]
\hypertarget{def-typeerrorconfig}{}
where $\TypeErrorConfig$ stands for a type error.

The structure of a type is the type that can hold the same set of values, but does not itself
contain any other type names.
This is essentially done by recursively replacing type names by their definition
(see \secref{TypingRule.structure}).
%
Since we assume the specification is well-typed (\secref{MeaningfulASLSpecifications}),
$\tstruct$ returns a valid type.

\subsubsection{Prose}
The base value of the type $\vt$ in the environment $\env$ is $\vv$,
as well as the execution graph $\vg$ that results
from evaluating any of the side-effect-free expressions appearing in $\vt$,
or an error, and one of the following applies:
\begin{itemize}
  \item all of the following apply (\textsc{boolean}):
  \begin{itemize}
    \item the structure of $\vt$ is the Boolean type;
    \item $\vv$ is the native Boolean "true" value;
    \item $\vg$ is the empty graph.
  \end{itemize}

  \item all of the following apply (\textsc{real}):
  \begin{itemize}
    \item the structure of $\vt$ is the real type;
    \item $\vv$ is the native real value $0$;
    \item $\vg$ is the empty graph.
  \end{itemize}

  \item all of the following apply (\textsc{string}):
  \begin{itemize}
    \item the structure of $\vt$ is the string type;
    \item $\vv$ is the \nativevalue\  for the empty string;
    \item $\vg$ is the empty graph.
  \end{itemize}

  \item all of the following apply (\textsc{bitvector}):
  \begin{itemize}
    \item the structure of $\vt$ is the bitvector with the length expression $\ve$;;
    \item evaluating the side-effect-free expression $\ve$ results in the \nativevalue\  $\length$
    and execution graph $\vg$\ProseOrError;
    \item $\vv$ is the bitvector of length $\length$ where all bits are $0$.
  \end{itemize}

  \item all of the following apply (\textsc{enum}):
  \begin{itemize}
    \item the structure of $\vt$ is the enumeration type where the first identifier is $\id_1$;
    \item $\vl$ is the literal associated with $\id_1$ in the static environment;
    \item $\vv$ is the \nativevalue\  literal for $\vl$;
    \item $\vg$ is the empty graph.
  \end{itemize}

  \item all of the following apply (\textsc{unconstrained\_integer}):
  \begin{itemize}
    \item the structure of $\vt$ is that of the unconstrained integer;
    \item $\vv$ is the \nativevalue\  integer $0$;
    \item $\vg$ is the empty graph.
  \end{itemize}

  \item all of the following apply (\textsc{well\_constrained\_integer}):
  \begin{itemize}
    \item the structure of $\vt$ is that of the well-constrained integer where the first constraint
    is exact with the expression $\ve$;
    \item $(\vv, \vg)$ is the result of evaluating the side-effect-free expression $\ve$.
  \end{itemize}

  \item all of the following apply (\textsc{record}):
  \begin{itemize}
    \item the structure of $\vt$ is that of a record or an exception;
    \item the base value of each field is obtained, and if any of the base values results in an error
    then the entire rule short-circuits with that error;
    \item $\vv$ is the \nativevalue\  record where each identifier in the record is mapped to its
    respective base value;
    \item $\vg$ is the parallel composition of the graphs resulting from the base value evaluation
    of all the fields.
  \end{itemize}

  \item all of the following apply (\textsc{tuple}):
  \begin{itemize}
    \item the structure of $\vt$ is that of a tuple of types;
    \item the base value of each type in the tuple is obtained, and if any of the base values results in an error
    then the entire rule short-circuits with that error;
    \item $\vv$ is the \nativevalue\  vector consisting of the base values in the order of the corresponding types
    of the tuple;
    \item $\vg$ is the parallel composition of the graphs resulting from the base value evaluation
    of all the tuple types.
  \end{itemize}

  \item all of the following apply (\textsc{array\_length\_global\_constant}):
  \begin{itemize}
    \item the structure of $\vt$ is that of an array with length expression $\elength$ and element type $\vvty$;
    \item $\elength$ is the value of a declared constant;
    \item $\elength$ is a variable expression with the variable name $\vx$;
    \item the constant value for $\vx$ in the static environment is the literal integer for $n$;
    \item the base value of $\vvty$ in $\env$ is $(\velem, \vg)$\ProseOrError;
    \item $\vv$ is the \nativevalue\  vector of length $n$ where each element is $\velem$;
  \end{itemize}

  \item all of the following apply (\textsc{array\_length\_expression}):
  \begin{itemize}
    \item the structure of $\vt$ is that of an array with length expression $\elength$ and element type $\vvty$;
    \item $\elength$ is not the value of a declared constant;
    \item the base value of $\vvty$ in $\env$ is $(\velem, \vg)$\ProseOrError;
    \item evaluating the side-effect-free expression $\elength$ in the environment $\env$
          results in $(\vlength, \vgtwo)$\ProseOrError;
    \item $\vlength$ is the \nativevalue\  integer for $n$;
    \item $\vv$ is the \nativevalue\  vector of length $n$ where each element is $\velem$;
    \item $\vg$ is the ordered composition of $\vgone$ and $\vgtwo$ with the $\asldata$ edge.
  \end{itemize}
\end{itemize}

\subsubsection{Formally}
Evaluating the inner expressions of the type $\vt$ is done via the relation\\
$\evalexprsef$~\ref{sec:SemanticsRule.ESideEffectFreeExpr}.
If evaluating an inner expression results in an error, there is no base value and an error configuration is returned.

\begin{mathpar}
\inferrule[bool]{
  \tstruct(\tenv, \vt) \typearrow \TBool
}{
  \basevalue((\tenv, \denv), \vt) \evalarrow (\overname{\nvbool(\True)}{\vv}, \overname{\emptygraph}{\vg})
}
\and
\inferrule[real]{
  \tstruct(\tenv, \vt) \typearrow \TReal
}{
  \basevalue((\tenv, \denv), \vt) \evalarrow (\overname{\nvreal(0)}{\vv}, \overname{\emptygraph}{\vg})
}
\and
\inferrule[string]{
  \tstruct(\tenv, \vt) \typearrow \TString
}{
  \basevalue((\tenv, \denv), \vt) \evalarrow (\overname{\nvliteral{\lstring(\emptylist)}}{\vv}, \overname{\emptygraph}{\vg})
}
\end{mathpar}

The base value of a bitvector is a bitvector \nativevalue\  consisting of a sequence of zeros
of the length specified by the type (\ve). If the length is $0$, the bitvector consists of an
empty sequence:
\begin{mathpar}
\inferrule[bitvector]{
  \env \eqname (\tenv, \denv)\\
  \tstruct(\tenv, \vt) \typearrow \TBits(\ve, \Ignore)\\
  \evalexprsef{\env, \ve} \evalarrow (\nvint(\length), \vg) \OrDynError
}{
  \basevalue(\env, \vt) \evalarrow (\overname{\nvbitvector(\overbrace{0\ldots 0}^{\length})}{\vv}, \vg)
}
\end{mathpar}

\hypertarget{def-constantvalues}{}
The base value of an enumeration is obtained from its first declared literal.
Accessing this literal is done via the \constantvalues map in the
global component of the static environment:
\begin{mathpar}
\inferrule[enum]{
  \tstruct(\vt) \typearrow \TEnum(\id_{1..k})\\
  \env \eqname (\tenv, \denv)\\
  \tenv \eqname (G^\tenv, L^\tenv)\\
  G^\tenv.\constantvalues(\id_1) \eqname \vl
}{
  \basevalue(\env, \vt) \evalarrow (\overname{\nvliteral{\vl}}{\vv}, \overname{\emptygraph}{\vg})
}
\end{mathpar}

\begin{mathpar}
\inferrule[integer\_unconstrained]{
  \tstruct(\tenv, \vt) \typearrow \TInt(\unconstrained)
}{
  \basevalue((\tenv, \denv), \vt) \evalarrow (\overname{\nvint(0)}{\vv}, \overname{\emptygraph}{\vg})
}
\and
\inferrule[integer\_constraint\_exact]{
  \env \eqname (\tenv, \denv)\\
  \tstruct(\tenv, \vt) \typearrow \TInt(\wellconstrained([\constraintexact(\ve)] \concat \Ignore))\\
  \evalexprsef{\env, \ve} \evalarrow C
}{
  \basevalue(\env, \vt) \evalarrow C
}
\and
\and
\inferrule[integer\_constraint\_range]{
  \env \eqname (\tenv, \denv)\\
  \tstruct(\tenv, \vt) \typearrow \TInt(\wellconstrained([\constraintrange(\ve, \Ignore)] \concat \Ignore))\\
  \evalexprsef{\env, \ve} \evalarrow C
}{
  \basevalue(\env, \vt) \evalarrow C
}
\end{mathpar}

\begin{mathpar}
\inferrule[record]{
  \env \eqname (\tenv, \denv)\\
  \tstruct(\tenv, \vt) \typearrow L([i=1..k: (\id_i, \vt_i)])\\
  L \in \{\TRecord, \TException\}\\
  i=1..k: \basevalue(\env, \vt_i) \evalarrow (\vv_i, \vg_i) \OrDynError
}{
  \basevalue(\env, \vt) \evalarrow
  (\overname{\nvrecord{\{i=1..k: \id_i\mapsto \vv_i\}}}{\vv}, \overname{\vg_1 \parallelcomp \ldots \parallelcomp \vg_k}{\vg})
}
\end{mathpar}

\begin{mathpar}
\inferrule[tuple]{
  \env \eqname (\tenv, \denv)\\
  \tstruct(\tenv, \vt) \typearrow \TTuple([i=1..k: \vt_i])\\
  i=1..k: \basevalue(\env, \vt_i) \evalarrow (\vv_i, \vg_i) \OrDynError
}{
  \basevalue(\env, \vt) \evalarrow (\overname{\nvvector{\vv_{1..k}}}{\vv}, \overname{\vg_1 \parallelcomp \ldots \parallelcomp \vg_k}{\vg})
}
\end{mathpar}

\newcommand\isconstant[0]{\hyperlink{def-isconstant}{\texttt{is\_contant}}}
\hypertarget{def-isconstant}{}
The predicate $\isconstant$ checks whether the expression $\ve$ is a variable
declared as a constant.
\begin{mathpar}
\inferrule[NotEVar]{
  \env \eqname (\tenv, \denv)\\
  \tenv \eqname (G^\tenv, L^\tenv)\\
  \astlabel(\ve) \neq \EVar
}{
  \isconstant(\env, \ve) \rightarrow \False
}
\and
\inferrule[EVar]{
  \env \eqname (\tenv, \denv)\\
  \tenv \eqname (G^\tenv, L^\tenv)\\
  \astlabel(\ve) = \EVar\\
  \ve\eqname\EVar(\vx)\\
  \vb \eqdef G^\tenv.\constantvalues(\vx) \neq \bot
}{
  \isconstant(\env, \ve) \rightarrow \vb
}
\end{mathpar}

\begin{mathpar}
\inferrule[array\_length\_global\_constant]{
  \env \eqname (\tenv, \denv)\\
  \tstruct(\tenv, \vt) \typearrow \TArray(\elength, \vvty)\\
  \basevalue(\env, \vvty) \evalarrow (\velem, \vg) \OrDynError\\\\
  \isconstant(\env, \elength) \rightarrow \True\\
  \elength \eqname \EVar(\vx)\\
  \env \eqname (\tenv, \denv)\\
  \tenv \eqname (G^\tenv, L^\tenv)\\
  G^\tenv.\constantvalues(\vx) \eqname \lint(n)
}{
  \basevalue(\env, \vt) \evalarrow (\overname{\nvvector{i=1..n: \velem}}{\vv}, \vg)
}
\and
\inferrule[array\_length\_expression]{
  \env \eqname (\tenv, \denv)\\
  \tstruct(\tenv, \vt) \typearrow \TArray(\elength, \vvty)\\
  \basevalue(\env, \vvty) \evalarrow (\velem, \vgone) \OrDynError\\\\
  \isconstant(\elength) \rightarrow \False\\
  \evalexprsef{\env, \elength} \evalarrow (\vlength, \vgtwo) \OrDynError\\\\
  \vlength \eqname \nvint(n)
}{
  \basevalue(\env, \vt) \evalarrow (\overname{\nvvector{i=1..n: \velem}}{\vv}, \overname{\ordered{\vgone}{\asldata}{\vgtwo}}{\vg})
}
\end{mathpar}
