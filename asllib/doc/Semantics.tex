\chapter{Semantics Definitions\label{chap:Semantics}}

The semantics of ASL define all valid behaviors of a given ASL specification.
More precisely, an ASL specification is first parsed into an \emph{abstract syntax tree},
or AST, for short. Second, a type checker analyzes the \emph{untyped AST} to determine whether it
is well-typed and, if successful, returns a \emph{static environment} and a \emph{typed AST}.
Otherwise, it returns a type error.

Tools such as interpreters, Verilog simulators, and verifiers can operate over the typed AST, based on the definition
of the semantics in this reference, to test and analyze a given specification.

\paragraph{Understanding the Semantics Formalization:}
We assume basic familiarity with the ASL language constructs.
The ASL semantics is defined in terms of its AST,
and as a consequence familiarity with the AST is required to understand the semantics.
The few components of the type system needed to understand the ASL semantics are explained in context.
The mathematical background needed to understand the mathematical formalization
of the ASL semantics appears in \chapref{FormalSystem} and \secref{semanticsbuildingblocks}.

\section{When Do ASL Specifications Have Meaning\label{sec:MeaningfulASLSpecifications}}
The ASL semantics defined here assign meaning only to \emph{well-typed specifications}.
That is, specifications for which the type-checker produces a static environment rather than
a type error.
Specifications that are not well-typed have no defined semantics.
In the rest of this reference, we assume well-typed specifications.

ASL admits non-determinism, for example via the \texttt{UNKNOWN} expression.
This means that a given specification might have (potentially infinitely) many
\hyperlink{def-derivationtree}{derivation trees}.

An ASL specification is \emph{terminating} when \underline{all} of its derivation trees are finite.

Although ASL does not require specifications to terminate, the semantics defined in this
reference assign meaning only to terminating specifications.
A future version of this reference, will assign meaning to non-terminating specifications.

\section{Basic Semantic Concepts}
The ASL semantics are given by relations between \emph{semantic configurations},
or \hyperlink{def-configuration}{\emph{semantic configurations}}~\cite{SemanticsWithApplicationsBook}, for short.
We refer to relations between semantic configurations as \emph{semantic relations}.
Semantic configurations encapsulate information needed to transition into other semantic configurations, such as:
\begin{itemize}
  \item a \emph{dynamic environment}, which binds variables to values;
  \item the typed AST node that needs to be evaluated;
  \item a \emph{concurrent execution graph}, as per a given memory model; and
  \item values resulting from evaluating expressions.
\end{itemize}
The semantic relations are constructively defined via \emph{semantic rules}.
These semantic rules are defined by induction over the typed AST.

\paragraph{Execution:}
A valid execution of an ASL specification transitions from an \emph{initial semantic configuration},
which consists of the given specification and the standard library specification, to an output semantic configuration
consisting of an output value and a concurrent execution graph.

\paragraph{Primitive Subprograms:}
The semantics of ASL are parameterized by a set of primitive subprograms ---
subprograms whose implementation is not defined by ASL statements and whose effect on the dynamic environment
is defined externally. Critically, access to memory is given by primitive subprograms.

We define two types of semantics --- \emph{sequential semantics} and \emph{concurrent semantics}.

\paragraph{Concurrent Semantics:}
The concurrent semantics operate over concurrent execution graphs.
Intuitively, these graphs define Read Effects and Write Effects to variables and constraints over those effects.
Together with the constraints that define a given memory model (such as the ARM memory model~\cite{AlglaveDGHM21}),
these graphs axiomatically define
the valid interactions of shared variables of a given specification.

\paragraph{Sequential Semantics:}
The sequential semantics correspond to executing an ASL specification in the context of a single thread
of execution; notice that ASL does not contain any concurrency constructs.
%
Technically, the sequential semantics are defined by omitting the concurrent execution graph components
from all semantic configurations.

\section{Semantics Building Blocks}
\label{sec:semanticsbuildingblocks}
This section defines the mathematical types over which our semantics are defined.
An \hyperlink{eval-example}{example} of semantic evaluation appears at the end.

\section{Semantic Configurations\label{sec:SemanticConfigurations}}

Semantic configurations express intermediate states related by \emph{semantic relations}.
%
More precisely, semantic relations relate two distinct sets of semantic configurations ---
\emph{input semantic configurations} and \emph{output semantic configurations}.
Input semantic configurations consist of an environment and an AST node.
Output semantic configurations consist of an output environment, values,
and concurrent execution graphs.
%
Semantic configurations wrap together elements such as environments and AST nodes
and associate them with a \emph{configuration domain}. Input semantic configuration domains
determine the semantic relation they pertain to, while output semantic configuration
domains distinguish between conceptually different kinds of outputs, for example
ones where an exception was raised, ones when a dynamic error occurred, etc.

We now explain the components over which semantic configurations are defined:
\begin{itemize}
    \item Native values.
    \item Static Environments, which consist of the information inferred
            by the type-checker for the specification.
    \item Dynamic Environments (\defref{dynamicenvironments}) associate \nativevalues\ to variables.
    \item Concurrent Execution Graphs (\secref{concurrentexecutiongraphs}) track Read and Write Effects over variables.
\end{itemize}

\hypertarget{def-vals}{}
\hypertarget{def-nativevalue}{}
\hypertarget{def-nativevalues}{}
\section{Native Values\label{sec:nativevalues}}
Semantic evaluation binds values to storage elements when a specification is semantically evaluated.
To formalize this, we define the set of \emph{\nativevalues}, denoted $\vals$
(\texttt{NV} stands for Native Value).

\subsection{Prose}
The set of \nativevalues\ $\vals$ is the minimal set satisfying all of the following rules:
\begin{itemize}
  \item \textsc{basis set}: if $\vv$ is a literal then $\nvliteral{\vv}$ is a \nativevalue;
  \item \textsc{tuple values and array values}: if $\vl$ is a list of \nativevalues\ then $\nvvector{\vl}$ is a \nativevalue;
  \item \textsc{record values}: if $\vr$ is a finite function from identifiers to \nativevalues\ then $\nvrecord{\vr}$ is a \nativevalue.
\end{itemize}

\subsection{Formally}
\hypertarget{def-nvliteral}{}
\hypertarget{def-nvvector}{}
\hypertarget{def-nvrecord}{}
\begin{mathpar}
  \inferrule[(Basis Set: Integers, Reals, Booleans, Strings, and Bitvectors)]
  {\vv \in \literal}
  { \nvliteral{\vv} \in \vals }
  \and
  \inferrule[(Tuple Values and Array Values)]{\vvl \in \vals^*}
  { \nvvector{\vvl} \in \vals }
  \and
  \inferrule[(Record Values)]
  {\vr : \Identifiers \rightarrowfin \vals}
  { \nvrecord{\vr} \in \vals }
\end{mathpar}

We define the following shorthands for \nativevalue\ literals:
\hypertarget{def-nvint}{}
\[
\begin{array}{rcl}
\nvint(z)       &\triangleq& \nvliteral{\lint(z)}           \hypertarget{def-nvbool}{}\\
\nvbool(b)      &\triangleq& \nvliteral{\lbool(b)}          \hypertarget{def-nvreal}{}\\
\nvreal(r)      &\triangleq& \nvliteral{\lreal(r)}          \hypertarget{def-nvstring}{}\\
\nvstring(s)    &\triangleq& \nvliteral{\lstring(s)}        \hypertarget{def-nvbitvector}{}\\
\nvbitvector(v) &\triangleq& \nvliteral{\lbitvector(v)}\\
\end{array}
\]

We define the following types of \nativevalues:
\hypertarget{def-tint}{}
\[
\begin{array}{rcl}
  \tint       &\triangleq& \{ \nvint(z) \;|\; z \in \Z\}                                        \hypertarget{def-tbool}{}\\
  \tbool      &\triangleq& \{ \nvbool(\True), \nvbool(\False) \}                                \hypertarget{def-treal}{}\\
  \treal      &\triangleq& \{ \nvreal(r) \;|\; r \in \Q\}                                       \hypertarget{def-tstring}{}\\
  \tstring    &\triangleq& \{ \nvstring(s) \;|\; s \in \Strings\}  \hypertarget{def-tbitvector}{}\\
  \tbitvector &\triangleq& \{ \nvbitvector(\textit{bits}) \;|\; \textit{bits} \in \{0,1\}^*\}   \hypertarget{def-tvector}{}\\
  \tvector    &\triangleq& \{ \nvvector{\textit{vals}} \;|\; \textit{vals} \in \vals^*\}        \hypertarget{def-trecord}{}\\
  \trecord  &\triangleq& \{ \nvrecord{\fieldmap} \;|\; \fieldmap \in \Identifiers\rightarrowfin\vals\}\\
\end{array}
\]

\hypertarget{def-dynamicenvs}{}
\begin{definition}[Dynamic Environments\label{def:dynamicenvironments}]
A \emph{sequential dynamic environment}, or \\ \emph{dynamic environment}, for short, is a structure which,
associates \nativevalues\ to variables.
Formally, a sequential environment $\denv \in \dynamicenvs$ is a pair consisting of a partial function
(see \defref{PartialFunction}) from global
variable names to their \nativevalue, and a partial function from local variable names to their
\nativevalues:
\[
\begin{array}{rcl}
\dynamicenvs        &\triangleq&  \globaldynamicenvs \times \localdynamicenvs \\
\globaldynamicenvs  &\triangleq&  (\Identifiers \partialto \vals) \\
\localdynamicenvs   &\triangleq&  (\Identifiers \partialto \vals) \\
\end{array}
\]
\end{definition}

\paragraph{Static Environments}
A \emph{static environment} (see \secref{StaticEnvironments}) $\tenv \in \staticenvs$ (also referred to as a \emph{type environment})
is produced by the type-checker from the untyped AST.

We assume that the static environment supports the following functions:
\hypertarget{def-findfunc}{}
\hypertarget{def-typesatisfies}{}
\[
  \begin{array}{rcl}
    \findfunc       &:& \staticenvs \times \Identifiers \partialto \func\\
    \typesat  &:& \staticenvs \times (\ty \times \ty) \rightarrow \{\True, \False\}
  \end{array}
\]
The partial function $\findfunc$ returns the typed AST of the subprogram for a given identifier.
(Recall that ASL allows subprogram overloading so a name does not uniquely identify
a specific subprogram.
However, the type-checker renames each function uniquely so that it can be accessed based
on its name alone.)
%
The function $\typesat(\vt, \vs)$ returns true
if the type $\vt$ \emph{type-satisfies} the type $\vs$
(see \secref{TypingRule.TypeSatisfaction}).
This is used in matching a raised exception to a corresponding catch clause.

\hypertarget{def-envs}{}
\begin{definition}[Environments]
Environments pair static environments with dynamic environments:
$\envs = \staticenvs \times \dynamicenvs$.
\end{definition}
\hypertarget{def-env}{}
We write $\env \in \envs$ to range over environments.
%
From the perspective of the semantics, the static environment is immutable.
That is, all environments share the same static environment.

\subsection{Concurrent Execution Graphs\label{sec:concurrentexecutiongraphs}}

The concurrent semantics of an ASL specification utilize \emph{concurrent execution graphs}
(\emph{execution graphs}, for short),
which track the Read and Write Effects over variables, yielded by the sequential semantics,
and the \emph{ordering constraints} between those effects.
The graphs resulting from executing an ASL specification are converted into
\emph{candidate execution graphs}, which are introduced, defined,
and used in~\cite{AlglaveMT14,alglave2016syntax,AlglaveDGHM21}.

\hypertarget{def-xgraphs}{}
Formally, an execution graph $\xgraph = (N^\xgraph, E^\xgraph, O^\xgraph) \in \XGraphs$
is defined via a set of \emph{nodes} ($N^\xgraph$), a set of \emph{edges} ($E^\xgraph$), and a set of \emph{output nodes} ($O^\xgraph$):
\hypertarget{def-nodes}{}
\[
\begin{array}{rcl}
\XGraphs &\triangleq& \pow{\Nodes} \times \pow{\Nodes \times \Nodes \times \Labels} \times \pow{\Nodes}  \hypertarget{def-read}{}\hypertarget{def-write}{}\\
\Nodes   &\triangleq& \N \times \{ \Read, \Write \} \times \Identifiers \hypertarget{def-labels}{}\\
\Labels  &\triangleq& \{ \asldata, \aslctrl, \aslpo \}
\end{array}
\]

Nodes represent unique Read and Write Effects. Formally, a node $(u,l,\id) \in \Nodes$ associates a unique instance counter $u$
to an \emph{ordering label} $l$, which specifies whether it represents a Read Effect of a Write Effect to a variable named $\id$.
%
We say that en Effect \texttt{E1} is \emph{$l$-before} another Effect \texttt{E2}, for $l\in \Labels$ and a given execution graph
$g$, when $(\texttt{E1}, l, \texttt{E2}) \in E^g$.

An edge represents an ordering constraint between two effects, which can be one of the following:
\hypertarget{def-asldata}{}
\begin{description}
\item[$\asldata$] Represents a \emph{data dependency}.
That is, when one effect hands over its data to another effect.
\hypertarget{def-aslctrl}{}
\item[$\aslctrl$] Represents a \emph{control dependency}.
That is, when a Read Effect to a variable determines the flow of control (e.g., which condition of a branch is taken),
which then leads to another Read/Write Effect.
\hypertarget{def-aslpo}{}
\item[$\aslpo$] Represents a \emph{program order}.
That is, when two Effects are generated by ASL constructs, which are separated by a semicolon in the text of the specification,
or appear in successive iterations of loop a unrolling.
% That is, when two effects are defined to be ordered according to the sequential semantics.
\end{description}

An execution graph is \emph{well-formed} if all nodes have unique instance counters, edges connect graph nodes,
and the output nodes are contained in the set of nodes:
\[
  \begin{array}{rcl}
  \forall n, n' \in N^\xgraph &.& n=(u,l,\id) \wedge n=(u',l',\id') \;\Rightarrow\; u \neq u' \\
  \forall e \in E^\xgraph &.& e=(n, n', l) \;\Rightarrow\; n, n' \in N^\xgraph \\
  & & O^\xgraph \subseteq N^\xgraph \enspace.
  \end{array}
\]
\hypertarget{def-emptygraph}{}

We denote the empty execution graph $\emptygraph \triangleq (\emptyset, \emptyset, \emptyset)$.
%
We define the following functions, which return an execution graph that represents a single Read/Write Effect to a variable \texttt{x}.
\hypertarget{def-writeeffect}{}
\hypertarget{def-readeffect}{}
\begin{definition}[Read/Write Effects]
\[
  \begin{array}{rclll}
    \WriteEffect(\vx) &\triangleq& (\{n\}, \emptyset, \{n\}) & \text{where } n=(u, \Write, \vx), & u\in\N \text{ is fresh}\\
    \ReadEffect(\vx)  &\triangleq& (\{n\}, \emptyset, \{n\}) & \text{where } n=(u, \Read, \vx),   & u\in\N \text{ is fresh}\\
  \end{array}
\]
\end{definition}

We also define two ways to compose execution graphs --- \emph{unordered composition} and
\emph{ordered composition with a given label}.

\hypertarget{def-parallel}{}
\begin{definition}[Unordered Graph Composition]
Given two execution graphs $S_1 = (N_1, E_1, O_1)$ and $S_1 = (N_2, E_2, O_2)$ their unordered composition,
denoted $S_1 \parallelcomp S_2$ is defined as follows:
\[
  S_1 \parallelcomp S_2\triangleq (N_1 \cup N_2, E_1 \cup E_2, O_1 \cup O_2) \enspace.
\]
Intuitively, this composition conveys the fact that there are no ordering constraints between the effects
in the arguments graphs.
\end{definition}

\hypertarget{def-ordered}{}
\begin{definition}[Ordered Graph Composition]
Given two execution graphs, \\ $S_1 = (N_1, E_1, O_1)$ and $S_1 = (N_2, E_2, O_2)$ and an ordering label $l$,
the ordered composition $\ordered{S_1}{l}{S_2}$ is defined as follows:
\[
  \ordered{S_1}{l}{S_2} \triangleq (N_1 \cup N_2,\ E_1 \cup E_2 \cup (O_1 \times \{l\} \times N_2),\ O_2) \enspace.
\]
Intuitively, this composition constrains the output effects of $S_1$ to appear before any effect of $S_2$ with respect
to the given ordering label.
\end{definition}

\subsection{Kinds of Semantic Configurations\label{sec:KindsOfSemanticConfigurations}}

Recall that the ASL semantics defines a relation between input semantic configurations and output semantic configurations
(\secref{SemanticConfigurations}).
Input semantic configuration domains are unique to the semantic relation that employs them.
For that reason, we name semantic relations by the name of the corresponding configuration domain of the input semantic configuration.
For example, the semantic relation that employs input semantic configurations with the domain $\texttt{eval\_expr}$
is named $\texttt{eval\_expr}$.
%
We will often use the prefix \texttt{eval} for semantic relations with the intuition being that their input semantic configurations
should be semantically evaluated, yielding an output semantic configuration.

ASL semantics mainly utilizes the following types of output semantic configurations:
\begin{description}
  \item[Normal Values.] \hypertarget{def-normal}{}
  Semantic configurations consisting of different combinations of values,
  execution graphs, and environments, representing intermediate results
  generated while evaluating statements:
  \begin{itemize}
  \item $\Normal(\vals \times \XGraphs)$,
  \item $\Normal((\vals \times \XGraphs), \envs)$,
  \item $\Normal(((\vals \times \vals)^* \times \XGraphs), \envs)$,
  \item $\Normal(\XGraphs, \envs)$,
  \item $\Normal((\vals^* \times \XGraphs), \envs)$, and
  \item $\Normal((\vals\times\XGraphs)^*, \envs)$.
  \end{itemize}

  \hypertarget{def-throwing}{}
  \item[Exceptions.] Semantic configurations in
  \[
    \Throwing(\langle \valuereadfrom(\vals,\Identifiers) \times \ty\rangle \times \XGraphs, \envs)
  \]
  represent thrown exceptions.
  \hypertarget{def-valuereadfrom}{}

  There are two flavors of exceptions:
  exceptions without an exception value (as in \texttt{throw;}), and ones with an exception value,
  an identifier to which the Read Effect is attributed, and an associated type.
  The type $\valuereadfrom(\vals,\Identifiers)$ is a semantic configuration nested inside an exception semantic configuration.
  The ASL semantics propagates these \emph{exceptional semantic configurations} to the nearest catch clause that matches
  them, and otherwise they are caught at the top-level and reported as errors (see dynamic errors below).

  \hypertarget{def-returning}{}
  \item[Returned Values.] Semantic configurations in $\Returning((\vals^* \times \XGraphs), \envs)$
  represent (tuples of) values being returned by the currently executing subprogram.
  The ASL semantics propagates these \emph{early return semantic configurations} to the respective call expression/statement.

  \hypertarget{def-continuing}{}
  \item[In-flight Subprogram.] Semantic semantic configurations in $\Continuing(\XGraphs, \envs)$
  represent the fact that a subprogram has more statements to execute.
  The ASL semantics treats these semantic configurations as a signal to keep evaluating the remainder
  of the subprogram currently being evaluated.
%where is this used?

  \hypertarget{def-error}{}
  \item[Dynamic Errors.] Semantic configurations in $\Error(\Strings)$
  represent dynamic errors (for example, division by zero).
  The ASL semantics is set up such that when these \emph{error semantic configurations} appear,
  the evaluation of the entire specification terminates by outputting them.
\end{description}
Helper relations often have output semantic configurations that are just tuples, without an associated configuration domain.

We define the following shorthands for types of output semantic configurations:
\hypertarget{def-tnormal}{}
\hypertarget{def-tcontinuing}{}
\hypertarget{def-tthrowing}{}
\hypertarget{def-treturning}{}
\hypertarget{def-terror}{}
\[
  \begin{array}{rcl}
    \TNormal          & \triangleq & \Normal(\vals, \XGraphs) \cup \Normal((\vals \times \XGraphs), \envs)\ \cup \\
                      &            & \Normal(((\vals \times \vals)^* \times \XGraphs), \envs) \cup  \Normal(\XGraphs, \envs)\ \cup   \\
                      &            & \Normal((\vals^* \times \XGraphs), \envs) \cup \Normal((\vals\times\XGraphs)^*, \envs)\\
    \TThrowing        & \triangleq & \Throwing(\langle\vals \times \ty\rangle \times \XGraphs, \envs)\\
    \TContinuing      & \triangleq & \Continuing(\XGraphs, \envs)\\
    \TReturning       & \triangleq & \Returning((\vals^* \times \XGraphs), \envs)\\
    \TDynError           & \triangleq & \Error(\Strings)\\
  \end{array}
\]

We will say that a semantic transition \emph{terminates}:
\begin{itemize}
\item \emph{normally} when the output semantic configuration domain is
$\Normal$,
\item \emph{exceptionally} when the output semantic configuration domain is
$\Throwing$,
\item \emph{erroneously} when the output semantic configuration domain is
$\Error$, and
\item \emph{abnormally} when it either terminates exceptionally or
erroneously.
\end{itemize}

We introduce the following shorthands for semantic configurations where all variables
appearing are \hyperlink{def-freshvariables}{fresh}:
\begin{itemize}
\hypertarget{def-throwingconfig}{}
\item $\ThrowingConfig \triangleq \Throwing((\vv, \vg), \newenv)$.
\hypertarget{def-errorconfig}{}
\item $\DynErrorConfig \triangleq \Error(\vs)$.
\hypertarget{def-returningconfig}{}
\item $\ReturningConfig \triangleq \Returning((\vvs,\newg), \newenv)$
is an early return semantic configuration.
\hypertarget{def-continuingconfig}{}
\item $\ContinuingConfig \triangleq \Continuing(\newg, \newenv)$.
\end{itemize}

\subsection{Extracting and Substituting Elements of Semantic configurations}

\hypertarget{def-graphof}{}
\hypertarget{def-environof}{}
Given a semantic configuration $C$, we define the graph component of the semantic configuration, \\
$\graphof{C}$, and the environment of the semantic configuration, $\environof{C}$, as follows:
\[
\begin{array}{lcc}
  C & \graphof{C} & \environof{C}\\
  \hline
  \Normal(\vv,\vg) & \vg & \text{undefined}\\
  \Normal((\vv,\vg), \env) & \vg & \env\\
  \Normal(([i=1..k: (\texttt{va}_i,\texttt{vb})],\vg), \env) & \vg & \env\\
  \Normal(\vg, \env) & \vg & \env\\
  \Normal([\vv_{1..k}], \vg) & \vg & \env\\
  \Normal([i=1..k: (\vv_i,\vg_i)], \env) & \text{undefined} & \env\\
  \Throwing((\valuereadfrom(\vx,\vv),\vg), \env) & \vg & \env\\
  \Returning(([\vv_{1..k}],\vg), \env) & \vg & \env\\
  \Continuing(\vg, \env) & \vg & \env\\
\end{array}
\]

\hypertarget{def-withgraph}{}
Given a semantic configuration $C$, we define $\withgraph{C}{\vgp}$ to be a semantic configuration
like $C$ where the graph component is substituted with $\vgp$:
\[
\begin{array}{ll}
  C & \withgraph{C}{\vgp}\\
  \hline
  \Normal(\vv,\vg) & \Normal(\vv,\vgp)\\
  \Normal((\vv,\vg), \env) & \Normal((\vv,\vgp), \env)\\
  \Normal((i=1..k: (\texttt{va}_i,\texttt{vb}),\vg), \env) & \Normal((i=1..k: (\texttt{va}_i,\texttt{vb}),\vgp), \env)\\
  \Normal(\vg, \env) & \Normal(\vgp, \env)\\
  \Normal(i=1..k: \vv_i, \vg) & \Normal(i=1..k: \vv_i, \vgp)\\
  \Normal([i=1..k: (\vv_i,\vg_i)], \env) & \text{undefined}\\
  \Throwing((\valuereadfrom(\vx,\vv),\vg), \env) & \Throwing((\valuereadfrom(\vx,\vv),\vgp), \env)\\
  \Returning((i=1..k: \vv_i,\vg), \env) & \Returning((i=1..k: \vv_i,\vgp), \env)\\
  \Continuing(\vg, \env) & \Continuing(\vgp, \env)\\
\end{array}
\]
\hypertarget{def-withenviron}{}
Similarly, we define the $\withenviron{C}{\envp}$ to be a semantic configuration
like $C$ where the environment component, if one exists, is substituted with $\envp$:
\[
\begin{array}{ll}
  \textbf{Semantic configuration} & \withenviron{C}{\envp}\\
  \hline
  \Normal(\vv,\vg) & \text{undefined}\\
  \Normal((\vv,\vg), \env) & \Normal((\vv,\vg), \envp)\\
  \Normal((i=1..k: (\texttt{va}_i,\texttt{vb}),\vg), \env) & \Normal((i=1..k: (\texttt{va}_i,\texttt{vb}),\vg), \envp)\\
  \Normal(\vg, \env) & \Normal(\vg, \envp)\\
  \Normal(i=1..k: \vv_i, \vg) & \Normal(i=1..k: \vv_i, \vg)\\
  \Normal([i=1..k: (\vv_i,\vg_i)], \env) & \Normal([i=1..k: (\vv_i,\vg_i)], \envp)\\
  \Throwing((\valuereadfrom(\vx,\vv),\vg), \env) & \Throwing((\valuereadfrom(\vx,\vv),\vg), \envp)\\
  \Returning((i=1..k: \vv_i,\vg), \env) & \Returning((i=1..k: \vv_i,\vg), \envp)\\
  \Continuing(\vg, \env) & \Continuing(\vg, \envp)\\
\end{array}
\]

\section{Semantic Evaluation}
\hypertarget{def-evalrel}{}
\hypertarget{def-primitiverel}{}
The semantics of ASL is given by the relations\footnote{The reason that relations, rather than functions, are used is due to the
potential non-determinism in the primitive subprograms
and the non-determinism inherent in the \texttt{UNKNOWN} expression.}
\evalrel\ and \primitiverel.
The relation \evalrel\ is defined as the disjoint union of the relations defined in this reference.
The relation \primitiverel\ provides the semantics of primitive subprograms and is not otherwise defined
constructively.

\subsection{Natural Operational Semantics}
We define the ASL semantics in the style of \emph{natural operational semantics}~\cite{SemanticsWithApplicationsBook}
(also known as \emph{big step semantics}).
Natural operational semantics evaluates the AST inductively.
That is, it concludes transitions for semantic configurations starting from
non-leaf AST nodes by concluding transitions from semantic configurations starting from their children nodes.

\subsubsection{No Undefined Behaviors}
When an input semantic configuration does not satisfy any semantic rule,
there is no output semantic configuration for it to transition to. We say that
the semantic configuration is \emph{stuck} and the ASL semantics is undefined for that
input semantic configuration.

The ASL semantics is defined for well-typed ASL specifications
and gets stuck only in cases of non-terminating specifications
(due to non-terminating loops, or infinite recursion).
Otherwise, for every input semantic configuration there is at least one rule that can be used to take a semantic transition.

\hypertarget{eval-example}{}
\subsubsection{Evaluation Example}
\newcommand\elint[1]{\ELiteral(\lint(#1))}
The following example shows how to utilize the rules for expression literals and binary operator expressions
to derive a transition from an input semantic configuration with the expression $(1+2)*(4+5)$,
given by the AST
\[
  \begin{array}{rl}
  \EBinop(\MUL, &\overname{\EBinop(\PLUS, \overname{\elint{1}}{\veone}, \overname{\elint{2}}{\vetwo})}{\veonetwo}, \\
                &\overname{\EBinop(\PLUS, \overname{\elint{4}}{\vefour}, \overname{\elint{5}}{\vefive})}{\vefourfive})
  \end{array}
\]
to an output semantic configuration with the value resulting from the calculation of the expression.

We annotate subexpressions to allow referring to them.

\hypertarget{def-emptyenv}{}
We write $\emptyenv$ to stand for a trivial environment (that is, one where all functions are empty).

Notice that, we have dropped the execution graph component and simplified pairs of the form $(\vv,\vg)$,
where $\vv$ is a \nativevalue\  and $\vg$ is an execution graph, to just $\vv$.
This is because we are interested in demonstrating the sequential semantics (also, the execution graphs
in this case are all empty).

The example shows (using references to the relevant rules on the right), how the expression for $1+2$ is evaluated
using the rule for literal expressions, the rule for binary operator (for addition), and the rules for binary expressions.
Similarly, the expression for $4+5$ is evaluated.
Finally, the transitions for both of the subexpressions are used as premises for the binary expression
rule, along with the rule for binary operator (for multiplication), to evaluate the entire expression.

\begin{mathpar}
  \inferrule*[right=\ref{sec:SemanticsRule.Binop}]{
    \inferrule*[right=\ref{sec:SemanticsRule.Lit}]{}{ \evalexpr{\emptyenv, \veone} \evalarrow \Normal(\nvint(1), \emptyenv) } \\\\
    \inferrule*[right=\ref{sec:SemanticsRule.Lit}]{}{ \evalexpr{\emptyenv, \vetwo} \evalarrow \Normal(\nvint(2), \emptyenv) } \\\\
    \inferrule*[right=\ref{sec:SemanticsRule.BinopValues}]{}{\binoprel(\PLUS, \nvint(1), \nvint(2)) \evalarrow \nvint(3)}
  }
  {
    \evalexpr{ \emptyenv, \veonetwo } \evalarrow
    \Normal(\nvint(3), \emptyenv)
  }
\end{mathpar}

\begin{mathpar}
  \inferrule*[right=\ref{sec:SemanticsRule.Binop}]{
    \inferrule*[right=\ref{sec:SemanticsRule.Lit}]{}{ \evalexpr{\emptyenv, \vefour} \evalarrow \Normal(\nvint(4), \emptyenv) } \\\\
    \inferrule*[right=\ref{sec:SemanticsRule.Lit}]{}{ \evalexpr{\emptyenv, \vefive} \evalarrow \Normal(\nvint(5), \emptyenv) } \\\\
    \inferrule*[right=\ref{sec:SemanticsRule.BinopValues}]{}{\binoprel(\PLUS, \nvint(4), \nvint(5)) \evalarrow \nvint(9)}
  }
  {
    \evalexpr{ \emptyenv, \vefourfive } \evalarrow
    \Normal(\nvint(9), \emptyenv)
  }
\end{mathpar}

\begin{mathpar}
  \inferrule*[right=\ref{sec:SemanticsRule.Binop}]{
    \evalexpr{ \emptyenv, \veonetwo } \evalarrow \Normal(\nvint(3), \emptyenv)\\
    \evalexpr{ \emptyenv, \vefourfive } \evalarrow \Normal(\nvint(9), \emptyenv)\\
    \inferrule*[right=\ref{sec:SemanticsRule.BinopValues}]{}{\binoprel(\MUL, \nvint(3), \nvint(9)) \evalarrow \nvint(27)}
  }
  {
    \evalexpr{ \emptyenv, \EBinop(\MUL, \veonetwo, \vefourfive) } \evalarrow
    \Normal(\nvint(27), \emptyenv)
  }
\end{mathpar}
