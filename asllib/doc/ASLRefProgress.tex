\chapter{Not Implemented by ASLRef\label{appendix:UnimplementedInASLRef}}
% ------------------------------------------------------------------------------

This chapter describes what is not yet present in the executable version of ASLRef
(Build from Dec 12, 2024).

% ------------------------------------------------------------------------------
\section{Syntax}

\subsection{Declaring Multiple Identifiers Without Initialization}
The following simultaneous declaration of three global variables does not currently parse with ASLRef.
\begin{verbatim}
 var x, y, z : integer;
\end{verbatim}

The same line does parse and correctly handled inside a subprogram.

\lrmcomment{This is related to \identr{QDQD}.}

\subsection{Guards}

Guards are used on \texttt{case} and \texttt{catch} statements, to restrict
matching on the evaluation of a boolean expression.
%
They are not yet implemented in ASLRef.

\lrmcomment{This relates to \identr{WGSY}.}

% ------------------------------------------------------------------------------
\section{Semantics}

\subsection{Non-\texttt{main} Entry Point}
Currently ASLRef only supports \texttt{main} as an entry point.

% ------------------------------------------------------------------------------
\section{Typing}

\subsection{Throwing Exceptions without Braces}
In the following example, the commented out \texttt{throw} statement should typecheck,
but it currently fails.

\begin{verbatim}
  type except of exception;

  func main() => integer
  begin
    // throw except; // Should typecheck
    throw except{}; // Okay

    return 0;
  end
\end{verbatim}

\subsection{Side-effect-free Subprograms with respect to dynamic errors}
ASLRef performs a side effect analysis (see \chapref{SideEffects}).
The analysis currently ignores dynamic errors that are not due to assertions.

\subsection{Restriction on Use of Parameterized Integer Types}

\subsubsection{As storage types}
Restrictions on the use of parameterized integer types as storage element types are not
implemented.

\lrmcomment{This is related to \identr{ZCVD}.}

\subsubsection{\texttt{as} Expression With a Constrained Type}

Restriction on the use of parameterized integer types as left-hand-side of a
Asserted Typed Conversion is not implemented in ASLRef.
%
For example, the following will not raise a type-error:
\VerbatimInput[firstline=1,lastline=4]{../tests/regressions.t/under-constrained-used.asl}

\lrmcomment{This is related to \identi{TBHH}, \identr{ZDKC}.}

% ------------------------------------------------------------------------------
\chapter{Issues Not Yet Addressed by the Reference\label{appendix:MissingTransliteration}}
% ------------------------------------------------------------------------------
\section{Semantics}

\subsection{Standard Library and Primitives}

The standard library is not yet defined in this reference.

\lrmcomment{This is related to \identr{RXYN}.}

% ------------------------------------------------------------------------------
\section{Typing}

\subsection{Checking Type Annotations for Absence of Side Effects}
Type annotations that contain expressions that may fail dynamically are not checked for.

% \subsection{Type inference from literals}
% Finding the type of a literal value, or a compile-time constant expression, is
% not yet transliterated.

% \lrmcomment{This is related to \identr{ZJKY} and \identi{RYRP}.}
