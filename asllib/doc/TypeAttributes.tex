%%%%%%%%%%%%%%%%%%%%%%%%%%%%%%%%%%%%%%%%%%%%%%%%%%%%%%%%%%%%%%%%%%%%%%%%%%%%%%%%%%%%
\section{Basic Type Attributes\label{sec:BasicTypeAttributes}}
%%%%%%%%%%%%%%%%%%%%%%%%%%%%%%%%%%%%%%%%%%%%%%%%%%%%%%%%%%%%%%%%%%%%%%%%%%%%%%%%%%%%

This section defines some basic predicates for classifying types as well as
functions that inspect the structure of types:
\begin{itemize}
  \item Builtin singular types (\secref{TypingRule.BuiltinSingularType})
  \item Builtin aggregate types (\secref{TypingRule.BuiltinAggregateType})
  \item Builtin types (\secref{TypingRule.BuiltinSingularOrAggregate})
  \item Named types (\secref{TypingRule.NamedType})
  \item Anonymous types (\secref{TypingRule.AnonymousType})
  \item Singular types (\secref{TypingRule.SingularType})
  \item Aggregate types (\secref{TypingRule.AggregateType})
  \item Structured types (\secref{TypingRule.StructuredType})
  \item Non-primitive types (\secref{TypingRule.NonPrimitiveType})
  \item Primitive types (\secref{TypingRule.PrimitiveType})
  \item The structure of a type (\secref{TypingRule.structure})
  \item The underlying type of a type (\secref{TypingRule.MakeAnonymous})
  \item Checked constrained integers (\secref{TypingRule.CheckConstrainedInteger})
\end{itemize}

Finally, constrained types are defined in \secref{ConstrainedTypes}.

\subsubsection{TypingRule.BuiltinSingularType \label{sec:TypingRule.BuiltinSingularType}}
\hypertarget{def-isbuiltinsingular}{}
The predicate
\[
  \isbuiltinsingular(\overname{\ty}{\tty}) \;\aslto\; \Bool
\]
tests whether the type $\tty$ is a \emph{builtin singular type}.

\subsubsection{Prose}
The \emph{builtin singular types} are:
\begin{itemize}
\item \texttt{integer};
\item \texttt{real};
\item \texttt{string};
\item \texttt{boolean};
\item \texttt{bits} (which also represents \texttt{bit}, as a special case);
\item \texttt{enumeration}.
\end{itemize}

\subsubsection{Example}
In this example:
\VerbatimInput[firstline=3,lastline=8]{\typingtests/TypingRule.BuiltinSingularTypes.asl}

Variables of builtin singular types \texttt{integer}, \texttt{real},
\texttt{boolean}, \texttt{bits(4)}, \\ and~\texttt{bits(2)} are defined.

\subsubsection{Example}
\VerbatimInput{\typingtests/TypingRule.EnumerationType.asl}
The builtin singular type \texttt{Color} consists in two constants
\texttt{RED}, and~\texttt{BLACK}.

\CodeSubsection{\BuiltinSingularBegin}{\BuiltinSingularEnd}{../types.ml}

\subsubsection{Formally}
\begin{mathpar}
\inferrule{
  \vb \eqdef \astlabel(\tty) \in \{\TReal, \TString, \TBool, \TBits, \TEnum, \TInt\}
}{
  \isbuiltinsingular(\tty) \typearrow \vb
}
\end{mathpar}

\isempty{\subsubsection{Comments}}
\lrmcomment{This is related to \identd{PQCK} and \identd{NZWT}.}

\subsubsection{TypingRule.BuiltinAggregateType \label{sec:TypingRule.BuiltinAggregateType}}
\hypertarget{def-isbuiltinaggregate}{}
The predicate
\[
  \isbuiltinaggregate(\overname{\ty}{\tty}) \;\aslto\; \Bool
\]
tests whether the type $\tty$ is a \emph{builtin aggregate type}.

\subsubsection{Prose}
The builtin aggregate types are:
\begin{itemize}
\item tuple;
\item \texttt{array};
\item \texttt{record};
\item \texttt{exception}.
\end{itemize}

\subsubsection{Example}
\VerbatimInput{\typingtests/TypingRule.BuiltinAggregateTypes.asl}
Type \texttt{Pair} is the type of integer and boolean pairs.

Arrays are declared with indices that are either integer-typed
or enumeration-typed.  In the example above, \texttt{T} is
declared as an array with an integer-typed index (as indicated
by the used of the integer-typed constant \texttt{3}) whereas
\texttt{PointArray} is declared with the index of
\texttt{Coord}, which is an enumeration type.

Arrays declared with integer-typed indices can be accessed only by integers ranging from $0$ to
the size of the array minus $1$. In the example above, $\texttt{T}$ can be accessed with
one of $0$, $1$, and $2$.

Arrays declared with an enumeration-typed index can only be accessed with labels from the corresponding
enumeration. In the example above, \texttt{PointArray} can only be accessed with one of the labels
\texttt{CX}, \texttt{CY}, and \texttt{CZ}.

The (builtin aggregate) type \verb|{ x : real, y : real, z : real }| is a record type with three fields
\texttt{x}, \texttt{y} and \texttt{z}.

\subsubsection{Example}
\VerbatimInput{\typingtests/TypingRule.BuiltinExceptionType.asl}
Two (builtin aggregate) exception types are defined:
\begin{itemize}
\item \verb|exception{}| (for \texttt{Not\_found}), which carries no value; and
\item \verb|exception { message:string }| (for \texttt{SyntaxException}), which carries a message.
\end{itemize}
Notice the similarity with record types and that the empty field list \verb|{}| can be
omitted in type declarations, as is the case for \texttt{Not\_found}.

\CodeSubsection{\BuiltinAggregateBegin}{\BuiltinAggregateEnd}{../types.ml}

\subsubsection{Formally}
\begin{mathpar}
\inferrule{ \vb \eqdef \astlabel(\tty) \in \{\TTuple, \TArray, \TRecord, \TException\} }
{ \isbuiltinaggregate(\tty) \typearrow \vb }
\end{mathpar}

\isempty{\subsubsection{Comments}}
\lrmcomment{This is related to \identd{PQCK} and \identd{KNBD}.}

\subsubsection{TypingRule.BuiltinSingularOrAggregate \label{sec:TypingRule.BuiltinSingularOrAggregate}}
\hypertarget{def-isbuiltin}{}
The predicate
\[
  \isbuiltin(\overname{\ty}{\tty}) \;\aslto\; \Bool
\]
tests whether the type $\tty$ is a \emph{builtin type}.

\subsubsection{Prose}
$\tty$ is a builtin type and one of the following applies:
\begin{itemize}
\item $\tty$ is singular;
\item $\tty$ is builtin aggregate.
\end{itemize}

\subsubsection{Example}
In the specification
\begin{verbatim}
  type ticks of integer;
\end{verbatim}
the type \texttt{integer} is a builtin type but the type of \texttt{ticks} is not.

\CodeSubsection{\BuiltinSingularOrAggregateBegin}{\BuiltinSingularOrAggregateEnd}{../types.ml}

\subsubsection{Formally}
\begin{mathpar}
  \inferrule{
    \isbuiltinsingular(\tty) \typearrow \vbone\\
    \isbuiltinaggregate(\tty) \typearrow \vbtwo
  }{
    \isbuiltin(\tty) \typearrow \vbone \lor \vbtwo
  }
\end{mathpar}

\subsubsection{TypingRule.NamedType \label{sec:TypingRule.NamedType} }
\hypertarget{def-isnamed}{}
The predicate
\[
  \isnamed(\overname{\ty}{\tty}) \;\aslto\; \Bool
\]
tests whether the type $\tty$ is a \emph{named type}.

Enumeration types, record types, and exception types must be declared
and associated with a named type.

\subsubsection{Prose}
A named type is a type that is declared by using the \texttt{type of} syntax.

\subsubsection{Example}
In the specification
\begin{verbatim}
  type ticks of integer;
\end{verbatim}
\texttt{ticks} is a named type.

\CodeSubsection{\NamedBegin}{\NamedEnd}{../types.ml}

\subsubsection{Formally}
\begin{mathpar}
\inferrule{
  \vb \eqdef \astlabel(\tty) = \TNamed
}{
  \isnamed(\tty) \typearrow \vb
}
\end{mathpar}

\isempty{\subsubsection{Comments}}
\lrmcomment{This is related to \identd{vmzx}.}

\subsubsection{TypingRule.AnonymousType \label{sec:TypingRule.AnonymousType}}
\hypertarget{def-isanonymous}{}
The predicate
\[
  \isanonymous(\overname{\ty}{\tty}) \;\aslto\; \Bool
\]
tests whether the type $\tty$ is an \anonymoustype.

\subsubsection{Prose}
\Anonymoustypes\ are types that are not declared using the type syntax:
integer types, the real type, the string type, the Boolean type,
bitvector types, tuple types, and array types.

\subsubsection{Example}
The tuple type \texttt{(integer, integer)} is an \anonymoustype.

\CodeSubsection{\AnonymousBegin}{\AnonymousEnd}{../types.ml}

\subsubsection{Formally}
\begin{mathpar}
\inferrule{ \vb \eqdef \astlabel(\tty) \neq \TNamed
}
{
  \isanonymous(\tty) \typearrow \vb
}
\end{mathpar}

\isempty{\subsubsection{Comments}}
\lrmcomment{This is related to \identd{VMZX}.}

\subsubsection{TypingRule.SingularType \label{sec:TypingRule.SingularType}}
\hypertarget{def-issingular}{}
The predicate
\[
  \issingular(\overname{\staticenvs}{\tenv} \aslsep \overname{\ty}{\tty}) \;\aslto\;
  \overname{\Bool}{\vb} \cup \overname{\TTypeError}{\TypeErrorConfig}
\]
tests whether the type $\tty$ is a \emph{singular type} in the static environment $\tenv$.

\subsubsection{Prose}
A type $\tty$ is singular if and only if all of the following apply:
\begin{itemize}
  \item obtaining the \underlyingtype\ of $\tty$ in the environment $\tenv$ yields $\vtone$\ProseOrTypeError;
  \item $\vtone$ is a builtin singular type.
\end{itemize}

\subsubsection{Example}
In the following example, the types \texttt{A}, \texttt{B}, and \texttt{C} are all singular types:
\begin{verbatim}
type A of integer;
type B of A;
type C of B;
\end{verbatim}

\CodeSubsection{\SingularBegin}{\SingularEnd}{../types.ml}

\subsubsection{Formally}
\begin{mathpar}
\inferrule{
  \makeanonymous(\tenv, \tty) \typearrow \vtone \OrTypeError\\\\
  \isbuiltinsingular(\vtone) \typearrow \vb
}{
\issingular(\tenv, \tty) \typearrow \vb
}
\end{mathpar}

\isempty{\subsubsection{Comments}}
\lrmcomment{This is related to \identr{GVZK}.}

\subsubsection{TypingRule.AggregateType \label{sec:TypingRule.AggregateType}}
\hypertarget{def-isaggregate}{}
The predicate
\[
  \isaggregate(\overname{\staticenvs}{\tenv} \aslsep \overname{\ty}{\tty}) \;\aslto\;
  \overname{\Bool}{\vb} \cup \overname{\TTypeError}{\TypeErrorConfig}
\]
tests whether the type $\tty$ is an \emph{aggregate type} in the static environment $\tenv$.

\subsubsection{Prose}
A type $\tty$ is aggregate in an environment $\tenv$ if and only if all of the following apply:
\begin{itemize}
  \item obtaining the \underlyingtype\ of $\tty$ in the environment $\tenv$ yields $\vtone$\ProseOrTypeError;
  \item $\vtone$ is a builtin aggregate.
\end{itemize}

\subsubsection{Example}
In the following example, the types \texttt{A}, \texttt{B}, and \texttt{C} are all aggregate types:
\begin{verbatim}
type A of (integer, integer);
type B of A;
type C of B;
\end{verbatim}

\CodeSubsection{\AggregateBegin}{\AggregateEnd}{../types.ml}

\subsubsection{Formally}
\begin{mathpar}
\inferrule{
  \makeanonymous(\tenv, \tty) \typearrow \vtone \OrTypeError\\\\
  \isbuiltinaggregate(\vtone) \typearrow \vb
}{
  \isaggregate(\tenv, \tty) \typearrow \vb
}
\end{mathpar}

\isempty{\subsubsection{Comments}}
\lrmcomment{This is related to \identr{GVZK}.}

\subsubsection{TypingRule.StructuredType\label{sec:TypingRule.StructuredType}}
\hypertarget{def-isstructured}{}
\hypertarget{def-structuredtype}{}
A \emph{\structuredtype} is any type that consists of a list of field identifiers
that denote individual storage elements. In ASL there are two such types --- record types and exception types.

The predicate
\[
  \isstructured(\overname{\ty}{\tty}) \;\aslto\; \overname{\Bool}{\vb}
\]
tests whether the type $\tty$ is a \structuredtype\ and yields the result in $\vb$.

\subsubsection{Prose}
The result $\vb$ is $\True$ if and only if $\tty$ is either a record type or an exception type,
which is determined via the AST label of $\tty$.

\subsubsection{Example}
In the following example, the types \texttt{SyntaxException} and \texttt{PointRecord}
are each an example of a \structuredtype:
\begin{verbatim}
type SyntaxException of exception {message: string };
type PointRecord of Record {x : real, y: real, z: real};
\end{verbatim}

\subsubsection{Formally}
\begin{mathpar}
\inferrule{}{
  \isstructured(\tty) \typearrow \overname{\astlabel(\tty) \in \{\TRecord, \TException\}}{\vb}
}
\end{mathpar}

\isempty{\subsubsection{Comments}}
\lrmcomment{This is related to \identd{WGQS}, \identd{QXYC}.}

\subsubsection{TypingRule.NonPrimitiveType \label{sec:TypingRule.NonPrimitiveType}}
\hypertarget{def-isnonprimitive}{}
The predicate
\[
  \isnonprimitive(\overname{\ty}{\tty}) \;\aslto\; \overname{\Bool}{\vb}
\]
tests whether the type $\tty$ is a \emph{non-primitive type}.

\subsubsection{Prose}
One of the following applies:
\begin{itemize}
  \item All of the following apply (\textsc{singular}):
  \begin{itemize}
  \item $\tty$ is a builtin singular type;
  \item $\vb$ is $\False$.
  \end{itemize}
  \item All of the following apply (\textsc{named}):
  \begin{itemize}
    \item $\tty$ is a named type;
    \item $\vb$ is $\True$.
  \end{itemize}
  \item All of the following apply (\textsc{tuple}):
  \begin{itemize}
    \item $\tty$ is a tuple type $\vli$;
    \item $\vb$ is $\True$ if and only if there exists a non-primitive type in $\vli$.
  \end{itemize}
  \item All of the following apply (\textsc{array}):
    \begin{itemize}
    \item $\tty$ is an array of type $\tty'$
    \item $\vb$ is $\True$ if and only if $\tty'$ is non-primitive.
    \end{itemize}
  \item All of the following apply (\textsc{structured}):
    \begin{itemize}
    \item $\tty$ is a \structuredtype\ with fields $\fields$;
    \item $\vb$ is $\True$ if and only if there exists a non-primitive type in $\fields$.
    \end{itemize}
\end{itemize}

\subsubsection{Example}
The following types are non-primitive:

\begin{tabular}{ll}
\textbf{Type definition} & \textbf{Reason for being non-primitive}\\
\hline
\texttt{type A of integer}  & Named types are non-primitive\\
\texttt{(integer, A)}       & The second component, \texttt{A}, has non-primitive type\\
\texttt{array[6] of A}      & Element type \texttt{A} has a non-primitive type\\
\verb|record { a : A }|     & The field \texttt{a} has a non-primitive type
\end{tabular}

\CodeSubsection{\NonPrimitiveBegin}{\NonPrimitiveEnd}{../types.ml}

\subsubsection{Formally}
The cases \textsc{tuple} and \textsc{structured} below, use the notation $\vb_\vt$ to name
Boolean variables by using the types denoted by $\vt$ as a subscript.
\begin{mathpar}
\inferrule[singular]{
  \astlabel(\tty) \in \{\TReal, \TString, \TBool, \TBits, \TEnum, \TInt\}
}{
  \isnonprimitive(\tty) \typearrow \False
}
\end{mathpar}

\begin{mathpar}
\inferrule[named]{\astlabel(\tty) = \TNamed}{\isnonprimitive(\tty) \typearrow \True}
\end{mathpar}

\begin{mathpar}
\inferrule[tuple]{
  \vt \in \tys: \isnonprimitive(\vt) \typearrow \vb_{\vt}\\
  \vb \eqdef \bigvee_{\vt \in \tys} \vb_{\vt}
}{
  \isnonprimitive(\overname{\TTuple(\tys)}{\tty}) \typearrow \vb
}
\end{mathpar}

\begin{mathpar}
\inferrule[array]{
  \isnonprimitive(\tty') \typearrow \vb
}{
  \isnonprimitive(\overname{\TArray(\Ignore, \tty')}{\tty}) \typearrow \vb
}
\end{mathpar}

\begin{mathpar}
\inferrule[structured]{
  L \in \{\TRecord, \TException\}\\
  (\Ignore,\vt) \in \fields : \isnonprimitive(\vt) \typearrow \vb_\vt\\
  \vb \eqdef \bigvee_{\vt \in \vli} \vb_{\vt}
}{
  \isnonprimitive(\overname{L(\fields)}{\tty}) \typearrow \vb
}
\end{mathpar}

\isempty{\subsubsection{Comments}}
\lrmcomment{This is related to \identd{GWXK}.}

\subsubsection{TypingRule.PrimitiveType \label{sec:TypingRule.PrimitiveType}}
\hypertarget{def-isprimitive}{}
The predicate
\[
  \isprimitive(\overname{\ty}{\tty}) \;\aslto\; \Bool
\]
tests whether the type $\tty$ is a \emph{primitive type}.

\subsubsection{Prose}
A type $\tty$ is primitive if it is not non-primitive.

\subsubsection{Example}
The following types are primitive:

\begin{tabular}{ll}
\textbf{Type definition} & \textbf{Reason for being primitive}\\
\hline
\texttt{integer} & Integers are primitive\\
\texttt{(integer, integer)} & All tuple elements are primitive\\
\texttt{array[5] of integer} & The array element type is primitive\\
\verb|record {ticks : integer}| & The single field \texttt{ticks} has a primitive type
\end{tabular}

\CodeSubsection{\PrimitiveBegin}{\PrimitiveEnd}{../types.ml}

\subsubsection{Formally}
\begin{mathpar}
\inferrule{
  \isnonprimitive(\tty) \typearrow \vb
}{
  \isprimitive(\tty) \typearrow \neg\vb
}
\end{mathpar}

\isempty{\subsubsection{Comments}}
\lrmcomment{This is related to \identd{GWXK}.}

\subsubsection{TypingRule.Structure\label{sec:TypingRule.structure}}
\hypertarget{def-structure}{}
The function
\[
  \tstruct(\overname{\staticenvs}{\tenv} \aslsep \overname{\ty}{\tty}) \aslto \overname{\ty}{\vt} \cup \overname{\TTypeError}{\TypeErrorConfig}
\]
assigns a type to its \hypertarget{def-tstruct}{\emph{\structure}}, which is the type formed by
recursively replacing named types by their type definition in the static environment $\tenv$.
If a named type is not associated with a declared type in $\tenv$, a type error is returned.

TypingRule.Specification ensures the absence of circular type definitions,
which ensures that TypingRule.Structure terminates\footnote{In mathematical terms,
this ensures that TypingRule.Structure is a proper \emph{structural induction.}}.

\subsubsection{Prose}
One of the following applies:
\begin{itemize}
\item All of the following apply (\textsc{named}):
  \begin{itemize}
  \item $\tty$ is a named type $\vx$;
  \item obtaining the declared type associated with $\vx$ in the static environment $\tenv$ yields $\vtone$\ProseOrTypeError;
  \item obtaining the structure of $\vtone$ static environment $\tenv$ yields $\vt$\ProseOrTypeError;
  \end{itemize}
\item All of the following apply (\textsc{builtin\_singular}):
  \begin{itemize}
  \item $\tty$ is a builtin singular type;
  \item $\vt$ is $\tty$.
  \end{itemize}
\item All of the following apply (\textsc{tuple}):
  \begin{itemize}
  \item $\tty$ is a tuple type with list of types $\tys$;
  \item the types in $\tys$ are indexed as $\vt_i$, for $i=1..k$;
  \item obtaining the structure of each type $\vt_i$, for $i=1..k$, in $\tys$ in the static environment $\tenv$,
  yields $\vtp_i$\ProseOrTypeError;
  \item $\vt$ is a tuple type with the list of types $\vtp_i$, for $i=1..k$.
  \end{itemize}
\item All of the following apply (\textsc{array}):
  \begin{itemize}
    \item $\tty$ is an array type of length $\ve$ with element type $\vt$;
    \item obtaining the structure of $\vt$ yields $\vtone$\ProseOrTypeError;
    \item $\vt$ is is an array type with of length $\ve$ with element type $\vtone$.
  \end{itemize}
\item All of the following apply (\textsc{structured}):
  \begin{itemize}
  \item $\tty$ is a \structuredtype\ with fields $\fields$;
  \item obtaining the structure for each type $\vt$ associated with field $\id$ yields a type $\vt_\id$\ProseOrTypeError;
  \item $\vt$ is a record or an exception, in correspondence to $\tty$, with the list of pairs $(\id, \vt\_\id)$;
  \end{itemize}
\end{itemize}

\subsubsection{Example}
In this example:
\texttt{type T1 of integer;} is the named type \texttt{T1}
whose structure is \texttt{integer}.

In this example:
\texttt{type T2 of (integer, T1);}
is the named type \texttt{T2} whose structure is (integer, integer). In this
example, \texttt{(integer, T1)} is non-primitive since it uses \texttt{T1}, which is builtin aggregate.

In this example:
\texttt{var x: T1;}
the type of $\vx$ is the named (hence non-primitive) type \texttt{T1}, whose structure
is \texttt{integer}.

In this example:
\texttt{var y: integer;}
the type of \texttt{y} is the anonymous primitive type \texttt{integer}.

In this example:
\texttt{var z: (integer, T1);}
the type of \texttt{z} is the anonymous non-primitive type
\texttt{(integer, T1)} whose structure is \texttt{(integer, integer)}.

\CodeSubsection{\StructureBegin}{\StructureEnd}{../types.ml}

\subsubsection{Formally}
\begin{mathpar}
\inferrule[named]{
  \declaredtype(\tenv, \vx) \typearrow \vtone \OrTypeError\\\\
  \tstruct(\tenv, \vtone)\typearrow\vt \OrTypeError
}{
  \tstruct(\tenv, \TNamed(\vx)) \typearrow \vt
}
\and
\inferrule[builtin\_singular]{
  \isbuiltinsingular(\tty) \typearrow \True
}{
  \tstruct(\tenv, \tty) \typearrow \tty
}
\and
\inferrule[tuple]{
  \tys \eqname \vt_{1..k}\\
  i=1..k: \tstruct(\tenv, \vt_i) \typearrow \vtp_i \OrTypeError
}{
  \tstruct(\tenv, \TTuple(\tys)) \typearrow  \TTuple(i=1..k: \vtp_i)
}
\and
\inferrule[array]{
  \tstruct(\tenv, \vt) \typearrow \vtone \OrTypeError
}{
  \tstruct(\tenv, \TArray(\ve, \vt)) \typearrow \TArray(\ve, \vtone)
}
\and
\inferrule[structured]{
  L \in \{\TRecord, \TException\}\\\\
  (\id,\vt) \in \fields : \tstruct(\tenv, \vt) \typearrow \vt_\id \OrTypeError
}{
  \tstruct(\tenv, L(\fields)) \typearrow
 L([ (\id,\vt) \in \fields : (\id,\vt_\id) ])
}
\end{mathpar}

\isempty{\subsubsection{Comments}}
\lrmcomment{This is related to \identd{FXQV}.}

\subsubsection{TypingRule.MakeAnonymous\label{sec:TypingRule.MakeAnonymous}}
\hypertarget{def-makeanonymous}{}
\hypertarget{def-underlyingtype}{}
The function
\[
  \makeanonymous(\overname{\staticenvs}{\tenv} \aslsep \overname{\ty}{\tty}) \aslto \overname{\ty}{\vt} \cup \overname{\TTypeError}{\TypeErrorConfig}
\]
returns the \emph{\underlyingtype} --- $\vt$ --- of the type $\tty$ in the static environment $\tenv$ or a type error.
Intuitively, $\tty$ is the first non-named type that is used to define $\tty$. Unlike $\tstruct$,
$\makeanonymous$ replaces named types by their definition until the first non-named type is found but
does not recurse further.

\subsubsection{Example}
Consider the following example:
\begin{verbatim}
type T1 of integer;
type T2 of T1;
type T3 of (integer, T2);
\end{verbatim}

The underlying types of \texttt{integer}, \texttt{T1}, and \texttt{T2} is \texttt{integer}.

The underlying type of \texttt{(integer, T2)} and \texttt{T3} is
\texttt{(integer, T2)}.  Notice how the underlying type does not replace
\texttt{T2} with its own underlying type, in contrast to the structure of
\texttt{T2}, which is \texttt{(integer, integer)}.

\subsubsection{Prose}
One of the following applies:
\begin{itemize}
  \item All of the following apply (\textsc{named}):
  \begin{itemize}
    \item $\tty$ is a named type $\vx$;
    \item obtaining the type declared for $\vx$ yields $\vtone$\ProseOrTypeError;
    \item the \underlyingtype\ of $\vtone$ is $\vt$.
  \end{itemize}

  \item All of the following apply (\textsc{non-named}):
  \begin{itemize}
    \item $\tty$ is not a named type $\vx$;
    \item $\vt$ is $\tty$.
  \end{itemize}
\end{itemize}

\subsubsection{Formally}
\begin{mathpar}
\inferrule[named]{
  \tty \eqname \TNamed(\vx) \\
  \declaredtype(\tenv, \vx) \typearrow \vtone \OrTypeError \\\\
  \makeanonymous(\tenv, \vtone) \typearrow \vt
}{
  \makeanonymous(\tenv, \tty) \typearrow \vt
}
\and
\inferrule[non-named]{
  \astlabel(\tty) \neq \TNamed
}{
  \makeanonymous(\tenv, \tty) \typearrow \tty
}
\end{mathpar}
\CodeSubsection{\MakeAnonymousBegin}{\MakeAnonymousEnd}{../types.ml}

\subsubsection{TypingRule.CheckConstrainedInteger \label{sec:TypingRule.CheckConstrainedInteger}}
\hypertarget{def-checkconstrainedinteger}{}
The function
\[
  \checkconstrainedinteger(\overname{\staticenvs}{\tenv} \aslsep \overname{\ty}{\tty}) \aslto \{\True\} \cup \overname{\TTypeError}{\TypeErrorConfig}
\]
checks whether the type $\vt$ is a \constrainedinteger. If so, the result is $\True$, otherwise a type error is returned.

\subsubsection{Prose}
One of the following applies:
\begin{itemize}
  \item All of the following apply (\textsc{well-constrained}):
  \begin{itemize}
    \item $\vt$ is a well-constrained integer;
    \item the result is $\True$.
  \end{itemize}

  \item All of the following apply (\textsc{parameterized}):
  \begin{itemize}
    \item $\vt$ is a \parameterizedintegertype;
    \item the result is $\True$.
  \end{itemize}

  \item All of the following apply (\textsc{unconstrained}):
  \begin{itemize}
    \item $\vt$ is an unconstrained integer or pending constrained integer;
    \item the result is a type error indicating that a constrained integer type is expected.
  \end{itemize}

  \item All of the following apply (\textsc{conflicting\_type}):
  \begin{itemize}
    \item $\vt$ is not an integer type;
    \item the result is a type error indicating the type conflict.
  \end{itemize}
\end{itemize}

\isempty{\subsubsection{Example}}

\CodeSubsection{\CheckConstrainedIntegerBegin}{\CheckConstrainedIntegerEnd}{../Typing.ml}

\subsubsection{Formally}
\begin{mathpar}
\inferrule[well-constrained]{}
{
  \checkconstrainedinteger(\tenv, \TInt(\wellconstrained(\Ignore))) \typearrow \True
}
\and
\inferrule[parameterized]{}
{
  \checkconstrainedinteger(\tenv, \TInt(\parameterized(\Ignore))) \typearrow \True
}
\and
\inferrule[unconstrained]{
  \astlabel(\vc) = \unconstrained \;\lor\; \astlabel(\vc) = \pendingconstrained
}
{
  \checkconstrainedinteger(\tenv, \TInt(\vc)) \typearrow \\
  \TypeErrorVal{ConstrainedIntegerExpected}
}
\and
\inferrule[conflicting\_type]{
  \astlabel(\vt) \neq \TInt
}{
  \checkconstrainedinteger(\tenv, \vt) \typearrow \TypeErrorVal{TypeConflict}
}
\end{mathpar}

\section{Constrained Types\label{sec:ConstrainedTypes}}
\begin{itemize}
  \item A \emph{constrained type} is a type whose definition is parameterized by an expression.
        In ASL only integer types and bitvector types can be constrained.
        An integer type with a non-empty list of constrained is referred to as a
        \hypertarget{def-wellconstrainedintegertype}{\wellconstrainedintegertype}.
  \item A type which is not constrained is \emph{unconstrained}.
        Specifically, the \hypertarget{def-unconstrainedintegertype}{\unconstrainedintegertype}.
  \item A constrained type with a non-empty constraint is \emph{well-constrained}.
  \hypertarget{def-parameterizedintegertype}
  \item A \hypertarget{def-pendingconstrainedintegertype}{\emph{\pendingconstrainedintegertype}} is an integer type whose constraints will be inferred during type checking.
  \item A \emph{\parameterizedintegertype} is an implicit type of a subprogram parameter.
  \end{itemize}
The widths of bitvector storage elements are constrained integers.

\hypertarget{def-isunconstrainedinteger}{}
\hypertarget{def-isparameterizedinteger}{}
\hypertarget{def-iswellconstrainedinteger}{}
We use the following helper predicates to classify integer types:
\[
  \begin{array}{rcl}
  \isunconstrainedinteger(\overname{\ty}{\vt}) &\aslto& \Bool\\
  \isparameterizedinteger(\overname{\ty}{\vt}) &\aslto& \Bool\\
  \iswellconstrainedinteger(\overname{\ty}{\vt}) &\aslto& \Bool
  \end{array}
\]
Those are defined as follows:
\[
  \begin{array}{rcl}
  \isunconstrainedinteger(\vt) &\triangleq& \vt = \TInt(c) \land \astlabel(c)=\unconstrained\\
  \isparameterizedinteger(\vt) &\triangleq& \vt = \TInt(c) \land \astlabel(c)=\parameterized\\
  \iswellconstrainedinteger(\vt) &\triangleq& \vt = \TInt(c) \land \astlabel(c)=\wellconstrained\\
\end{array}
\]\lrmcomment{This is related to \identd{ZTPP}, \identr{WJYH}, \identr{HJPN}, \identr{CZTX}, \identr{TPHR}.}

\paragraph{Shorthand Notations:}

\hypertarget{def-unconstrainedinteger}{}
We use the shorthand notation $\unconstrainedinteger$ to denote the unconstrained integer type: $\TInt(\unconstrained)$.
