%%%%%%%%%%%%%%%%%%%%%%%%%%%%%%%%%%%%%%%%%%%%%%%%%%%%%%%%%%%%%%%%%%%%%%%%%%%%%%%
\chapter{Semantics Utility Rules\label{chap:SemanticsUtilityRules}}
%%%%%%%%%%%%%%%%%%%%%%%%%%%%%%%%%%%%%%%%%%%%%%%%%%%%%%%%%%%%%%%%%%%%%%%%%%%%%%%

In this chapter, we define helper relations for operating on \nativevalues,
\hyperlink{def-envs}{environments}, and operations involving values and types.

We now define the following relations:
\begin{itemize}
  \item SemanticsRule.GetStackSize (see \secref{SemanticsRule.GetStackSize})
  \item SemanticsRule.SetStackSize (see \secref{SemanticsRule.SetStackSize})
  \item SemanticsRule.IncrStackSize (see \secref{SemanticsRule.IncrStackSize})
  \item SemanticsRule.DecrStackSize (see \secref{SemanticsRule.DecrStackSize})
  \item SemanticsRule.RemoveLocal \secref{SemanticsRule.RemoveLocal};
  \item SemanticsRule.ReadIdentifier \secref{SemanticsRule.ReadIdentifier};
  \item SemanticsRule.WriteIdentifier \secref{SemanticsRule.WriteIdentifier};
  \item SemanticsRule.CreateBitvector \secref{SemanticsRule.CreateBitvector};
  \item SemanticsRule.ConcatBitvectors \secref{SemanticsRule.ConcatBitvectors};
  \item SemanticsRule.ReadFromBitvector \secref{SemanticsRule.ReadFromBitvector};
  \item SemanticsRule.WriteToBitvector \secref{SemanticsRule.WriteToBitvector};
  \item SemanticsRule.GetIndex \secref{SemanticsRule.GetIndex};
  \item SemanticsRule.SetIndex \secref{SemanticsRule.SetIndex};
  \item SemanticsRule.GetField \secref{SemanticsRule.GetField};
  \item SemanticsRule.SetField \secref{SemanticsRule.SetField};
  \item SemanticsRule.DeclareLocalIdentifier \secref{SemanticsRule.DeclareLocalIdentifier};
  \item SemanticsRule.DeclareLocalIdentifierM\secref{SemanticsRule.DeclareLocalIdentifierM};
  \item SemanticsRule.DeclareLocalIdentifierMM \secref{SemanticsRule.DeclareLocalIdentifierMM};
\end{itemize}

\subsubsection{SemanticsRule.GetStackSize\label{sec:SemanticsRule.GetStackSize}}
\hypertarget{def-getstacksize}{}
The function
\[
\getstacksize(\overname{\denv}{\dynamicenvs} \aslsep \overname{\name}{\identifier}) \aslto \overname{\vs}{\N}
\]
retrieves the value associated with $\name$ in $\denv.\stacksize$ or $0$ if no value is associated with it.

\subsubsection{Prose}
define $\vs$ is $0$ if no value is associated with $\name$ in $\denv.\stacksize$ and the value bound to
$\name$ in $\denv.\stacksize$ otherwise.

\subsubsection{Formally}
\begin{mathpar}
\inferrule{
  \vs \eqdef \choice{\name \in \dom(\denv.\stacksize)}{\denv.\stacksize(\name)}{0}
}{
  \getstacksize(\denv, \name) \evalarrow \vs
}
\end{mathpar}

\subsubsection{SemanticsRule.SetStackSize\label{sec:SemanticsRule.SetStackSize}}
\hypertarget{def-setstacksize}{}
The function
\[
\setstacksize(\overname{\genv}{\globaldynamicenvs} \aslsep \overname{\name}{\identifier} \aslsep \overname{\vv}{\N}) \aslto
\overname{\newgenv}{\dynamicenvs}
\]
updates the value bound to $\name$ in $\genv.\storage$ to $\vv$, yielding the new global dynamic environment $\newgenv$.

\subsubsection{Prose}
define $\newdenv$ as $\genv$ updated to bind $\name$ to $\vv$ in $\genv.\stacksize$.

\subsubsection{Formally}
\begin{mathpar}
\inferrule{}{
  \setstacksize(\genv, \name, \vv) \evalarrow \overname{\genv.\stacksize[\name\mapsto\vv]}{\newgenv}
}
\end{mathpar}

\subsubsection{SemanticsRule.IncrStackSize\label{sec:SemanticsRule.IncrStackSize}}
\hypertarget{def-incrstacksize}{}
The function
\[
\incrstacksize(\overname{\genv}{\globaldynamicenvs} \aslsep \overname{\name}{\identifier}) \aslto
\overname{\newgenv}{\globaldynamicenvs}
\]
increments the value associated with $\name$ in $\genv.\stacksize$, yielding the updated global dynamic environment $\newgenv$.

\subsubsection{Prose}
All of the following apply:
\begin{itemize}
  \item applying $\getstacksize$ to $\name$ in $(\genv, \emptyfunc)$ yields $\vprev$;
  \item applying $\setstacksize$ to $\name$ and $\vprev + 1$ in $\genv$ yields $\newgenv$.
\end{itemize}

\subsubsection{Formally}
\begin{mathpar}
\inferrule{
  \getstacksize((\genv, \emptyfunc)v, \name) \evalarrow \vprev\\
  \setstacksize(\genv, \name, \vprev + 1) \evalarrow \newgenv
}{
  \incrstacksize(\genv, \name) \evalarrow \newgenv
}
\end{mathpar}

\subsubsection{SemanticsRule.DecrStackSize\label{sec:SemanticsRule.DecrStackSize}}
\hypertarget{def-decrstacksize}{}
The function
\[
\decrstacksize(\overname{\genv}{\globaldynamicenvs} \aslsep \overname{\name}{\identifier}) \aslto
\overname{\newgenv}{\globaldynamicenvs} \cup \overname{\newdenv}{\dynamicenvs}
\]
decrements the value associated with $\name$ in $\genv.\stacksize$, yielding the updated global dynamic environment $\newgenv$.
It is assumed that $\getstacksize((\genv, \emptyfunc), \name)$ yields a positive value.

\subsubsection{Prose}
All of the following apply:
\begin{itemize}
  \item applying $\getstacksize$ to $\name$ in $(\genv, \emptyfunc)$ yields $\vprev$;
  \item applying $\setstacksize$ to $\name$ and $\vprev - 1$ in $\genv$ yields $\newgenv$.
\end{itemize}

\subsubsection{Formally}
\begin{mathpar}
\inferrule{
  \getstacksize((\genv, \emptyfunc), \name) \evalarrow \vprev\\
  \setstacksize(\genv, \name, \vprev - 1) \evalarrow \newgenv
}{
  \decrstacksize(\genv, \name) \evalarrow \newgenv
}
\end{mathpar}

\subsubsection{SemanticsRule.RemoveLocal \label{sec:SemanticsRule.RemoveLocal}}
\subsubsection{Prose}
The relation
\hypertarget{def-removelocal}{}
\[
  \removelocal(\overname{\envs}{\env} \aslsep \overname{\Identifiers}{\name}) \;\aslrel\; \overname{\envs}{\newenv}
\]
removes the binding of the identifier $\name$ from the local storage of the environment $\env$.

Removal of the identifier $\name$ from the local storage of the environment $\env$
is the environment $\newenv$ and all of the following apply:
\begin{itemize}
  \item $\env$ consists of the static environment $\tenv$ and dynamic environment $\denv$;
  \item $\newenv$ consists of the static environment $\tenv$ and the dynamic environment
  with the same global component as $\denv$ --- $G^\denv$, and local component $L^\denv$,
  with the identifier $\name$ removed from its domain.
\end{itemize}

\subsubsection{Formally}
(Recall that $[\name\mapsto\bot]$ means that $\name$ is not in the domain of the resulting function.)
\begin{mathpar}
  \inferrule{
    \env \eqname (\tenv, (G^\denv, L^\denv))\\
    \newenv \eqdef (\tenv, (G^\denv, L^\denv[\name \mapsto \bot]))
  }
  {
    \removelocal(\env, \name) \evalarrow \newenv
  }
\end{mathpar}

\subsubsection{SemanticsRule.ReadIdentifier \label{sec:SemanticsRule.ReadIdentifier}}
\subsubsection{Prose}
The relation
\hypertarget{def-readidentifier}{}
\[
  \readidentifier(\overname{\Identifiers}{\name}\aslsep\overname{\vals}{\vv}) \;\aslrel\; (\overname{\vals}{\vv} \times \XGraphs)
\]
reads a value $\vv$ into a storage element given by an identifier $\name$.
The result is the value and an execution graph containing a single Read Effect,
which denoting reading from $\name$.

\subsubsection{Formally}
\begin{mathpar}
  \inferrule{}
  {
    \readidentifier(\name, \vv) \evalarrow (\vv, \ReadEffect(\name))
  }
\end{mathpar}

\subsubsection{SemanticsRule.WriteIdentifier \label{sec:SemanticsRule.WriteIdentifier}}
\subsubsection{Prose}
The relation
\hypertarget{def-writeidentifier}{}
\[
  \writeidentifier(\overname{\Identifiers}{\name}\aslsep\overname{\vals}{\vv}) \;\aslrel\; \XGraphs
\]
writes the value $\vv$ into a storage element given by an identifier $\name$.
The result is an execution graph containing a single Write Effect,
which denotes writing into $\name$.

\subsubsection{Formally}
\begin{mathpar}
  \inferrule{}
  {
    \writeidentifier(\name, \vv) \evalarrow \WriteEffect(\name)
  }
\end{mathpar}

\subsubsection{SemanticsRule.CreateBitvector \label{sec:SemanticsRule.CreateBitvector}}
\subsubsection{Prose}
The relation
\[
  \createbitvector(\overname{\vals^*}{\vvs}) \;\aslrel\; \tbitvector
\]
creates a native vector value bitvector from a sequence of values $\vvs$.

\subsubsection{Formally}
\begin{mathpar}
  \inferrule{}
  {
    \createbitvector(\vvs) \evalarrow \nvbitvector{\vvs}
  }
\end{mathpar}

\subsubsection{SemanticsRule.ConcatBitvectors \label{sec:SemanticsRule.ConcatBitvectors}}
\subsubsection{Prose}
The relation
\[
  \concatbitvectors(\overname{\tbitvector^*}{\vvs}) \;\aslrel\; \tbitvector
\]
transforms a (possibly empty) list of bitvector \nativevalues\ $\vvs$ into a single bitvector.

\subsubsection{Formally}
\begin{mathpar}
  \inferrule[ConcatBitvector.Empty]{}
  {
    \concatbitvectors(\emptylist) \evalarrow \nvbitvector(\emptylist)
  }
  \and
  \inferrule[ConcatBitvector.NonEmpty]{
    \vvs \eqname [\vv] \concat \vvs'\\
    \vv\eqname\nvbitvector(\bv)\\
    \concatbitvectors(\vvs') \evalarrow \nvbitvector(\bv')\\
    \vres \eqdef \bv \concat \bv'
  }
  {
    \concatbitvectors(\vvs) \evalarrow \nvbitvector(\vres)
  }
\end{mathpar}

\subsubsection{SemanticsRule.ReadFromBitvector \label{sec:SemanticsRule.ReadFromBitvector}}
\hypertarget{def-readfrombitvector}{}
The relation
\[
  \readfrombitvector(\overname{\tbitvector}{\bv} \aslsep \overname{(\tint\times\tint)^*}{\slices}) \;\aslrel\;
  \overname{\tbitvector}{\vv} \cup \overname{\TDynError}{\DynErrorConfig}
\]
reads from a bitvector $\bv$, or an integer seen as a bitvector, the indices specified by the list of slices $\slices$,
thereby concatenating their values.

\subsubsection{Prose}
One of the following applies:
\begin{itemize}
  \item all indices are in range for $\bv$ and the returned bitvector consists of the concatenated bits specified
  by the slices.
  \item there exists an out-of-range index and an error is returned.
\end{itemize}

\subsubsection{Formally}
We start by introducing a few helper relations.

\hypertarget{def-positioninrange}{}
The predicate $\positioninrange(\vs, \vl, n)$ checks whether the indices starting at index $\vs$ and
up to $\vs + \vl$, inclusive, would refer to actual indices of a bitvector of length $n$:
\[
  \positioninrange(\vs, \vl, \vn) \triangleq (\vs \geq 0) \land (\vl \geq 0) \land (\vs + \vl < \vn) \enspace.
\]

The relation
\hypertarget{def-slicestopositions}{}
\[
  \slicestopositions(\overname{\N}{\vn} \aslsep \overname{(\overname{\tint}{\vs_i}\times\overname{\tint}{\vl_i})^+}{\slices}) \;\aslrel\;
  (\overname{\N^*}{\positions} \cup\ \TDynError)
\]
returns the list of positions (indices) specified by the slices $\slices$,
unless an index would be out of range for a bitvector of length $\vn$, in which case it returns an error configuration.

\begin{mathpar}
  \inferrule[SlicesToPositionsOutOfRange]{
    \slices \eqname [i=1..k: (\nvint(\vs_i), \nvint(\vl_i))]\\
    j \in 1..k: \neg\positioninrange(\vs_j, \vl_j, \vn)
  }
  {
    \slicestopositions(\vn, \slices) \evalarrow \ErrorVal{Slice\_PositionOutOfRange}
  }
  \and
  \inferrule[SlicesToPositionsInRange]{
    \slices \eqname [i=1..k: (\nvint(\vs_i), \nvint(\vl_i))]\\
    i=1..k: \positioninrange(\vs_i, \vl_i, \vn)\\
    \positions \eqdef [\vs_1,\ldots,\vs_1+\vl_1] \concat \ldots \concat [\vs_k,\ldots,\vs_k+\vl_k]\\
  }
  {
    \slicestopositions(\vn, \slices) \evalarrow \positions
  }
\end{mathpar}

\hypertarget{def-asbitvector}{}
The function $\asbitvector : (\tbitvector\cup\tint) \rightarrow \{0,1\}^*$ transforms \nativevalue\  integers and \nativevalue\  bitvectors into
a sequence of binary values:
\begin{mathpar}
  \inferrule[AsBitvectorBitvector]{}
  {
    \asbitvector(\nvbitvector(\bv)) \evalarrow \bv
  }
  \and
  \inferrule[AsBitvectorInt]{
    \bv \eqdef \text{ two's complement representation of }n
  }
  {
    \asbitvector(\nvint(n)) \evalarrow \bv
  }
\end{mathpar}

Finally, the rules below distinguish between empty bitvectors and non-empty bitvectors.
\begin{mathpar}
  \inferrule[ReadFromBitvector.Empty]{}
  {
    \readfrombitvector(\bv, \emptylist) \evalarrow \nvbitvector(\emptylist)
  }
  \and
  \inferrule[ReadFromBitvector.NonEmpty]{
    \asbitvector(\bv) \eqdef \vb_n \ldots \vb_1\\
    \slicestopositions(n, \slices) \evalarrow [j_{1..m}] \OrDynError\\\\
    \vv \eqdef \nvbitvector(\vb_{j_m + 1}\ldots\vb_{j_1 + 1})
  }
  {
    \readfrombitvector(\bv, \slices) \evalarrow \vv
  }
\end{mathpar}
Notice that the bits of a bitvector go from the least significant bit being on the right to the most significant bit being on the left,
which is reflected by how the rules list the bits.
The effect of placing the bits in sequence is that of concatenating the results
from all of the given slices.
Also notice that bitvector bits are numbered from 1 and onwards, which is why we add 1 to the indices specified
by the slices when accessing a bit.

\subsubsection{SemanticsRule.WriteToBitvector \label{sec:SemanticsRule.WriteToBitvector}}
\subsubsection{Prose}
The relation
\[
  \writetobitvector(\overname{(\tint\times\tint)^*}{\slices} \aslsep \overname{\tbitvector}{\src} \aslsep \overname{\tbitvector}{\dst})
  \;\bigtimes\; \overname{\tbitvector}{\vv} \cup \overname{\TDynError}{\DynErrorConfig}
\]
overwrites the bits of $\dst$ at the positions given by $\slices$ with the bits of $\src$
and one of the following applies:
\begin{itemize}
  \item all positions specified by $\slices$ are within range for $\dst$ and the modified version
  of $\dst$ with the bits of $\src$ at the specified positions is returned;
  \item there exists a position in $\slices$ that is not in range for $\dst$ and an error is returned.
\end{itemize}

\subsubsection{Formally}
\begin{mathpar}
  \inferrule[WriteToBitvector.Empty]{}
  {
    \writetobitvector(\emptylist, \nvbitvector(\emptylist), \nvbitvector(\emptylist)) \evalarrow \nvbitvector(\emptylist)
  }
  \and
  \inferrule[WriteToBitvector.NonEmpty]{
    \vs_n \ldots \vs_1 \eqdef \asbitvector(\src)\\
    \vd_n \ldots \vd_1 \eqdef \asbitvector(\dst)\\
    \slicestopositions(n, \slices) \evalarrow \positions \OrDynError\\\\
    {\mathit{bit} = \lambda i \in 1..n.\left\{ \begin{array}{ll}
     \vs_i & i \in \positions\\
     \vd_i & \text{otherwise}
    \end{array} \right.}\\
    \vv\eqdef\nvbitvector(\mathit{bit}(n-1)\ldots \mathit{bit}(0))\\
  }
  {
    \writetobitvector(\slices, \src, \dst) \evalarrow \vv
  }
\end{mathpar}

\subsubsection{SemanticsRule.GetIndex \label{sec:SemanticsRule.GetIndex}}
\subsubsection{Prose}
The relation
\hypertarget{def-getindex}{}
\[
  \getindex(\overname{\N}{\vi} \aslsep \overname{\tvector}{\vvec}) \;\aslrel\; \overname{\tvector}{\vv_{\vi}}
\]
reads the value $\vv_i$ from the vector of values $\vvec$ at the index $\vi$.

\subsubsection{Formally}
\begin{mathpar}
  \inferrule{
    \vvec \eqname \vv_{0..k}\\
    \vi \leq k\\
  }
  {
    \getindex(\vi, \vvec) \evalarrow \vv_{\vi}
  }
\end{mathpar}
Notice that there is no rule to handle the case where the index is out of range ---
this is guaranteed by the type-checker not to happen. Specifically,
\begin{itemize}
  \item \texttt{TypingRule.EGetArray} ensures that an index is within the bounds of the array
  being accessed via a check that the type of the index satisfies the type of the array size.
  \item Typing rules \texttt{TypingRule.LEDestructuring}, \texttt{TypingRule.PTuple},
  and \\ \texttt{TypingRule.LDTuple} use the same index sequences for the tuples
  involved and the corresponding lists of expressions.
\end{itemize}
If the rules listed above do not hold the type checker fails.

\subsubsection{SemanticsRule.SetIndex \label{sec:SemanticsRule.SetIndex}}
\subsubsection{Prose}
The relation
\hypertarget{def-setindex}{}
\[
  \setindex(\overname{\N}{\vi} \aslsep \overname{\vals}{\vv} \aslsep \overname{\tvector}{\vvec}) \;\aslrel\; \overname{\tvector}{\vres}
\]
overwrites the value at the given index $\vi$ in a vector of values $\vvec$ with the new value $\vv$.

\subsubsection{Formally}
\begin{mathpar}
  \inferrule{
    \vvec \eqname \vu_{0..k}\\
    \vi \leq k\\
    \vres \eqname \vw_{0..k}\\
    \vv \eqdef \vw_{\vi} \\
    j \in \{0..k\} \setminus \{\vi\}.\ \vw_{j} = \vu_j\\
  }
  {
    \setindex(\vi, \vv, \vvec) \evalarrow \vres
  }
\end{mathpar}
Similar to $\getindex$, there is no need to handle the out-of-range index case.

\subsubsection{SemanticsRule.GetField \label{sec:SemanticsRule.GetField}}
\subsubsection{Prose}
The relation
\hypertarget{def-getfield}{}
\[
  \getfield(\overname{\Identifiers}{\name} \aslsep \overname{\trecord}{\record}) \;\aslrel\; \vals
\]
retrieves the value corresponding to the field name $\name$ from the record value $\record$.

\subsubsection{Formally}
\begin{mathpar}
  \inferrule{
    \record \eqname \nvrecord{\fieldmap}
  }
  {
    \getfield(\name, \record) \evalarrow \fieldmap(\name)
  }
\end{mathpar}
The type-checker ensures, via TypingRule.EGetRecordField, that the field $\name$ exists in $\record$.

\subsubsection{SemanticsRule.SetField \label{sec:SemanticsRule.SetField}}
\subsubsection{Prose}
The relation
\hypertarget{def-setfield}{}
\[
  \setfield(\overname{\Identifiers}{\name} \aslsep \overname{\vals}{\vv} \aslsep \overname{\trecord}{\record}) \;\aslrel\; \trecord
\]
overwrites the value corresponding to the field name $\name$ in the record value $\record$ with the value $\vv$.

\subsubsection{Formally}
\begin{mathpar}
  \inferrule{
    \record \eqname \nvrecord{\fieldmap}\\
    \fieldmap' \eqdef \fieldmap[\name\mapsto\vv]
  }
  {
    \setfield(\name, \vv, \record) \evalarrow \nvrecord{\fieldmap'}
  }
\end{mathpar}
The type-checker ensures that the field $\name$ exists in $\record$.

\subsubsection{SemanticsRule.DeclareLocalIdentifier \label{sec:SemanticsRule.DeclareLocalIdentifier}}
\subsubsection{Prose}
The relation
\hypertarget{def-declarelocalidentifier}{}
\[
  \declarelocalidentifier(\overname{\envs}{\env} \aslsep \overname{\Identifiers}{\name} \aslsep \overname{\vals}{\vv}) \;\aslrel\;
  (\overname{\envs}{\newenv}\times\overname{\XGraphs}{\vg})
\]
associates $\vv$ to $\name$ as a local storage element in the environment $\env$ and
returns the updated environment $\newenv$ with the execution graph consisting of a Write Effect to $\name$.

\subsubsection{Formally}
\begin{mathpar}
  \inferrule{
    \vg \eqdef \WriteEffect(\name)\\
    \env \eqname (\tenv, (G^\denv, L^\denv))\\
    \newenv \eqdef (\tenv, (G^\denv, L^\denv[\name\mapsto \vv]))
  }
  { \declarelocalidentifier(\env, \name, \vv) \evalarrow (\newenv, \vg)  }
\end{mathpar}

\subsubsection{SemanticsRule.DeclareLocalIdentifierM \label{sec:SemanticsRule.DeclareLocalIdentifierM}}
\subsubsection{Prose}
\hypertarget{def-declarelocalidentifierm}{}
The relation
\[
  \declarelocalidentifierm(\overname{\envs}{\env} \aslsep
   \overname{\Identifiers}{\vx} \aslsep
   \overname{(\overname{\vals}{\vv}\times\overname{\XGraphs}{\vg})}{\vm}) \;\aslrel\;
  (\overname{\envs}{\newenv} \times \overname{\XGraphs}{\newg})
\]
declares the local identifier $\vx$ in the environment $\env$, in the context
of the value-graph pair $(\vv, \vg)$, and all of the following apply:
\begin{itemize}
  \item \newenv\ is the environment $\env$ modified to declare the variable $\vx$ as a local storage element;
  \item $\vgone$ is the execution graph resulting from the declaration of $\vx$;
  \item $\vgtwo$ is the execution graph resulting from the ordered composition of $\vg$ and $\vgone$
  with the $\asldata$ edge.
\end{itemize}

\subsubsection{Formally}
\begin{mathpar}
  \inferrule{
    \vm \eqname (\vv, \vg)\\
    \declarelocalidentifier(\env, \vx, \vv) \evalarrow (\newenv, \vgone)\\
    \newg \eqdef \ordered{\vg}{\asldata}{\vgone}
  }
  {
    \declarelocalidentifierm(\env, \vx, \vm) \evalarrow (\newenv, \newg)
  }
\end{mathpar}

\subsubsection{SemanticsRule.DeclareLocalIdentifierMM \label{sec:SemanticsRule.DeclareLocalIdentifierMM}}
\subsubsection{Prose}
\hypertarget{def-declarelocalidentifermm}{}
The relation
\[
  \declarelocalidentifiermm(\overname{\envs}{\env} \aslsep
   \overname{\Identifiers}{\vx} \aslsep
   \overname{(\overname{\vals}{\vv}\times\overname{\XGraphs}{\vg})}{\vm}) \;\aslrel\;
  (\overname{\envs}{\newenv} \times \overname{\XGraphs}{\vgtwo})
\]
declares the local identifier $\vx$ in the environment $\env$,
in the context of the value-graph pair $(\vv, \vg)$,
and all of the following apply:
\begin{itemize}
  \item \newenv\ is the environment $\env$ modified to declare the variable $\vx$ as a local storage element;
  \item $\vgone$ is the execution graph resulting from the declaration of $\vx$;
  \item $\vgtwo$ is the execution graph resulting from the ordered composition of $\vg$ and $\vgone$
  with the $\aslpo$ edge.
\end{itemize}

\subsubsection{Formally}
\begin{mathpar}
  \inferrule{
    \declarelocalidentifierm(\env, \vm) \evalarrow (\newenv, \vgone)\\
    \vgtwo \eqdef \ordered{\vg}{\aslpo}{\vgone}
  }
  {
    \declarelocalidentifiermm(\env, \vx, \vm) \evalarrow (\newenv, \vgtwo)
  }
\end{mathpar}
