\chapter{Bitfields\label{chap:Bitfields}}

Bitvector types allow defining bitslices of bitvectors, to be treated as named
fields, which can be read or written. \lrmcomment{\identi{KGMC}}

Individual bitfields are grammatically derived from $\Nbitfield$ and represented as ASTs by $\bitfield$.
Bitfields are not associated with a semantic relation.

\subsection{Example}
The following code declares a global variable whose type is a bitvector with bitfields.
\ASLExample{\definitiontests/Bitfields.asl}
\begin{itemize}
  \item The expression \texttt{myData.flag} evaluates to the value \texttt{myData[4]} of type \texttt{bits(1)};
  \item The expression \texttt{myData.data} evaluates to the value \texttt{[myData[3:0], myData[8:5]]} of type \texttt{bits(8)};
  \item There is no bitfield which accesses \texttt{myData[15:10]};
  \item The value field overlaps with the other fields;
  \item The slices \texttt{3:0} and \texttt{8:5} which define \texttt{data} do not overlap.
\end{itemize}
Note that in the \texttt{data} bitfield, bits \texttt{3:0} come before bits \texttt{8:5},
which is a different order from their occurrence in \texttt{myData}.

\hypertarget{def-singleslice}{}
\hypertarget{def-rangeslice}{}
\hypertarget{def-lengthslice}{}
\hypertarget{def-scaledslice}{}
We refer to a slice of the form \texttt{[$\ve$]} as a \singleslice,
a slice of the form \texttt{[$\veone$:$\vetwo$]} as a \rangeslice,
a slice of the form \texttt{[$\veone$+:$\vetwo$]} as a \lengthslice,
and slice of the form \texttt{[$\veone$*:$\vetwo$]} as a \scaledslice.

\section{Nested Bitfields}
Bitfields may have nested bitfields. This can have several uses, one of which being being able to define two
different views of a register.

\subsection{Example}
\ASLExample{\definitiontests/Bitfields_nested.asl}

\subsection{Syntax}
\begin{flalign*}
\Nbitfields \derivesinline\ & \Tlbrace \parsesep \TClist{\Nbitfield} \parsesep \Trbrace &\\
\Nbitfield \derivesinline\ & \Nnamedslices \parsesep \Tidentifier &\\
                  |\ & \Nnamedslices \parsesep \Tidentifier \parsesep \Nbitfields &\\
                  |\ & \Nnamedslices \parsesep \Tidentifier \parsesep \Tcolon \parsesep \Nty &\\
\Nnamedslices \derivesinline\ & \Tlbracket \parsesep \NClist{\Nslice} \parsesep \Trbracket &
\end{flalign*}

\subsection{Abstract Syntax}
\begin{flalign*}
\bitfield \derives\ & \BitFieldSimple(\identifier, \slice^{*})
  & \hypertarget{ast-bitfieldnested}{}\\
  |\ & \BitFieldNested(\identifier, \slice^{*}, \bitfield^{*})
  & \hypertarget{ast-bitfieldtype}{}\\
  |\ & \BitFieldType(\identifier, \slice^{*}, \ty) &
\end{flalign*}

\subsubsection{ASTRule.Bitfields\label{sec:ASTRule.Bitfields}}
\hypertarget{build-bitfields}{}
The function
\[
  \buildbitfields(\overname{\parsenode{\Nbitfields}}{\vparsednode}) \;\aslto\; \overname{\bitfield^*}{\vastnode}
\]
transforms a parse node $\vparsednode$ into an AST node $\vastnode$.

\begin{mathpar}
\inferrule{
  \buildtclist[\buildbitfield](\vbitfields) \astarrow \vbitfieldasts
}{
  \buildbitfields(\Nbitfields(\Tlbrace, \namednode{\vbitfields}{\TClist{\Nbitfield}}, \Trbrace)) \astarrow
  \overname{\vbitfieldasts}{\vastnode}
}
\end{mathpar}

\subsubsection{ASTRule.Bitfield\label{sec:ASTRule.Bitfield}}
\hypertarget{build-bitfield}{}
The function
\[
  \buildbitfield(\overname{\parsenode{\Nbitfield}}{\vparsednode}) \;\aslto\; \overname{\bitfield}{\vastnode}
\]
transforms a parse node $\vparsednode$ into an AST node $\vastnode$.

\begin{mathpar}
\inferrule[simple]{}{
  \buildbitfields(\Nbitfield(\punnode{\Nnamedslices}, \Tidentifier(\vx))) \astarrow
  \overname{\BitFieldSimple(\vx, \astof{\vnamedslices})}{\vastnode}
}
\end{mathpar}

\begin{mathpar}
\inferrule[nested]{}{
  {
    \begin{array}{r}
      \buildbitfields(\Nbitfield(\punnode{\Nnamedslices}, \Tidentifier(\vx), \punnode{\Nbitfields})) \astarrow\\
      \overname{\BitFieldNested(\vx, \astof{\vnamedslices}, \astof{\vbitfields})}{\vastnode}
    \end{array}
  }
}
\end{mathpar}

\begin{mathpar}
\inferrule[type]{}{
  {
    \begin{array}{r}
      \buildbitfields(\Nbitfield(\punnode{\Nnamedslices}, \Tidentifier(\vx), \Tcolon, \punnode{\Nty})) \astarrow\\
      \overname{\BitFieldType(\vx, \astof{\vnamedslices}, \astof{\tty})}{\vastnode}
    \end{array}
  }
}
\end{mathpar}

\subsubsection{ASTRule.NamedSlices\label{sec:ASTRule.NamedSlices}}
\hypertarget{build-namedslices}{}
The function
\[
  \buildnamedslices(\overname{\parsenode{\Nnamedslices}}{\vparsednode}) \;\aslto\; \overname{\slice^+}{\vastnode}
\]
transforms a parse node $\vparsednode$ into an AST node $\vastnode$.

\begin{mathpar}
\inferrule{
  \buildclist[\buildslice](\vslices) \astarrow \vsliceasts
}{
  \buildnamedslices(\Nnamedslices(\Tlbracket, \namednode{\vslices}{\NClist{\Nslice}}, \Trbracket)) \astarrow
  \overname{\vsliceasts}{\vastnode}
}
\end{mathpar}

\section{Typing Bitfields}
\subsubsection{TypingRule.TBitFields\label{sec:TypingRule.TBitFields}}
\hypertarget{def-annotatebitfields}{}
The function
\[
  \annotatebitfields(\overname{\staticenvs}{\tenv} \aslsep \overname{\expr}{\ewidth} \aslsep \overname{\bitfield^*}{\fields})
  \aslto \overname{\bitfield^*}{\newfields} \cup \overname{\TTypeError}{\TypeErrorConfig}
\]
annotates a list of bitfields --- $\fields$ --- with an expression denoting the overall number of bits in the containing
bitvector type --- $\ewidth$,
in an environment $\tenv$,
resulting in $\newfields$ --- the \typedast\ for $\fields$ and $\ewidth$. \ProseOtherwiseTypeError

\subsubsection{Prose}
All of the following apply:
\begin{itemize}
  \item checking that the list of bitfield names in $\bitfields$ does not contain duplicates yields $\True$\ProseOrTypeError;
  \item symbolically simplifying $\ewidth$ in $\tenv$ via $\reduceconstants$ yields the literal integer for $\width$\ProseOrTypeError;
  \item annotating each bitfield $\field$ in $\fields$ with width $\width$ in $\tenv$ yields the corresponding annotated
  bitfield $\newfield$\ProseOrTypeError;
  \item $\newfields$ is the list of annotated bitfields.
\end{itemize}

\subsubsection{Formally}
\begin{mathpar}
\inferrule{
  \names \eqdef [\field\in\fields: \bitfieldgetname(\field)]\\
  \checknoduplicates(\names) \typearrow \True \OrTypeError\\\\
  \reduceconstants(\tenv, \ewidth) \typearrow \lint(\width) \OrTypeError\\\\
  \field\in\fields: \annotatebitfield(\tenv, \width, \field) \typearrow \newfield \OrTypeError\\\\
  \newfields \eqdef [\field\in\fields: \newfield]
}{
  \annotatebitfields(\tenv, \ewidth, \fields) \typearrow \newfields
}
\end{mathpar}

\CodeSubsection{\TBitFieldsBegin}{\TBitFieldsEnd}{../Typing.ml}

\subsubsection{TypingRule.BitFieldGetName \label{sec:TypingRule.BitFieldGetName}}
\hypertarget{def-bitfieldgetname}{}
The function
\[
  \bitfieldgetname : \overname{\bitfield}{\vbf} \aslto \overname{\identifier}{\name}
\]
returns the name of a bitfield --- $\name$, given a bitfield $\vbf$.

\subsubsection{Prose}
One of the following applies:
\begin{itemize}
  \item $\vb$ is a simple bitfield with name $\name$, that is, $\BitFieldSimple(\name, \Ignore)$;
  \item $\vb$ is a nested bitfield with name $\name$, that is, $\BitFieldNested(\name, \Ignore, \Ignore)$;
  \item $\vb$ is a typed bitfield with name $\name$, that is, $\BitFieldType(\name, \Ignore, \Ignore)$.
\end{itemize}

\subsubsection{Formally}
\begin{mathpar}
  \inferrule[simple]{}{
    \bitfieldgetname(\BitFieldSimple(\name, \Ignore)) \typearrow \name
  }
  \and
  \inferrule[nested]{}{
    \bitfieldgetname(\BitFieldNested(\name, \Ignore, \Ignore)) \typearrow \name
  }
  \and
  \inferrule[type]{}{
    \bitfieldgetname(\BitFieldType(\name, \Ignore, \Ignore)) \typearrow \name
  }
\end{mathpar}

\subsubsection{TypingRule.TBitField\label{sec:TypingRule.TBitField}}
\hypertarget{def-annotatebitfield}{}
The function
\[
  \annotatebitfield(\overname{\staticenvs}{\tenv} \aslsep \overname{\Z}{\width} \aslsep \overname{\bitfield}{\vfield})
  \aslto \overname{\bitfield}{\newfield} \cup \overname{\TTypeError}{\TypeErrorConfig}
\]
annotates a bitfield --- $\vfield$ --- with an integer --- $\width$ --- indicating the number of bits in
the bitvector type that contains $\field$,
in an environment $\tenv$, resulting in an
annotated bitfield --- $\newfield$ --- or a type error, if one is detected.

\subsubsection{Prose}
\begin{itemize}
  \item $\vfield$ is a bitfield with list of slices $\vslices$;
  \item annotating the slices $\vslices$ yields $\slicesone$\ProseOrTypeError;
  \item One of the following applies:
  \begin{itemize}
    \item All of the following apply (\textsc{simple}):
    \begin{itemize}
      \item checking whether the range of positions in $\slicesone$ fits inside $0..\width-1$ yields $\True$\ProseOrTypeError;
      \item $\newfield$ is a bitfield named $\name$ with list of slices $\slicesone$, that is, $\BitFieldSimple(\name, \slicesone)$.
    \end{itemize}

    \item All of the following apply (\textsc{nested}):
    \begin{itemize}
      \item converting the $\slicesone$ into a list of positions with $\width$ and static environment $\tenv$
      yields $\positions$\ProseOrTypeError;
      \item checking that all positions in $\positions$ fit inside $0..\width$ yields \\
            $\True$\ProseOrTypeError;
      \item let $\widthp$ be the length of the list $\positions$;
      \item annotating the bitfields $\bitfieldsp$ with $\widthp$ in static environment $\tenv$ yields $\bitfieldspp$\ProseOrTypeError;
      \item $\newfields$ is the nested bitfield with $\slicesone$ and bitfields $\bitfieldspp$, that is, $\BitFieldNested(\slicesone, \bitfieldspp)$.
    \end{itemize}

    \item All of the following apply (\textsc{type}):
    \begin{itemize}
      \item Annotating the type $\vt$ yields $\vtp$\ProseOrTypeError;
      \item checking whether the range of positions in $\slicesone$ fit inside $0..\width$ yields $\True$\ProseOrTypeError;
      \item converting the list of slices $\slicesone$ into a list of positions in $\tenv$ yields $\positions$\ProseOrTypeError;
      \item checking that all positions in $\positions$ fit inside $0..\width$ yields $\True$\ProseOrTypeError;
      \item let $\widthp$ be the length of the list $\positions$;
      \item checking whether the $\vt$ and the bitvector with $\widthp$ bits have the same width yields $\True$\ProseOrTypeError
      \item $\newfield$ is the typed-bitfield with name $\name$, list of slices $\slicesone$ and type $\vtp$, that is, \BitFieldType(\name, \slicesone, \vtp).
    \end{itemize}
  \end{itemize}
\end{itemize}

\subsubsection{Formally}
\begin{mathpar}
\inferrule[simple]{
  \annotateslices(\tenv, \vslices) \typearrow \slicesone \OrTypeError\\\\
  \commonprefixline\\\\
  \checkslicesinwidth(\tenv, \width, \slicesone) \typearrow \True \OrTypeError
}{
  \annotatebitfield(\tenv, \width, \BitFieldSimple(\name, \vslices)) \typearrow \\
  \BitFieldSimple(\name, \slicesone)
}
\end{mathpar}

\begin{mathpar}
\inferrule[nested]{
  \annotateslices(\tenv, \vslices) \typearrow \slicesone \OrTypeError\\\\
  \commonprefixline\\\\
  \disjointslicestopositions(\tenv, \slicesone) \typearrow \positions \OrTypeError\\\\
  \checkpositionsinwidth(\tenv, \width, \positions) \typearrow \True \OrTypeError\\\\
  \widthp \eqdef \listlen{\positions}\\
  \annotatebitfields(\tenv, \widthp, \bitfieldsp) \typearrow \bitfieldspp \OrTypeError\\
}{
  \annotatebitfield(\tenv, \width, \BitFieldNested(\name, \vslices, \bitfieldsp)) \typearrow \\
  \BitFieldNested(\slicesone, \bitfieldspp)
}
\end{mathpar}

\begin{mathpar}
\inferrule[type]{
  \annotateslices(\tenv, \vslices) \typearrow \slicesone \OrTypeError\\\\
  \commonprefixline\\\\
  \annotatetype{\tenv, \vt} \typearrow \vtp \OrTypeError\\\\
  \checkslicesinwidth(\tenv, \width, \slicesone) \typearrow \True \OrTypeError\\\\
  \disjointslicestopositions(\tenv, \slicesone) \typearrow \positions \OrTypeError\\\\
  \checkpositionsinwidth(\tenv, \slicesone, \width, \positions) \typearrow \True \OrTypeError\\\\
  \widthp \eqdef \listlen{\positions}\\
  \checkbitsequalwidth(\TBits(\widthp, \emptylist), \vt) \typearrow \True \OrTypeError
}{
  \annotatebitfield(\tenv, \width, \BitFieldType(\name, \vslices, \vt)) \typearrow \\
  \BitFieldType(\name, \slicesone, \vtp)
}
\end{mathpar}

\subsubsection{Example}
In the following example, all the uses of bitvector types with bitfields are well-typed:
\ASLExample{\typingtests/TypingRule.TBitField.asl}

\CodeSubsection{\TBitFieldBegin}{\TBitFieldEnd}{../Typing.ml}

\subsubsection{TypingRule.CheckSlicesInWidth\label{sec:TypingRule.CheckSlicesInWidth}}
\hypertarget{def-checkslicesinwidth}{}
The function
\[
  \checkslicesinwidth(\overname{\staticenvs}{\tenv} \aslsep \overname{\Z}{\vwidth} \aslsep \overname{\slice^*}{\vslices})
  \aslto \{\True\} \cup \overname{\TTypeError}{\TypeErrorConfig}
\]
checks whether the slices in $\vslices$ fit within the bitvector width given by $\vwidth$ in $\tenv$,
yielding $\True$. \ProseOtherwiseTypeError

\subsubsection{Prose}
All of the following apply:
\begin{itemize}
    \item applying $\disjointslicestopositions$ to $\vslices$ in $\tenv$ checks whether the
    slices in $\vslices$ are disjoint and yields the set of their positions\ProseOrTypeError;
    \item applying $\checkpositionsinwidth$ to $\vwidth$ and $\positions$ to check that
    all of the positions fit with the width given by $\vwidth$ yields $\True$\ProseOrError.
\end{itemize}

\subsubsection{Formally}
\begin{mathpar}
\inferrule{
    \disjointslicestopositions(\tenv, \vslices) \typearrow \positions \OrTypeError\\\\
    \checkpositionsinwidth(\vwidth, \positions) \typearrow \True \OrTypeError
}{
    \checkslicesinwidth(\tenv, \vwidth, \vslices) \typearrow \True
}
\end{mathpar}
\CodeSubsection{\CheckSlicesInWidthBegin}{\CheckSlicesInWidthEnd}{../Typing.ml}

\subsubsection{TypingRule.CheckPositionsInWidth\label{sec:TypingRule.CheckPositionsInWidth}}
\hypertarget{def-checkpositionsinwidth}{}
The function
\[
  \checkpositionsinwidth(\overname{\Z}{\vwidth} \aslsep \overname{\pow{\Z}}{\positions})
  \aslto \{\True\} \cup \overname{\TTypeError}{\TypeErrorConfig}
\]
checks whether the set of positions in $\positions$ fit within the bitvector width given by $\vwidth$,
yielding $\True$. \ProseOtherwiseTypeError

\subsubsection{Prose}
All of the following apply:
\begin{itemize}
    \item define $\minpos$ as the minimal position in $\positions$;
    \item define $\maxpos$ as the maximal position in $\positions$;
    \item checking that $\minpos$ is non-negative and that $\maxpos$ is less than or equal to $\vwidth$
            yields $\True$\ProseTerminateAs{\BitfieldsOutOfRange}.
    \item the result is $\True$.
\end{itemize}

\subsubsection{Formally}
\begin{mathpar}
\inferrule{
    \minpos \eqdef \min(\positions)\\
    \maxpos \eqdef \max(\positions)\\
    \checktrans{0 \leq \minpos \land \maxpos \leq \vwidth}{\BitfieldsOutOfRange} \typearrow \True \OrTypeError
}{
    \checkpositionsinwidth(\vwidth, \positions) \typearrow \True
}
\end{mathpar}

\subsubsection{TypingRule.DisjointSlicesToPositions\label{sec:TypingRule.DisjointSlicesToPositions}}
\hypertarget{def-disjointslicestopositions}{}
The function
\[
  \disjointslicestopositions(\overname{\staticenvs}{\tenv}, \overname{\slice^*}{\vslices})
  \aslto \overname{\powfin{\Z}}{\positions} \cup \overname{\TTypeError}{\TypeErrorConfig}
\]
returns the set of integers defined by the list of slices in $\vslices$ in $\positions$.
In particular, this rule checks that the bitfield slices do not overlap and that they are not defined in reverse
(e.g., \texttt{0:1} rather than \texttt{1:0})
\ProseOtherwiseTypeError

\subsubsection{Prose}
One of the following applies:
\begin{itemize}
  \item All of the following apply (\textsc{empty}):
  \begin{itemize}
    \item $\vslices$ is the empty list;
    \item $\positions$ is the empty set.
  \end{itemize}

  \item All of the following apply (\textsc{non\_empty}):
  \begin{itemize}
    \item $\vslices$ is the list with $\vs$ as its \head\ and $\vslicesone$ as its \tail;
    \item applying $\bitfieldslicetopositions$ to $\vs$ in $\tenv$ yields the optional set of positions \\
          $\positionsoneopt$\ProseOrTypeError;
    \item define $\positionsone$ as $\vsone$ if $\positionsoneopt$ is $\langle\vsone\rangle$ and the empty set, otherwise;
    \item applying $\disjointslicestopositions$ to $\vslicesone$ in $\tenv$ yields the optional set of positions
          $\positionstwoopt$\ProseOrTypeError;
    \item define $\positionstwo$ as $\vsone$ if $\positionstwoopt$ is $\langle\vstwo\rangle$ and the empty set, otherwise;
    \item checking that $\positionsone$ is disjoint from $\positionstwo$ yields $\True$\ProseTerminateAs{\BitfieldSlicesOverlap}
    \item $\positions$ is the union of $\positionsone$ and $\positionstwo$.
  \end{itemize}
\end{itemize}
\subsubsection{Formally}
\begin{mathpar}
\inferrule[empty]{}{
  \disjointslicestopositions(\tenv, \overname{\emptylist}{\vslices}) \typearrow \overname{\emptyset}{\positions}
}
\end{mathpar}

\begin{mathpar}
\inferrule[non\_empty]{
  \bitfieldslicetopositions(\tenv, \vs) \typearrow \positionsoneopt \OrTypeError\\\\
  \positionsone \eqdef \choice{\positionsoneopt = \langle\vsone\rangle}{\vsone}{\emptyset}\\
  \disjointslicestopositions(\tenv, \vslicesone) \typearrow \positionstwoopt \OrTypeError\\\\
  \positionstwo \eqdef \choice{\positionstwoopt = \langle\vstwo\rangle}{\vstwo}{\emptyset}\\
  \checktrans{\positionsone \cap \positionstwo = \emptyset}{\BitfieldSlicesOverlap} \checktransarrow \True \OrTypeError
}{
  \disjointslicestopositions(\tenv, \overname{\vs \concat \vslicesone}{\vslices}) \typearrow \overname{\positionsone \cup \positionstwo}{\positions}
}
\end{mathpar}

\subsubsection{TypingRule.BitfieldSliceToPositions\label{sec:TypingRule.BitfieldSliceToPositions}}
\hypertarget{def-bitfieldslicetopositions}{}
The function
\[
  \bitfieldslicetopositions(\overname{\staticenvs}{\tenv}, \overname{\slice}{\vslice})
  \aslto \overname{\langle\powfin{\Z}\rangle}{\positions} \cup \overname{\TTypeError}{\TypeErrorConfig}
\]
\hypertarget{def-nonstatic}{}
returns the set of integers defined by the bitfield slice $\vslice$ in $\positions$,
if they can be statically evaluated, or $\None$ if they cannot be statically evaluated.
\ProseOtherwiseTypeError

\subsubsection{Prose}
One of the following applies:
\begin{itemize}
  \item All of the following apply (\textsc{single}):
  \begin{itemize}
    \item $\vslice$ is a \singleslice\ defined by the expression $\ve$, that is, $\SliceSingle(\ve)$;
    \item applying $\reducetozopt$ to $\ve$ in $\tenv$ yields the integer literal for $\vx$\ProseTerminateAs{\None};
    \item $\positions$ is the singleton set for $\vx$.
  \end{itemize}

  \item All of the following apply (\textsc{range}):
  \begin{itemize}
    \item $\vslice$ is \rangeslice\ defined by expressions $\veone$ and $\vetwo$, that is, \\
          $\SliceRange(\veone, \vetwo)$;
    \item applying $\reducetozopt$ to $\veone$ in $\tenv$ yields the integer literal for $\vx$\ProseTerminateAs{\None};
    \item applying $\reducetozopt$ to $\vetwo$ in $\tenv$ yields the integer literal for $\vy$\ProseTerminateAs{\None};
    \item checking that $\vx$ is less than or equal to $\vy$ yields $\True$\ProseTerminateAs{\BitfieldSliceReversed};
    \item $\positions$ is the set of integers between $\vx$ and $\vy$, inclusive.
  \end{itemize}

  \item All of the following apply (\textsc{length}):
  \begin{itemize}
    \item $\vslice$ is \lengthslice\ defined by expressions $\veone$ and $\vetwo$, that is, \\
          $\SliceLength(\veone, \vetwo)$;
    \item applying $\reducetozopt$ to $\veone$ in $\tenv$ yields the integer literal for $\vx$\ProseTerminateAs{\None};
    \item applying $\reducetozopt$ to $\vetwo$ in $\tenv$ yields the integer literal for $\vy$\ProseTerminateAs{\None};
    \item checking that $\vy > 0$ holds (which implies that $\vx \leq \vx+\vy-1$ holds) yields $\True$\ProseTerminateAs{\BitfieldSliceReversed};
    \item $\positions$ is the set of integers between $\vx$ and $\vx+\vy-1$, inclusive.
  \end{itemize}

  \item All of the following apply (\textsc{scaled}):
  \begin{itemize}
    \item $\vslice$ is \scaledslice\ defined by expressions $\veone$ and $\vetwo$, that is, \\
          $\SliceStar(\veone, \vetwo)$;
    \item applying $\reducetozopt$ to $\veone$ in $\tenv$ yields the integer literal for $\vx$\ProseTerminateAs{\None};
    \item applying $\reducetozopt$ to $\vetwo$ in $\tenv$ yields the integer literal for $\vy$\ProseTerminateAs{\None};
    \item checking that $\vx > 0$ holds (which implies that $\vx \times \vy \leq \vx \times (\vy + 1) - 1$ holds) yields $\True$\ProseTerminateAs{\BitfieldSliceReversed};
    \item $\positions$ is the set of integers between $\vx \times \vy$ and $\vx \times (\vy + 1) - 1$, inclusive.
  \end{itemize}
\end{itemize}
\subsubsection{Formally}
\begin{mathpar}
\inferrule[single\_static]{
  \reducetozopt(\tenv, \ve) \typearrow \langle\vx\rangle \terminateas\None
}{
  \bitfieldslicetopositions(\tenv, \overname{\SliceSingle(\ve)}{\vslice}) \typearrow \overname{\langle\{\vx\}\rangle}{\positions}
}
\end{mathpar}

\begin{mathpar}
\inferrule[range]{
  \reducetozopt(\tenv, \veone) \typearrow \langle\vx\rangle \terminateas\None\\\\
  \reducetozopt(\tenv, \vetwo) \typearrow \langle\vy\rangle \terminateas\None\\\\
  \checktrans{\vy \leq \vx}{\BitfieldSliceReversed} \checktransarrow \True \OrTypeError\\\\
}{
  \bitfieldslicetopositions(\tenv, \overname{\SliceRange(\veone, \vetwo)}{\vslice}) \typearrow
  \overname{\langle\{n \;|\; \vx \leq n \leq \vy\}\rangle}{\positions}
}
\end{mathpar}

\begin{mathpar}
\inferrule[length]{
  \reducetozopt(\tenv, \veone) \typearrow \langle\vx\rangle \terminateas\None\\\\
  \reducetozopt(\tenv, \vetwo) \typearrow \langle\vy\rangle \terminateas\None\\\\
  \checktrans{\vy > 0}{\BitfieldSliceReversed} \checktransarrow \True \OrTypeError\\\\
}{
  \bitfieldslicetopositions(\tenv, \overname{\SliceRange(\veone, \vetwo)}{\vslice}) \typearrow
  \overname{\langle\{n \;|\; \vx \leq n \leq \vx+\vy-1\}\rangle}{\positions}
}
\end{mathpar}
\begin{mathpar}

\inferrule[scaled]{
  \reducetozopt(\tenv, \veone) \typearrow \langle\vx\rangle \terminateas\None\\\\
  \reducetozopt(\tenv, \vetwo) \typearrow \langle\vy\rangle \terminateas\None\\\\
  \checktrans{\vx > 0}{\BitfieldSliceReversed} \checktransarrow \True \OrTypeError\\\\
}{
  \bitfieldslicetopositions(\tenv, \overname{\SliceStar(\veone, \vetwo)}{\vslice}) \typearrow
  \overname{\langle\{n \;|\; \vx \times \vy \leq n \leq \vx \times (\vy + 1) - 1\}\rangle}{\positions}
}
\end{mathpar}
