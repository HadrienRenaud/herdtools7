\chapter{Literals\label{chap:Literals}}
ASL allows specifying literal values for the following types:
integers, Booleans, real numbers, bitvectors, and strings.

Enumeration labels are also literal values.
However, they are syntactically indistinguishable from identifiers,
so they cannot be input directly in concrete syntax.
Rather, they are parsed as identifiers, and during typechecking
converted to enumeration label literal values (instance of $\llabel$).

In the remainder of this reference, we often refer to literal values simply as literals.

\section{Syntax}
\begin{flalign*}
\Nvalue \derives         \ & \Tintlit &\\
                        |\ & \Tboollit &\\
                        |\ & \Treallit &\\
                        |\ & \Tbitvectorlit &\\
                        |\ & \Tstringlit &
\end{flalign*}

\section{Abstract Syntax}
\begin{flalign*}
\literal \derives\ & \lint(\overtext{n}{$\Z$}) & \\
    |\ & \lbool(\overtext{b}{$\{\True, \False\}$})
    & \\
    |\ & \lreal(\overtext{q}{$\Q$})
    & \\
    |\ & \lbitvector(\overtext{B}{$B \in \{0, 1\}^*$})
    & \\
    |\ & \lstring(\overtext{S}{$S \in \Strings$})
    &\\
    |\ & \llabel(\overtext{l}{enumeration label})
\end{flalign*}

\subsection{ASTRule.Value \label{sec:ASTRule.Value}}
\hypertarget{build-value}{}
The function
\[
  \buildvalue(\overname{\parsenode{\Nvalue}}{\vparsednode}) \;\aslto\; \overname{\literal}{\vastnode}
\]
transforms a parse node $\vparsednode$ for $\Nvalue$ into an AST node $\vastnode$ for $\literal$.

\begin{mathpar}
\inferrule[integer]{}{
  \buildvalue(\Nvalue(\Tintlit(\vi))) \astarrow
  \overname{\lint(\vi)}{\vastnode}
}
\end{mathpar}

\begin{mathpar}
\inferrule[boolean]{}{
  \buildvalue(\Nvalue(\Tboollit(\vb))) \astarrow
  \overname{\lbool(\vb)}{\vastnode}
}
\end{mathpar}

\begin{mathpar}
\inferrule[real]{}{
  \buildvalue(\Nvalue(\Treallit(\vr))) \astarrow
  \overname{\lreal(\vr)}{\vastnode}
}
\end{mathpar}

\begin{mathpar}
\inferrule[bitvector]{}{
  \buildvalue(\Nvalue(\Tbitvectorlit(\vb))) \astarrow
  \overname{\lbitvector(\vb)}{\vastnode}
}
\end{mathpar}

\begin{mathpar}
\inferrule[string]{}{
  \buildvalue(\Nvalue(\Tstringlit(\vs))) \astarrow
  \overname{\lstring(\vs)}{\vastnode}
}
\end{mathpar}

\section{Typing}
\ExampleDef{Well-typed literals}
\listingref{literals1} shows literals and their corresponding types in comments:
\ASLListing{Literals and their corresponding types}{literals1}{\typingtests/TypingRule.Lit.asl}

\TypingRuleDef{Lit}
\hypertarget{def-annotateliteral}{}
The function
\[
  \annotateliteral{\overname{\staticenvs}{\tenv}, \overname{\literal}{\vl}} \aslto \overname{\ty}{\vt}
\]
annotates a literal $\vl$ in the static environment $\tenv$, resulting in a type $\vt$.

\ProseParagraph
\OneApplies
\begin{itemize}
  \item \AllApplyCase{int}
  \begin{itemize}
    \item $\vl$ is an integer literal $\vn$;
    \item \Proseeqdef{$\vt$}{the well-constrained integer type, constraining
          its set to the single value $\vn$}.
  \end{itemize}

  \item \AllApplyCase{bool}
  \begin{itemize}
    \item $\vl$ is a Boolean literal;
    \item \Proseeqdef{$\vt$}{the \booleantypeterm}.
  \end{itemize}

  \item \AllApplyCase{real}
  \begin{itemize}
    \item $\vl$ is real literal;
    \item \Proseeqdef{$\vt$}{the \realtypeterm}.
  \end{itemize}

  \item \AllApplyCase{string}
  \begin{itemize}
    \item $\vl$ is a string literal;
    \item \Proseeqdef{$\vt$}{the \stringtypeterm}.
  \end{itemize}

  \item \AllApplyCase{string}
  \begin{itemize}
    \item $\vl$ is a bitvector literal of length $\vn$ and $\vt$;
    \item \Proseeqdef{$\vt$}{the bitvector type of fixed width $\vn$}.
  \end{itemize}

  \item \AllApplyCase{label}
  \begin{itemize}
    \item $\vl$ is an enumeration label for $\vlabel$;
    \item \Proseeqdef{$\vt$}{the type to which $\vlabel$ is bound in the
        $\declaredtypes$ map of the global environment $\tenv$}.
  \end{itemize}
\end{itemize}

\FormallyParagraph
\begin{mathpar}
\inferrule[int]{}{\annotateliteral{\Ignore, \lint(n)}\typearrow \TInt(\langle[\ConstraintExact(\ELInt{n})]\rangle)}
\end{mathpar}

\begin{mathpar}
\inferrule[bool]{}{\annotateliteral{\Ignore, \lbool(\Ignore)}\typearrow \TBool}
\end{mathpar}

\begin{mathpar}
\inferrule[real]{}{\annotateliteral{\Ignore, \lreal(\Ignore)}\typearrow \TReal}
\end{mathpar}

\begin{mathpar}
\inferrule[string]{}{\annotateliteral{\Ignore, \lstring(\Ignore)}\typearrow \TString}
\end{mathpar}

\begin{mathpar}
\inferrule[bitvector]{
  n \eqdef \listlen{\bits}
}{
  \annotateliteral{\Ignore, \lbitvector(\bits)}\typearrow \TBits(\ELInt{n}, \emptylist)
}
\end{mathpar}

\begin{mathpar}
\inferrule[label]{
  G^\tenv.\declaredtypes(\vlabel) = (\vt, \Ignore)
}{
  \annotateliteral{\tenv, \llabel(\vlabel)}\typearrow \vt
}
\end{mathpar}
\CodeSubsection{\LitBegin}{\LitEnd}{../Typing.ml}

\section{Semantics}
A literal $\vl$ can be converted to the \nativevalue\ $\nvliteral{\vl}$.

\subsubsection{Example}
The literal $\lint(5)$ can be used as a \nativevalue\ $\nvliteral{\lint(5)}$,
which we will usually abbreviate as $\nvint(5)$.
