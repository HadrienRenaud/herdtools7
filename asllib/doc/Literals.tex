\chapter{Literals}
ASL allows specifying literal values for the following types:
integers, Booleans, real numbers, bitvectors, and strings.

Enumeration labels are also considered literal values but are technically identifiers.
As such, they are not built by the Grammar, but by the Type-checker.

\section{Syntax}

\begin{flalign*}
\Nvalue \derives         \ & \Tintlit &\\
                        |\ & \Tboollit &\\
                        |\ & \Treallit &\\
                        |\ & \Tbitvectorlit &\\
                        |\ & \Tstringlit &
\end{flalign*}

\section{Abstract Syntax}
\begin{flalign*}
\literal \derives\ & \lint(\overname{n}{\Z}) & \\
    |\ & \lbool(\overname{b}{\{\True, \False\}})
    & \\
    |\ & \lreal(\overname{q}{\Q})
    & \\
    |\ & \lbitvector(\overname{B}{B \in \{0, 1\}^*})
    & \\
    |\ & \lstring(\overname{S}{S \in \{C \;|\; \texttt{"$C$"}\ \in\ \Strings\}}) &
\end{flalign*}

\subsection{ASTRule.Value \label{sec:ASTRule.Value}}
\hypertarget{build-value}{}
The function
\[
  \buildvalue(\overname{\parsenode{\Nvalue}}{\vparsednode}) \;\aslto\; \overname{\literal}{\vastnode}
\]
transforms a parse node $\vparsednode$ for $\Nvalue$ into an AST node $\vastnode$ for $\literal$.

\begin{mathpar}
\inferrule[integer]{}{
  \buildvalue(\Nvalue(\Tintlit(\vi))) \astarrow
  \overname{\lint(\vi)}{\vastnode}
}
\end{mathpar}

\begin{mathpar}
\inferrule[boolean]{}{
  \buildvalue(\Nvalue(\Tboollit(\vb))) \astarrow
  \overname{\lbool(\vb)}{\vastnode}
}
\end{mathpar}

\begin{mathpar}
\inferrule[real]{}{
  \buildvalue(\Nvalue(\Treallit(\vr))) \astarrow
  \overname{\lreal(\vr)}{\vastnode}
}
\end{mathpar}

\begin{mathpar}
\inferrule[bitvector]{}{
  \buildvalue(\Nvalue(\Tbitvectorlit(\vb))) \astarrow
  \overname{\lbitvector(\vb)}{\vastnode}
}
\end{mathpar}

\begin{mathpar}
\inferrule[string]{}{
  \buildvalue(\Nvalue(\Tstringlit(\vs))) \astarrow
  \overname{\lstring(\vs)}{\vastnode}
}
\end{mathpar}

\section{Typing}
\subsubsection{TypingRule.Lit \label{sec:TypingRule.Lit}}
\hypertarget{def-annotateliteral}{}
The function
\[
  \annotateliteral{\overname{\staticenvs}{\tenv}, \overname{\literal}{\vl}} \aslto \overname{\ty}{\vt}
\]
annotates a literal $\vl$ in the static environment $\tenv$, resulting in a type $\vt$.

\subsubsection{Prose}
The result of annotating a literal $\vl$ in a static environment $\tenv$ is $\vt$ and one of the following applies:
\begin{itemize}
\item (\textsc{int}): $\vl$ is an integer literal $\vn$ and $\vt$ is the well-constrained integer type, constraining
its set to the single value $\vn$;
\item (\textsc{bool}): $\vl$ is a Boolean literal and $\vt$ is the Boolean type;
\item (\textsc{real}): $\vl$ is a real literal and $\vt$ is the real type;
\item (\textsc{string}): $\vl$ is a string literal and $\vt$ is the string type;
\item (\textsc{bitvector}): $\vl$ is a bitvector literal of length $\vn$ and $\vt$ is the bitvector type of fixed width $\vn$.
\item (\textsc{label}): $\vl$ is an enumeration label for $\vlabel$ and $\vlabel$ is bound to the type $\vt$ in the
      $\declaredtypes$ map of the global environment $\tenv$.
\end{itemize}

\CodeSubsection{\LitBegin}{\LitEnd}{../Typing.ml}

\subsubsection{Formally}
\begin{mathpar}
\inferrule[int]{}{\annotateliteral{\Ignore, \lint(n)}\typearrow \TInt(\langle[\ConstraintExact(\ELInt{n})]\rangle)}
\end{mathpar}

\begin{mathpar}
\inferrule[bool]{}{\annotateliteral{\Ignore, \lbool(\Ignore)}\typearrow \TBool}
\end{mathpar}

\begin{mathpar}
\inferrule[real]{}{\annotateliteral{\Ignore, \lreal(\Ignore)}\typearrow \TReal}
\end{mathpar}

\begin{mathpar}
\inferrule[string]{}{\annotateliteral{\Ignore, \lstring(\Ignore)}\typearrow \TString}
\end{mathpar}

\begin{mathpar}
\inferrule[bitvector]{
  n \eqdef \listlen{\bits}
}{
  \annotateliteral{\Ignore, \lbitvector(\bits)}\typearrow \TBits(\ELInt{n}, \emptylist)
}
\end{mathpar}

\begin{mathpar}
\inferrule[label]{
  G^\tenv.\declaredtypes(\vlabel) = (\vt, \Ignore)
}{
  \annotateliteral{\tenv, \llabel(\vlabel)}\typearrow \vt
}
\end{mathpar}

\subsection{Example}
\listingref{literals1} shows literals and their corresponding types in comments:
\ASLListing{Literals and their corresponding types}{literals1}{\typingtests/TypingRule.Lit.asl}

\section{Semantics}
A literal $\vl$ can be converted to the \nativevalue\ $\nvliteral{\vl}$.

\subsection{Printing}%
\hypertarget{def-outputtoconsole}{}

The following table describes how literals can be printed to a command line.
%
Please note that surrounding quotations mark for $\lstring(S)$ are not included
in $S$, so they will not be printed.

\begin{tabular}{rl}
  \textbf{literal} & \textbf{prints} \\
  \hline
  $\lint(n)$ & $n$ in decimal format, without any leading zeros, \\
             & preceded by a ``\texttt{-}'' sign if $n$ is negative. \\
  $\lbool(\True)$ & ``\texttt{TRUE}'' \\
  $\lbool(\False)$ & ``\texttt{FALSE}'' \\
  $\lreal(q)$ & $q$ as an irreducible fraction of positive integers, \\
              & preceded by a ``\texttt{-}'' sign when $q$ is negative, \\
              & with the denominator omitted if it is equal to 1. \\
  $\lbitvector(b)$ & $b$ in hexadecimal, preceded by ``\texttt{0x}'', with enough leading zeros \\
                   & to make the number of hexadecimal digits printed equal the width \\
                   & of $b$ divided by 4, and rounded up to the following integer. \\
  $\lstring(S)$ & $S$. \\
  $\llabel(s)$ & $s$. \\
\end{tabular}
