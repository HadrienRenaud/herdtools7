\chapter{Subprogram Calls\label{chap:SubprogramCalls}}

%%%%%%%%%%%%%%%%%%%%%%%%%%%%%%%%%%%%%%%%%%%%%%%%%%%%%%%%%%%%%%%%%%%%%%%%%%%%%%%%
\section{Syntax}
%%%%%%%%%%%%%%%%%%%%%%%%%%%%%%%%%%%%%%%%%%%%%%%%%%%%%%%%%%%%%%%%%%%%%%%%%%%%%%%%
\begin{flalign*}
\Nexpr \derives\  & \Tidentifier \parsesep \Plist{\Nexpr} &\\
\Nstmt \derivesinline\ & \Tidentifier \parsesep \Plist{\Nexpr} \parsesep \Tsemicolon &
\end{flalign*}

%%%%%%%%%%%%%%%%%%%%%%%%%%%%%%%%%%%%%%%%%%%%%%%%%%%%%%%%%%%%%%%%%%%%%%%%%%%%%%%%
\section{Abstract Syntax}
%%%%%%%%%%%%%%%%%%%%%%%%%%%%%%%%%%%%%%%%%%%%%%%%%%%%%%%%%%%%%%%%%%%%%%%%%%%%%%%%
\subsection{Untyped AST}
\begin{flalign*}
\expr \derives\ & \ECall(\overtext{\identifier}{subprogram name}, \overtext{\expr^{*}}{actual arguments}) &\\
\stmt \derives\ & \SCall(\overtext{\identifier}{subprogram name}, \overtext{\expr^{*}}{actual arguments}) &
\end{flalign*}

\subsection{Typed AST}
The AST node for call expressions includes an extra component that explicitly associates expressions
with parameters:
\begin{flalign*}
\expr \derives\ & \ECall(\overtext{\identifier}{subprogram name}, \overtext{\expr^{*}}{actual arguments}, \overtext{(\identifier, \expr)^{*}}{parameters with initializers}) &
\end{flalign*}

Similar to expressions, the AST node for call statements includes an extra component that explicitly associates expressions
with parameters:
\begin{flalign*}
\stmt \derives\ & \SCall(\overtext{\identifier}{subprogram name}, \overtext{\expr^{*}}{actual arguments}, \overtext{(\identifier, \expr)^{*}}{parameters with initializers}) &
\end{flalign*}

%%%%%%%%%%%%%%%%%%%%%%%%%%%%%%%%%%%%%%%%%%%%%%%%%%%%%%%%%%%%%%%%%%%%%%%%%%%%%%%%
\section{Typing}
%%%%%%%%%%%%%%%%%%%%%%%%%%%%%%%%%%%%%%%%%%%%%%%%%%%%%%%%%%%%%%%%%%%%%%%%%%%%%%%%
\hypertarget{def-annotatecall}
The function
\[
  \begin{array}{rl}
  \annotatecall( &
    \overname{\staticenvs}{\tenv} \aslsep
    \overname{\identifier}{\name} \aslsep
    \overname{\expr^*}{\vargs} \aslsep
    \overname{\subprogramtype}{\calltype}
   ) \aslto \\ &
  (\overname{\identifier}{\nameone} \aslsep
  \overname{\expr^*}{\vargsone} \aslsep
  \overname{(\identifier\times\expr)^*}{\eqs} \aslsep
  \overname{\langle \ty \rangle}{\rettyopt})
\end{array}
\]
annotates the call to subprogram $\name$ with arguments $\vargs$
and call type $\calltype$, resulting in the following:
\begin{itemize}
  \item $\nameone$ --- a string, which uniquely identifies $\name$ among the set of overloading subprograms declared with $\name$;
  \item $\vargsone$ --- the annotated argument expressions;
  \item $\eqs$ --- the expressions providing values to the parameters;
  \item $\rettyopt$ --- the \optional\ annotated return type.
\end{itemize}
\ProseOtherwiseTypeError

The function is defined by the rule TypingRule.AnnotateCall (see \secref{TypingRule.AnnotateCall}).

We also define helper functions via respective rules:
\begin{itemize}
  \item TypingRule.AnnotateCallArgTyped (see \secref{TypingRule.AnnotateCallArgTyped})
  \item TypingRule.CheckCalleeParams (see \secref{TypingRule.CheckCalleeParams})
  \item TypingRule.RenameTyEqs (see \secref{TypingRule.RenameTyEqs})
  \item TypingRule.SubstExprNormalize (see \secref{TypingRule.SubstExprNormalize})
  \item TypingRule.SubstExpr (see \secref{TypingRule.SubstExpr})
  \item TypingRule.SubstConstraint (see \secref{TypingRule.SubstConstraint})
  \item TypingRule.CheckArgsTypeSat (see \secref{TypingRule.CheckArgsTypeSat})
  \item TypingRule.AnnotateParameterDefining (see \secref{TypingRule.AnnotateParameterDefining})
  \item TypingRule.AnnotateRetTy (\secref{TypingRule.AnnotateRetTy})
  \item TypingRule.SubprogramForName (see \secref{TypingRule.SubprogramForName})
  \item TypingRule.DeduceEqs (see \secref{TypingRule.DeduceEqs})
  \item TypingRule.FilterCallCandidates (see \secref{TypingRule.FilterCallCandidates})
  \item TypingRule.HasArgClash (see \secref{TypingRule.HasArgClash})
  \item TypingRule.ExpressionList (see \secref{TypingRule.ExpressionList})
\end{itemize}

\subsubsection{TypingRule.AnnotateCall \label{sec:TypingRule.AnnotateCall}}
\subsubsection{Prose}
All of the following apply:
\begin{itemize}
  \item applying $\annotateexprlist$ to annotate the expression list $\vargs$ in $\tenv$ yields \\
        $\callerargtyped$\ProseOrTypeError;
  \item applying $\annotatecallargtyped$ to $\name$, $\callerargtyped$, $\calltype$ in $\tenv$ yields
        $(\nameone, \vargsone, \eqs, \retty)$\ProseOrTypeError.
\end{itemize}

\CodeSubsection{\AnnotateCallBegin}{\AnnotateCallEnd}{../Typing.ml}

\subsubsection{Formally}
\begin{mathpar}
\inferrule{
  \annotateexprlist(\tenv, \vargs) \typearrow \callerargtyped \OrTypeError\\\\
  {
    \begin{array}{r}
  \annotatecallargtyped(\tenv, \name, \callerargtyped, \calltype) \typearrow \\
   (\nameone, \vargsone, \eqs, \retty) \OrTypeError
    \end{array}
  }
}{
  \annotatecall(\tenv, \name, \vargs, \calltype) \typearrow
  (\nameone, \vargsone, \eqs, \retty)
}
\end{mathpar}\lrmcomment{
  This is related to \identi{VFDP}, \identd{TRFW}, \identr{KMDB},
  \identi{YMHX}, \identr{CCVD}, \identr{QYBH}, \identr{PFWQ}, \identr{ZLWD},
  \identi{FLKF}, \identd{PMBL}, \identr{MWBN}, \identr{TZSP}, \identr{SBWR},
  \identi{CMLP}, \identr{BQJG}, \identr{RTCF}.
}

\subsubsection{TypingRule.AnnotateCallArgTyped \label{sec:TypingRule.AnnotateCallArgTyped}}
\hypertarget{def-annotatecallargtyped}{}
The function
\[
  \begin{array}{r}
  \annotatecallargtyped(
    \overname{\staticenvs}{\tenv} \aslsep
    \overname{\identifier}{\name} \aslsep
    \overname{(\ty \times\expr)^*}{\callerargstyped} \aslsep
    \overname{\subprogramtype}{\calltype}
   ) \aslto \\
  (\overname{\identifier}{\nameone} \aslsep
  \overname{\expr^*}{\vargsone} \aslsep
  \overname{(\identifier\times\expr)^*}{\eqs} \aslsep
  \overname{\langle \ty \rangle}{\rettyopt})
  \cup \overname{\TTypeError}{\TypeErrorConfig}
\end{array}
\]
is similar to $\annotatecall$, except that the argument expressions are replaced by
the annotated expressions. That is, pairs consisting of a type and an expression.
\ProseOtherwiseTypeError

\subsubsection{Prose}
All of the following apply:
\begin{itemize}
  \item applying $\unziplist$ to $\callerargtyped$ yields the corresponding lists of types
        and expressions $\callerargtypes$ and $\vargsone$;
  \item applying $\subprogramforname$ to match $\name$ and $\callerargtypes$ in $\tenv$
        yields $(\eqsone, \nameone, \callee)$\ProseOrTypeError;
  \item checking that $\subprogramtype$ of $\callee$ equals $\calltype$ yields $\True$\ProseOrTypeError;
  \item checking that the lengths of $\callee.\funcargs$ and $\vargsone$ are the same yields $\True$\ProseOrTypeError;
  \item applying $\annotateparameterdefining$ to $\callee.\funcargs$,
        $\callerargstyped$, and \\
        $\calleeparams$ in $\tenv$ to annotate the implicit parameters
        yields $\eqsthreep$\ProseOrTypeError;
  \item define $\eqsthree$ is the concatenation of $\eqsthreep$ and $\eqsone$;
  \item applying $\checkargstypesat$ to $\calleeargtypes$
        to check that the actual arguments have correct types with respect to $\callerargtypes$
        in $\tenv$ yields $\True$\ProseOrTypeError;
  \item applying $\checkcalleeparams$ to $\calleeparams$ to check they have correct types
        with respect to $\eqsthree$ in $\tenv$ yields $\True$\ProseOrTypeError;
  \item applying $\annotateretty$ to $\calltype$ and $\callee.\funcreturntype$
        to check that the two call types match and to substitute actual parameter arguments in the formal return type
        yields $\rettyopt$ \ProseOrTypeError;
  \item define $\eqs$ as $\eqsthree$.
\end{itemize}

\CodeSubsection{\AnnotateCallArgTypedBegin}{\AnnotateCallArgTypedEnd}{../Typing.ml}

\subsubsection{Formally}
\begin{mathpar}
\inferrule{
  \unziplist(\callerargstyped) = (\callerargtypes, \vargsone)\\
  {
    \begin{array}{r}
    \subprogramforname(\tenv, \name, \callerargtypes) \typearrow \\
  (\eqsone, \nameone, \callee) \OrTypeError
    \end{array}
  }\\
  \checktrans{\callee.\subprogramtype = \calltype}{\MismatchedReturnValue} \checktransarrow \True\OrTypeError\\\\
  \equallength(\callee.\funcargs, \vargsone) \typearrow \aritymatch\\
  \checktrans{\aritymatch}{\CallBadArity} \checktransarrow \True \OrTypeError\\\\
  {
  \annotateparameterdefining\left(
    \begin{array}{l}
      \tenv,\\
      \callee.\funcargs, \\
      \callerargstyped, \\
      \callee.\funcparameters
    \end{array}
    \right)
    \typearrow \eqsthreep \OrTypeError
  }\\
  \eqsthree \eqdef \eqsthreep \concat \eqsone\\
  \checkargstypesat(\tenv, \callee.\funcargs, \callerargtypes, \eqsthree) \typearrow \True \OrTypeError\\\\
  \checkcalleeparams(\tenv, \callee.\funcparameters, \eqsthree) \typearrow \True \OrTypeError\\\\
  \annotateretty(\tenv, \calltype, \callee.\funcreturntype) \typearrow \rettyopt \OrTypeError
}{
  {
    \begin{array}{r}
  \annotatecallargtyped(\tenv, \name, \callerargstyped, \calltype) \typearrow \\
  (\nameone, \vargsone, \overname{\eqsthree}{\eqs}, \rettyopt)
    \end{array}
  }
}
\end{mathpar}

\subsubsection{TypingRule.CheckCalleeParams \label{sec:TypingRule.CheckCalleeParams}}
\hypertarget{def-checkcalleeparams}{}
The function
\[
\checkcalleeparams(
  \overname{\staticenvs}{\tenv} \aslsep
  \overname{(\identifier\times\langle\ty\rangle)^*}{\calleeparams}
  \overname{(\identifier\times\expr)^*}{\eqsthree}
  ) \aslto \\
  \{\True\}\ \cup \overname{\TTypeError}{\TypeErrorConfig}
\]
checks that the parameters in $\calleeparams$ are correct with respect
to the parameter expressions $\eqsthree$.
\ProseOtherwiseTypeError

\subsubsection{Prose}
One of the following applies:
\begin{itemize}
  \item All of the following apply (\textsc{empty}):
  \begin{itemize}
    \item $\calleeparams$ is an empty list;
    \item the result is $\True$.
  \end{itemize}

  \item All of the following apply:
  \begin{itemize}
    \item $\calleeparams$ is a non-empty list with \head\ $\calleeparam$ and \tail\ \\
          $\calleeparamsone$;
    \item One of the following applies:
    \begin{itemize}
      \item All of the following apply (\textsc{no\_type}):
      \begin{itemize}
        \item $\calleeparam$ does not have a type annotation, that is, $(\Ignore, \None)$.
      \end{itemize}

      \item All of the following apply (\textsc{parameterized}):
      \begin{itemize}
        \item $\calleeparam$ is a parameter $\vs$ with a type annotation of a
              \parameterizedintegertype\ for the same parameter, that is, \\
              $(\vs, \langle\TInt(\parameterized(\vs))\rangle)$.
      \end{itemize}

      \item All of the following apply (\textsc{other}):
      \begin{itemize}
        \item $\calleeparam$ is a parameter $\vs$ whose type annotation is \\
              $\calleeparamt$, that is, $(\vs, \langle\calleeparamt\rangle)$;
        \item $\calleeparamt$ is not the \parameterizedintegertype\ for the same parameter;
        \item substituting the parameter expressions from $\eqsthree$ in $\calleeparamt$
              yields $\calleeparamtrenamed$\ProseOrTypeError;
        \item applying $\assocopt$ to $\eqsthree$ and $\vs$ yields the expression $\callerparame$
              (that is, the parameter $\vs$ is associated with the expression \\
              $\callerparame$);
        \item annotating the expression $\callerparame$ in $\tenv$ yields \\
              $(\callerparamt, \Ignore)$\ProseOrTypeError;
        \item checking that $\callerparamt$ \typesatisfies\ $\calleeparamtrenamed$ in $\tenv$ yields
              $\True$\ProseOrTypeError;
      \end{itemize}
    \end{itemize}
    \item applying $\checkcalleeparams$ to $\calleeparamsone$ and $\eqsthree$ in $\tenv$
          yields \\ $\True$\ProseOrTypeError.
  \end{itemize}
\end{itemize}

\subsubsection{Formally}
\begin{mathpar}
\inferrule[empty]{
}{
  \checkcalleeparams(\tenv, \overname{\emptylist}{\calleeparams}, \eqsthree) \typearrow \True
}
\and
\inferrule[no\_type]{
  \calleeparams = [(\Ignore, \None)] \concat \calleeparamsone\\
  \checkcalleeparams(\tenv, \calleeparamsone, \eqsthree) \typearrow \True \OrTypeError
}{
  \checkcalleeparams(\tenv, \calleeparams, \eqsthree) \typearrow \True
}
\and
\inferrule[parameterized]{
  \calleeparams = [(\vs, \langle\TInt(\parameterized(\vs))\rangle)] \concat \calleeparamsone\\
  \checkcalleeparams(\tenv, \calleeparamsone, \eqsthree) \typearrow \True \OrTypeError
}{
  \checkcalleeparams(\tenv, \calleeparams, \eqsthree) \typearrow \True
}
\and
\inferrule[other]{
  \calleeparams = [(\vs, \langle\calleeparamt\rangle)] \concat \calleeparamsone\\
  \calleeparamt \neq \TInt(\parameterized(\vs))\\
  \renametyeqs(\tenv, \eqsthree, \calleeparamt) \typearrow \calleeparamtrenamed \OrTypeError\\\\
  \assocopt(\eqsthree, \vs) \typearrow \langle \callerparame \rangle\\
  \annotateexpr{\tenv, \callerparame} \typearrow \callerparamt \OrTypeError\\\\
  \checktypesat(\tenv, \callerparamt, \calleeparamtrenamed) \typearrow \True \OrTypeError\\\\
  \checkcalleeparams(\tenv, \calleeparamsone, \eqsthree) \typearrow \True \OrTypeError
}{
  \checkcalleeparams(\tenv, \calleeparams, \eqsthree) \typearrow \True
}
\end{mathpar}

\subsubsection{TypingRule.RenameTyEqs \label{sec:TypingRule.RenameTyEqs}}
\hypertarget{def-renametyeqs}{}
The function
\[
\renametyeqs(
  \overname{\staticenvs}{\tenv} \aslsep
  \overname{(\identifier\times\expr)^*}{\eqs} \aslsep
  \overname{\ty}{\tty}
) \aslto
\overname{\ty}{\newty} \cup \overname{\TTypeError}{\TypeErrorConfig}
\]
transforms the type $\tty$ in the static environment $\tenv$,
by substituting parameter names with their corresponding expressions in
$\eqs$, yielding the type $\newty$.
\ProseOtherwiseTypeError

\subsubsection{Prose}
One of the following applies:
\begin{itemize}
  \item All of the following apply (\textsc{t\_bits}):
  \begin{itemize}
    \item $\tty$ is a bitvector type with width expression $\ve$ and fields $\fields$, that is, $\TBits(\ve, \fields)$;
    \item applying $\substexprnormalize$ to $\eqs$ and $\ve$ in $\tenv$ yields the expression $\newe$;
    \item define $\newty$ as a bitvector type with with expression $\newe$ and fields $\fields$.
  \end{itemize}

  \item All of the following apply (\textsc{t\_int\_wellconstrained}):
  \begin{itemize}
    \item $\tty$ is a well-constrained integer type with constraints $\constraints$;
    \item applying $\substconstraint$ to each constraint $\constraints[\vi]$, for $\vi$ in \\
          $\listrange(\constraints)$, yields the constraint $\newc_\vi$;
    \item define $\newconstraints$ as the list of constraints $\newc_\vi$, for $\vi$ in \\
          $\listrange(\constraints)$;
    \item define $\newty$ as the well-constrained integer type with constraints \\
          $\newconstraints$.
  \end{itemize}

  \item All of the following apply (\textsc{t\_int\_parameterized}):
  \begin{itemize}
    \item $\tty$ is a \parameterizedintegertype\ for the parameter $\name$;
    \item applying $\substexprnormalize$ to $\eqs$ and the expression $\EVar(\name)$ yields $\ve$;
    \item define $\newty$ as the well-constrained integer type with the single constraint for $\ve$, that is,
          $\TInt(\wellconstrained(\ConstraintExact(\ve)))$.
  \end{itemize}

  \item All of the following apply (\textsc{t\_tuple}):
  \begin{itemize}
    \item $\tty$ is the tuple type over the list of tuples $\tys$, that is, $\TTuple(\tys)$;
    \item applying $\renametyeqs$ to $\eqs$ and the type $\tys[\vi]$, for each $\vi$ in \\
          $\listrange(\tys)$, yields the type $\newty_\vi$;
    \item define $\newtys$ as the list of types $\newty_\vi$, for each $\vi$ in $\listrange(\tys)$;
    \item define $\newty$ as the tuple type over $\newtys$, that is, $\TTuple(\newtys)$.
  \end{itemize}

  \item All of the following apply (\textsc{other}):
  \begin{itemize}
    \item $\tty$ is not one of the types in the previous cases,
          that is, $\tty$ is not a bitvector type, nor an integer type, nor a tuple type;
    \item $\newty$ is $\tty$.
  \end{itemize}
\end{itemize}

\subsubsection{Formally}
\begin{mathpar}
\inferrule[t\_bits]{
  \substexprnormalize(\tenv, \eqs, \ve) \typearrow \newe
}{
  \renametyeqs(\tenv, \eqs, \overname{\TBits(\ve, \fields)}{\tty}) \typearrow \overname{\TBits(\newe, \fields)}{\newty}
}
\and
\inferrule[t\_int\_wellconstrained]{
  \vi\in\listrange(\constraints): \substconstraint(\tenv, \constraints[\vi]) \typearrow \newc_\vi\\
  \newconstraints \eqdef [\vi\in\listrange(\constraints): \newc_\vi]\\
  \newty \eqdef \TInt(\wellconstrained(\newconstraints))
}{
  \renametyeqs(\tenv, \eqs, \overname{\TInt(\wellconstrained(\constraints))}{\tty}) \typearrow
  \newty
}
\and
\inferrule[t\_int\_parameterized]{
  \substexprnormalize(\eqs, \EVar(\name)) \typearrow \ve\\
  \newty \eqdef \TInt(\wellconstrained(\ConstraintExact(\ve)))
}{
  \renametyeqs(\tenv, \eqs, \overname{\TInt(\parameterized(\name))}{\tty}) \typearrow \newty
}
\and
\inferrule[t\_tuple]{
  \vi\in\listrange(\tys): \renametyeqs(\eqs, \tys[\vi]) \typearrow \newty_\vi\\
  \newtys \eqdef [\vi\in\listrange(\tys): \newty_\vi]
}{
  \renametyeqs(\tenv, \eqs, \overname{\TTuple(\tys)}{\tty}) \typearrow
  \overname{\TTuple(\newtys)}{\newty}
}
\and
\inferrule[other]{
  \astlabel(\tty) \not\in \{\TBits, \TInt, \TTuple\}
}{
  \renametyeqs(\tenv, \eqs, \tty) \typearrow \overname{\tty}{\newty}
}
\end{mathpar}

\subsubsection{TypingRule.SubstExprNormalize \label{sec:TypingRule.SubstExprNormalize}}
\hypertarget{def-substexprnormalize}{}
The function
\[
\substexprnormalize(
  \overname{\staticenvs}{\tenv} \aslsep
  \overname{(\identifier\times\expr)^*}{\eqs} \aslsep
  \overname{\expr}{\ve}
) \aslto \overname{\newe}{\expr}
\]
transforms the expression $\ve$ in the static environment $\tenv$,
by substituting parameter names with their corresponding expressions in
$\eqs$, and then attempting to symbolically simplify the result, yielding the expression $\newe$.
\ProseOtherwiseTypeError

\subsubsection{Prose}
All of the following apply:
\begin{itemize}
  \item transforming $\ve$ in the static environment $\tenv$, by substituting the parameter expressions
        $\eqs$, yields $\veone$;
  \item symbolically simplifying $\veone$ in $\tenv$ yields $\newe$.
\end{itemize}

\subsubsection{Formally}
\begin{mathpar}
\inferrule{
  \substexpr(\tenv, \ve) \typearrow \veone\\
  \normalize(\tenv, \veone) \typearrow \newe
}{
  \substexprnormalize(\tenv, \eqs, \ve) \typearrow \newe
}
\end{mathpar}

\subsubsection{TypingRule.SubstExpr \label{sec:TypingRule.SubstExpr}}
\hypertarget{def-substexpr}{}
The function
\[
\substexpr(
  \overname{\staticenvs}{\tenv} \aslsep
  \overname{(\identifier\times\expr)^*}{\substs} \aslsep
  \overname{\expr}{\ve}
) \aslto \overname{\expr}{\newe}
\]
transforms the expression $\ve$ in the static environment $\tenv$,
by substituting parameter names with their corresponding expressions in
$\substs$, yielding the expression $\newe$.
\ProseOtherwiseTypeError

\subsubsection{Prose}
One of the following applies:
\begin{itemize}
  \item All of the following apply (\textsc{e\_var\_in\_substs}):
  \begin{itemize}
    \item $\ve$ is a variable expression for the identifier $\vs$, that is, $\EVar(\vs)$;
    \item applying $\assocopt$ to $\vs$ and $\substs$ yields the expression $\newe$.
          That is, $\vs$ is a parameter with an associated expression;
  \end{itemize}

  \item All of the following apply (\textsc{e\_var\_not\_in\_substs}):
  \begin{itemize}
    \item $\ve$ is the variable expression for the identifier $\vs$, that is, $\EVar(\vs)$;
    \item applying $\assocopt$ to $\vs$ and $\substs$ yields $\None$.
          That is, $\vs$ is not a parameter with an associated expression;
    \item define $\newe$ is $\ve$.
  \end{itemize}

  \item All of the following apply (\textsc{e\_unop}):
  \begin{itemize}
    \item $\ve$ is the unary operator expression for the operator $\op$ and expression $\ve$, that is, $\EUnop(\op, \veone)$;
    \item applying $\substexpr$ to $\substs$ and $\veone$ in $\tenv$ yields $\veonep$;
    \item define $\newe$ as the unary operator expression for the operator $\op$ and expression $\veonep$, that is, $\EUnop(\op, \veonep)$.
  \end{itemize}

  \item All of the following apply (\textsc{e\_binop}):
  \begin{itemize}
    \item $\ve$ is the binary operator expression for the operator $\op$ and expressions $\veone$ and $\vetwo$, that is, $\EBinop(\op, \veone, \vetwo)$;
    \item applying $\substexpr$ to $\substs$ and $\veone$ in $\tenv$ yields $\veonep$;
    \item applying $\substexpr$ to $\substs$ and $\vetwo$ in $\tenv$ yields $\vetwop$;
    \item define $\newe$ as the unary operator expression for the operator $\op$ and expression $\veonep$, that is, $\EUnop(\op, \veonep)$.
  \end{itemize}

  \item All of the following apply (\textsc{e\_cond}):
  \begin{itemize}
    \item $\ve$ is the conditional expression for expressions $\veone$, $\vetwo$, and $\vethree$, that is, \\
          $\ECond(\veone, \vetwo, \vethree)$;
    \item applying $\substexpr$ to $\substs$ and $\veone$ in $\tenv$ yields $\veonep$;
    \item applying $\substexpr$ to $\substs$ and $\vetwo$ in $\tenv$ yields $\vetwop$;
    \item applying $\substexpr$ to $\substs$ and $\vethree$ in $\tenv$ yields $\vethreep$;
    \item define $\newe$ as the conditional expression for expressions $\veonep$, $\vetwop$, and $\vethreep$, that is, $\ECond(\veonep, \vetwop, \vethreep)$.
  \end{itemize}

  \item All of the following apply (\textsc{e\_concat}):
  \begin{itemize}
    \item $\ve$ is the concatenation of expressions $\ves$, that is, $\EConcat(\ves)$;
    \item applying $\substexpr$ to $\substs$ and every expression $\ves[\vi]$, for $\vi$ in \\
          $\listrange(\ves)$ yields $\newes_\vi$;
    \item define $\vesp$ as the list of expressions $\newes_\vi$, for $\vi$ in $\listrange(\ves)$;
    \item define $\newe$ as the concatenation of expressions $\vesp$, that is, $\EConcat(\vesp)$.
  \end{itemize}

  \item All of the following apply (\textsc{e\_call}):
  \begin{itemize}
    \item $\ve$ is the call expression for subprogram $\vx$ with arguments $\vargs$ and parameter expressions $\paramargs$,
          that is, $\ECall(\vx, \vargs, \paramargs)$;
    \item applying $\substexpr$ to $\substs$ and every argument expression $\vargs[\vi]$, for $\vi$ in
          $\listrange(\vargs)$ yields $\ve_\vi$;
    \item define $\vargsp$ as $\ve_\vi$ for each $\vi$ in $\listrange(\vargs)$;
    \item define $\newe$ as the call expression for subprogram $\vx$ with arguments $\vargsp$ and parameter expressions $\paramargs$,
    that is, $\ECall(\vx, \vargsp, \paramargs)$.
  \end{itemize}

  \item All of the following apply (\textsc{e\_getarray}):
  \begin{itemize}
    \item $\ve$ is the \arrayaccess\ expression for base expression $\veone$ and index expression $\vetwo$,
          that is, $\EGetArray(\veone, \vetwo)$;
    \item applying $\substexpr$ to $\substs$ and $\veone$ in $\tenv$ yields $\veonep$;
    \item applying $\substexpr$ to $\substs$ and $\vetwo$ in $\tenv$ yields $\vetwop$;
    \item define $\newe$ as the \arrayaccess\ expression for base expression $\veonep$ and index expression $\vetwop$,
    that is, $\EGetArray(\veonep, \vetwop)$.
  \end{itemize}

  \item All of the following apply (\textsc{e\_getfield}):
  \begin{itemize}
    \item $\ve$ is the field access expression for base expression $\ve$ and field $\vx$,
          that is, $\EGetField(\veone, \vx)$;
    \item applying $\substexpr$ to $\substs$ and $\veone$ in $\tenv$ yields $\veonep$;
    \item define $\newe$ as the field access expression for base expression $\ve$ and field $\vx$,
          that is, $\EGetField(\veonep, \vx)$.
  \end{itemize}

  \item All of the following apply (\textsc{e\_getfields}):
  \begin{itemize}
    \item $\ve$ is the access to fields $\fields$ with base expression $\veone$, that is, \\
          $\EGetFields(\veone, \fields)$;
    \item applying $\substexpr$ to $\substs$ and $\veone$ in $\tenv$ yields $\veonep$;
    \item define $\newe$ as the access to fields $\fields$ with base expression $\veonep$, that is, $\EGetFields(\veonep, \fields)$.
  \end{itemize}

  \item All of the following apply (\textsc{e\_getitem}):
  \begin{itemize}
    \item $\ve$ is the access to tuple item $\vi$ of the tuple expression $\veone$, that is, \\
          $\EGetItem(\veone, \vi)$;
    \item applying $\substexpr$ to $\substs$ and $\veone$ in $\tenv$ yields $\veonep$;
    \item define $\newe$ as the access to tuple item $\vi$ of the tuple expression $\veonep$, that is, \\
          $\EGetItem(\veonep, \vi)$.
  \end{itemize}

  \item All of the following apply (\textsc{e\_pattern}):
  \begin{itemize}
    \item $\ve$ is the pattern expression of expression $\veone$ and patterns $\vps$, that is, \\
          $\EPattern(\veone, \vps)$;
    \item applying $\substexpr$ to $\substs$ and $\veone$ in $\tenv$ yields $\veonep$;
    \item define $\newe$ as the pattern expression of expression $\veonep$ and patterns $\vps$, that is, $\EPattern(\veonep, \vps)$.
  \end{itemize}

  \item All of the following apply (\textsc{e\_record}):
  \begin{itemize}
    \item $\ve$ is the record expression of record type $\vt$ and list of fields $\fields$;
    \item for every pair $(\vx,\veone)$ in $\fields$, applying $\substexpr$ to $\substs$ $\veone$ in $\tenv$ yields $\veonep_\vx$;
    \item define $\fieldsp$ as the list of pairs $(\vx,\veonep_\vx)$ for every pair $(\vx,\veone)$ in $\fields$;
    \item define $\newe$ as the record expression of record type $\vt$ and list of fields $\fieldsp$.
  \end{itemize}

  \item All of the following apply (\textsc{e\_slice}):
  \begin{itemize}
    \item $\ve$ is the slicing expression for sub-expression $\veone$ and list of slices $\vslices$, that is, $\ESlice(\veone, \vslices)$;
    \item applying $\substexpr$ to $\veone$ in $\tenv$ yields $\veonep$;
    \item define $\newe$ as slicing expression for sub-expression $\veonep$ and list of slices $\vslices$, that is, $\ESlice(\veonep, \vslices)$.
  \end{itemize}

  \item All of the following apply (\textsc{e\_tuple}):
  \begin{itemize}
    \item $\ve$ is the tuple expression of expressions $\ves$, that is, $\ETuple(\ves)$;
    \item applying $\substexpr$ to $\substs$ and every expression $\ves[\vi]$ in $\tenv$, for every $\vi$ in $\listrange(\ves)$
          yields $\newe_\vi$;
    \item define $\vesp$ as the list of expressions $\newe_\vi$, for every $\vi$ in $\listrange(\ves)$;
    \item define $\newe$ as the tuple expression of expressions $\vesp$, that is, $\ETuple(\vesp)$.
  \end{itemize}

  \item All of the following apply (\textsc{e\_atc}):
  \begin{itemize}
    \item $\ve$ is the type assertion of expression $\veone$ and type $\vt$, that is, $\EATC(\veone, \vt)$;
    \item applying $\substexpr$ to $\substs$ and $\veone$ in $\tenv$ yields $\veonep$;
    \item define $\newe$ as the type assertion of expression $\veonep$ and type $\vt$, that is, $\EATC(\veonep, \vt)$.
  \end{itemize}

  \item All of the following apply (\textsc{other}):
  \begin{itemize}
    \item $\ve$ is either a literal expression or an unknown value expression;
    \item define $\newe$ as $\ve$.
  \end{itemize}
\end{itemize}

\subsubsection{Formally}
\begin{mathpar}
\inferrule[e\_var\_in\_substs]{
  \assocopt(\vs, \substs) \typearrow \langle\newe\rangle
}{
  \substexpr(\tenv, \substs, \overname{\EVar(\vs)}{\ve}) \typearrow \newe
}
\end{mathpar}

\begin{mathpar}
\inferrule[e\_var\_not\_in\_substs]{
  \assocopt(\vs, \substs) \typearrow \None
}{
  \substexpr(\tenv, \substs, \overname{\EVar(\vs)}{\ve}) \typearrow \overname{\ve}{\newe}
}
\end{mathpar}

\begin{mathpar}
\inferrule[e\_unop]{
  \substexpr(\tenv, \substs, \veone) \typearrow \veonep\
}{
  \substexpr(\tenv, \substs, \overname{\EUnop(\op, \veone)}{\ve}) \typearrow \overname{\EUnop(\op, \veonep)}{\newe}
}
\end{mathpar}

\begin{mathpar}
\inferrule[e\_binop]{
  \substexpr(\tenv, \substs, \veone) \typearrow \veonep\\
  \substexpr(\tenv, \substs, \vetwop) \typearrow \vetwop
}{
  \substexpr(\tenv, \substs, \overname{\EBinop(\op, \veone, \vetwo)}{\ve}) \typearrow \overname{\EBinop(\op, \veonep, \vetwop)}{\newe}
}
\end{mathpar}

\begin{mathpar}
\inferrule[e\_cond]{
  \substexpr(\tenv, \substs, \veone) \typearrow \veonep\\
  \substexpr(\tenv, \substs, \vetwop) \typearrow \vetwop\\
  \substexpr(\tenv, \substs, \vethreep) \typearrow \vethreep\\
}{
  \substexpr(\tenv, \substs, \overname{\ECond(\veone, \vetwo, \vethree)}{\ve}) \typearrow \overname{\ECond(\veonep, \vetwop, \vethreep)}{\newe}
}
\end{mathpar}

\begin{mathpar}
\inferrule[e\_concat]{
  \vi\in\listrange(\ves): \substexpr(\tenv, \substs, \ves[\vi]) \typearrow \newes_\vi\\
  \vesp \eqdef [\vi\in\listrange(\ves): \newes_\vi]
}{
  \substexpr(\tenv, \substs, \overname{\EConcat(\ves)}{\ve}) \typearrow \overname{\EConcat(\vesp)}{\newe}
}
\end{mathpar}

\begin{mathpar}
\inferrule[e\_call]{
  \vi\in\listrange(\vargs): \substexpr(\tenv, \substs, \vargs[\vi]) \typearrow \ve_\vi\\
  \vargsp \eqdef [\vi\in\listrange(\vargs): \ve_\vi]
}{
  {
    \begin{array}{r}
  \substexpr(\tenv, \substs, \overname{\ECall(\vx, \vargs, \paramargs)}{\ve}) \typearrow \\ \overname{\ECall(\vx, \vargsp, \paramargs)}{\newe}
    \end{array}
  }
}
\end{mathpar}

\begin{mathpar}
\inferrule[e\_getarray]{
  \substexpr(\tenv, \substs, \veone) \typearrow \veonep\\
  \substexpr(\tenv, \substs, \vetwop) \typearrow \vetwop
}{
  \substexpr(\tenv, \substs, \overname{\EGetArray(\veone, \vetwo)}{\ve}) \typearrow \overname{\EGetArray(\veonep, \vetwop)}{\newe}
}
\end{mathpar}

\begin{mathpar}
\inferrule[e\_getfield]{
  \substexpr(\tenv, \substs, \veone) \typearrow \veonep
}{
  \substexpr(\tenv, \substs, \overname{\EGetField(\veone, \vx)}{\ve}) \typearrow \overname{\EGetField(\veonep, \vx)}{\newe}
}
\end{mathpar}

\begin{mathpar}
\inferrule[e\_getfields]{
  \substexpr(\tenv, \substs, \veone) \typearrow \veonep
}{
  \substexpr(\tenv, \substs, \overname{\EGetFields(\veone, \fields)}{\ve}) \typearrow \overname{\EGetFields(\veonep, \fields)}{\newe}
}
\end{mathpar}

\begin{mathpar}
\inferrule[e\_getitem]{
  \substexpr(\tenv, \substs, \veone) \typearrow \veonep
}{
  \substexpr(\tenv, \substs, \overname{\EGetItem(\veone, \vi)}{\ve}) \typearrow \overname{\EGetItem(\veonep, \vi)}{\newe}
}
\end{mathpar}

\begin{mathpar}
\inferrule[e\_pattern]{
  \substexpr(\tenv, \substs, \veone) \typearrow \veonep
}{
  \substexpr(\tenv, \substs, \overname{\EPattern(\veone, \vps)}{\ve}) \typearrow \overname{\EPattern(\veonep, \vps)}{\newe}
}
\end{mathpar}

\begin{mathpar}
\inferrule[e\_record]{
  (\vx, \veone)\in\fields: \substexpr(\tenv, \substs, \veone) \typearrow \veone_\vx\\
  \fieldsp \eqdef [(\vx, \veone)\in\fields: (\vx, \veone_\vx)]
}{
  \substexpr(\tenv, \substs, \overname{\ERecord(\vt, \fields)}{\ve}) \typearrow \overname{\ERecord(\vt, \fieldsp)}{\newe}
}
\end{mathpar}

\begin{mathpar}
\inferrule[e\_slice]{
  \substexpr(\tenv, \substs, \veone) \typearrow \veonep
}{
  \substexpr(\tenv, \substs, \overname{\ESlice(\veone, \slices)}{\ve}) \typearrow \overname{\ESlice(\veonep, \slices)}{\newe}
}
\end{mathpar}

\begin{mathpar}
\inferrule[e\_tuple]{
  \vi\in\listrange(\ves): \substexpr(\tenv, \substs, \ves[\vi]) \typearrow \newe_\vi\\
  \vesp \eqdef [\vi\in\listrange(\ves): \newe_\vi]
}{
  \substexpr(\tenv, \substs, \overname{\ETuple(\ves)}{\ve}) \typearrow \overname{\ETuple(\vesp)}{\newe}
}
\end{mathpar}

\begin{mathpar}
\inferrule[e\_atc]{
  \substexpr(\tenv, \substs, \veone) \typearrow \veonep
}{
  \substexpr(\tenv, \substs, \overname{\EATC(\veone, \vt)}{\ve}) \typearrow \overname{\EATC(\veonep, \vt)}{\newe}
}
\end{mathpar}

\begin{mathpar}
\inferrule[other]{
  \astlabel(\ve) \in \{\ELiteral, \EUnknown\}\
}{
  \substexpr(\tenv, \substs, \ve) \typearrow \overname{\ve}{\newe}
}
\end{mathpar}

\CodeSubsection{\SubstExprBegin}{\SubstExprEnd}{../ASTUtils.ml}

\subsubsection{TypingRule.SubstConstraint \label{sec:TypingRule.SubstConstraint}}
\hypertarget{def-substconstraint}{}
The function
\[
\substconstraint(
  \overname{\staticenvs}{\tenv} \aslsep
  \overname{(\identifier\times\expr)^*}{\eqs} \aslsep
  \overname{\intconstraint}{\vc}
) \aslto \overname{\newc}{\intconstraint}
\]
transforms the integer constraint $\vc$ in the static environment $\tenv$,
by substituting parameter names with their corresponding expressions in
$\eqs$, and then attempting to symbolically simplify the result,
yielding the integer constraint $\newc$.
\ProseOtherwiseTypeError

\subsubsection{Prose}
One of the following applies:
\begin{itemize}
  \item All of the following apply (\textsc{exact}):
  \begin{itemize}
    \item $\vc$ is an exact constraint for the expression $\ve$, that is, $\ConstraintExact(\ve)$;
    \item applying $\substexprnormalize$ in $\tenv$ to $\eqs$ and $\ve$ yields $\newe$;
    \item define $\newc$ as the exact constraint for the expression $\newe$, that is, \\
          $\ConstraintExact(\newe)$.
  \end{itemize}

  \item All of the following apply (\textsc{range}):
  \begin{itemize}
    \item $\vc$ is a range constraint for the expressions $\veone$ and $\vetwo$, that is, \\
          $\ConstraintRange(\veone, \vetwo)$;
    \item applying $\substexprnormalize$ in $\tenv$ to $\eqs$ and $\veone$ yields $\veonep$;
    \item applying $\substexprnormalize$ in $\tenv$ to $\eqs$ and $\vetwo$ yields $\vetwop$;
    \item define $\newc$ as the range constraint for the expressions $\veonep$ and $\vetwop$, that is, $\ConstraintRange(\veonep, \vetwop)$.
  \end{itemize}
\end{itemize}

\subsubsection{Formally}
\begin{mathpar}
\inferrule[exact]{
  \substexprnormalize(\tenv, \eqs, \ve) \typearrow \newe
}{
  \substconstraint(\tenv, \eqs, \overname{\ConstraintExact(\ve)}{\vc}) \typearrow \overname{\ConstraintExact(\newe)}{\newc}
}
\and
\inferrule[range]{
  \substexprnormalize(\tenv, \eqs, \veone) \typearrow \veonep\\
  \substexprnormalize(\tenv, \eqs, \vetwo) \typearrow \vetwop
}{
  \substconstraint(\tenv, \eqs, \overname{\ConstraintRange(\veone, \vetwo)}{\vc}) \typearrow \overname{\ConstraintRange(\veonep, \vetwop)}{\newc}
}
\end{mathpar}

\subsubsection{TypingRule.CheckArgsTypeSat \label{sec:TypingRule.CheckArgsTypeSat}}
\hypertarget{def-checkargstypesat}{}
The function
\[
\begin{array}{r}
  \checkargstypesat(
    \overname{\staticenvs}{\tenv} \aslsep
    \overname{(\identifier\times\ty)^*}{\calleeargs} \aslsep
    \overname{\ty^*}{\callerargtypes} \aslsep
    \overname{(\identifier\times\expr)^*}{\eqsthree}
  ) \aslto \\
  \{\True\}\ \cup \overname{\TTypeError}{\TypeErrorConfig}
\end{array}
\]
checks that the types $\callerargtypes$ \typesatisfy\ the types of the corresponding
formal arguments $\calleeargs$ with the parameters substituted with their corresponding
arguments as per $\eqsthree$ and results in a type error otherwise.

\subsubsection{Prose}
One of the following applies:
\begin{itemize}
  \item All of the following apply (\textsc{empty}):
  \begin{itemize}
    \item both $\calleeargs$ and $\callerargtypes$ are empty;
    \item the result is $\True$.
  \end{itemize}

  \item All of the following apply (\textsc{non\_empty}):
  \begin{itemize}
    \item view $\calleeargs$ as a list with \head\ $(\calleeargname, \calleearg)$ and \tail\ $\calleeargsone$;
    \item view $\callerargtypes$ as a list with \head\ $\callerarg$ and \tail\ \\
          $\callerargtypesone$;
    \item applying $\renametyeqs$ to $\eqsthree$ and $\calleearg$ in $\tenv$
          to substitute parameter arguments in $\calleearg$ yields $\calleeargone$\ProseOrTypeError;
    \item checking that $\callerarg$ \typesatisfies\ $\calleeargone$ in $\tenv$ yields $\True$\ProseOrTypeError;
    \item applying $\checkargstypesat$ to $\calleeargsone$, $\callerargtypesone$, \\
          and $\eqsthree$ in $\tenv$
          yields $\True$\ProseOrTypeError;
    \item the result is $\True$.
  \end{itemize}
\end{itemize}

\subsubsection{Formally}
We note that it is guaranteed by TypingRule.AnnotateCallArgTyped
that $\vargs$ and $\callerargstyped$ have the same length.

\begin{mathpar}
\inferrule[empty]{}{
  \checkargstypesat(\tenv, \overname{\emptylist}{\calleeargs}, \overname{\emptylist}{\callerargtypes}, \eqsthree)
  \typearrow \True
}
\and
\inferrule[non\_empty]{
  \calleeargs \eqname [(\calleeargname, \calleearg)] \concat \calleeargsone\\
  \callerargtypes \eqname [\callerarg] \concat \callerargtypesone\\
  \renametyeqs(\tenv, \eqsthree, \calleearg) \typearrow \calleeargone\OrTypeError\\\\
  \checktypesat(\tenv, \callerarg, \calleeargone) \typearrow \True\OrTypeError\\\\
  \checkargstypesat(\tenv, \calleeargsone, \callerargtypesone, \eqsthree) \typearrow \True
}{
  \checkargstypesat(\tenv, \calleeargs, \callerargtypes, \eqsthree) \typearrow \True
}
\end{mathpar}

\subsubsection{TypingRule.AnnotateParameterDefining \label{sec:TypingRule.AnnotateParameterDefining}}
\hypertarget{def-annotateparameterdefining}{}
The function
\[
  \annotateparameterdefining
  \left(
    \begin{array}{c}
      \overname{\staticenvs}{\tenv} \aslsep\\
      \overname{(\identifier\times\expr)^*}{\eqsone} \aslsep\\
      \overname{(\identifier\times\ty)^*}{\vargs} \aslsep\\
      \overname{(\ty\times\expr)^*}{\callerargstyped} \aslsep\\
      \overname{(\identifier\times\langle\ty\rangle)^*}{\calleeparams}
    \end{array}
  \right)
  \aslto
  \begin{array}{c}
  \overname{(\identifier\times\expr)^*}{\eqs} \cup \\
  \overname{\TTypeError}{\TypeErrorConfig}
  \end{array}
\]
checks that all parameter-defining arguments
in $\calleeparams$ are \staticallyevaluable\ constrained integers.
The result --- $\eqs$ --- is the list of parameter identifiers and their corresponding expressions,
added to $\eqsone$.
\ProseOtherwiseTypeError

\subsubsection{Prose}
One of the following applies:
\begin{itemize}
  \item All of the following apply (\textsc{empty}):
  \begin{itemize}
    \item both $\vargs$ and $\callerargstyped$ are empty;
    \item define $\eqs$ as $\eqsone$.
  \end{itemize}

  \item All of the following apply:
  \begin{itemize}
    \item view $\vargs$ as a list with \head\ $(\calleex, \Ignore)$ and \tail\ $\vargsone$;
    \item view $\callerargstyped$ as a list with \head\ $(\callerty, \callere)$ and \tail\ $\callerargstypedone$;
    \item define $\calleeargisparam$ as $\True$ if and only if $\calleex$ is listed as a parameter in $\calleeparams$;
    \item One of the following applies:
    \begin{itemize}
      \item All of the following apply (\textsc{arg\_is\_param}):
      \begin{itemize}
        \item $\calleeargisparam$ is $\True$;
        \item checking that $\callere$ is \staticallyevaluable\ in $\tenv$ yields $\True$\ProseOrTypeError;
        \item checking that $\callerty$ is a constrained integer in $\tenv$ yields \\
              $\True$\ProseOrTypeError;
        \item applying $\annotateparameterdefining$ to $\vargsone$, \\
              $\callerargstypedone$, $\eqsone$ in $\tenv$ yields $\eqstwo$\ProseOrTypeError;
        \item define $\eqs$ as a list with \head\ $(\calleex, \callere)$ and \tail\ $\eqstwo$.
      \end{itemize}

      \item All of the following apply (\textsc{arg\_is\_not\_param}):
      \begin{itemize}
        \item $\calleeargisparam$ is $\False$;
        \item applying $\annotateparameterdefining$ to $\vargsone$, \\
              $\callerargstypedone$, $\eqsone$ in $\tenv$ yields $\eqs$\ProseOrTypeError.
      \end{itemize}
    \end{itemize}
  \end{itemize}
\end{itemize}

\subsubsection{Formally}
We note that it is guaranteed by TypingRule.AnnotateCallArgTyped
that $\vargs$ and $\callerargstyped$ have the same length.

\begin{mathpar}
\inferrule[empty]{}{
  {
    \begin{array}{r}
  \annotateparameterdefining(\tenv, \eqsone, \overname{\emptylist}{\vargs}, \overname{\emptylist}{\callerargstyped}, \calleeparams)
  \typearrow \\
  \overname{\eqsone}{\eqs}
    \end{array}
  }
}
\end{mathpar}

\begin{mathpar}
\inferrule[arg\_is\_param]{
  \vargs \eqname [(\calleex, \Ignore)] \concat \vargsone\\
  \callerargstyped \eqname [(\callerty, \callere)] \concat \callerargstypedone\\
  {
    \left(\begin{array}{l}
  \calleeargisparam \eqdef \\
  \exists \vi\in\listrange(\calleeparams).\ \calleeparams[\vi]=(\calleex, \Ignore)
    \end{array}\right)
  }\\\\
  \commonprefixline\\\\
  \calleeargisparam\ = \True\\
  \checkstaticallyevaluable(\tenv, \callere) \typearrow \True\OrTypeError\\\\
  \checkconstrainedinteger(\tenv, \callerty) \typearrow \True\OrTypeError\\\\
  {
    \begin{array}{r}
  \annotateparameterdefining(\tenv, \vargsone, \callerargstypedone, \eqsone) \typearrow \\ \eqstwo \OrTypeError
    \end{array}
  }\\
  \eqs \eqdef [(\calleex, \callere)] \concat \eqstwo
}{
  {
    \annotateparameterdefining
    \left(
      \begin{array}{c}
      \tenv, \\
      \eqsone, \\
      \vargs, \\
      \callerargstyped, \\
      \calleeparams
      \end{array}
    \right)
    \typearrow \eqs \OrTypeError
  }
}
\end{mathpar}

\begin{mathpar}
\inferrule[arg\_is\_not\_param]{
  \vargs \eqname [(\calleex, \Ignore)] \concat \vargsone\\
  \callerargstyped \eqname [(\callerty, \callere)] \concat \callerargstypedone\\
  {
    \left(\begin{array}{l}
  \calleeargisparam \eqdef \\
  \exists \vi\in\listrange(\calleeparams).\ \calleeparams[\vi]=(\calleex, \Ignore)
    \end{array}\right)
  }\\\\
  \commonprefixline\\\\
  \calleeargisparam = \False\\
  \annotateparameterdefining(\tenv, \vargsone, \callerargstypedone, \eqsone) \typearrow \eqs\\
}{
  {
    \annotateparameterdefining
    \left(
      \begin{array}{c}
      \tenv, \\
      \eqsone, \\
      \vargs, \\
      \callerargstyped, \\
      \calleeparams
      \end{array}
    \right)
    \typearrow \eqs \OrTypeError
  }
}
\end{mathpar}

\subsubsection{TypingRule.AnnotateRetTy \label{sec:TypingRule.AnnotateRetTy}}
\hypertarget{def-annotateretty}{}
The function
\[
\begin{array}{r}
\annotateretty(
  \overname{\staticenvs}{\tenv} \aslsep
  \overname{\subprogramtype}{\calltype} \aslsep
  \overname{\langle\ty\rangle}{\calleerettyopt}
  \overname{(\identifier\times\expr)^*}{\eqsthree}
) \aslto \\
\overname{\langle\ty\rangle}{\rettyopt}
\cup \overname{\TTypeError}{\TypeErrorConfig}
\end{array}
\]
annotates the \optional\ return type $\calleerettyopt$ given with the subprogram type
$\calltype$ with respect to the parameter expressions $\eqsthree$,
yielding the \optional\ annotated type $\rettyopt$.
\ProseOtherwiseTypeError

\subsubsection{Prose}
One of the following applies:
\begin{itemize}
  \item All of the following apply (\textsc{function\_or\_getter}):
  \begin{itemize}
    \item $\calltype$ is one of $\STFunction$, $\STGetter$, or $\STEmptyGetter$;
    \item $\calleerettyopt$ is $\langle\tty\rangle$;
    \item applying $\renametyeqs$ to $\eqsthree$ and $\tty$ yields $\ttyone$\ProseOrTypeError;
    \item $\rettyopt$ is $\langle\ttyone\rangle$.
  \end{itemize}

  \item All of the following apply (\textsc{procedure\_or\_setter}):
  \begin{itemize}
    \item $\calltype$ is one of $\STProcedure$, $\STSetter$, or $\STEmptySetter$;
    \item $\calleerettyopt$ is $\None$;
    \item define $\rettyopt$ as $\None$.
  \end{itemize}

  \item All of the following apply (\textsc{ret\_type\_mismatch}):
  \begin{itemize}
    \item the condition that $\calltype$ is one of $\STProcedure$, $\STSetter$, or \\
          $\STEmptySetter$
          if and only if $\calleerettyopt$ is $\None$ does not hold;
    \item the result is a type error indicating the mismatch.
  \end{itemize}
\end{itemize}

\subsubsection{Formally}
\begin{mathpar}
\inferrule[function\_or\_getter]{
  \calltype \in \{\STFunction, \STGetter, \STEmptyGetter\}\\
  \renametyeqs(\eqsthree, \tty) \typearrow \ttyone \OrTypeError
}{
  \annotateretty(\tenv, \calltype, \overname{\langle\tty\rangle}{\calleerettyopt}, \eqsthree) \typearrow
  \overname{\langle\ttyone\rangle}{\rettyopt}
}
\end{mathpar}

\begin{mathpar}
\inferrule[procedure\_or\_setter]{
  \calltype \in \{\STProcedure, \STSetter, \STEmptySetter\}\\
}{
  \annotateretty(\tenv, \calltype, \overname{\None}{\calleerettyopt}, \eqsthree) \typearrow
  \overname{\None}{\rettyopt}
}
\end{mathpar}

\begin{mathpar}
\inferrule[ret\_type\_mismatch]{
  {
  \neg\left(
    \begin{array}{c}\calltype \in \{\STProcedure, \STSetter, \STEmptySetter\} \leftrightarrow \\
      \calleerettyopt=\None
    \end{array}
  \right)
  }
}{
  \annotateretty(\tenv, \calltype, \calleerettyopt, \eqsthree) \typearrow
  \TypeErrorVal{\MismatchedReturnValue}
}
\end{mathpar}

\subsubsection{TypingRule.SubprogramForName \label{sec:TypingRule.SubprogramForName}}
\hypertarget{def-subprogramforname}{}
The function
\[
  \begin{array}{rl}
    \subprogramforname(
    \overname{\staticenvs}{\tenv} \aslsep
    \overname{\Strings}{\name} \aslsep
    \overname{\ty^*}{\callerargtypes}
  )
  \aslto &
    (\overname{(\identifier \times \expr)^*}{\extranargs} \aslsep
    \overname{\Strings}{\namep} \aslsep
    \overname{\func}{\callee}
    )\\
  & \cup\ \overname{\TTypeError}{\TypeErrorConfig}
  \end{array}
\]
looks up the static environment $\tenv$ for a subprogram associated with $\name$
and the list of argument types $\calleeargtypes$ and determines which one of the following
cases holds:
\begin{itemize}
  \item there is no declared subprogram that matches $\name$ and $\calleeargtypes$;
  \item there is exactly one subprogram that matches $\name$ and $\calleeargtypes$;
  \item there is more than one subprogram that matches $\name$ and $\calleeargtypes$;
\end{itemize}
The first and last cases result in a type error.
If the second case holds, the function returns a tuple comprised of:
\begin{itemize}
\item $\extranargs$ --- the list of extra named arguments (parameters);
\item $\namep$ --- the string that uniquely identifies this subprogram;
\item $\callee$ --- the AST node defining the called subprogram.
\end{itemize}
\ProseOtherwiseTypeError

\subsubsection{Prose}
One of the following applies:
\begin{itemize}
  \item All of the following apply (\textsc{undefined}):
  \begin{itemize}
    \item $\tenv$ does not contain a binding for $\name$ in the $\subprogramrenamings$ map
          ($G^\tenv.\subprogramrenamings$);
    \item the result is a type error indicating that the identifier has not been declared (as a subprogram).
  \end{itemize}

  \item All of the following apply (\textsc{no\_candidates}):
  \begin{itemize}
    \item $\tenv$ binds $\name$ via $\subprogramrenamings$ map to $\renamingset$;
    \item filtering the subprograms in $\renamingset$ with the caller argument types $\callerargtypes$
          in $\tenv$ (see \secref{TypingRule.FilterCallCandidates}) yields an empty set\ProseOrTypeError;
    \item the result is a type error indicating that the call given by $\name$ and \\ $\callerargtypes$
          does not match any defined subprogram.
  \end{itemize}

  \item All of the following apply (\textsc{too\_many\_candidates}):
  \begin{itemize}
    \item $\tenv$ binds $\name$ via $\subprogramrenamings$ map to $\renamingset$;
    \item filtering the subprograms in $\renamingset$ with the caller argument types $\callerargtypes$
          in $\tenv$ (see \secref{TypingRule.FilterCallCandidates}) yields $\matchingrenamings$\ProseOrTypeError;
    \item $\matchingrenamings$ contains at least two elements;
    \item the result is a type error indicating that the call given by $\name$ and \\
          $\callerargtypes$ matches more than one defined subprogram.
  \end{itemize}

  \item All of the following apply (\textsc{one\_candidate}):
  \begin{itemize}
    \item $\tenv$ binds $\name$ via $\subprogramrenamings$ map to $\renamingset$;
    \item filtering the subprograms in $\renamingset$ with the caller argument types $\callerargtypes$
          in $\tenv$ (see \secref{TypingRule.FilterCallCandidates}) yields $\matchingrenamings$\ProseOrTypeError;
    \item $\matchingrenamings$ contains a single element --- $(\matchedname, \funcsig)$;
    \item deducing the argument values for the parameters via $\deduceeqs$ with \\
          $\callerargtypes$, $\funcsig.\funcargs$ in $\tenv$ yields \\
          $(\extranargs, \namep, \callee)$\ProseOrTypeError.
  \end{itemize}
\end{itemize}
\CodeSubsection{\SubprogramForNameBegin}{\SubprogramForNameEnd}{../Typing.ml}

\subsubsection{Formally}
\begin{mathpar}
\inferrule[undefined]{
  G^\tenv.\subprogramrenamings(\name) = \bot
}{
  \subprogramforname(\tenv, \name, \callerargtypes) \typearrow \TypeErrorVal{\UndefinedIdentifier}
}
\end{mathpar}

\begin{mathpar}
\inferrule[no\_candidates]{
  G^\tenv.\subprogramrenamings(\name) = \renamingset\\
  {
    \begin{array}{r}
      \filtercallcandidates(\tenv, \callerargtypes, \renamingset) \typearrow \emptyset \OrTypeError
    \end{array}
  }
}{
  \subprogramforname(\tenv, \name, \callerargtypes) \typearrow \TypeErrorVal{\NoCallCandidates}
}
\end{mathpar}

\begin{mathpar}
\inferrule[too\_many\_candidates]{
  G^\tenv.\subprogramrenamings(\name) = \renamingset\\
  {
    \begin{array}{r}
      \filtercallcandidates(\tenv, \callerargtypes, \renamingset) \typearrow \\ \matchingrenamings \OrTypeError
    \end{array}
  }\\\\
  \cardinality{\matchingrenamings} \geq 2
}{
  \subprogramforname(\tenv, \name, \callerargtypes) \typearrow \TypeErrorVal{\TooManyCandidates}
}
\end{mathpar}

\begin{mathpar}
\inferrule[one\_candidate]{
  G^\tenv.\subprogramrenamings(\name) = \renamingset\\
  {
    \begin{array}{r}
      \filtercallcandidates(\tenv, \callerargtypes, \renamingset) \typearrow \\ \matchingrenamings \OrTypeError
    \end{array}
  }\\\\
  \matchingrenamings = [(\matchedname, \funcsig)]\\
  {
    \begin{array}{r}
  \deduceeqs(\tenv, \callerargtypes, \funcsig.\funcargs)
  \typearrow \\ (\extranargs, \namep, \callee) \OrTypeError
    \end{array}
  }
}{
  {
    \begin{array}{r}
  \subprogramforname(\tenv, \name, \callerargtypes) \typearrow \\ (\extranargs, \namep, \callee)
    \end{array}
  }
}
\end{mathpar}

\subsubsection{TypingRule.DeduceEqs \label{sec:TypingRule.DeduceEqs}}
\hypertarget{def-deduceeqs}{}
The function
\[
\deduceeqs(
  \overname{\staticenvs}{\tenv} \aslsep
  \overname{\ty^*}{\callerargtypes} \aslsep
  \overname{(\identifier\times\ty)^*}{\vargs}
) \aslto
\overname{(\identifier\times\expr)^*}{\eqs}
\cup \overname{\TTypeError}{\TypeErrorConfig}
\]
takes the types of the actual arguments of a call --- $\callerargtypes$,
the list of formal arguments --- $\vargs$ --- which consist of the names of a subprogram arguments
and their associated types, and
infers the expressions associated with parameters that correspond to bitvector widths,
yielding the result in $\eqs$.
\ProseOtherwiseTypeError

It guaranteed that by TypingRule.HasArgClash,
which is used by \\ TypingRule.FilterCallCandidates before calling $\deduceeqs$,
that $\callerargtypes$ and $\vargs$ have the same length.

\subsubsection{Prose}
One of the following applies:
\begin{itemize}
  \item All of the following apply (\textsc{empty}):
  \begin{itemize}
    \item both $\callerargtypes$ and $\vargs$ are empty lists;
    \item define $\eqs$ as the empty list.
  \end{itemize}

  \item All of the following apply (\textsc{not\_bits\_parameter}):
  \begin{itemize}
    \item $\callerargtypes$ has \head\ $\caller$ and \tail\ $\callerargtypesone$;
    \item $\vargs$ has \head\ $(\Ignore, \callee)$ and \tail\ $\vargsone$;
    \item $\caller$ is not a bitvector type with a width expression that is a variable expression;
    \item applying $\deduceeqs$ to $\callerargtypesone$ and $\vargsone$ in $\tenv$ yields $\eqs$.
  \end{itemize}

  \item All of the following apply (\textsc{bits\_parameter}):
  \begin{itemize}
    \item $\callerargtypes$ has \head\ $\caller$ and \tail\ $\callerargtypesone$;
    \item $\vargs$ has \head\ $(\Ignore, \callee)$ and \tail\ $\vargsone$;
    \item $\caller$ is bitvector type whose width expression is the variable expression for $\vx$;
    \item obtaining the \structure\ of $\caller$ in $\tenv$ yields the bitvector type with width expression $\ecaller$\ProseOrTypeError;
    \item applying $\deduceeqs$ to $\callerargtypesone$ and $\vargsone$ in $\tenv$ yields $\eqsone$;
    \item define $\eqs$ as the list with \head\ $(\vx, \ecaller)$.
  \end{itemize}
\end{itemize}

\subsubsection{Formally}
\begin{mathpar}
\inferrule[empty]{}{
  \deduceeqs(\tenv, \overname{\emptylist}{\callerargtypes}, \overname{\emptylist}{\vargs}) \typearrow \overname{\emptylist}{\eqs}
}
\end{mathpar}

\begin{mathpar}
\inferrule[not\_bits\_parameter]{
  \caller \neq \TBits(\EVar(\Ignore))\\
  \deduceeqs(\tenv, \callerargtypesone, \vargsone) \typearrow \eqs
}{
  \deduceeqs(
    \tenv,
    \overname{[\caller] \concat \callerargtypesone}{\callerargtypes},
    \overname{[(\Ignore, \callee)] \concat \vargsone}{\vargs}) \typearrow \eqs
}
\end{mathpar}

\begin{mathpar}
\inferrule[bits\_parameter]{
  \caller = \TBits(\EVar(\vx))\\
  \tstruct(\tenv, \caller) \typearrow \TBits(\ecaller, \Ignore) \OrTypeError\\\\
  \deduceeqs(\tenv, \callerargtypesone, \vargsone) \typearrow \eqsone\\
  \eqs \eqdef [(\vx, \ecaller)] \concat \eqsone
}{
  \deduceeqs(
    \tenv,
    \overname{[\caller] \concat \callerargtypesone}{\callerargtypes},
    \overname{[(\Ignore, \callee)] \concat \vargsone}{\vargs}) \typearrow \eqs
}
\end{mathpar}

\subsubsection{TypingRule.FilterCallCandidates \label{sec:TypingRule.FilterCallCandidates}}
\hypertarget{def-filtercandidates}{}
The helper function
\[
  \filtercallcandidates(
    \overname{\staticenvs}{\tenv} \aslsep
    \overname{\ty^*}{\formaltypes} \aslsep
    \overname{\pow{\Strings}}{\candidates}
    )
  \aslto \overname{(\Strings\times\func)^*}{\matches}
\]
iterates over the list of unique subprogram names in $\candidates$ and checks whether
their lists of arguments clash with the types in $\formaltypes$ in $\tenv$.
The result is the set of pairs consisting of the names and function definitions of the
subprograms whose arguments clash in $\candidates$.
\ProseOtherwiseTypeError

The names $\candidates$ are assumed to exist in $G^\tenv.\subprograms$.

\subsubsection{Prose}
One of the following applies:
\begin{itemize}
  \item All of the following apply (\textsc{no\_candidates}):
  \begin{itemize}
    \item $\candidates$ is empty;
    \item $\matches$ is empty.
  \end{itemize}

  \item All of the following apply (\textsc{candidates\_exist}):
  \begin{itemize}
    \item $\candidates$ is a list with \head\ $\name$ and \tail\ $\candidatesone$;
    \item the function definition associated with $\name$ in $\tenv$ is $\funcdef$;
    \item determining whether there is an argument clash between $\formaltypes$ and the arguments in $\funcdef$
          (that is, $\funcdef.\funcargs$) yields $\vb$\ProseOrTypeError;
    \item filtering the call candidates in $\candidatesone$ with $\formaltypes$ in $\tenv$ yields $\matchesone$\ProseOrTypeError;
    \item if $\vb$ is $\True$ then $\matches$ is the list with \head\ $(\name,\funcdef)$ and \tail\ $\matchesone$,
          and otherwise it is $\matchesone$.
  \end{itemize}
\end{itemize}

\subsubsection{Formally}
\begin{mathpar}
\inferrule[no\_candidates]
{}{
  \filtercallcandidates(\tenv, \formaltypes, \overname{\emptylist}{\candidates}) \typearrow \overname{\emptylist}{\matches}
}
\and
\inferrule[candidates\_exist]
{
  \funcdef \eqdef G^\tenv.\subprograms(\name) \\
  \hasargclash(\tenv, \formaltypes, \funcdef.\funcargs) \typearrow \vb \OrTypeError\\\\
  \filtercallcandidates(\tenv, \formaltypes, \candidatesone) \typearrow \matchesone \OrTypeError\\\\
  \matches \eqdef \choice{\vb}{\;[(\name,\funcdef)] \concat \matchesone}{\;\matchesone}
}{
  \filtercallcandidates(\tenv, \formaltypes, \overname{[\name] \concat \candidatesone}{\candidates}) \typearrow \matches
}
\end{mathpar}

\subsubsection{TypingRule.HasArgClash \label{sec:TypingRule.HasArgClash}}
\hypertarget{def-hasargclash}{}
The function
\[
  \hasargclash(
    \overname{\staticenvs}{\tenv} \aslsep
    \overname{\ty^*}{\formaltys} \aslsep
    \overname{(\identifier\times\ty)^*}{\vargs}
  )
  \aslto \overname{\Bool}{\vb} \cup \overname{\TTypeError}{\TypeErrorConfig}
\]
checks whether a list of types $\formaltys$ clashes with the list of types appearing
in the list of arguments $\vargs$ in $\tenv$, yielding the result in $\vb$.
\ProseOtherwiseTypeError

\subsubsection{Prose}
All of the following apply:
\begin{itemize}
  \item equating the list lengths of $\formaltys$ and $\vargs$ either yields $\True$
        or $\False$, which short-circuits the entire rule;
  \item $\argtys$ is the list of types appearing in $\vargs$, in the same order;
  \item for each $\vi$ in the list of indices of $\formaltys$, applying $\typeclashes$ to
        $\formaltys[\vi]$ and $\argtys[\vi]$ in $\tenv$ yields $\True$\ProseTerminateAs{\False, \TypeErrorConfig};
  \item $\vb$ is $\True$ (unless the rule short-circuited with $\False$ or a type error).
\end{itemize}

\CodeSubsection{\HasArgClashBegin}{\HasArgClashEnd}{../Typing.ml}

\subsubsection{Formally}
\begin{mathpar}
\inferrule{
  \equallength(\formaltypes, \vargs) \typearrow \True \terminateas \False\\
  \argtys \eqdef [(\Ignore, \vt) \in \vargs: \vt]\\
  \vi\in\listrange(\formaltys): \typeclashes(\tenv, \formaltys[\vi], \argtys[\vi]) \typearrow \True \terminateas \False,\TypeErrorConfig
}{
  \hasargclash(\tenv, \formaltys, \vargs) \typearrow \overname{\True}{\vb}
}
\end{mathpar}

\subsubsection{TypingRule.ExpressionList\label{sec:TypingRule.ExpressionList}}
\hypertarget{def-annotateexprs}{}
The helper function
\[
  \annotateexprlist(\overname{\staticenvs}{\tenv} \aslsep \overname{\expr^*}{\exprs})
  \aslto \overname{(\ty \times \expr)^*}{\typedexprs}
  \cup \overname{\TTypeError}{\TypeErrorConfig}
\]
annotates a list of expressions $\exprs$ from left to right, yielding a list of pairs $\typedexprs$,
each consisting of a type and expression.
\ProseOtherwiseTypeError

\subsubsection{Prose}
One of the following applies:
\begin{itemize}
  \item All of the following apply (\textsc{empty}):
  \begin{itemize}
    \item $\exprs$ is empty;
    \item $\typedexprs$ is empty.
  \end{itemize}

  \item All of the following apply (\textsc{non\_empty}):
  \begin{itemize}
    \item $\exprs$ has $\ve$ as its \head\ expression and $\exprsone$ as its \tail;
    \item annotating $\ve$ in $\tenv$ yields the pair $\typedexpr$ consisting of a type and an expression
    \ProseOrTypeError;
    \item annotating the expression list $\exprsone$ in $\tenv$ yields
    $\typedexprs$\ProseOrTypeError;
    \item $\typedexprs$ is the list with $\typedexpr$ as its \head\ and $\typedexprs$ as its \tail.
  \end{itemize}
\end{itemize}

\subsubsection{Formally}
\begin{mathpar}
\inferrule[empty]{}
{
  \annotateexprlist(\tenv, \overname{\emptylist}{\exprs}) \typearrow \overname{\emptylist}{\typedexprs}
}
\end{mathpar}

\begin{mathpar}
\inferrule[non\_empty]{
  \annotateexpr{\tenv, \ve} \typearrow \typedexpr \OrTypeError\\\\
  \annotateexprlist(\tenv, \exprsone) \typearrow \typedexprsone \OrTypeError
}{
  \annotateexprlist(\tenv, \overname{[\ve] \concat \exprsone}{\exprs}) \typearrow \overname{[\typedexpr] \concat \typedexprsone}{\typedexprs}
}
\end{mathpar}

%%%%%%%%%%%%%%%%%%%%%%%%%%%%%%%%%%%%%%%%%%%%%%%%%%%%%%%%%%%%%%%%%%%%%%%%%%%%%%%%
\section{Semantics}
%%%%%%%%%%%%%%%%%%%%%%%%%%%%%%%%%%%%%%%%%%%%%%%%%%%%%%%%%%%%%%%%%%%%%%%%%%%%%%%%
The relation
\hypertarget{def-evalcall}{}
\[
  \begin{array}{c}
    \evalcall{\overname{\envs}{\env} \aslsep
    \overbracket{\Identifiers}^{\texttt{name}} \aslsep
    \overbracket{\expr^*}^{\texttt{args}} \aslsep
    \overbracket{(\overname{\Identifiers}{\id_i}\times\overname{\expr}{\ve_i})^*}^{\texttt{named\_args}}} \;\aslrel\; \\
    \Normal(\overbracket{(\vals\times\XGraphs)^*}^{\vmstwo}, \overname{\envs}{\newenv}) \cup
    \overname{\TThrowing}{\ThrowingConfig} \cup \overname{\TError}{\ErrorConfig}
  \end{array}
\]
evaluates a call to the subprogram named $\name$ in the environment $\env$,
with the argument expressions
$\args$, and the parameter expressions $\namedargs$.
The evaluation results in either a list of returned values, each one associated
with an execution graph, and a new environment;
or an abnormal configuration.

The evaluation first evaluates the expressions corresponding to the arguments
and parameters and then passes their values in a resulting configuration
to the helper relation \texttt{eval\_subprogram}.

The relation
\hypertarget{def-evalsubprogram}{}
\[
  \begin{array}{c}
    \evalsubprogram{\overname{\envs}{\env} \aslsep
    \overname{\Identifiers}{\name} \aslsep
    \overname{(\vals \times \XGraphs)^*}{\actualargs} \aslsep
    \overname{(\Identifiers\times\vals)^*}{\params}} \aslrel \\
    \Normal(\overname{(\vals^* \aslsep \XGraphs)}{\vvs}, \overname{\envs}{\newenv}) \cup
    \overname{\TThrowing}{\ThrowingConfig} \cup
    \overname{\TError}{\ErrorConfig}
  \end{array}
\]
evaluates the subprogram named $\name$ in the environment $\env$, with
$\actualargs$ the list of actual arguments, and $\params$ the
list of arguments deduced by type equality.
The result is either a normal configuration or an abnormal configuration.
In the case of a normal configuration, it consists of a list of pairs
with a value and an identifier, and a new environment $\newenv$.
The values represent values returned by the subprogram call and the
identifiers are used in generating execution graph constraints for the
returned values.

The main subprogram call relation is given by
SemanticsRule.Call (see \secref{SemanticsRule.Call}).
%
The different types of subprogram calls are evaluated via one of the following rules:
\begin{itemize}
\item SemanticsRule.FPrimitive (see \secref{SemanticsRule.FPrimitive})
\item SemanticsRule.FCall (see \secref{SemanticsRule.FCall})
\end{itemize}

We also define the following helper rules:
\begin{itemize}
  \item SemanticsRule.ReadValueFrom (see \secref{SemanticsRule.ReadValueFrom})
  \item SemanticsRule.WriteRetVals (see \secref{SemanticsRule.WriteRetVals})
  \item SemanticsRule.AssignArgs (see \secref{SemanticsRule.AssignArgs})
  \item SemanticsRule.AssignNamedArgs (see \secref{SemanticsRule.AssignNamedArgs})
  \item SemanticsRule.MatchFuncRes (see \secref{SemanticsRule.MatchFuncRes})
\end{itemize}

\subsubsection{SemanticsRule.Call \label{sec:SemanticsRule.Call}}
\subsubsection{Prose}
All of the following apply:
\begin{itemize}
  \item $\namedargs$ is a list of identifier-expression pairs $(\id_i, \ve_i)$, for $i=1..k$;
  \item $\names$ is the list of identifiers in $\namedargs$;
  \item $\nargsone$ is the list of argument expressions in $\namedargs$;
  \item evaluating each expression in $\args$ separately in $\env$ as per \secref{SemanticsRule.EExprListM}
  is \\ $\Normal(\vargs, \envone)$\ProseOrAbnormal;
  \item evaluating each expression in $\nargs$ separately in $\envone$ as per \secref{SemanticsRule.EExprListM}
  is \\ $\Normal(\nargstwo, \envtwo)$\ProseOrAbnormal;
  \item $\nargstwo$ is the list of value-execution graph pairs $\vm_i$, for $i=1..k$;
  \item $\nargsthree$ is the list of pairs $(\id_i, \vm_i)$, for $i=1..k$ (this is the format needed
  for \texttt{eval\_subprogram});
  \item $\envtwo$ consists of the static environment $\tenv$ and the dynamic environment $\denvtwo$;
  \item the environment $\envtwo'$ is defined as the environment
  consisting of the static environment $\tenv$ and the dynamic environment with the global component
  of $\denvtwo$ and an empty local component (intuitively, this is because the called subprogram does not have access
  to the local environment of the caller);
  \item evaluating the subprogram named $\name$ with arguments $\vvargs$ and parameters $\nargsthree$ in
  $\denvtwo'$ is $\Normal(\vms, (\vglobal, \Ignore))$ (that is, we ignore the local environment
  of the callee)\ProseOrAbnormal;
  \item the list $\vms$ consists of value-identifier pairs $(\vv_j, \rid_j)$, for $i=1..n$;
  \item applying the helper relation $\readvaluefrom$ to each $(\vv_j, \rid_j)$
  results in $\vmstwo_j$, for $i=1..n$;
  \item $\vmstwo$ is defined as the list of $\vmstwo_j$, for $i=1..n$;
  \item $\newenv$ consists of the static environment $\tenv$ and the dynamic environment consisting
  of $\vglobal$ as the global component and the local component of $\denvtwo$ (that is, we restore
  the local environment to that of the caller and drop the local environment of the callee).
  \item the entire evaluation results in $\Normal(\vmstwo, \newenv)$.
\end{itemize}

\CodeSubsection{\EvalCallBegin}{\EvalCallEnd}{../Interpreter.ml}

\subsubsection{Formally}
\begin{mathpar}
\inferrule{
  \namedargs \eqname [i=1..k: (\id_i, \ve_i)]\\
  \names \eqdef [i=1..k: \id_i]\\
  \nargsone \eqdef [i=1..k: \ve_i]\\
  \evalexprlistm(\env, \args) \evalarrow \Normal(\vvargs, \envone)\OrAbnormal\\
  \evalexprlistm(\envone, \nargs1) \evalarrow \Normal(\nargstwo, \envtwo)\OrAbnormal\\
  \nargstwo \eqname [i=1..k: \vm_i]\\
  \nargsthree \eqdef [i=1...k: (\id_i, \vm_i)]\\
  \envtwo \eqname (\tenv, \denvtwo)\\
  \envtwo' \eqdef (\tenv,(G^\denvtwo,\emptyfunc))\\\\
  \evalsubprogram{\envtwo', \name, \vvargs, \nargsthree} \evalarrow \Normal(\vms, (\vglobal, \Ignore)) \OrAbnormal\\\\
  \vms \eqname [j=1..n: (\vv_j, \rid_j)]\\
  j=1..n: \readvaluefrom(\vv_j, \rid_j) \evalarrow \vmstwo_j\\
  \vmstwo \eqdef [j=1..n: \vmstwo_j]\\
  \newenv \eqdef (\tenv, (\vglobal, L^{\denvtwo}))
}{
  \evalcall{\env, \name, \args, \namedargs} \evalarrow \Normal(\vmstwo, \newenv)
}
\end{mathpar}

\subsubsection{SemanticsRule.FPrimitive \label{sec:SemanticsRule.FPrimitive}}
\subsubsection{Prose}
All of the following apply:
\begin{itemize}
  \item $\env$ consists of the static environment $\tenv$ and the dynamic environment with $\genv$ as its
        global component and an empty local component;
  \item finding the function named $\name$ in the static environment $\tenv$ gives a $\func$ AST node
        with the body field \SBPrimitive;
  \item evaluating the primitive subprogram $\name$ with the actual arguments $\actualargs$
        is $\Normal(\vms, \vgone)$\ProseOrError;
  \item writing the returned values $\vms$ as per \secref{SemanticsRule.WriteRetVals} gives $\vvsm$;
  \item $\vvsm$ is a pair consisting of the list of values $\vvs$ and execution graph $\vgtwo$;
  \item $\newg$ is the ordered composition of $\vgone$ and $\vgtwo$ with the $\asldata$ label;
  \item $\newenv$ is the environment with $\tenv$ as its static environment component
        and the dynamic environment consisting of $\genv$ as its global component and an empty local component;
  \item the result of the entire evaluation is $\Normal((\vvs, \newg), \newenv)$.
\end{itemize}

\subsubsection{Example}
In the specification:
\VerbatimInput{../tests/ASLSemanticsReference.t/SemanticsRule.FPrimitive.asl}
\texttt{print ("Hello, world!");} calls the primitive \texttt{print} on the evaluation of \texttt{"Hello, world!"}.

\CodeSubsection{\EvalFPrimitiveBegin}{\EvalFPrimitiveEnd}{../Interpreter.ml}

\subsubsection{Formally}
The following rule utilizes the transition relation
\hypertarget{def-evalprimitive}{}
\[
  \evalprimitive{\overname{\Identifiers}{\name} \aslsep \overname{(\vals\times\XGraphs)^*}{\actualargs}} \bigtimes
  \Normal(\overname{(\vals\times\XGraphs)^*}{\vms} \aslsep \overname{\XGraphs}{\vgone}) \cup \overname{\TError}{\ErrorConfig} \enspace,
\]
which parameterizes the ASL semantics and allows evaluating primitive subprograms.
That is, it is not a part of $\evalarrow$ but rather a separate transition relation denoted $\evalprimitivearrow$.

\begin{mathpar}
\inferrule{
  \env \eqname (\tenv, (\genv, \emptyfunc))\\
  \findfunc(\tenv, \name) = \{ \body = \SBPrimitive \ldots \}\\
  \evalprimitive{\name, \actualargs} \evalprimitivearrow \Normal(\vms, \vgone) \OrDynError\\\\
  \writeretvals(\vms) \evalarrow \vvsm\\
  \vvsm \eqname (\vvs, \vgtwo)\\
  \newg \eqdef \ordered{\vgone}{\asldata}{\vgtwo}\\
  \newenv \eqdef (\tenv, (\genv, \emptyfunc))
}{
  \evalsubprogram{\env, \name, \actualargs, \params} \evalarrow \Normal((\vvs, \newg), \newenv)
}
\end{mathpar}

\subsubsection{SemanticsRule.FCall \label{sec:SemanticsRule.FCall}}
\subsubsection{Prose}
All of the following apply:
\begin{itemize}
  \item $\env$ consists of the static environment $\tenv$ and the dynamic environment with the global
        component $\genv$ and an empty local component;
  \item finding the function named $\name$ in $\tenv$ gives the AST $\func$ node with body
        $\SBASL(\vbody)$ and arguments $\argdecls$;
  \item $\envone$ is the environment consisting of the static environment $\tenv$ and the dynamic1
        environment consisting of the dynamic component from $\denv$ and an empty local component;
  \item assigning the actual arguments with $((\envone, \emptygraph), \argdecls, \actualargs)$
        as per \secref{SemanticsRule.AssignArgs} gives $(\envtwo, vgtwo)$ make sure that each
        formal argument in $\argdecls$ is
        locally bound to the corresponding actual argument in $\actualargs$;
  \item declaring and assigning the parameter values with $((\envtwo, \vgtwo), \params)$
        as per \secref{SemanticsRule.AssignNamedArgs} gives $(\envthree, \vgthree)$;
  \item evaluating the body of the subprogram $\vbody$ as a statement in in $\envthree$
        is $\vres$\ProseOrAbnormal;
  \item matching the result $\vres$ to obtain a normal configuration as per \secref{SemanticsRule.MatchFuncRes}
        gives $C$;
  \item $\newg$ is the ordered composition of $\vgtwo$ and $\vgthree$ with the $\aslpo$ edge;
  \item the result is $C$ with its graph substituted for $\newg$.
\end{itemize}

\subsubsection{Example}
The specification:
\VerbatimInput{../tests/ASLSemanticsReference.t/SemanticsRule.FCall.asl}
calls the function \texttt{foo} and the procedure \texttt{bar}.

\CodeSubsection{\EvalFCallBegin}{\EvalFCallEnd}{../Interpreter.ml}

\subsubsection{Formally}
\begin{mathpar}
\inferrule{
  \env \eqname (\tenv, \denv)\\
  \findfunc(\tenv, \name) \eqname \{ \body: \SBASL(\vbody), \args: \argdecls, \ldots \}\\
    \envone \eqdef (\tenv, (G^\denv, \emptyfunc))\\
    \assignargs((\envone, \emptygraph), \argdecls, \actualargs) \evalarrow (\envtwo, \vgtwo)\\
    \assignnamedargs((\envtwo, \vgtwo), \params) \evalarrow (\envthree, \vgthree)\\
    \evalstmt{\envthree, \vbody} \evalarrow \vres \OrAbnormal\\
    \matchfuncres(\vres) \evalarrow C\\
    \newg \eqdef \ordered{\vgtwo}{\aslpo}{\vgthree}\\
}{
  \evalsubprogram{\env, \name, \actualargs, \params} \evalarrow \withgraph{C}{\newg}
}
\end{mathpar}

\subsubsection{Comments}
\lrmcomment{This is related to \identr{DFWZ}:}
It is not an error for execution of a procedure or setter to end without a
return statement.

\subsubsection{SemanticsRule.ReadValueFrom \label{sec:SemanticsRule.ReadValueFrom}}
\hypertarget{def-readvaluefrom}{}
The helper relation
\[
  \readvaluefrom(\vals \aslsep \Identifiers) \;\aslrel\; (\vals \times \XGraphs)
\]
generates an execution graph for reading the given value to a variable given
by the identifier, and pairs it with the given value.

\subsubsection{Prose}
All of the following apply:
\begin{itemize}
  \item reading the value $\vv$ into the variable named $\id$ gives $\newg$;
  \item the result is $(\vv, \newg)$.
\end{itemize}

\CodeSubsection{\EvalReadValueFromBegin}{\EvalReadValueFromEnd}{../Interpreter.ml}

\subsubsection{Formally}
\begin{mathpar}
\inferrule{
  \readidentifier(\vv, \id) \evalarrow \newg
}{
  \readvaluefrom(\vv, \id) \evalarrow (\vv, \newg)
}
\end{mathpar}

\subsubsection{SemanticsRule.WriteRetVals \label{sec:SemanticsRule.WriteRetVals}}
\hypertarget{def-writeretvals}{}
The relation
\[
  \writeretvals(\overname{(\overname{\overname{\vals}{\vv}\times\overname{\XGraphs}{\vgone}}{\vm})^*}{\vvsm}) \;\aslrel\;
  (\overname{\vals^*}{\vvs}\times\overname{\XGraphs}{\newg}) \enspace.
\]
generates Write Effects for the values
returned by the evaluation of a primitive subprogram:

\subsubsection{Prose}
  one of the following applies:
  \begin{itemize}
  \item All of the following apply (\textsc{empty}):
  \begin{itemize}
    \item the list of value-execution graphs $\vvsm$ is empty;
    \item the result is a pair consisting of an empty list and an empty graph.
  \end{itemize}

  \item All of the following apply (\textsc{non\_empty}):
  \begin{itemize}
    \item the list of value-execution graphs $\vvsm$ has $\vm$ as its head and $\vvsmone$ as its tail;
    \item $\vx$ is a fresh identifier;
    \item $\vm$ consists of the value $\vv$ and execution graph $\vgone$;
    \item the execution graph $\vgtwo$ is generating by writing the value $\vv$ for the variable named $\vx$;
    \item writing the returned values in $\vvsmone$ gives $(\vvsone, \vgthree)$;
    \item $\vs$ is defined as the list with $\vv$ as its head and $\vvsone$ as its tail;
    \item $\newg$ is defined by first taking the ordered composition of $\vgone$ and $\vgtwo$ with the $\asldata$ edge
    and then composing the resulting execution graph in parallel with $\vgthree$;
    \item the result of the entire evaluation is $(\vvs, \newg)$.
  \end{itemize}
\end{itemize}

%% Annotating the OCaml code is problematic, since the code for this rule is inside the code
%% for FPrimitive. One option is to factor it out into a separate function.
%
%   \CodeSubsection{\EvalWriteRetValsBegin}{\EvalWriteRetValsEnd}{../Interpreter.ml}

\subsubsection{Formally}
  \begin{mathpar}
    \inferrule[empty]{}
    {
      \writeretvals(\emptylist) \evalarrow (\emptylist, \emptygraph)
    }
    \and
    \inferrule[non\_empty]{
      \vvsm \eqname [\vm] \concat \vvsmone\\
      \vx \in \Identifiers \text{ is fresh}\\
      \vm \eqname (\vv, \vgone)\\
      \writeidentifier(\vx, \vv) \evalarrow \vgtwo\\
      \writeretvals(\vvsmone) \evalarrow (\vvsone, \vgthree)\\
      \vvs \eqdef [\vv] \concat \vvsone\\
      \newg \eqdef (\ordered{\vgone}{\asldata}{\vgtwo}) \parallelcomp \vgthree
    }
    {
      \writeretvals(\vvsm) \evalarrow (\vvs, \newg)
    }
  \end{mathpar}

\subsubsection{SemanticsRule.AssignArgs \label{sec:SemanticsRule.AssignArgs}}
The helper relation
\hypertarget{def-assignargs}{}
\[
  \assignargs((\overname{\envs}{\env}\times\overname{\XGraphs}{\vgone})\aslsep \overname{(\Identifiers\times\ty)^*}{\texttt{arg\_decls}}
                  \aslsep \overname{(\vals\times\XGraphs)^*} {\texttt{actual\_args}}) \;\aslrel\;
              (\overname{\envs}{\newenv} \times \overname{\XGraphs}{\newg})
\]
assigns the values of (the actual) arguments to the
formal variables of a given subprogram.

\subsubsection{Prose}
One of the following applies:
\begin{itemize}
  \item All of the following apply (\textsc{empty}):
  \begin{itemize}
    \item both $\argdecls$ and $\actualargs$ are empty lists;
    \item the result is $(\env, \vgone)$.
  \end{itemize}

  \item All of the following apply (\textsc{non\_empty}):
  \begin{itemize}
    \item $\argdecls$ has $(\vx, \Ignore)$ as its head and $\argdecls$ as its tail,
    and $\actualargs$ has $\vm$ as its head and $\actualargs$ as its tail;
    \item declaring the local identifier $\vx$ with $\vm$ in $\env$ as per \secref{SemanticsRule.DeclareLocalIdentifierMM}
    gives \\ $(\envone, \vgtwo)$.
    \item assigning the remaining lists $\argdecls$ and $\actualargs$ with the environment $\envone$ and the ordered
    composition of $\vgone$ and $\vgtwo$ with the $\aslpo$ edge gives $(\newenv, \newg)$.
    \item the entire result of the evaluation is $(\newenv, \newg)$.
  \end{itemize}
\end{itemize}

\subsubsection{Formally}
  \begin{mathpar}
    \inferrule[empty]{}
    {
      \assignargs((\env, \vgone), \emptylist, \emptylist) \evalarrow (\env, \vgone)
    }
    \and
    \inferrule[non\_empty]{
      \declarelocalidentifiermm(\env, \vx, \vm) \evalarrow (\envone, \vgtwo)\\
      \assignargs((\envone, \ordered{\vgone}{\aslpo}{\vgtwo}), \argdecls, \actualargs) \evalarrow (\newenv, \vg)
    }
    {
      \assignargs((\env, \vgone), [(\vx, \Ignore)] \concat \argdecls, [\vm] \concat \actualargs) \evalarrow (\newenv, \vg)
    }
  \end{mathpar}

\subsubsection{SemanticsRule.AssignNamedArgs \label{sec:SemanticsRule.AssignNamedArgs}}
The helper relation
\hypertarget{def-assignnamedargs}{}
\[
  \begin{array}{c}
    \assignnamedargs((\overname{\envs}{\env}\times\overname{\XGraphs}{\vgone}) \aslsep
    \overname{(\overname{\Identifiers}{\vx}\times\overname{(\vals\times\XGraphs)}{\vm})^*}{\params}) \;\aslrel\;
    (\overname{\envs}{\newenv}\times\overname{\XGraphs}{\newg})
  \end{array}
\]
assigns values to the variables that correspond
to the parameters of a given subprogram.

\subsubsection{Prose}
One of the following applies:
\begin{itemize}
  \item All of the following apply (\textsc{empty}):
  \begin{itemize}
    \item $\params$ is an empty list;
    \item the result is $\env, \vgone$;
  \end{itemize}

  \item All of the following apply (\textsc{declared}):
  \begin{itemize}
    \item $\params$ has $(\vx,\vm)$ as its head and $\paramsone$ as its tail;
    \item $\env$ consists of the static environment $\tenv$ and dynamic environment $\denv$;
    \item $\vx$ is bound to a value in $\denv$;
    \item $\vacc$ is defined as $(\env,\vgone)$;
    \item assigning the named args with $\vacc$ and $\paramsone$ gives $(\newenv, \vgtwo)$;
    \item $\newg$ is the ordered composition of $\vgone$ and $\vgtwo$ with the $\aslpo$ edge.
    \item the result is $(\newenv, \newg)$.
  \end{itemize}

  \item All of the following apply (\textsc{not\_declared}):
  \begin{itemize}
    \item $\params$ has $(\vx,\vm)$ as its head and $\paramsone$ as its tail;
    \item $\env$ consists of the static environment $\tenv$ and dynamic environment $\denv$;
    \item $\vx$ is not bound to a value in $\denv$;
    \item declaring the local identifier $\vx$ with $\vm$ in $\env$, as per \secref{SemanticsRule.DeclareLocalIdentifierM},
    gives $(\envone, \vgtwo)$;
    \item $\vacc$ is defined as $(\envone,\vgtwo)$;
    \item assigning the named args with $\vacc$ and $\paramsone$ gives $(\newenv, \vgthree)$;
    \item $\newg$ is the ordered composition of $\vgone$, $\vgtwo$, and $\vgthree$ with the $\aslpo$ edge.
    \item the result is $(\newenv, \newg)$.
  \end{itemize}
\end{itemize}

\subsubsection{Formally}
\newcommand\isbound[0]{\hyperlink{def-isbound}{\texttt{is\_bound}}}
\hypertarget{def-isbound}{}
We use the helper predicate
\[
\isbound(\denv, \vx) \triangleq G^\denv(\vx) \neq \bot \vee L^\denv(\vx) \neq \bot
\]
to test whether the variable $\vx$ is bound in the dynamic environment $\denv$.

\begin{mathpar}
  \inferrule[empty]{}
  {
    \assignnamedargs((\env, \vgone), \emptylist) \evalarrow (\env, \vgone)
  }
  \and
  \inferrule[declared]{
    \params \eqname [(\vx, \vm)] \concat \paramsone\\
    \env \eqname (\tenv, \denv)\\
    \isbound(\denv,\vx)\\
    \vacc \eqdef (\env, \vgone) \\
    \assignnamedargs(\vacc, \paramsone) \evalarrow (\newenv, \vgtwo)\\
    \newg \eqdef \ordered{\vgone}{\aslpo}{\vgtwo}
  }
  {
    \assignnamedargs((\env, \vgone), \params) \evalarrow (\newenv, \newg)
  }
  \and
  \inferrule[not\_declared]{
    \params \eqname [(\vx, \vm)] \concat \paramsone\\
    \env \eqname (\tenv, \denv)\\
    \neg\isbound(\denv,\vx)\\
    \declarelocalidentifierm(\env, \vx, \vm) \evalarrow (\envone, \vgtwo)\\
    \vacc \eqdef (\envone, \vgtwo) \\
    \assignnamedargs(\vacc, \paramsone) \evalarrow (\newenv, \vgthree)\\
    \newg \eqdef \ordered{\ordered{\vgone}{\aslpo}{\vgtwo}}{\aslpo}{\vgthree}
  }
  {
    \assignnamedargs((\env, \vgone), \params) \evalarrow (\newenv, \newg)
  }
\end{mathpar}

\subsubsection{SemanticsRule.MatchFuncRes \label{sec:SemanticsRule.MatchFuncRes}}
The helper relation
\hypertarget{def-matchfuncres}{}
\[
    \matchfuncres(\TContinuing \cup \TReturning) \;\aslrel\;
                  \Normal(((\Identifiers\times\vals)^*\times\XGraphs)\aslsep\envs)
\]
converts normal and throwing configurations
into corresponding normal configurations that can be returned by a subprogram evaluation.

\subsubsection{Prose}
One of the following applies:
\begin{itemize}
  \item All of the following apply (\textsc{continuing}):
  \begin{itemize}
    \item the given configuration is $\Continuing(\vg, \env)$. This happens when,
    for example, the subprogram called is either a setter or a procedure;
    \item the result is $\Normal((\emptylist, \vg), \env)$.
  \end{itemize}

  \item All of the following apply (\textsc{returning}):
  \begin{itemize}
    \item the given configuration is $\Returning(\xs, \retenv)$, which is the case of a function;
    \item $\xs$ is the list $\vv_i$, for $i=1..k$;
    \item define the list of fresh identifiers $\id_i$, for $i=1..k$;
    \item define $\vvs$ to be $(\vv_i, \id_i)$, for $i=1..k$;
    \item the result is $\Normal((\vvs, \emptygraph), \retenv)$.
  \end{itemize}
\end{itemize}

\subsubsection{Formally}
\begin{mathpar}
  \inferrule[continuing]{}
  {
    \matchfuncres(\Continuing(\vg, \env)) \evalarrow \Normal((\emptylist, \vg), \env)
  }
  \and
  \inferrule[returning]{
    \xs \eqname [i=1..k: \vv_i]\\
    i=1..k: \id_i \in \Identifiers\text{ is fresh}\\
    \vvs \eqdef [i=1..k: (\vv_i, \id_i)]
  }
  {
    \matchfuncres(\Returning(\xs, \retenv)) \evalarrow \Normal((\vvs, \emptygraph), \retenv)
  }
\end{mathpar}
