\chapter{Static Evaluation\label{chap:staticevaluation}}

In this chapter, we define how to statically evaluate a subset of expressions
via TypingRule.StaticEval (see \secref{TypingRule.StaticEval}) and
the primitive operations defined in \chapref{PrimitiveOperations}.
We also define the following helper rules:
\begin{itemize}
  \item TypingRule.SlicesToPositions (see \secref{TypingRule.SlicesToPositions})
  \item TypingRule.SliceToPositions (see \secref{TypingRule.SliceToPositions})
  \item TypingRule.EvalToInt (see \secref{TypingRule.EvalToInt})
  \item TypingRule.ExtractSlice (see \secref{TypingRule.ExtractSlice})
\end{itemize}

\hypertarget{def-unsupportedexpression}
In this chapter, as well as \chapref{SymbolicSubsumptionTesting} and
\chapref{SymbolicEquivalenceTesting}, we use the special value $\CannotBeTransformed$
to represent a failure in transforming an expression into a desired form (the specific
desired form varies according to the functions utilizing this value).

\hypertarget{def-staticeval}{}
\subsection{TypingRule.StaticEval \label{sec:TypingRule.StaticEval}}
The function
\[
  \staticeval(\overname{\staticenvs}{\tenv} \aslsep \overname{\expr}{\ve}) \;\aslto\;
  \overname{\literal}{\vv} \cup
  \{\CannotBeTransformed\}
  \overname{\TTypeError}{\TypeErrorConfig}
\]
evaluates an expression $\ve$, from a restricted subset of all expressions,
in the static environment $\tenv$, returning a literal $\vv$.
If $\ve$ is not in the restricted set of expressions or cannot be statically evaluated to a compile-time
constant, the result is $\CannotBeTransformed$.
\ProseOtherwiseTypeError

\subsubsection{Prose}
One of the following applies:
\begin{itemize}
  \item All of the following apply (\textsc{e\_literal}):
  \begin{itemize}
    \item $\ve$ is the literal expression for the literal $\vv$, that is, $\ELiteral(\vv)$.
  \end{itemize}

  \item All of the following apply (\textsc{e\_var\_constant}):
  \begin{itemize}
    \item $\ve$ is a variable expression with the identifier $\vx$, that is, $\EVar(\vx)$;
    \item determining whether $\vx$ is bound to a constant in $\tenv$ via $\lookupconstant$ yields the literal $\vv$.
  \end{itemize}

  \item All of the following apply (\textsc{e\_var\_non\_constant}):
  \begin{itemize}
    \item $\ve$ is a variable expression with the identifier $\vx$, that is, $\EVar(\vx)$;
    \item determining whether $\vx$ is bound to a constant in $\tenv$ via $\lookupconstant$ yields $\bot$
          (that is, $\vx$ is not bound to a constant);
    \item checking whether $\vx$ is defined in $\tenv$ yields $\True$\ProseOrTypeError;
    \item the result is $\CannotBeTransformed$.
  \end{itemize}

  \item All of the following apply (\textsc{e\_binop}):
  \begin{itemize}
    \item $\ve$ is a binary operation expression with operator $\op$ and operand expressions $\veone$ and $\vetwo$,
          that is, $\EBinop(\op, \veone, \vetwo)$;
    \item applying $\staticeval$ to $\veone$ in $\tenv$ yields the literal $\vvone$\ProseTerminateAs{\CannotBeTransformed, \TypeErrorConfig};
    \item applying $\staticeval$ to $\vetwo$ in $\tenv$ yields the literal $\vvtwo$\ProseTerminateAs{\CannotBeTransformed, \TypeErrorConfig};
    \item applying $\op$ to $\vvone$ and $\vvtwo$ via $\binopliterals$ yields $\vv$\ProseOrTypeError.
  \end{itemize}

  \item All of the following apply (\textsc{e\_unop}):
  \begin{itemize}
    \item $\ve$ is a unary operation expression with operator $\op$ and operand expression $\veone$,
          that is, $\EUnop(\op, \veone)$;
    \item applying $\staticeval$ to $\veone$ in $\tenv$ yields the literal $\vvone$\ProseTerminateAs{\CannotBeTransformed, \TypeErrorConfig};
    \item applying $\op$ to $\vvone$ via $\unopliterals$ yields $\vv$\ProseOrTypeError.
  \end{itemize}

  \item All of the following apply (\textsc{e\_slice\_int}):
  \begin{itemize}
    \item $\ve$ is a slicing expression of the integer literal for $\vi$ and slice list $\vslices$, that is,
          $\ESlice(\lint(\vi), \vslices)$;
    \item obtaining the indices of the slice list $\vslices$ in $\tenv$ via $\typingslicestopositions$
          yields $\positions$\ProseOrTypeError;
    \item $\posmax$ is the maximum index in $\positions$;
    \item converting the first $\posmax+1$ digits of the binary representation of $\vi$ into a bitvector
          via $\inttobits$ yields $\bvone$;
    \item extracting the slice of $\bvone$ given by $\positions$ yields $\bvtwo$\ProseOrTypeError;
    \item $\vv$ is the bitvector literal for $\bvtwo$.
  \end{itemize}

  \item All of the following apply (\textsc{e\_slice\_bitvector}):
  \begin{itemize}
    \item $\ve$ is a slicing expression of the bitvector literal for $\bv$ and slice list $\vslices$, that is,
          $\ESlice(\lbitvector(\bv), \vslices)$;
    \item obtaining the indices of the slice list $\vslices$ in $\tenv$ via $\typingslicestopositions$
          yields $\positions$\ProseOrTypeError;
    \item $\posmax$ is the maximum index in $\positions$;
    \item checking that the length of $\bv$ is greater than $\posmax$ (which is $0$-based) yields $\True$\ProseOrTypeError;
    \item extracting the slice of $\bv$ given by $\positions$ yields $\bvtwo$\ProseOrTypeError;
    \item $\vv$ is the bitvector literal for $\bvtwo$.
  \end{itemize}

  \item All of the following apply (\textsc{e\_slice\_type\_error}):
  \begin{itemize}
    \item $\ve$ is a slicing expression of subexpression $\veone$ and slice list $\vslices$, that is, $\ESlice(\veone, \vslices)$;
    \item $\veone$ is neither an integer literal nor a bitvector literal;
    \item the result is a type error indicating that either an integer literal or a bitvector literal were expected.
  \end{itemize}

  \item All of the following apply (\textsc{e\_cond}):
  \begin{itemize}
    \item $\ve$ is a conditional expression with condition subexpression $\econd$ and subexpressions $\veone$ (true case)
          and $\vetwo$ (false case), that is, $\ECond(\econd, \veone, \vetwo)$;
    \item evaluating $\econd$ in $\tenv$ either yields a Boolean literal $\vb$ or a type error or $\CannotBeTransformed$,
          either of which short-circuits the rule;
    \item $\vep$ is $\veone$ if $\vb$ is $\True$ and $\vetwo$ otherwise;
    \item the result is given by applying $\staticeval$ to $\vep$ in $\tenv$.
  \end{itemize}

  \item All of the following apply (\textsc{unsupported}):
  \begin{itemize}
    \item $\ve$ is an expression that is not one of the following: a literal, a variable, a binary operation expression,
          a unary operation expression, a slice expression, and a conditional expression;
    \item the result is a type error indicating that $\ve$ is not an expression that is supported
          for static evaluation.
  \end{itemize}
\end{itemize}

\subsubsection{Formally}
\begin{mathpar}
\inferrule[e\_literal]{}
{
  \staticeval(\tenv, \overname{\ELiteral(\vv)}{\ve}) \typearrow \vv
}
\end{mathpar}

\begin{mathpar}
\inferrule[e\_var\_constant]{
  \lookupconstant(\tenv, \vx) \typearrow \vv
}{
  \staticeval(\tenv, \overname{\EVar(\vx)}{\ve}) \typearrow \vv
}
\and
\inferrule[e\_var\_non\_constant]{
  \lookupconstant(\tenv, \vx) \typearrow \bot\\
  \isundefined(\tenv, \vx) \typearrow \vb\\
  \checktrans{\neg\vb}{\UndefinedIdentifier} \typearrow \True \OrTypeError
}{
  \staticeval(\tenv, \overname{\EVar(\vx)}{\ve}) \typearrow \CannotBeTransformed
}
\end{mathpar}

\begin{mathpar}
\inferrule[e\_binop]{
  \staticeval(\tenv, \veone) \typearrow \vvone \terminateas \TypeErrorConfig, \CannotBeTransformed\\\\
  \staticeval(\tenv, \vetwo) \typearrow \vvtwo \terminateas \TypeErrorConfig, \CannotBeTransformed\\\\
  \binopliterals(\op, \vvone, \vvtwo) \typearrow \vv \OrTypeError
}{
  \staticeval(\tenv, \overname{\EBinop(\op, \veone, \vetwo)}{\ve}) \typearrow \vv
}
\end{mathpar}

\begin{mathpar}
\inferrule[e\_unop]{
  \staticeval(\tenv, \veone) \typearrow \vvone \terminateas \TypeErrorConfig, \CannotBeTransformed\\\\
  \unopliterals(\op, \vvone) \typearrow \vv \OrTypeError
}{
  \staticeval(\tenv, \overname{\EUnop(\op, \veone)}{\ve}) \typearrow \vv
}
\end{mathpar}

\begin{mathpar}
\inferrule[e\_slice\_int]{
  \typingslicestopositions(\tenv, \slices) \typearrow \positions \OrTypeError\\\\
  \posmax \eqdef \max(\positions)\\
  \bvone \eqdef \inttobits(\vi, \posmax + 1)\\
  \extractslice(\bvone, \positions) \typearrow \bvtwo \OrTypeError
}{
  \staticeval(\tenv, \overname{\ESlice(\lint(\vi), \slices)}{\ve}) \typearrow \overname{\lbitvector(\bvtwo)}{\vv}
}
\end{mathpar}

\begin{mathpar}
\inferrule[e\_slice\_bitvector]{
  \typingslicestopositions(\tenv, \slices) \typearrow \positions \OrTypeError\\\\
  \posmax \eqdef \max(\positions)\\
  \checktrans{\listlen{\bv} > \posmax}{SliceOutOfRange} \checktransarrow \True \OrTypeError\\\\
  \extractslice(\bv, \positions) \typearrow \bvtwo \OrTypeError
}{
  \staticeval(\tenv, \overname{\ESlice(\lbitvector(\bv), \slices)}{\ve}) \typearrow \overname{\lbitvector(\bvtwo)}{\vv}
}
\end{mathpar}

\begin{mathpar}
\inferrule[e\_slice\_type\_error]{
  \astlabel(\veone) \not\in \{\lint, \lbitvector\}
}{
  \staticeval(\tenv, \overname{\ESlice(\veone, \slices)}{\ve}) \typearrow \TypeErrorVal{TypeMismatch}
}
\end{mathpar}

\begin{mathpar}
\inferrule[e\_cond]{
  \staticeval(\tenv, \econd) \typearrow \vcond \terminateas \TypeErrorConfig, \CannotBeTransformed\\\\
  \vcond \eqname \lbool(\vb)\\
  \vep \eqdef \choice{\vb}{\veone}{\vetwo}\\
  \staticeval(\tenv, \vep) \typearrow \vv \terminateas \TypeErrorConfig, \CannotBeTransformed
}{
  \staticeval(\tenv, \overname{\ECond(\econd, \veone, \vetwo)}{\ve}) \typearrow \vv
}
\end{mathpar}

\begin{mathpar}
\inferrule[unsupported]{
  \astlabel(\ve) \not\in \{ \ELiteral, \EVar, \EBinop, \EUnop, \ESlice, \ECond\}
}{
  \staticeval(\tenv, \ve) \typearrow \TypeErrorVal{UnsupportedExpression}
}
\end{mathpar}

\subsubsection{TypingRule.LookupConstant}
\hypertarget{def-lookupconstant}{}
The function
\[
  \lookupconstant(\overname{\staticenvs}{\tenv} \aslsep \overname{\identifier}{\vs})
  \;\aslto\; \overname{\literal}{\vv}\ \cup\ \{\bot\}
\]
looks up the environment $\tenv$ for a constant $\vv$ associated with an identifier
$\vs$. The result is $\bot$ if $\vs$ is not associated with any constant.

\subsubsection{Prose}
One of the following applies:
\begin{itemize}
  \item All of the following apply (\textsc{local}):
  \begin{itemize}
    \item $\vs$ is associated with a constant $\vv$ in the local environment of $\tenv$;
  \end{itemize}

  \item All of the following apply (\textsc{global}):
  \begin{itemize}
    \item $\vs$ is not associated with a constant in the local environment of $\tenv$;
    \item $\vs$ is associated with a constant $\vv$ in the global environment of $\tenv$;
  \end{itemize}

  \item All of the following apply (\textsc{none}):
  \begin{itemize}
    \item $\vs$ is not associated with a constant in the local environment of $\tenv$;
    \item $\vs$ is not associated with a constant in the global environment of $\tenv$;
    \item the result is $\bot$.
  \end{itemize}
\end{itemize}

\subsubsection{Formally}
\begin{mathpar}
\inferrule[local]{
  L^\tenv.\constantvalues(\vs) = \vv
}{
  \lookupconstant(\tenv, \vs) \typearrow \vv
}
\and
\inferrule[global]{
  L^\tenv.\constantvalues(\vs) = \bot\\
  G^\tenv.\constantvalues(\vs) = \vv
}{
  \lookupconstant(\tenv, \vs) \typearrow \vv
}
\and
\inferrule[none]{
  L^\tenv.\constantvalues(\vs) = \bot\\
  G^\tenv.\constantvalues(\vs) = \bot
}{
  \lookupconstant(\tenv, \vs) \typearrow \bot
}
\end{mathpar}

\subsection{TypingRule.SlicesToPositions \label{sec:TypingRule.SlicesToPositions}}
\hypertarget{def-typingslicestopositions}{}
The function
\[
  \typingslicestopositions(\overname{\staticenvs}{\tenv} \aslsep \overname{\slice^*}{\slices}) \aslto
  \overname{\Z^*}{\positions} \cup \{\CannotBeTransformed\} \cup \overname{\TTypeError}{\TypeErrorConfig}
\]
transforms the list of slices $\slices$ in $\tenv$ into a list of indices $\positions$.
The result is $\CannotBeTransformed$ if $\slices$ cannot be statically evaluated to
a list of positions.
\ProseOtherwiseTypeError

\subsubsection{Prose}
One of the following applies:
\begin{itemize}
  \item All of the following apply (\textsc{empty}):
  \begin{itemize}
    \item $\slices$ is the empty list;
    \item $\positions$ is the empty list.
  \end{itemize}

  \item All of the following apply (\textsc{non\_empty}):
  \begin{itemize}
    \item \view\ $\slices$ as the list with $\vs$ as its \head\ and $\slicesone$ as its \tail;
    \item applying $\slicetopositions$ to $\vs$ in $\tenv$ yields the list of positions \\
          $\positionsone$\ProseTerminateAs{\CannotBeTransformed, \TypeErrorConfig};
    \item transforming $\slicesone$ to a list of positions in $\tenv$ via $\typingslicestopositions$ yields
          $\positionstwo$\ProseTerminateAs{\CannotBeTransformed, \TypeErrorConfig};
    \item $\positions$ is the concatenation of $\positionsone$ and $\positionstwo$.
  \end{itemize}
\end{itemize}

\subsubsection{Formally}
\begin{mathpar}
\inferrule[empty]{}
{
  \typingslicestopositions(\tenv, \overname{\emptylist}{\slices}) \typearrow \overname{\emptylist}{\positions}
}
\and
\inferrule[non\_empty]{
  \slicetopositions(\tenv, \vs) \typearrow \positionsone \terminateas \CannotBeTransformed,\TypeErrorConfig\\\\
  \typingslicestopositions(\tenv, \slicesone) \typearrow \positionstwo \terminateas \CannotBeTransformed,\TypeErrorConfig
}{
  \typingslicestopositions(\tenv, \overname{[\vs] \concat \slicesone}{\slices}) \typearrow \overname{\positionsone \concat \positionstwo}{\positions}
}
\end{mathpar}

\subsection{TypingRule.SliceToPositions \label{sec:TypingRule.SliceToPositions}}
\hypertarget{def-slicetopositions}{}
The function
\[
  \slicetopositions(\overname{\staticenvs}{\tenv} \aslsep \overname{\slice}{\vs}) \aslto
  \overname{\Z^+}{\positions} \cup \{\CannotBeTransformed\} \cup \overname{\TTypeError}{\TypeErrorConfig}
\]
transforms a slice $\vs$ in $\tenv$ into a list of indices $\positions$.
The result is $\CannotBeTransformed$ if $\slices$ cannot be statically evaluated to
a list of positions.
\ProseOtherwiseTypeError

\subsubsection{Prose}
One of the following applies:
\begin{itemize}
  \item All of the following apply (\textsc{single}):
  \begin{itemize}
    \item $\vs$ is a slice for a single position given by the expression $\ve$, that is, \\ $\SliceSingle(\ve)$;
    \item applying $\evaltoint$ to $\ve$ in $\tenv$ yields the integer $n$\ProseOrTypeError;
    \item checking that $n$ is non-negative yields $\True$\ProseOrTypeError;
    \item $\positions$ is the list containing the single element $n$.
  \end{itemize}

  \item All of the following apply (\textsc{range}):
  \begin{itemize}
    \item $\vs$ is a slice for a range given by the expression $\etop$
          for the top position and $\ebot$ for the bottom position, that is, \\ $\SliceRange(\etop, \ebot)$;
    \item applying $\evaltoint$ to $\ebot$ in $\tenv$ yields the integer $b$\ProseOrTypeError;
    \item applying $\evaltoint$ to $\etop$ in $\tenv$ yields the integer $t$\ProseOrTypeError;
    \item checking that $t$ is greater or equal to $b$ and that $b$ is greater or equal to $0$ yields $\True$\ProseOrTypeError;
    \item $\positions$ is the list of integers from $t$ down to $b$, inclusive.
  \end{itemize}

  \item All of the following apply (\textsc{length}):
  \begin{itemize}
    \item $\vs$ is a slice for a length slice given by the expression $\ebot$
          for the bottom position and $\elength$ for the length of the slice, that is, \\ $\SliceLength(\ebot, \elength)$;
    \item applying $\evaltoint$ to $\ebot$ in $\tenv$ to an integer yields the integer $b$\ProseOrTypeError;
    \item applying $\evaltoint$ to $\elength$ in $\tenv$ to an integer yields the integer $l$\ProseOrTypeError;
    \item $t$ is $b + l - 1$;
    \item checking that $t$ is greater or equal to $b$ and that $b$ is greater or equal to $0$ yields $\True$\ProseOrTypeError;
    \item $\positions$ is the list of integers from $t$ down to $b$, inclusive.
  \end{itemize}

  \item All of the following apply (\textsc{star}):
  \begin{itemize}
    \item $\vs$ is a slice for a scaled slice given by the expression $\efactor$
          for the factor and $\elength$ for the length of the slice (\texttt{$\elength$*:$\efactor$}),
          that is, \\ $\SliceStar(\efactor, \elength)$;
    \item applying $\evaltoint$ to $\efactor$ in $\tenv$ to an integer yields the integer $f$\ProseOrTypeError;
    \item applying $\evaltoint$ to $\elength$ in $\tenv$ to an integer yields the integer $l$\ProseOrTypeError;
    \item $t$ is $(f \times l) + l - 1$;
    \item $b$ is $t - l + 1$;
    \item checking that $t$ is greater or equal to $b$ and that $b$ is greater or equal to $0$ yields $\True$\ProseOrTypeError;
    \item $\positions$ is the list of integers from $t$ down to $b$, inclusive.
  \end{itemize}
\end{itemize}

\subsubsection{Formally}

\begin{mathpar}
\inferrule[single]{
  \evaltoint(\tenv, \ve) \typearrow n \terminateas \CannotBeTransformed,\TypeErrorConfig\\\\
  \checktrans{n \geq 0}{BadSlice} \checktransarrow \True \OrTypeError
}{
  \slicetopositions(\tenv, \overname{\SliceSingle(\ve)}{\vs}) \typearrow [ n ]
}
\and
\inferrule[range]{
  \evaltoint(\tenv, \ebot) \typearrow b \terminateas \CannotBeTransformed,\TypeErrorConfig\\\\
  \evaltoint(\tenv, \etop) \typearrow t \terminateas \CannotBeTransformed,\TypeErrorConfig\\\\
  \checktrans{t \geq b \geq 0}{BadSlice} \checktransarrow \True \OrTypeError
}{
  \slicetopositions(\tenv, \overname{\SliceRange(\etop, \ebot)}{\vs}) \typearrow [ t..b ]
}
\and
\inferrule[length]{
  \evaltoint(\tenv, \ebot) \typearrow b \terminateas \CannotBeTransformed,\TypeErrorConfig\\\\
  \evaltoint(\tenv, \elength) \typearrow l \terminateas \CannotBeTransformed,\TypeErrorConfig\\\\
  t \eqdef b + l - 1\\
  \checktrans{t \geq b \geq 0}{BadSlice} \checktransarrow \True \OrTypeError
}{
  \slicetopositions(\tenv, \overname{\SliceLength(\ebot, \elength)}{\vs}) \typearrow [ t..b ]
}
\and
\inferrule[star]{
  \evaltoint(\tenv, \efactor) \typearrow f \terminateas \CannotBeTransformed,\TypeErrorConfig\\\\
  \evaltoint(\tenv, \elength) \typearrow l \terminateas \CannotBeTransformed,\TypeErrorConfig\\\\
  t \eqdef (f \times l) + l - 1\\
  b \eqdef t - l\\
  \checktrans{t \geq b \geq 0}{BadSlice} \checktransarrow \True \OrTypeError
}{
  \slicetopositions(\tenv, \overname{\SliceStar(\efactor, \elength)}{\vs}) \typearrow [ t..b ]
}
\end{mathpar}

\subsection{TypingRule.EvalToInt \label{sec:TypingRule.EvalToInt}}
\hypertarget{def-evaltoint}{}
The function
\[
\evaltoint(\overname{\staticenvs}{\tenv} \aslsep \overname{\expr}{\ve})
\aslto \overname{\Z}{n} \cup \{\CannotBeTransformed\} \cup \overname{\TTypeError}{\TypeErrorConfig}
\]
statically evaluates the expression $\ve$ to the integer $n$.
The result is $\CannotBeTransformed$ if $\ve$ cannot be statically evaluated
to an integer.
\ProseOtherwiseTypeError

\subsubsection{Prose}
All of the following apply:
\begin{itemize}
  \item applying $\staticeval$ to $\ve$ in $\tenv$ has one of three outcomes:
  \begin{itemize}
  \item a literal $\vl$, which satisfies the premise;
  \item $\CannotBeTransformed$ (which means the expression could not be evaluated to a literal),
  which short-circuits the entire rule;
  or
  \item $\TypeErrorConfig$ (indicating a type error was detected), which short-circuits the entire rule;
  \end{itemize}
  \item checking that $\vl$ is an integer literal yields $\True$\ProseOrTypeError;
  \item define $\vl$ as the literal integer for $n$.
\end{itemize}

\subsubsection{Formally}
\begin{mathpar}
\inferrule{
  \staticeval(\tenv, \ve) \typearrow \vl \terminateas \CannotBeTransformed, \TypeErrorConfig\\\\
  \checktrans{\astlabel(\vl) = \lint}{ExpectedIntegerType} \checktransarrow \True \OrTypeError\\\\
  \vl \eqname \lint(n)
}{
  \evaltoint(\tenv, \ve) \typearrow n
}
\end{mathpar}

\subsection{TypingRule.ExtractSlice \label{sec:TypingRule.ExtractSlice}}
\hypertarget{def-extractslice}{}
The function
\[
  \extractslice(\overname{\{0,1\}^*}{\bits} \aslsep \overname{\Z^*}{\positions}) \aslto
  \overname{\{0,1\}^*}{\vr} \cup\ \TTypeError
\]
extracts from the list of bits $\bits$ the sublist $\vr$ of bits appearing at the positions given by $\positions$.
\ProseOtherwiseTypeError

\subsubsection{Prose}
Define $\vr$ the sublist of $\bits$ given by taking $\bits[\vi]$, for every value given in $\positions$ (in the order they appear).

\subsubsection{Formally}
\begin{mathpar}
\inferrule{}{
  \extractslice(\tenv, \bits, \positions) \typearrow \overname{[ \vi\in\positions: \bits[\vi] ]}{\vr}
}
\end{mathpar}