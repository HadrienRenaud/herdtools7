\chapter{Bitvector Slicing\label{chap:BitvectorSlicing}}
\hypertarget{bitsliceterm}{}
\ExampleDef{Well-typed Bitvector Slices}
\listingref{BitvectorSlices} shows different ways of expressing bitvector slices
and how they relate to one another in terms of the order of bits they represent.
\ASLListing{Examples of bitvector slices and operations on slices}{BitvectorSlices}{\definitiontests/Bitvector_slices.asl}

\section{A List of Slices}
A list of bitvector slices is grammatically derived from $\Nslices$ and the AST
is given by a list of $\slice$ AST nodes.
%
The function $\buildslices$ builds the AST for a list of slices.
%
The function $\annotateslices$ (see \TypingRuleRef{Slices}) annotates a list of slices.
%
The relation $\evalslices$ (see \SemanticsRuleRef{Slices}) evaluates a list of slices.

\subsection{Syntax}
\begin{flalign*}
\Nslices \derives \ & \Tlbracket \parsesep \ClistZero{\Nslice} \parsesep \Trbracket &
\end{flalign*}

\subsection{Abstract Syntax}
\ASTRuleDef{Slices}
\hypertarget{build-slices}{}
The function
\[
  \buildslices(\overname{\parsenode{\Nslices}}{\vparsednode}) \;\aslto\; \overname{\slice^+}{\vastnode}
\]
transforms a parse node for a list of slices $\vparsednode$ into an AST node for a list of slices $\vastnode$.

\begin{mathpar}
\inferrule{
  \buildclist[\buildslice](\vslices) \astarrow \vsliceasts
}{
  \buildslices(\Nslices(\Tlbracket, \namednode{\vslices}{\ClistOne{\Nslice}}, \Trbracket)) \astarrow
  \overname{\vsliceasts}{\vastnode}
}
\end{mathpar}

\subsection{Typing}
\TypingRuleDef{Slices}
\hypertarget{def-annotateslices}{}
The function
\[
\annotateslices(\overname{\staticenvs}{\tenv}, \overname{\slice^*}{\vslices}) \aslto
(\overname{\slice^*}{\slicesp} \times \overname{\TSideEffectSet}{\vses})
\]
annotates a list of slices $\vslices$ in the static environment $\tenv$, yielding a list of annotated slices (that is,
slices in the \typedast) and \sideeffectsetterm\ $\vses$.
\ProseOtherwiseTypeError

See \ExampleRef{Well-typed Bitvector Slices}.

\ProseParagraph
\AllApply
\begin{itemize}
  \item annotating the slice $\vslices[\vi]$ in $\tenv$, for each $\vi\in\listrange(\vslices)$, yields \\
        $(\vs_\vi, \vxs_i)$\ProseOrTypeError;
  \item define $\slicesp$ as the list of slices $\vs_\vi$, for each $\vi\in\listrange(\vslices)$;
  \item define $\vses$ as the union of all $\vxs_i$, for every \Proselistrange{$i$}{$\vslices$}.
\end{itemize}
\FormallyParagraph
\begin{mathpar}
\inferrule{
  i\in\listrange(\vslices): \annotateslice(\tenv, \vslices[i]) \typearrow (\vs_i, \vxs_i) \OrTypeError\\\\
  \slicesp \eqdef [i\in\listrange(\vslices): \vs_i]\\
  \vses \eqdef \bigcup_{i\in\listrange(\vslices)} \vxs_i
}{
  \annotateslices(\tenv, \vslices) \typearrow (\slicesp, \vses)
}
\end{mathpar}

\subsection{Semantics}
\SemanticsRuleDef{Slices}
The relation
\hypertarget{def-evalslices}{}
\[
  \begin{array}{rl}
  \evalslices(\overname{\envs}{\env} \aslsep \overname{\slice^*}{\slices}) \;\aslrel &
    \Normal((\overname{(\vals \times \vals)^*}{\ranges} \times \overname{\XGraphs}{\newg}), \overname{\envs}{\newenv})\ \cup \\
    & \overname{\Throwing}{\ThrowingConfig} \cup \overname{\TDynError}{\DynErrorConfig}
  \end{array}
\]
evaluates a list of slices $\slices$ in an environment $\env$, resulting in either \\
$\Normal((\ranges, \newg), \newenv)$ or an abnormal configuration.
\ProseParagraph
\OneApplies
\begin{itemize}
  \item \AllApplyCase{empty}
  \begin{itemize}
    \item the list of slices is empty;
    \item $\ranges$ is the empty list;
    \item $\newg$ is the empty graph;
    \item $\newenv$ is $\env$;
  \end{itemize}

  \item \AllApplyCase{nonempty}
  \begin{itemize}
    \item the list of slices has $\slice$ as the head and $\slicesone$ as the tail;
    \item evaluating the slice $\vslice$ in $\env$ results in \\
    $\Normal((\range, \vgone), \envone)$\ProseOrAbnormal;
    \item evaluating the tail list $\slicesone$ in $\envone$ results in \\
    $\Normal((\rangesone, \vgtwo), \newenv)$\ProseOrAbnormal;
    \item $\ranges$ is the concatenation of $\range$ to $\rangesone$;
    \item $\newg$ is the parallel composition of $\vgone$ and $\vgtwo$.
  \end{itemize}
\end{itemize}

$\evalslices(\env, \vslices)$ is the list of pairs \texttt{(start\_n, length\_n)} that
correspond to the start (included) and the length of each slice in
$\slices$.

\FormallyParagraph
\begin{mathpar}
\inferrule[empty]{}
{
  \evalslices(\env, \emptylist) \evalarrow \Normal((\emptylist, \emptygraph), \env)
}
\end{mathpar}
\begin{mathpar}
\inferrule[nonempty]{
  \slices \eqname [\vslice] \concat \slicesone\\
  \evalslice(\env, \vslice) \evalarrow \Normal((\range, \vgone), \envone) \OrAbnormal\\
  \evalslices(\envone, \slicesone) \evalarrow \Normal((\rangesone, \vgtwo), \newenv) \OrAbnormal\\
  \ranges \eqdef [\range] \concat \rangesone\\
  \newg \eqdef \vgone \parallelcomp \vgtwo
}{
  \evalslices(\env, \slices) \evalarrow \Normal((\ranges, \newg), \newenv)
}
\end{mathpar}
\CodeSubsection{\EvalSlicesBegin}{\EvalSlicesEnd}{../Interpreter.ml}

\section{Slicing Constructs}
An individual slice construct is grammatically derived from $\Nslice$ and
represented as an AST by $\slice$.
%
The function $\buildslice$ (see \ASTRuleRef{Slice}) builds the AST for an individual slice construct.
%
the function $\annotateslice$ (see \TypingRuleRef{Slice}) annotates a single slice.

\subsection{Syntax}
\begin{flalign*}
\Nslice \derives \ & \Nexpr &\\
            |\  & \Nexpr \parsesep \Tcolon \parsesep \Nexpr &\\
            |\  & \Nexpr \parsesep \Tpluscolon \parsesep \Nexpr &\\
            |\  & \Nexpr \parsesep \Tstarcolon \parsesep \Nexpr &\\
            |\  & \Tcolon \parsesep \Nexpr &
\end{flalign*}

\subsection{Abstract Syntax}
\begin{flalign*}
\slice \derives\ & \SliceSingle(\overname{\expr}{\vi}) &\\
  |\ & \SliceRange(\overname{\expr}{\vj}, \overname{\expr}{\vi}) &\\
  |\ & \SliceLength(\overname{\expr}{\vi}, \overname{\expr}{\vn}) &\\
  |\ & \SliceStar(\overname{\expr}{\vi}, \overname{\expr}{\vn}) &
\end{flalign*}

Note that the syntax \texttt{[:width]} is a shorthand for \texttt{x[width-1:0]}.

\ASTRuleDef{Slice}
\hypertarget{build-slice}{}
The function
\[
  \buildslice(\overname{\parsenode{\Nslice}}{\vparsednode}) \;\aslto\; \overname{\slice}{\vastnode}
\]
transforms a parse node for a slice $\vparsednode$ into an AST node for a slice $\vastnode$.

\begin{mathpar}
\inferrule[single]{}{
  \buildslices(\Nslice(\punnode{\Nexpr})) \astarrow
  \overname{\SliceSingle(\astof{\vexpr})}{\vastnode}
}
\end{mathpar}

\begin{mathpar}
\inferrule[range]{
  \buildexpr(\veone) \astarrow \astversion{\veone}\\
  \buildexpr(\vetwo) \astarrow \astversion{\vetwo}
}{
  \buildslices(\Nslice(\namednode{\veone}{\Nexpr}, \Tcolon, \namednode{\vetwo}{\Nexpr})) \astarrow
  \overname{\SliceRange(\astversion{\veone}, \astversion{\vetwo})}{\vastnode}
}
\end{mathpar}

\begin{mathpar}
\inferrule[length]{
  \buildexpr(\veone) \astarrow \astversion{\veone}\\
  \buildexpr(\vetwo) \astarrow \astversion{\vetwo}
}{
  \buildslices(\Nslice(\namednode{\veone}{\Nexpr}, \Tpluscolon, \namednode{\vetwo}{\Nexpr})) \astarrow
  \overname{\SliceLength(\astversion{\veone}, \astversion{\vetwo})}{\vastnode}
}
\end{mathpar}

\begin{mathpar}
\inferrule[scaled]{
  \buildexpr(\veone) \astarrow \astversion{\veone}\\
  \buildexpr(\vetwo) \astarrow \astversion{\vetwo}
}{
  \buildslices(\Nslice(\namednode{\veone}{\Nexpr}, \Tstarcolon, \namednode{\vetwo}{\Nexpr})) \astarrow
  \overname{\SliceStar(\astversion{\veone}, \astversion{\vetwo})}{\vastnode}
}
\end{mathpar}

\begin{mathpar}
\inferrule[width]{}{
  \buildslices(\Nslice(\Tcolon, \punnode{\Nexpr})) \astarrow
  \overname{\SliceLength(\ELiteral(\lint(0)), \astof{\vexpr})}{\vastnode}
}
\end{mathpar}

\subsection{Typing}
\TypingRuleDef{Slice}
\hypertarget{def-annotateslice}{}
the function
\[
  \annotateslice(\overname{\staticenvs}{\tenv} \aslsep \overname{\slice}{\vs})
  \aslto
  \overname{\slice}{\vsp} \cup \overname{\TTypeError}{\TypeErrorConfig}
\]
annotates a single slice $\vs$ in the static environment $\tenv$,
resulting in an annotated slice $\vsp$.
\ProseOtherwiseTypeError
\ProseParagraph
\OneApplies
\begin{itemize}
  \item \AllApplyCase{single}
  \begin{itemize}
    \item $\vs$ is a \singleslice\ at index \vi, that is $\SliceSingle(\vi)$;
    \item annotating the slice at offset $\vi$ of length $1$ yields $\vsp$\ProseOrTypeError.
  \end{itemize}

  \item \AllApplyCase{range}
  \begin{itemize}
    \item $\vs$ is a slice for the range \texttt{(j, i)}, that is $\SliceRange(\vj, \vi)$;
    \item $\prelength$ is \texttt{i+:(j-i+1)};
    \item annotating the slice at offset $\vi$ of length $\prelength$ yields $\vsp$\ProseOrTypeError.
  \end{itemize}

  \item \AllApplyCase{length}
  \begin{itemize}
    \item $\vs$ is a \lengthslice\ of length $\elength$ and offset $\eoffset$, that is, \\
          $\SliceLength(\eoffset, \elength)$;
    \item annotating the expression $\eoffset$ in $\tenv$ yields \\
          $(\toffset, \eoffsetp, \vsesoffset)$\ProseOrTypeError;
    \item annotating the \symbolicallyevaluable\ \constrainedinteger\ expression $\elength$ in $\tenv$ yields
    $(\elengthp, \vseslength)$\ProseOrTypeError;
    \item determining whether $\toffset$ has the \structureofinteger\ yields $\True$\ProseOrTypeError;
    \item $\vsp$ is the slice at offset $\eoffsetp$ and length $\elength'$, that is,\\
     $\SliceLength(\eoffsetp, \elength')$;
     \item define $\vses$ as the union of $\vsesoffset$ and $\vseslength$.
  \end{itemize}

  \item \AllApplyCase{scaled}
  \begin{itemize}
    \item $\vs$ is a \scaledslice\ \texttt{[factor *: pre\_length]}, that is, \\
          $\SliceStar(\factor, \prelength)$;
    \item $\preoffset$ is $\factor * \prelength$;
    \item annotating the slice at offset $\preoffset$ of length $\prelength$ yields $\vsp$\ProseOrTypeError.
  \end{itemize}
\end{itemize}

\FormallyParagraph
\begin{mathpar}
\inferrule[single]{
  \annotateslice(\SliceLength(\vi, \eliteral{1})) \typearrow \vsp \OrTypeError
}{
  \annotateslice(\tenv, \overname{\SliceSingle(\vi)}{\vs}) \typearrow \vsp
}
\end{mathpar}

\begin{mathpar}
\inferrule[range]{
  \binopliterals(\MINUS, \vj, \vi) \typearrow \prelengthp\\
  \binopliterals(\PLUS, \prelengthp, \eliteral{1}) \typearrow \prelength\\
  \annotateslice(\SliceLength(\vi, \prelength)) \typearrow \vsp \OrTypeError
}{
  \annotateslice(\tenv, \overname{\SliceRange(\vj, \vi)}{\vs}) \typearrow \vsp
}
\end{mathpar}

\begin{mathpar}
\inferrule[length]{
  \annotateexpr{\tenv, \eoffset} \typearrow (\toffset, \eoffsetp, \vsesoffset) \OrTypeError\\\\
  \annotatesymbolicconstrainedinteger(\tenv, \elength) \typearrow (\elengthp, \vseslength) \OrTypeError\\\\
  \checkunderlyinginteger(\tenv, \toffset) \typearrow \True \OrTypeError\\\\
  \vses \eqdef \vsesoffset \cup \vseslength
}{
  {
    \begin{array}{r}
  \annotateslice(\tenv, \overname{\SliceLength(\eoffset, \elength)}{\vs}) \typearrow \\
    (\overname{\SliceLength(\eoffsetp, \elength')}{\vsp}, \vses)
    \end{array}
  }
}
\end{mathpar}

\begin{mathpar}
\inferrule[scaled]{
  \binopliterals(\MUL, \factor, \prelength) \typearrow \preoffset\\
  \annotateslice(\SliceLength(\preoffset, \prelength)) \typearrow \vsp \OrTypeError
}{
  \annotateslice(\tenv, \overname{\SliceStar(\factor, \prelength)}{\vs}) \typearrow \vsp
}
\end{mathpar}
\lrmcomment{\identr{GXKG}: The notation \texttt{b[j:i]} is syntactic sugar for \texttt{b[i+:(j-i+1)]}.}
\lrmcomment{\identr{GXKG}: The notation \texttt{b[i]} is syntactic sugar for \texttt{b[i +: 1]}.}
\lrmcomment{\identr{GXQG}: The notation \texttt{b[i *: n]} is syntactic sugar for \texttt{b[i*n +: n]}}

\TypingRuleDef{SlicesWidth}
\hypertarget{def-sliceswidth}{}
The helper function
\[
  \sliceswidth(\overname{\staticenvs}{\tenv} \aslsep \overname{\slice^*}{\vslices}) \aslto
  \overname{\expr}{\vwidth} \cup \overname{\TTypeError}{\TypeErrorConfig}
\]
returns an expression $\vslices$ that represents the width of all slices given by $\vslices$
in the static environment $\tenv$.

\ProseParagraph
\OneApplies
\begin{itemize}
  \item \AllApplyCase{empty}
  \begin{itemize}
    \item $\vslices$ is the empty list;
    \item $\vwidth$ is the literal integer expression for $0$.
  \end{itemize}

  \item \AllApplyCase{non\_empty}
  \begin{itemize}
    \item $\vslices$ is the list with \head\ $\vs$ and \tail\ $\slicesone$;
    \item applying $\slicewidth$ to $\vs$ yields $\veone$;
    \item applying $\sliceswidth$ to $\slicesone$ yields $\vetwo$;
    \item symbolically simplifying the binary expression summing $\veone$ with $\vetwo$ yields $\vwidth$\ProseOrTypeError.
  \end{itemize}
\end{itemize}
\FormallyParagraph
\begin{mathpar}
\inferrule[empty]{}{
  \sliceswidth(\tenv, \overname{\emptylist}{\vslices}) \typearrow \overname{\ELInt{0}}{\vwidth}
}
\end{mathpar}

\begin{mathpar}
\inferrule[non\_empty]{
  \slicewidth(\vs) \typearrow \veone\\
  \sliceswidth(\slicesone) \typearrow \vetwo\\
  \normalize(\AbbrevEBinop{\PLUS}{\veone}{\vetwo}) \typearrow \vwidth \OrTypeError
}{
  \sliceswidth(\tenv, \overname{[\vs]\concat\slicesone}{\vslices}) \typearrow \vwidth
}
\end{mathpar}

\TypingRuleDef{SliceWidth}
\hypertarget{def-slicewidth}{}
The helper function
\[
  \slicewidth(\overname{\slice}{\vslice}) \aslto
  \overname{\expr}{\vwidth}
\]
returns an expression $\vwidth$ that represents the width of the slices given by $\vslice$.

\ProseParagraph
\OneApplies
\begin{itemize}
  \item \AllApplyCase{single}
  \begin{itemize}
    \item $\vslice$ is a single slice, that is, $\SliceSingle(\Ignore)$;
    \item $\vwidth$ is the literal integer expression for $1$;
  \end{itemize}

  \item \AllApplyCase{scaled, length}
  \begin{itemize}
    \item $\vslice$ is either a slice of the form \texttt{\_*:$\ve$} or \texttt{\_+:$\ve$};
    \item $\vwidth$ is $\ve$;
  \end{itemize}

  \item \AllApplyCase{range}
  \begin{itemize}
    \item $\vslice$ is a slice of the form \texttt{$\veone$:$\vetwo$};
    \item $\vwidth$ is the expression for $1 + (\veone - \vetwo)$.
  \end{itemize}
\end{itemize}
\FormallyParagraph
\begin{mathpar}
\inferrule[single]{}{
  \slicewidth(\overname{\SliceSingle(\Ignore)}{\vslice}) \typearrow \overname{\ELInt{1}}{\vwidth}
}
\and
\inferrule[scaled]{}{
  \slicewidth(\overname{\SliceStar(\Ignore, \ve)}{\vslice}) \typearrow \overname{\ve}{\vwidth}
}
\and
\inferrule[length]{}{
  \slicewidth(\overname{\SliceLength(\Ignore, \ve)}{\vslices}) \typearrow \overname{\ve}{\vwidth}
}
\and
\inferrule[range]{}{
  \slicewidth(\overname{\SliceRange(\veone, \vetwo)}{\vslices}) \typearrow
  \overname{\AbbrevEBinop{\PLUS}{\ELInt{1}}{(\AbbrevEBinop{\MINUS}{\veone}{\vetwo})}}{\vwidth}
}
\end{mathpar}

\TypingRuleDef{SymbolicConstrainedInteger}
\hypertarget{def-annotatesymbolicconstrainedinteger}{}
The function
\[
  \annotatesymbolicconstrainedinteger(\overname{\staticenvs}{\tenv} \aslsep \overname{\expr}{\ve}) \aslto
  (\overname{\expr}{\vepp} \times \overname{\TSideEffectSet}{\vses}) \cup \overname{\TTypeError}{\TypeErrorConfig}
\]
annotates a \symbolicallyevaluable\ integer expression $\ve$ of a constrained integer type in the static environment $\tenv$
and returns the annotated expression $\vepp$ and \sideeffectsetterm\ $\vses$.
\ProseOtherwiseTypeError

\ProseParagraph
\AllApply
\begin{itemize}
  \item \Proseannotatesymbolicallyevaluableexpr{$\tenv$}{$\ve$}{$(\vt, \vep, \vses)$\ProseOrTypeError};
  \item determining whether $\vt$ is a symbolically \constrainedinteger\ in $\tenv$ yields $\True$\ProseOrTypeError;
  \item determining whether $\vep$ is \symbolicallyevaluable\  in $\tenv$ yields $\True$\ProseOrTypeError;
  \item applying $\normalize$ to $\vep$ in $\tenv$ yields $\vepp$.
\end{itemize}
\FormallyParagraph
\begin{mathpar}
\inferrule{
  \annotatesymbolicallyevaluableexpr(\tenv, \ve) \typearrow (\vt, \vep, \vses) \OrTypeError\\\\
  \checkconstrainedinteger(\tenv, \vt) \typearrow \True \OrTypeError\\\\
  \checksymbolicallyevaluable(\tenv, \vep) \typearrow \True \OrTypeError\\\\
  \normalize(\tenv, \vep) \typearrow \vepp
}{
  \annotatesymbolicconstrainedinteger(\tenv, \ve) \typearrow (\vepp, \vses)
}
\end{mathpar}
\CodeSubsection{\SymbolicConstrainedIntegerBegin}{\SymbolicConstrainedIntegerEnd}{../Typing.ml}

\subsection{Semantics}
\SemanticsRuleDef{Slice}
The relation
\hypertarget{def-evalslice}{}
\[
  \begin{array}{rl}
  \evalslice(\overname{\envs}{\env} \aslsep \overname{\slice}{\vs}) \;\aslrel &
    \Normal(((\overname{\tint}{\vstart} \times \overname{\tint}{\vlength}) \times \overname{\XGraphs}{\newg}), \overname{\envs}{\newenv})\ \cup \\
    & \overname{\Throwing}{\ThrowingConfig} \cup \overname{\TDynError}{\DynErrorConfig}
  \end{array}
\]
evaluates an individual slice $\vs$ in an environment $\env$ is,
resulting either in \\
$\Normal(((\vstart, \vlength), \vg), \newenv)$, a throwing configuration, or a dynamic error configuration.

\ExampleDef{Single Slice}
In \listingref{semantics-slicesingle},
the slice \texttt{[2]} evaluates to \texttt{(2, 1)}, that is, the slice of
length 1 starting at index 2.
\ASLListing{A single expression slice}{semantics-slicesingle}{\semanticstests/SemanticsRule.SliceSingle.asl}

\ExampleDef{Range Slice}
In \listingref{semantics-slicerange}, \texttt{4:2} evaluates to \texttt{(2, 3)}.
\ASLListing{A range slice}{semantics-slicerange}{\semanticstests/SemanticsRule.SliceRange.asl}

\ExampleDef{Length Slice}
In \listingref{semantics-slicelength}, \texttt{2+:3} evaluates to \texttt{(2, 3)}.
\ASLListing{A slice by length}{semantics-slicelength}{\semanticstests/SemanticsRule.SliceLength.asl}

\ExampleDef{Scaled Slice}
In \listingref{semantics-scaledslice},
\texttt{x[3*:2]} evaluates to \texttt{'11'}.
\ASLListing{A scaled slice}{semantics-scaledslice}{\semanticstests/SemanticsRule.SliceStar.asl}

\ProseParagraph
\OneApplies
\begin{itemize}
  \item \AllApplyCase{single}
  \begin{itemize}
    \item $\vs$ is a \singleslice\ with the expression $\ve$, $\SliceSingle(\ve)$;
    \item evaluating $\ve$ in $\env$ results in $\Normal((\vstart, \newg), \newenv)$\ProseOrAbnormal;
    \item $\vlength$ is the integer value 1.
  \end{itemize}

  \item \AllApplyCase{range}
  \begin{itemize}
    \item $\vs$ is the \rangeslice\ between the
      expressions $\estart$ and $\etop$, that is, \\ $\SliceRange(\etop, \estart)$;
    \item evaluating $\etop$ in $\env$ is $\Normal(\mtop, \envone)$\ProseOrAbnormal;
    \item $\mtop$ is a pair consisting of the native integer $\vvsubtop$ and execution graph $\vgone$;
    \item evaluating $\estart$ in $\envone$ is $\Normal(\mstart, \newenv)$\ProseOrAbnormal;
    \item $\mstart$ is a pair consisting of the native integer $\vstart$ and execution graph $\vgtwo$;
    \item $\vlength$ is the integer value \texttt{(v\_top - v\_start) + 1};
    \item $\newg$ is the parallel composition of $\vgone$ and $\vgtwo$.
  \end{itemize}

  \item \AllApplyCase{length}
  \begin{itemize}
    \item $\vs$ is the \lengthslice, which starts at expression~$\estart$ with length~$\elength$,
    that is, $\SliceLength(\estart, \elength)$;
    \item evaluating $\estart$ in $\env$ is \Normal(\mstart, \envone)\ProseOrAbnormal;
    \item evaluating $\elength$ in $\envone$ is \Normal(\mlength, \newenv)\ProseOrAbnormal;
    \item $\mstart$ is a pair consisting of the native integer $\vstart$ and execution graph $\vgone$;
    \item $\mlength$ is a pair consisting of the native integer $\vlength$ and execution graph $\vgtwo$;
    \item $\newg$ is the parallel composition of $\vgone$ and $\vgtwo$.
  \end{itemize}

  \item \AllApplyCase{scaled}
  \begin{itemize}
    \item $\vs$ is the \scaledslice\ with factor given by the
      expression $\efactor$ and length given by the
      expression $\elength$, that is, $\SliceStar(\efactor, \elength)$;
    \item evaluating $\efactor$ in $\env$ is $\Normal(\mfactor, \envone)$\ProseOrAbnormal;
    \item $\mfactor$ is a pair consisting of the native integer $\vfactor$ and execution graph $\vgone$;
    \item evaluating $\elength$ in $\env$ is $\Normal(\mlength, \newenv)$\ProseOrAbnormal;
    \item $\mlength$ is a pair consisting of the native integer $\vlength$ and execution graph $\vgtwo$;
    \item $\vstart$ is the native integer $\vfactor \times \vlength$;
    \item $\newg$ is the parallel composition of $\vgone$ and $\vgtwo$.
  \end{itemize}
\end{itemize}

\FormallyParagraph
\begin{mathpar}
\inferrule[single]{
  \evalexpr{\env, \ve} \evalarrow \Normal((\vstart, \newg), \newenv) \OrAbnormal\\
  \vlength \eqdef \nvint(1)
}{
  \evalslice(\env, \SliceSingle(\ve)) \evalarrow \Normal(((\vstart, \vlength), \newg), \newenv)
}
\end{mathpar}

\begin{mathpar}
\inferrule[range]{
  \evalexpr{\env, \etop} \evalarrow \Normal(\mtop, \envone) \OrAbnormal\\\\
  \mtop \eqname (\vvsubtop, \vgone)\\
  \evalexpr{\envone, \estart} \evalarrow \Normal(\mstart, \newenv) \OrAbnormal\\\\
  \mstart \eqname (\vstart, \vgtwo)\\
  \binoprel(\MINUS, \vvsubtop, \vstart) \evalarrow \vdiff\\
  \binoprel(\PLUS, \nvint(1), \vdiff) \evalarrow \vlength\\
  \newg \eqdef \vgone \parallelcomp \vgtwo
}{
  \evalslice(\env, \SliceRange(\etop, \estart)) \evalarrow \\ \Normal(((\vstart, \vlength), \newg), \newenv)
}
\end{mathpar}

\begin{mathpar}
\inferrule[length]{
  \evalexpr{\env, \estart} \evalarrow \Normal(\mstart, \envone) \OrAbnormal\\
  \evalexpr{\envone, \elength} \evalarrow \Normal(\mlength, \newenv) \OrAbnormal\\
  \mstart \eqname (\vstart, \vgone)\\
  \mlength \eqname (\vlength, \vgtwo)\\
  \newg \eqdef \vgone \parallelcomp \vgtwo
}{
  \evalslice(\env, \SliceLength(\estart, \elength)) \evalarrow \\ \Normal(((\vstart, \vlength), \newg), \newenv)
}
\end{mathpar}

\begin{mathpar}
\inferrule[scaled]{
  \evalexpr{\env, \efactor} \evalarrow \Normal(\mfactor, \envone) \OrAbnormal\\
  \mfactor \eqname (\vfactor, \vgone)\\
  \evalexpr{\envone, \elength} \evalarrow \Normal(\mlength, \newenv) \OrAbnormal\\
  \mlength \eqname (\vlength, \vgtwo)\\
  \binoprel(\MUL, \vfactor, \vlength) \evalarrow \vstart \\
  \newg \eqdef \vgone \parallelcomp \vgtwo
}{
  \evalslice(\env, \SliceStar(\efactor, \elength)) \evalarrow \\ \Normal(((\vstart, \vlength), \newg), \newenv)
}
\end{mathpar}
