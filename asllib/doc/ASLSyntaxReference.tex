\documentclass{book}
\usepackage{amsmath}  % Classic math package
\usepackage{amssymb}  % Classic math package
\usepackage{mathtools}  % Additional math package
\usepackage{amssymb}  % Classic math package
\usepackage{mathtools}  % Additional math package
\usepackage{graphicx}  % For figures
\usepackage{caption}  % For figure captions
\usepackage{subcaption}  % For subfigure captions
\usepackage{url}  % Automatically escapes urls
\usepackage{hyperref}  % Insert links inside pdfs
\hypersetup{
    colorlinks=true,
    linkcolor=blue,
    filecolor=magenta,
    urlcolor=cyan,
}
\makeatletter
\makeatletter
  \newcommand{\linkdest}[1]{\Hy@raisedlink{\hypertarget{#1}{}}}
% \newcommand{\linkdest}[1]{\Hy@raisedlink{\hypertarget{#1}{}}}
% \newcommand{\pac}[2]{\hyperlink{{#1}1}{\Hy@raisedlink{\hypertarget{{#1}0}{}}{#2}}}
% \newcommand{\jac}[1]{\Hy@raisedlink{\hypertarget{{#1}1}{}}{\hyperlink{{#1}0}{\ac{#1}}}}
\makeatother
\usepackage[inline]{enumitem}  % For inline lists
\usepackage[export]{adjustbox}  % For centering too wide figures
\usepackage[export]{adjustbox}  % For centering too wide figures
\usepackage{mathpartir}  % For deduction rules and equations paragraphs
\usepackage{comment}
\usepackage{fancyvrb}
\usepackage[
  % Even pages have notes on the left, odd on the right
  twoside,
  % Notes on the right, should be less than outer
  marginparwidth=100pt,
  % margins
  top=4.5cm, bottom=4.5cm, inner=3.5cm, outer=4.5cm
  % To visualize:
  % showframe
]{geometry}
%\usepackage{stmaryrd} % for \llbracket and \rrbracket
\input{ifempty}
\input{ifformal}
\input{ifcode}
%% Should be functional
\ifempty
\newcommand{\isempty}[1]{#1}
\else
\newcommand{\isempty}[1]{}
\fi

%%%Safety net
\iffalse
\makeatletter
\newcommand{\isempty}[1]{#1}
\makeatother
\fi


\newcommand\herd[0]{\texttt{herd7}}

\fvset{fontsize=\small}

\newcommand\tododefine[1]{\hyperlink{tododefine}{\color{red}{\texttt{#1}}}}
\newcommand\lrmcomment[1]{}

\usepackage{enumitem}
\renewlist{itemize}{itemize}{20}
\setlist[itemize,1]{label=\textbullet}
\setlist[itemize,2]{label=\textasteriskcentered}
\setlist[itemize,3]{label=\textendash}
\setlist[itemize,4]{label=$\triangleright$}
\setlist[itemize,5]{label=+}
\setlist[itemize,6]{label=\textbullet}
\setlist[itemize,7]{label=\textasteriskcentered}
\setlist[itemize,8]{label=\textendash}
\setlist[itemize,9]{label=$\triangleright$}
\setlist[itemize,10]{label=+}
\setlist[itemize,11]{label=\textbullet}
\setlist[itemize,12]{label=\textasteriskcentered}
\setlist[itemize,13]{label=\textendash}
\setlist[itemize,14]{label=\textendash}
\setlist[itemize,15]{label=$\triangleright$}
\setlist[itemize,16]{label=+}
\setlist[itemize,17]{label=\textbullet}
\setlist[itemize,18]{label=\textasteriskcentered}
\setlist[itemize,19]{label=\textendash}
\setlist[itemize,20]{label=\textendash}

\ifcode
% First argument is \<rule>Begin, second is \<rule>End, third is the file name.
% Example: for SemanticsRule.Lit, use the following:
% \CodeSubsection{\LitBegin}{\LitEnd}{../Interpreter.ml}
\newcommand\CodeSubsection[3]{\subsection{Code} \VerbatimInput[firstline=#1, lastline=#2]{#3}}
\else
\newcommand\CodeSubsection[3]{}
\fi

\ifcode
% First argument is \<rule>Begin, second is \<rule>End, third is the file name.
% Example: for SemanticsRule.Lit, use the following:
% \CodeSubsubsection{\LitBegin}{\LitEnd}{../Interpreter.ml}
\newcommand\CodeSubsubsection[3]{\subsubsection{Code} \VerbatimInput[firstline=#1, lastline=#2]{#3}}
\else
\newcommand\CodeSubsubsection[3]{}
\fi

%%%%%%%%%%%%%%%%%%%%%%%%%%%%%%%%%%%%%%%%%%%%%%%%%%
% Typesetting macros
\newtheorem{definition}{Definition}
\newcommand\defref[1]{Def.~\ref{def:#1}}
\newcommand\secref[1]{Section~\ref{sec:#1}}
\newcommand\chapref[1]{Chapter~\ref{chap:#1}}
\newcommand\ie{i.\,e.}
\newcommand\eg{e.\,g.}

%%%%%%%%%%%%%%%%%%%%%%%%%%%%%%%%%%%%%%%%%%%%%$%%%%%
%% Mathematical notations and Inference Rule macros
%%%%%%%%%%%%%%%%%%%%%%%%%%%%%%%%%%%%%%%%%%%%$%%%%%%
\usepackage{relsize}
\newcommand\view[0]{\hyperlink{def-deconstruction}{view}}
%\newcommand\defpoint[1]{\underline{#1}}
\newcommand\defpoint[1]{#1}
\newcommand\eqname[0]{\hyperlink{def-deconstruction}{\stackrel{\mathsmaller{\mathsf{is}}}{=}}}
\newcommand\eqdef[0]{\hyperlink{def-eqdef}{:=}}
\newcommand\overname[2]{\overbracket{#1}^{#2}}
\newcommand\overtext[2]{\overbracket{#1}^{\text{#2}}}
\newcommand\emptyfunc[0]{\hyperlink{def-emptyfunc}{{\emptyset}_\lambda}}
\newcommand\restrictfunc[2]{{#1}\hyperlink{def-restrictfunc}{|}_{#2}}

\newcommand\choice[3]{\hyperlink{def-choice}{\textsf{choice}}(#1,#2,#3)}
\newcommand\equal[0]{\hyperlink{def-equal}{\texttt{equal}}}
\newcommand\equalarrow[0]{\rightarrow}
\newcommand\equallength[0]{\hyperlink{def-equallength}{\texttt{equal\_length}}}
\newcommand\listrange[0]{\hyperlink{def-listrange}{\texttt{indices}}}
\newcommand\listlen[1]{\hyperlink{def-listlen}{|}#1\hyperlink{def-listlen}{|}}
\newcommand\cardinality[1]{\hyperlink{def-cardinality}{|}#1\hyperlink{def-cardinality}{|}}
\newcommand\splitlist[0]{\hyperlink{def-splitlist}{\texttt{split}}}
\newcommand\concat[0]{\hyperlink{def-concat}{+}}
\newcommand\prepend[0]{\hyperlink{def-prepend}{{+}{+}}}
\newcommand\listcomprehension[2]{[#1 : #2]} % #1 is a list element predicate, #2 is the output list element.
\newcommand\setcomprehension[2]{\{#1 : #2\}} % #1 is a set element predicate, #2 is the output set element.

\newcommand\Ignore[0]{\hyperlink{def-ignore}{\underline{\;\;}}}
\newcommand\None[0]{\hyperlink{def-none}{\texttt{None}}}

% Set types
\newcommand\N[0]{\hyperlink{def-N}{\mathbb{N}}}
\newcommand\Npos[0]{\hyperlink{def-Npos}{\mathbb{N}^{+}}}
\newcommand\Q[0]{\hyperlink{def-Q}{\mathbb{Q}}}
\newcommand\Z[0]{\hyperlink{def-Z}{\mathbb{Z}}}
\newcommand\Bool[0]{\hyperlink{def-bool}{\mathbb{B}}}
\newcommand\Identifiers[0]{\hyperlink{def-identifier}{\mathbb{I}}}
\newcommand\Strings[0]{\hyperlink{def-strings}{\mathbb{S}}}
\newcommand\astlabels[0]{\hyperlink{def-astlabels}{\mathbb{L}}}
\newcommand\literals[0]{\mathcal{L}}

\newcommand\pow[1]{\hyperlink{def-pow}{\mathcal{P}}(#1)}
\newcommand\partialto[0]{\hyperlink{def-partialfunc}{\rightharpoonup}}
\newcommand\rightarrowfin[0]{\hyperlink{def-finfunction}{\rightarrow_{\text{fin}}}}
\newcommand\funcgraph[0]{\hyperlink{def-funcgraph}{\texttt{func\_graph}}}
\DeclareMathOperator{\dom}{\hyperlink{def-dom}{dom}}
\newcommand\sign[0]{\hyperlink{def-sign}{\texttt{sign}}}

\newcommand\configdomain[1]{\hyperlink{def-configdomain}{\texttt{config\_domain}}({#1})}

\newcommand\sslash[0]{\mathbin{/\mkern-6mu/}}
\newcommand\terminateas[0]{\hyperlink{def-terminateas}{\sslash}\;}
\newcommand\booltrans[1]{\hyperlink{def-booltrans}{\texttt{bool\_transition}}(#1)}
\newcommand\booltransarrow[0]{\longrightarrow}
\newcommand\checktrans[2]{\hyperlink{def-checktrans}{\texttt{check}}(#1, \texttt{#2})}
\newcommand\checktransarrow[0]{\longrightarrow}

%%%%%%%%%%%%%%%%%%%%%%%%%%%%%%%%%%%%%%%%%%%%%%%%%%
% Abstract Syntax macros
% These are used by the AST reference, typing reference, and semantics reference.
\newcommand\specification[0]{\textsf{specification}}

\newcommand\emptylist[0]{\hyperlink{def-emptylist}{[\ ]}}

\newcommand\BNOT[0]{\texttt{BNOT}} % Boolean inversion
\newcommand\NEG[0]{\texttt{NEG}} % Integer or real negation
\newcommand\NOT[0]{\texttt{NOT}} % Bitvector bitwise inversion

\newcommand\AND[0]{\texttt{AND}} % Bitvector bitwise and
\newcommand\BAND[0]{\texttt{BAND}} % Boolean and
\newcommand\BEQ[0]{\texttt{BEQ}} % Boolean equivalence
\newcommand\BOR[0]{\texttt{BOR}} % Boolean or
\newcommand\DIV[0]{\texttt{DIV}} % Integer division
\newcommand\DIVRM[0]{\texttt{DIVRM}} % Inexact integer division, with rounding towards negative infinity.
\newcommand\EOR[0]{\texttt{XOR}} % Bitvector bitwise exclusive or
\newcommand\EQOP[0]{\texttt{EQ\_OP}} % Equality on two base values of same type
\newcommand\GT[0]{\texttt{GT}} % Greater than for int or reals
\newcommand\GEQ[0]{\texttt{GEQ}} % Greater or equal for int or reals
\newcommand\IMPL[0]{\texttt{IMPL}} % Boolean implication
\newcommand\LT[0]{\texttt{LT}} % Less than for int or reals
\newcommand\LEQ[0]{\texttt{LEQ}} % Less or equal for int or reals
\newcommand\MOD[0]{\texttt{MOD}} % Remainder of integer division
\newcommand\MINUS[0]{\texttt{MINUS}} % Subtraction for int or reals or bitvectors
\newcommand\MUL[0]{\texttt{MUL}} % Multiplication for int or reals or bitvectors
\newcommand\NEQ[0]{\texttt{NEQ}} % Non equality on two base values of same type
\newcommand\OR[0]{\texttt{OR}} % Bitvector bitwise or
\newcommand\PLUS[0]{\texttt{PLUS}} % Addition for int or reals or bitvectors
\newcommand\POW[0]{\texttt{POW}} % Exponentiation for ints
\newcommand\RDIV[0]{\texttt{RDIV}} % Division for reals
\newcommand\SHL[0]{\texttt{SHL}} % Shift left for ints
\newcommand\SHR[0]{\texttt{SHR}} % Shift right for ints

\newcommand\UNKNOWN[0]{\texttt{UNKNOWN}}

% For loop direction
\newcommand\UP[0]{\texttt{Up}}
\newcommand\DOWN[0]{\texttt{Down}}

% Non-terminal names
\newcommand\unop[0]{\textsf{unop}}
\newcommand\binop[0]{\textsf{binop}}
\newcommand\literal[0]{\textsf{literal}}
\newcommand\expr[0]{\textsf{expr}}
\newcommand\lexpr[0]{\textsf{lexpr}}
\newcommand\slice[0]{\textsf{slice}}
\newcommand\arrayindex[0]{\textsf{array\_index}}
\newcommand\leslice[0]{\texttt{LE\_Slice}}

\newcommand\ty[0]{\textsf{ty}}
\newcommand\pattern[0]{\textsf{pattern}}
\newcommand\intconstraints[0]{\textsf{int\_constraints}}
\newcommand\intconstraint[0]{\textsf{int\_constraint}}
\newcommand\unconstrained[0]{\textsf{Unconstrained}}
\newcommand\wellconstrained[0]{\textsf{WellConstrained}}
\newcommand\underconstrained[0]{\textsf{Underconstrained}}
\newcommand\constraintexact[0]{\textsf{Constraint\_Exact}}
\newcommand\constraintrange[0]{\textsf{Constraint\_Range}}
\newcommand\bitfield[0]{\textsf{bitfield}}
\newcommand\version[0]{\textsf{version}}
\newcommand\spec[0]{\textsf{spec}}
\newcommand\typedidentifier[0]{\textsf{typed\_identifier}}
\newcommand\localdeclkeyword[0]{\textsf{local\_decl\_keyword}}
\newcommand\globaldeclkeyword[0]{\textsf{global\_decl\_keyword}}
\newcommand\localdeclitem[0]{\textsf{local\_decl\_item}}
\newcommand\globaldecl[0]{\textsf{global\_decl}}
\newcommand\fordirection[0]{\textsf{for\_direction}}
\newcommand\stmt[0]{\textsf{stmt}}
\newcommand\decl[0]{\textsf{decl}}
\newcommand\casealt[0]{\textsf{case\_alt}}
\newcommand\catcher[0]{\textsf{catcher}}
\newcommand\subprogramtype[0]{\textsf{sub\_program\_type}}
\newcommand\subprogrambody[0]{\textsf{sub\_program\_body}}
\newcommand\func[0]{\textsf{func}}
\newcommand\Field[0]{\textsf{field}}

% Expression labels
\newcommand\ELiteral[0]{\textsf{E\_Literal}}
\newcommand\EVar[0]{\textsf{E\_Var}}
\newcommand\EATC[0]{\textsf{E\_ATC}}
\newcommand\EBinop[0]{\textsf{E\_Binop}}
\newcommand\EUnop[0]{\textsf{E\_Unop}}
\newcommand\ECall[0]{\textsf{E\_Call}}
\newcommand\ESlice[0]{\textsf{E\_Slice}}
\newcommand\ECond[0]{\textsf{E\_Cond}}
\newcommand\EGetArray[0]{\textsf{E\_GetArray}}
\newcommand\EGetField[0]{\textsf{E\_GetField}}
\newcommand\EGetItem[0]{\textsf{E\_GetItem}}
\newcommand\EGetFields[0]{\textsf{E\_GetFields}}
\newcommand\ERecord[0]{\textsf{E\_Record}}
\newcommand\EConcat[0]{\textsf{E\_Concat}}
\newcommand\ETuple[0]{\textsf{E\_Tuple}}
\newcommand\EUnknown[0]{\textsf{E\_Unknown}}
\newcommand\EPattern[0]{\textsf{E\_Pattern}}

% Left-hand-side expression labels
\newcommand\LEConcat[0]{\textsf{LE\_Concat}}
\newcommand\LEDiscard[0]{\textsf{LE\_Discard}}
\newcommand\LEVar[0]{\textsf{LE\_Var}}
\newcommand\LESlice{\textsf{LE\_Slice}}
\newcommand\LESetArray[0]{\textsf{LE\_SetArray}}
\newcommand\LESetField[0]{\textsf{LE\_SetField}}
\newcommand\LESetFields[0]{\textsf{LE\_SetFields}}
\newcommand\LEDestructuring[0]{\textsf{LE\_Destructuring}}

% Statement labels
\newcommand\SPass[0]{\textsf{S\_Pass}}
\newcommand\SAssign[0]{\textsf{S\_Assign}}
\newcommand\SReturn[0]{\textsf{S\_Return}}
\newcommand\SSeq[0]{\textsf{S\_Seq}}
\newcommand\SCall[0]{\textsf{S\_Call}}
\newcommand\SCond[0]{\textsf{S\_Cond}}
\newcommand\SCase[0]{\textsf{S\_Case}}
\newcommand\SDecl[0]{\textsf{S\_Decl}}
\newcommand\SAssert[0]{\textsf{S\_Assert}}
\newcommand\SWhile[0]{\textsf{S\_While}}
\newcommand\SRepeat[0]{\textsf{S\_Repeat}}
\newcommand\SFor[0]{\textsf{S\_For}}
\newcommand\SThrow[0]{\textsf{S\_Throw}}
\newcommand\STry[0]{\textsf{S\_Try}}
\newcommand\SPrint[0]{\textsf{S\_Print}}

% Literal labels
\newcommand\lint[0]{\textsf{L\_Int}}
\newcommand\lbool[0]{\textsf{L\_Bool}}
\newcommand\lreal[0]{\textsf{L\_Real}}
\newcommand\lbitvector[0]{\textsf{L\_Bitvector}}
\newcommand\lstring[0]{\textsf{L\_String}}

\newcommand\True[0]{\hyperlink{def-true}{\texttt{TRUE}}}
\newcommand\False[0]{\hyperlink{def-false}{\texttt{FALSE}}}

% Type labels
\newcommand\TInt[0]{\textsf{T\_Int}}
\newcommand\TReal[0]{\textsf{T\_Real}}
\newcommand\TString[0]{\textsf{T\_String}}
\newcommand\TBool[0]{\textsf{T\_Bool}}
\newcommand\TBits[0]{\textsf{T\_Bits}}
\newcommand\TEnum[0]{\textsf{T\_Enum}}
\newcommand\TTuple[0]{\textsf{T\_Tuple}}
\newcommand\TArray[0]{\textsf{T\_Array}}
\newcommand\TRecord[0]{\textsf{T\_Record}}
\newcommand\TException[0]{\textsf{T\_Exception}}
\newcommand\TNamed[0]{\textsf{T\_Named}}

\newcommand\BitFieldSimple[0]{\textsf{BitField\_Simple}}
\newcommand\BitFieldNested[0]{\textsf{BitField\_Nested}}
\newcommand\BitFieldType[0]{\textsf{BitField\_Type}}

\newcommand\ConstraintExact[0]{\textsf{Constraint\_Exact}}
\newcommand\ConstraintRange[0]{\textsf{Constraint\_Range}}

% Array index labels
\newcommand\ArrayLengthExpr[0]{\textsf{ArrayLength\_Expr}}
\newcommand\ArrayLengthEnum[0]{\textsf{ArrayLength\_Enum}}

% Slice labels
\newcommand\SliceSingle[0]{\textsf{Slice\_Single}}
\newcommand\SliceRange[0]{\textsf{Slice\_Range}}
\newcommand\SliceLength[0]{\textsf{Slice\_Length}}
\newcommand\SliceStar[0]{\textsf{Slice\_Star}}

% Pattern labels
\newcommand\PatternAll[0]{\textsf{Pattern\_All}}
\newcommand\PatternAny[0]{\textsf{Pattern\_Any}}
\newcommand\PatternGeq[0]{\textsf{Pattern\_Geq}}
\newcommand\PatternLeq[0]{\textsf{Pattern\_Leq}}
\newcommand\PatternNot[0]{\textsf{Pattern\_Not}}
\newcommand\PatternRange[0]{\textsf{Pattern\_Range}}
\newcommand\PatternSingle[0]{\textsf{Pattern\_Single}}
\newcommand\PatternMask[0]{\textsf{Pattern\_Mask}}
\newcommand\PatternTuple[0]{\textsf{Pattern\_Tuple}}

% Local declarations
\newcommand\LDIDiscard[0]{\textsf{LDI\_Discard}}
\newcommand\LDIVar[0]{\textsf{LDI\_Var}}
\newcommand\LDITyped[0]{\textsf{LDI\_Typed}}
\newcommand\LDITuple[0]{\textsf{LDI\_Tuple}}

\newcommand\LDKVar[0]{\textsf{LDK\_Var}}
\newcommand\LDKConstant[0]{\textsf{LDK\_Constant}}
\newcommand\LDKLet[0]{\textsf{LDK\_Let}}

\newcommand\GDKConstant[0]{\textsf{GDK\_Constant}}
\newcommand\GDKConfig[0]{\textsf{GDK\_Config}}
\newcommand\GDKLet[0]{\textsf{GDK\_Let}}
\newcommand\GDKVar[0]{\textsf{GDK\_Var}}

\newcommand\SBASL[0]{\textsf{SB\_ASL}}
\newcommand\SBPrimitive[0]{\textsf{SB\_Primitive}}

\newcommand\STFunction[0]{\texttt{ST\_Function}}
\newcommand\STGetter[0]{\texttt{ST\_Getter}}
\newcommand\STEmptyGetter[0]{\texttt{ST\_EmptyGetter}}
\newcommand\STSetter[0]{\texttt{ST\_Setter}}
\newcommand\STEmptySetter[0]{\texttt{ST\_EmptySetter}}
\newcommand\STProcedure[0]{\texttt{ST\_Procedure}}

% Fields
\newcommand\funcname[0]{\text{name}}
\newcommand\funcparameters[0]{\text{parameters}}
\newcommand\funcargs[0]{\text{args}}
\newcommand\funcbody[0]{\text{body}}
\newcommand\funcreturntype[0]{\text{return\_type}}
\newcommand\funcsubprogramtype[0]{\text{subprogram\_type}}
\newcommand\GDkeyword[0]{\text{keyword}}
\newcommand\GDname[0]{\text{name}}
\newcommand\GDty[0]{\text{ty}}
\newcommand\GDinitialvalue[0]{\text{initial\_value}}

\newcommand\DFunc[0]{\texttt{D\_Func}}
\newcommand\DGlobalStorage[0]{\texttt{D\_GlobalStorage}}
\newcommand\DTypeDecl[0]{\texttt{D\_TypeDecl}}

\newcommand\identifier[0]{\textsf{identifier}}
%%%%%%%%%%%%%%%%%%%%%%%%%%%%%%%%%%%%%%%%%%%%%%%%%%

\newcommand\torexpr[0]{\hyperlink{def-rexpr}{\textsf{rexpr}}}

%%%%%%%%%%%%%%%%%%%%%%%%%%%%%%%%%%%%%%%%%%%%%%%%%%
%% Type System macros
%%%%%%%%%%%%%%%%%%%%%%%%%%%%%%%%%%%%%%%%%%%%%%%%%%
\newcommand\TypeErrorConfig[0]{\hyperlink{def-typeerrorconfig}{\texttt{\#TE}}}
\newcommand\TTypeError[0]{\hyperlink{def-ttypeerror}{\textsf{TTypeError}}}
\newcommand\TypeError[0]{\hyperlink{def-typeerror}{\textsf{TypeError}}}
\newcommand\TypeErrorVal[1]{\TypeError(\texttt{#1})}

\newcommand\Val[0]{\textsf{Val}}
\newcommand\aslto[0]{\longrightarrow}

\newcommand\staticenvs[0]{\hyperlink{def-staticenvs}{\mathbb{SE}}}
\newcommand\emptytenv[0]{\hyperlink{def-emptytenv}{\emptyset_{\tenv}}}
\newcommand\tstruct[0]{\hyperlink{def-tstruct}{\texttt{get\_structure}}}
\newcommand\astlabel[0]{\hyperlink{def-astlabel}{\textsf{ast\_label}}}
\newcommand\typesat[0]{\hyperlink{def-typesatisfies}{\texttt{type\_satisfies}}}

\newcommand\constantvalues[0]{\hyperlink{def-constantvalues}{\textsf{constant\_values}}}
\newcommand\globalstoragetypes[0]{\hyperlink{def-globalstoragetypes}{\textsf{global\_storage\_types}}}
\newcommand\localstoragetypes[0]{\hyperlink{def-localstoragetypes}{\textsf{local\_storage\_types}}}
\newcommand\returntype[0]{\hyperlink{def-returntype}{\textsf{return\_type}}}
\newcommand\declaredtypes[0]{\hyperlink{def-declaredtypes}{\textsf{declared\_types}}}
\newcommand\subtypes[0]{\hyperlink{def-subtypes}{\textsf{subtypes}}}
\newcommand\subprograms[0]{\hyperlink{def-subprograms}{\textsf{subprograms}}}
\newcommand\subprogramrenamings[0]{\hyperlink{def-subprogramrenamings}{\textsf{subprogram\_renamings}}}
\newcommand\parameters[0]{\hyperlink{def-parameters}{\textsf{parameters}}}

\newcommand\unopliterals[0]{\hyperlink{def-unopliterals}{\texttt{unop\_literals}}}
\newcommand\binopliterals[0]{\hyperlink{def-binopliterals}{\texttt{binop\_literals}}}

%%%%%%%%%%%%%%%%%%%%%%%%%%%%%%%%%%%%%%%%%%%%%%%%%%
%% Semantics macros
%%%%%%%%%%%%%%%%%%%%%%%%%%%%%%%%%%%%%%%%%%%%%%%%%%
\newcommand\aslrel[0]{\bigtimes}
\newcommand\vals[0]{\hyperlink{def-vals}{\mathbb{V}}}
\newcommand\nvliteral[1]{\hyperlink{def-nvliteral}{\texttt{NV\_Literal}}(#1)}
\newcommand\nvvector[1]{\hyperlink{def-nvvector}{\texttt{NV\_Vector}}(#1)}
\newcommand\nvrecord[1]{\hyperlink{def-nvrecord}{\texttt{NV\_Record}}(#1)}
\newcommand\nvint[0]{\hyperlink{def-nvint}{\texttt{Int}}}
\newcommand\nvbool[0]{\hyperlink{def-nvbool}{\texttt{Bool}}}
\newcommand\nvreal[0]{\hyperlink{nv-real}{\texttt{Real}}}
\newcommand\nvstring[0]{\hyperlink{def-nvstring}{\texttt{String}}}
\newcommand\nvbitvector[0]{\hyperlink{def-nvbitvector}{\texttt{Bitvector}}}
\newcommand\tint[0]{\hyperlink{def-tint}{\mathcal{Z}}}
\newcommand\tbool[0]{\hyperlink{def-tbool}{\mathcal{B}}}
\newcommand\treal[0]{\hyperlink{def-treal}{\mathcal{R}}}
\newcommand\tstring[0]{\hyperlink{def-tstring}{\mathcal{S}\mathcal{T}\mathcal{R}}}
\newcommand\tbitvector[0]{\hyperlink{def-tbitvector}{\mathcal{B}\mathcal{V}}}
\newcommand\tvector[0]{\hyperlink{def-tvector}{\mathcal{V}\mathcal{E}\mathcal{C}}}
\newcommand\trecord[0]{\hyperlink{def-trecord}{\mathcal{R}\mathcal{E}\mathcal{C}}}

\newcommand\evalrel[0]{\hyperlink{def-evalrel}{\textsf{eval}}}
\newcommand\evalarrow[0]{\xrightarrow{\evalrel}}
\newcommand\envs[0]{\hyperlink{def-envs}{\mathbb{E}}}
\newcommand\dynamicenvs[0]{\hyperlink{def-dynamicenvs}{\mathbb{DE}}}
\newcommand\xgraph[0]{\textsf{g}}
\newcommand\emptygraph[0]{{\hyperlink{def-emptygraph}{\emptyset_\xgraph}}}
\newcommand\asldata[0]{\hyperlink{def-asldata}{\mathtt{asl\_data}}}
\newcommand\aslctrl[0]{\hyperlink{def-aslctrl}{\mathtt{asl\_ctrl}}}
\newcommand\aslpo[0]{\hyperlink{def-aslpo}{\mathtt{asl\_po}}}

%%%%%%%%%%%%%%%%%%%%%%%%%%%%%%%%%%%%%%%%%%%%%%%%%%
% Semantics/Typing Shared macros
\newcommand\ProseTerminateAs[1]{\hyperlink{def-proseterminateas}{${}^{\sslash #1 }$}}
\newcommand\tenv[0]{\textsf{tenv}}
\newcommand\denv[0]{\textsf{denv}}
\newcommand\aslsep[0]{\mathbf{,}}

% Glossary
\newcommand\head[0]{\hyperlink{def-head}{\texttt{head}}}
\newcommand\tail[0]{\hyperlink{def-tail}{\texttt{tail}}}

%%%%%%%%%%%%%%%%%%%%%%%%%%%%%%%%%%%%%%%%%%%%%%%%%%
% LRM Ident info
\newcommand\ident[2]{\texttt{#1}\textsubscript{\texttt{\MakeUppercase{#2}}}}
\newcommand\identi[1]{\ident{I}{#1}}
\newcommand\identr[1]{\ident{R}{#1}}
\newcommand\identd[1]{\ident{D}{#1}}
\newcommand\identg[1]{\ident{G}{#1}}

%%%%%%%%%%%%%%%%%%%%%%%%%%%%%%%%%%%%%%%%%%%%%%%%%%
% Variable Names
\newcommand\id[0]{\texttt{id}}
\newcommand\idone[0]{\texttt{id1}}
\newcommand\idtwo[0]{\texttt{id2}}
\newcommand\idthree[0]{\texttt{id3}}
\newcommand\op[0]{\texttt{op}}
\newcommand\vx[0]{\texttt{x}}
\newcommand\va[0]{\texttt{a}}
\newcommand\vb[0]{\texttt{b}}
\newcommand\vbfour[0]{\texttt{b4}}
\newcommand\ve[0]{\texttt{e}}
\newcommand\vi[0]{\texttt{i}}
\newcommand\vj[0]{\texttt{j}}
\newcommand\vn[0]{\texttt{n}}
\newcommand\vtone[0]{\texttt{t1}}
\newcommand\vttwo[0]{\texttt{t2}}
\newcommand\vtthree[0]{\texttt{t3}}
\newcommand\vle[0]{\texttt{le}}
\newcommand\vargs[0]{\texttt{args}}
\newcommand\vnewargs[0]{\texttt{new\_args}}
\newcommand\vgone[0]{\texttt{g1}}
\newcommand\vgtwo[0]{\texttt{g2}}
\newcommand\envone[0]{\texttt{env1}}
\newcommand\envtwo[0]{\texttt{env2}}
\newcommand\newenv[0]{\texttt{new\_env}}
\newcommand\vv[0]{\texttt{v}}
\newcommand\vk[0]{\texttt{k}}
\newcommand\vvl[0]{\texttt{vl}}
\newcommand\vr[0]{\texttt{r}}
\newcommand\fieldmap[0]{\textit{field\_map}}
\newcommand\length[0]{\texttt{length}}
\newcommand\factor[0]{\texttt{factor}}
\newcommand\width[0]{\texttt{width}}
\newcommand\widthp[0]{\texttt{width'}}
\newcommand\vcs[0]{\texttt{vcs}}
\newcommand\vis[0]{\texttt{vis}}
\newcommand\vd[0]{\texttt{d}}
\newcommand\isone[0]{\texttt{is1}}
\newcommand\istwo[0]{\texttt{is2}}
\newcommand\dt[0]{\texttt{dt}}
\newcommand\ds[0]{\texttt{ds}}
\newcommand\widtht[0]{\texttt{width\_t}}
\newcommand\widths[0]{\texttt{width\_s}}
\newcommand\ws[0]{\texttt{w\_s}}
\newcommand\wt[0]{\texttt{w\_t}}
\newcommand\bfss[0]{\texttt{bfs\_s}}
\newcommand\bfst[0]{\texttt{bfs\_t}}
\newcommand\bfsone[0]{\texttt{bfs1}}
\newcommand\bfstwo[0]{\texttt{bfs2}}
\newcommand\vlengtht[0]{\texttt{length\_t}}
\newcommand\vlengths[0]{\texttt{length\_s}}
\newcommand\vlengthexprt[0]{\texttt{length\_expr\_t}}
\newcommand\vlengthexprs[0]{\texttt{length\_expr\_s}}
\newcommand\vtyt[0]{\texttt{ty\_t}}
\newcommand\vtys[0]{\texttt{ty\_s}}
\newcommand\vnamet[0]{\texttt{name\_t}}
\newcommand\vnames[0]{\texttt{name\_s}}
\newcommand\vlis[0]{\texttt{lis\_s}}
\newcommand\vlit[0]{\texttt{lis\_t}}
\newcommand\vfieldss[0]{\texttt{fields\_s}}
\newcommand\vfieldst[0]{\texttt{fields\_t}}
\newcommand\vnamest[0]{\texttt{names\_t}}
\newcommand\vnamess[0]{\texttt{names\_s}}
\newcommand\vsstruct[0]{\texttt{s\_struct}}
\newcommand\vtstruct[0]{\texttt{t\_struct}}
\newcommand\vtystruct[0]{\texttt{ty\_struct}}
\newcommand\calltype[0]{\texttt{call\_type}}
\newcommand\eqsone[0]{\texttt{eqs1}}
\newcommand\eqstwo[0]{\texttt{eqs2}}
\newcommand\eqsthree[0]{\texttt{eqs3}}
\newcommand\eqsthreep[0]{\texttt{eqs3'}}
\newcommand\eqsfour[0]{\texttt{eqs4}}
\newcommand\retty[0]{\texttt{ret\_ty}}
\newcommand\rettyone[0]{\texttt{ret\_ty1}}
\newcommand\rettyopt[0]{\texttt{ret\_ty\_opt}}
\newcommand\exprs[0]{\texttt{exprs}}
\newcommand\exprsone[0]{\texttt{exprs1}}
\newcommand\typedexpr[0]{\texttt{typed\_exprs}}
\newcommand\typedexprs[0]{\texttt{typed\_exprs}}
\newcommand\typedexprsone[0]{\texttt{typed\_exprs1}}
\newcommand\cst[0]{\texttt{cs\_t}}
\newcommand\css[0]{\texttt{cs\_s}}
\newcommand\vlione[0]{\texttt{li1}}
\newcommand\vlitwo[0]{\texttt{li2}}
\newcommand\structone[0]{\texttt{struct1}}
\newcommand\structtwo[0]{\texttt{struct2}}
\newcommand\newty[0]{\texttt{new\_ty}}
\newcommand\tenvp[0]{\texttt{tenv'}}
\newcommand\tenvthree[0]{\texttt{tenv3}}
\newcommand\tenvthreep[0]{\texttt{tenv3'}}
\newcommand\tenvfour[0]{\texttt{tenv4}}
\newcommand\tenvfive[0]{\texttt{tenv5}}
\newcommand\funcsig[0]{\texttt{func\_sig}}
\newcommand\funcsigp[0]{\texttt{func\_sig'}}
\newcommand\gsd[0]{\texttt{gsd}}
\newcommand\gsdp[0]{\texttt{gsd'}}
\newcommand\newgsd[0]{\texttt{new\_gsd}}
\newcommand\decls[0]{\texttt{decls}}
\newcommand\declsp[0]{\texttt{decls'}}
\newcommand\ordereddecls[0]{\texttt{ordered\_decls}}
\newcommand\constraints[0]{\texttt{constraints}}
\newcommand\constraintsp[0]{\texttt{constraints'}}
\newcommand\ewidth[0]{\texttt{e\_width}}
\newcommand\ewidthp[0]{\texttt{e\_width'}}
\newcommand\bitfieldsp[0]{\texttt{bitfields'}}
\newcommand\bitfieldspp[0]{\texttt{bitfields''}}
\newcommand\fieldsp[0]{\texttt{fields'}}
\newcommand\vstmt[0]{\texttt{stmt}}
\newcommand\newstmt[0]{\texttt{new\_stmt}}
\newcommand\newstmtp[0]{\texttt{new\_stmt'}}
\newcommand\newopt[0]{\texttt{new\_opt}}
\newcommand\callerargtyped[0]{\texttt{caller\_arg\_typed}}
\newcommand\callerargtypes[0]{\texttt{caller\_arg\_types}}
\newcommand\callee[0]{\texttt{callee}}
\newcommand\calleeargtypes[0]{\texttt{callee\_arg\_types}}
\newcommand\calleeparam[0]{\texttt{callee\_param}}
\newcommand\calleeparams[0]{\texttt{callee\_params}}
\newcommand\calleeparamsone[0]{\texttt{callee\_params1}}
\newcommand\extranargs[0]{\texttt{extra\_nargs}}
\newcommand\veq[0]{\texttt{eq}}
\newcommand\newconstraints[0]{\texttt{new\_constraints}}
\newcommand\newc[0]{\texttt{new\_c}}
\newcommand\tysp[0]{\texttt{tys'}}
\newcommand\ttyp[0]{\texttt{ty'}}
\newcommand\field[0]{\texttt{field}}
\newcommand\newfield[0]{\texttt{new\_field}}
\newcommand\newfields[0]{\texttt{new\_fields}}
\newcommand\names[0]{\texttt{names}}
\newcommand\bits[0]{\texttt{bits}}
\newcommand\veonep[0]{\texttt{e1'}}
\newcommand\vetwop[0]{\texttt{e2'}}
\newcommand\vethreep[0]{\texttt{e2'}}
\newcommand\vepp[0]{\texttt{e''}}
\newcommand\vtpp[0]{\texttt{t''}}
\newcommand\vsp[0]{\texttt{s'}}
\newcommand\vspp[0]{\texttt{s''}}
\newcommand\econdp[0]{\texttt{e\_cond'}}
\newcommand\etruep[0]{\texttt{e\_true'}}
\newcommand\efalsep[0]{\texttt{e\_false'}}
\newcommand\slicesp[0]{\texttt{slices'}}
\newcommand\tep[0]{\texttt{t\_e'}}
\newcommand\structtep[0]{\texttt{struct\_t\_e'}}
\newcommand\structtleone[0]{\texttt{struct\_t\_le1}}
\newcommand\varg[0]{\texttt{arg}}
\newcommand\vnewarg[0]{\texttt{new\_arg}}
\newcommand\vargsp[0]{\texttt{args'}}
\newcommand\eqsp[0]{\texttt{eqs'}}
\newcommand\namep[0]{\texttt{name'}}
\newcommand\namepp[0]{\texttt{name''}}
\newcommand\eindexp[0]{\texttt{e\_index'}}
\newcommand\tindexp[0]{\texttt{t\_index'}}
\newcommand\fieldtypes[0]{\texttt{field\_types}}
\newcommand\tspecp[0]{\texttt{t\_spec'}}
\newcommand\pairs[0]{\texttt{pairs}}
\newcommand\initializedfields[0]{\texttt{initialized\_fields}}
\newcommand\vteone[0]{\texttt{t\_e1}}
\newcommand\vtetwo[0]{\texttt{t\_e2}}
\newcommand\vtethree[0]{\texttt{t\_e3}}
\newcommand\vtefour[0]{\texttt{t\_e4}}
\newcommand\vtefive[0]{\texttt{t\_e5}}
\newcommand\ldip[0]{\texttt{ldi'}}
\newcommand\newldip[0]{\texttt{new\_ldi'}}
\newcommand\newldis[0]{\texttt{new\_ldis'}}
\newcommand\newlocalstoragetypes[0]{\texttt{new\_local\_storagetypes}}
\newcommand\eoffset[0]{\texttt{offset}}
\newcommand\eoffsetp[0]{\texttt{offset'}}
\newcommand\elength[0]{\texttt{length}}
\newcommand\elengthp[0]{\texttt{length'}}
\newcommand\efactor[0]{\texttt{factor}}
\newcommand\toffset[0]{\texttt{t\_offset}}
\newcommand\tlength[0]{\texttt{t\_length}}
\newcommand\prelength[0]{\texttt{pre\_length}}
\newcommand\prelengthp[0]{\texttt{pre\_length'}}
\newcommand\preoffset[0]{\texttt{pre\_offset}}
\newcommand\newp[0]{\texttt{new\_p}}
\newcommand\newq[0]{\texttt{new\_q}}
\newcommand\newli[0]{\texttt{new\_li}}
\newcommand\vlp[0]{\texttt{l'}}
\newcommand\useset[0]{\texttt{use\_set}}
\newcommand\gdk[0]{\texttt{gdk}}
\newcommand\testruct[0]{\texttt{t\_e\_struct}}
\newcommand\vtestruct[0]{\texttt{t\_e\_struct}}
\newcommand\vq[0]{\texttt{q}}
\newcommand\vteonestruct[0]{\texttt{t\_e1\_struct}}
\newcommand\vtetwostruct[0]{\texttt{t\_e2\_struct}}
\newcommand\vlip[0]{\texttt{li'}}
\newcommand\vfp[0]{\texttt{f'}}
\newcommand\body[0]{\texttt{body}}
\newcommand\newbody[0]{\texttt{new\_body}}
\newcommand\newd[0]{\texttt{new\_d}}
\newcommand\funcsigone[0]{\texttt{func\_sig1}}
\newcommand\namesubs[0]{\texttt{name\_s}}
\newcommand\namesubt[0]{\texttt{name\_t}}
\newcommand\vstructt[0]{\texttt{struct\_t}}
\newcommand\vstructs[0]{\texttt{struct\_s}}
\newcommand\vcsnew[0]{\texttt{cs\_new}}
\newcommand\vbot[0]{\texttt{bot}}
\newcommand\vvtop[0]{\texttt{top}}
\newcommand\vneg[0]{\texttt{neg}}
\newcommand\vtonestruct[0]{\texttt{t1\_struct}}
\newcommand\vttwostruct[0]{\texttt{t2\_struct}}
\newcommand\vtoneanon[0]{\texttt{t1\_anon}}
\newcommand\vttwoanon[0]{\texttt{t2\_anon}}
\newcommand\vcsone[0]{\texttt{cs1}}
\newcommand\vcstwo[0]{\texttt{cs2}}
\newcommand\vanons[0]{\texttt{anon\_s}}
\newcommand\vanont[0]{\texttt{anon\_t}}
\newcommand\vet[0]{\texttt{e\_t}}
\newcommand\ves[0]{\texttt{e\_s}}
\newcommand\vtsupers[0]{\texttt{t\_supers}}
\newcommand\vindex[0]{\texttt{index}}
\newcommand\widthsum[0]{\texttt{width\_sum}}
\newcommand\ttyone[0]{\texttt{ty1}}
\newcommand\ttytwo[0]{\texttt{ty2}}
\newcommand\vpat[0]{\texttt{pat}}
\newcommand\vpatp[0]{\texttt{pat'}}
\newcommand\lesp[0]{\texttt{les'}}
\newcommand\lesone[0]{\texttt{les1}}
\newcommand\vteeq[0]{\texttt{t\_e\_eq}}
\newcommand\vtleonestruct[0]{\texttt{t\_le1\_struct}}
\newcommand\tsubs[0]{\texttt{t\_s}}
\newcommand\vdebug[0]{\texttt{debug}}
\newcommand\vreone[0]{\texttt{re1}}
\newcommand\vtre[0]{\texttt{t\_re}}
\newcommand\vtep[0]{\texttt{t\_e'}}
\newcommand\vsonep[0]{\texttt{s1'}}
\newcommand\vstwop[0]{\texttt{s2'}}
\newcommand\vcases[0]{\texttt{cases}}
\newcommand\vcasesone[0]{\texttt{cases1}}
\newcommand\vcase[0]{\texttt{case}}
\newcommand\vcaseone[0]{\texttt{case1}}
\newcommand\vpzero[0]{\texttt{p0}}
\newcommand\vwzero[0]{\texttt{w0}}
\newcommand\vszero[0]{\texttt{s0}}
\newcommand\vewzero[0]{\texttt{e\_w0}}
\newcommand\vewone[0]{\texttt{e\_w1}}
\newcommand\vtwe[0]{\texttt{twe}}
\newcommand\otherwisep[0]{\texttt{otherwise'}}
\newcommand\vcp[0]{\texttt{c'}}
\newcommand\catchersp[0]{\texttt{catchers'}}
\newcommand\nameopt[0]{\texttt{name\_opt}}
\newcommand\botcs[0]{\texttt{bot\_cs}}
\newcommand\topcs[0]{\texttt{top\_cs}}
\newcommand\ebot[0]{\texttt{e\_bot}}
\newcommand\etop[0]{\texttt{e\_top}}
\newcommand\vbf[0]{\texttt{bf}}
\newcommand\vbfone[0]{\texttt{bf1}}
\newcommand\vbftwo[0]{\texttt{bf2}}
\newcommand\bfstwop[0]{\texttt{bfs2'}}
\newcommand\vpp[0]{\texttt{p'}}
\newcommand\acc[0]{\texttt{acc}}
\newcommand\accp[0]{\texttt{acc'}}
\newcommand\annotateddecls[0]{\texttt{annotated\_decls}}
\newcommand\funcdef[0]{\texttt{func\_def}}
\newcommand\funcdefone[0]{\texttt{func\_def1}}
\newcommand\newfuncdef[0]{\texttt{new\_func\_def}}
\newcommand\newfuncsig[0]{\texttt{new\_func\_sig}}
\newcommand\bd[0]{\texttt{bd}}
\newcommand\initialvalue[0]{\texttt{initial\_value}}
\newcommand\keyword[0]{\texttt{keyword}}
\newcommand\subpgmtype[0]{\texttt{subpgm\_type}}
\newcommand\subpgmtypeone[0]{\texttt{subpgm\_type1}}
\newcommand\subpgmtypetwo[0]{\texttt{subpgm\_type2}}
\newcommand\formals[0]{\texttt{formals}}
\newcommand\formaltypes[0]{\texttt{formal\_types}}
\newcommand\nameformaltypes[0]{\texttt{name\_formal\_types}}
\newcommand\nameargs[0]{\texttt{name\_args}}
\newcommand\nameformals[0]{\texttt{name\_formals}}
\newcommand\namesubpgmtype[0]{\texttt{name\_subpgmtype}}
\newcommand\othernames[0]{\texttt{other\_names}}
\newcommand\newrenamings[0]{\texttt{new\_renamings}}
\newcommand\issetter[0]{\texttt{is\_setter}}
\newcommand\rettype[0]{\texttt{ret\_type}}
\newcommand\argtypes[0]{\texttt{arg\_types}}
\newcommand\argtypesp[0]{\texttt{arg\_types'}}
\newcommand\watendgettertype[0]{\texttt{wanted\_getter\_type}}
\newcommand\ret[0]{\texttt{ret}}
\newcommand\renamingset[0]{\texttt{renaming\_set}}
\newcommand\formaltys[0]{\texttt{f\_tys}}
\newcommand\argtys[0]{\texttt{a\_tys}}
\newcommand\candidates[0]{\texttt{candidates}}
\newcommand\candidatesone[0]{\texttt{candidates1}}
\newcommand\matches[0]{\texttt{matches}}
\newcommand\matchesone[0]{\texttt{matches1}}
\newcommand\matchingrenamings[0]{\texttt{matching\_renamings}}
\newcommand\vres[0]{\texttt{res}}
\newcommand\matchedname[0]{\texttt{matched\_name}}
\newcommand\potentialparams[0]{\texttt{potential\_params}}
\newcommand\declaredparams[0]{\texttt{declared\_params}}
\newcommand\argparams[0]{\texttt{arg\_params}}
\newcommand\vreturntype[0]{\texttt{return\_type}}
\newcommand\vparameters[0]{\texttt{parameters}}
\newcommand\vparametersone[0]{\texttt{parameters1}}
\newcommand\vparams[0]{\texttt{params}}
\newcommand\vparam[0]{\texttt{param}}
\newcommand\vused[0]{\texttt{used}}
\newcommand\vusedone[0]{\texttt{used1}}
\newcommand\tyoptp[0]{\texttt{ty\_opt'}}
\newcommand\expropt[0]{\texttt{expr\_opt}}
\newcommand\exproptp[0]{\texttt{expr\_opt'}}
\newcommand\initialvaluetype[0]{\texttt{initial\_value\_type}}
\newcommand\initialvaluep[0]{\texttt{initial\_value'}}
\newcommand\typeannotation[0]{\texttt{type\_annotation}}
\newcommand\declaredt[0]{\texttt{declared\_t}}
\newcommand\ids[0]{\texttt{ids}}
\newcommand\extrafields[0]{\texttt{extra\_fields}}
\newcommand\vsuper[0]{\texttt{super}}
\newcommand\idsone[0]{\texttt{ids1}}
\newcommand\idstwo[0]{\texttt{ids2}}
\newcommand\newreturntype[0]{\texttt{new\_return\_type}}
\newcommand\bvtwo[0]{\texttt{bv2}}
\newcommand\positionsone[0]{\texttt{positions1}}
\newcommand\positionstwo[0]{\texttt{positions2}}
\newcommand\vms[0]{\texttt{vms}}
\newcommand\vmsone[0]{\texttt{vms1}}
\newcommand\vmstwo[0]{\texttt{vms2}}
\newcommand\monoms[0]{\texttt{monoms}}
\newcommand\monome[0]{\texttt{monome}}
\newcommand\monomsone[0]{\texttt{monoms1}}
\newcommand\compare[0]{\texttt{compare}}
\newcommand\vqone[0]{\texttt{q1}}
\newcommand\vqtwo[0]{\texttt{q2}}
\newcommand\tty[0]{\texttt{ty}}
\newcommand\aritymatch[0]{\texttt{arity\_match}}
\newcommand\paramname[0]{\texttt{param\_name}}
\newcommand\calleeargs[0]{\texttt{callee\_args}}
\newcommand\callerargstyped[0]{\texttt{caller\_args\_typed}}
\newcommand\callerargstypedone[0]{\texttt{caller\_args\_typed1}}
\newcommand\calleex[0]{\texttt{callee\_x}}
\newcommand\callerty[0]{\texttt{caller\_ty}}
\newcommand\callere[0]{\texttt{caller\_e}}
\newcommand\calleearg[0]{\texttt{callee\_arg}}
\newcommand\calleeargone[0]{\texttt{callee\_arg1}}
\newcommand\callerarg[0]{\texttt{caller\_arg}}
\newcommand\calleeargname[0]{\texttt{callee\_arg\_name}}
\newcommand\calleeargisparam[0]{\texttt{callee\_arg\_is\_param}}
\newcommand\calleeargsone[0]{\texttt{callee\_args\_one}}
\newcommand\callerargtypesone[0]{\texttt{caller\_arg\_types1}}
\newcommand\calleeparamt[0]{\texttt{callee\_param\_t}}
\newcommand\calleeparamtrenamed[0]{\texttt{callee\_param\_t\_renamed}}
\newcommand\callerparame[0]{\texttt{caller\_param\_e}}
\newcommand\callerparamt[0]{\texttt{caller\_param\_t}}
\newcommand\callerparamname[0]{\texttt{caller\_param\_name}}
\newcommand\calleerettyopt[0]{\texttt{callee\_ret\_ty\_opt}}
\newcommand\newtys[0]{\texttt{new\_tys}}
\newcommand\substs[0]{\texttt{substs}}
\newcommand\vesp[0]{\texttt{es'}}
\newcommand\paramargs[0]{\texttt{param\_args}}

\usepackage{tikz}

\newcommand\nonterminal[1]{\texttt{#1}}
\newcommand\terminal[1]{\mathtt{\mathbf{#1}}}
%\newcommand\verbatimterminal[2]{\mathtt{\mathbf{#1}}}
\newcommand\verbatimterminal[2]{\texttt{"}\texttt{#2}\texttt{"}}
\newcommand\emptysentence[0]{\hyperlink{def-emptysentence}{\epsilon}}
\newcommand\astof[1]{\overline{{#1}}}
\newcommand\parsenode[1]{\hyperlink{def-parsenode}{\textsf{Node}}[#1]}
\newcommand\namednode[2]{#1:#2} % #1 is a free variable, #2 is the grammar symbol
\newcommand\punnode[1]{#1}
\newcommand\epsilonnode[0]{\hyperlink{def-epsilonnode}{\textsf{epsilon\_node}}}
\newcommand\yield[0]{\hyperlink{def-yield}{\textsf{yield}}}

%%%%%%%%%%%%%%%%%%%%%%%%%%%%%%%%%%%%%%%%%%%%%%%%%%%%%%%%%%%%%%%%%%%%%%%%%%%%%%
% Macros for terminal tokens
\newcommand\Tlooplimit[0]{\verbatimterminal{LOOP\_LIMIT}{@looplimit}}
\newcommand\Tand[0]{\verbatimterminal{AND}{AND}}
\newcommand\Tarray[0]{\verbatimterminal{ARRAY}{array}}
\newcommand\Tarrow[0]{\verbatimterminal{ARROW}{=>}}
\newcommand\Tas[0]{\verbatimterminal{AS}{as}}
\newcommand\Tassert[0]{\verbatimterminal{ASSERT}{assert}}
\newcommand\Tband[0]{\verbatimterminal{BAND}{\&\&}}
\newcommand\Tbegin[0]{\verbatimterminal{BEGIN}{begin}}
\newcommand\Tbeq[0]{\verbatimterminal{BEQ}{<->}}
\newcommand\Tbit[0]{\verbatimterminal{BIT}{bit}}
\newcommand\Tbits[0]{\verbatimterminal{BITS}{bits}}
\newcommand\Tbnot[0]{\verbatimterminal{BNOT}{!}}
\newcommand\Tboolean[0]{\verbatimterminal{BOOLEAN}{boolean}}
\newcommand\Tbor[0]{\verbatimterminal{BOR}{||}}
\newcommand\Tcase[0]{\verbatimterminal{CASE}{case}}
\newcommand\Tcatch[0]{\verbatimterminal{CATCH}{catch}}
\newcommand\Tcolon[0]{\verbatimterminal{COLON}{:}}
\newcommand\Tcoloncolon[0]{\verbatimterminal{COLON\_COLON}{::}}
\newcommand\Tcomma[0]{\verbatimterminal{COMMA}{,}}
\newcommand\Tconcat[0]{\verbatimterminal{CONCAT}{++}}
\newcommand\Tconfig[0]{\verbatimterminal{CONFIG}{config}}
\newcommand\Tconstant[0]{\verbatimterminal{CONSTANT}{constant}}
\newcommand\Tdebug[0]{\verbatimterminal{DEBUG}{debug}}
\newcommand\Tdiv[0]{\verbatimterminal{DIV}{DIV}}
\newcommand\Tdivrm[0]{\verbatimterminal{DIVRM}{DIVRM}}
\newcommand\Tdo[0]{\verbatimterminal{DO}{do}}
\newcommand\Tdot[0]{\verbatimterminal{DOT}{.}}
\newcommand\Tdownto[0]{\verbatimterminal{DOWNTO}{downto}}
\newcommand\Telse[0]{\verbatimterminal{ELSE}{else}}
\newcommand\Telseif[0]{\verbatimterminal{ELSIF}{elseif}}
\newcommand\Tend[0]{\verbatimterminal{END}{end}}
\newcommand\Tenumeration[0]{\verbatimterminal{ENUMERATION}{enumeration}}
\newcommand\Txor[0]{\verbatimterminal{XOR}{XOR}}
\newcommand\Teq[0]{\verbatimterminal{EQ}{=}}
\newcommand\Teqop[0]{\verbatimterminal{EQ\_OP}{==}}
\newcommand\Texception[0]{\verbatimterminal{EXCEPTION}{exception}}
\newcommand\Tfor[0]{\verbatimterminal{FOR}{for}}
\newcommand\Tfunc[0]{\verbatimterminal{FUNC}{func}}
\newcommand\Tgeq[0]{\verbatimterminal{GEQ}{>=}}
\newcommand\Tgetter[0]{\verbatimterminal{GETTER}{getter}}
\newcommand\Tgt[0]{\verbatimterminal{GT}{>}}
\newcommand\Tif[0]{\verbatimterminal{IF}{if}}
\newcommand\Timpl[0]{\verbatimterminal{IMPL}{-->}}
\newcommand\Tin[0]{\verbatimterminal{IN}{in}}
\newcommand\Tinteger[0]{\verbatimterminal{INTEGER}{integer}}
\newcommand\Tlbrace[0]{\verbatimterminal{LBRACE}{\{}}
\newcommand\Tlbracket[0]{\verbatimterminal{LBRACKET}{[}}
\newcommand\Tleq[0]{\verbatimterminal{LEQ}{<=}}
\newcommand\Tlet[0]{\verbatimterminal{LET}{let}}
\newcommand\Tlpar[0]{\verbatimterminal{LPAR}{(}}
\newcommand\Tlt[0]{\verbatimterminal{LT}{<}}
\newcommand\Tminus[0]{\verbatimterminal{MINUS}{-}}
\newcommand\Tmod[0]{\verbatimterminal{MOD}{MOD}}
\newcommand\Tmul[0]{\verbatimterminal{MUL}{*}}
\newcommand\Tneq[0]{\verbatimterminal{NEQ}{!=}}
\newcommand\Tnot[0]{\verbatimterminal{NOT}{NOT}}
\newcommand\Tof[0]{\verbatimterminal{OF}{of}}
\newcommand\Tor[0]{\verbatimterminal{OR}{OR}}
\newcommand\Totherwise[0]{\verbatimterminal{OTHERWISE}{otherwise}}
\newcommand\Tpass[0]{\verbatimterminal{PASS}{pass}}
\newcommand\Tplus[0]{\verbatimterminal{PLUS}{+}}
\newcommand\Tpluscolon[0]{\verbatimterminal{PLUS\_COLON}{+:}}
\newcommand\Tpow[0]{\verbatimterminal{POW}{\^{}}}
\newcommand\Tpragma[0]{\verbatimterminal{PRAGMA}{pragma}}
\newcommand\Tprint[0]{\verbatimterminal{PRINT}{print}}
\newcommand\Trbrace[0]{\verbatimterminal{RBRACE}{\}}}
\newcommand\Trbracket[0]{\verbatimterminal{RBRACKET}{]}}
\newcommand\Trdiv[0]{\verbatimterminal{RDIV}{RDIV}}
\newcommand\Treal[0]{\verbatimterminal{REAL}{real}}
\newcommand\Trecord[0]{\verbatimterminal{RECORD}{record}}
\newcommand\Trepeat[0]{\verbatimterminal{REPEAT}{repeat}}
\newcommand\Treturn[0]{\verbatimterminal{RETURN}{return}}
\newcommand\Trpar[0]{\verbatimterminal{RPAR}{)}}
\newcommand\Tstarcolon[0]{\verbatimterminal{STAR\_COLON}{*:}}
\newcommand\Tsemicolon[0]{\verbatimterminal{SEMI\_COLON}{;}}
\newcommand\Tsetter[0]{\verbatimterminal{SETTER}{setter}}
\newcommand\Tshl[0]{\verbatimterminal{SHL}{<<}}
\newcommand\Tshr[0]{\verbatimterminal{SHR}{>>}}
\newcommand\Tslicing[0]{\verbatimterminal{SLICING}{..}}
\newcommand\Tstring[0]{\verbatimterminal{STRING}{string}}
\newcommand\Tsubtypes[0]{\verbatimterminal{SUBTYPES}{subtypes}}
\newcommand\Tthen[0]{\verbatimterminal{THEN}{then}}
\newcommand\Tthrow[0]{\verbatimterminal{THROW}{throw}}
\newcommand\Tto[0]{\verbatimterminal{TO}{to}}
\newcommand\Ttry[0]{\verbatimterminal{TRY}{try}}
\newcommand\Ttype[0]{\verbatimterminal{TYPE}{type}}
\newcommand\Tunknown[0]{\verbatimterminal{UNKNOWN}{UNKNOWN}}
\newcommand\Tuntil[0]{\verbatimterminal{UNTIL}{until}}
\newcommand\Tvar[0]{\verbatimterminal{VAR}{var}}
\newcommand\Twhen[0]{\verbatimterminal{WHEN}{when}}
\newcommand\Twhere[0]{\verbatimterminal{WHERE}{where}}
\newcommand\Twhile[0]{\verbatimterminal{WHILE}{while}}
\newcommand\Twith[0]{\verbatimterminal{WITH}{with}}

% \newcommand\Tidentifier[1]{\terminal{IDENTIFIER(#1)}}
\newcommand\Tidentifier[0]{\hyperlink{def-tidentifier}{\terminal{ID}}}
\newcommand\Tstringlit[0]{\hyperlink{def-tstringlit}{\terminal{STRING\_LIT}}}
\newcommand\Tmasklit[0]{\hyperlink{def-tmasklit}{\terminal{MASK\_LIT}}}
\newcommand\Tbitvectorlit[0]{\hyperlink{def-tbitvectorlit}{\terminal{BITVECTOR\_LIT}}}
\newcommand\Tintlit[0]{\hyperlink{def-tintlit}{\terminal{INT\_LIT}}}
\newcommand\Treallit[0]{\hyperlink{def-treallit}{\terminal{REAL\_LIT}}}
\newcommand\Tboollit[0]{\hyperlink{def-tboollit}{\terminal{BOOL\_LIT}}}
\newcommand\Tlexeme[0]{\hyperlink{def-tlexeme}{\terminal{LEXEME}}}

\newcommand\Tunops[0]{\terminal{UNOPS}}
\newcommand\precedence[1]{\textsf{precedence: }#1}

%%%%%%%%%%%%%%%%%%%%%%%%%%%%%%%%%%%%%%%%%%%%%%%%%%%%%%%%%%%%%%%%%%%%%%%%%%%%%%
% Macros for non-terminals
\newcommand\Nast[0]{\hyperlink{def-nast}{\nonterminal{ast}}}
\newcommand\Ndecl[0]{\hyperlink{def-ndecl}{\nonterminal{decl}}}
\newcommand\Nparamsopt[0]{\hyperlink{def-nparamsopt}{\nonterminal{params\_opt}}}
\newcommand\Nfuncargs[0]{\hyperlink{def-nfuncargs}{\nonterminal{func\_args}}}
\newcommand\Nreturntype[0]{\hyperlink{def-nreturntype}{\nonterminal{return\_type}}}
\newcommand\Nfuncbody[0]{\hyperlink{def-nfuncbody}{\nonterminal{func\_body}}}
\newcommand\Naccessargs[0]{\hyperlink{def-naccessargs}{\nonterminal{access\_args}}}
\newcommand\Ntypedidentifier[0]{\hyperlink{def-ntypedidentifier}{\nonterminal{typed\_identifier}}}
\newcommand\Nasty[0]{\hyperlink{def-nasty}{\nonterminal{as\_ty}}}
\newcommand\Ntydecl[0]{\hyperlink{def-ntydecl}{\nonterminal{ty\_decl}}}
\newcommand\Nsubtype[0]{\hyperlink{def-nsubtype}{\nonterminal{subtype}}}
\newcommand\Nsubtypeopt[0]{\hyperlink{def-nsubtypeopt}{\nonterminal{subtype\_opt}}}
\newcommand\Nstoragekeyword[0]{\hyperlink{def-nstoragekeyword}{\nonterminal{storage\_keyword}}}
\newcommand\Nignoredoridentifier[0]{\hyperlink{def-nignoredoridentifier}{\nonterminal{ignored\_or\_identifier}}}
\newcommand\Ninitialvalue[0]{\hyperlink{def-ninitialvalue}{\nonterminal{initial\_value}}}
\newcommand\Nty[0]{\hyperlink{def-nty}{\nonterminal{ty}}}
\newcommand\Nexpr[0]{\hyperlink{def-nexpr}{\nonterminal{expr}}}
\newcommand\Nlexpr[0]{\hyperlink{def-nlexpr}{\nonterminal{lexpr}}}
\newcommand\Nlexpratom[0]{\hyperlink{def-nlexpratom}{\nonterminal{lexpr\_atom}}}
\newcommand\Nfields[0]{\hyperlink{def-nfields}{\nonterminal{fields}}}
\newcommand\Nfieldsopt[0]{\hyperlink{def-nfieldsopt}{\nonterminal{fields\_opt}}}
\newcommand\Nopttypedidentifier[0]{\hyperlink{def-nopttypeidentifier}{\nonterminal{opt\_typed\_identifier}}}
\newcommand\Nstmtlist[0]{\hyperlink{def-nstmtlist}{\nonterminal{stmt\_list}}}
\newcommand\Nmaybeemptystmtlist[0]{\hyperlink{def-nmaybeemptystmtlist}{\nonterminal{maybe\_empty\_stmt\_list}}}
\newcommand\Nlocaldeclkeyword[0]{\hyperlink{def-nlocaldeclkeyword}{\nonterminal{local\_decl\_keyword}}}
\newcommand\Ndirection[0]{\hyperlink{def-ndirection}{\nonterminal{direction}}}
\newcommand\Nalt[0]{\hyperlink{def-nalt}{\nonterminal{alt}}}
\newcommand\Npatternlist[0]{\hyperlink{def-npatternlist}{\nonterminal{pattern\_list}}}
\newcommand\Notherwiseopt[0]{\hyperlink{def-notherwiseopt}{\nonterminal{otherwise\_opt}}}
\newcommand\Ncatcher[0]{\hyperlink{def-ncatcher}{\nonterminal{catcher}}}
\newcommand\Nstmt[0]{\hyperlink{def-nstmt}{\nonterminal{stmt}}}
\newcommand\Nselse[0]{\hyperlink{def-nselse}{\nonterminal{s\_else}}}
\newcommand\Ndeclitem[0]{\hyperlink{def-ndeclitem}{\nonterminal{decl\_item}}}
\newcommand\Nuntypeddeclitem[0]{\hyperlink{def-nuntypeddeclitem}{\nonterminal{untyped\_decl\_item}}}
\newcommand\Nslice[0]{\hyperlink{def-nslice}{\nonterminal{slice}}}
\newcommand\Nslices[0]{\hyperlink{def-nslices}{\nonterminal{slices}}}
\newcommand\Nnslices[0]{\hyperlink{def-nnslice}{\nonterminal{nslices}}}
\newcommand\Nintconstraints[0]{\hyperlink{def-nintconstraints}{\nonterminal{int\_constraints}}}
\newcommand\Nintconstraintsopt[0]{\hyperlink{def-nintconstraintsopt}{\nonterminal{int\_constraints\_opt}}}
\newcommand\Nintconstraint[0]{\hyperlink{def-nintconstraint}{\nonterminal{int\_constraint}}}
\newcommand\Nexprpattern[0]{\hyperlink{def-nexprpattern}{\nonterminal{expr\_pattern}}}
\newcommand\Npatternset[0]{\hyperlink{def-npatternset}{\nonterminal{pattern\_set}}}
\newcommand\Npattern[0]{\hyperlink{def-npattern}{\nonterminal{pattern}}}
\newcommand\Nbitfields[0]{\hyperlink{def-nbitfields}{\nonterminal{bitfields}}}
\newcommand\Nbitfield[0]{\hyperlink{def-nbitfield}{\nonterminal{bitfield}}}
\newcommand\Nvalue[0]{\hyperlink{def-nvalue}{\nonterminal{value}}}
\newcommand\Nbinop[0]{\hyperlink{def-nbinop}{\nonterminal{binop}}}
\newcommand\Nunop[0]{\hyperlink{def-nunop}{\nonterminal{unop}}}
\newcommand\Neelse[0]{\hyperlink{def-neelse}{\nonterminal{e\_else}}}
\newcommand\Nfieldassign[0]{\hyperlink{def-nfieldassign}{\nonterminal{field\_assign}}}

%%%%%%%%%%%%%%%%%%%%%%%%%%%%%%%%%%%%%%%%%%%%%%%%%%%%%%%%%%%%%%%%%%%%%%%%%%%%%%
% Macros for defining associativity
\newcommand\nonassoc[0]{\textsf{nonassoc}}
\newcommand\leftassoc[0]{\textsf{left}}
\newcommand\rightassoc[0]{\textsf{right}}

%%%%%%%%%%%%%%%%%%%%%%%%%%%%%%%%%%%%%%%%%%%%%%%%%%%%%%%%%%%%%%%%%%%%%%%%%%%%%%
% Macros for generic parsing symbols and parsing functions
% \newcommand\derives[0]{\texttt{::=}}
\newcommand\derives[0]{\longrightarrow}
\newcommand\derivesinline[0]{\xlongrightarrow{\textsf{inline}}}
\newcommand\parsesep[0]{\ } % separates symbols in a single production

\newcommand\maybeemptylist[1]{\hyperlink{def-maybeemptylist}{\textsf{list}^{*}}(#1)} % This stands for list(x)
\newcommand\nonemptylist[1]{\hyperlink{def-nonemptylist}{\textsf{list}^{+}}(#1)} % This stands for non_empty_list(x)
\newcommand\NClist[1]{\hyperlink{def-nclist}{\textsf{clist}^{+}}(#1)}
\newcommand\Plist[1]{\hyperlink{def-plist}{\textsf{plist}}(#1)}
\newcommand\Plisttwo[1]{\hyperlink{def-plisttwo}{\textsf{plist2}}(#1)}
\newcommand\Clist[1]{\hyperlink{def-clist}{\textsf{clist}}(#1)}
\newcommand\Clisttwo[1]{\hyperlink{def-clisttwo}{\textsf{clist2}}(#1)}
\newcommand\NTClist[1]{\hyperlink{def-ntclist}{\textsf{ntclist}}(#1)}
\newcommand\TClist[1]{\hyperlink{def-tclist}{\textsf{tclist}}(#1)}
\newcommand\option[1]{\hyperlink{def-option}{\textsf{option}}(#1)}
%\newcommand\option[1]{(#1)\hyperlink{def-option}{?}}

%%%%%%%%%%%%%%%%%%%%%%%%%%%%%%%%%%%%%%%%%%%%%%%%%%%%%%%%%%%%%%%%%%%%%%%%%%%%%%%
%% AST macros with hyperlinks
\renewcommand\identifier[0]{\hyperlink{ast-identifier}{\textsf{identifier}}} % Boolean inversion

\renewcommand\BNOT[0]{\hyperlink{ast-bnot}{\texttt{BNOT}}} % Boolean inversion
\renewcommand\NEG[0]{\hyperlink{ast-neg}{\texttt{NEG}}} % Integer or real negation
\renewcommand\NOT[0]{\hyperlink{ast-not}{\texttt{NOT}}} % Bitvector bitwise inversion

\renewcommand\AND[0]{\hyperlink{ast-and}{\texttt{AND}}} % Bitvector bitwise and
\renewcommand\BAND[0]{\hyperlink{ast-band}{\texttt{BAND}}} % Boolean and
\renewcommand\BEQ[0]{\hyperlink{ast-beq}{\texttt{BEQ}}} % Boolean equivalence
\renewcommand\BOR[0]{\hyperlink{ast-bor}{\texttt{BOR}}} % Boolean or
\renewcommand\DIV[0]{\hyperlink{ast-div}{\texttt{DIV}}} % Integer division
\renewcommand\DIVRM[0]{\hyperlink{ast-divrm}{\texttt{DIVRM}}} % Inexact integer division, with rounding towards negative infinity.
\newcommand\XOR[0]{\hyperlink{ast-xor}{\texttt{XOR}}} % Bitvector bitwise exclusive or
\renewcommand\EQOP[0]{\hyperlink{ast-eqop}{\texttt{EQ\_OP}}} % Equality on two base values of same type
\renewcommand\GT[0]{\hyperlink{ast-gt}{\texttt{GT}}} % Greater than for int or reals
\renewcommand\GEQ[0]{\hyperlink{ast-geq}{\texttt{GEQ}}} % Greater or equal for int or reals
\renewcommand\IMPL[0]{\hyperlink{ast-impl}{\texttt{IMPL}}} % Boolean implication
\renewcommand\LT[0]{\hyperlink{ast-lt}{\texttt{LT}}} % Less than for int or reals
\renewcommand\LEQ[0]{\hyperlink{ast-leq}{\texttt{LEQ}}} % Less or equal for int or reals
\renewcommand\MOD[0]{\hyperlink{ast-mod}{\texttt{MOD}}} % Remainder of integer division
\renewcommand\MINUS[0]{\hyperlink{ast-minus}{\texttt{MINUS}}} % Subtraction for int or reals or bitvectors
\renewcommand\MUL[0]{\hyperlink{ast-mul}{\texttt{MUL}}} % Multiplication for int or reals or bitvectors
\renewcommand\NEQ[0]{\hyperlink{ast-neq}{\texttt{NEQ}}} % Non equality on two base values of same type
\renewcommand\OR[0]{\hyperlink{ast-or}{\texttt{OR}}} % Bitvector bitwise or
\renewcommand\PLUS[0]{\hyperlink{ast-plus}{\texttt{PLUS}}} % Addition for int or reals or bitvectors
\renewcommand\POW[0]{\hyperlink{ast-pow}{\texttt{POW}}} % Exponentiation for integers
\renewcommand\RDIV[0]{\hyperlink{ast-rdiv}{\texttt{RDIV}}} % Division for reals
\renewcommand\SHL[0]{\hyperlink{ast-shl}{\texttt{SHL}}} % Shift left for integers
\renewcommand\SHR[0]{\hyperlink{ast-shr}{\texttt{SHR}}} % Shift right for integers

\renewcommand\UNKNOWN[0]{\hyperlink{ast-unknown}{\texttt{UNKNOWN}}}

% For loop direction
\renewcommand\UP[0]{\hyperlink{ast-up}{\texttt{Up}}}
\renewcommand\DOWN[0]{\hyperlink{ast-down}{\texttt{Down}}}

% Non-terminal names
\renewcommand\unop[0]{\hyperlink{ast-unop}{\textsf{unop}}}
\renewcommand\binop[0]{\hyperlink{ast-binop}{\textsf{binop}}}
\renewcommand\literal[0]{\hyperlink{ast-literal}{\textsf{literal}}}
\renewcommand\expr[0]{\hyperlink{ast-expr}{\textsf{expr}}}
\renewcommand\lexpr[0]{\hyperlink{ast-lexpr}{\textsf{lexpr}}}
\renewcommand\slice[0]{\hyperlink{ast-slice}{\textsf{slice}}}
\renewcommand\arrayindex[0]{\hyperlink{ast-arrayindex}{\textsf{array\_index}}}

\renewcommand\ty[0]{\hyperlink{ast-ty}{\textsf{ty}}}
\renewcommand\pattern[0]{\hyperlink{ast-pattern}{\textsf{pattern}}}
\renewcommand\intconstraints[0]{\hyperlink{ast-intconstraints}{\textsf{int\_constraints}}}
\renewcommand\intconstraint[0]{\hyperlink{ast-intconstraint}{\textsf{int\_constraint}}}
\renewcommand\unconstrained[0]{\hyperlink{ast-unconstrained}{\textsf{Unconstrained}}}
\renewcommand\wellconstrained[0]{\hyperlink{ast-wellconstrained}{\textsf{WellConstrained}}}
\renewcommand\parameterized[0]{\hyperlink{ast-parameterized}{\textsf{Parameterized}}}
\renewcommand\constraintexact[0]{\hyperlink{ast-constraintexact}{\textsf{Constraint\_Exact}}}
\renewcommand\constraintrange[0]{\hyperlink{ast-constraintrange}{\textsf{Constraint\_Range}}}
\renewcommand\bitfield[0]{\hyperlink{ast-bitfield}{\textsf{bitfield}}}
\renewcommand\spec[0]{\hyperlink{ast-spec}{\textsf{spec}}}
\renewcommand\typedidentifier[0]{\hyperlink{ast-typedidentifier}{\textsf{typed\_identifier}}}
\renewcommand\localdeclkeyword[0]{\hyperlink{ast-localdeclkeyword}{\textsf{local\_decl\_keyword}}}
\renewcommand\globaldeclkeyword[0]{\hyperlink{ast-globaldeclkeyword}{\textsf{global\_decl\_keyword}}}
\renewcommand\localdeclitem[0]{\hyperlink{ast-localdeclitem}{\textsf{local\_decl\_item}}}
\renewcommand\globaldecl[0]{\hyperlink{ast-globaldecl}{\textsf{global\_decl}}}
\renewcommand\fordirection[0]{\hyperlink{ast-fordirection}{\textsf{for\_direction}}}
\renewcommand\stmt[0]{\hyperlink{ast-stmt}{\textsf{stmt}}}
\renewcommand\decl[0]{\hyperlink{ast-decl}{\textsf{decl}}}
\renewcommand\casealt[0]{\hyperlink{ast-casealt}{\textsf{case\_alt}}}
\renewcommand\catcher[0]{\hyperlink{ast-catcher}{\textsf{catcher}}}
\renewcommand\subprogramtype[0]{\hyperlink{ast-subprogramtype}{\textsf{sub\_program\_type}}}
\renewcommand\subprogrambody[0]{\hyperlink{ast-subprogrambody}{\textsf{sub\_program\_body}}}
\renewcommand\func[0]{\hyperlink{ast-func}{\textsf{func}}}
\renewcommand\Field[0]{\hyperlink{ast-field}{\textsf{field}}}

% Expression labels
\renewcommand\ELiteral[0]{\hyperlink{ast-eliteral}{\textsf{E\_Literal}}}
\renewcommand\EVar[0]{\hyperlink{ast-evar}{\textsf{E\_Var}}}
\renewcommand\EATC[0]{\hyperlink{ast-eatc}{\textsf{E\_ATC}}}
\renewcommand\EBinop[0]{\hyperlink{ast-ebinop}{\textsf{E\_Binop}}}
\renewcommand\EUnop[0]{\hyperlink{ast-eunop}{\textsf{E\_Unop}}}
\renewcommand\ECall[0]{\hyperlink{ast-ecall}{\textsf{E\_Call}}}
\renewcommand\ESlice[0]{\hyperlink{ast-eslice}{\textsf{E\_Slice}}}
\renewcommand\ECond[0]{\hyperlink{ast-econd}{\textsf{E\_Cond}}}
\renewcommand\EGetArray[0]{\hyperlink{ast-egetarray}{\textsf{E\_GetArray}}}
\renewcommand\EGetField[0]{\hyperlink{ast-egetfield}{\textsf{E\_GetField}}}
\renewcommand\EGetItem[0]{\hyperlink{ast-egetitem}{\textsf{E\_GetItem}}}
\renewcommand\EGetFields[0]{\hyperlink{ast-getfields}{\textsf{E\_GetFields}}}
\renewcommand\ERecord[0]{\hyperlink{ast-erecord}{\textsf{E\_Record}}}
\renewcommand\EConcat[0]{\hyperlink{ast-econcat}{\textsf{E\_Concat}}}
\renewcommand\ETuple[0]{\hyperlink{ast-etuple}{\textsf{E\_Tuple}}}
\renewcommand\EUnknown[0]{\hyperlink{ast-eunknown}{\textsf{E\_Unknown}}}
\renewcommand\EPattern[0]{\hyperlink{ast-epattern}{\textsf{E\_Pattern}}}

% Left-hand-side expression labels
\renewcommand\LEConcat[0]{\hyperlink{ast-leconcat}{\textsf{LE\_Concat}}}
\renewcommand\LEDiscard[0]{\hyperlink{ast-lediscard}{\textsf{LE\_Discard}}}
\renewcommand\LEVar[0]{\hyperlink{ast-levar}{\textsf{LE\_Var}}}
\renewcommand\LESlice{\hyperlink{ast-leslice}{\textsf{LE\_Slice}}}
\renewcommand\LESetArray[0]{\hyperlink{ast-lesetarray}{\textsf{LE\_SetArray}}}
\renewcommand\LESetField[0]{\hyperlink{ast-lesetfield}{\textsf{LE\_SetField}}}
\renewcommand\LESetFields[0]{\hyperlink{ast-lesetfields}{\textsf{LE\_SetFields}}}
\renewcommand\LEDestructuring[0]{\hyperlink{ast-ledestructuring}{\textsf{LE\_Destructuring}}}

% Statement labels
\renewcommand\SPass[0]{\hyperlink{ast-spass}{\textsf{S\_Pass}}}
\renewcommand\SAssign[0]{\hyperlink{ast-sassign}{\textsf{S\_Assign}}}
\renewcommand\SReturn[0]{\hyperlink{ast-sreturn}{\textsf{S\_Return}}}
\renewcommand\SSeq[0]{\hyperlink{ast-sseq}{\textsf{S\_Seq}}}
\renewcommand\SCall[0]{\hyperlink{ast-scall}{\textsf{S\_Call}}}
\renewcommand\SCond[0]{\hyperlink{ast-scond}{\textsf{S\_Cond}}}
\renewcommand\SCase[0]{\hyperlink{ast-scase}{\textsf{S\_Case}}}
\renewcommand\SDecl[0]{\hyperlink{ast-sdecl}{\textsf{S\_Decl}}}
\renewcommand\SAssert[0]{\hyperlink{ast-sassert}{\textsf{S\_Assert}}}
\renewcommand\SWhile[0]{\hyperlink{ast-swhile}{\textsf{S\_While}}}
\renewcommand\SRepeat[0]{\hyperlink{ast-srepeat}{\textsf{S\_Repeat}}}
\renewcommand\SFor[0]{\hyperlink{ast-sfor}{\textsf{S\_For}}}
\renewcommand\SThrow[0]{\hyperlink{ast-sthrow}{\textsf{S\_Throw}}}
\renewcommand\STry[0]{\hyperlink{ast-stry}{\textsf{S\_Try}}}
\renewcommand\SPrint[0]{\hyperlink{ast-sprint}{\textsf{S\_Print}}}

% Literal labels
\renewcommand\lint[0]{\hyperlink{ast-lint}{\textsf{L\_Int}}}
\renewcommand\lbool[0]{\hyperlink{ast-lbool}{\textsf{L\_Bool}}}
\renewcommand\lreal[0]{\hyperlink{ast-lreal}{\textsf{L\_Real}}}
\renewcommand\lbitvector[0]{\hyperlink{ast-lbitvector}{\textsf{L\_Bitvector}}}
\renewcommand\lstring[0]{\hyperlink{ast-lstring}{\textsf{L\_String}}}

% Type labels
\renewcommand\TInt[0]{\hyperlink{ast-tint}{\textsf{T\_Int}}}
\renewcommand\TReal[0]{\hyperlink{ast-treal}{\textsf{T\_Real}}}
\renewcommand\TString[0]{\hyperlink{ast-tstring}{\textsf{T\_String}}}
\renewcommand\TBool[0]{\hyperlink{ast-tbool}{\textsf{T\_Bool}}}
\renewcommand\TBits[0]{\hyperlink{ast-tbits}{\textsf{T\_Bits}}}
\renewcommand\TEnum[0]{\hyperlink{ast-tenum}{\textsf{T\_Enum}}}
\renewcommand\TTuple[0]{\hyperlink{ast-ttuple}{\textsf{T\_Tuple}}}
\renewcommand\TArray[0]{\hyperlink{ast-tarray}{\textsf{T\_Array}}}
\renewcommand\TRecord[0]{\hyperlink{ast-trecord}{\textsf{T\_Record}}}
\renewcommand\TException[0]{\hyperlink{ast-texception}{\textsf{T\_Exception}}}
\renewcommand\TNamed[0]{\hyperlink{ast-tnamed}{\textsf{T\_Named}}}

\renewcommand\BitFieldSimple[0]{\hyperlink{ast-bitfieldsimple}{\textsf{BitField\_Simple}}}
\renewcommand\BitFieldNested[0]{\hyperlink{ast-bitfieldnested}{\textsf{BitField\_Nested}}}
\renewcommand\BitFieldType[0]{\hyperlink{ast-bitfieldtype}{\textsf{BitField\_Type}}}

\renewcommand\ConstraintExact[0]{\hyperlink{ast-constraintexact}{\textsf{Constraint\_Exact}}}
\renewcommand\ConstraintRange[0]{\hyperlink{ast-constraintrange}{\textsf{Constraint\_Range}}}

% Array index labels
\renewcommand\ArrayLengthExpr[0]{\hyperlink{ast-arraylengthexpr}{\textsf{ArrayLength\_Expr}}}
\renewcommand\ArrayLengthEnum[0]{\hyperlink{ast-arraylengthenum}{\textsf{ArrayLength\_Enum}}}

% Slice labels
\renewcommand\SliceSingle[0]{\hyperlink{ast-slicesingle}{\textsf{Slice\_Single}}}
\renewcommand\SliceRange[0]{\hyperlink{ast-slicerange}{\textsf{Slice\_Range}}}
\renewcommand\SliceLength[0]{\hyperlink{ast-slicelength}{\textsf{Slice\_Length}}}
\renewcommand\SliceStar[0]{\hyperlink{ast-slicestar}{\textsf{Slice\_Star}}}

% Pattern labels
\renewcommand\PatternAll[0]{\hyperlink{ast-patternall}{\textsf{Pattern\_All}}}
\renewcommand\PatternAny[0]{\hyperlink{ast-patternany}{\textsf{Pattern\_Any}}}
\renewcommand\PatternGeq[0]{\hyperlink{ast-patterngeq}{\textsf{Pattern\_Geq}}}
\renewcommand\PatternLeq[0]{\hyperlink{ast-patternleq}{\textsf{Pattern\_Leq}}}
\renewcommand\PatternNot[0]{\hyperlink{ast-patternnot}{\textsf{Pattern\_Not}}}
\renewcommand\PatternRange[0]{\hyperlink{ast-patternrange}{\textsf{Pattern\_Range}}}
\renewcommand\PatternSingle[0]{\hyperlink{ast-patternsingle}{\textsf{Pattern\_Single}}}
\renewcommand\PatternMask[0]{\hyperlink{ast-patternmask}{\textsf{Pattern\_Mask}}}
\renewcommand\PatternTuple[0]{\hyperlink{ast-patterntuple}{\textsf{Pattern\_Tuple}}}

% Local declarations
\renewcommand\LDIDiscard[0]{\hyperlink{ast-ldidiscard}{\textsf{LDI\_Discard}}}
\renewcommand\LDIVar[0]{\hyperlink{ast-ldivar}{\textsf{LDI\_Var}}}
\renewcommand\LDITyped[0]{\hyperlink{ast-ldityped}{\textsf{LDI\_Typed}}}
\renewcommand\LDITuple[0]{\hyperlink{ast-ldituple}{\textsf{LDI\_Tuple}}}

\renewcommand\LDKVar[0]{\hyperlink{ast-ldkvar}{\textsf{LDK\_Var}}}
\renewcommand\LDKConstant[0]{\hyperlink{ast-ldkconstant}{\textsf{LDK\_Constant}}}
\renewcommand\LDKLet[0]{\hyperlink{ast-ldklet}{\textsf{LDK\_Let}}}

\renewcommand\GDKConstant[0]{\hyperlink{ast-gdkconstant}{\textsf{GDK\_Constant}}}
\renewcommand\GDKConfig[0]{\hyperlink{ast-gdkconfig}{\textsf{GDK\_Config}}}
\renewcommand\GDKLet[0]{\hyperlink{ast-gdklet}{\textsf{GDK\_Let}}}
\renewcommand\GDKVar[0]{\hyperlink{ast-gdkvar}{\textsf{GDK\_Var}}}

\renewcommand\SBASL[0]{\hyperlink{ast-sbasl}{\textsf{SB\_ASL}}}
\renewcommand\SBPrimitive[0]{\hyperlink{ast-sbprimitive}{\textsf{SB\_Primitive}}}

\renewcommand\STFunction[0]{\hyperlink{ast-stfunction}{\texttt{ST\_Function}}}
\renewcommand\STGetter[0]{\hyperlink{ast-stgetter}{\texttt{ST\_Getter}}}
\renewcommand\STEmptyGetter[0]{\hyperlink{ast-stemptygetter}{\texttt{ST\_EmptyGetter}}}
\renewcommand\STSetter[0]{\hyperlink{ast-stsetter}{\texttt{ST\_Setter}}}
\renewcommand\STEmptySetter[0]{\hyperlink{ast-stemptysetter}{\texttt{ST\_EmptySetter}}}
\renewcommand\STProcedure[0]{\hyperlink{ast-stprocedure}{\texttt{ST\_Procedure}}}

\renewcommand\DFunc[0]{\hyperlink{ast-dfunc}{\texttt{D\_Func}}}
\renewcommand\DGlobalStorage[0]{\hyperlink{ast-dglobalstorage}{\texttt{D\_GlobalStorage}}}
\renewcommand\DTypeDecl[0]{\hyperlink{ast-dtypedecl}{\texttt{D\_TypeDecl}}}

\renewcommand\specification[0]{\hyperlink{ast-specification}{\texttt{specification}}}
%%%%%%%%%%%%%%%%%%%%%%%%%%%%%%%%%%%%%%%%%%%%%%%%%%%%%%%%%%%%%%%%%%%%%%%%%%%%%%%

\newcommand\astarrow[0]{\xrightarrow{\textsf{ast}}}
\newcommand\scanarrow[0]{\xrightarrow{\textsf{scan}}}
%\newcommand\productionname[2]{[\hyperlink{build-#1}{\textsc{#2}}]}
\newcommand\productionname[2]{}

\newcommand\buildidentity[0]{\hyperlink{build-identity}{\textsf{build\_identity}}}
\newcommand\buildlist[0]{\hyperlink{build-list}{\textsf{build\_list}}}
\newcommand\buildplist[0]{\hyperlink{build-plist}{\textsf{build\_plist}}}
\newcommand\buildclist[0]{\hyperlink{build-clist}{\textsf{build\_clist}}}
\newcommand\buildntclist[0]{\hyperlink{build-ntclist}{\textsf{build\_ntclist}}}
\newcommand\buildtclist[0]{\hyperlink{build-tclist}{\textsf{build\_tclist}}}
\newcommand\buildoption[0]{\hyperlink{build-option}{\textsf{build\_option}}}

\newcommand\buildast[0]{\hyperlink{build-ast}{\textsf{build\_ast}}}
\newcommand\builddecl[0]{\hyperlink{build-decl}{\textsf{build\_decl}}}
\newcommand\builddeclitem[0]{\hyperlink{build-declitem}{\textsf{build\_decl\_item}}}
\newcommand\builduntypeddeclitem[0]{\hyperlink{build-untypeddeclitem}{\textsf{build\_untyped\_decl\_item}}}
\newcommand\buildstmt[0]{\hyperlink{build-stmt}{\textsf{build\_stmt}}}
\newcommand\buildexpr[0]{\hyperlink{build-expr}{\textsf{build\_expr}}}
\newcommand\buildlexpr[0]{\hyperlink{build-lexpr}{\textsf{build\_lexpr}}}
\newcommand\buildlexpratom[0]{\hyperlink{build-lexpratom}{\textsf{build\_lexpr\_atom}}}
\newcommand\buildparamsopt[0]{\hyperlink{build-paramsopt}{\textsf{build\_params\_opt}}}
\newcommand\buildaccessargs[0]{\hyperlink{build-accessargs}{\textsf{build\_access\_args}}}
\newcommand\buildfuncargs[0]{\hyperlink{build-funcargs}{\textsf{build\_func\_args}}}
\newcommand\buildreturntype[0]{\hyperlink{build-returntype}{\textsf{build\_return\_type}}}
\newcommand\buildfuncbody[0]{\hyperlink{build-funcbody}{\textsf{build\_func\_body}}}
\newcommand\buildtypedidentifier[0]{\hyperlink{build-typedidentifier}{\textsf{build\_typed\_identifier}}}
\newcommand\buildopttypedidentifier[0]{\hyperlink{build-opttypedidentifier}{\textsf{build\_opt\_typed\_identifier}}}
\newcommand\buildtydecl[0]{\hyperlink{build-tydecl}{\textsf{build\_ty\_decl}}}
\newcommand\buildsubtype[0]{\hyperlink{build-subtype}{\textsf{build\_subtype}}}
\newcommand\buildsubtypeopt[0]{\hyperlink{build-subtypeopt}{\textsf{build\_subtype\_opt}}}
\newcommand\buildstoragekeyword[0]{\hyperlink{build-storagekeyword}{\textsf{build\_storage\_keyword}}}
\newcommand\builddirection[0]{\hyperlink{build-direction}{\textsf{build\_direction}}}
\newcommand\buildalt[0]{\hyperlink{build-alt}{\textsf{build\_alt}}}
\newcommand\buildotherwiseopt[0]{\hyperlink{build-otherwiseopt}{\textsf{build\_otherwise\_opt}}}
\newcommand\buildcatcher[0]{\hyperlink{build-catcher}{\textsf{build\_catcher}}}
\newcommand\buildasty[0]{\textsf{build\_as\_ty}}
\newcommand\buildignoredoridentifier[0]{\hyperlink{build-ignoredoridentifier}{\textsf{build\_ignored\_or\_identifier}}}
\newcommand\buildlocaldeclkeyword[0]{\hyperlink{build-localdeclkeyword}{\textsf{build\_local\_decl\_keyword}}}
\newcommand\buildstmtlist[0]{\hyperlink{build-stmtlist}{\textsf{build\_stmt\_list}}}
\newcommand\buildmaybeemptystmtlist[0]{\hyperlink{build-maybeemptystmtlist}{\textsf{build\_maybe\_empty\_stmt\_list}}}
\newcommand\buildselse[0]{\hyperlink{build-selse}{\textsf{build\_s\_else}}}
\newcommand\buildintconstraints[0]{\hyperlink{build-intconstraints}{\textsf{build\_int\_constraints}}}
\newcommand\buildintconstraintsopt[0]{\hyperlink{build-intconstraintsopt}{\textsf{build\_int\_constraints\_opt}}}
\newcommand\buildintconstraint[0]{\hyperlink{build-intconstraint}{\textsf{build\_int\_constraint}}}
\newcommand\buildexprpattern[0]{\hyperlink{build-exprpattern}{\textsf{build\_expr\_pattern}}}
\newcommand\buildfieldassign[0]{\hyperlink{build-fieldassign}{\textsf{build\_field\_assign}}}
\newcommand\buildpatternset[0]{\hyperlink{build-patternset}{\textsf{build\_pattern\_set}}}
\newcommand\buildpatternlist[0]{\hyperlink{build-patternlist}{\textsf{build\_pattern\_list}}}
\newcommand\buildpattern[0]{\hyperlink{build-pattern}{\textsf{build\_pattern}}}
\newcommand\buildfields[0]{\hyperlink{build-fields}{\textsf{build\_fields}}}
\newcommand\buildfieldsopt[0]{\hyperlink{build-fieldsopt}{\textsf{build\_fields\_opt}}}
\newcommand\buildnslices[0]{\hyperlink{build-nslices}{\textsf{build\_nslices}}}
\newcommand\buildslices[0]{\hyperlink{build-slices}{\textsf{build\_slices}}}
\newcommand\buildslice[0]{\hyperlink{build-slice}{\textsf{build\_slice}}}
\newcommand\buildbitfields[0]{\hyperlink{build-bitfields}{\textsf{build\_bitfields}}}
\newcommand\buildbitfield[0]{\hyperlink{build-bitfield}{\textsf{build\_bitfield}}}
\newcommand\buildty[0]{\hyperlink{build-ty}{\textsf{build\_ty}}}
\newcommand\buildeelse[0]{\hyperlink{build-eelse}{\textsf{build\_e\_else}}}
\newcommand\buildvalue[0]{\hyperlink{build-value}{\textsf{build\_value}}}
\newcommand\buildunop[0]{\hyperlink{build-unop}{\textsf{build\_unop}}}
\newcommand\buildbinop[0]{\hyperlink{build-binop}{\textsf{build\_binop}}}

\newcommand\stmtfromlist[0]{\hyperlink{def-stmtfromlist}{\textsf{stmt\_from\_list}}}
\newcommand\sequencestmts[0]{\hyperlink{def-sequencestmts}{\textsf{sequence\_stmts}}}

\newcommand\eof[0]{\hyperlink{def-eof}{\textsf{eof}}}

\newcommand\Char[1]{\fbox{#1}}
\newcommand\SpaceChar[0]{\framebox[1cm]{\color{white}A}}
%\newcommand\Underscore[0]{\fbox{\_}}
\newcommand\Underscore[0]{'\_'}

\newcommand\REasciichar[0]{\hyperlink{def-reasciichar}{\texttt{<}\textsf{ascii\_char}\texttt{>}}}
\newcommand\REdigit[0]{\hyperlink{def-redigit}{\texttt{<}\textsf{digit}\texttt{>}}}
\newcommand\REintlit[0]{\hyperlink{def-reintlit}{\texttt{<}\textsf{int\_lit}\texttt{>}}}
\newcommand\REhexlit[0]{\hyperlink{def-rehexlit}{\texttt{<}\textsf{hex\_lit}\texttt{>}}}
\newcommand\REreallit[0]{\hyperlink{def-reallit}{\texttt{<}\textsf{real\_lit}\texttt{>}}}
\newcommand\REstrchar[0]{\hyperlink{def-restrchar}{\texttt{<}\textsf{str\_char}\texttt{>}}}
\newcommand\REstringlit[0]{\hyperlink{def-restringlit}{\texttt{<}\textsf{string\_lit}\texttt{>}}}
\newcommand\REbitvectorlit[0]{\hyperlink{def-rebitvectorlit}{\texttt{<}\textsf{bitvector\_lit}\texttt{>}}}
\newcommand\REbitmasklit[0]{\hyperlink{def-rebitmasklit}{\texttt{<}\textsf{bitmask\_lit}\texttt{>}}}
\newcommand\REletter[0]{\hyperlink{def-reletter}{\texttt{<}\textsf{letter}\texttt{>}}}
\newcommand\REidentifier[0]{\hyperlink{def-reidentifier}{\texttt{<}\textsf{identifier}\texttt{>}}}
\newcommand\REcomment[0]{\hyperlink{def-recomment}{\texttt{<}\textsf{comment}\texttt{>}}}

\newcommand\aslparse[0]{\textsf{asl\_parse}}
\newcommand\aslscan[0]{\hyperlink{def-aslscan}{\textsf{scan}}}
\newcommand\maxmatches[0]{\hyperlink{def-maxmatch}{\textsf{max\_matches}}}
\newcommand\remaxmatch[0]{\hyperlink{def-rematch}{\textsf{re\_max\_match}}}
\newcommand\Token[0]{\hyperlink{def-token}{\textsf{Token}}}
\newcommand\LexicalError[0]{\hyperlink{def-lexicalerrorresult}{\textsf{\#LE}}}
\newcommand\ParseError[0]{\textsf{\#PE}}
\newcommand\discard[0]{\hyperlink{def-discard}{\textsf{discard}}}
\newcommand\booltolit[0]{\hyperlink{def-booltolit}{\textsf{bool\_to\_lit}}}
\newcommand\decimaltolit[0]{\hyperlink{def-decimaltolit}{\textsf{dec\_to\_lit}}}
\newcommand\hextolit[0]{\hyperlink{def-hextolit}{\textsf{hex\_to\_lit}}}
\newcommand\realtolit[0]{\hyperlink{def-realtolit}{\textsf{real\_to\_lit}}}
\newcommand\strtolit[0]{\hyperlink{def-strtolit}{\textsf{str\_to\_lit}}}
\newcommand\bitstolit[0]{\hyperlink{def-bitstolit}{\textsf{bits\_to\_lit}}}
\newcommand\masktolit[0]{\hyperlink{def-masktolit}{\textsf{mask\_to\_lit}}}
\newcommand\truetolit[0]{\hyperlink{def-truetolit}{\textsf{true\_to\_lit}}}
\newcommand\falsetolit[0]{\hyperlink{def-falsetolit}{\textsf{false\_to\_lit}}}
\newcommand\tokenid[0]{\hyperlink{def-tokenid}{\textsf{token\_id}}}
\newcommand\lexicalerror[0]{\hyperlink{def-lexicalerror}{\textsf{lexical\_error}}}
\newcommand\toidentifier[0]{\hyperlink{def-toidentifier}{\textsf{to\_identifier}}}
\newcommand\eoftoken[0]{\hyperlink{def-eoftoken}{\textsf{eof\_token}}}
\newcommand\Teof[0]{\hyperlink{def-teof}{\textsf{EOF}}}
\newcommand\Terror[0]{\hyperlink{def-terror}{\textsf{ERROR}}}
\newcommand\Twhitespace[0]{\hyperlink{def-twhitespace}{\textsf{WHITE\_SPACE}}}
\newcommand\LexSpec[0]{\hyperlink{def-lexspec}{\textsf{LexSpec}}}
%\newcommand\Teof[0]{\verbatimterminal{EOF}{EOF}}
\newcommand\Lang[0]{\hyperlink{def-lang}{\textsf{Lang}}}
\newcommand\RegExp[0]{\hyperlink{def-regex}{\textsf{RegExp}}}

\newcommand\name[0]{\texttt{name}}

\newcommand\astversion[1]{#1\texttt{\_ast}}

\author{Arm Architecture Technology Group}
\title{ASL Syntax Reference \\
       DDI 0620}
\begin{document}
\maketitle

\tableofcontents{}

\chapter{Non-Confidential Proprietary Notice}

This document is protected by copyright and other related rights and the
practice or implementation of the information contained in this document may be
protected by one or more patents or pending patent applications. No part of
this document may be reproduced in any form by any means without the express
prior written permission of Arm. No license, express or implied, by estoppel or
otherwise to any intellectual property rights is granted by this document
unless specifically stated.

Your access to the information in this document is conditional upon your
acceptance that you will not use or permit others to use the information for
the purposes of determining whether implementations infringe any third party
patents.

THIS DOCUMENT IS PROVIDED “AS IS”. ARM PROVIDES NO REPRESENTATIONS AND NO
WARRANTIES, EXPRESS, IMPLIED OR STATUTORY, INCLUDING, WITHOUT LIMITATION, THE
IMPLIED WARRAN-TIES OF MERCHANTABILITY, SATISFACTORY QUALITY, NON-INFRIN-GEMENT
OR FITNESS FOR A PARTICULAR PURPOSE WITH RESPECT TO THE DOCUMENT. For the
avoidance of doubt, Arm makes no representation with respect to, and has
undertaken no analysis to identify or understand the scope and content of, any
patents, copyrights, trade secrets, trademarks, or other rights.

This document may include technical inaccuracies or typographical errors.

TO THE EXTENT NOT PROHIBITED BY LAW, IN NO EVENT WILL ARM BE LIABLE FOR ANY
DAMAGES, INCLUDING WITHOUT LIMITATION ANY DIRECT, INDIRECT, SPECIAL,
INCIDENTAL, PUNITIVE, OR CONSEQUENTIAL DAMAGES, HOWEVER CAUSED AND REGARDLESS
OF THE THEORY OF LIABILITY, ARISING OUT OF ANY USE OF THIS DOCUMENT, EVEN IF
ARM HAS BEEN ADVISED OF THE POSSIBILITY OF SUCH DAMAGES.

This document consists solely of commercial items. You shall be responsible for
ensuring that any use, duplication or disclosure of this document complies
fully with any relevant export laws and regulations to assure that this
document or any portion thereof is not exported, directly or indirectly, in
violation of such export laws. Use of the word “partner” in reference to Arm’s
customers is not intended to create or refer to any partnership relationship
with any other company. Arm may make changes to this document at any time and
without notice.

This document may be translated into other languages for convenience, and you
agree that if there is any conflict between the English version of this
document and any translation, the terms of the English version of this document
shall prevail.

The Arm corporate logo and words marked with ® or ™ are registered trademarks
or trademarks of Arm Limited (or its affiliates) in the US and/or elsewhere.
All rights reserved.  Other brands and names mentioned in this document may be
the trademarks of their respective owners. Please follow Arm’s trademark usage
guidelines at

\url{https://www.arm.com/company/policies/trademarks.}

Copyright © [2023,2024] Arm Limited (or its affiliates). All rights reserved.

Arm Limited. Company 02557590 registered in England.  110 Fulbourn Road,
Cambridge, England CB1 9NJ.  (LES-PRE-20349)


\chapter{Disclaimer}

This document is part of the ASLRef material.

This material covers ASLv1, a new, experimental, and as yet unreleased version of ASL.

The development version of ASLRef can be found here: \\
\url{https://github.com/herd/herdtools7}.

A list of open items being worked on can be found here: \\
\url{https://github.com/herd/herdtools7/blob/master/asllib/doc/ASLRefProgress.tex}.

This material is work in progress, more precisely at Alpha quality as
per Arm’s quality standards. In particular, this means that it would be
premature to base any production tool development on this material.

However, any feedback, question, query and feature request would be most
welcome; those can be sent to Arm’s Architecture Formal Team Lead Jade Alglave
\texttt{(jade.alglave@arm.com)} or by raising issues or PRs to the herdtools7
github repository.


%%%%%%%%%%%%%%%%%%%%%%%%%%%%%%%%%%%%%%%%%%%%%%%%%%%%%%%%%%%%%%%%%%%%%%%%%%%%%%%%
\chapter{Introduction}
%%%%%%%%%%%%%%%%%%%%%%%%%%%%%%%%%%%%%%%%%%%%%%%%%%%%%%%%%%%%%%%%%%%%%%%%%%%%%%%%
This document defines how an ASL specification, given as text, can be transformed into an \emph{abstract syntax tree},
which is a tree-like data structure. This transformation occurs in three stages:

\begin{description}
  \item[Lexical Analysis] The text is first transformed into a list of \emph{tokens}.
        This stage is defined in \chapref{lexicalanalysis};
  \item[Parsing] The list of tokens is transformed into a \emph{parse tree}.
        This stage is defined in \chapref{parsing};
  \item[Abstraction] The parse tree is transformed into an abstract syntax tree. This is a conceptual stage. In actuality,
        the parsing stage transforms the list of tokens directly into an abstract syntax tree. However, it is useful to
        distinguish between the parsing state and the abstraction stage.
        ASL abstract syntax trees are defined in \chapref{ASLAbstractSyntax}.
        This stage is defined in \chapref{BuildingAbstractSyntaxTrees}.
\end{description}

% The following macros will be moved to ASLmacros.tex when we unify all reference documents.
\newcommand\ReadEffect[0]{\textsf{ReadEffect}}
\newcommand\Normal[0]{\textsf{Normal}}
\newcommand\ThrowingConfig[0]{\texttt{\#T}}
\newcommand\OrAbnormal[0]{\terminateas \ThrowingConfig, \ErrorConfig}
\newcommand\vg[0]{\texttt{g}}
\newcommand\env[0]{\texttt{env}}
\newcommand\parallelcomp[0]{\parallel}
\newcommand\binoprel[0]{\texttt{binop}}
\newcommand\unoprel[0]{\texttt{unop}}
\newcommand\evalexprsef[1]{\hyperlink{def-evalexprsef}{\texttt{eval\_expr\_sef}}(#1)}
\newcommand\XGraphs[0]{\mathcal{G}}
\newcommand\TError[0]{\textsf{TDynError}}
\newcommand\ErrorConfig[0]{\hyperlink{def-errorconfig}{\texttt{\#DE}}}
\newcommand\annotaterel[0]{\textsf{type}}
\newcommand\typearrow[0]{\xrightarrow{\annotaterel}}
\newcommand\evalexpr[1]{\texttt{eval\_expr}(#1)}
\newcommand\vt[0]{\texttt{t}}
\newcommand\veone[0]{\texttt{e1}}
\newcommand\vetwo[0]{\texttt{e2}}
\newcommand\vvone[0]{\texttt{v1}}
\newcommand\vvtwo[0]{\texttt{v2}}
\newcommand\vl[0]{\texttt{l}}
\newcommand\vm[0]{\texttt{m}}
\newcommand\vmone[0]{\texttt{m1}}
\newcommand\vmtwo[0]{\texttt{m2}}
\chapter{Formal System \label{chap:FormalSystem}}

In this part, we define the mathematical concepts and notations used throughout.
We start by defining general mathematical concepts
and then describe how sets of rules formally define functions and relations.

\section{Mathematical Definitions and Notations}

\hypertarget{def-triangleq}{}
We use $\triangleq$ to define mathematical concepts.

We define the following sets:
\begin{itemize}
\item \hypertarget{def-N}{
    $\N$ is the set of natural numbers, including $0$.
}

\item \hypertarget{def-Npos}{
    $\Npos$ is the set of natural numbers, excluding $0$.
}

\item
\hypertarget{def-Z}{
    $\Z$ is the set of integers.
}

\item
\hypertarget{def-Q}{
    $\Q$ is the set of rationals.
}

\hypertarget{def-bool}{}
\hypertarget{def-false}{}
\hypertarget{def-true}{}
\item $\Bool$ is the set of ASL Boolean literals, which consists of $\True$ and $\False$.
We employ these literals to represent the corresponding mathematical truth values,
which are used to denote whether logical assertions hold or not.
\hypertarget{def-land}{}
\hypertarget{def-lor}{}
We also employ the mathematical meaning of logical conjunction $\land$, logical disjunction $\lor$,
and logical negation $\neg$, given next.
For a set of Boolean values $A$:
\[
  \begin{array}{rcl}
  \land A &\triangleq&
  \begin{cases}
    \True & \text{if all values in A are }\True\\
    \False & \text{otherwise}
  \end{cases}\\
  \lor A &\triangleq&
  \begin{cases}
    \False & \text{if all values in A are }\False\\
    \True & \text{otherwise}
  \end{cases}\\
\end{array}
\]
\hypertarget{def-neg}{}
For a pair of Boolean values $a,b\in\Bool$, we define $a \land b \triangleq \land\{a, b\}$
and $a \lor b \triangleq \lor\{a, b\}$.
Finally, $\neg\True\triangleq\False$ and $\neg\False\triangleq\True$.

\item
\hypertarget{def-identifier}{}
    $\Identifiers$ is the set of all ASL identifiers.

\item
\hypertarget{def-astlabels}{}
    $\astlabels$ is the set of all labels of Abstract Syntax Tree (AST) nodes.

\item
\hypertarget{def-strings}{}
    $\Strings$ is the set of all ASCII strings.
\end{itemize}

We utilize the notation $\overname{a}{b}$ to enable us to name the mathematical term $a$ as $b$ so that
we can refer to it in text. We especially use this to name the input arguments and
output results of functions and relations. For example, the input argument of $\sign$,
which is defined next is named $q$.

\hypertarget{def-sign}{}
\begin{definition}[Sign of a Rational Number]
  The function $\sign : \overname{\Q}{q} \rightarrow \{-1,0,1\}$ returns the sign of $\vq$:
\[
\sign(q) \triangleq \begin{cases}
  1 & \text{if }q > 0\\
  0 & \text{if }q = 0\\
  -1 & \text{if }q < 0
\end{cases}
\]
\end{definition}

\begin{definition}[Empty Set]
  The \emph{empty set} --- the set that does not contain any element --- is denoted as $\emptyset$.
\end{definition}

\hypertarget{def-cardinality}{}
\begin{definition}[Set Cardinality]
  For a set $S$, the notation $\cardinality{S}$ stands for the number of elements in $S$.
\end{definition}

\hypertarget{def-pow}{}
\begin{definition}[Powerset]
  The \emph{powerset} of a set $A$, denoted as $\pow{A}$, is the set of all subsets of $A$, including the empty set and $A$ itself:
  \[
     \pow{A} \triangleq \{ B \;|\; B \subseteq A\} \enspace.
  \]
\end{definition}

\hypertarget{def-powfin}{}
\begin{definition}[Powerset of Finite Subsets]
  The \emph{powerset of finite subsets} of a set $A$, denoted as $\powfin{A}$, is the set of all finite subsets (including the empty set) of $A$:
  \[
     \powfin{A} \triangleq \{ B \;|\; B \subseteq A, |B| \in \N\} \enspace.
  \]
\end{definition}

\hypertarget{def-cartimes}{}
\begin{definition}[Cartesian Product]
    The \emph{Cartesian product} of sets $A$ and $B$, denoted $A \cartimes B$
    is $A \cartimes B \triangleq \{(a,b) \;|\; a \in A, b \in B\}$.
\end{definition}

\hypertarget{def-partialfunc}{}
\hypertarget{def-dom}{}
\begin{definition}[Partial Function\label{def:PartialFunction}]
  A \emph{partial function}, denoted $f : A \partialto B$, is a function from a \underline{subset} of $A$ to $B$.
  The \emph{domain} of a partial function $f$, denoted $\dom(f)$, is the subset of $A$ for which it is defined.
  We write $f(x) = \bot$ to denote that $x$ is not in the domain of $f$, that is, $x \not\in \dom(f)$.
\end{definition}

Notice that the domain of a partial function need not be finite, which is what the following definition covers.

\hypertarget{def-finfunction}{}
\begin{definition}[Finite-domain Function]
The notation $\rightarrowfin$ stands for a function \\ whose domain is finite.
\end{definition}

\hypertarget{def-emptyfunc}{}
\begin{definition}[Empty Function\label{def:EmptyFunction}]
The function with an empty domain is denoted as $\emptyfunc$.
\end{definition}

\begin{definition}[Function Update\label{def:FunctionUpdate}]
  The function denoted as $f[x \mapsto v]$ is a function identical to $f$, except that $x$ is bound
  to $v$. That is, if  $g = f[x \mapsto v]$ then
  \[
    g(z) =
  \begin{cases}
    v     & \text{if } z = x\\
    f(z)  & \text{otherwise } \enspace.
  \end{cases}
  \]

  The notation $\{i=1..k: a_i\mapsto b_i\}$ stands for the function formed from the corresponding input-output pairs:
  $\emptyfunc[a_1\mapsto b_1]\ldots[a_k\mapsto b_k]$.
\end{definition}

\begin{definition}[Function Restriction]
\hypertarget{def-restrictfunc}{}
The \emph{restriction} of a function $f : X \rightarrow Y$ to a subset of its domain
$A \subseteq \dom(f)$, denoted as $\restrictfunc{f}{A}$, is defined
in terms of the set of input-output pairs:
\[
  \restrictfunc{f}{A} \triangleq \{ (x, f(x)) \;|\; x \in A \} \enspace.
\]
\end{definition}

\begin{definition}[Function Graph]
\hypertarget{def-funcgraph}{}
The \emph{graph} of a finite-domain function $f : X \rightarrowfin Y$
is the list of input-output pairs for $f$, given in any order:
\[
\funcgraph(f) \triangleq \{ (x, f(x)) \;|\; x \in \dom(f) \} \enspace.
\]
\end{definition}

Throughout this document, we will annotate arguments of relations and functions, wherever it is useful,
by writing a name or an expression above the corresponding argument type.
This makes convenient to refer to arguments by referring to the corresponding names and helps identify
the expressions corresponding to the arguments.
For example,
\[
    \textsf{choice} : \overname{\Bool}{b} \cartimes \overname{T}{x} \cartimes \overname{T}{y} \rightarrow \overname{T}{z}
\]
defines a function type and lets us refer to the first argument as $b$, the second argument as $x$,
the third argument as $y$, and to the result as $z$.

A \emph{parametric function} is a function whose domain is not a priori fixed but rather
parameterized by the type of its arguments. An example is the $\textsf{choice}$ function where the type $T$ of
$x$, $y$, and $z$ is unspecified and inferred from the context where the function is used.

\hypertarget{def-choice}{}
\begin{definition}[Choice]
The parametric function $\textsf{choice} : \overname{\Bool}{b} \cartimes \overname{T}{x} \cartimes \overname{T}{y} \rightarrow \overname{T}{z}$,
is defined as follows:
\[
  \choice{\vb}{x}{y} \triangleq
  \begin{cases}
    x & \text{ if }\vb \text{ is }\True\\
    y & \text{ otherwise}\\
  \end{cases}
\]
\end{definition}

\subsection{Lists}
In the remainder of this document, we use the term \emph{list} and \emph{sequence} interchangeably.

A list of elements \hypertarget{def-emptylist}{is either empty, denoted by $\emptylist$}, or non-empty.
A non-empty list is either denoted by listing the elements in sequence, $v_1 \ldots\ v_k$,
or in bracketed form, $[v_1,\ldots,v_k]$, which is used to aesthetically separate it from surrounding mathematical expressions.
The commas carry no special meaning.

\hypertarget{def-head}{}
\hypertarget{def-tail}{}
For a non-empty list $v_1 \ldots\ v_k$, the \emph{\head} of the list is the first element --- $v_1$ ---
and the \emph{\tail} of the list is the suffix obtained by removing $v_1$ from the list.

We refer to individual elements of a non-empty list $V$ by the index notation $V[i]$ where $i\in\Npos$.

\hypertarget{def-listlen}{}
\begin{definition}[List Length]
The \emph{length} of a list is the number of elements in that list:
$\listlen{\emptylist} \triangleq 0$ and $\listlen{v_1,\ldots,v_k}=k$.
\end{definition}

We use the notation $a..b$, where $a,b\in\Z$ and as a shorthand for the interval $[a\ldots b]$
(counting up when $a \leq b$ and counting down when $a \geq b$).
We write $x_{a..b}$ as a shorthand for the sequence $x_a \ldots x_b$.
%
We write $i=1..k: V(i)$, where $V(i)$ is a mathematical expression parameterized by $i$,
to denote the sequence of expressions $V(1) \ldots V(k)$.
The notation $a \in A: V(a)$, where $A$ is a set and $V$ is an expression parameterized by the free variable $a$,
stands for $V(a_1) \ldots V(a_k)$ where $a_{1..k}$ is an arbitrary ordering of the elements of $A$.

We write $T^*$ to denote a the type of a possibly-empty list of elements of type $T$,
and $T^+$ for a non-empty list of elements of type $T$.

\hypertarget{def-listset}{}
\begin{definition}[Listing a Set]
The parametric relation $\listset : \pow{T} \times T^*$
lists the elements of a set in an arbitrary order:
\[
\begin{array}{c}
  \listset(X) = x_{1..k}\\
  |X| = k\\
  \forall x\in X.\ \exists 1 \leq i \leq k.\ x = x_i
\end{array}
\]
\end{definition}

\hypertarget{def-concat}{}
\begin{definition}[List Concatenation]
The parametric function $\concat : T^* \cartimes T^* \rightarrow T^*$ concatenates two lists:
\[
    \begin{array}{rcl}
    \emptylist \concat L &\triangleq& L\\
    L \concat \emptylist &\triangleq& L\\
    l_{1..k} \concat m_{1..n} &\triangleq& [l_{1..k}, m_{1..n}]
    \end{array}
\]
\end{definition}

\hypertarget{def-equallength}{}
\begin{definition}[Equating List Lengths]
The parametric function
\[
  \equallength : \overname{L}{a} \cartimes \overname{L}{b} \rightarrow \Bool
\]
compares the length of two lists:
\[
\equallength(a, b) \triangleq \listlen{a}=\listlen{b} \enspace.
\]
\end{definition}

\hypertarget{def-listprefix}{}
\begin{definition}[List Prefix]
The parametric function $\listprefix : \overname{T^*}{\vlone} \cartimes \overname{T^*}{\vltwo} \rightarrow \Bool$ checks whether
the list $\vlone$ is a \emph{prefix} of the list $\vltwo$:
\[
\listprefix(\vlone, \vltwo) \triangleq \exists \vlthree.\ \vltwo = \vlone \concat \vlthree \enspace.
\]
\end{definition}

\hypertarget{def-listrange}{}
\begin{definition}[Indices of a List]
The parametric function $\listrange : T^* \rightarrow \N^*$ returns the ($1$-based) list of indices for a given list:
\[
    \begin{array}{rcl}
        \listrange(\emptylist) &\triangleq& \emptylist\\
        \listrange(v_{1..k}) &\triangleq& [1..k] \enspace.
    \end{array}
\]
\end{definition}

% \hypertarget{def-filterlist}{}
% \begin{definition}[Filtering a List]
% The parametric function
% \[
% \filterlist : \overname{T^*}{l} \times \overname{(T \rightarrow \Bool)}{p} \rightarrow T^*
% \]
% retains the elements of the list $l$ for which the predicate $p$ returns $\True$:
% \[
% \begin{array}{rcl}
% \filterlist(\emptylist, p) &\triangleq& \emptylist\\
% \filterlist([h] \concat t, p) &\triangleq& \begin{cases}
% [h]\concat \filterlist(t) & \text{if }p(h) = \True\\
%  \filterlist(t) & \text{else}\\
% \end{cases}\\
% \end{array}
% \]
% \end{definition}

\hypertarget{def-unziplist}{}
\begin{definition}[Unzipping a List of Pairs]
The parametric function
\[
\unziplist : (T_1 \cartimes T_2)^* \rightarrow (T_1^* \cartimes T_2^*)
\]
transforms a list of pairs into the corresponding pair of lists:
\[
  \unziplist(\pairs) \triangleq \begin{cases}
    (\emptylist, \emptylist)  & \text{if }\pairs = \emptylist\\
    (a_{1..k}, b_{1..k})      & \text{else }\pairs = (a_1,b_1) \ldots (a_k,b_k)  \enspace.
  \end{cases}
\]
\end{definition}

\hypertarget{def-unziplistthree}{}
\begin{definition}[Unzipping a List of Triples]
The parametric function
\[
\unziplistthree : (T_1 \cartimes T_2 \cartimes T_3)^* \rightarrow (T_1^* \cartimes T_2^* \cartimes T_3^*)
\]
transforms a list of triples into the corresponding triple of lists:
\[
  \unziplistthree(\triples) \triangleq \begin{cases}
    (\emptylist, \emptylist, \emptylist)  & \text{if }\triples = \emptylist\\
    (a_{1..k}, b_{1..k}, c_{1..k})      & \text{else }\triples = (a_1,b_1,c_1) \ldots (a_k,b_k,c_k)  \enspace.
  \end{cases}
\]
\end{definition}

\hypertarget{def-uniquelist}{}
\hypertarget{def-uniquep}{}
\begin{definition}[Finding unique elements of a list]
The parametric function
\[
\uniquelist : \overname{T^*}{l} \rightarrow T^*
\]
retains only the first occurrence of each element of the list $l$.
It relies on the helper function $\uniquep$:
\[
\begin{array}{rcl}
\uniquelist(l) &\triangleq& \uniquep(l, \emptylist)\\\\
\uniquep(\emptylist, \acc) &\triangleq& \acc\\
\uniquep([h] \concat t, \acc) &\triangleq&
  \begin{cases}
    \uniquep(t, \acc) & \text{if }h \in t\\
    \uniquep(t, \acc \concat [h]) & \text{otherwise}\\
  \end{cases}
\end{array}
\]
\end{definition}

\subsection{Strings}
\hypertarget{def-stringconcat}{}
The function $\stringconcat : \Strings \times \Strings \rightarrow \Strings$
concatenates two strings.

\hypertarget{def-stringofnat}{}
The function $\stringofnat : \N \rightarrow \Strings$ converts a natural number
to the corresponding string.

\subsection{OCaml-style Notations}
We use the following notations, which are in the style of the OCaml programming language,
to facilitate correspondence with our
\href{https://github.com/herd/herdtools7/tree/master/asllib}{reference implementation}.

The notation $L(v_{1..k})$ is a compound term where $L$ is a label and $v_{1..k}$ is a (possibly singleton) list of mathematical values.
We also write $L(T_{1..k})$, where $T_{1..k}$ denotes mathematical types of values, to stand for the type
$\{ L(v_{1..k}) \;|\; v_1\in T_1,\ldots,v_k\in T_k \}$.

\hypertarget{def-optional}{}
\begin{definition}[Optional]
\hypertarget{def-none}{}
\hypertarget{def-some}
The notation $\Some{\cdot}$ stands for either an empty set or a singleton set,
where $\None\triangleq\langle\rangle$ denotes an empty set
and $\Some{v}$ denotes a set containing the single element $v$.
%
The notation $\Some{T}$, where $T$ denotes a mathematical type, stands for
$\{ \None \} \cup \{\Some{v} \;|\; v \in T\}$.
%
We refer to $\Some{T}$ as an \emph{\optional}.
\end{definition}

%%%%%%%%%%%%%%%%%%%%%%%%%%%%%%%%%%%%%%%%%%%%%%%%%%%%%%%%%%%%%%%%%%%%%%%%%%%%%%%%
\section{Inference Rules}
%%%%%%%%%%%%%%%%%%%%%%%%%%%%%%%%%%%%%%%%%%%%%%%%%%%%%%%%%%%%%%%%%%%%%%%%%%%%%%%%
\hypertarget{def-inferencerule}{}
An \emph{\inferencerule} (rule, for short) is an implication between a set of logical assertions,
called the \emph{premises} of the rule,
and a \emph{conclusion} assertion.
The conclusion holds when the \underline{conjunction} of its premises holds.

We use the following rule notation, where $P_{1..k}$ are the rule premises and $C$ is the conclusion:
\begin{mathpar}
  \inferrule{P_1 \and \ldots \and P_k}{C}
\end{mathpar}

For example, the rule \TypingRuleRef{ELit} has one premise:
\begin{mathpar}
\inferrule{
  \annotateliteral{\vv} \typearrow \vt
}{
  \annotateexpr{\tenv, \ELiteral(\vv)} \typearrow (\vt, \ELiteral(\vv))
}
\end{mathpar}

and the rule \TypingRuleRef{EBinop} (somewhat simplified here) has three premises:
\begin{mathpar}
\inferrule{
  \annotateexpr{\tenv, \veone} \typearrow (\vtone, \veone') \\\\
  \annotateexpr{\tenv, \vetwo} \typearrow (\vttwo, \vetwo') \\\\
  \applybinoptypes(\tenv, \op, \vtone, \vttwo) \typearrow \vt
}{
  \annotateexpr{\tenv, \EBinop(\op, \veone, \vetwo)} \typearrow (\vt, \EBinop(\op, \veone', \vetwo'))
}
\end{mathpar}

The free variables appearing in the premises and conclusion are interpreted \underline{universally}.
That is, the rules apply to any values (of the appropriate types) assigned to their free variables.
%
For example, the rule \TypingRuleRef{EBinop} applies to any choice of values for the free variables
$\tenv$ (a static environment),
$\veone$, $\vetwo$, $\veone'$, $\vetwo'$ (expressions),
$\vt$, $\vtone$, and $\vttwo$ (types).

\begin{definition}[Grounding]
Assertions can be \emph{grounded} by substituting their free variables with values.
A \emph{ground rule} is a rule with all its assertions (premises and conclusion) grounded.
\end{definition}
For example,
the following is a grounding of \TypingRuleRef{EBinop}
\begin{mathpar}
\inferrule{
  \annotateexpr{\emptytenv, \ELiteral(\lint(2))} \typearrow (\TInt, \ELiteral(\lint(2))) \\\\
  \annotateexpr{\emptytenv, \ELiteral(\lint(3))} \typearrow (\TInt, \ELiteral(\lint(3))) \\\\
  \applybinoptypes(\emptytenv, \MUL, \TInt, \TInt) \typearrow \TInt
}{
  \annotateexpr{\emptytenv, \EBinop(\MUL, \ELiteral(\lint(2)), \ELiteral(\lint(3)))} \typearrow \\ (\TInt, \EBinop(\MUL, \ELiteral(\lint(2)), \ELiteral(\lint(3))))
}
\end{mathpar}
obtained by the following substitutions:
\begin{tabular}{ll}
  \textbf{free variable} & \textbf{value}\\
  \hline
  $\tenv$   & $\emptytenv$\\
  $\veone$  & $\ELiteral(\lint(2))$\\
  $\veone'$  & $\ELiteral(\lint(2))$\\
  $\vetwo$  & $\ELiteral(\lint(3))$\\
  $\vetwo'$  & $\ELiteral(\lint(3))$\\
  $\vt$    & $\TInt$\\
  $\vtone$    & $\TInt$\\
  $\vttwo$    & $\TInt$\\
  $\op$       & $\MUL$
\end{tabular}

A set of rules is interpreted \underline{disjunctively}. That is, each rule is used to determine whether its conclusion
holds independently of other rules.

\begin{definition}[Axiom]
An \emph{axiom} is a rule with an empty set of premises.
An axiom is denoted by simply stating its conclusion.
\end{definition}

An example of an axiom in the ASL type system is \TypingRuleRef{SPass}:
\begin{mathpar}
\inferrule{}{\annotatestmt(\tenv, \SPass) \typearrow (\SPass,\tenv)}
\end{mathpar}
\hypertarget{SemanticsRule.PAll-example}{}
An example of an axiom in the ASL semantics is \SemanticsRuleRef{PAll}:
\begin{mathpar}
\inferrule{}{
  \evalpattern{\env, \Ignore, \PatternAll} \evalarrow \Normal(\nvbool(\True), \emptygraph)
}
\end{mathpar}

To show that a specification is correct, with respect to the set of type rules,
or to show that a specification evaluates to a certain value, with respect to
the set of semantic rules, we must apply rules to form a \emph{derivation tree}.

\hypertarget{def-derivationtree}{}
\begin{definition}[Derivation Tree]
  A \emph{derivation tree} is a tree whose vertices correspond to ground assertions.
  More specifically, the leaves of a derivation tree correspond to ground axioms,
  and an internal vertex corresponds to a ground conclusion of a rule with its children
  corresponding to the ground premises of the same rule.
\end{definition}

\subsection{Transitions\label{sec:transitions}}

We use rules as a structured way for defining relations (and therefore functions, as a special case).

To define a relation $R \subseteq X \cartimes Y$, we use assertions of the form $\termx \rulearrow \termy$
where $\termx$ and $\termy$ are logical terms denoting sets of elements from $X$ and $Y$, respectively.
%
We call such assertions \emph{transitions}.
A set of rules $M$ with transition assertions defines the relation
\[
    R = \{ (x,y) \;|\; x \rulearrow y \text{ can be derived from rules in } M\} \enspace.
\]

For example, the rule \TypingRuleRef{ELit} defines a relation
between the infinite set of elements of the form
$\annotateexpr{\tenv, \ELiteral(\vv)}$ (for the
infinite choice of values for the free variables $\tenv$ and
$\vv$) to the infinite set of pairs of the form $(\vt,
\ELiteral(\vv))$, such that the premise holds.

\paragraph{Mutual Exclusion Principle:}
Our rules follow (with very few deviations, which we point out
in context) a mutual exclusion principle, where each rule
defines a relation disjoint from the ones defined by the other
rules.  This makes it easy to determine the rule responsible
for a given transition.

\hypertarget{def-configuration}{}
\subsection{Configurations}

\hypertarget{def-configdomain}{}
Our relations range over compound values. That is, values that often nest tuples and lists inside other tuples and lists.
We refer to such values as \emph{configurations}. To make it easier to distinguish between different configurations,
we will sometimes attach labels to tuples using the OCaml-style notation discussed earlier.
We refer to those labels as \emph{configuration domains}.
The domain of a configuration $C=L(\ldots)$, denoted $\configdomain{C}$, is the label $L$.

We refer to configurations at the origin of a transition as \emph{input configurations} and to the
configurations at the destination of a transition as \emph{output transitions}.

For example, the conclusion of the rule \TypingRuleRef{ELit} has \\
$\annotateexpr{\tenv, \ELiteral(\vv)}$ as its input configuration
and $(\vt, \ELiteral(\vv))$ as its output configuration.
Further, \\
$\configdomain{\annotateexpr{\tenv, \ELiteral(\vv)}} = \textit{annotate\_expr}$,
while the output configuration does not have a configuration domain, since it is an unlabelled pair.

Our rules always make use of labelled input configurations. This makes it easier to ensure
the mutual exclusion rule principle.

Our rules always define relations whose sets of input configurations and output configurations are disjoint.

\hypertarget{def-freshvariables}{}
\begin{definition}[Fresh Element]
  Premises of the form \texttt{$x\in T$ is fresh} mean that in any
  instantiation in a derivation tree, the value of $x$ is unique.
  That is, different from all other values instantiated for any other variable.
\end{definition}

\hypertarget{def-ignore}{}
\begin{definition}[Ignore Variable]
To keep rules succinct, we write $\Ignore$ for a mathematical variable whose name is
irrelevant for understanding the rule, and can thus be omitted.
Each \underline{occurrence} of $\Ignore$ represents a variable whose name is
different from any other free variable in the rule.
\end{definition}

For example, the rule \SemanticsRuleRef{PAll}, shown \hyperlink{SemanticsRule.PAll-example}{above},
uses an ignore variable to stand for the value being matched by a \texttt{-} pattern.
Since the rule does not need to refer to the value, we do not name it and use an ignore variable
instead.

\subsection{Flavors of Equality In Rules \label{sec:FlavoursOfEqualityInRules}}
We now explain equality notations in rules, two of which are used in \SemanticsRuleRef{ECond},
shown here:
\begin{mathpar}
\inferrule{
  \evalexpr{\env, \econd} \evalarrow \Normal(\mcond, \envone) \OrAbnormal\\\\
  \mcond \eqname (\nvbool(\vb), \vgone)\\
  \vep \eqdef \choice{\vb}{\veone}{\vetwo}\\\\
  \evalexpr{\envone, \vep} \evalarrow \Normal((\vv, \vgtwo), \newenv)  \OrAbnormal\\\\
  \vg \eqdef \ordered{\vgone}{\aslctrl}{\vgtwo}
}{
  \evalexpr{\env, \overname{\ECond(\econd, \veone, \vetwo)}{\ve}} \evalarrow
  \Normal((\vv, \vg), \newenv)
}
\end{mathpar}

\begin{description}
  \item[Range:] we write $i=1..k$ to allow listing premises parameterized by $i$ or constructing
  lists from expressions parameterized by $i$.
  For example, given two lists $a$ and $b$,
  \[
    i=1..k: a[i] > b[i]
  \]
  is the list of premises
  \[
    \begin{array}{l}
    a[1] > b[1]\\
    \ldots\\
    a[k] > b[k] \enspace.
    \end{array}
  \]

  \item[Predicate:] we write $a = b$ as an assertion of the equality of $a$ and $b$.
  For example, the mathematical identity $x \times (y + z) = x \times y + x \times z$.

  \hypertarget{def-deconstruction}{}
  \item[Deconstruction / ``View as'':] some values, such as tuples, are compound. In order to refer to the structure
  of compound values, we write $v \eqname \textit{f}(u_{1..k})$ where the expression on the right
  hand side exposes the internal structure of $v$ by introducing the variables
  $u_{1..k}$, allowing us to alias internal components of $v$.
  Intuitively, $v$ is re-interpreted as $\textit{f}(u_{1..k})$.
  For example, suppose we know that $v$ is a pair of values.
  Then, $v \eqname (a, b)$ allows us to alias $a$ and $b$.
  In \SemanticsRuleRef{ECond}, we know from the definition of $\evalexpr$ that
  $\mcond$ is a pair datatype.
  Therefore, writing $\mcond \eqname (\nvbool(\vb), \vgone)$ allows us to name each component of this pair
  and then refer to it, while \hyperlink{def-ignore}{ignoring} the static environment component.
  Similarly, if $v$ is a non-empty list, then $v \eqname [h] + t$ deconstructs the list into the
  head of the list $h$ and its tail $t$.
  Given that a variable $v$ represents a list, we write $v \eqname v_{1..k}$ to list its elements and allow
  referring to them by index.

  \hypertarget{def-eqdef}{}
  \item[Definition / ``Define as'':] the notation $\vx \eqdef \ve$ denotes that $\vx$ is a new name serving as an alias for the expression $\ve$.
  For example, in the rule \SemanticsRuleRef{ECond}, we use $\vg$ to name the mathematical expression
  $\ordered{\vgone}{\aslctrl}{\vgtwo}$.
  Aliases allow us to break down complex expressions, but rules can always be rewritten without them,
  by inlining their right-hand sides:
\begin{mathpar}
\inferrule{
  \evalexpr{\env, \econd} \evalarrow \Normal(\mcond, \envone) \OrAbnormal\\\\
  \mcond \eqname (\nvbool(\vb), \vgone)\\
  \evalexpr{\envone, \choice{\vb}{\veone}{\vetwo}} \evalarrow \Normal((\vv, \vgtwo), \newenv) \OrAbnormal
}{
  \evalexpr{\env, \overname{\ECond(\econd, \veone, \vetwo)}{\ve}} \evalarrow
  \Normal((\vv, \ordered{\vgone}{\aslctrl}{\vgtwo}), \newenv)
}
\end{mathpar}
\end{description}

\subsection{AST-related Notations}

When deconstructing AST record nodes such as $\{f_1:t_1,\ldots,f_k:t_k\}$,
we sometimes only care about a subset of the fields $\{f_{i_1},\ldots,f_{i_m}\} \subset \{f_{1..k}\}$.
In such cases, we write $\{f_{i_1}:t_{i_1},\ldots,f_{i_m}:t_{i_m},\ldots\}$,
where $\ldots$ stands for fields that are irrelevant for the rule.

For example\footnote{This example is from \SemanticsRuleRef{FCall}.}, the \func\ non-terminal is of a record type and has the following fields:
$\textsf{name}$, $\textsf{parameters}$, $\textsf{args}$, $\textsf{body}$, $\textsf{return\_type}$, and $\textsf{subprogram\_type}$.
The notation $\{ \textsf{body}:\SBASL(\texttt{body}),\ \textsf{args}:\texttt{arg\_decls}, \ldots \}$
allows us to deconstruct a given \func\ node by matching only the \textsf{body} and \textsf{args} fields.

Recall that a subset of AST nodes are either labels or labelled tuples.
\hypertarget{def-astlabel}{}
The partial function $\astlabel$ returns the label $\vl\in\astlabels$ of an AST node, when it exists.
For example, $\astlabel(\TBool) = \TBool$ and $\astlabel(\TNamed(\texttt{x})) = \TNamed$.

\subsection{How to Parse Rules Efficiently}
Consider the following examples, which is a simplified version of \SemanticsRuleRef{Binop}
\begin{mathpar}
  \inferrule{\op \not\in \{\BAND, \BOR, \IMPL\}\\\\
    \evalexpr{ \env, \veone} \evalarrow \Normal(\vmone, \envone) \\\\
    \evalexpr{ \envone, \vetwo } \evalarrow \Normal(\vmtwo, \newenv) \\\\
    \vmone \eqname (\vvone, \vgone) \\
    \vmtwo \eqname (\vvtwo, \vgtwo) \\
    \binoprel(\op, \vvone, \vvtwo) \evalarrow \vv \\\\
    \vg \eqdef \vgone \parallelcomp \vgtwo
  }
  {
    \evalexpr{ \env, \EBinop(\op, \veone, \vetwo) } \evalarrow
    \Normal((\vv, \vg), \newenv)
  }
\end{mathpar}

To parse a rule, start by examining the conclusion and the variables appearing in the rule.
In this case, the rule describes a transition from an input configuration \\
$\evalexpr{ \env, \EBinop(\op, \veone, \vetwo) }$,
whose configuration domain is \texttt{eval\_expr}, to an output configuration $\Normal((\vv, \vg), \newenv)$
whose configuration domain is $\Normal$.
%
A rule uses the free variables appearing in the input configuration of the conclusion
($\env$, $\op$, $\veone$, and $\vetwo$ in our example),
with the goal of assigning values to the free variables in the output configuration
of the conclusion ($\vv$, $\vg$, and $\newenv$, in our example).

Now, scan the premises in order to see where $\env$, $\op$, $\veone$, and $\vetwo$ are used and how
premises assign values to $\vv$, $\vg$, and $\newenv$.
%
In this case, $\vv$ is assigned as the result of the transition assertion
$\binoprel(\op, \vvone, \vvtwo) \evalarrow \vv$,
$\vg$ is assigned the expression $\vgone \parallelcomp \vgtwo$,
and $\newenv$ is assigned as the result of the transition assertion
$\evalexpr{ \envone, \vetwo } \evalarrow \Normal(\vmtwo, \newenv)$.
%
Notice that to assign values to the variables $\vv$, $\vg$, and $\newenv$,
intermediate values have to be assigned first.
For example, $\evalexpr{ \env, \veone} \evalarrow \Normal(\vmone, \envone)$
assigned values to $\envone$, which is then used by the transition \\
$\evalexpr{ \envone, \vetwo } \evalarrow \Normal(\vmtwo, \newenv)$.
Similarly, $\vg$ requires first assigning values to $\vgone$ and $\vgtwo$,
which are components of the previously assigned variables $\vmone$ and $\vmtwo$.

\subsection{Short-Circuit Rule Macros}

\emph{Short-circuit rule macros}, or \emph{rule macros}, for short, allow us to succinctly define sets of rules.
Specifically, they allow us to capture situations where
transitions have two alternative output configurations.
If the transition results in the first of the alternative output configurations, the following premises are considered.
However, if the result is the second, short-circuit output configuration, then the following premises are ignored
and the conclusion transitions into the short-circuit output configuration.
These short-circuit output configurations are typically, but not always, due to (type or dynamic) errors.

\hypertarget{def-terminateas}{}
In the following, $\XP$ and $\XQ$ stand for, possibly empty, sequences of premises.
%
A rule macro includes the special premise form $C \rulearrow C' \terminateas E$,
which introduces alternative output configurations $C'$ and short-circuit $E$:
\begin{mathpar}
  \inferrule{
    \XP\\\\
    C \rulearrow C' \terminateas E\\\\
    \XQ\\
  }
  {
    V \rulearrow V'
  }
\end{mathpar}
Such a rule macro expands to the following pair of rules:
\begin{mathpar}
  \inferrule[(Option 1)]{
    \XP\\\\
    C \rulearrow C' \\\\
    \XQ\\
  }
  {
    V \rulearrow V'
  }
  \and
  \inferrule[(Option 2:Short-circuited)]{
    \XP\\\\
    C \rulearrow E
  }
  {
    V \rulearrow E
  }
\end{mathpar}
Intuitively, if $C$ transitions to $C'$ then $\sslash E$ can be ignored
and the rule is interpreted as usual (Option 1).
However, if $C$ transitions into $E$ (Option 2) then the premises $\XQ$ are ignored,
thereby short-circuiting the rule, and the input configuration
in the conclusion also transitions into $E$.

We allow more than one premise to include short-circuiting alternatives and also
a single premise to include several alternatives.
That is, a rule macro of the form
\begin{mathpar}
  \inferrule{
    \XP\\\\
    C \rulearrow C' \terminateas E_{1...m}\\\\
    \XQ\\
  }
  {
    V \rulearrow V'
  }
\end{mathpar}
Stands for the set of rule macros
\begin{mathpar}
  \inferrule{
    \XP\\\\
    C \rulearrow C' \terminateas E_1\\\\
    \XQ\\
  }
  {
    V \rulearrow V'
  }
\and
\inferrule{\ldots}{}
\and
\inferrule{
  \XP\\\\
  C \rulearrow C' \terminateas E_m\\\\
  \XQ\\
}
{
  V \rulearrow V'
}
\end{mathpar}

Notice that after all rule macros are expanded, in a top-to-bottom and left-to-right order, into normal rules,
they behave like normal rules where the order of premises does
not matter.

\hypertarget{def-proseterminateas}{}
\paragraph{Alternative Outcomes Expressed in English Prose:}
In English prose, we use
\ProseTerminateAs{x, y, \ldots} to mean
``if the outcome is one of $x, y, \ldots$ then the result short-circuits the rule.

As an example, consider the rule \SemanticsRuleRef{Binop}.
This time, not simplified:
\begin{mathpar}
  \inferrule{\op \not\in \{\BAND, \BOR, \IMPL\}\\\\
    \evalexpr{ \env, \veone} \evalarrow \Normal(\vmone, \envone) \OrAbnormal \\\\
    \evalexpr{ \envone, \vetwo } \evalarrow \Normal(\vmtwo, \newenv) \OrAbnormal \\\\
    \vmone \eqname (\vvone, \vgone) \\
    \vmtwo \eqname (\vvtwo, \vgtwo) \\
    \binoprel(\op, \vvone, \vvtwo) \evalarrow \vv \terminateas \DynErrorConfig\\\\
    \vg \eqdef \vgone \parallelcomp \vgtwo
  }
  {
    \evalexpr{ \env, \EBinop(\op, \veone, \vetwo) } \evalarrow
    \Normal((\vv, \vg), \newenv)
  }
\end{mathpar}

In this rule, $\ThrowingConfig$ and $\DynErrorConfig$ are just shorthand notations for
actual configurations, which are properly defined in \chapref{Semantics}.
Intuitively, the alternative configurations $\ThrowingConfig$ and $\DynErrorConfig$
represent situations where a transition may result in a raised exception and a dynamic error,
respectively.

One may first read the rule ignoring these alternative configurations, to see how the
goal of transitioning into the output configuration appearing in the conclusion ---
$\Normal((\vv, \vg), \newenv)$ --- is achieved.
Then, re-reading the rule would indicate where exceptions and dynamic errors may result
in other output configurations.
%
For example, if the first transition assertion results in a throwing configuration $\ThrowingConfig$
then the output configuration of the conclusion is also $\ThrowingConfig$.
This corresponds to the following rule in the expanded macro:

\begin{mathpar}
  \inferrule{\op \not\in \{\BAND, \BOR, \IMPL\}\\\\
    \evalexpr{ \env, \veone} \evalarrow \ThrowingConfig
  }
  {
    \evalexpr{ \env, \EBinop(\op, \veone, \vetwo) } \evalarrow
    \ThrowingConfig
  }
\end{mathpar}

Similarly, if the first transition assertion results in a dynamic error, the output configuration of
the conclusion is that dynamic error, which corresponds to the following rule in the expansion:
\begin{mathpar}
  \inferrule{\op \not\in \{\BAND, \BOR, \IMPL\}\\\\
    \evalexpr{ \env, \veone} \evalarrow \DynErrorConfig
  }
  {
    \evalexpr{ \env, \EBinop(\op, \veone, \vetwo) } \evalarrow
    \DynErrorConfig
  }
\end{mathpar}

The following rules correspond to the cases where the first transition results in \\
$\Normal(\vmone, \envone)$, but the second transition assertion results in either
$\ThrowingConfig$ or $\DynErrorConfig$, respectively:
\begin{mathpar}
  \inferrule{\op \not\in \{\BAND, \BOR, \IMPL\}\\\\
    \evalexpr{ \env, \veone} \evalarrow \Normal(\vmone, \envone) \\\\
    \evalexpr{ \envone, \vetwo } \evalarrow \ThrowingConfig
  }
  {
    \evalexpr{ \env, \EBinop(\op, \veone, \vetwo) } \evalarrow
    \ThrowingConfig
  }
\end{mathpar}

\begin{mathpar}
  \inferrule{\op \not\in \{\BAND, \BOR, \IMPL\}\\\\
    \evalexpr{ \env, \veone} \evalarrow \Normal(\vmone, \envone) \\\\
    \evalexpr{ \envone, \vetwo } \evalarrow \DynErrorConfig
  }
  {
    \evalexpr{ \env, \EBinop(\op, \veone, \vetwo) } \evalarrow
    \DynErrorConfig
  }
\end{mathpar}

Expanding the last transition assertion, gives us the case:
\begin{mathpar}
  \inferrule{\op \not\in \{\BAND, \BOR, \IMPL\}\\\\
    \evalexpr{ \env, \veone} \evalarrow \Normal(\vmone, \envone) \\\\
    \evalexpr{ \envone, \vetwo } \evalarrow \Normal(\vmtwo, \newenv) \\\\
    \vmone \eqname (\vvone, \vgone) \\
    \vmtwo \eqname (\vvtwo, \vgtwo) \\
    \binoprel(\op, \vvone, \vvtwo) \evalarrow \DynErrorConfig
  }
  {
    \evalexpr{ \env, \EBinop(\op, \veone, \vetwo) } \evalarrow
    \DynErrorConfig
  }
\end{mathpar}

All these cases are succinctly encoded in a single rule with the alternative output configurations.

\subsection{Boolean Transition Assertions}
\hypertarget{def-booltrans}{}
We define the following rules to allow us to treat assertions as transition assertions:
\begin{mathpar}
  \inferrule[bool\_trans\_true]{}{ \booltrans{\True} \booltransarrow\True }
  \and
  \inferrule[bool\_trans\_false]{}{ \booltrans{\False} \booltransarrow\False }
\end{mathpar}
This is useful in that it allows us to use assertions in rule macros.

\subsection{Assertions Over Optional Data Types}
\hypertarget{def-mapopt}{}
Optional data types are prevalent in the AST.
To facilitate transition assertions over optional data types,
we introduce the parametric function,
which accepts a one-argument relation (or function) $f : A \aslrel B$
and applies it to an optional value $A?$:
\[
\mapopt{\cdot} : \overname{A?}{\vvopt} \aslrel \overname{B?}{\vvoptnew}
\]

\ProseParagraph
\OneApplies
\begin{itemize}
  \item \AllApplyCase{Some}
  \begin{itemize}
    \item $\vvopt$ consists of the value $v$;
    \item applying $f$ to $v$ yields $v'$;
    \item \Proseeqdef{$\vvoptnew$}{the singleton set consisting of $v'$}.
  \end{itemize}

  \item \AllApplyCase{None}
  \begin{itemize}
    \item $\vvopt$ is $\None$;
    \item \Proseeqdef{$\vvoptnew$}{$\None$}.
  \end{itemize}
\end{itemize}

\FormallyParagraph
\begin{mathpar}
\inferrule[Some]{
  f(v) \longrightarrow v'
}{
  \mapopt{f}(\overname{\Some{v}}{\vvopt}) \longrightarrow{r} \Some{v'}
}
\and
\inferrule[None]{}{
  \mapopt{f}(\overname{\None}{\vvopt}) \longrightarrow \None
}
\end{mathpar}

\subsection{Rule Naming}
To name a rule, we place it in a section with its name.
However, some relations are defined by a group of rules.
\hypertarget{def-caserules}{}
In such cases, we refer to the individual rules in a group as \emph{case rules},
or simply \emph{cases}. We annotate case rules by names
appearing above and to the left of the rule. The name of these case rules
is the name of the group, given by its section, followed by the name of the case.

For example, the rule \TypingRuleRef{BaseValue} is defined via multiple
cases. Two of these cases are the following ones:
\begin{mathpar}
\inferrule[t\_bool]{}{
    \basevalue(\tenv, \overname{\TBool}{\vt}) \typearrow \overname{\ELiteral(\lbool(\False))}{\veinit}
}
\end{mathpar}

\begin{mathpar}
\inferrule[t\_real]{}{
    \basevalue(\tenv, \overname{\TReal}{\vt}) \typearrow \overname{\ELiteral(\lreal(0))}{\veinit}
}
\end{mathpar}

The full name of the first case is then \TypingRuleRef{BaseValue}.BOOL
and the full name of the second case is \TypingRuleRef{BaseValue}.REAL.

When explaining rules in English prose, we include the name of the case rules
in parenthesis to make it easier to relate the prose to the corresponding mathematical
definitions (see, for example, the Prose paragraph of \TypingRuleRef{BaseValue}
or that of \TypingRuleRef{ApplyUnopType}).

\subsection{Generic Notations}
\hypertarget{def-wrapline}{}
\begin{itemize}
\item
The notation $\wrappedline$ denotes that a line that is longer than the page width continues on the next line.

\hypertarget{def-commonprefixline}{}
\item The notation $\commonprefixline$ serves as a visual aid to delimit a common prefix of premises shared by rule cases.

\hypertarget{def-commonsuffixline}{}
\item The notation $\commonsuffixline$ serves as a visual aid to delimit a common suffix of premises shared by rule cases.

\hypertarget{tododefine}{}
\item \tododefine{Missing:} Red hyperlinks indicate items that are yet to be defined.
\end{itemize}


%%%%%%%%%%%%%%%%%%%%%%%%%%%%%%%%%%%%%%%%%%%%%%%%%%%%%%%%%%%%%%%%%%%%%%%%%%%%%%%%
\chapter{ASL Lexical Definition \label{chap:lexicalanalysis}}
%%%%%%%%%%%%%%%%%%%%%%%%%%%%%%%%%%%%%%%%%%%%%%%%%%%%%%%%%%%%%%%%%%%%%%%%%%%%%%%%
This chapter defines the various elements of an ASL specification text in a high-level way
and then formalizes the lexical analysis as a function that takes a text and returns
a list of \emph{tokens} or a lexical error.

\section{ASL Specification Text}
An ASL specification is a string --- a list of ASCII characters --- consisting of a \emph{content text} followed by an \emph{end-of-file}.
The context text is a list of
ASCII characters that have the decimal encoding of 32 through 126 (inclusive),
which includes the space character (decimal encoding 32),
as well as
carriage return (decimal encoding 13) and line feed (decimal encoding 10).
\hypertarget{def-eof}{}
The end of file is the combination of two characters --- a carriage return followed by a line feed,
which we denote by $\eof$.
The context text does not contain an end-of-file.

In particular, it is an error to use a tab character in ASL specification text (decimal encoding 9).

\section{Lexical Regular Expressions}

\hypertarget{def-regex}{}
Table~\ref{ta:LexicalRegularExpressions} defines the regular expressions $\RegExp$ used to define
\emph{lexemes} --- substrings of the ASL specification text that are used to form \emph{tokens}.
We use the notation \Char{c} for characters such as the apostrophe.

\begin{table}
\caption{Lexical Regular Expressions \label{ta:LexicalRegularExpressions}}
\begin{center}
\begin{tabular}{ll}
\hline
\textbf{RegExp} & \textbf{Matches}\\
\hline
\texttt{'c'}              & The character c\\
\Char{c}                  & The character c\\
\texttt{ASCII\{a-b\}}       & The ASCII range between decimals 'a' and 'b'\\
\texttt{( $A$ )}          & $A$\\
$A$ $B$                   & $A$ followed by $B$\\
\texttt{$A$ | $B$}        & A or B\\
\texttt{$A$ - $B$}        & $A$ but not $B$\\
\texttt{$A$*}             & Zero or more repetitions of A\\
\texttt{$A$+}             & One or more repetitions of A\\
\texttt{["a\_string"]}    & Any character in \texttt{a\_string}\\
\texttt{\{"a\_string"\}}  & The string \texttt{a\_string} verbatim\\
\texttt{<}r\texttt{>}     & The lexical regular expression defined for \texttt{<}r\texttt{>}\\
\hline
\end{tabular}
\end{center}
\end{table}

\hypertarget{def-reasciichar}{}
Let $\REasciichar$ stand for any ASCII character:
\begin{center}
\begin{tabular}{rcl}
$\REasciichar$  &$\triangleq$& \texttt{ASCII\{0-255\}}
\end{tabular}
\end{center}

\hypertarget{def-lang}{}
The notation $\Lang(e)$ stands for \emph{formal language} of a regular expression $e$.
That is, the set of strings that match that regular expression.

\section{Whitespace}
Comments, newlines and space characters are treated as whitespace.

\section{Comments}
ASL supports comments in the style of C++:
\begin{itemize}
\item Single-line comments: the text from \text{//} until the end of the line
is a comment.
\item Multi-line comments: the text between \texttt{/*} and \texttt{*/} is a comment.
\end{itemize}
Comments do not nest and the two styles of comments do not interact with each other.

\section{Integer Literals}
Integers are written either in decimal using one or more of the characters \texttt{0-9} and underscore, or in hexadecimal
using \texttt{0x} at the start followed by the characters \texttt{0-9, a-f, A-F} and underscore. An integer literal cannot start with
an underscore.

This is formalized by the following lexical regular expression:
\hypertarget{def-redigit}{}
\hypertarget{def-reintlit}{}
\hypertarget{def-rehexlit}{}
\begin{center}
\begin{tabular}{rcl}
$\REdigit$  &$\triangleq$& \texttt{['0-9']}\\
$\REintlit$ &$\triangleq$& \texttt{\REdigit\ ('\_' | \REdigit)*}\\
$\REhexlit$ &$\triangleq$& \texttt{'0' 'x' (\REdigit\ | ["abcdefABCDEF"])} \\
          &            & $\wrappedline$ \texttt{('\_' | \REdigit\ | ["abcdefABCDEF"])*}
\end{tabular}
\end{center}

\section{Fixed Point Real Number Literals}
Fixed point real numbers are written in decimal and consist of one or more decimal digits, a decimal point and one
or more decimal digits. Underscores can be added between digits to aid readability

Underscores in numbers are not significant, and their only purpose is to separate groups of digits to make constants
such as \texttt{0xefff\_fffe}, \texttt{1\_000\_000} or \texttt{3.141\_592\_654} easier to read,

\hypertarget{def-reallit}{}
This is formalized by the following lexical regular expression:
\begin{center}
\begin{tabular}{rcl}
$\REreallit$ &$\triangleq$& \texttt{\REdigit\ (\Underscore\ | \REdigit)* '.' \REdigit\ (\Underscore\ | \REdigit)*}
\end{tabular}
\end{center}

\section{Boolean Literals}
Boolean literals are written using \texttt{TRUE} or \texttt{FALSE}.

\section{Bitvector Literals}
Constant bit-vectors are written using 1, 0 and spaces surrounded by single-quotes.
\hypertarget{def-rebitvectorlit}{}
\begin{center}
\begin{tabular}{rcl}
$\REbitvectorlit$ &$\triangleq$& \texttt{\Char{'} ["01 "]* \Char{'}}
\end{tabular}
\end{center}

The spaces in a bitvector are not significant and are only used to improve readability.
For example, \texttt{'1111 1111 1111 1111'} is the same as \texttt{'1111111111111111'}.

Constant bitmasks are written using \texttt{1}, \texttt{0}, \texttt{x} and spaces surrounded by single-quotes.
The \texttt{x} represents a don’t care character.

\section{Bitmask Literals}

\hypertarget{def-rebitmasklit}{}
\begin{center}
\begin{tabular}{rcl}
$\REbitmasklit$ &$\triangleq$& \texttt{\Char{'} ["01x "]* \Char{'}}
\end{tabular}
\end{center}

The spaces in a constant bitmask are not significant and are only used to improve readability.

\section{String Literals}

A string value is a string of zero or more characters, where a character is a printable ASCII character, tab (ASCII
code \texttt{0x09}) or newline (ASCII code \texttt{0x0A}). String values are created by string literals.
String literals consist of printable characters surrounded by double-quotes. Actual tabs and newlines are not
permitted in string literals, meaning that string literals cannot span multiple source lines. The backslash character,
\verb|`\'|, is treated as an escape character.

The escape sequences allowed in string literals appear in Table~\ref{ta:SscapeSeuqnces}.
\begin{table}
\caption{Escape Sequences in String Literals\label{ta:SscapeSeuqnces}}
\begin{center}
\begin{tabular}{ll}
\hline
\textbf{Escape sequence} & \textbf{Meaning}\\
\hline
\verb|\n| & The newline, ASCII code \texttt{0x0A}\\
\verb|\t| & The tab, ASCII code \texttt{0x09}\\
\verb|\\| & The backslash character, \verb|\|\\
\verb|\"| & The double-quote character, \texttt{"}\\
\hline
\end{tabular}
\end{center}
\end{table}

\hypertarget{def-restringlit}{}
\hypertarget{def-restrchar}{}
\begin{center}
\begin{tabular}{rcl}
$\REstrchar$ &$\triangleq$& ASCII\{32-126\}\\
$\REstringlit$ &$\triangleq$& \Char{\texttt{"}} ( ($\REstrchar$ \texttt{-} [ \Char{\texttt{"}} \Char{$\backslash$}]) $|$ (\Char{\textbackslash} [\texttt{"n t \Char{\texttt{"}} \Char{\textbackslash}"}])  )* \Char{\texttt{"}}
\end{tabular}
\end{center}

\section{Identifiers}
Identifiers start with a letter or underscore and continue with zero or more letters, underscores or digits.
Identifiers are case sensitive. To improve readability, it is recommended to avoid the use of identifiers that differ
only by the case of some characters.

By convention, identifiers that begin with double-underscore are reserved for use in the implementation and should
not be used in specifications.

\hypertarget{def-reletter}{}
\hypertarget{def-reidentifier}{}
\begin{center}
\begin{tabular}{rcl}
$\REletter$ &$\triangleq$& \texttt{'a-z' $|$ 'A-Z'}\\
$\REidentifier$ &$\triangleq$& \texttt{($\REletter$ $|$ '\_') ($\REletter$ $|$ '\_' $|$ $\REdigit$)*}\\
\end{tabular}
\end{center}

An enumeration literal is also classed as a literal constant, but is syntactically an identifier.

Tuple element selectors are classed as identifiers. That is, in cases like \texttt{(1, 2).item0},
the selector \texttt{item0} is classed as an identifier.
\lrmcomment{This is related to \identiTSXL}

\section{Lexical Analysis}
Lexical analysis is defined via the function
\hypertarget{def-aslscan}{}
\[
\aslscan : \LexSpec \times \REasciichar^* \aslto (\Token^* \cup \{\LexicalError\})
\]
\hypertarget{def-lexicalerrorresult}{}
which takes a \emph{lexical specification} (explained soon), an ASL specification string
(where characters are simply numbers representing ASCII characters)
and returns a sequence of tokens (tokens are defined below) or a \emph{lexical error} $\LexicalError$.

Tokens have one of two forms:
\begin{description}
  \item[Value-carrying] Tokens that carry value have the form $L(v)$ where $L$ is a token label,
        signifying the meaning of the token, and $v$ is a value carried by the token,
        which is used to construct the respective Abstract Syntax Tree nodes.
  \item[Valueless] Tokens that do not carry values have the form $L$ where $L$ is a token label.
\end{description}

The set of tokens used for the lexical analysis of ASL strings is defined below.

\hypertarget{def-token}{}
\[
\begin{array}{rcl}
\Token &\triangleq& \{\ \Tintlit(n) \;|\; n\in\Z\ \}\ \cup\\
        & & \{\ \Treallit(q) \;|\; q\in\Q\ \}\ \cup\\
        & & \{\ \Tstringlit(s) \;|\; s\in \Lang(\REstringlit)\ \}\ \cup\\
        & & \{\ \Tbitvectorlit(b) \;|\; b\in\{0,1\}^*\ \}\ \cup\\
        & & \{\ \Tmasklit(m) \;|\; m\in\{0,1,x\}^*\ \}\ \cup\\
        & & \{\ \Tboollit(\True), \Tboollit(\False)\ \}\ \cup \hypertarget{def-tidentifier}{}\\
        & & \{\ \Tidentifier(\id) \;|\; \id\in \Lang(\REidentifier)\ \} \\
        & & \{\ \Tlexeme(s) \;|\; s\in\Strings\ \} \\
        & & \{\ \Twhitespace, \Teof, \Terror\ \}
\end{array}
\]

\hypertarget{def-tintlit}{}
\begin{itemize}
  \item Tokens of the form $\Tintlit(n)$ represent integer literals; \hypertarget{def-treallit}{}
  \item Tokens of the form $\Treallit(q)$ represent real literals; \hypertarget{def-tstringlit}{}
  \item Tokens of the form $\Tstringlit(s)$ represent string literals; \hypertarget{def-tbitvectorlit}{}
  \item Tokens of the form $\Tbitvectorlit(b)$ represent bitvector literals; \hypertarget{def-tmasklit}
  \item Tokens of the form $\Tmasklit(m)$ represent bitmasks; \hypertarget{def-tboollit}{}
  \item Tokens of the form $\Tboollit(b)$ represent Boolean literals; \hypertarget{def-tidentifier}{}
  \item Tokens of the form $\Tidentifier(b)$ represent identifiers; \hypertarget{def-tlexeme}{}
  \item Tokens with the label $\Tlexeme$ are ones where the value $s$ is simply the \emph{lexeme} for that token.
  That is, the substring representing that token. Later when we will refer to such token by simply quoting
  the lexeme of the token and dropping the label, for brevity. For example, instead of $\Tlexeme(\texttt{for})$,
  we will write $\Tfor$. \hypertarget{def-twhitespace}{}
  \item The valueless token $\Twhitespace$ represents white spaces; \hypertarget{def-terror}{}
  \item The valueless token $\Terror$ represents an illegal lexeme such as the use of a reserved keyword;
  \hypertarget{def-teof}{}
  \item The valueless token $\Teof$ represents $\eof$.
\end{itemize}

\hypertarget{def-lexspec}{}
\begin{definition}[Lexical Specification]
A \emph{lexical specification} consists of a list of pairs $[(r_1,f_1),\ldots,(r_k,f_k)] \in \LexSpec$
where each pair $(r_i,f_i)$ consists of a lexical regular expression $r_i$
and a \emph{token function} $f_i : \Strings \aslto \Token$.
\end{definition}

The function $\remaxmatch : \overname{\RegExp}{e} \times \overname{\Strings}{s} \aslto (\overname{\Strings}{s_1} \times \overname{\Strings}{s_2}) \cup \{\bot\}$
returns the \emph{longest} match of a regular expression $e$ for a prefix of a string $s$.
More precisely:
$\remaxmatch(e, s) = (s_1,s_2)$ means that $s_1\in\Lang(e)$ and $s = s_1 \concat s_2$.
If no match exists, it is indicated by returning $\bot$.

The function $\maxmatches : \overname{\LexSpec}{R} \times \overname{\Strings}{s} \aslto \overname{\LexSpec}{R'}$
returns the sublist of $R$ consisting of pairs whose maximal matching $s$ is equal. Importantly, the result sublist $R'$ maintains
the order of pairs in $R$. If all expressions in $R$ do not match (that is $\remaxmatch$ return $\bot$), then $R'$ is the empty list.

The function $\aslscan$ is constructively defined via the following inference rules:

\begin{mathpar}
\inferrule[no\_match]{
  \maxmatches(R, s) = \emptylist
}{
  \aslscan(R, s) \scanarrow \LexicalError
}
\end{mathpar}

\begin{mathpar}
\inferrule[whitespace]{
  \maxmatches(R, s) = [(e_1,g_1),\ldots,(e_n,g_n)]\\
  \remaxmatch(s, e_1) = (s_1, s_2)\\
  s = s_1 \concat s_2\\
  g_1(s_1) = \Twhitespace\\
  \aslscan(R, s_2) \scanarrow \tstwo \terminateas \LexicalError
}{
  \aslscan(R, s) \scanarrow \tstwo
}
\end{mathpar}

\begin{mathpar}
\inferrule[eof]{
  \maxmatches(R, s) = [(e_1,g_1),\ldots,(e_n,g_n)]\\
  \remaxmatch(s, e_1) = (s_1, s_2)\\
  s = s_1 \concat s_2\\
  g_1(s_1) = \Teof
}{
  \aslscan(R, s) \scanarrow [t_1]
}
\end{mathpar}

\begin{mathpar}
\inferrule[error]{
  \maxmatches(R, s) = [(e_1,g_1),\ldots,(e_n,g_n)]\\
  \remaxmatch(s, e_1) = (s_1, s_2)\\
  s = s_1 \concat s_2\\
  g_1(s_1) = \Terror
}{
  \aslscan(R, s) \scanarrow \LexicalError
}
\end{mathpar}

\begin{mathpar}
\inferrule[token]{
  \maxmatches(R, s) = [(e_1,g_1),\ldots,(e_n,g_n)]\\
  \remaxmatch(s, e_1) = (s_1, s_2)\\
  s = s_1 \concat s_2\\
  g_1(s_1) = t_1\\
  t_1 \not\in \{\Teof, \Twhitespace, \Terror\}\\
  \aslscan(s_2) \scanarrow \tstwo \terminateas \LexicalError
}{
  \aslscan(R, s) \scanarrow [t_1] \concat \tstwo
}
\end{mathpar}

This form of lexical analysis is referred to as ``Maximal Munch'' in Compiler Theory
and is the most common form of lexical analysis.
See ``Compilers: Principles, Techniques, and Tools''~\cite{ASU86} for more details.

\section{ASL Lexical Specification}
We now define the lexical specification for ASL via token functions and tables
that associate regular expressions to token functions.

\subsection{Token Functions}
\hypertarget{def-discard}{}
\begin{itemize}
\item The function $\discard$ is a constant function that always returns $\Twhitespace$.
\hypertarget{def-decimaltolit}{}
\item The function $\decimaltolit(s)$ returns $\Tintlit(n)$ where $n$ is the integer represented by $s$
by decimal representation.
\hypertarget{def-hextolit}{}
\item The function $\hextolit(s)$ returns $\Tintlit(n)$ where $n$ is the integer represented by $s$
by hexadecimal representation.
\hypertarget{def-realtolit}{}
\item The function $\realtolit(s)$ returns $\Treallit(q)$ where $q$ is the real value represented by $s$
by floating point representation.
\hypertarget{def-strtolit}{}
\item The function $\strtolit(s)$ returns $\Tstringlit(s')$ where $s'$ is the string value represented by $s$.
\hypertarget{def-bitstolit}{}
\item The function $\bitstolit(s)$ returns $\Tbitvectorlit(b)$ where $b$ is the sequence of bits
given by $s$.
\hypertarget{def-masktolit}{}
\item The function $\masktolit(s)$ returns $\Tmasklit(m)$ where $m$ is the bitmask given by $s$.
\hypertarget{def-falsetolit}{}
\item The function $\falsetolit(s)$ returns $\Tboollit(\False)$ ($s$ is ensured to be \texttt{FALSE}).
\hypertarget{def-truetolit}{}
\item The function $\truetolit(s)$ returns $\Tboollit(\True)$ ($s$ is ensured to be \texttt{TRUE}).
\hypertarget{def-tokenid}{}
\item The function $\tokenid(s)$ returns $\Tlexeme(s)$.
\hypertarget{def-lexicalerror}{}
\item The function $\lexicalerror$ returns $\Terror$.
\hypertarget{def-toidentifier}{}
\item The function $\toidentifier(s)$ returns $\Tidentifier(s)$.
\hypertarget{def-eoftoken}{}
\item The function $\eoftoken$ returns $\Teof$.
\end{itemize}

\subsection{Regular Expressions and Corresponding Token Functions}
When several regular expressions are listed in a row, it means that they are all associated with the same
token function.

The lexical specification is given by the following four tables.

\begin{center}
\begin{tabular}{ll}
\textbf{Lexical Regular Expressions} & \textbf{Token Function}\\
\hline
\texttt{'\textbackslash n'}           & $\discard$ \\
\texttt{[' ' '\textbackslash r']+}    & $\discard$ \\
$\REcomment$                          & $\discard$ \\
$\REintlit$                           & $\decimaltolit$ \\
$\REhexlit$                           & $\hextolit$ \\
$\REreallit$                          & $\realtolit$ \\
$\REstringlit$                        & $\strtolit$ \\
$\REbitvectorlit$                     & $\bitstolit$ \\
$\REbitmasklit$                       & $\masktolit$ \\
\texttt{'!'}, \texttt{','}, \texttt{'<'}, \texttt{">>"}, \texttt{"\&\&"}, \texttt{"-->"}, \texttt{"<<"}                         & $\tokenid$  \\
\texttt{']'}, \texttt{')'}, \texttt{".."}, \texttt{'='}, \texttt{'\{'}, \texttt{"!="}, \texttt{'-'}, \texttt{"<->"}                        & $\tokenid$  \\
\texttt{'['}, \texttt{'('}, \texttt{'.'}, \texttt{"<="}, \texttt{'\textasciicircum'}, \texttt{'*'}, \texttt{'/'}                          & $\tokenid$  \\
\texttt{"=="}, \texttt{"||"}, \texttt{'+'}, texttt{':'}, \texttt{"=>"},                          & $\tokenid$  \\
\texttt{'\}'}, \texttt{"++"}, \texttt{'>'}, \texttt{"+:"}, \texttt{"*:"}, \texttt{';'}, \texttt{">="}                         & $\tokenid$  \\
"@looplimit"                          & $\tokenid$  \\
\hline
\end{tabular}
\end{center}

\begin{center}
\begin{tabular}{ll}
\textbf{Lexical Regular Expressions} & \textbf{Token Function}\\
\hline
\texttt{"AND"}, \texttt{"array"}, \texttt{"as"}, \texttt{"assert"},      & $\tokenid$ \\
\texttt{"begin"}, \texttt{"bit"}, \texttt{"bits"}, \texttt{"boolean"}       & $\tokenid$ \\
\texttt{"case"}, \texttt{"catch"}, \texttt{"config"}, \texttt{"constant"}      & $\tokenid$ \\
\texttt{"DIV"}, \texttt{"DIVRM"}, \texttt{"do"},\texttt{"downto"}        & $\tokenid$ \\
\texttt{"else"}, \texttt{"elsif"}, \texttt{"end"}, \texttt{"enumeration"}   & $\tokenid$ \\
\texttt{"XOR"}           & $\tokenid$ \\
\texttt{"exception"}     & $\tokenid$ \\
\texttt{"FALSE"} & $\falsetolit$  \\
\texttt{"for"}, \texttt{"func"}          & $\tokenid$ \\
\texttt{"getter"}        & $\tokenid$ \\
\texttt{"if"}, \texttt{"IN"}, \texttt{"integer"}       & $\tokenid$ \\
\texttt{"let"}           & $\tokenid$ \\
%"limit"}         & LIMIT \\
\texttt{"MOD"}           & $\tokenid$ \\
\texttt{"NOT"}           & $\tokenid$ \\
\texttt{"of"},      \texttt{"OR"},      \texttt{"otherwise"}                  & $\tokenid$ \\
\texttt{"pass"},    \texttt{"pragma"},  \texttt{"print"}                      & $\tokenid$ \\
\texttt{"real"},    \texttt{"record"},  \texttt{"repeat"}, \texttt{"return"}  & $\tokenid$ \\
\texttt{"setter"},  \texttt{"string"},  \texttt{"subtypes"}                   & $\tokenid$ \\
\texttt{"then"},    \texttt{"throw"},   \texttt{"to"}, \texttt{"try"}         & $\tokenid$ \\
\texttt{"TRUE"}          & $\truetolit$ \\
\texttt{"type"}          & $\tokenid$ \\
\texttt{"UNKNOWN"}, \texttt{"until"}         & $\tokenid$ \\
\texttt{"var"}           & $\tokenid$ \\
\texttt{"when"}, \texttt{"where"}, \texttt{"while"}, \texttt{"with"}          & $\tokenid$ \\
\hline
\end{tabular}
\end{center}

The following list represents keywords that are reserved for future use.
\begin{center}
\begin{tabular}{ll}
\textbf{Lexical Regular Expressions} & \textbf{Token Function}\\
\hline
\texttt{"SAMPLE"}, \texttt{"UNSTABLE"} & $\lexicalerror$ \\
\texttt{"\_"}, \texttt{"access"}, \texttt{"advice"}, \texttt{"after"} & $\lexicalerror$ \\
\texttt{"any"}, \texttt{"aspect"} & $\lexicalerror$ \\
\texttt{"assume"}, \texttt{"assumes"}, \texttt{"before"} & $\lexicalerror$ \\
\texttt{"call"}, \texttt{"cast"} & $\lexicalerror$ \\
\texttt{"class"}, \texttt{"dict"} & $\lexicalerror$ \\
\texttt{"endcase"}, \texttt{"endcatch"}, \texttt{"endclass"} & $\lexicalerror$ \\
\texttt{"endevent"}, \texttt{"endfor"}, \texttt{"endfunc"}, \texttt{"endgetter"} & $\lexicalerror$ \\
\texttt{"endif"}, \texttt{"endmodule"}, \texttt{"endnamespace"}, \texttt{"endpackage"} & $\lexicalerror$ \\
\texttt{"endproperty"}, \texttt{"endrule"}, \texttt{"endsetter"}, \texttt{"endtemplate"} & $\lexicalerror$ \\
\texttt{"endtry"}, \texttt{"endwhile"}, \texttt{"entry"} & $\lexicalerror$ \\
\texttt{"event"}, \texttt{"export"}, \texttt{"expression"} & $\lexicalerror$ \\
\texttt{"extends"}, \texttt{"extern"}, \texttt{"feature"} & $\lexicalerror$ \\
\texttt{"get"}, \texttt{"gives"} & $\lexicalerror$ \\
\texttt{"iff"}, \texttt{"implies"}, \texttt{"import"} & $\lexicalerror$ \\
\texttt{"intersect"}, \texttt{"intrinsic"} & $\lexicalerror$ \\
\texttt{"invariant"}, \texttt{"is"}, \texttt{"list"} & $\lexicalerror$ \\
\texttt{"map"}, \texttt{"module"}, \texttt{"namespace"}, \texttt{"newevent"} & $\lexicalerror$ \\
\texttt{"newmap"}, \texttt{"original"} & $\lexicalerror$ \\
\texttt{"package"}, \texttt{"parallel"} & $\lexicalerror$ \\
\texttt{"pointcut"}, \texttt{"port"}, \texttt{"private"} & $\lexicalerror$ \\
\texttt{"profile"}, \texttt{"property"}, \texttt{"protected"}, \texttt{"public"} & $\lexicalerror$ \\
\texttt{"replace"} & $\lexicalerror$ \\
\texttt{"requires"}, \texttt{"rethrow"}, \texttt{"rule"} & $\lexicalerror$ \\
\texttt{"set"}, \texttt{"shared"}, \texttt{"signal"} & $\lexicalerror$ \\
\texttt{"statements"}, \texttt{"template"} & $\lexicalerror$ \\
\texttt{"typeof"}, \texttt{"union"} & $\lexicalerror$ \\
\texttt{"using"}, \texttt{"watch"} & $\lexicalerror$ \\
\texttt{"ztype"} & $\lexicalerror$ \\
% "pattern"
\hline
\end{tabular}
\end{center}

\begin{center}
\begin{tabular}{ll}
\textbf{Lexical Regular Expression} & \textbf{Token Function}\\
\hline
$\REidentifier$   & $\toidentifier$ \\
$\eof$            & $\eoftoken$ \\
\hline
\end{tabular}
\end{center}

%%%%%%%%%%%%%%%%%%%%%%%%%%%%%%%%%%%%%%%%%%%%%%%%%%%%%%%%%%%%%%%%%%%%%%%%%%%%%%%%
\chapter{ASL Concrete Syntax \label{chap:parsing}}
%%%%%%%%%%%%%%%%%%%%%%%%%%%%%%%%%%%%%%%%%%%%%%%%%%%%%%%%%%%%%%%%%%%%%%%%%%%%%%%%

This chapter defines the grammar of ASL. The grammar is presented via two extensions
to context-free grammars --- \emph{inlined derivations} and \emph{parametric productions},
inspired by the Menhir Parser Generator~\cite{MenhirManual} for the OCaml language.
Those extensions can be viewed as macros over context-free grammars, which can be
expanded to yield a standard context-free grammar.

Our definition of the grammar and description of the parsing mechanism heavily relies
on the theory of parsing via LR(1) grammars and LR(1) parser generators.
%
See ``Compilers: Principles, Techniques, and Tools''~\cite{ASU86} for a detailed
definition of LR(1) grammars and parser construction.

The expanded context-free grammar is an LR(1) grammar, modulo shift-reduce
conflicts that are resolved via appropriate precedence definitions.
That is, given a list of token, returned from $\aslscan$, it is possible to apply
an LR(1) parser to obtain a parse tree if the list of tokens is in the formal language
of the grammar and return a parse error otherwise.

The outline of this chapter is as follows:
\begin{itemize}
  \item Definition of inlined derivations (see \secref{InlinedDerivations})
  \item Definition of parametric productions (see \secref{ParametricProductions})
  \item Definition of the ASL grammar (see \secref{ASLGrammar})
  \item Definition of priority and associativity of operators (see \secref{PriorityAndAssociativity})
  \item Definition of parse trees (see \secref{ParseTrees})
\end{itemize}

\section{Inlined Derivations \label{sec:InlinedDerivations}}
Context-free grammars consist of a list of \emph{derivations} $N \derives S^*$
where $N$ is a non-terminal symbol and $S$ is a list of non-terminal symbols and terminal symbols,
which correspond to tokens.
We refer to a list of such symbols as a \emph{sentence}.
A special form of a sentence is the \emph{empty sentence}, written $\emptysentence$.

As commonly done, we aggregate all derivations associated with the same non-terminal symbol
by writing $N \derives R_1 \;|\; \ldots \;|\; R_k$.
We refer to the right-hand-side sentences $R_{1..k}$ as the \emph{alternatives} of $N$.

Our grammar contains another form of derivation --- \emph{inlined derivation} ---
written as $N \derivesinline R_1 \;|\; \ldots \;|\; R_k$.
Expanding an inlined derivation consists of replacing each instance of $N$
in a right-hand-side sentence of a derivation with each of $R_{1..k}$, thereby
creating $k$ variations of it (and removing $N \derivesinline R_1 \;|\; \ldots \;|\; R_k$
from the set of derivations).
For example, given the inlined derivation
\[
  A \derivesinline a \;|\; b
\]
and another derivation where $A$ appears:
\[
B \derives A \parsesep c
\]
expanding $A$ yields:
\[
B \derives a \parsesep c \;|\; b \parsesep c
\]
Note that the same would apply if $B$ was defined via an inlined derivation.
That is, an inlined derivation can itself be expanded if it contains non-terminal
symbols defined via inlined derivations.
For example, if we had
\[
B \derivesinline A \parsesep c
\]
expanding $A$ in $B$ would result in
\[
B \derivesinline a \parsesep c \;|\; b \parsesep c
\]

Barring mutually-recursive derivations involving inlined derivations, it is possible to expand
all inlined derivations to obtain a context-free grammar without any inlined derivations.

\section{Parametric Productions \label{sec:ParametricProductions}}
A parametric production has the form
$N(p_{1..m}) \derives R_1 \;|\; \ldots \;|\; R_k$
where $p_{1..m}$ are place holders for grammar symbols and may appear in any of the alternatives $R_{1..k}$.
We refer to $N(p_{1..m})$ as a \emph{parametric non-terminal}.

\newcommand\uniquesymb[1]{\textsf{unique}(#1)}
Given sentences $S_{1..m}$, we can expand $N(p_{1..m}) \derives R_1 \;|\; \ldots \;|\; R_k$
by creating a unique symbols for $N(p_{1..m})$, denoted as $\uniquesymb{S_{1..m}}$, defining the
derivations
\[
  \uniquesymb{S_{1..m}} \derives R_1[S_1/p_1,\ldots,S_m/p_m] \;|\; \ldots \;|\; R_k[S_1/p_1,\ldots,S_m/p_m]
\]
where $R_i[S_1/p_1,\ldots,S_m/p_m]$ means replacing each instance of $p_i$ with $S_i$.
Then, the each instance of $S_{1..m}$ in the grammar is replaced by $\uniquesymb{S_{1..m}}$.
If all instances of a parametric non-terminal are expanded this way, we can remove the derivations of the parametric
non-terminal altogether.

We note that a parametric production can be either a normal derivation of an inlined derivation.

For example,
\[
  \option{x} \derives \emptysentence \;|\; x
\]
is useful for compactly defining derivations where part of a sentence may or may not appear.

Suppose we have
\[
B \derives a \parsesep \option{A \parsesep b}
\]
then expanding the instance $\option{A, b}$ produces
\[
\uniquesymb{\option{A \parsesep b}} \derives \emptysentence \;|\; A \parsesep b
\]
since $x[A \parsesep b/x]$ yields $A \parsesep b$ and $\emptysentence[A \parsesep b/x]$ yields $\emptysentence$,
and the derivations for $B$ are replaced by
\[
B \derives a \parsesep \uniquesymb{\option{A \parsesep b}}
\]

Expanding all instances of parametric productions results in a grammar without any parametric productions.

\section{ASL Parametric Productions \label{sec:ASLParametricProductions}}
We define the following parametric productions for various types of lists and optional productions.

\paragraph{Optional Symbol}
\hypertarget{def-option}{}
\begin{flalign*}
\option{x}   \derives\ & \emptysentence \;|\; x &\\
\end{flalign*}

\paragraph{Possibly-empty List}
\hypertarget{def-maybeemptylist}{}
\begin{flalign*}
\maybeemptylist{x}   \derives\ & \emptysentence \;|\; x \parsesep \maybeemptylist{x} &\\
\end{flalign*}

\paragraph{Non-empty List}
\hypertarget{def-nonemptylist}{}
\begin{flalign*}
\nonemptylist{x}   \derives\ & x \;|\; x \parsesep \nonemptylist{x}&\\
\end{flalign*}

\paragraph{Non-empty Comma-separated List}
\hypertarget{def-nclist}{}
\begin{flalign*}
\NClist{x}   \derives\ & x \;|\; x \parsesep \Tcomma \parsesep \NClist{x} &\\
\end{flalign*}

\paragraph{Possibly-empty Comma-separated List}
\hypertarget{def-clist}{}
\begin{flalign*}
\Clist{x}   \derivesinline\ & \emptysentence \;|\; \NClist{x} &\\
\end{flalign*}

\paragraph{Comma-separated List With At Least Two Elements}
\hypertarget{def-clisttwo}{}
\begin{flalign*}
\Clisttwo{x}   \derivesinline\ & x \parsesep \Tcomma \parsesep \NClist{x} &\\
\end{flalign*}

\paragraph{Possibly-empty Parenthesized, Comma-separated List}
\hypertarget{def-plist}{}
\begin{flalign*}
\Plist{x}   \derivesinline\ & \Tlpar \parsesep \Clist{x} \parsesep \Trpar &\\
\end{flalign*}

\paragraph{Parenthesized Comma-separated List With At Least Two Elements}
\hypertarget{def-plisttwo}{}
\begin{flalign*}
\Plisttwo{x}   \derivesinline\ & \Tlpar \parsesep x \parsesep \Tcomma \parsesep \NClist{x} \parsesep \Trpar &\\
\end{flalign*}

\paragraph{Non-empty Comma-separated Trailing List}
\hypertarget{def-tclist}{}
\begin{flalign*}
\NTClist{x}   \derives\ & x \parsesep \option{\Tcomma} &\\
                          |\  & x \parsesep \Tcomma \parsesep \NTClist{x}
\end{flalign*}

\paragraph{Comma-separated Trailing List}
\hypertarget{def-tclist}{}
\begin{flalign*}
\TClist{x}   \derivesinline\ & \option{\NTClist{x}} &\\
\end{flalign*}

\section{ASL Grammar \label{sec:ASLGrammar}}
We now present the list of derivations for the ASL Grammar where the start non-terminal is $\Nast$.

Notice that two of the derivations (for $\Nexprpattern$ and for $\Nexpr$) end with \\
$\precedence{\Tunops}$.
This is a precedence annotation, which is not part of the right-hand-side sentence, and is explained in \secref{PriorityAndAssociativity}
and can be ignored upon first reading.

For brevity, tokens are presented via their label only, dropping their associated value.
For example, instead of $\Tidentifier(\id)$, we simply write $\Tidentifier$.

\hypertarget{def-nast}{}
\begin{flalign*}
\Nast   \derives\ & \productionname{ast}{ast}\ \maybeemptylist{\Ndecl} &
\end{flalign*}

\hypertarget{def-ndecl}{}
\hypertarget{def-funcdecl}{}
\begin{flalign*}
\Ndecl  \derivesinline\ & \productionname{funcdecl}{func\_decl}\ \Tfunc \parsesep \Tidentifier \parsesep \Nparamsopt \parsesep \Nfuncargs \parsesep \Nreturntype \parsesep \Nfuncbody &
\hypertarget{def-proceduredecl}{}\\
|\ & \productionname{proceduredecl}{procedure\_decl}\ \Tfunc \parsesep \Tidentifier \parsesep \Nparamsopt \parsesep \Nfuncargs \parsesep \Nfuncbody &
\hypertarget{def-getter}{}\\
|\ & \productionname{getter}{getter}\ \Tgetter \parsesep \Tidentifier \parsesep \Nparamsopt \parsesep \Naccessargs \parsesep \Nreturntype \parsesep \Nfuncbody&
\hypertarget{def-noarggetter}{}\\
|\ & \productionname{noarggetter}{no\_arg\_getter}\ \Tgetter \parsesep \Tidentifier \parsesep \Nreturntype \parsesep \Nfuncbody &
\hypertarget{def-setter}{}\\
|\ & \productionname{setter}{setter}\ \Tsetter \parsesep \Tidentifier \parsesep \Nparamsopt \parsesep \Naccessargs \parsesep \Teq \parsesep \Ntypedidentifier \parsesep \Nfuncbody &
\hypertarget{def-noargsetter}{}\\
|\ & \productionname{noargsetter}{no\_arg\_setter}\ \Tsetter \parsesep \Tidentifier \parsesep \Teq \parsesep \Ntypedidentifier \parsesep \Nfuncbody&
\hypertarget{def-typedecl}{}\\
|\ & \productionname{typedecl}{type\_decl}\ \Ttype \parsesep \Tidentifier \parsesep \Tof \parsesep \Ntydecl \parsesep \Nsubtypeopt \parsesep \Tsemicolon&
\hypertarget{def-subtypedecl}{}\\
|\ & \productionname{subtypedecl}{subtype\_decl}\ \Ttype \parsesep \Tidentifier \parsesep \Nsubtype \parsesep \Tsemicolon&
\hypertarget{def-globalstorage}{}\\
|\ & \productionname{globalstorage}{global\_storage}\ \Nstoragekeyword \parsesep \Nignoredoridentifier \parsesep \option{\Tcolon \parsesep \Nty} \parsesep \Teq \parsesep &\\
   & \wrappedline\ \Nexpr \parsesep \Tsemicolon &
\hypertarget{def-globaluninitvar}{}\\
|\ & \productionname{globaluninitvar}{global\_uninit\_var}\ \Tvar \parsesep \Nignoredoridentifier \parsesep \Tcolon \parsesep \Nty \parsesep \Tsemicolon&
\hypertarget{def-globalpragma}{}\\
|\ & \productionname{globalpragma}{global\_pragma}\ \Tpragma \parsesep \Tidentifier \parsesep \Clist{\Nexpr} \parsesep \Tsemicolon&
\end{flalign*}

\hypertarget{def-nsubtype}{}
\begin{flalign*}
\Nsubtype \derivesinline\ & \Tsubtypes \parsesep \Tidentifier \parsesep \Twith \parsesep \Nfields &\\
          |\              & \Tsubtypes \parsesep \Tidentifier &\\
\end{flalign*}

\hypertarget{def-nsubtypeopt}{}
\begin{flalign*}
\Nsubtypeopt           \derivesinline\ & \option{\Nsubtype} &
\end{flalign*}

\hypertarget{def-ntypedidentifier}{}
\begin{flalign*}
\Ntypedidentifier \derivesinline\ & \Tidentifier \parsesep \Nasty &
\end{flalign*}

\hypertarget{def-nopttypeidentifier}{}
\begin{flalign*}
\Nopttypedidentifier \derivesinline\ & \Tidentifier \parsesep \option{\Nasty} &
\end{flalign*}

\hypertarget{def-nasty}{}
\begin{flalign*}
\Nasty \derivesinline\ & \Tcolon \parsesep \Nty &
\end{flalign*}

\hypertarget{def-nreturntype}{}
\begin{flalign*}
\Nreturntype        \derivesinline\ & \Tarrow \parsesep \Nty &
\end{flalign*}

\hypertarget{def-nparamsopt}{}
\begin{flalign*}
\Nparamsopt \derivesinline\ & \emptysentence &\\
                   |\ & \Tlbrace \parsesep \Clist{\Nopttypedidentifier} \parsesep \Trbrace &
\end{flalign*}

\hypertarget{def-naccessargs}{}
\begin{flalign*}
\Naccessargs        \derivesinline\ & \Tlbracket \parsesep \Clist{\Ntypedidentifier} \parsesep \Trbracket &
\end{flalign*}

\hypertarget{def-nfuncargs}{}
\begin{flalign*}
\Nfuncargs          \derivesinline\ & \Tlpar \parsesep \Clist{\Ntypedidentifier} \parsesep \Trpar &
\end{flalign*}

\hypertarget{def-nmaybeemptystmtlist}{}
\begin{flalign*}
\Nmaybeemptystmtlist          \derivesinline\ & \emptysentence \;|\; \Nstmtlist &
\end{flalign*}

\hypertarget{def-nfuncbody}{}
\begin{flalign*}
\Nfuncbody          \derivesinline\ & \Tbegin \parsesep \Nmaybeemptystmtlist \parsesep \Tend &
\end{flalign*}

\hypertarget{def-nignoredoridentifier}{}
\begin{flalign*}
\Nignoredoridentifier \derivesinline\ & \Tminus \;|\; \Tidentifier &
\end{flalign*}

\vspace*{-\baselineskip}
\paragraph{Parsing note:} $\Tvar$ is not derived by $\Nlocaldeclkeyword$ to avoid an LR(1) conflict.
\hypertarget{def-nlocaldeclkeyword}{}
\begin{flalign*}
\Nlocaldeclkeyword \derivesinline\ & \Tlet \;|\; \Tconstant&
\end{flalign*}

\hypertarget{def-nstoragekeyword}{}
\begin{flalign*}
\Nstoragekeyword \derivesinline\ & \Tlet \;|\; \Tconstant \;|\; \Tvar \;|\; \Tconfig&
\end{flalign*}

\hypertarget{def-ndirection}{}
\begin{flalign*}
\Ndirection \derivesinline\ & \Tto \;|\; \Tdownto &
\end{flalign*}

\hypertarget{def-nalt}{}
\begin{flalign*}
\Nalt       \derivesinline\ & \Twhen \parsesep \Npatternlist \parsesep \option{\Twhere \parsesep \Nexpr} \parsesep \Tarrow \parsesep \Nstmtlist &\\
|\ & \Totherwise \parsesep \Nstmtlist &
\end{flalign*}

\hypertarget{def-notherwiseopt}{}
\begin{flalign*}
\Notherwiseopt     \derives\ & \option{\Totherwise \parsesep \Tarrow \parsesep \Nstmtlist} &
\end{flalign*}

\hypertarget{def-ncatcher}{}
\begin{flalign*}
\Ncatcher \derivesinline\ & \Twhen \parsesep \Tidentifier \parsesep \Tcolon \parsesep \Nty \parsesep \Tarrow \parsesep \Nstmtlist &\\
          |\              & \Twhen \parsesep \Nty \parsesep \Tarrow \parsesep \Nstmtlist &\\
\end{flalign*}

\hypertarget{def-nstmt}{}
\begin{flalign*}
\Nstmt \derivesinline\ & \Tif \parsesep \Nexpr \parsesep \Tthen \parsesep \Nstmtlist \parsesep \Nselse \parsesep \Tend &\\
|\ & \Tcase \parsesep \Nexpr \parsesep \Tof \parsesep \maybeemptylist{\Nalt} \parsesep \Tend &\\
|\ & \Twhile \parsesep \Nexpr \parsesep \Tdo \parsesep \Nstmtlist \parsesep \Tend &\\
|\ & \Tlooplimit \parsesep \Tlpar \parsesep \Nexpr \parsesep \Trpar \parsesep \Twhile \parsesep \Nexpr \parsesep \Tdo \parsesep \Nstmtlist \parsesep \Tend &\\
|\ & \Tfor \parsesep \Tidentifier \parsesep \Teq \parsesep \Nexpr \parsesep \Ndirection \parsesep
                    \Nexpr \parsesep \Tdo \parsesep \Nstmtlist \parsesep \Tend &\\
|\ & \Ttry \parsesep \Nstmtlist \parsesep \Tcatch \parsesep \nonemptylist{\Ncatcher} \parsesep \Notherwiseopt \parsesep \Tend &\\
|\ & \Tpass \parsesep \Tsemicolon &\\
|\ & \Treturn \parsesep \option{\Nexpr} \parsesep \Tsemicolon &\\
|\ & \Tidentifier \parsesep \Plist{\Nexpr} \parsesep \Tsemicolon &\\
|\ & \Tassert \parsesep \Nexpr \parsesep \Tsemicolon &\\
|\ & \Nlocaldeclkeyword \parsesep \Ndeclitem \parsesep \Teq \parsesep \Nexpr \parsesep \Tsemicolon &\\
|\ & \Nlexpr \parsesep \Teq \parsesep \Nexpr \parsesep \Tsemicolon &\\
|\ & \Tvar \parsesep \Ndeclitem \parsesep \option{\Teq \parsesep \Nexpr} \parsesep \Tsemicolon &\\
|\ & \Tvar \parsesep \Clisttwo{\Tidentifier} \parsesep \Tcolon \parsesep \Nty \parsesep \Tsemicolon &\\
|\ & \Tprint \parsesep \Plist{\Nexpr} \parsesep \Tsemicolon &\\
%|\ & \Tdebug \parsesep \Plist{\Nexpr} \parsesep \Tsemicolon &\\
|\ & \Trepeat \parsesep \Nstmtlist \parsesep \Tuntil \parsesep \Nexpr \parsesep \Tsemicolon &\\
|\ & \Tlooplimit \parsesep \Tlpar \parsesep \Nexpr \parsesep \Trpar \parsesep \Trepeat \parsesep \Nstmtlist \parsesep \Tuntil \parsesep \Nexpr \parsesep \Tsemicolon &\\
|\ & \Tthrow \parsesep \Nexpr \parsesep \Tsemicolon &\\
|\ & \Tthrow \parsesep \Tsemicolon &\\
|\ & \Tpragma \parsesep \Tidentifier \parsesep \Clist{\Nexpr} \parsesep \Tsemicolon &
\end{flalign*}

\hypertarget{def-nstmtlist}{}
\begin{flalign*}
\Nstmtlist \derivesinline\ & \nonemptylist{\Nstmt} &
\end{flalign*}

\hypertarget{def-nselse}{}
\begin{flalign*}
\Nselse \derives\ & \Telseif \parsesep \Nexpr \parsesep \Twhen \parsesep \Nstmtlist \parsesep \Nselse &\\
|\ & \Tpass &\\
|\ & \Telse \parsesep \Nstmtlist &
\end{flalign*}

\hypertarget{def-nlexpr}{}
\begin{flalign*}
\Nlexpr \derivesinline\ & \Nlexpratom &\\
|\ & \Tminus &\\
|\ & \Tlpar \parsesep \NClist{\Nlexpr} \parsesep \Trpar &
\end{flalign*}

\hypertarget{def-nlexpratom}{}
\begin{flalign*}
\Nlexpratom \derives\ & \Tidentifier &\\
|\ & \Nlexpratom \parsesep \Nslices &\\
|\ & \Nlexpratom \parsesep \Tdot \parsesep \Tidentifier{\field} &\\
|\ & \Nlexpratom \parsesep \Tdot \parsesep \Tlbracket \parsesep \Clist{{\Tidentifier}} \parsesep \Trbracket &\\
|\ & \Tlbracket \parsesep \NClist{{\Nlexpratom}} \parsesep \Trbracket &
\end{flalign*}

A $\Ndeclitem$ is another kind of left-hand-side expression,
which appears only in declarations. It cannot have setter calls or set record fields,
it must declare a new variable.
\hypertarget{def-ndeclitem}{}
\begin{flalign*}
\Ndeclitem \derives\ & \Nuntypeddeclitem \parsesep \Nasty&\\
|\ & \Nuntypeddeclitem  &
\end{flalign*}

\hypertarget{def-nuntypeddeclitem}{}
\begin{flalign*}
\Nuntypeddeclitem \derivesinline\ & \Tidentifier &\\
|\ & \Tminus &\\
|\ & \Plisttwo{\Ndeclitem} &
\end{flalign*}

\hypertarget{def-nintconstraints}{}
\begin{flalign*}
\Nintconstraints \derivesinline\ & \Tlbrace \parsesep \NClist{\Nintconstraint} \parsesep \Trbrace &
\end{flalign*}

\hypertarget{def-nintconstraintsopt}{}
\begin{flalign*}
\Nintconstraintsopt \derivesinline\ & \Nintconstraints \;|\; \emptysentence &
\end{flalign*}

\hypertarget{def-nintconstraint}{}
\begin{flalign*}
\Nintconstraint \derivesinline\ & \Nexpr &\\
|\ & \Nexpr \parsesep \Tslicing \parsesep \Nexpr &
\end{flalign*}

\hypertarget{def-nexprpattern}{}
\begin{flalign*}
\Nexprpattern \derives\ & \Nvalue &\\
                    |\  & \Tidentifier &\\
                    |\  & \Nexprpattern \parsesep \Nbinop \parsesep \Nexpr &\\
                    |\  & \Nunop \parsesep \Nexpr & \precedence{\Tunops}\\
                    |\  & \Tif \parsesep \Nexpr \parsesep \Tthen \parsesep \Nexpr \parsesep \Neelse &\\
                    |\  & \Tidentifier \parsesep \Plist{\Nexpr} &\\
                    |\  & \Nexprpattern \parsesep \Nslices &\\
                    |\  & \Nexprpattern \parsesep \Tdot \parsesep \Tidentifier&\\
                    |\  & \Nexprpattern \parsesep \Tdot \parsesep \Tlbracket \parsesep \NClist{\Tidentifier} \parsesep \Trbracket &\\
                    |\  & \Tlbracket \parsesep \NClist{\Nexpr} \parsesep \Trbracket &\\
                    |\  & \Nexprpattern \parsesep \Tas \parsesep \Nty &\\
                    |\  & \Nexprpattern \parsesep \Tas \parsesep \Nintconstraints &\\
                    |\  & \Nexprpattern \parsesep \Tin \parsesep \Npatternset &\\
                    |\  & \Nexprpattern \parsesep \Tin \parsesep \Tmasklit &\\
                    |\  & \Tunknown \parsesep \Tcolon \parsesep \Nty &\\
                    |\  & \Tidentifier \parsesep \Tlbrace \parsesep \Clist{\Nfieldassign} \parsesep \Trbrace &\\
                    |\  & \Tlpar \parsesep \Nexprpattern \parsesep \Trpar &
\end{flalign*}

\hypertarget{def-npatternset}{}
\begin{flalign*}
\Npatternset \derivesinline\  & \Tbnot \parsesep \Tlbrace \parsesep \Npatternlist \parsesep \Trbrace &\\
                  |\    & \Tlbrace \parsesep \Npatternlist \parsesep \Trbrace &
\end{flalign*}

\hypertarget{def-npatternlist}{}
\begin{flalign*}
\Npatternlist \derivesinline\ & \NClist{\Npattern} &
\end{flalign*}

\hypertarget{def-npattern}{}
\begin{flalign*}
\Npattern \derives\ & \Nexprpattern &\\
                |\  & \Nexprpattern \parsesep \Tslicing \parsesep \Nexpr &\\
                |\  & \Tminus &\\
                |\  & \Tleq \parsesep \Nexpr &\\
                |\  & \Tgeq \parsesep \Nexpr &\\
                |\  & \Tmasklit &\\
                |\  & \Plisttwo{\Npattern} &\\
                |\  & \Npatternset &
\end{flalign*}

\hypertarget{def-nfields}{}
\begin{flalign*}
\Nfields \derivesinline\ & \Tlbrace \parsesep \TClist{\Ntypedidentifier} \parsesep \Trbrace &
\end{flalign*}

\hypertarget{def-nfieldsopt}{}
\begin{flalign*}
\Nfieldsopt \derivesinline\ & \Nfields \;|\; \emptysentence &
\end{flalign*}

\hypertarget{def-nnslices}{}
\begin{flalign*}
\Nnslices \derivesinline\ & \Tlbracket \parsesep \NClist{\Nslice} \parsesep \Trbracket &
\end{flalign*}

\hypertarget{def-nslices}{}
\begin{flalign*}
\Nslices \derivesinline\ & \Tlbracket \parsesep \Clist{\Nslice} \parsesep \Trbracket &
\end{flalign*}

\hypertarget{def-nslice}{}
\begin{flalign*}
\Nslice \derivesinline\ & \Nexpr &\\
              |\  & \Nexpr \parsesep \Tcolon \parsesep \Nexpr &\\
              |\  & \Nexpr \parsesep \Tpluscolon \parsesep \Nexpr &\\
              |\  & \Nexpr \parsesep \Tstarcolon \parsesep \Nexpr &
\end{flalign*}

\hypertarget{def-nbitfields}{}
\begin{flalign*}
\Nbitfields \derivesinline\ & \Tlbrace \parsesep \TClist{\Nbitfield} \parsesep \Trbrace &
\end{flalign*}

\hypertarget{def-nbitfield}{}
\begin{flalign*}
\Nbitfield \derivesinline\ & \Nnslices \parsesep \Tidentifier &\\
                  |\ & \Nnslices \parsesep \Tidentifier \parsesep \Nbitfields &\\
                  |\ & \Nnslices \parsesep \Tidentifier \parsesep \Tcolon \parsesep \Nty &
\end{flalign*}

\hypertarget{def-nty}{}
\begin{flalign*}
\Nty \derives\ & \Tinteger \parsesep \option{\Nintconstraints} &\\
            |\ & \Treal &\\
            |\ & \Tboolean &\\
            |\ & \Tstring &\\
            |\ & \Tbit &\\
            |\ & \Tbits \parsesep \Tlpar \parsesep \Nexpr \parsesep \Trpar \parsesep \maybeemptylist{\Nbitfields} &\\
            |\ & \Plist{\Nty} &\\
            |\ & \Tidentifier &\\
            |\ & \Tarray \parsesep \Tlbracket \parsesep \Nexpr \parsesep \Trbracket \parsesep \Tof \parsesep \Nty &
\end{flalign*}

\hypertarget{def-ntydecl}{}
\begin{flalign*}
\Ntydecl \derives\ & \Nty &\\
            |\ & \Tenumeration \parsesep \Tlbrace \parsesep \NTClist{\Tidentifier} \parsesep \Trbrace &\\
            |\ & \Trecord \parsesep \Nfieldsopt &\\
            |\ & \Texception \parsesep \Nfieldsopt &
\end{flalign*}

\hypertarget{def-nfieldassign}{}
\begin{flalign*}
\Nfieldassign \derivesinline\ & \Tidentifier \parsesep \Teq \parsesep \Nexpr &
\end{flalign*}

\hypertarget{def-neelse}{}
\begin{flalign*}
\Neelse \derives\ & \Telse \parsesep \Nexpr &\\
                     |\ & \Telseif \parsesep \Nexpr \parsesep \Tthen \parsesep \Nexpr \parsesep \Neelse &
\end{flalign*}

\hypertarget{def-nexpr}{}
\begin{flalign*}
\Nexpr \derives\  & \Nvalue &\\
                    |\  & \Tidentifier &\\
                    |\  & \Nexpr \parsesep \Nbinop \parsesep \Nexpr &\\
                    |\  & \Nunop \parsesep \Nexpr & \precedence{\Tunops}\\
                    |\  & \Tif \parsesep \Nexpr \parsesep \Tthen \parsesep \Nexpr \parsesep \Neelse &\\
                    |\  & \Tidentifier \parsesep \Plist{\Nexpr} &\\
                    |\  & \Nexpr \parsesep \Nslices &\\
                    |\  & \Nexpr \parsesep \Tdot \parsesep \Tidentifier&\\
                    |\  & \Nexpr \parsesep \Tdot \parsesep \Tlbracket \parsesep \NClist{\Tidentifier} \parsesep \Trbracket &\\
                    |\  & \Tlbracket \parsesep \NClist{\Nexpr} \parsesep \Trbracket &\\
                    |\  & \Nexpr \parsesep \Tas \parsesep \Nty &\\
                    |\  & \Nexpr \parsesep \Tas \parsesep \Nintconstraints &\\
                    |\  & \Nexpr \parsesep \Tin \parsesep \Npatternset &\\
                    |\  & \Nexpr \parsesep \Tin \parsesep \Tmasklit &\\
                    |\  & \Tunknown \parsesep \Tcolon \parsesep \Nty &\\
                    |\  & \Tidentifier \parsesep \Tlbrace \parsesep \Clist{\Nfieldassign} \parsesep \Trbrace &\\
                    |\  & \Tlpar \parsesep \Nexpr \parsesep \Trpar &\\
                    |\  & \Plisttwo{\Nexpr} &
\end{flalign*}

\hypertarget{def-nvalue}{}
\begin{flalign*}
\Nvalue \derivesinline\ & \Tintlit &\\
                     |\ & \Tboollit &\\
                     |\ & \Treallit &\\
                     |\ & \Tbitvectorlit &\\
                     |\ & \Tstringlit &
\end{flalign*}

\hypertarget{def-nunop}{}
\begin{flalign*}
\Nunop \derivesinline\ & \Tbnot \;|\; \Tminus \;|\; \Tnot &
\end{flalign*}

\hypertarget{def-nbinop}{}
\begin{flalign*}
\Nbinop \derivesinline\ & \Tand \;|\; \Tband \;|\; \Tbor \;|\; \Tbeq \;|\; \Tdiv \;|\; \Tdivrm \;|\; \Txor \;|\; \Teqop \;|\; \Tneq &\\
                     |\ & \Tgt \;|\; \Tgeq \;|\; \Timpl \;|\; \Tlt \;|\; \Tleq \;|\; \Tplus \;|\; \Tminus \;|\; \Tmod \;|\; \Tmul &\\
                     |\ & \Tor \;|\; \Trdiv \;|\; \Tshl \;|\; \Tshr \;|\; \Tpow \;|\; \Tconcat
\end{flalign*}

\section{Parse Trees \label{sec:ParseTrees}}
We now define \emph{parse trees} for the ASL expanded grammar. Those are later used for build Abstract Syntax Trees.

\begin{definition}[Parse Trees]
A \emph{parse tree} has one of the following forms:
\begin{itemize}
  \item A \emph{token node}, given by the token itself, for example, $\Tlexeme(\Tarrow)$ and $\Tidentifier(\id)$;
  \item \hypertarget{def-epsilonnode}{} $\epsilonnode$, which represents the empty sentence --- $\emptysentence$.
  \item A \emph{non-terminal node} of the form $N(n_{1..k})$ where $N$ is a non-terminal symbol,
        which is said to label the node,
        and $n_{1..k}$ are its children parse nodes,
        for example,
        $\Ndecl(\Tfunc, \Tidentifier(\id), \Nparamsopt, \Nfuncargs, \Nfuncbody)$
        is labeled by $\Ndecl$ and has five children nodes.
\end{itemize}
\end{definition}
(In the literature, parse trees are also referred to as \emph{derivation trees}.)

\begin{definition}[Well-formed Parse Trees]
A parse tree is \emph{well-formed} if its root is labelled by the start non-terminal ($\Nast$ for ASL)
and each non-terminal node $N(n_{1..k})$ corresponds to a grammar derivation
$N \derives l_{1..k}$ where $l_i$ is the label of node $n_i$ if it is a non-terminal node and $n_i$
itself when it is a token.
A non-terminal node $N(\epsilonnode)$ is well-formed if the grammar includes a derivation
$N \derives \emptysentence$.
\end{definition}

\hypertarget{def-yield}{}
\begin{definition}[Parse Tree Yield]
The \emph{\yield} of a parse tree is the list of its tokens
given by an in-order walk of the tree:
\[
\yield(n) \triangleq \begin{cases}
  [t] & n \text{ is a token }t\\
  \emptylist & n = \epsilonnode\\
  \yield(n_1) \concat \ldots \concat \yield(n_k) & n = N(n_{1..k})\\
\end{cases}
\]
\end{definition}

% \newcommand\cleantokens[0]{\textsf{clean\_tokens}}
% The function $\cleantokens : \Token^* \aslto \Token^*$
% filters out all whitespace tokens and the $\eoftoken$,
% leaving all other tokens in the same order.

\hypertarget{def-parsenode}{}
We denote the set of well-formed parse trees for a non-terminal symbol $S$ by $\parsenode{S}$.

A parser is a function
\[
\aslparse : \Token^* \setminus \{\Terror\} \aslto \parsenode{\Nast} \cup \{\ParseError\}
\]
such that if $\aslparse(\ts) = n$ then $\yield(n)=\ts$ $\yield(n)=\ts$
and if $\aslparse(\ts) = \ParseError$ then there is no well-formed tree
$n$ such that $\yield(n)=\ts$.
(Notice that we do not define a parser if $\ts$ is lexically illegal.)

The language of a grammar $G$ is defined as follows:
\[
\Lang(G) = \{\yield(n) \;|\; n \text{ is a well-formed parse tree for }G\} \enspace.
\]

\section{Priority and Associativity \label{sec:PriorityAndAssociativity}}
A context-free grammar $G$ is \emph{ambiguous} if there can be more than one parse tree for a given list of tokens
$\ts \in \Lang(G)$.
Indeed the expanded ASL grammar is ambiguous, for example, due to its definition of binary operation expressions.
To allow assigning a unique parse tree to each sequence of tokens in the language of the ASL grammar,
we utilize the standard technique of associating priority levels to productions and using them to resolve
any shift-reduce conflicts in the LR(1) parser associated with our grammar (our grammar does not have any
reduce-reduce conflicts).

The priority of a grammar derivation is defined as the priority of its rightmost token.
Derivations that do not contain tokens do not require a priority as they do not induce shift-reduce conflicts.

The table below assigns priorities to tokens in increasing order, starting from the lowest priority (for $\Telse$)
to the highest priority (for $\Tdot$).
When a shift-reduce conflict arises during the LR(1) grammar construction
it resolve in favor of the action (shift or reduce) associated with the derivation that has the higher priority.
If two derivations have the same priority due to them both having the same rightmost token,
the conflict is resolved based on the associativity associated with the token below:
reduce for $\leftassoc$, shift for $\rightassoc$, and a parsing error for $\nonassoc$.

The two rules involving a unary minus operation are not assigned the priority level of $\Tminus$,
but rather then the priority level $\Tunops$, as denoted by the notation \\
$\precedence{\Tunops}$
appearing to their right. This is a standard way of dealing with a unary minus operation
in many programming languages, which involves defining an artificial token $\Tunops$,
which is never returned by the lexical analysis.

\begin{center}
\begin{tabular}{ll}
\textbf{Terminals} & \textbf{Associativity}\\
\hline
\Telse & \nonassoc\\
\Tbor, \Tband, \Timpl, \Tbeq, \Tas & \leftassoc\\
\Teqop, \Tneq & \leftassoc\\
\Tgt, \Tgeq, \Tlt, \Tleq & \nonassoc\\
\Tplus, \Tminus, \Tor, \Txor, \Tand & \leftassoc\\
\Tmul, \Tdiv, \Tdivrm, \Trdiv, \Tmod, \Tshl, \Tshr & \leftassoc\\
\Tpow, \Tconcat & \leftassoc\\
$\Tunops$ & \nonassoc\\
\Tin & \nonassoc\\
\Tdot, \Tlbracket & \leftassoc
\end{tabular}
\end{center}

%%%%%%%%%%%%%%%%%%%%%%%%%%%%%%%%%%%%%%%%%%%%%%%%%%%%%%%%%%%%%%%%%%%%%%%%%%%%
\chapter{ASL Abstract Syntax \label{chap:ASLAbstractSyntax}}
%%%%%%%%%%%%%%%%%%%%%%%%%%%%%%%%%%%%%%%%%%%%%%%%%%%%%%%%%%%%%%%%%%%%%%%%%%%%
An abstract syntax is a form of context-free grammar over structured trees.
Compilers and interpreters typically start by parsing the text of a program and producing an abstract syntax tree (AST, for short),
and then continue to operate over that tree.
%
The reason for this is that abstract syntax trees abstract away details that are irrelevant to the semantics of the program,
such as punctuation and scoping syntax, which are mostly there to facilitate parsing.

Technically, there are two abstract syntaxes:
a \emph{parsed abstract syntax} and a \emph{typed abstract syntax}.
The first syntax results from parsing the text of an ASL specification.
The type checker checks whether the parsed AST is valid and if so produces
a typed AST where some nodes in the parsed AST have been transformed to
more explicit representation. For example, the parsed AST may contain
what looks like a slicing expression, which turns out to be a call to a getter.
The typed AST represents that call directly, making it easier for an interpreter
to evaluate that expression.

\section{ASL Abstract Syntax Trees}

In an ASL abstract syntax tree, a node is one the following data types:
\begin{description}
\item[Token Node.] A lexical token, denoted as in the lexical description of ASL;
\item[Label Node.] A label
\item[Unlabelled Tuple Node.] A tuple of children nodes, denoted as $(n_1,\ldots,n_k)$;
\item[Labelled Tuple Node.] A tuple labelled~$L$, denoted as~$L(n_1,\ldots,n_k)$;
\item[List Node.] A list of~$0$ or more children nodes, denoted as~$\emptylist$
      when the list is empty and~$[n_1,\ldots,n_k]$ for non-empty lists;
\item[Optional.] An optional node stands for a list of 0 or 1 occurrences of a sub-node $n$. We denote an empty optional by $\langle\rangle$ and the non-empty optional by $\langle n \rangle$;
\item[Record Node.] A record node, denoted as $\{\text{name}_1 : n_1,\ldots,\text{name}_k : n_k\}$, where \\
      $\text{name}_1 \ldots \text{name}_k$ are names, which associates names with corresponding nodes.
\end{description}

\newpage

\section{ASL Abstract Syntax Grammar}

An abstract syntax is defined in terms of derivation rules containing variables (also referred to as non-terminals).
%
A \emph{derivation rule} has the form $v \derives \textit{rhs}$ where $v$ is a non-terminal variable and \textit{rhs} is a \emph{node type}. We write $n$, $n_1,\ldots,n_k$ to denote node types.
%
Node types are defined recursively as follows:
\begin{description}
\item[Non-terminal.] A non-terminal variable;
\item[Terminal.] A lexical token $t$ or a label $L$;
\item[Unlabelled Tuple.] A tuple of node types, denoted as~$(n_1,\ldots,n_k)$;
\item[Labelled Tuple.] A tuple labelled~$L$, denoted as~$L(n_1,\ldots,n_k)$;
\item[List.] A list node type, denoted as $n^{*}$;
\item[Optional.] An optional node type, denoted as $n?$;
\item[Record.] A record, denoted as $\{\text{name}_1 : n_1,\ldots,\text{name}_k : n_k\}$ where $\text{name}_i$, which associates names with corresponding node types.
\end{description}

\newpage

An abstract syntax consists of a set of derivation rules and a start non-terminal.

\newcommand\ASTComment[1]{//\quad\textit{#1}\ }

\section{ASL Parsed Abstract Syntax}

The abstract syntax of ASL is given in terms of the derivation rules below and the start non-terminal $\specification$.
%
We sometimes provide extra details to individual derivations by adding comments below them, in the form \ASTComment{this is a comment}.

\hypertarget{ast-unop}{} \hypertarget{ast-bnot}{} \hypertarget{ast-neg}{} \hypertarget{ast-not}{}
\[
\begin{array}{rcl}
\unop &\derives& \overname{\BNOT}{\texttt{"!"}} \;|\; \overname{\NEG}{\texttt{"-"}} \;|\; \overname{\NOT}{\texttt{"NOT"}}
\hypertarget{ast-binop}{} \hypertarget{ast-bor}{} \hypertarget{ast-impl}{} \hypertarget{ast-beq}{}
\\
\binop  &\derives& \overname{\BAND}{\texttt{"\&\&"}} \;|\; \overname{\BOR}{\texttt{"||"}} \;|\; \overname{\IMPL}{\texttt{"-->"}}
              \;|\; \overname{\BEQ}{\texttt{"<->"}}
\hypertarget{ast-eqop}{} \hypertarget{ast-neq}{} \hypertarget{ast-gt}{} \hypertarget{ast-geq}{} \hypertarget{ast-lt}{} \hypertarget{ast-leq}{}
\\
        &|& \overname{\EQOP}{\texttt{"=="}} \;|\; \overname{\NEQ}{\texttt{"!="}} \;|\; \overname{\GT}{\texttt{"<"}}
        \;|\; \overname{\GEQ}{\texttt{">="}} \;|\; \overname{\LT}{\texttt{"<"}} \;|\; \overname{\LEQ}{\texttt{"<="}}
\hypertarget{ast-plus}{} \hypertarget{ast-minus}{} \hypertarget{ast-or}{} \hypertarget{ast-xor}{} \hypertarget{ast-and}{}
\\
        &|& \overname{\PLUS}{\texttt{"+"}} \;|\; \overname{\MINUS}{\texttt{"-"}} \;|\; \overname{\OR}{\texttt{"OR"}}
        \;|\; \overname{\XOR}{\texttt{"XOR"}} \;|\; \overname{\AND}{\texttt{"AND"}}
\hypertarget{ast-mul}{} \hypertarget{ast-div}{} \hypertarget{ast-divrm}{} \hypertarget{ast-mod}{} \hypertarget{ast-shl}{} \hypertarget{ast-shr}{}
\\
        &|& \overname{\MUL}{\texttt{"*"}} \;|\; \overname{\DIV}{\texttt{"DIV"}} \;|\; \overname{\DIVRM}{\texttt{"DIVRM"}}
        \;|\; \overname{\MOD}{\texttt{"MOD"}} \;|\; \overname{\SHL}{\texttt{"<<"}}  \;|\; \overname{\SHR}{\texttt{">>"}}
\hypertarget{ast-rdiv}{} \hypertarget{ast-pow}{}
\\
        &|& \overname{\RDIV}{\texttt{"/"}} \;|\; \overname{\POW}{\texttt{"\^{}"}}
\hypertarget{ast-literal}{} \hypertarget{ast-lint}{}
\\
\literal &\derives& \lint(\overname{n}{\Z})
\hypertarget{ast-lbool}{}
\\
 &|& \lbool(\overname{b}{\{\True, \False\}})
\hypertarget{ast-lreal}{}
\\
 &|& \lreal(\overname{q}{\Q})
\hypertarget{ast-lbitvector}{}
\\
 &|& \lbitvector(\overname{B}{B \in \{0, 1\}^*})
\hypertarget{ast-lstring}{}
\\
 &|& \lstring(\overname{S}{S \in \{C \;|\; \texttt{"$C$"} \in \texttt{<string\_lit>}\}})\\
\end{array}
\]

\hypertarget{ast-expr}{} \hypertarget{ast-eliteral}{}
\[
\begin{array}{rcl}
\expr &\derives& \ELiteral(\literal)
\hypertarget{ast-evar}{} \hypertarget{ast-identifier}{}\\
	&|& \EVar(\overtext{\identifier}{variable name})
\hypertarget{ast-eatc}{}\\
	&|& \overtext{\EATC}{Type assertion}(\overtext{\expr}{given expression}, \overtext{\ty}{asserted type})
\hypertarget{ast-ebinop}{}\\
	&|& \EBinop(\binop, \expr, \expr)
  \hypertarget{ast-eunop}{}\\
	&|& \EUnop(\unop, \expr)
  \hypertarget{ast-ecall}{}\\
	&|& \ECall(\overtext{\identifier}{subprogram name}, \overtext{\expr^{*}}{actual arguments})
  \hypertarget{ast-eslice}{}\\
	&|& \ESlice(\expr, \slice^{*})
  \hypertarget{ast-econd}{}\\
	&|& \ECond(\overtext{\expr}{condition}, \overtext{\expr}{then}, \overtext{\expr}{else})
  \hypertarget{ast-egetfield}{}\\
	&|& \EGetField(\overtext{\expr}{record}, \overtext{\identifier}{field name})
  \hypertarget{ast-egetfields}{}\\
	&|& \EGetFields(\overtext{\expr}{record}, \overtext{\identifier^{*}}{field names})
  \hypertarget{ast-erecord}{}\\
	&|& \ERecord(\overtext{\ty}{record type}, \overtext{(\identifier, \expr)^{*}}{field initializers})
  \hypertarget{ast-econcat}{}\\
    & & \ASTComment{Both record construction and exception construction}\\
	&|& \EConcat(\expr^{+})
  \hypertarget{ast-etuple}{}\\
	&|& \ETuple(\expr^{+})
  \hypertarget{ast-eunknown}{}\\
	&|& \EUnknown(\ty)
  \hypertarget{ast-epattern}{}\\
	&|& \EPattern(\expr, \pattern)
\end{array}
\]

\hypertarget{ast-pattern}{} \hypertarget{ast-patternall}{}
\[
\begin{array}{rcl}
\pattern &\derives& \PatternAll
\hypertarget{ast-patternany}{}\\
  &|& \PatternAny(\pattern^{*})
  \hypertarget{ast-patterngeq}{}\\
  &|& \PatternGeq(\expr)
  \hypertarget{ast-patternleq}{}\\
  &|& \PatternLeq(\expr)
  \hypertarget{ast-patternmask}{}\\
  &|& \PatternMask(\Tmasklit)
  \hypertarget{ast-patternnot}{}\\
  &|& \PatternNot(\pattern)
  \hypertarget{ast-patternrange}{}\\
  &|& \PatternRange(\overtext{\expr}{lower}, \overtext{\expr}{upper included})
  \hypertarget{ast-patternsingle}{}\\
  &|& \PatternSingle(\expr)
  \hypertarget{ast-patterntuple}{}\\
  &|& \PatternTuple(\pattern^{*}) \\
\end{array}
\]

\hypertarget{ast-slice}{} \hypertarget{ast-slicesingle}{}
\[
\begin{array}{rcl}
&&\ASTComment{Indexes an array or a bitvector.}\\
&&\ASTComment{All positions mentioned below are inclusive}\\
\hline
\slice &\derives& \SliceSingle(\overname{\expr}{\vi})
\hypertarget{ast-slicerange}{}\\
  & & \ASTComment{the slice of length \texttt{1} at position \vi.}\\
  &|& \SliceRange(\overname{\expr}{\vj}, \overname{\expr}{\vi})
  \hypertarget{ast-slicelength}{}\\
  & & \ASTComment{the slice from \vi\ to \texttt{j - 1}.}\\
  &|& \SliceLength(\overname{\expr}{\vi}, \overname{\expr}{\vn})
  \hypertarget{ast-slicestar}{}\\
  & & \ASTComment{the slice starting at \vi\ of length \vn.}\\
  &|& \SliceStar(\overname{\expr}{\vi}, \overname{\expr}{\vn}) \\
  & & \ASTComment{the slice starting at \texttt{i*n} of length \vn}
\end{array}
\]

\hypertarget{ast-ty}{} \hypertarget{ast-tint}{}
\[
\begin{array}{rcl}
\ty &\derives& \TInt(\intconstraints)
\hypertarget{ast-treal}{}\\
  &|& \TReal
  \hypertarget{ast-tstring}{}\\
  &|& \TString
  \hypertarget{ast-tbool}{}\\
  &|& \TBool
  \hypertarget{ast-tbits}{}\\
  &|& \TBits(\overtext{\expr}{width}, \bitfield^{*})
  \hypertarget{ast-tenum}{}\\
  &|& \TEnum(\overtext{\identifier^{*}}{labels})
  \hypertarget{ast-ttuple}{}\\
  &|& \TTuple(\ty^{*})
  \hypertarget{ast-tarray}{}\\
  &|& \TArray(\arrayindex, \ty)
  \hypertarget{ast-trecord}{}\\
  &|& \TRecord(\Field^{*})
  \hypertarget{ast-texception}{}\\
  &|& \TException(\Field^{*})
  \hypertarget{ast-tnamed}{}\\
  &|& \TNamed(\overtext{\identifier}{type name})\\
%  & & \ASTComment{This is related to \identi{LDNP}}
\end{array}
\]

\hypertarget{ast-intconstraints}{} \hypertarget{ast-unconstrained}{}
\[
  \begin{array}{rcl}
    & & \ASTComment{Constraints that may be assigned to integer types.}  \\
    \hline
    \intconstraints & \derives
      & \unconstrained                                      \\
    & & \ASTComment{The unconstrained integer type.}
    \hypertarget{ast-wellconstrained}{}\\
    &|& \wellconstrained(\intconstraint^{+})                \\
    & & \ASTComment{An integer type with explicit constraints.}
    \hypertarget{ast-parameterized}{}\\
    &|& \parameterized(\overtext{\identifier}{parameter})        \\
    % & & \ASTComment{Implicitly constrained integer from function declaration.} \\
    % & & \ASTComment{Attributes are:} \\
    % & & \ASTComment{- a unique integer identifier and the variable} \\
    % & & \ASTComment{- the type was implicitly constructed from.} \\
  \end{array}
\]

\hypertarget{ast-intconstraint}{} \hypertarget{ast-constraintexact}{}
\[
\begin{array}{rcl}
& & \ASTComment{A constraint on an integer part.}\\
\hline
\intconstraint &\derives& \ConstraintExact(\expr) \\
  & & \ASTComment{A single value, given by a statically evaluable expression.}
  \hypertarget{ast-constraintrange}{}\\
  &|& \ConstraintRange(\expr, \expr) \\
  & & \ASTComment{An interval between two statically evaluable expression.}\\
\end{array}
\]

\hypertarget{ast-bitfield}{} \hypertarget{ast-bitfieldsimple}{}
\[
\begin{array}{rcl}
& & \ASTComment{Represent static slices on a given bitvector type.}\\
\hline
\bitfield &\derives& \BitFieldSimple(\identifier, \slice^{*}) \\
  & & \ASTComment{A name and its corresponding slice.}
  \hypertarget{ast-bitfieldnested}{}\\
  &|& \BitFieldNested(\identifier, \slice^{*}, \bitfield^{*}) \\
  & & \ASTComment{A name, its corresponding slice and some nested bitfields.}
  \hypertarget{ast-bitfieldtype}{}\\
  &|& \BitFieldType(\identifier, \slice^{*}, \ty) \\
  & & \ASTComment{A name, its corresponding slice, and the type of the bitfield.}\\
\end{array}
\]

\hypertarget{ast-arrayindex}{} \hypertarget{ast-arraylengthexpr}{}
\[
  \begin{array}{rcl}
    & & \ASTComment{The type of indexes for an array.}  \\
    \hline
    \arrayindex & \derives
      & \ArrayLengthExpr(\overtext{\expr}{array length})
      \hypertarget{ast-arraylengthenum}{}\\
    &|& \ArrayLengthEnum(\overtext{\identifier}{name of enumeration}, \overtext{\Z}{length}) \\
  \end{array}
\]

\hypertarget{ast-field}{}
\[
\begin{array}{rcl}
\Field &\derives& (\identifier, \ty)\\
  & & \ASTComment{A field of a record-like structure.}
  \hypertarget{ast-typedidentifier}{}\\
\typedidentifier &\derives& (\identifier, \ty)\\
  & & \ASTComment{An identifier declared with its type.}\\
\end{array}
\]

\hypertarget{ast-lexpr}{} \hypertarget{ast-lediscard}{}
\[
\begin{array}{rcl}
& & \ASTComment{Type of left-hand side of assignments.}\\
\hline
\lexpr &\derives& \LEDiscard\\
  & & \ASTComment{\texttt{"-"}}\\
  &|& \LEVar(\identifier)
  \hypertarget{ast-leslice}{}\\
  &|& \LESlice(\lexpr, \slice^*)
  \hypertarget{ast-lesetarray}{}\\
  &|& \LESetArray(\lexpr, \expr)
  \hypertarget{ast-lesetfield}{}\\
  &|& \LESetField(\lexpr, \identifier)
  \hypertarget{ast-lesetfields}{}\\
  &|& \LESetFields(\lexpr, \identifier^*)
  \hypertarget{ast-ledestructuring}{}\\
  &|& \LEDestructuring(\lexpr^*)
  \hypertarget{ast-leconcat}{}\\
  &|& \LEConcat(\lexpr^+)\\
  & & \ASTComment{$\LEConcat(\texttt{les}, \_)$ unpacks a list of \lexpr.}
\end{array}
\]

\hypertarget{ast-localdeclkeyword}{} \hypertarget{ast-ldkvar}{} \hypertarget{ast-ldkconstant}{} \hypertarget{ast-ldklet}{}
\[
\begin{array}{rcl}
\localdeclkeyword &\derives& \LDKVar \;|\; \LDKConstant \;|\; \LDKLet\\
\end{array}
\]

\hypertarget{ast-localdeclitem}{} \hypertarget{ast-ldidiscard}{}
\[
\begin{array}{rcl}
  & & \ASTComment{A left-hand side of a declaration statement.}\\
  & & \ASTComment{In the following example of a declaration statement:}\\
  & & \ASTComment{\Verb|let (x, -, z): (integer, integer, integer \{0..32\}) = (2, 3, 4);|}\\
  & & \ASTComment{\Verb|(x, -, z): (integer, integer, integer \{0..32\})| is the}\\
  & & \ASTComment{local declaration item:}\\
  \hline
\localdeclitem &\derives
    & \LDIDiscard\\
  & & \ASTComment{The ignored local declaration item, for example used in: \Verb!let - = 42;!.}
  \hypertarget{ast-ldivar}{}\\
  &|& \LDIVar(\identifier)\\
  & & \ASTComment{\texttt{LDI\_Var x} is the variable declaration of the variable \texttt{x}, used for example in:}\\
  & & \ASTComment{\texttt{let x = 42;}.}
  \hypertarget{ast-ldituple}{}\\
  &|& \LDITuple(\localdeclitem^*)\\
  & & \ASTComment{\texttt{LDI\_Tuple ldis} is the tuple declarations of the items in \texttt{ldis},}\\
  & & \ASTComment{used for example in: \texttt{let (x, y, -, z) = (1, 2, 3, 4);}}\\
  & & \ASTComment{Note that a the list here must be at least 2 items long.}
  \hypertarget{ast-ldityped}{}\\
  &|& \LDITyped(\localdeclitem, \ty)\\
  & & \ASTComment{\texttt{LDI\_Typed (ldi, t)} declares the item \texttt{ldi} with type \texttt{t}, used for example in:} \\
  & & \ASTComment{\texttt{let x: integer = 4;}}
\end{array}
\]

\hypertarget{ast-fordirection}{} \hypertarget{ast-up}{} \hypertarget{ast-down}{}
\[
\begin{array}{rcl}
\fordirection &\derives& \UP \;|\; \DOWN\\
\end{array}
\]

\hypertarget{ast-stmt}{} \hypertarget{ast-spass}{}
\begin{flalign*}
\stmt \derives\ & \SPass
\hypertarget{ast-sseq}{} &\\
  |\ & \SSeq(\stmt, \stmt)
  \hypertarget{ast-sdecl}{} &\\
  |\ & \SDecl(\localdeclkeyword, \localdeclitem, \expr?)
  \hypertarget{ast-sassign}{} &\\
  |\ & \SAssign(\lexpr, \expr)
  \hypertarget{ast-scall}{} &\\
  |\ & \SCall(\overtext{\identifier}{subprogram name}, \overtext{\expr^{*}}{actual arguments})
  \hypertarget{ast-sreturn}{} &\\
  |\ & \SReturn(\expr?)
  \hypertarget{ast-scond}{} &\\
  |\ & \SCond(\expr, \stmt, \stmt)
  \hypertarget{ast-scase}{} &\\
  |\ & \SCase(\expr, \casealt^*)
  \hypertarget{ast-sassert}{} &\\
  |\ & \SAssert(\expr)
  \hypertarget{ast-sfor}{} &\\
  |\ & \SFor(\identifier, \expr, \fordirection, \expr, \stmt)
  \hypertarget{ast-swhile}{} &\\
  |\ & \SWhile(\overtext{\expr}{condition}, \overtext{\expr?}{loop limit}, \stmt)
  \hypertarget{ast-srepeat}{} &\\
  |\ & \SRepeat(\stmt, \overtext{\expr}{condition}, \overtext{\expr?}{loop limit})
  \hypertarget{ast-sthrow}{} &\\
  |\ & \SThrow((\expr, \None)?) &\\
     & \ASTComment{The option represents an implicit throw: \texttt{throw;}.}
  \hypertarget{ast-stry}{} &\\
  |\ & \STry(\stmt, \catcher^*, \overtext{\stmt?}{otherwise})
  \hypertarget{ast-sprint}{} &\\
  |\ & \SPrint(\overtext{\expr^*}{args}, \overtext{\Bool}{debug})
\end{flalign*}

\hypertarget{ast-casealt}{}
\begin{flalign*}
\casealt \derives\ & \{ \CasePattern : \pattern, \CaseWhere : \expr?, \CaseStmt : \stmt \} &
\end{flalign*}

\hypertarget{ast-catcher}{}
\begin{flalign*}
\catcher \derives\ & (\overtext{\identifier?}{exception to match}, \overtext{\ty}{guard type}, \overtext{\stmt}{statement to execute on match}) &
\end{flalign*}

\hypertarget{ast-subprogramtype}{} \hypertarget{ast-stprocedure}{} \hypertarget{ast-stfunction}{}
\begin{flalign*}
\subprogramtype \derives\ & \STProcedure \;|\; \STFunction
\hypertarget{ast-stgetter}{} \hypertarget{ast-stemptygetter}{} &\\
                |\  & \STGetter \;|\; \STEmptyGetter
                \hypertarget{ast-stsetter}{} \hypertarget{ast-stemptysetter}{} &\\
                |\  & \STSetter \;|\; \STEmptySetter &
\end{flalign*}

\hypertarget{ast-subprogrambody}{} \hypertarget{ast-sbasl}{}
\begin{flalign*}
\subprogrambody \derives\ & \SBASL(\stmt) \hypertarget{ast-sbprimitive}{} \;|\; \SBPrimitive &
\end{flalign*}

\hypertarget{ast-func}{}
\begin{flalign*}
\func \derives\ &
{
\left\{
  \begin{array}{rcl}
 \funcname &:& \Strings, \\
 \funcparameters &:& (\identifier, \ty?)^*,\\
 \funcargs &:& \typedidentifier^*,\\
 \funcbody &:& \subprogrambody,\\
 \funcreturntype &:& \ty?,\\
 \funcsubprogramtype &:& \subprogramtype
\end{array}
\right\}
} &
\end{flalign*}

Declaration keyword for global storage elements:
\hypertarget{ast-globaldeclkeyword}{} \hypertarget{ast-gdkconstant}{} \hypertarget{ast-gdkconfig}{} \hypertarget{ast-gdklet}{} \hypertarget{ast-gdkvar}{}
\begin{flalign*}
\globaldeclkeyword \derives\ & \GDKConstant \;|\; \GDKConfig \;|\; \GDKLet \;|\; \GDKVar &
\end{flalign*}

\hypertarget{ast-globaldecl}{}
\begin{flalign*}
\globaldecl \derives\ &
{\left\{
  \begin{array}{rcl}
  \GDkeyword &:& \globaldeclkeyword, \\
  \GDname &:& \identifier,\\
  \GDty &:& \ty?,\\
  \GDinitialvalue &:& \expr?
  \end{array}
  \right\}
 } &
\end{flalign*}

\hypertarget{ast-decl}{}
\hypertarget{ast-dfunc}{}
\begin{flalign*}
\decl \derives\ & \DFunc(\func) & \hypertarget{ast-dglobalstorage}{}\\
  |\ & \DGlobalStorage(\globaldecl) & \hypertarget{ast-dtypedecl}{}\\
  |\ & \DTypeDecl(\identifier, \ty, (\identifier, \overtext{\Field^*}{with fields})?) &
\end{flalign*}

\hypertarget{ast-specification}{}
\begin{flalign*}
\specification \derives\ & \decl^* &
\end{flalign*}

\section{ASL Typed Abstract Syntax}

The derivation rules for the typed abstract syntax are the same as the rules for the parsed abstract syntax,
except for the following differences.

The rules for expressions have the extra derivation rule:
\hypertarget{ast-egetarray}{}
\[
\begin{array}{rcl}
\expr &\derives& \EGetArray(\overtext{\expr}{base}, \overtext{\expr}{index}) \\
\end{array}
\]

The AST node for call expressions includes an extra component that explicitly associates expressions
with parameters:
\[
\begin{array}{rcl}
\expr &\derives& \ECall(\overtext{\identifier}{subprogram name}, \overtext{\expr^{*}}{actual arguments}, \overtext{(\identifier, \expr)^{*}}{parameters with initializers})
\end{array}
\]

The AST node for a left-hand-side tuple of expressions contains a second component, which is a type annotation:
\[
\begin{array}{rcl}
\lexpr &\derives& \LEConcat(\lexpr^+, \Tintlit^+?)
\end{array}
\]

The rules for statements refine the throw statement by annotating it with
the type of the throw exception.
\[
\begin{array}{rcl}
\stmt &\derives& \SThrow((\expr, \overtext{\langle\ty\rangle}{exception type})?)
\end{array}
\]

Similar to expressions, the AST node for call statements includes an extra component that explicitly associates expressions
with parameters:
\[
\begin{array}{rcl}
  \stmt &\derives&  \SCall(\overtext{\identifier}{subprogram name}, \overtext{\expr^{*}}{actual arguments}, \overtext{(\identifier, \expr)^{*}}{parameters with initializers})
\end{array}
\]

The rules for slices is replaced by the following:
\[
\begin{array}{rcl}
\slice &\derives& \SliceLength(\expr, \expr) \\
\end{array}
\]
This reflects the fact that all other slicing constructs are syntactic sugar
for \SliceLength.

%%%%%%%%%%%%%%%%%%%%%%%%%%%%%%%%%%%%%%%%%%%%%%%%%%%%%%%%%%%%%%%%%%%%%%%%%%%%%%%%
\chapter{Building Abstract Syntax Trees \label{chap:BuildingAbstractSyntaxTrees}}
%%%%%%%%%%%%%%%%%%%%%%%%%%%%%%%%%%%%%%%%%%%%%%%%%%%%%%%%%%%%%%%%%%%%%%%%%%%%%%%%
This chapter define how to transform a parse tree into the corresponding AST
via recursively traversing the parse tree and applying a \emph{builder} function
for each non-terminal node.

(Some of the builders are relations due to non-determinism induced by naming global variables
for assignments whose left-hand-side variable is discarded ($\Tminus$).)

For each non-terminal $N \derives R_1 \;|\; \ldots R_k$, we define a builder function
$\textsf{build\_}N $ which takes a parse tree $\parsenode{N}$ and returns the corresponding
AST. The builder function is defined in terms of one inference rule per alternative $R_i$.
The input for the builder for an alternative $R = S_{1..m}$ is a parse node
$N(S_{1..m})$. To allow the builder to refer to the children nodes of $N$,
we use the notation $\namednode{n_i}{S_i}$, which names the child node $S_i$ as $n_i$.

\section{Example}
Consider the derivation
\[
\Nstmt \derives \Twhile \parsesep \Nexpr \parsesep \Tdo \parsesep \Nstmtlist \parsesep \Tend
\]
for while loops.

The parse node for a while statement has the form
\[
\Nstmt(\Twhile, \namednode{\ve}{\Nexpr}, \Tdo, \namednode{\vstmtlist}{\Nstmtlist}, \Tend)
\]
where $\ve$ names the node representing the condition of the loop and $\vstmtlist$ names
the list of statements that form the body of the loop.

To build the corresponding AST, we would employ the builder function $\buildstmt$, since
the non-terminal labelling the parse node is $\Nstmt$.

We would also employ the following rule:
\begin{mathpar}
\inferrule{
  \buildexpr(\ve) \astarrow \astversion{\ve}\\
  \buildstmtlist(\vstmtlist) \astarrow \astversion{\vstmtlist}
}{
{
\begin{array}{r}
  \buildstmt(\Nstmt(\Twhile, \namednode{\ve}{\Nexpr}, \Tdo, \namednode{\vstmtlist}{\Nstmtlist}, \Tend))
  \astarrow\\
  \SWhile(\astversion{\ve}, \None, \astversion{\vstmtlist})
\end{array}
}
}
\end{mathpar}
That is, we apply the $\buildexpr$ to transform the condition parse node into the corresponding AST node,
we apply $\buildstmtlist$ to transform the body of the list into the corresponding AST node,
and finally return the AST node with AST label $\SWhile$ with the two nodes as its children.

\subsection{Abbreviated Rule Notation}
Notice that there is only one instance of $\Nexpr$ and one instance of $\Nstmtlist$ in this production.
This is very common and we therefore use the following shorthand notation for such cases, as explained next.

In a non-terminal $N$ appears only once in the right-hand-side of a derivation,
we use the name $\texttt{N}$ to name the corresponding child parse node.
For example, $\namednode{\vexpr}{\Nexpr}$ and $\namednode{\vstmtlist}{\Nstmtlist}$.
In such cases, we always have the premise $\textsf{build\_}N(\texttt{N}) \astarrow \astversion{N}$
to obtain the corresponding AST node.
We therefore make this premise implicit, by dropping it entirely and using $\astof{N}$ to mean that
the parse node $N$ is named $\texttt{N}$, the premise $\textsf{build\_}N(\texttt{N}) \astarrow \astversion{N}$
is considered part of the rule and $\astversion{N}$ itself stands for $\astversion{N}$.

In our example, this results in the abbreviated rule notation
\begin{mathpar}
\inferrule{}{
{
\begin{array}{r}
  \buildstmt(\Nstmt(\Twhile, \punnode{\Nexpr}, \Tdo, \punnode{\Nstmtlist}, \Tend))
  \astarrow\\
  \SWhile(\astof{\vexpr}, \None, \astof{\vstmtlist})
  \end{array}
}
}
\end{mathpar}

\section{AST Builder Functions and Relations}
We define the following rules for transforming the various non-terminal parse nodes into
the corresponding AST nodes:
\begin{itemize}
  \item SyntaxRule.AST (see \secref{SyntaxRule.AST})
  \item SyntaxRule.GlobalDecl (see \secref{SyntaxRule.GlobalDecl})
  \item SyntaxRule.Subtype (see \secref{SyntaxRule.Subtype})
  \item SyntaxRule.Subtypeopt (see \secref{SyntaxRule.Subtypeopt})
  \item SyntaxRule.TypedIdentifier (see \secref{SyntaxRule.TypedIdentifier})
  \item SyntaxRule.OptTypedIdentifier (see \secref{SyntaxRule.OptTypedIdentifier})
  \item SyntaxRule.ReturnType (see \secref{SyntaxRule.ReturnType})
  \item SyntaxRule.ParamsOpt (see \secref{SyntaxRule.ParamsOpt})
  \item SyntaxRule.AccessArgs (see \secref{SyntaxRule.AccessArgs})
  \item SyntaxRule.FuncArgs (see \secref{SyntaxRule.FuncArgs})
  \item SyntaxRule.MaybeEmptyStmtList (see \secref{SyntaxRule.MaybeEmptyStmtList})
  \item SyntaxRule.FuncBody (see \secref{SyntaxRule.FuncBody})
  \item SyntaxRule.IgnoredOrIdentifier (see \secref{SyntaxRule.IgnoredOrIdentifier})
  \item SyntaxRule.LocalDeclKeyword (see \secref{SyntaxRule.LocalDeclKeyword})
  \item SyntaxRule.StorageKeyword (see \secref{SyntaxRule.StorageKeyword})
  \item SyntaxRule.Direction (see \secref{SyntaxRule.Direction})
  \item SyntaxRule.Alt (see \secref{SyntaxRule.Alt})
  \item SyntaxRule.OtherwiseOpt (see \secref{SyntaxRule.OtherwiseOpt})
  \item SyntaxRule.Catcher (see \secref{SyntaxRule.Catcher})
  \item SyntaxRule.Stmt (see \secref{SyntaxRule.Stmt})
  \item SyntaxRule.StmtList (see \secref{SyntaxRule.StmtList})
  \item SyntaxRule.SElse (see \secref{SyntaxRule.SElse})
  \item SyntaxRule.LExpr (see \secref{SyntaxRule.LExpr})
  \item SyntaxRule.LExprAtom (see \secref{SyntaxRule.LExprAtom})
  \item SyntaxRule.DeclItem (see \secref{SyntaxRule.DeclItem})
  \item SyntaxRule.UntypedDeclItem (see \secref{SyntaxRule.UntypedDeclItem})
  \item SyntaxRule.IntConstraints (see \secref{SyntaxRule.IntConstraints})
  \item SyntaxRule.IntConstraintsopt (see \secref{SyntaxRule.IntConstraintsopt})
  \item SyntaxRule.IntConstraint (see \secref{SyntaxRule.IntConstraint})
  \item SyntaxRule.ExprPattern (see \secref{SyntaxRule.ExprPattern})
  \item SyntaxRule.PatternSet (see \secref{SyntaxRule.PatternSet})
  \item SyntaxRule.PatternList (see \secref{SyntaxRule.PatternList})
  \item SyntaxRule.Pattern (see \secref{SyntaxRule.Pattern})
  \item SyntaxRule.Fields (see \secref{SyntaxRule.Fields})
  \item SyntaxRule.FieldsOpt (see \secref{SyntaxRule.FieldsOpt})
  \item SyntaxRule.NSlices (see \secref{SyntaxRule.NSlices})
  \item SyntaxRule.Slices (see \secref{SyntaxRule.Slices})
  \item SyntaxRule.Slice (see \secref{SyntaxRule.Slice})
  \item SyntaxRule.Bitfields (see \secref{SyntaxRule.Bitfields})
  \item SyntaxRule.Bitfield (see \secref{SyntaxRule.Bitfield})
  \item SyntaxRule.Ty (see \secref{SyntaxRule.Ty})
  \item SyntaxRule.TyDecl (see \secref{SyntaxRule.TyDecl})
  \item SyntaxRule.FieldAssign (see \secref{SyntaxRule.FieldAssign})
  \item SyntaxRule.EElse (see \secref{SyntaxRule.EElse})
  \item SyntaxRule.Expr (see \secref{SyntaxRule.Expr})
  \item SyntaxRule.Value (see \secref{SyntaxRule.Value})
  \item SyntaxRule.Unop (see \secref{SyntaxRule.Unop})
  \item SyntaxRule.Binop (see \secref{SyntaxRule.Binop})
\end{itemize}

We also define the following helper functions:
\begin{itemize}
\item SyntaxRule.StmtFromList (see \secref{SyntaxRule.StmtFromList})
\item SyntaxRule.SequenceStmts (see \secref{SyntaxRule.SequenceStmts})
\end{itemize}

\section{SyntaxRule.AST \label{sec:SyntaxRule.AST}}
\hypertarget{build-ast}{}
The relation
\[
  \buildast : \overname{\parsenode{\Nast}}{\vparsednode} \;\aslrel\; \overname{\specification}{\vastnode}
\]
transforms an $\Nast$ node $\vparsednode$ into an AST specification node $\vastnode$.

\isempty{\subsection{Prose}}

\subsection{Formally}
\begin{mathpar}
\inferrule[ast]{
  \buildlist[\builddecl](\vdecls) \astarrow \vadecls
}{
  \buildast(\overname{\Nast(\namednode{\vdecls}{\maybeemptylist{\Ndecl}})}{\vparsednode}) \astarrow \overname{\vadecls}{\vastnode}
}
\end{mathpar}

\section{SyntaxRule.GlobalDecl \label{sec:SyntaxRule.GlobalDecl}}
\hypertarget{build-decl}{}
The relation
\[
  \builddecl : \overname{\parsenode{\Ndecl}}{\vparsednode} \;\aslrel\; \overname{\decl}{\vastnode}
\]
transforms a parse node $\vparsednode$ into an AST node $\vastnode$.

\isempty{\subsection{Prose}}

\subsection{Formally}
\hypertarget{build-funcdecl}{}
\begin{mathpar}
\inferrule[func\_decl]{}{
    \builddecl(
      \overname{\Ndecl(
      \Tfunc, \Tidentifier(\name), \punnode{\Nparamsopt}, \punnode{\Nfuncargs}, \punnode{\Nreturntype}, \punnode{\Nfuncbody}
    )}{\vparsednode})
  \astarrow \\
  {
    \overname{
  \DFunc\left(\left\{
    \begin{array}{rcl}
            \funcname &:& \name,\\
            \funcparameters &:& \astof{\vparamsopt},\\
            \funcargs &:& \astof{\vfuncargs},\\
            \funcbody &:& \SBASL(\astof{\vfuncbody}),\\
            \funcreturntype &:& \langle \astof{\vreturntype} \rangle,\\
            \funcsubprogramtype &:& \STFunction
    \end{array}
  \right\}\right)}{\vastnode}
  }
}
\end{mathpar}

\hypertarget{build-procedureecl}{}
\begin{mathpar}
\inferrule[procedure\_decl]{}{
  \builddecl(
    \overname{\Ndecl(\Tfunc, \Tidentifier(\name), \punnode{\Nparamsopt}, \punnode{\Nfuncargs}, \punnode{\Nfuncbody})}{\vparsednode}
    )
  \astarrow \\
  {
    \overname{
  \DFunc\left(\left\{
    \begin{array}{rcl}
            \funcname &:& \name,\\
            \funcparameters &:& \astof{\vparamsopt},\\
            \funcargs &:& \astof{\vfuncargs},\\
            \funcbody &:& \SBASL(\astof{\vfuncbody}),\\
            \funcreturntype &:& \None,\\
            \funcsubprogramtype &:& \STProcedure
    \end{array}
  \right\}\right)
    }{\vastnode}
  }
}
\end{mathpar}

\hypertarget{build-getter}{}
\begin{mathpar}
\inferrule[getter]{}{
  {
    \begin{array}{r}
  \builddecl\left(\overname{\Ndecl\left(
    \begin{array}{l}
      \Tgetter, \Tidentifier(\name), \punnode{\Nparamsopt}, \punnode{\Naccessargs}, \\
      \wrappedline\ \punnode{\Nreturntype}, \punnode{\Nfuncbody}
    \end{array}
      \right)}{\vparsednode}\right)
  \astarrow \\
  \overname{
  \DFunc\left(\left\{
    \begin{array}{rcl}
            \funcname &:& \name,\\
            \funcparameters &:& \astof{\vparamsopt},\\
            \funcargs &:& \astof{\vaccessargs},\\
            \funcbody &:& \SBASL(\astof{\vfuncbody}),\\
            \funcreturntype &:& \langle \astof{\vreturntype}\rangle,\\
            \funcsubprogramtype &:& \STGetter
    \end{array}
  \right\}\right)
  }{\vastnode}
  \end{array}
  }
}
\end{mathpar}

\hypertarget{build-noarggetter}{}
\begin{mathpar}
\inferrule[no\_arg\_getter]{}{
  \builddecl(\overname{\Ndecl(\Tgetter, \Tidentifier(\name), \punnode{\Nreturntype}, \punnode{\Nfuncbody})}{\vparsednode})
  \astarrow \\
  {
    \overname{
  \DFunc\left(\left\{
    \begin{array}{rcl}
            \funcname &:& \name,\\
            \funcparameters &:& \emptylist,\\
            \funcargs &:& \emptylist,\\
            \funcbody &:& \SBASL(\astof{\vfuncbody}),\\
            \funcreturntype &:& \langle \astof{\vreturntype}\rangle,\\
            \funcsubprogramtype &:& \STEmptyGetter
    \end{array}
  \right\}\right)
    }{\vastnode}
  }
}
\end{mathpar}

\hypertarget{build-setter}{}
\begin{mathpar}
\inferrule[setter]{}{
  {
      \builddecl\left(\overname{\Ndecl\left(
        \begin{array}{r}
          \Tsetter, \Tidentifier(\name), \punnode{\Nparamsopt}, \punnode{\Naccessargs}, \Teq, \\
   \wrappedline\ \namednode{\vv}{\Ntypedidentifier}, \punnode{\Nfuncbody}
        \end{array}
   \right)}{\vparsednode}\right)
  } \astarrow
  \\
  {
    \overname{
  \DFunc\left(\left\{
    \begin{array}{rcl}
            \funcname &:& \name,\\
            \funcparameters &:& \astof{\vparamsopt},\\
            \funcargs &:& [\vv] \concat \astof{\vaccessargs},\\
            \funcbody &:& \SBASL(\astof{\vfuncbody}),\\
            \funcreturntype &:& \None,\\
            \funcsubprogramtype &:& \STSetter
    \end{array}
  \right\}\right)
    }{\vastnode}
  }
}
\end{mathpar}

\hypertarget{build-noargsetter}{}
\begin{mathpar}
\inferrule[no\_arg\_setter]{}{
  \builddecl(\overname{\Ndecl(\Tsetter, \Tidentifier(\name), \Teq, \namednode{\vv}{\Ntypedidentifier}, \punnode{\Nfuncbody})}{\vparsednode})
  \astarrow \\
  {
    \overname{
  \DFunc\left(\left\{
    \begin{array}{rcl}
            \funcname &:& \name,\\
            \funcparameters &:& \emptylist,\\
            \funcargs &:& [\vv],\\
            \funcbody &:& \SBASL(\astof{\vfuncbody}),\\
            \funcreturntype &:& \None,\\
            \funcsubprogramtype &:& \STEmptySetter
    \end{array}
  \right\}\right)
    }{\vastnode}
  }
}
\end{mathpar}

\hypertarget{build-typedecl}{}
\begin{mathpar}
\inferrule[type\_decl]{}
{
  {
    \begin{array}{c}
      \builddecl(\overname{\Ndecl(\Ttype, \Tidentifier(\vx), \Tof, \punnode{\Ntydecl}, \Nsubtypeopt, \Tsemicolon)}{\vparsednode})
  \astarrow \\
  \overname{\DTypeDecl(\vx, \astof{\vt}, \astof{\vsubtypeopt})}{\vastnode}
    \end{array}
  }
}
\end{mathpar}

\hypertarget{build-subtypedecl}{}
\begin{mathpar}
\inferrule[subtype\_decl]{
  \buildsubtype(\vsubtype) \astarrow \vs\\
  \vs \eqname (\name, \vfields)
}{
  {
    \begin{array}{c}
      \builddecl(\overname{\Ndecl(\Ttype, \Tidentifier(\vx), \Tof, \punnode{\Nsubtype}, \Tsemicolon)}{\vparsednode})
  \astarrow \\
  \overname{\DTypeDecl(\vx, \TNamed(\name), \langle(\name, \vfields)\rangle)}{\vastnode}
    \end{array}
  }
}
\end{mathpar}

\hypertarget{build-globalstorage}{}
\begin{mathpar}
\inferrule[global\_storage]{
  \buildstoragekeyword(\keyword) \astarrow \astof{\keyword}\\
  \buildoption[\buildasty](\tty) \astarrow \ttyp\\
  \buildexpr(\vinitialvalue) \typearrow \astof{\vinitialvalue}
}
{
  {
      \builddecl\left(\overname{\Ndecl\left(
      \begin{array}{r}
      \namednode{\vkeyword}{\Nstoragekeyword}, \namednode{\name}{\Nignoredoridentifier},  \\
  \wrappedline\ \namednode{\tty}{\option{\Nasty}}, \Teq, \namednode{\vinitialvalue}{\Nexpr}, \Tsemicolon
      \end{array}
  \right)}{\vparsednode}\right)
  } \astarrow \\
  {
    \overname{
  \DGlobalStorage\left(\left\{
    \begin{array}{rcl}
    \GDkeyword &:& \astof{\vkeyword},\\
    \GDname &:& \astof{\name},\\
    \GDty &:& \ttyp,\\
    \GDinitialvalue &:& \astof{\vinitialvalue}
  \end{array}
  \right\}\right)
  }{\vastnode}
  }
}
\end{mathpar}

\hypertarget{build-globaluninitvar}{}
\begin{mathpar}
\inferrule[global\_uninit\_var]{
  \buildignoredoridentifier(\cname) \astarrow \name
}
{
  {
    \begin{array}{r}
      \builddecl(\overname{\Ndecl(\Tvar, \namednode{\cname}{\Nignoredoridentifier}, \Nasty, \Tsemicolon)}{\vparsednode}) \astarrow
    \end{array}
  } \\
  \overname{\DGlobalStorage(\{\GDkeyword: \GDKVar, \GDname: \name, \GDty: \langle\astof{\Nasty}\rangle, \GDinitialvalue: \None\})}{\vastnode}
}
\end{mathpar}

\hypertarget{build-globalpragma}{}
\begin{mathpar}
\inferrule[global\_pragma]{}
{
  \builddecl(\overname{\Ndecl(\Tpragma, \Tidentifier(\vx), \Clist{\Nexpr}, \Tsemicolon)}{\vparsednode}) \astarrow \overname{\tododefine{pragma\_node}}{\vastnode}
}
\end{mathpar}

\section{SyntaxRule.Subtype \label{sec:SyntaxRule.Subtype}}
\hypertarget{build-subtype}{}
The function
\[
  \buildsubtype(\overname{\parsenode{\Nsubtype}}{\vparsednode}) \aslto \overname{(\identifier \times (\identifier\times \ty)^*)}{\vastnode}
\]
transforms a parse node $\vparsednode$ into an AST node $\vastnode$.

\isempty{\subsection{Prose}}
\subsection{Formally}
\hypertarget{build-subtype}{}
\begin{mathpar}
\inferrule[with\_fields]{}{
  {
    \begin{array}{r}
  \buildsubtype(\overname{\Nsubtype(
    \Tsubtypes, \Tidentifier(\id), \Twith, \punnode{\Nfields}
    )}{\vparsednode})
  \astarrow \\
  \overname{(\id, \astof{\vfields})}{\vastnode}
  \end{array}
  }
}
\end{mathpar}

\begin{mathpar}
  \inferrule[no\_fields]{}{
  \buildsubtype(\overname{\Nsubtype(
    \Tsubtypes, \Tidentifier(\id))}{\vparsednode})
  \astarrow
  \overname{(\id, \emptylist)}{\vastnode}
}
\end{mathpar}

\section{SyntaxRule.Subtypeopt \label{sec:SyntaxRule.Subtypeopt}}
\hypertarget{build-subtypeopt}{}
The function
\[
   \buildsubtypeopt(\overname{\parsenode{\Nsubtypeopt}}{\vparsednode}) \aslto
    \overname{\langle(\identifier \times \langle (\identifier\times \ty)^* \rangle)\rangle}{\vastnode}
\]
transforms a parse node $\vparsednode$ into an AST node $\vastnode$.

\isempty{\subsection{Prose}}
\subsection{Formally}
\begin{mathpar}
\inferrule[subtype\_opt]{
  \buildoption[\Nsubtype](\vsubtypeopt) \astarrow \vastnode
}{
  \buildsubtypeopt(\overname{\Nsubtypeopt(\namednode{\vsubtypeopt}{\option{\Nsubtype}})}{\vparsednode}) \astarrow \vastnode
}
\end{mathpar}

\section{SyntaxRule.TypedIdentifier \label{sec:SyntaxRule.TypedIdentifier}}
\hypertarget{build-typedidentifier}{}
The function
\[
\buildtypedidentifier(\overname{\parsenode{\Ntypedidentifier}}{\vparsednode}) \aslto \overname{(\identifier \times \ty)}{\vastnode}
\]
transforms a parse node $\vparsednode$ into an AST node $\vastnode$.

\begin{mathpar}
\inferrule{}{
  \buildtypedidentifier(\overname{\Ntypedidentifier(\Tidentifier(\id), \punnode{\Nasty})}{\vparsednode}) \astarrow \overname{(\id,\astof{\vasty})}{\vastnode}
}
\end{mathpar}

\section{SyntaxRule.OptTypedIdentifier \label{sec:SyntaxRule.OptTypedIdentifier}}
\hypertarget{build-opttypedidentifier}{}
The function
\[
\buildopttypedidentifier(\overname{\parsenode{\Nopttypedidentifier}}{\vparsednode}) \aslto \overname{(\identifier \times \langle\ty\rangle)}{\vastnode}
\]
transforms a parse node $\vparsednode$ into an AST node $\vastnode$.

\isempty{\subsection{Prose}}
\subsection{Formally}
\begin{mathpar}
\inferrule{
  \buildoption[\Nasty](\vastyopt) \astarrow \astversion{\vastyopt}
}{
  {
  \begin{array}{r}
  \buildopttypedidentifier(\overname{\Ntypedidentifier(\Tidentifier(\id), \namednode{\vastyopt}{\option{\Nasty}})}{\vparsednode}) \astarrow \\
  \overname{(\id, \astversion{\vastyopt})}{\vastnode}
  \end{array}
  }
}
\end{mathpar}

\section{SyntaxRule.ReturnType \label{sec:SyntaxRule.ReturnType}}
\hypertarget{build-returntype}{}
The function
\[
\buildreturntype(\overname{\parsenode{\Nreturntype}}{\vparsednode}) \aslto \overname{\ty}{\vastnode}
\]
transforms a parse node $\vparsednode$ into an AST node $\vastnode$.

\isempty{\subsection{Prose}}
\subsection{Formally}
\begin{mathpar}
\inferrule{}{
  \buildreturntype(\overname{\Nreturntype(\Tarrow, \punnode{\Nty})}{\vparsednode}) \astarrow
  \overname{\astof{\tty}}{\vastnode}
}
\end{mathpar}

\section{SyntaxRule.ParamsOpt \label{sec:SyntaxRule.ParamsOpt}}
\hypertarget{build-paramsopt}{}
The function
\[
\buildparamsopt(\overname{\parsenode{\Nparamsopt}}{\vparsednode}) \aslto
  \overname{(\identifier\times\langle\ty\rangle)^*}{\vastnode}
\]
transforms a parse node $\vparsednode$ into an AST node $\vastnode$.

\isempty{\subsection{Prose}}
\subsection{Formally}
\begin{mathpar}
\inferrule[empty]{}{
  \buildparamsopt(\overname{\Nparamsopt(\epsilonnode)}{\vparsednode}) \astarrow
  \overname{\emptylist}{\vastnode}
}
\end{mathpar}

\begin{mathpar}
\inferrule[non\_empty]{
  \buildclist[\Nopttypedidentifier](\ids) \astarrow \astversion{\ids}
}{
  \buildparamsopt(\overname{\Nparamsopt(\Tlbrace, \namednode{\ids}{\Clist{\Nopttypedidentifier}}, \Trbrace)}{\vparsednode}) \astarrow
  \overname{\astversion{\ids}}{\vastnode}
}
\end{mathpar}

\section{SyntaxRule.AccessArgs \label{sec:SyntaxRule.AccessArgs}}
\hypertarget{build-accessargs}{}
The function
\[
\buildaccessargs(\overname{\parsenode{\Naccessargs}}{\vparsednode}) \aslto
  \overname{(\identifier\times\ty)^*}{\vastnode}
\]
transforms a parse node $\vparsednode$ into an AST node $\vastnode$.

\isempty{\subsection{Prose}}
\subsection{Formally}
\begin{mathpar}
\inferrule{
  \buildclist[\Ntypedidentifier](\ids) \astarrow \astversion{\ids}
}{
  \buildaccessargs(\overname{\Naccessargs(\Tlbracket, \namednode{\ids}{\Clist{\Ntypedidentifier}}, \Trbracket)}{\vparsednode}) \astarrow
  \overname{\astversion{\ids}}{\vastnode}
}
\end{mathpar}

\section{SyntaxRule.FuncArgs \label{sec:SyntaxRule.FuncArgs}}
\hypertarget{build-funcargs}{}
The function
\[
\buildfuncargs(\overname{\parsenode{\Nfuncargs}}{\vparsednode}) \aslto
  \overname{(\identifier\times\ty)^*}{\vastnode}
\]
transforms a parse node $\vparsednode$ into an AST node $\vastnode$.

\isempty{\subsection{Prose}}
\subsection{Formally}
\begin{mathpar}
\inferrule{
  \buildclist[\Ntypedidentifier](\ids) \astarrow \astversion{\ids}
}{
  \buildfuncargs(\overname{\Nfuncargs(\Tlpar, \namednode{\ids}{\Clist{\Ntypedidentifier}}, \Trpar)}{\vparsednode}) \astarrow
  \overname{\astversion{\ids}}{\vastnode}
}
\end{mathpar}

\section{SyntaxRule.MaybeEmptyStmtList \label{sec:SyntaxRule.MaybeEmptyStmtList}}
\hypertarget{build-maybeemptystmtlist}{}
The function
\[
\buildmaybeemptystmtlist(\overname{\parsenode{\Nmaybeemptystmtlist}}{\vparsednode}) \aslto
  \overname{\stmt}{\vastnode}
\]
transforms a parse node $\vparsednode$ into an AST node $\vastnode$.

\isempty{\subsection{Prose}}
\subsection{Formally}
\begin{mathpar}
\inferrule[empty]{}{
  \buildfuncbody(\overname{\Nmaybeemptystmtlist(\epsilonnode)}{\vparsednode}) \astarrow
  \overname{\SPass}{\vastnode}
}
\end{mathpar}

\begin{mathpar}
\inferrule[non\_empty]{}{
  \buildfuncbody(\overname{\Nmaybeemptystmtlist(\Nstmtlist)}{\vparsednode}) \astarrow
  \overname{\astof{\vstmtlist}}{\vastnode}
}
\end{mathpar}

\section{SyntaxRule.FuncBody \label{sec:SyntaxRule.FuncBody}}
\hypertarget{build-funcbody}{}
The function
\[
\buildfuncargs(\overname{\parsenode{\Nfuncbody}}{\vparsednode}) \aslto
  \overname{\stmt}{\vastnode}
\]
transforms a parse node $\vparsednode$ into an AST node $\vastnode$.

\isempty{\subsection{Prose}}
\subsection{Formally}
\begin{mathpar}
\inferrule{}{
  {
  \begin{array}{r}
  \buildfuncbody(\overname{\Nfuncbody(\Tbegin, \namednode{\vstmts}{\Nmaybeemptystmtlist}, \Tend)}{\vparsednode}) \astarrow \\
  \overname{\astof{\vmaybeemptystmtlist}}{\vastnode}
  \end{array}
  }
}
\end{mathpar}

\section{SyntaxRule.IgnoredOrIdentifier \label{sec:SyntaxRule.IgnoredOrIdentifier}}
\hypertarget{build-ignoredoridentifier}{}
The relation
\[
\buildfuncargs(\overname{\parsenode{\Nignoredoridentifier}}{\vparsednode}) \;\aslrel\;
  \overname{\identifier}{\vastnode}
\]
transforms a parse node $\vparsednode$ into an AST node $\vastnode$.

\isempty{\subsection{Prose}}
\subsection{Formally}
\begin{mathpar}
\inferrule[discard]{
  \id \in \identifier \text{ is fresh}
}{
  \buildignoredoridentifier(\overname{\Nignoredoridentifier(\Tminus)}{\vparsednode}) \astarrow
  \overname{\id}{\vastnode}
}
\end{mathpar}

\begin{mathpar}
\inferrule[id]{}{
  \buildignoredoridentifier(\overname{\Nignoredoridentifier(\Tidentifier(\id))}{\vparsednode}) \astarrow
  \overname{\id}{\vastnode}
}
\end{mathpar}

\section{SyntaxRule.LocalDeclKeyword \label{sec:SyntaxRule.LocalDeclKeyword}}
\hypertarget{build-localdeclkeyword}{}
The function
\[
\buildlocaldeclkeyword(\overname{\parsenode{\Nlocaldeclkeyword}}{\vparsednode}) \;\aslto\;
  \overname{\localdeclkeyword}{\vastnode}
\]
transforms a parse node $\vparsednode$ into an AST node $\vastnode$.

\isempty{\subsection{Prose}}
\subsection{Formally}
\begin{mathpar}
\inferrule[let]{}{
  \buildlocaldeclkeyword(\overname{\Nlocaldeclkeyword(\Tlet)}{\vparsednode}) \astarrow \overname{\LDKLet}{\vastnode}
}
\end{mathpar}

\begin{mathpar}
\inferrule[constant]{}{
  \buildlocaldeclkeyword(\overname{\Nlocaldeclkeyword(\Tconstant)}{\vparsednode}) \astarrow \overname{\LDKConstant}{\vastnode}
}
\end{mathpar}

\section{SyntaxRule.StorageKeyword \label{sec:SyntaxRule.StorageKeyword}}
\hypertarget{build-storagekeyword}{}
The function
\[
\buildstoragekeyword(\overname{\parsenode{\Nstoragekeyword}}{\vparsednode}) \;\aslto\;
  \overname{\globaldeclkeyword}{\vastnode}
\]
transforms a parse node $\vparsednode$ into an AST node $\vastnode$.

\isempty{\subsection{Prose}}
\subsection{Formally}
\begin{mathpar}
\inferrule[let]{}{
  \buildstoragekeyword(\overname{\Nstoragekeyword(\Tlet)}{\vparsednode}) \astarrow \overname{\GDKLet}{\vastnode}
}
\end{mathpar}

\begin{mathpar}
\inferrule[constant]{}{
  \buildstoragekeyword(\overname{\Nstoragekeyword(\Tconstant)}{\vparsednode}) \astarrow \overname{\GDKConstant}{\vastnode}
}
\end{mathpar}

\begin{mathpar}
\inferrule[var]{}{
  \buildstoragekeyword(\overname{\Nstoragekeyword(\Tvar)}{\vparsednode}) \astarrow \overname{\GDKVar}{\vastnode}
}
\end{mathpar}

\begin{mathpar}
\inferrule[config]{}{
  \buildstoragekeyword(\overname{\Nstoragekeyword(\Tconfig)}{\vparsednode}) \astarrow \overname{\GDKConfig}{\vastnode}
}
\end{mathpar}

\section{SyntaxRule.Direction \label{sec:SyntaxRule.Direction}}
\hypertarget{build-direction}{}
The function
\[
\builddirection(\overname{\parsenode{\Ndirection}}{\vparsednode}) \;\aslto\; \overname{\fordirection}{\vastnode}
\]
transforms a parse node $\vparsednode$ into an AST node $\vastnode$.

\isempty{\subsection{Prose}}
\subsection{Formally}
\begin{mathpar}
\inferrule[to]{}{
  \builddirection(\overname{\Ndirection(\Tto)}{\vparsednode}) \astarrow \overname{\UP}{\vastnode}
}
\end{mathpar}

\begin{mathpar}
\inferrule[downto]{}{
  \builddirection(\overname{\Ndirection(\Tdownto)}{\vparsednode}) \astarrow \overname{\DOWN}{\vastnode}
}
\end{mathpar}

\section{SyntaxRule.Alt \label{sec:SyntaxRule.Alt}}
\hypertarget{build-alt}{}
The function
\[
\buildalt(\overname{\parsenode{\Nalt}}{\vparsednode}) \;\aslto\; \overname{\casealt}{\vastnode}
\]
transforms a parse node $\vparsednode$ into an AST node $\vastnode$.

\isempty{\subsection{Prose}}
\subsection{Formally}
\begin{mathpar}
\inferrule[when]{
  \buildoption[\buildexpr](\vwhereopt) \astarrow \vwhereast
}{
  {
    \begin{array}{r}
  \buildalt\left(\overname{\Nalt\left(
    \begin{array}{l}
    \Twhen, \punnode{\Npatternlist}, \\
    \wrappedline\ \namednode{\vwhereopt}{\option{\Twhere, \Nexpr}}, \Tarrow, \\
    \wrappedline\ \namednode{\vstmts}{\Nstmtlist}
    \end{array}
    \right)}{\vparsednode}\right)
  \astarrow \\
  \overname{\casealt(\CasePattern: \astof{\vpatternlist}, \CaseWhere: \vwhereast, \CaseStmt: \astof{\vstmtlist})}{\vastnode}
    \end{array}
  }
}
\end{mathpar}

\begin{mathpar}
\inferrule[otherwise]{}{
  {
    \begin{array}{r}
  \buildalt(\overname{\Nalt(\Totherwise, \namednode{\vstmts}{\Nstmtlist})}{\vparsednode})
  \astarrow \\
  \overname{\casealt(\CasePattern: \PatternAll, \CaseWhere: \None, \CaseStmt: \astof{\vstmtlist})}{\vastnode}
    \end{array}
  }
}
\end{mathpar}

\section{SyntaxRule.OtherwiseOpt \label{sec:SyntaxRule.OtherwiseOpt}}
\hypertarget{build-otherwiseopt}{}
The function
\[
\buildotherwiseopt(\overname{\parsenode{\Notherwiseopt}}{\vparsednode}) \;\aslto\; \overname{\stmt?}{\vastnode}
\]
transforms a parse node $\vparsednode$ into an AST node $\vastnode$.

\isempty{\subsection{Prose}}
\subsection{Formally}
\begin{mathpar}
\inferrule{
  \buildoption[\buildstmtlist] (\vv)\astarrow \vastnode
}{
  {
  \begin{array}{r}
  \buildotherwiseopt(\overname{\Notherwiseopt(\namednode{\vv}{\option{\Totherwise, \Tarrow, \Nstmtlist}})}{\vparsednode})
  \astarrow \\
  \vastnode
  \end{array}
  }
}
\end{mathpar}

\section{SyntaxRule.Catcher \label{sec:SyntaxRule.Catcher}}
\hypertarget{build-catcher}{}
The function
\[
\buildcatcher(\overname{\parsenode{\Ncatcher}}{\vparsednode}) \;\aslto\; \overname{\catcher}{\vastnode}
\]
transforms a parse node $\vparsednode$ into an AST node $\vastnode$.

\isempty{\subsection{Prose}}
\subsection{Formally}
\begin{mathpar}
\inferrule[named]{}{
  {
  \begin{array}{r}
  \buildcatcher(\overname{\Ncatcher(\Twhen, \Tidentifier(\id), \Tcolon, \Nty, \Tarrow, \Nstmtlist)}{\vparsednode})
  \astarrow \\
  \overname{(\langle\id\rangle, \astof{\tty}, \astof{\vstmtlist})}{\vastnode}
  \end{array}
  }
}
\end{mathpar}

\begin{mathpar}
\inferrule[unnamed]{}{
  {
  \begin{array}{r}
  \buildcatcher(\overname{\Ncatcher(\Twhen, \Nty, \Tarrow, \Nstmtlist)}{\vparsednode})
  \astarrow \\
  \overname{(\None, \astof{\tty}, \astof{\vstmtlist})}{\vastnode}
  \end{array}
  }
}
\end{mathpar}

\section{SyntaxRule.Stmt \label{sec:SyntaxRule.Stmt}}
\hypertarget{build-stmt}{}
The function
\[
\buildstmt(\overname{\parsenode{\Nstmt}}{\vparsednode}) \;\aslto\; \overname{\stmt}{\vastnode}
\]
transforms a parse node $\vparsednode$ into an AST node $\vastnode$.

\isempty{\subsection{Prose}}
\subsection{Formally}

\begin{mathpar}
\inferrule[if]{}{
  {
    \begin{array}{r}
  \buildstmt(\overname{\Nstmt(\Tif, \punnode{\Nexpr}, \Tthen, \punnode{\Nstmtlist}, \punnode{\Nselse}, \Tend)}{\vparsednode})
  \astarrow \\
  \overname{\SCond(\astof{\vexpr}, \astof{\vstmtlist}, \astof{\velse})}{\vastnode}
    \end{array}
  }
}
\end{mathpar}

\begin{mathpar}
\inferrule[case]{
  \buildlist[\Nalt](\valtlist) \astarrow \valtlistast
}{
  {
    \begin{array}{r}
  \buildstmt(\overname{\Nstmt(\Tcase, \punnode{\Nexpr}, \Tof, \namednode{\valtlist}{\maybeemptylist{\Nalt}}, \Tend)}{\vparsednode})
  \astarrow \\
  \overname{\SCase(\astof{\vexpr}, \valtlistast)}{\vastnode}
    \end{array}
  }
}
\end{mathpar}

\begin{mathpar}
\inferrule[while]{}{
  {
    \begin{array}{r}
  \buildstmt(\overname{\Nstmt(\Twhile, \punnode{\Nexpr}, \Tdo, \punnode{\Nstmtlist}, \Tend)}{\vparsednode})
  \astarrow\\
  \overname{\SWhile(\astof{\vexpr}, \None, \astof{\vstmtlist})}{\vastnode}
\end{array}
}
}
\end{mathpar}

\begin{mathpar}
\inferrule[looplimit\_while]{
  \buildexpr(\vlimitexpr) \astarrow \astversion{\vlimitexpr}
}{
  {
    \begin{array}{r}
  \buildstmt\left(\overname{\Nstmt\left(
    \begin{array}{r}
    \Tlooplimit, \Tlpar, \namednode{\vlimitexpr}{\Nexpr}, \Trpar, \Twhile,  \\
    \wrappedline\ \punnode{\Nexpr}, \Tdo, \punnode{\Nstmtlist}, \Tend
    \end{array}
    \right)}{\vparsednode}\right)
  \astarrow\\
  \overname{\SWhile(\astof{\vexpr}, \langle\astversion{\vlimitexpr}\rangle, \astof{\vstmtlist})}{\vastnode}
\end{array}
}
}
\end{mathpar}

\begin{mathpar}
\inferrule[for]{
  \buildexpr(\vfromexpr) \astarrow \astversion{\vfromexpr}\\
  \buildexpr(\vtoexpr) \astarrow \astversion{\vtoexpr}\\
}{
  {
    \begin{array}{r}
  \buildstmt\left(\overname{\Nstmt\left(
    \begin{array}{l}
    \Tfor, \Tidentifier(\id), \Teq, \namednode{\vfromexpr}{\Nexpr}, \Ndirection, \\
    \wrappedline\ \namednode{\vtoexpr}{\Nexpr}, \Tdo, \punnode{\Nstmtlist}, \Tend
    \end{array}
                       \right)}{\vparsednode}\right)
  \astarrow \\
  \overname{\SFor(\id, \astversion{\vfromexpr}, \astof{\vdirection}, \astversion{\vtoexpr}, \astof{\vstmtlist})}{\vastnode}
\end{array}
}
}
\end{mathpar}

\begin{mathpar}
\inferrule[try]{
  \buildlist[\Ncatcher] \astarrow \astversion{\vcatcherlist}
}{
  {
    \begin{array}{r}
  \buildstmt\left(\overname{\Nstmt\left(
    \begin{array}{r}
    \Ttry, \Nstmtlist, \Tcatch,  \\
    \wrappedline\ \namednode{\vcatcherlist}{\nonemptylist{\Ncatcher}}, \\
    \wrappedline\ \Notherwiseopt, \Tend
    \end{array}
    \right)}{\vparsednode}\right)
  \astarrow \\
  \overname{\STry(\astof{\vstmtlist}, \astversion{\vcatcherlist}, \astof{\votherwiseopt})}{\vastnode}
\end{array}
}
}
\end{mathpar}

\begin{mathpar}
\inferrule[pass]{}{
  \buildstmt(\overname{\Nstmt(\Tpass, \Tsemicolon)}{\vparsednode})
  \astarrow
  \overname{\SPass}{\vastnode}
}
\end{mathpar}

\begin{mathpar}
\inferrule[return]{
  \buildoption[\Nexpr](\vexpr) \astarrow \astversion{\vexpr}
}{
  \buildstmt(\overname{\Nstmt(\Treturn, \namednode{\vexpr}{\option{\Nexpr}}, \Tsemicolon)}{\vparsednode})
  \astarrow
  \overname{\SReturn(\astversion{\vexpr})}{\vastnode}
}
\end{mathpar}

\begin{mathpar}
\inferrule[call]{
  \buildplist[\Nexpr](\vargs) \astarrow \astversion{\vargs}
}{
  \buildstmt(\overname{\Nstmt(\Tidentifier(\vx), \namednode{\vargs}{\Plist{\Nexpr}}, \Tsemicolon)}{\vparsednode})
  \astarrow
  \overname{\SCall(\vx, \astversion{\vargs})}{\vastnode}
}
\end{mathpar}

\begin{mathpar}
\inferrule[assert]{}{
  \buildstmt(\overname{\Nstmt(\Tassert, \Nexpr, \Tsemicolon)}{\vparsednode})
  \astarrow
  \overname{\SAssert(\astof{\vexpr})}{\vastnode}
}
\end{mathpar}

\begin{mathpar}
\inferrule[decl]{}{
  {
  \begin{array}{r}
  \buildstmt(\overname{\Nstmt(\Nlocaldeclkeyword, \Ndeclitem, \Teq, \punnode{\Nexpr}, \Tsemicolon)}{\vparsednode})
  \astarrow\\
  \overname{\SDecl(\astof{\vlocaldeclkeyword}, \astof{\vdeclitem}, \langle\astof{\vexpr}\rangle)}{\vastnode}
  \end{array}
  }
}
\end{mathpar}

\begin{mathpar}
\inferrule[assignment]{}{
  \buildstmt(\overname{\Nstmt(\punnode{\Nlexpr}, \Teq, \punnode{\Nexpr}, \Tsemicolon)}{\vparsednode})
  \astarrow
  \overname{\SAssign(\astof{\vlexpr}, \astof{\vexpr})}{\vastnode}
}
\end{mathpar}

\begin{mathpar}
\inferrule[var\_decl]{
  \buildoption[\buildexpr](\ve) \astarrow \astversion{\ve}
}{
  {
    \begin{array}{r}
  \buildstmt(\overname{\Nstmt(\Tvar, \Ndeclitem, \namednode{\ve}{\option{\Teq, \Nexpr}}, \Tsemicolon)}{\vparsednode})
  \astarrow \\
  \overname{\SDecl(\LDKVar, \astof{\vdeclitem}, \astversion{\ve})}{\vastnode}
\end{array}
}
}
\end{mathpar}

\begin{mathpar}
\inferrule[multi\_var\_decl]{
  \buildclist[\buildidentity](\vids) \astarrow \astversion{\vids}\\
  \vstmts \eqdef [\vx\in\astversion{\vids}: \SDecl(\LDKVar, \vx, \astof{\tty})]\\
  \stmtfromlist(\vstmts) \astarrow \vastnode
}{
  \buildstmt(\overname{\Nstmt(\Tvar, \namednode{\vids}{\Clisttwo{\Tidentifier}}, \Tcolon, \punnode{\Nty}, \Tsemicolon)}{\vparsednode})
  \astarrow
  \vastnode
}
\end{mathpar}

\begin{mathpar}
\inferrule[print]{
  \buildplist[\Nexpr](\vargs) \astarrow \astversion{\vargs}
}{
  \buildstmt(\overname{\Nstmt(\Tprint, \namednode{\vargs}{\Plist{\Nexpr}}, \Tsemicolon)}{\vparsednode})
  \astarrow
  \overname{\SPrint(\astversion{\vargs})}{\vastnode}
}
\end{mathpar}

\begin{mathpar}
\inferrule[repeat]{}{
  {
    \begin{array}{r}
  \buildstmt(\overname{\Nstmt(\Trepeat, \Nstmtlist, \Tuntil, \Nexpr, \Tsemicolon)}{\vparsednode})
  \astarrow\\
  \overname{\SRepeat(\astof{\vstmtlist}, \astof{\vexpr}, \None)}{\vastnode}
    \end{array}
  }
}
\end{mathpar}

\begin{mathpar}
\inferrule[looplimit\_repeat]{
  \buildexpr(\vlimitexpr) \astarrow \astversion{\vlimitexpr}
}{
  {
    \begin{array}{r}
  \buildstmt\left(\overname{\Nstmt\left(
    \begin{array}{r}
    \Tlooplimit, \Tlpar, \namednode{\vlimitexpr}{\Nexpr}, \Trpar, \Trepeat, \\
    \wrappedline\ \Nstmtlist, \Tuntil, \Nexpr, \Tsemicolon
    \end{array}
    \right)}{\vparsednode}\right)
  \astarrow\\
  \overname{\SRepeat(\astof{\vstmtlist}, \astof{\vexpr}, \langle\astversion{\vlimitexpr}\rangle)}{\vastnode}
    \end{array}
  }
}
\end{mathpar}

\begin{mathpar}
\inferrule[throw\_some]{}{
  \buildstmt(\overname{\Nstmt(\Tthrow, \Nexpr, \Tsemicolon)}{\vparsednode})
  \astarrow
  \overname{\SThrow(\langle(\astof{\vexpr}, \None)\rangle)}{\vastnode}
}
\end{mathpar}

\begin{mathpar}
\inferrule[throw\_none]{}{
  \buildstmt(\overname{\Nstmt(\Tthrow, \Tsemicolon)}{\vparsednode})
  \astarrow
  \overname{\SThrow(\None)}{\vastnode}
}
\end{mathpar}

\begin{mathpar}
\inferrule[pragma]{}{
  \buildstmt(\overname{\Nstmt(\Tpragma, \Tidentifier, \Clist{\Nexpr}, \Tsemicolon)}{\vparsednode})
  \astarrow
  \overname{\tododefine{pragma\_node}}{\vastnode}
}
\end{mathpar}

\section{SyntaxRule.StmtList \label{sec:SyntaxRule.StmtList}}
\hypertarget{build-stmtlist}{}
The function
\[
  \buildstmtlist(\overname{\parsenode{\Nstmtlist}}{\vparsednode}) \;\aslto\; \overname{\stmt}{\vastnode}
\]
transforms a parse node $\vparsednode$ into an AST node $\vastnode$.

\isempty{\subsection{Prose}}
\subsection{Formally}
\begin{mathpar}
\inferrule{
  \buildlist[\Nstmt](\vstmts) \astarrow \vstmtlist\\
  \stmtfromlist(\vstmtlist) \astarrow \vastnode
}{
  \buildstmtlist(\Nstmtlist(\namednode{\vstmts}{\nonemptylist{\Nstmt}})) \astarrow \vastnode
}
\end{mathpar}

\section{SyntaxRule.SElse \label{sec:SyntaxRule.SElse}}
\hypertarget{build-selse}{}
The function
\[
  \buildselse(\overname{\parsenode{\Nselse}}{\vparsednode}) \;\aslto\; \overname{\stmt}{\vastnode}
\]
transforms a parse node $\vparsednode$ into an AST node $\vastnode$.

\isempty{\subsection{Prose}}
\subsection{Formally}
\begin{mathpar}
\inferrule[elseif]{}{
  {
    \begin{array}{r}
  \buildselse(\Nselse(\Telseif, \Nexpr, \Twhen, \Nstmtlist, \Nselse)) \astarrow \\
  \overname{\SCond(\astof{\vexpr}, \astof{\vstmtlist}, \astof{\vselse})}{\vastnode}
    \end{array}
  }
}
\end{mathpar}

\begin{mathpar}
\inferrule[pass]{}{
  \buildselse(\Nselse(\Tpass)) \astarrow \overname{\SPass}{\vastnode}
}
\end{mathpar}

\begin{mathpar}
\inferrule[else]{}{
  \buildselse(\Nselse(\Telse, \punnode{\Nstmtlist})) \astarrow \overname{\astof{\vstmtlist}}{\vastnode}
}
\end{mathpar}

\section{SyntaxRule.LExpr \label{sec:SyntaxRule.LExpr}}
\hypertarget{build-lexpr}{}
The function
\[
  \buildlexpr(\overname{\parsenode{\Nlexpr}}{\vparsednode}) \;\aslto\; \overname{\lexpr}{\vastnode}
\]
transforms a parse node $\vparsednode$ into an AST node $\vastnode$.

\isempty{\subsection{Prose}}
\subsection{Formally}
\begin{mathpar}
\inferrule[lexpr\_atom]{}{
  \buildlexpr(\Nlexpr(\punnode{\Nlexpratom})) \astarrow \overname{\astof{\vlexpratom}}{\vastnode}
}
\end{mathpar}

\begin{mathpar}
\inferrule[discard]{}{
  \buildlexpr(\Nlexpr(\Tminus)) \astarrow \overname{\LEDiscard}{\vastnode}
}
\end{mathpar}

\begin{mathpar}
\inferrule[multi\_lexpr]{
  \buildclist[\Nlexpr](\vlexprs) \astarrow \vlexprasts
}{
  \buildlexpr(\Nlexpr(\Tlpar, \namednode{\vlexprs}{\NClist{\Nlexpr}}, \Trpar)) \astarrow
  \overname{\LEDestructuring(\vlexprasts)}{\vastnode}
}
\end{mathpar}

\section{SyntaxRule.LExprAtom \label{sec:SyntaxRule.LExprAtom}}
\hypertarget{build-lexpratom}{}
The function
\[
  \buildlexpratom(\overname{\parsenode{\Nlexpratom}}{\vparsednode}) \;\aslto\; \overname{\lexpr}{\vastnode}
\]
transforms a parse node $\vparsednode$ into an AST node $\vastnode$.

\isempty{\subsection{Prose}}
\subsection{Formally}
\begin{mathpar}
\inferrule[var]{}{
  \buildlexpratom(\Nlexpr(\Tidentifier(\id))) \astarrow
  \overname{\LEVar(\id)}{\vastnode}
}
\end{mathpar}

\begin{mathpar}
\inferrule[slice]{}{
  \buildlexpratom(\Nlexpr(\punnode{\Nlexpratom}, \punnode{\Nslices})) \astarrow
  \overname{\LESlice(\astof{\vlexpratom}, \astof{\vslices})}{\vastnode}
}
\end{mathpar}

\begin{mathpar}
\inferrule[set\_field]{}{
  \buildlexpratom(\Nlexpr(\punnode{\Nlexpratom}, \Tdot, \Tidentifier(\id))) \astarrow
  \overname{\LESetField(\astof{\vlexpratom}, \id)}{\vastnode}
}
\end{mathpar}

\begin{mathpar}
\inferrule[set\_fields]{
  \buildclist[\buildidentity](\vfields) \astarrow \vfieldasts
}{
  {
  \begin{array}{r}
  \buildlexpratom(\Nlexpr(\punnode{\Nlexpratom}, \Tdot, \Tlbracket, \namednode{\vfields}{\Clist{\Tidentifier}}, \Trbracket)) \astarrow\\
  \overname{\LESetFields(\astof{\vlexpratom}, \vfieldasts)}{\vastnode}
  \end{array}
  }
}
\end{mathpar}

\begin{mathpar}
\inferrule[concat]{
  \buildclist[\buildlexpratom](\vlexprs) \astarrow \vlexprasts
}{
  {
    \begin{array}{r}
  \buildlexpratom(\Nlexpr(\Tlbracket, \namednode{\vlexprs}{\NClist{{\Nlexpratom}}}, \Trbracket)) \astarrow\\
  \overname{\LEConcat(\vlexprasts)}{\vastnode}
    \end{array}
  }
}
\end{mathpar}

\section{SyntaxRule.DeclItem \label{sec:SyntaxRule.DeclItem}}
\hypertarget{build-declitem}{}
The function
\[
  \builddeclitem(\overname{\parsenode{\Ndeclitem}}{\vparsednode}) \;\aslto\; \overname{\localdeclitem}{\vastnode}
\]
transforms a parse node $\vparsednode$ into an AST node $\vastnode$.

\isempty{\subsection{Prose}}
\subsection{Formally}
\begin{mathpar}
\inferrule[typed]{}{
  {
    \begin{array}{r}
  \builddeclitem(\Ndeclitem(\punnode{\Nuntypeddeclitem}, \punnode{\Nasty})) \astarrow \\
  \overname{\LDITyped(\astof{\vuntypedlocaldeclitem}, \astof{\vasty})}{\vastnode}
    \end{array}
  }
}
\end{mathpar}

\begin{mathpar}
\inferrule[untyped]{}{
  \builddeclitem(\Ndeclitem(\punnode{\Nuntypeddeclitem})) \astarrow
  \overname{\astof{\vuntypedlocaldeclitem}}{\vastnode}
}
\end{mathpar}

\section{SyntaxRule.UntypedDeclItem \label{sec:SyntaxRule.UntypedDeclItem}}
\hypertarget{build-untypeddeclitem}{}
The function
\[
  \builduntypeddeclitem(\overname{\parsenode{\Nuntypeddeclitem}}{\vparsednode}) \;\aslto\; \overname{\localdeclitem}{\vastnode}
\]
transforms a parse node $\vparsednode$ into an AST node $\vastnode$.

\isempty{\subsection{Prose}}
\subsection{Formally}
\begin{mathpar}
\inferrule[var]{}{
  \builduntypeddeclitem(\Nuntypeddeclitem(\Tidentifier(\id))) \astarrow
  \overname{\LDIVar(\id)}{\vastnode}
}
\end{mathpar}

\begin{mathpar}
\inferrule[discard]{}{
  \builduntypeddeclitem(\Nuntypeddeclitem(\Tminus)) \astarrow
  \overname{\LDIDiscard}{\vastnode}
}
\end{mathpar}

\begin{mathpar}
\inferrule[tuple]{
  \buildclist[\builddeclitem](\vdeclitems) \astarrow \vdeclitemasts
}{
  {
    \begin{array}{r}
  \builduntypeddeclitem(\Nuntypeddeclitem(\namednode{\vdeclitems}{\Plisttwo{\Ndeclitem}})) \astarrow \\
  \overname{\LDITuple(\vdeclitemasts)}{\vastnode}
    \end{array}
  }
}
\end{mathpar}

\section{SyntaxRule.IntConstraints \label{sec:SyntaxRule.IntConstraints}}
\hypertarget{build-intconstraints}{}
The function
\[
  \buildintconstraints(\overname{\parsenode{\Nintconstraints}}{\vparsednode}) \;\aslto\; \overname{\intconstraints}{\vastnode}
\]
transforms a parse node $\vparsednode$ into an AST node $\vastnode$.

\isempty{\subsection{Prose}}
\subsection{Formally}
\begin{mathpar}
\inferrule{
  \buildclist[\buildintconstraint](\vconstraintasts) \astarrow \vconstraintasts
}{
  {
    \begin{array}{r}
  \buildintconstraints(\Nintconstraints(\Tlbrace, \namednode{\vconstraints}{\NClist{\Nintconstraint}}, \Trbrace)) \astarrow\\
  \overname{\wellconstrained(\vconstraintasts)}{\vastnode}
    \end{array}
  }
}
\end{mathpar}

\section{SyntaxRule.IntConstraintsopt \label{sec:SyntaxRule.IntConstraintsopt}}
\hypertarget{build-intconstraintsopt}{}
The function
\[
  \buildintconstraintsopt(\overname{\parsenode{\Nintconstraintsopt}}{\vparsednode}) \;\aslto\; \overname{\intconstraints}{\vastnode}
\]
transforms a parse node $\vparsednode$ into an AST node $\vastnode$.

\isempty{\subsection{Prose}}
\subsection{Formally}
\begin{mathpar}
\inferrule[constrained]{}{
  \buildintconstraintsopt(\Nintconstraints(\punnode{\Nintconstraints})) \astarrow
  \overname{\astof{\vintconstraints}}{\vastnode}
}
\end{mathpar}

\begin{mathpar}
\inferrule[unconstrained]{}{
  \buildintconstraintsopt(\Nintconstraints(\emptysentence)) \astarrow
  \overname{\unconstrained}{\vastnode}
}
\end{mathpar}

\section{SyntaxRule.IntConstraint \label{sec:SyntaxRule.IntConstraint}}
\hypertarget{build-intconstraint}{}
The function
\[
  \buildintconstraint(\overname{\parsenode{\Nintconstraint}}{\vparsednode}) \;\aslto\; \overname{\intconstraint}{\vastnode}
\]
transforms a parse node $\vparsednode$ into an AST node $\vastnode$.

\isempty{\subsection{Prose}}
\subsection{Formally}
\begin{mathpar}
\inferrule[exact]{}{
  \buildintconstraint(\Nintconstraint(\punnode{\Nexpr})) \astarrow
  \overname{\ConstraintExact(\astof{\vexpr})}{\vastnode}
}
\end{mathpar}

\begin{mathpar}
\inferrule[range]{
  \buildexpr(\vfromexpr) \astarrow \astversion{\vfromexpr}\\
  \buildexpr(\vtoexpr) \astarrow \astversion{\vtoexpr}\\
}{
  {
    \begin{array}{r}
  \buildintconstraint(\Nintconstraint(\namednode{\vfromexpr}{\Nexpr}, \Tslicing, \namednode{\vtoexpr}{\Nexpr})) \astarrow\\
  \overname{\ConstraintRange(\astversion{\vfromexpr}, \astversion{\vtoexpr})}{\vastnode}
    \end{array}
  }
}
\end{mathpar}

\section{SyntaxRule.ExprPattern \label{sec:SyntaxRule.ExprPattern}}
\hypertarget{build-exprpattern}{}
The function
\[
  \buildexprpattern(\overname{\parsenode{\Nexprpattern}}{\vparsednode}) \;\aslto\; \overname{\expr}{\vastnode}
\]
transforms a parse node $\vparsednode$ into an AST node $\vastnode$.

\isempty{\subsection{Prose}}
\subsection{Formally}
\begin{mathpar}
\inferrule[literal]{}{
  \buildexprpattern(\Nexprpattern(\punnode{\Nvalue})) \astarrow
  \overname{\ELiteral(\astof{\vvalue})}{\vastnode}
}
\end{mathpar}

\begin{mathpar}
  \inferrule[var]{}{
  \buildexprpattern(\Nexprpattern(\Tidentifier(\id))) \astarrow
  \overname{\EVar(\id)}{\vastnode}
}
\end{mathpar}

\begin{mathpar}
  \inferrule[binop]{}{
    {
      \begin{array}{r}
  \buildexprpattern(\Nexprpattern(\punnode{\Nexprpattern}, \punnode{\Nbinop}, \punnode{\Nexpr})) \astarrow\\
  \overname{\EBinop(\astof{\vexprpattern}, \astof{\vbinop}, \astof{\vexpr})}{\vastnode}
      \end{array}
    }
}
\end{mathpar}

\begin{mathpar}
  \inferrule[unop]{}{
  \buildexprpattern(\Nexprpattern(\punnode{\Nunop}, \punnode{\Nexpr})) \astarrow
  \overname{\EUnop(\astof{\vunop}, \astof{\vexpr})}{\vastnode}
}
\end{mathpar}

\begin{mathpar}
  \inferrule[cond]{
    \buildexpr(\vcondexpr) \astarrow \astversion{\vcondexpr}\\
    \buildexpr(\vthenexpr) \astarrow \astversion{\vthenexpr}
  }{
    {
      \begin{array}{r}
  \buildexprpattern\left(\Nexprpattern\left(
    \begin{array}{l}
    \Tif, \namednode{\vcondexpr}{\Nexpr}, \Tthen, \\
    \wrappedline\ \namednode{\vthenexpr}{\Nexpr}, \punnode{\Neelse}
    \end{array}
    \right)\right) \astarrow\\
  \overname{\ECond(\astversion{\vcondexpr}, \astversion{\vthenexpr}, \astof{\veelse})}{\vastnode}
      \end{array}
    }
}
\end{mathpar}

\begin{mathpar}
  \inferrule[call]{
    \buildplist[\buildexpr](\vargs) \astarrow \vexprasts
  }{
  \buildexprpattern(\Nexprpattern(\Tidentifier(\id), \namednode{\vargs}{\Plist{\Nexpr}})) \astarrow
  \overname{\ECall(\id, \vexprasts)}{\vastnode}
}
\end{mathpar}

\begin{mathpar}
  \inferrule[slice]{}{
    {
      \begin{array}{r}
  \buildexprpattern(\Nexprpattern(\punnode{\Nexprpattern}, \punnode{\Nslice})) \astarrow\\
  \overname{\ESlice(\astof{\vexprpattern}, \astof{\vslice})}{\vastnode}
      \end{array}
    }
}
\end{mathpar}

\begin{mathpar}
  \inferrule[get\_field]{}{
  \buildexprpattern(\Nexprpattern(\Nexprpattern, \Tdot, \Tidentifier)) \astarrow
  \overname{\EGetField()}{\vastnode}
}
\end{mathpar}

\begin{mathpar}
  \inferrule[get\_fields]{
    \buildclist[\buildidentity](\vids) \astarrow \vidasts
  }{
    {
      \begin{array}{r}
  \buildexprpattern(\Nexprpattern(\punnode{\Nexprpattern}, \Tdot, \Tlbracket, \namednode{\vids}{\NClist{\Tidentifier}}, \Trbracket)) \astarrow\\
  \overname{\EGetFields(\astof{\vexprpattern}, \vidasts)}{\vastnode}
      \end{array}
    }
}
\end{mathpar}

\begin{mathpar}
  \inferrule[concat]{
    \buildclist[\buildexpr](\vexprs) \astarrow \vexprasts
  }{
  \buildexprpattern(\Nexprpattern(\Tlbracket, \namednode{\vexprs}{\NClist{\Nexpr}}, \Trbracket)) \astarrow
  \overname{\EConcat(\vexprasts)}{\vastnode}
}
\end{mathpar}

\begin{mathpar}
  \inferrule[atc]{}{
    {
      \begin{array}{r}
  \buildexprpattern(\Nexprpattern(\punnode{\Nexprpattern}, \Tas, \punnode{\Nty})) \astarrow\\
  \overname{\EATC(\astof{\vexprpattern}, \astof{\tty})}{\vastnode}
      \end{array}
    }
}
\end{mathpar}

\begin{mathpar}
  \inferrule[atc\_int\_constraints]{}{
    {
      \begin{array}{r}
  \buildexprpattern(\Nexprpattern(\punnode{\Nexprpattern}, \Tas, \punnode{\Nintconstraints})) \astarrow\\
  \overname{\EATC(\astof{\vexprpattern}, \TInt(\astof{\vintconstraints}))}{\vastnode}
\end{array}
}
}
\end{mathpar}

\begin{mathpar}
  \inferrule[pattern\_set]{}{
    {
      \begin{array}{r}
  \buildexprpattern(\Nexprpattern(\punnode{\Nexprpattern}, \Tin, \punnode{\Npatternset})) \astarrow\\
  \overname{\EPattern(\astof{\vexprpattern}, \astof{\vpatternset})}{\vastnode}
\end{array}
}
}
\end{mathpar}

\begin{mathpar}
  \inferrule[pattern\_mask]{}{
    {
      \begin{array}{r}
  \buildexprpattern(\Nexprpattern(\punnode{\Nexprpattern}, \Tin, \Tmasklit(\vm))) \astarrow\\
  \overname{\EPattern(\astof{\vexprpattern}, \vm)}{\vastnode}
\end{array}
}
}
\end{mathpar}

\begin{mathpar}
  \inferrule[unknown]{}{
  \buildexprpattern(\Nexprpattern(\Tunknown, \Tcolon, \punnode{\Nty})) \astarrow
  \overname{\EUnknown(\astof{\tty})}{\vastnode}
}
\end{mathpar}

\begin{mathpar}
  \inferrule[record]{
    \buildclist[\buildfieldassign](\vfieldassigns) \astarrow \vfieldassignasts
  }{
    {
      \begin{array}{r}
  \buildexprpattern\left(\Nexprpattern\left(
    \begin{array}{l}
    \Tidentifier(\vt), \Tlbrace, \\
    \wrappedline\ \namednode{\vfieldassigns}{\Clist{\Nfieldassign}}, \\
    \wrappedline\ \Trbrace
    \end{array}
    \right)\right) \\
    \astarrow\ \overname{\ERecord(\TNamed(\vt), \vfieldassignasts)}{\vastnode}
\end{array}
}
}
\end{mathpar}

\begin{mathpar}
  \inferrule[sub\_expr]{}{
  \buildexprpattern(\Nexprpattern(\Tlpar, \punnode{\Nexprpattern}, \Trpar)) \astarrow
  \overname{\astof{\vexprpattern}}{\vastnode}
}
\end{mathpar}

\section{SyntaxRule.PatternSet \label{sec:SyntaxRule.PatternSet}}
\hypertarget{build-patternset}{}
The function
\[
  \buildpatternset(\overname{\parsenode{\Npatternset}}{\vparsednode}) \;\aslto\; \overname{\pattern}{\vastnode}
\]
transforms a parse node $\vparsednode$ into an AST node $\vastnode$.

\isempty{\subsection{Prose}}
\subsection{Formally}
\begin{mathpar}
\inferrule[not]{}{
  {
    \begin{array}{r}
  \buildpatternset(\Npatternset(\Tbnot, \Tlbrace, \punnode{\Npatternlist}, \Trbrace)) \astarrow\\
  \overname{\PatternNot(\astof{\vpatternlist})}{\vastnode}
    \end{array}
  }
}
\end{mathpar}

\begin{mathpar}
\inferrule[list]{}{
  \buildpatternset(\Npatternset(\Tlbrace, \punnode{\Npatternlist}, \Trbrace)) \astarrow
  \overname{\astof{\vpatternlist}}{\vastnode}
}
\end{mathpar}

\section{SyntaxRule.PatternList \label{sec:SyntaxRule.PatternList}}
\hypertarget{build-patternlist}{}
The function
\[
  \buildpatternlist(\overname{\parsenode{\Npatternlist}}{\vparsednode}) \;\aslto\; \overname{\pattern}{\vastnode}
\]
transforms a parse node $\vparsednode$ into an AST node $\vastnode$.

\isempty{\subsection{Prose}}
\subsection{Formally}
\begin{mathpar}
\inferrule{
  \buildclist[\buildpattern](\vpatterns) \astarrow \vpatternasts
}{
  {
    \begin{array}{r}
  \buildpatternlist(\Npatternlist(\namednode{\vpatterns}{\NClist{\Npattern}})) \astarrow\\
  \overname{\PatternAny(\vpatternasts)}{\vastnode}
    \end{array}
  }
}
\end{mathpar}

\section{SyntaxRule.Pattern \label{sec:SyntaxRule.Pattern}}
\hypertarget{build-pattern}{}
The function
\[
  \buildpattern(\overname{\parsenode{\Npattern}}{\vparsednode}) \;\aslto\; \overname{\pattern}{\vastnode}
\]
transforms a parse node $\vparsednode$ into an AST node $\vastnode$.

\isempty{\subsection{Prose}}
\subsection{Formally}
\begin{mathpar}
\inferrule[single]{}{
  \buildpattern(\Npattern(\punnode{\Nexprpattern})) \astarrow
  \overname{\PatternSingle(\astof{\vexprpattern})}{\vastnode}
}
\end{mathpar}

\begin{mathpar}
\inferrule[range]{}{
  {
    \begin{array}{r}
  \buildpattern(\Npattern(\punnode{\Nexprpattern}, \Tslicing, \punnode{\Nexpr})) \astarrow\\
  \overname{\PatternRange(\astof{\vexprpattern}, \astof{\vexpr})}{\vastnode}
    \end{array}
  }
}
\end{mathpar}

\begin{mathpar}
\inferrule[all]{}{
  \buildpattern(\Npattern(\Tminus)) \astarrow
  \overname{\PatternAll}{\vastnode}
}
\end{mathpar}

\begin{mathpar}
\inferrule[leq]{}{
  \buildpattern(\Npattern(\Tleq, \punnode{\Nexpr})) \astarrow
  \overname{\PatternLeq(\astof{\vexpr})}{\vastnode}
}
\end{mathpar}

\begin{mathpar}
\inferrule[geq]{}{
  \buildpattern(\Npattern(\Tgeq, \punnode{\Nexpr})) \astarrow
  \overname{\PatternGeq(\astof{\vexpr})}{\vastnode}
}
\end{mathpar}

\begin{mathpar}
\inferrule[mask]{}{
  \buildpattern(\Npattern(\Tmasklit(\vm))) \astarrow
  \overname{\PatternMask(\vm)}{\vastnode}
}
\end{mathpar}

\begin{mathpar}
\inferrule[tuple]{
  \buildplist[\buildpattern](\vpatterns) \astarrow \vpatternasts
}{
  \buildpattern(\Npattern(\namednode{\vpatterns}{\Plisttwo{\Npattern}})) \astarrow
  \overname{\PatternTuple(\vpatternasts)}{\vastnode}
}
\end{mathpar}

\begin{mathpar}
\inferrule[set]{}{
  \buildpattern(\Npattern(\punnode{\Npatternset})) \astarrow
  \overname{\astof{\vpatternset}}{\vastnode}
}
\end{mathpar}

\section{SyntaxRule.Fields \label{sec:SyntaxRule.Fields}}
\hypertarget{build-fields}{}
The function
\[
  \buildfields(\overname{\parsenode{\Nfields}}{\vparsednode}) \;\aslto\; \overname{\Field^*}{\vastnode}
\]
transforms a parse node $\vparsednode$ into an AST node $\vastnode$.

\isempty{\subsection{Prose}}
\subsection{Formally}
\begin{mathpar}
\inferrule{
  \buildtclist[\buildtypedidentifier](\vfields) \astarrow \vfieldasts
}{
  \buildfields(\Nfields(\Tlbrace, \namednode{\vfields}{\TClist{\Ntypedidentifier}}, \Trbrace)) \astarrow
  \overname{\vfieldasts}{\vastnode}
}
\end{mathpar}

\section{SyntaxRule.FieldsOpt \label{sec:SyntaxRule.FieldsOpt}}
\hypertarget{build-fieldsopt}{}
The function
\[
  \buildfieldsopt(\overname{\parsenode{\Nfieldsopt}}{\vparsednode}) \;\aslto\; \overname{\Field^*}{\vastnode}
\]
transforms a parse node $\vparsednode$ into an AST node $\vastnode$.

\isempty{\subsection{Prose}}
\subsection{Formally}
\begin{mathpar}
\inferrule[fields]{}{
  \buildfieldsopt(\Nfieldsopt(\punnode{\Nfields})) \astarrow
  \overname{\astof{\vfields}}{\vastnode}
}
\end{mathpar}

\begin{mathpar}
\inferrule[empty]{}{
  \buildfieldsopt(\Nfieldsopt(\emptysentence)) \astarrow
  \overname{\emptylist}{\vastnode}
}
\end{mathpar}

\section{SyntaxRule.NSlices \label{sec:SyntaxRule.NSlices}}
\hypertarget{build-nslices}{}
The function
\[
  \buildnslices(\overname{\parsenode{\Nnslices}}{\vparsednode}) \;\aslto\; \overname{\slice^+}{\vastnode}
\]
transforms a parse node $\vparsednode$ into an AST node $\vastnode$.

\isempty{\subsection{Prose}}
\subsection{Formally}
\begin{mathpar}
\inferrule{
  \buildclist[\buildslice](\vslices) \astarrow \vsliceasts
}{
  \buildnslices(\Nnslices(\Tlbracket, \namednode{\vslices}{\NClist{\Nslice}}, \Trbracket)) \astarrow
  \overname{\vsliceasts}{\vastnode}
}
\end{mathpar}

\section{SyntaxRule.Slices \label{sec:SyntaxRule.Slices}}
\hypertarget{build-slices}{}
The function
\[
  \buildslices(\overname{\parsenode{\Nslices}}{\vparsednode}) \;\aslto\; \overname{\slice^*}{\vastnode}
\]
transforms a parse node $\vparsednode$ into an AST node $\vastnode$.

\isempty{\subsection{Prose}}
\subsection{Formally}
\begin{mathpar}
\inferrule{
  \buildclist[\buildslice](\vslices) \astarrow \vsliceasts
}{
  \buildslices(\Nslices(\Tlbracket, \namednode{\vslices}{\Clist{\Nslice}}, \Trbracket)) \astarrow
  \overname{\vsliceasts}{\vastnode}
}
\end{mathpar}

\section{SyntaxRule.Slice \label{sec:SyntaxRule.Slice}}
\hypertarget{build-slice}{}
The function
\[
  \buildslice(\overname{\parsenode{\Nslice}}{\vparsednode}) \;\aslto\; \overname{\slice}{\vastnode}
\]
transforms a parse node $\vparsednode$ into an AST node $\vastnode$.

\isempty{\subsection{Prose}}
\subsection{Formally}
\begin{mathpar}
\inferrule[single]{}{
  \buildslices(\Nslice(\punnode{\Nexpr})) \astarrow
  \overname{\SliceSingle(\astof{\vexpr})}{\vastnode}
}
\end{mathpar}

\begin{mathpar}
\inferrule[range]{
  \buildexpr(\veone) \astarrow \astversion{\veone}\\
  \buildexpr(\vetwo) \astarrow \astversion{\vetwo}
}{
  \buildslices(\Nslice(\namednode{\veone}{\Nexpr}, \Tcolon, \namednode{\vetwo}{\Nexpr})) \astarrow
  \overname{\SliceRange(\astversion{\veone}, \astversion{\vetwo})}{\vastnode}
}
\end{mathpar}

\begin{mathpar}
\inferrule[length]{
  \buildexpr(\veone) \astarrow \astversion{\veone}\\
  \buildexpr(\vetwo) \astarrow \astversion{\vetwo}
}{
  \buildslices(\Nslice(\namednode{\veone}{\Nexpr}, \Tpluscolon, \namednode{\vetwo}{\Nexpr})) \astarrow
  \overname{\SliceLength(\astversion{\veone}, \astversion{\vetwo})}{\vastnode}
}
\end{mathpar}

\begin{mathpar}
\inferrule[star]{
  \buildexpr(\veone) \astarrow \astversion{\veone}\\
  \buildexpr(\vetwo) \astarrow \astversion{\vetwo}
}{
  \buildslices(\Nslice(\namednode{\veone}{\Nexpr}, \Tstarcolon, \namednode{\vetwo}{\Nexpr})) \astarrow
  \overname{\SliceStar(\astversion{\veone}, \astversion{\vetwo})}{\vastnode}
}
\end{mathpar}

\section{SyntaxRule.Bitfields \label{sec:SyntaxRule.Bitfields}}
\hypertarget{build-bitfields}{}
The function
\[
  \buildbitfields(\overname{\parsenode{\Nbitfields}}{\vparsednode}) \;\aslto\; \overname{\bitfield^*}{\vastnode}
\]
transforms a parse node $\vparsednode$ into an AST node $\vastnode$.

\isempty{\subsection{Prose}}
\subsection{Formally}
\begin{mathpar}
\inferrule{
  \buildtclist[\buildbitfield](\vbitfields) \astarrow \vbitfieldasts
}{
  \buildbitfields(\Nbitfields(\Tlbrace, \namednode{\vbitfields}{\TClist{\Nbitfield}}, \Trbrace)) \astarrow
  \overname{\vbitfieldasts}{\vastnode}
}
\end{mathpar}

\section{SyntaxRule.Bitfield \label{sec:SyntaxRule.Bitfield}}
\hypertarget{build-bitfield}{}
The function
\[
  \buildbitfield(\overname{\parsenode{\Nbitfield}}{\vparsednode}) \;\aslto\; \overname{\bitfield}{\vastnode}
\]
transforms a parse node $\vparsednode$ into an AST node $\vastnode$.

\isempty{\subsection{Prose}}
\subsection{Formally}
\begin{mathpar}
\inferrule[simple]{}{
  \buildbitfields(\Nbitfield(\punnode{\Nnslices}, \Tidentifier(\vx))) \astarrow
  \overname{\BitFieldSimple(\vx, \astof{\vslices})}{\vastnode}
}
\end{mathpar}

\begin{mathpar}
\inferrule[nested]{}{
  {
    \begin{array}{r}
  \buildbitfields(\Nbitfield(\punnode{\Nnslices}, \Tidentifier(\vx), \punnode{\Nbitfields})) \astarrow\\
  \overname{\BitFieldNested(\vx, \astof{\vslices}, \astof{\vbitfields})}{\vastnode}
    \end{array}
  }
}
\end{mathpar}

\begin{mathpar}
\inferrule[type]{}{
  \buildbitfields(\Nbitfield(\punnode{\Nnslices}, \Tidentifier(\vx), \Tcolon, \punnode{\Nty})) \astarrow
  \overname{\BitFieldType(\vx, \astof{\vslices}, \astof{\tty})}{\vastnode}
}
\end{mathpar}

\section{SyntaxRule.Ty \label{sec:SyntaxRule.Ty}}
\hypertarget{build-ty}{}
The function
\[
  \buildty(\overname{\parsenode{\Nty}}{\vparsednode}) \;\aslto\; \overname{\ty}{\vastnode}
\]
transforms a parse node $\vparsednode$ into an AST node $\vastnode$.

\isempty{\subsection{Prose}}
\subsection{Formally}
\begin{mathpar}
\inferrule[integer]{}{
  \buildty(\Nty(\Tinteger, \punnode{\Nintconstraintsopt})) \astarrow
  \overname{\TInt(\vintconstraintsopt)}{\vastnode}
}
\end{mathpar}

\begin{mathpar}
\inferrule[real]{}{
  \buildty(\Nty(\Treal)) \astarrow
  \overname{\TReal}{\vastnode}
}
\end{mathpar}

\begin{mathpar}
\inferrule[boolean]{}{
  \buildty(\Nty(\Tboolean)) \astarrow
  \overname{\TBool}{\vastnode}
}
\end{mathpar}

\begin{mathpar}
\inferrule[string]{}{
  \buildty(\Nty(\Tstring)) \astarrow
  \overname{\TString}{\vastnode}
}
\end{mathpar}

\begin{mathpar}
\inferrule[bit]{}{
  \buildty(\Nty(\Tbit)) \astarrow
  \overname{\TBits(\ELiteral(\lint(1)), \emptylist)}{\vastnode}
}
\end{mathpar}

\begin{mathpar}
\inferrule[bits]{
  \buildlist[\buildbitfield](\vbitfields) \astarrow \vbitfieldasts
}{
  {
    \begin{array}{r}
  \buildty(\Nty(\Tbits, \Tlpar, \punnode{\Nexpr}, \Trpar, \namednode{\vbitfields}{\maybeemptylist{\Nbitfields}})) \astarrow\\
  \overname{\TBits(\astof{\vexpr}, \vbitfieldasts)}{\vastnode}
    \end{array}
  }
}
\end{mathpar}

\begin{mathpar}
\inferrule[tuple]{
  \buildplist[\buildty](\vtypes) \astarrow \vtypeasts
}{
  \buildty(\Nty(\namednode{\vtypes}{\Plist{\Nty}})) \astarrow
  \overname{\TTuple(\vtypeasts)}{\vastnode}
}
\end{mathpar}

\begin{mathpar}
\inferrule[named]{}{
  \buildty(\Nty(\Tidentifier(\id))) \astarrow
  \overname{\TNamed(\id)}{\vastnode}
}
\end{mathpar}

\begin{mathpar}
\inferrule[array]{}{
  \buildty(\Nty(\Tarray, \Tlbracket, \punnode{\Nexpr}, \Trbracket, \Tof, \punnode{\Nty})) \astarrow
  \overname{\TArray(\ArrayLengthExpr(\astof{\vexpr}), \astof{\tty})}{\vastnode}
}
\end{mathpar}

\section{SyntaxRule.TyDecl \label{sec:SyntaxRule.TyDecl}}
\hypertarget{build-tydecl}{}
The function
\[
  \buildtydecl(\overname{\parsenode{\Ntydecl}}{\vparsednode}) \;\aslto\; \overname{\ty}{\vastnode}
\]
transforms a parse node $\vparsednode$ into an AST node $\vastnode$.

\isempty{\subsection{Prose}}
\subsection{Formally}
\begin{mathpar}
\inferrule[ty]{}{
  \buildtydecl(\Ntydecl(\punnode{\Nty})) \astarrow
  \overname{\astversion{\tty}}{\vastnode}
}
\end{mathpar}

\begin{mathpar}
\inferrule[enumeration]{
  \buildtclist[\buildidentity](\vids) \astarrow \vidasts
}{
  {
    \begin{array}{r}
  \buildtydecl(\Ntydecl(\Tenumeration, \Tlbrace, \namednode{\vids}{\NTClist{\Tidentifier}}, \Trbrace)) \astarrow\\
  \overname{\TEnum(\vidasts)}{\vastnode}
\end{array}
  }
}
\end{mathpar}

\begin{mathpar}
\inferrule[record]{}{
  \buildtydecl(\Ntydecl(\Trecord, \punnode{\Nfieldsopt})) \astarrow
  \overname{\TRecord(\astof{\vfieldsopt})}{\vastnode}
}
\end{mathpar}

\begin{mathpar}
\inferrule[exception]{}{
  \buildtydecl(\Ntydecl(\Texception, \punnode{\Nfieldsopt})) \astarrow
  \overname{\TException(\astof{\vfieldsopt})}{\vastnode}
}
\end{mathpar}

\section{SyntaxRule.FieldAssign \label{sec:SyntaxRule.FieldAssign}}
\hypertarget{build-fieldassign}{}
The function
\[
  \buildfieldassign(\overname{\parsenode{\Nfieldassign}}{\vparsednode}) \;\aslto\; \overname{(\identifier\times\expr)}{\vastnode}
\]
transforms a parse node $\vparsednode$ into an AST node $\vastnode$.

\isempty{\subsection{Prose}}
\subsection{Formally}
\begin{mathpar}
\inferrule{}{
  \buildfieldassign(\Nfieldassign(\Tidentifier(\id), \Teq, \punnode{\Nexpr})) \astarrow
  \overname{(\id, \astof{\vexpr})}{\vastnode}
}
\end{mathpar}

\section{SyntaxRule.EElse \label{sec:SyntaxRule.EElse}}
\hypertarget{build-eelse}{}
The function
\[
  \buildeelse(\overname{\parsenode{\Nfieldassign}}{\vparsednode}) \;\aslto\; \overname{\expr}{\vastnode}
\]
transforms a parse node $\vparsednode$ into an AST node $\vastnode$.

\isempty{\subsection{Prose}}
\subsection{Formally}
\begin{mathpar}
\inferrule[else]{}{
  \buildeelse(\Neelse(\Telse, \punnode{\Nexpr})) \astarrow
  \overname{\astof{\vexpr}}{\vastnode}
}
\end{mathpar}

\begin{mathpar}
\inferrule[else\_if]{
  \buildexpr(\vcondexpr) \astarrow \astversion{\vcondexpr}\\
  \buildexpr(\vthenexpr) \astarrow \astversion{\vthenexpr}
}{
  {
    \begin{array}{r}
  \buildeelse\left(\Neelse\left(
    \begin{array}{l}
    \Telseif, \namednode{\vcondexpr}{\Nexpr},  \\
    \wrappedline\ \Tthen, \namednode{\vthenexpr}{\Nexpr}, \punnode{\Neelse}
  \end{array}
    \right)\right) \astarrow\\
  \overname{\ECond(\astversion{\vcondexpr}, \astversion{\vthenexpr}, \astof{\veelse})}{\vastnode}
\end{array}
  }
}
\end{mathpar}

\section{SyntaxRule.Expr \label{sec:SyntaxRule.Expr}}
\hypertarget{build-expr}{}
The function
\[
  \buildexpr(\overname{\parsenode{\Nexpr}}{\vparsednode}) \;\aslto\; \overname{\expr}{\vastnode}
\]
transforms a parse node $\vparsednode$ into an AST node $\vastnode$.

\isempty{\subsection{Prose}}
\subsection{Formally}
\begin{mathpar}
\inferrule[literal]{}{
  \buildexpr(\overname{\Nexpr(\punnode{\Nvalue})}{\vparsednode}) \astarrow
  \overname{\ELiteral(\astof{\vvalue})}{\vastnode}
}
\end{mathpar}

\begin{mathpar}
  \inferrule[var]{}{
  \buildexpr(\overname{\Nexpr(\Tidentifier(\id))}{\vparsednode}) \astarrow
  \overname{\EVar(\id)}{\vastnode}
}
\end{mathpar}

\begin{mathpar}
  \inferrule[binop]{
    \buildexpr(\veone) \astarrow \astversion{\veone}\\
    \buildexpr(\vetwo) \astarrow \astversion{\vetwo}
  }{
    {
      \begin{array}{r}
  \buildexpr(\overname{\Nexpr(\namednode{\veone}{\Nexpr}, \punnode{\Nbinop}, \namednode{\vetwo}{\Nexpr})}{\vparsednode}) \astarrow\\
  \overname{\EBinop(\astversion{\veone}, \astof{\vbinop}, \astversion{\vetwo})}{\vastnode}
      \end{array}
    }
}
\end{mathpar}

\begin{mathpar}
  \inferrule[unop]{}{
  \buildexpr(\overname{\Nexpr(\punnode{\Nunop}, \punnode{\Nexpr})}{\vparsednode}) \astarrow
  \overname{\EUnop(\astof{\vunop}, \astof{\vexpr})}{\vastnode}
}
\end{mathpar}

\begin{mathpar}
  \inferrule[cond]{
    \buildexpr(\vcondexpr) \astarrow \astversion{\vcondexpr}\\
    \buildexpr(\vthenexpr) \astarrow \astversion{\vthenexpr}
  }{
    {
      \begin{array}{r}
  \buildexpr\left(\overname{\Nexpr\left(
    \begin{array}{l}
    \Tif, \namednode{\vcondexpr}{\Nexpr}, \Tthen, \\
    \wrappedline\ \namednode{\vthenexpr}{\Nexpr}, \punnode{\Neelse}
    \end{array}
    \right)}{\vparsednode}\right) \astarrow\\
  \overname{\ECond(\astversion{\vcondexpr}, \astversion{\vthenexpr}, \astof{\veelse})}{\vastnode}
      \end{array}
    }
}
\end{mathpar}

\begin{mathpar}
  \inferrule[call]{
    \buildplist[\buildexpr](\vargs) \astarrow \vexprasts
  }{
  \buildexpr(\overname{\Nexpr(\Tidentifier(\id), \namednode{\vargs}{\Plist{\Nexpr}})}{\vparsednode}) \astarrow
  \overname{\ECall(\id, \vexprasts)}{\vastnode}
}
\end{mathpar}

\begin{mathpar}
  \inferrule[slice]{}{
    {
      \begin{array}{r}
  \buildexpr(\overname{\Nexpr(\punnode{\Nexpr}, \punnode{\Nslice})}{\vparsednode}) \astarrow\\
  \overname{\ESlice(\astof{\vexpr}, \astof{\vslice})}{\vastnode}
      \end{array}
    }
}
\end{mathpar}

\begin{mathpar}
  \inferrule[get\_field]{}{
  \buildexpr(\overname{\Nexpr(\Nexpr, \Tdot, \Tidentifier)}{\vparsednode}) \astarrow
  \overname{\EGetField()}{\vastnode}
}
\end{mathpar}

\begin{mathpar}
  \inferrule[get\_fields]{
    \buildclist[\buildidentity](\vids) \astarrow \vidasts
  }{
    {
      \begin{array}{r}
  \buildexpr(\overname{\Nexpr(\punnode{\Nexpr}, \Tdot, \Tlbracket, \namednode{\vids}{\NClist{\Tidentifier}}, \Trbracket)}{\vparsednode}) \astarrow\\
  \overname{\EGetFields(\astof{\vexpr}, \vidasts)}{\vastnode}
      \end{array}
    }
}
\end{mathpar}

\begin{mathpar}
  \inferrule[concat]{
    \buildclist[\buildexpr](\vexprs) \astarrow \vexprasts
  }{
  \buildexpr(\overname{\Nexpr(\Tlbracket, \namednode{\vexprs}{\NClist{\Nexpr}}, \Trbracket)}{\vparsednode}) \astarrow
  \overname{\EConcat(\vexprasts)}{\vastnode}
}
\end{mathpar}

\begin{mathpar}
  \inferrule[atc]{}{
    {
      \begin{array}{r}
  \buildexpr(\overname{\Nexpr(\punnode{\Nexpr}, \Tas, \punnode{\Nty})}{\vparsednode}) \astarrow\\
  \overname{\EATC(\astof{\vexpr}, \astof{\tty})}{\vastnode}
      \end{array}
    }
}
\end{mathpar}

\begin{mathpar}
  \inferrule[atc\_int\_constraints]{}{
    {
      \begin{array}{r}
  \buildexpr(\overname{\Nexpr(\punnode{\Nexpr}, \Tas, \punnode{\Nintconstraints})}{\vparsednode}) \astarrow\\
  \overname{\EATC(\astof{\vexpr}, \TInt(\astof{\vintconstraints}))}{\vastnode}
\end{array}
}
}
\end{mathpar}

\begin{mathpar}
  \inferrule[pattern\_set]{}{
    {
      \begin{array}{r}
  \buildexpr(\overname{\Nexpr(\punnode{\Nexpr}, \Tin, \punnode{\Npatternset})}{\vparsednode}) \astarrow\\
  \overname{\EPattern(\astof{\vexpr}, \astof{\vpatternset})}{\vastnode}
\end{array}
}
}
\end{mathpar}

\begin{mathpar}
  \inferrule[pattern\_mask]{}{
    {
      \begin{array}{r}
  \buildexpr(\overname{\Nexpr(\punnode{\Nexpr}, \Tin, \Tmasklit(\vm))}{\vparsednode}) \astarrow\\
  \overname{\EPattern(\astof{\vexpr}, \vm)}{\vastnode}
\end{array}
}
}
\end{mathpar}

\begin{mathpar}
  \inferrule[unknown]{}{
  \buildexpr(\overname{\Nexpr(\Tunknown, \Tcolon, \punnode{\Nty})}{\vparsednode}) \astarrow
  \overname{\EUnknown(\astof{\tty})}{\vastnode}
}
\end{mathpar}

\begin{mathpar}
  \inferrule[record]{
    \buildclist[\buildfieldassign](\vfieldassigns) \astarrow \vfieldassignasts
  }{
    {
      \begin{array}{r}
  \buildexpr\left(\overname{\Nexpr\left(
    \begin{array}{l}
    \Tidentifier(\vt), \Tlbrace, \\
    \wrappedline\ \namednode{\vfieldassigns}{\Clist{\Nfieldassign}}, \\
    \wrappedline\ \Trbrace
    \end{array}
    \right)}{\vparsednode}\right) \\
    \astarrow\ \overname{\ERecord(\TNamed(\vt), \vfieldassignasts)}{\vastnode}
\end{array}
}
}
\end{mathpar}

\begin{mathpar}
  \inferrule[sub\_expr]{}{
  \buildexpr(\overname{\Nexpr(\Tlpar, \punnode{\Nexpr}, \Trpar)}{\vparsednode}) \astarrow
  \overname{\astof{\vexpr}}{\vastnode}
}
\end{mathpar}

\begin{mathpar}
  \inferrule[tuple]{
    \buildplist[\buildexpr](\vexprs) \astarrow \vexprasts
  }{
  \buildexpr(\overname{\Nexpr(\namednode{\vexprs}{\Plisttwo{\Nexpr}})}{\vparsednode}) \astarrow
  \overname{\ETuple(\vexprasts)}{\vastnode}
}
\end{mathpar}

\section{SyntaxRule.Value \label{sec:SyntaxRule.Value}}
\hypertarget{build-value}{}
The function
\[
  \buildvalue(\overname{\parsenode{\Nvalue}}{\vparsednode}) \;\aslto\; \overname{\literal}{\vastnode}
\]
transforms a parse node $\vparsednode$ into an AST node $\vastnode$.

\isempty{\subsection{Prose}}
\subsection{Formally}
\begin{mathpar}
\inferrule[integer]{}{
  \buildvalue(\Nvalue(\Tintlit(\vi))) \astarrow
  \overname{\lint(\vi)}{\vastnode}
}
\end{mathpar}

\begin{mathpar}
\inferrule[boolean]{}{
  \buildvalue(\Nvalue(\Tboollit(\vb))) \astarrow
  \overname{\lbool(\vb)}{\vastnode}
}
\end{mathpar}

\begin{mathpar}
\inferrule[real]{}{
  \buildvalue(\Nvalue(\Treallit(\vr))) \astarrow
  \overname{\lreal(\vr)}{\vastnode}
}
\end{mathpar}

\begin{mathpar}
\inferrule[bitvector]{}{
  \buildvalue(\Nvalue(\Tbitvectorlit(\vb))) \astarrow
  \overname{\lbitvector(\vb)}{\vastnode}
}
\end{mathpar}

\begin{mathpar}
\inferrule[string]{}{
  \buildvalue(\Nvalue(\Tstringlit(\vs))) \astarrow
  \overname{\lstring(\vs)}{\vastnode}
}
\end{mathpar}

\section{SyntaxRule.Unop \label{sec:SyntaxRule.Unop}}
\hypertarget{build-unop}{}
The function
\[
  \buildunop(\overname{\parsenode{\Nunop}}{\vparsednode}) \;\aslto\; \overname{\unop}{\vastnode}
\]
transforms a parse node $\vparsednode$ into an AST node $\vastnode$.

\isempty{\subsection{Prose}}
\subsection{Formally}
\begin{mathpar}
\inferrule[bnot]{}{
  \buildunop(\Nunop(\Tbnot)) \astarrow \overname{\BNOT}{\vastnode}
}
\end{mathpar}

\begin{mathpar}
\inferrule[neg]{}{
  \buildunop(\Nunop(\Tminus)) \astarrow \overname{\NEG}{\vastnode}
}
\end{mathpar}

\begin{mathpar}
\inferrule[not]{}{
  \buildunop(\Nunop(\Tnot)) \astarrow \overname{\NOT}{\vastnode}
}
\end{mathpar}

\section{SyntaxRule.Binop \label{sec:SyntaxRule.Binop}}
\hypertarget{build-binop}{}
The function
\[
  \buildbinop(\overname{\parsenode{\Nbinop}}{\vparsednode}) \;\aslto\; \overname{\binop}{\vastnode}
\]
transforms a parse node $\vparsednode$ into an AST node $\vastnode$.

\isempty{\subsection{Prose}}
\subsection{Formally}
\begin{mathpar}
\inferrule[]{}{
  \buildunop(\Nbinop(\Tand)) \astarrow \overname{\AND}{\vastnode}
}
\end{mathpar}

\begin{mathpar}
\inferrule[]{}{
  \buildunop(\Nbinop(\Tband)) \astarrow \overname{\BAND}{\vastnode}
}
\end{mathpar}

\begin{mathpar}
\inferrule[]{}{
  \buildunop(\Nbinop(\Tbor)) \astarrow \overname{\BOR}{\vastnode}
}
\end{mathpar}

\begin{mathpar}
\inferrule[]{}{
  \buildunop(\Nbinop(\Tbeq)) \astarrow \overname{\EQOP}{\vastnode}
}
\end{mathpar}

\begin{mathpar}
\inferrule[]{}{
  \buildunop(\Nbinop(\Tdiv)) \astarrow \overname{\DIV}{\vastnode}
}
\end{mathpar}

\begin{mathpar}
\inferrule[]{}{
  \buildunop(\Nbinop(\Tdivrm)) \astarrow \overname{\DIVRM}{\vastnode}
}
\end{mathpar}

\begin{mathpar}
\inferrule[]{}{
  \buildunop(\Nbinop(\Txor)) \astarrow \overname{\XOR}{\vastnode}
}
\end{mathpar}

\begin{mathpar}
\inferrule[]{}{
  \buildunop(\Nbinop(\Teqop)) \astarrow \overname{\EQOP}{\vastnode}
}
\end{mathpar}

\begin{mathpar}
\inferrule[]{}{
  \buildunop(\Nbinop(\Tneq)) \astarrow \overname{\NEQ}{\vastnode}
}
\end{mathpar}

\begin{mathpar}
\inferrule[]{}{
  \buildunop(\Nbinop(\Tgt)) \astarrow \overname{\GT}{\vastnode}
}
\end{mathpar}

\begin{mathpar}
\inferrule[]{}{
  \buildunop(\Nbinop(\Tgeq)) \astarrow \overname{\GEQ}{\vastnode}
}
\end{mathpar}

\begin{mathpar}
\inferrule[]{}{
  \buildunop(\Nbinop(\Timpl)) \astarrow \overname{\IMPL}{\vastnode}
}
\end{mathpar}

\begin{mathpar}
\inferrule[]{}{
  \buildunop(\Nbinop(\Tlt)) \astarrow \overname{\LT}{\vastnode}
}
\end{mathpar}

\begin{mathpar}
\inferrule[]{}{
  \buildunop(\Nbinop(\Tleq)) \astarrow \overname{\LEQ}{\vastnode}
}
\end{mathpar}

\begin{mathpar}
\inferrule[]{}{
  \buildunop(\Nbinop(\Tplus)) \astarrow \overname{\PLUS}{\vastnode}
}
\end{mathpar}

\begin{mathpar}
\inferrule[]{}{
  \buildunop(\Nbinop(\Tminus)) \astarrow \overname{\MINUS}{\vastnode}
}
\end{mathpar}

\begin{mathpar}
\inferrule[]{}{
  \buildunop(\Nbinop(\Tmod)) \astarrow \overname{\MOD}{\vastnode}
}
\end{mathpar}

\begin{mathpar}
\inferrule[]{}{
  \buildunop(\Nbinop(\Tmul)) \astarrow \overname{\MUL}{\vastnode}
}
\end{mathpar}

\begin{mathpar}
\inferrule[]{}{
  \buildunop(\Nbinop(\Tor)) \astarrow \overname{\OR}{\vastnode}
}
\end{mathpar}

\begin{mathpar}
\inferrule[]{}{
  \buildunop(\Nbinop(\Trdiv)) \astarrow \overname{\RDIV}{\vastnode}
}
\end{mathpar}

\begin{mathpar}
\inferrule[]{}{
  \buildunop(\Nbinop(\Tshl)) \astarrow \overname{\SHL}{\vastnode}
}
\end{mathpar}

\begin{mathpar}
\inferrule[]{}{
  \buildunop(\Nbinop(\Tshr)) \astarrow \overname{\SHR}{\vastnode}
}
\end{mathpar}

\begin{mathpar}
\inferrule[]{}{
  \buildunop(\Nbinop(\Tpow)) \astarrow \overname{\POW}{\vastnode}
}
\end{mathpar}

\begin{mathpar}
\inferrule[]{}{
  \buildunop(\Nbinop(\Tconcat)) \astarrow \overname{\tododefine{CONCAT}}{\vastnode}
}
\end{mathpar}

\section{SyntaxRule.StmtFromList \label{sec:SyntaxRule.StmtFromList}}
\hypertarget{def-stmtfromlist}{}
The function
\[
\stmtfromlist(\overname{\stmt^*}{\vstmts}) \aslto \overname{\stmt}{\news}
\]
builds a statement $\news$ from a possibly-empty list of statements $\vstmts$.

\isempty{\subsection{Prose}}
\subsection{Formally}
\begin{mathpar}
\inferrule[empty]{
}{
  \stmtfromlist(\overname{\emptylist}{\vstmts}) \astarrow \overname{\SPass}{\news}
}
\and
\inferrule[non\_empty]{
  \stmtfromlist(\vstmtsone) \astarrow \vsone\\
  \sequencestmts(\vs, \vsone) \astarrow \news
}{
  \stmtfromlist(\overname{[\vs] \concat \vstmtsone}{\vstmts}) \astarrow \news
}
\end{mathpar}

\section{SyntaxRule.SequenceStmts \label{sec:SyntaxRule.SequenceStmts}}
\hypertarget{def-sequencestmts}{}
The function
\[
\sequencestmts(\overname{\stmt}{\vsone}, \overname{\stmt}{\vstwo}) \aslto \overname{\stmt}{\news}
\]
Combines the statement $\vsone$ with $\vstwo$ into the statement $\news$, while filtering away
instances of $\SPass$.

\isempty{\subsection{Prose}}
\subsection{Formally}
\begin{mathpar}
\inferrule[s1\_spass]{}{
  \sequencestmts(\overname{\SPass}{\vsone}, \vstwo) \astarrow \overname{\vstwo}{\news}
}
\and
\inferrule[s2\_spass]{
  \vsone \neq \SPass
}{
  \sequencestmts(\vsone, \overname{\SPass}{\vstwo}) \astarrow \overname{\vsone}{\news}
}
\and
\inferrule[no\_spass]{
  \vsone \neq \SPass\\
  \vstwo \neq \SPass
}{
  \sequencestmts(\vsone, \vstwo) \astarrow \overname{\SSeq(\vsone, \vstwo)}{\news}
}
\end{mathpar}

%%%%%%%%%%%%%%%%%%%%%%%%%%%%%%%%%%%%%%%%%%%%%%%%%%%%%%%%%%%%%%%%%%%%%%%%%%%%%%%%
\chapter{Building Macro Productions}
%%%%%%%%%%%%%%%%%%%%%%%%%%%%%%%%%%%%%%%%%%%%%%%%%%%%%%%%%%%%%%%%%%%%%%%%%%%%%%%%
This chapter defines builder relations for the
subset of macro productions in \secref{ParametricProductions}
that are not inlined:
\begin{itemize}
  \item SyntaxRule.List (see \secref{SyntaxRule.List})
  \item SyntaxRule.CList (see \secref{SyntaxRule.CList})
  \item SyntaxRule.NTCList (see \secref{SyntaxRule.NTCList})
  \item SyntaxRule.Option (see \secref{SyntaxRule.Option})
\end{itemize}

We also define SyntaxRule.Identity (see \secref{SyntaxRule.Identity}),
which can be used in conjunction with the rules above in application
to terminals.

\section{SyntaxRule.List \label{sec:SyntaxRule.List}}
\hypertarget{build-list}{}
The meta relation
\[
\buildlist[b](\overname{N}{\vsyms}) \;\aslrel\; \overname{A}{\vsymasts}
\]
which is parameterized by an AST building relation $b : E \aslrel A$,
takes a parse node that represents a possibly-empty list of $E$ values --- $\vsyms$ --- and returns the result of applying $b$
to each of them --- $\vsymasts$.

\subsection{Formally}
\begin{mathpar}
\inferrule[empty]{}{
  \buildlist[b](\overname{\emptysentence}{\vsyms}) \astarrow \overname{\emptylist}{\vsymasts}
}
\end{mathpar}

\begin{mathpar}
\inferrule[non\_empty]{
  b(\vv) \astarrow \astversion{\vv}\\
  \buildlist[b](\vsymsone) \astarrow \vsymastsone
}{
  \buildlist[b](\overname{
    \maybeemptylist{N}(\namednode{\vv}{E}, \namednode{\vsymsone}{\maybeemptylist{N}})
    }{\vsyms}) \astarrow
  \overname{[\astversion{\vv}] \concat \vsymastsone}{\vsymasts}
}
\end{mathpar}

\section{SyntaxRule.CList \label{sec:SyntaxRule.CList}}
\hypertarget{build-clist}{}
The meta relation
\[
\buildclist[b](\overname{N}{\vsyms}) \;\aslrel\; \overname{A}{\vsymasts}
\]
which is parameterized by an AST building relation $b : E \aslrel A$,
takes a parse node that represents a possibly-empty comma-separated list of $E$ values --- $\vsyms$ --- and returns the result of applying $b$
to each of them --- $\vsymasts$.

\subsection{Formally}
\begin{mathpar}
\inferrule[empty]{}{
  \buildclist[b](\overname{\emptysentence}{\vsyms}) \astarrow \overname{\emptylist}{\vsymasts}
}
\end{mathpar}

\begin{mathpar}
\inferrule[non\_empty]{
  b(\vv) \astarrow \astversion{\vv}\\
  \buildclist[b](\vsymsone) \astarrow \vsymastsone
}{
  \buildclist[b](\overname{
    \Clist{N}(\namednode{\vv}{E}, \Tcomma, \namednode{\vsymsone}{\NClist{N}})
    }{\vsyms}) \astarrow \overname{[\astversion{\vv}] \concat \vsymastsone}{\vsymasts}
}
\end{mathpar}

\section{SyntaxRule.NTCList \label{sec:SyntaxRule.NTCList}}
\hypertarget{build-ntclist}{}
The meta relation
\[
\buildntclist[b](\overname{N}{\vsyms}) \;\aslrel\; \overname{A}{\vsymasts}
\]
which is parameterized by an AST building relation $b : E \aslrel A$,
takes a parse node that represents a non-empty comma-separated trailing list of $E$ values --- $\vsyms$ --- and returns the result of applying $b$
to each of them --- $\vsymasts$.

\begin{mathpar}
\inferrule[empty]{
  b(\vv) \astarrow \astversion{\vv}
}{
  \buildntclist[b](\overname{\vv \parsesep \option{\Tcomma}}{\vsyms}) \astarrow \overname{[\astversion{\vv}]}{\vsymasts}
}
\end{mathpar}

\begin{mathpar}
\inferrule[non\_empty]{
  b(\vv) \astarrow \astversion{\vv}\\
  \buildntclist[b](\vsymsone) \astarrow \vsymastsone
}{
  \buildntclist[b](\overname{\namednode{\vv}{E}, \Tcomma, \namednode{\vsymsone}{\NTClist{N}}}{\vsyms}) \astarrow \overname{[\astversion{\vv}] \concat \vsymastsone}{\vsymasts}
}
\end{mathpar}

\section{SyntaxRule.Option \label{sec:SyntaxRule.Option}}
\hypertarget{build-option}{}
The meta relation
\[
\buildoption[b](\overname{N}{\vsym}) \;\aslrel\; \overname{\langle A \rangle}{\vsymast}
\]
which is parameterized by an AST building relation $b : E \aslrel A$,
takes a parse node that represents an optional $E$ value --- $\vsym$ --- and returns the result of applying $b$
to the value if it exists --- $\vsymasts$.

\begin{mathpar}
\inferrule[none]{}{
  \buildoption[b](\overname{\emptysentence}{\vsym}) \astarrow \overname{\None}{\vsymast}
}
\end{mathpar}

\begin{mathpar}
\inferrule[some]{
  b(\vv) \astarrow \astversion{\vv}
}{
  \buildoption[b](\overname{\namednode{\vv}{E}}{\vsym}) \astarrow \overname{\langle\astversion{\vv}\rangle}{\vsymast}
}
\end{mathpar}

When this relation is applied to a sentence consisting of a prefix of terminals $t_{1..k}$, ending with a non-terminal $\vv$,
it ignore the terminals and returns the result for the non-terminal.
\begin{mathpar}
\inferrule[last]{
  \buildoption[b](\vv) \astarrow \vsymast
}{
  \buildoption[b](t_{1..k}, \namednode{\vv}{E}) \astarrow \vsymast
}
\end{mathpar}

\section{SyntaxRule.Identity \label{sec:SyntaxRule.Identity}}
\hypertarget{build-identity}{}
The meta function
\[
\buildidentity(\overname{T}{x}) \aslto \overname{T}{x}
\]
is the identity function, which can be used as an argument to meta functions such as $\buildlist$ when they are applied
to terminals.

\subsection{Formally}
\begin{mathpar}
\inferrule{}{
  \buildidentity(x) \astarrow x
}
\end{mathpar}

%%%%%%%%%%%%%%%%%%%%%%%%%%%%%%%%%%%%%%%%%%%%%%%%%%%%%%%%%%%%%%%%%%%%%%%%%%%%%%%%
\chapter{Correspondence Between Left-hand-side Expressions and Right-hand-side Expressions
\label{ch:LeftToRight}}
%%%%%%%%%%%%%%%%%%%%%%%%%%%%%%%%%%%%%%%%%%%%%%%%%%%%%%%%%%%%%%%%%%%%%%%%%%%%%%%%

The recursive function $\torexpr : \lexpr \rightarrow \expr$ transforms
left-hand-side expressions to corresponding right-hand-side expressions,
which is utilized both for the type system and semantics:
\[
\begin{array}{lcl}
  \textbf{Left hand side expression} & & \textbf{Right hand side expression}\\
  \hline
  \torexpr(\LEVar(\vx)) &=& \EVar(\vx)\\
  \torexpr(\LESlice(\vle, \vargs)) &=& \ESlice(\torexpr(\vle), \vargs)\\
  \torexpr(\LESetArray(\vle, \ve)) &=& \EGetArray(\torexpr(\vle), \ve)\\
  \torexpr(\LESetField(\vle, \vx)) &=& \EGetField(\torexpr(\vle), \vx)\\
  \torexpr(\LESetFields(\vle, \vx)) &=& \EGetFields(\torexpr(\vle), \vx)\\
  \torexpr(\LEDiscard) &=& \EVar(\texttt{-})\\
  \torexpr(\LEDestructuring([\vle_{1..k}])) &=& \ETuple([i=1..k: \torexpr(\vle_i)])\\
  \torexpr(\LEConcat([\vle_{1..k}], \Ignore)) &=& \EConcat([i=1..k: \torexpr(\vle_i)])\\
\end{array}
\]

\bibliographystyle{plain}
\bibliography{ASL}
\end{document}
