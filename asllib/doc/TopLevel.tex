\chapter{Top Level\label{chap:TopLevel}}
In previous chapters, we defined the following components:
\begin{itemize}
    \item Lexical analysis (\chapref{LexicalStructure}),
    \item Parsing (\chapref{Syntax}),
    \item AST building (\secref{BuildingAbstractSyntaxTrees}),
    \item Type checking (\secref{TypingSpecifications}), and
    \item Semantic evaluation (\secref{SemanticsOfSpecifications}).
\end{itemize}

In this chapter, we show how these components can be combined to form an interpreter
for an ASL standard library and a given ASL specification.
%
We emphasize that this is only an example usage of the components listed above.
One can think of other combinations where, for example, semantic evaluation is
replaced with a translation to a hardware description language.

\hypertarget{def-checkandinterpret}{}
The relation
\[
\checkandinterpret(\overname{\Strings}{\vspectext}, \overname{\Strings}{\vstdtext}) \aslrel
\left(
    \begin{array}{cc}
        (\overname{\tint}{\vret} \times \overname{\XGraphs}{\vg})   & \cup\\
        \{\ \LexicalError, \ParseError, \BuildErrorConfig\ \}    & \cup\\
        \overname{\TTypeError}{\TypeErrorConfig}    & \cup\\
        \overname{\TDynError}{\DynErrorConfig}      &
    \end{array}
\right)
\]
accepts a textual description of a specification in $\vspectext$ and a textual description of the standard library in $\vstdtext$.
The descriptions are statically checked for validity.
If found invalid, an error configuration corresponding to the phase where the error
exists is returned --- scanning, parsing, or AST building.
If found valid, an AST is built and semantically evaluated, yielding an integer return code $\vret$ and an execution graph $\vg$,
or a dynamic error.

\subsubsection{TopLevelRule.CheckAndInterpret}
\subsubsection{Prose}
All of the following apply:
\begin{itemize}
    \item \Proseaslscan{$\vstdtext$}{$\spectoken$}{$\vstdtokens$\ProseTerminateAs{\LexicalError}};
    \item \Proseaslparse{$\vstdtokens$}{$\vstdparse$\ProseTerminateAs{\ParseError}};
    \item \Prosebuildast{$\vstdparse$}{$\vstdast$\ProseTerminateAs{\BuildErrorConfig}};
    \item define $\vstdasbuiltin$ by applying $\setbuiltin$ to each top-level declaration in\\
          $\vstdast$\ProseTerminateAs{\BuildErrorConfig};
    %%%%%
    \item \Proseaslscan{$\vspectext$}{$\spectoken$}{$\vspectokens$\ProseTerminateAs{\LexicalError}};
    \item \Proseaslparse{$\vspectokens$}{$\vspecparse$\ProseTerminateAs{\ParseError}};
    \item \Prosebuildast{$\vspecparse$}{$\vspecast$\ProseTerminateAs{\BuildErrorConfig}};
    %%%%%
    \item define $\vuntypedast$ as the concatenation of the AST $\vstdasbuiltin$ and
          the AST $\vspecast$ (both are lists of $\decl$);
    \item \Prosetypecheckast{$\vuntypedast$}{empty static global environment}{$\vtypedast$}{$\tenv$\ProseOrTypeError};
    \item \Proseevalspec{$\vtypedast$}{$\tenv$}{the integer $\vret$ and the execution graph $\vg$\ProseOrError}.
\end{itemize}

\subsubsection{Formally}
\begin{mathpar}
\inferrule{
    \aslscan(\spectoken, \vstdtext) \scanarrow \vstdtokens \terminateas \LexicalError\\\\
    \aslparse(\vstdtokens) \parsearrow \vstdparse \terminateas \ParseError\\\\
    \buildast(\vstdparse) \astarrow \vstdast \terminateas \BuildErrorConfig\\\\
    \vi \in \listrange(\vstdast): \setbuiltin(\vstdast_i) \typearrow \vstddeclast_i \terminateas \BuildErrorConfig\\\\
    \vstdasbuiltin \eqdef [\vi \in \listrange(\vstdast): \vstddeclast_i]\\
    \aslscan(\spectoken, \vspectext) \scanarrow \vspectokens \terminateas \LexicalError\\\\
    \aslparse(\vspectokens) \parsearrow \vspecparse \terminateas \ParseError\\\\
    \buildast(\vspecparse) \astarrow \vspecast \terminateas \BuildErrorConfig\\\\
    \vuntypedast \eqdef \vstdasbuiltin \concat \vspecast\\
    \typecheckast(G^{\emptytenv}, \vuntypedast) \typearrow (\vtypedast, \tenv) \OrTypeError\\\\
    \evalspec(\tenv, \vtypedast) \evalarrow (\nvint(\vret), \vg) \OrDynError
}{
    \checkandinterpret(\vspectext, \vstdtext) \longrightarrow (\nvint(\vret), \vg)
}
\end{mathpar}

\ASTRuleDef{SetBuiltin}
The helper function
\hypertarget{def-setbuiltin}{}
\[
  \setbuiltin(\overname{\decl}{\vdecl}) \aslto
  \overname{\decl}{\vdeclp}
  \cup \overname{\TBuildError}{\BuildErrorConfig}
\]
sets the $\funcbuiltin$ flag of a top-level function declaration, which is used to identify standard library functions in \TypingRuleRef{InsertStdlibParam}.
It produces a build error when given a top-level declaration which is not a function.

\subsubsection{Prose}
All of the following apply:
\begin{itemize}
  \item \Prosechecktrans{that $\vdecl$ is a function}{\BuiltinExpectedToBeFunction};
  \item view $\vdecl$ as $\DFunc(\funcsig)$;
  \item define $\funcsigp$ as $\funcsig$ with its $\funcbuiltin$ flag set to $\True$;
  \item define $\vdeclp$ as $\DFunc(\funcsigp)$.
\end{itemize}

\subsubsection{Formally}
\begin{mathpar}
\inferrule{
  \checktrans{\astlabel(\vdecl) = \DFunc}{\BuiltinExpectedToBeFunction} \typearrow \True \terminateas \BuildErrorConfig\\\\
  \vdecl \eqname \DFunc(\funcsig)\\
  \funcsigp \eqdef \funcsig[\funcbuiltin \mapsto \True]
}{
  \setbuiltin(\vdecl) \astarrow \DFunc(\funcsigp)
}
\end{mathpar}
