\chapter{Top Level\label{chap:TopLevel}}
In previous chapters, we defined the following components:
\begin{itemize}
    \item Lexical analysis (\chapref{LexicalStructure}),
    \item Parsing (\chapref{Syntax}),
    \item AST building (\secref{BuildingAbstractSyntaxTrees}),
    \item Type checking (\secref{TypingSpecifications}), and
    \item Semantic evaluation (\secref{SemanticsOfSpecifications}).
\end{itemize}

In this chapter, we show how these components can be combined to form an interpreter
for an ASL standard library and a given ASL specification.
%
We emphasize that this is only an example usage of the components listed above.
One can think of other combinations where, for example, semantic evaluation is
replaced with a translation to a hardware description language.

\section{An Example Interpreter}

\hypertarget{def-checkandinterpret}{}
The relation
\[
\checkandinterpret(\overname{\Strings}{\vspectext}, \overname{\Strings}{\vstdtext}) \aslrel
\left(
    \begin{array}{cc}
        (\overname{\tint}{\vret} \times \overname{\XGraphs}{\vg})   & \cup\\
        \{\ \LexicalError, \ParseError, \BuildErrorConfig\ \}    & \cup\\
        \overname{\TTypeError}{\TypeErrorConfig}    & \cup\\
        \overname{\TDynError}{\DynErrorConfig}      &
    \end{array}
\right)
\]
accepts a textual description of a specification in $\vspectext$ and a textual description of the standard library in $\vstdtext$.
The descriptions are statically checked for validity.
If found invalid, an error configuration corresponding to the phase where the error
exists is returned --- scanning, parsing, or AST building.
If found valid, an AST is built and semantically evaluated, yielding an integer return code $\vret$ and an execution graph $\vg$,
or a dynamic error.

\subsubsection{TopLevelRule.CheckAndInterpret}
\subsubsection{Prose}
All of the following apply:
\begin{itemize}
    \item \Proseaslscan{$\vstdtext$}{$\spectoken$}{$\vstdtokens$\ProseTerminateAs{\LexicalError}};
    \item \Proseaslparse{$\vstdtokens$}{$\vstdparse$\ProseTerminateAs{\ParseError}};
    \item \Prosebuildast{$\vstdparse$}{$\vstdast$\ProseTerminateAs{\BuildErrorConfig}};
    \item \Proserenamelocals{\vstdast}{\vstdastrenamed};
    \item define $\vstdasbuiltin$ by applying $\setbuiltin$ to each top-level declaration in\\
          $\vstdastrenamed$\ProseTerminateAs{\BuildErrorConfig};
    %%%%%
    \item \Proseaslscan{$\vspectext$}{$\spectoken$}{$\vspectokens$\ProseTerminateAs{\LexicalError}};
    \item \Proseaslparse{$\vspectokens$}{$\vspecparse$\ProseTerminateAs{\ParseError}};
    \item \Prosebuildast{$\vspecparse$}{$\vspecast$\ProseTerminateAs{\BuildErrorConfig}};
    %%%%%
    \item define $\vuntypedast$ as the concatenation of the AST $\vstdasbuiltin$ and
          the AST $\vspecast$ (both are lists of $\decl$);
    \item \Prosetypecheckast{$\vuntypedast$}{empty static global environment}{$\vtypedast$}{$\tenv$\ProseOrTypeError};
    \item \Proseevalspec{$\vtypedast$}{$\tenv$}{the integer $\vret$ and the execution graph $\vg$\ProseOrError}.
\end{itemize}

\subsubsection{Formally}
\begin{mathpar}
\inferrule{
    \aslscan(\spectoken, \vstdtext) \scanarrow \vstdtokens \terminateas \LexicalError\\\\
    \aslparse(\vstdtokens) \parsearrow \vstdparse \terminateas \ParseError\\\\
    \buildast(\vstdparse) \astarrow \vstdast \terminateas \BuildErrorConfig\\\\
    \renamelocals(\vstdast) \astarrow \vstdastrenamed\\
    {
    \begin{array}{r}
      \vi \in \listrange(\vstdastrenamed): \setbuiltin(\vstdastrenamed[\vi]) \typearrow \\
      \vstddeclast_i \terminateas \BuildErrorConfig
    \end{array}
    }\\
    \vstdasbuiltin \eqdef [\vi \in \listrange(\vstdast): \vstddeclast_i]\\
    \aslscan(\spectoken, \vspectext) \scanarrow \vspectokens \terminateas \LexicalError\\\\
    \aslparse(\vspectokens) \parsearrow \vspecparse \terminateas \ParseError\\\\
    \buildast(\vspecparse) \astarrow \vspecast \terminateas \BuildErrorConfig\\\\
    \vuntypedast \eqdef \vstdasbuiltin \concat \vspecast\\
    \typecheckast(G^{\emptytenv}, \vuntypedast) \typearrow (\vtypedast, \tenv) \OrTypeError\\\\
    \evalspec(\tenv, \vtypedast) \evalarrow (\nvint(\vret), \vg) \OrDynError
}{
    \checkandinterpret(\vspectext, \vstdtext) \longrightarrow (\nvint(\vret), \vg)
}
\end{mathpar}

%%%%%%%%%%%%%%%%%%%%%%%%%%%%%%%%%%%%%%%%%%%%%%%%%%%%%%%%%%%%%%%%%%%%%%%%%%%%%%%%
%% Beginning of rename locals-related functions
%%%%%%%%%%%%%%%%%%%%%%%%%%%%%%%%%%%%%%%%%%%%%%%%%%%%%%%%%%%%%%%%%%%%%%%%%%%%%%%%

\section{Renaming Local Storage Elements in the Standard Library}
% Transliteration comment: this transformation only applies to untyped AST node types.

In order to combine the standard library declarations with a given specification,
we need to avoid name clashes between the identifiers for local storage elements
in the standard library and the identifiers used for global declarations in the
specification.
%
This is done by renaming the identifiers corresponding to local storage elements
in the standard library. Specifically, we prefix these identifiers with the
string \verb|__stdlib_local_|.
%
The rest of this section consists of functions that recursively transform an
untyped AST, renaming local storage elements accordingly.

\ASTRuleDef{RenameLocals}
\hypertarget{def-renamelocals}{}
The helper function
\[
\renamelocals(\overname{\decl^*}{\vdecls}) \aslto \overname{\decl^*}{\vnewdecls}
\]
renames the local storage elements appearing in $\vdecls$,
yielding the list of declarations $\vnewdecls$.

\ProseParagraph
\AllApply:
\begin{itemize}
  \item \Proserenamelocalsdecl{$\vdecls[\vi]$}{$\vnewdecl_\vi$},
        for each \Proselistrange{$\vi$}{$\vdecls$};
  \item \Proseeqdef{$\vnewdecls$}{the list of declarations $\vnewdecl_\vi$},
        for each \Proselistrange{$\vi$}{$\vdecls$}.
\end{itemize}

\FormallyParagraph
\begin{mathpar}
\inferrule{
  \vi\in\listrange(\vdecls): \renamelocalsdecl(\vdecls[\vi]) \astarrow \vnewdecl_\vi
}{
  \renamelocals(\vdecls) \astarrow \overname{[\vi\in\listrange(\vdecls): \vnewdecl_\vi]}{\vnewdecls}
}
\end{mathpar}
\CodeSubsection{\RenameLocalsBegin}{\RenameLocalsEnd}{../ASTUtils.ml}

\ASTRuleDef{RenameLocalsDecl}
\hypertarget{def-renamelocalsdecl}{}
The helper function
\[
\renamelocalsdecl(\overname{\decl}{\vdecl}) \aslto \overname{\decl^*}{\vnewdecl}
\]
renames the local storage elements appearing in $\vdecl$,
yielding the declaration $\vnewdecl$.

\ProseParagraph
\OneApplies
\begin{itemize}
  \item \AllApplyCase{subprogram}
  \begin{itemize}
    \item $\vf$ is a subprogram declaration with description $\vf$, that is, $\DFunc(\vf)$;
    \item \Proserenamelocalsfunc{$\vf$}{$\vnewf$};
    \item \Proseeqdef{$\vnewdecl$}{the subprogram declaration for $\vnewf$, that is, \\
          $\DFunc(\vnewf)$}.
  \end{itemize}

  \item \AllApplyCase{other}
  \begin{itemize}
    \item $\vf$ is not a subprogram declaration;
    \item \Proseeqdef{$\vnewdecl$}{$\vdecl$}.
  \end{itemize}
\end{itemize}

\FormallyParagraph
\begin{mathpar}
\inferrule[subprogram]{
  \renamelocalsfunc(\vf) \astarrow \vnewf
}{
  \renamelocalsdecl(\overname{\DFunc(\vf)}{\vdecl}) \astarrow \overname{\DFunc(\vnewf)}{\vnewdecl}
}
\end{mathpar}

\begin{mathpar}
\inferrule[other]{
  \astlabel(\vdecl) \neq \DFunc
}{
  \renamelocalsdecl(\vdecl) \astarrow \overname{\vdecl}{\vnewdecl}
}
\end{mathpar}

\ASTRuleDef{RenameLocalsFunc}
\hypertarget{def-renamelocalsfunc}{}
The helper function
\[
\renamelocalsfunc(\overname{\func}{\vf}) \aslto \overname{\func}{\vnewf}
\]
renames the local storage elements appearing in the subprogram description $\vf$,
yielding the subprogram description $\vnewf$.

\ProseParagraph
\AllApply
\begin{itemize}
  \item view $\vf$ as a subprogram description consisting of a subprogram named $\name$,
        list of parameters $\params$, list of arguments $\args$, body $\vbody$,
        optional return type $\rettyopt$, optional subtype $\vsubtype$,
        optional limit expression $\vlimitexpr$, and builtin flag $\vbuiltin$;
  \item \Proserenamelocalsargs{$\args$}{$\newargs$};
  \item \Proserenamelocalsnamedargs{$\params$}{$\newparams$};
  \item \Proserenamelocalsstmt{$\vbody$}{$\vnewbody$};
  \item \Prosemapopt{$\renamelocalsty$}{$\rettyopt$}{\\ $\newrettyopt$};
  \item define $\vnewf$ as the subprogram description consisting of a subprogram named $\name$,
    list of parameters $\newparams$, list of arguments $\newargs$, body $\vnewbody$,
    optional return type $\newrettyopt$, optional subtype $\vsubtype$,
    optional limit expression $\vlimitexpr$, and builtin flag $\vbuiltin$.
\end{itemize}

\FormallyParagraph
\begin{mathpar}
\inferrule{
  {
  \vf \eqname
  {
  \left\{
    \begin{array}{rcl}
   \funcname &:& \name, \\
   \funcparameters &:& \params,\\
   \funcargs &:& \args,\\
   \funcbody &:& \vbody,\\
   \funcreturntype &:& \rettyopt,\\
   \funcsubprogramtype &:& \vsubtype\\
   \funcrecurselimit    &:& \vlimitexpr\\
   \funcbuiltin &:& \vbuiltin
  \end{array}
  \right\}
  }
  }\\
  \renamelocalsargs(\args) \astarrow \newargs\\
  \renamelocalsnamedargs(\params) \astarrow \newparams\\
  \renamelocalsstmt(\vbody) \astarrow \vnewbody\\
  \mapopt{\renamelocalsty}(\rettyopt) \astarrow \newrettyopt\\
  {
    \vnewf \eqdef
    {
    \left\{
      \begin{array}{rcl}
     \funcname &:& \name, \\
     \funcparameters &:& \newparams,\\
     \funcargs &:& \newargs,\\
     \funcbody &:& \vnewbody,\\
     \funcreturntype &:& \newrettyopt,\\
     \funcsubprogramtype &:& \vsubtype\\
     \funcrecurselimit    &:& \vlimitexpr\\
     \funcbuiltin &:& \vbuiltin
    \end{array}
    \right\}
    }
  }
}{
  \renamelocalsfunc(\vf) \astarrow \vnewf
}
\end{mathpar}

\ASTRuleDef{RenameLocalsArgs}
\hypertarget{def-renamelocalsargs}{}
The helper function
\[
\renamelocalsargs(\overname{(\identifier\times\ty)^*}{\args}) \aslto
\overname{(\identifier\times\ty)^*}{\newargs}
\]
renames the local storage elements appearing in the list of arguments
$\args$, yielding the list of arguments $\newargs$.

\ProseParagraph
\OneApplies
\begin{itemize}
  \item \AllApplyCase{empty}
  \begin{itemize}
    \item $\args$ is the empty list;
    \item \Proseeqdef{$\newargs$}{the empty list}.
  \end{itemize}

  \item \AllApplyCase{non\_empty}
  \begin{itemize}
    \item $\args$ is the list with \head\ $(\name,\vt)$ and \tail\ $\vargsone$;
    \item \Proserenamelocalsname{$\name$}{$\namep$};
    \item \Proserenamelocalsty{$\vt$}{$\vtp$};
    \item \Proserenamelocalsargs{$\vargsone$}{$\vargsonep$};
    \item \Proseeqdef{$\newargs$}{the list with headp $(\namep,\vtp)$ and \tail\ $\vargsonep$}.
  \end{itemize}
\end{itemize}

\FormallyParagraph
\begin{mathpar}
\inferrule[empty]{}{
  \renamelocalsargs(\overname{\emptylist}{\args}) \astarrow \overname{\emptylist}{\newargs}
}
\end{mathpar}

\begin{mathpar}
\inferrule[non\_empty]{
  \renamelocalsname(\name) \astarrow \namep\\
  \renamelocalsty(\vt) \astarrow \vtp\\
  \renamelocalsargs(\vargsone) \astarrow \vargsonep
}{
  \renamelocalsargs(\overname{(\name,\vt) \concat \vargsone}{\args}) \astarrow \overname{[(\namep, \vtp)] \concat \vargsonep}{\newargs}
}
\end{mathpar}

\ASTRuleDef{RenameLocalsNamedArgs}
\hypertarget{def-renamelocalsnamedargs}{}
The helper function
\[
\renamelocalsnamedargs(\overname{(\identifier\times\langle\ty\rangle)^*}{\params}) \aslto
\overname{(\identifier\times\langle\ty\rangle)^*}{\newparams}
\]
renames the local storage elements appearing in the list of parameters $\params$,
yielding the list of parameters $\newparams$.

\ProseParagraph
\OneApplies
\begin{itemize}
  \item \AllApplyCase{empty}
  \begin{itemize}
    \item $\params$ is the empty list;
    \item \Proseeqdef{$\newparams$}{the empty list}.
  \end{itemize}

  \item \AllApplyCase{non\_empty}
  \begin{itemize}
    \item $\newparams$ is the list with \head\ $(\name,\tyopt)$ and \tail\ $\paramsone$;
    \item \Proserenamelocalsname{$\name$}{$\namep$};
    \item \Prosemapopt{$\renamelocalsty$}{$\tyopt$}{$\tyoptp$};
    \item \Proserenamelocalsnamedargs{$\paramsone$}{$\paramsonep$};
    \item \Proseeqdef{$\newargs$}{the list with headp $(\namep,\tyoptp)$ and \tail\ $\paramsonep$}.
  \end{itemize}
\end{itemize}

\FormallyParagraph
\begin{mathpar}
\inferrule[empty]{}{
  \renamelocalsnamedargs(\overname{\emptylist}{\params}) \astarrow \overname{\emptylist}{\newparams}
}
\end{mathpar}

\begin{mathpar}
\inferrule[non\_empty]{
  \renamelocalsname(\name) \astarrow \namep\\
  \mapopt{\renamelocalsty}(\tyopt) \astarrow \tyoptp\\
  \renamelocalsnamedargs(\paramsone) \astarrow \paramsonep
}{
  \renamelocalsnamedargs(\overname{(\name,\tyopt) \concat \vargsone}{\params}) \astarrow \overname{[(\namep, \tyoptp)] \concat \paramsonep}{\newparams}
}
\end{mathpar}

\ASTRuleDef{RenameLocalsTy}
\hypertarget{def-renamelocalsty}{}
The helper function
\[
\renamelocalsty(\overname{\ty}{\tty}) \aslto \overname{\ty}{\newty}
\]
renames the local storage elements appearing in the type $\tty$,
yielding the type $\newty$.

\ProseParagraph
\OneApplies
\begin{itemize}
  \item \AllApplyCase{unchanged}
  \begin{itemize}
    \item $\tty$ is either one of the following types: the \realtypeterm, the \stringtypeterm,
          the \booleantypeterm, an \enumerationtypeterm, a \namedtypeterm,
          or $\tty$ is an \unconstrainedintegertype\ or a
          \pendingconstrainedintegertype.
    \item \Proseeqdef{$\newty$}{$\tty$}.
  \end{itemize}

  \item \AllApplyCase{t\_int\_parameterized}
  \begin{itemize}
    \item $\tty$ is the \parameterizedintegertype.
    \item this case is not implemented yet.
  \end{itemize}

  \item \AllApplyCase{t\_int\_wellconstrained}
  \begin{itemize}
    \item $\tty$ is the \wellconstrainedintegertype\ with list of constraints $\vcs$.
    \item \Proseeqdef{$\newcs$}{the list obtained by applying $\renamelocalsconstraint$ to every constraint in $\vcs$};
    \item \Proseeqdef{$\newty$}{the \wellconstrainedintegertype\ with list of constraints $\newcs$}.
  \end{itemize}

  \item \AllApplyCase{t\_bits}
  \begin{itemize}
    \item $\tty$ is the \bitvectortypeterm\ with width expression $\ve$ and bitfields $\bitfields$;
    \item \Proserenamelocalsexpr{$\ve$}{$\vep$};
    \item \Proseeqdef{$\newty$}{the \bitvectortypeterm\ with width expression $\vep$ bitfields\\ $\bitfields$}.
  \end{itemize}

  \item \AllApplyCase{t\_tuple}
  \begin{itemize}
    \item $\tty$ is the \tupletypeterm\ for the list of types $\vli$;
    \item \Proserenamelocalsexpr{$\newli$}{the list obtained by applying $\renamelocalsty$ to each type in $\vli$};
    \item \Proseeqdef{$\newty$}{the \tupletypeterm\ for the list of types $\newli$}.
  \end{itemize}

  \item \AllApplyCase{tarray}
  \begin{itemize}
    \item $\tty$ is the \arraytypeterm;
    \item this case is not implemented yet
  \end{itemize}

  \item \AllApplyCase{structured}
  \begin{itemize}
    \item $\tty$ is \structuredtype\ with AST label $L$ list of pairs consisting of identifiers and types $\vli$;
    \item \Proseeqdef{$\newli$}{the list of pairs $(\name, \renamelocalsty(\vt))$ for each pair
          $(\name, \vt)$ in $\vli$}
    \item \Proseeqdef{$\newty$}{the \structuredtype\ with AST label $L$ and $\newli$}.
  \end{itemize}
\end{itemize}

\FormallyParagraph
\begin{mathpar}
\inferrule[unchanged]{
  {
  \begin{array}{c}
    \astlabel(\tty) \in \{\TReal, \TString, \TBool, \TEnum, \TNamed\}\ \lor\\
    (\tty = \TInt(\vc) \land \astlabel(\vc) \in \{\unconstrained, \pendingconstrained\})
  \end{array}
  }
}{
  \renamelocalsty(\tty) \astarrow \overname{\tty}{\newty}
}
\end{mathpar}

\begin{mathpar}
\inferrule[t\_int\_parameterized]{
  \astlabel(\vc) = \parameterized
}{
  \renamelocalsty(\overname{\TInt(\vc)}{\tty}) \astarrow \tododefine{unimplemented}
}
\end{mathpar}

\begin{mathpar}
\inferrule[t\_int\_wellconstrained]{
  \newcs \eqdef [\vc\in \vcs: \renamelocalsconstraint(\vcs)]
}{
  \renamelocalsty(\overname{\TInt(\wellconstrained(\vcs))}{\tty}) \astarrow \overname{\TInt(\wellconstrained(\newcs))}{\newty}
}
\end{mathpar}

\begin{mathpar}
\inferrule[t\_bits]{
  \renamelocalsexpr(\ve) \astarrow \vep
}{
  \renamelocalsty(\overname{\TBits(\ve, \bitfields)}{\tty}) \astarrow \overname{\TBits(\vep, \bitfields)}{\newty}
}
\end{mathpar}

\begin{mathpar}
\inferrule[t\_tuple]{
  \newli \eqdef [\vt\in\vli: \renamelocalsty(\vt)]
}{
  \renamelocalsty(\overname{\TTuple(\vli)}{\tty}) \astarrow \overname{\TTuple(\newli)}{\newty}
}
\end{mathpar}

\begin{mathpar}
\inferrule[t\_array]{
  \astlabel(\tty) = \TArray
}{
  \renamelocalsty(\tty) \astarrow \tododefine{unimplemented}
}
\end{mathpar}

\begin{mathpar}
\inferrule[structured]{
  L \in \{\TRecord, \TException\}\\
  \newli \eqdef [(\name, \vt) \in \vli: (\name, \renamelocalsty(\vt))]
}{
  \renamelocalsty(\overname{L(\vli)}{\tty}) \astarrow \overname{L(\newli)}{\newty}
}
\end{mathpar}

\ASTRuleDef{RenameLocalsStmt}
\hypertarget{def-renamelocalsstmt}{}
The helper function
\[
\renamelocalsstmt(\overname{\stmt}{\vs}) \aslto \overname{\stmt}{\news}
\]
renames the local storage elements appearing in the statement $\vs$,
yielding the statement $\news$.

\ProseParagraph
\begin{itemize}
  \item \AllApplyCase{s\_pass}
  \begin{itemize}
    \item $\vs$ is a \passstatementterm;
    \item \Proseeqdef{$\news$}{$\vs$}.
  \end{itemize}

  \item \AllApplyCase{s\_seq}
  \begin{itemize}
    \item $\vs$ is a \sequencingstatementterm\ for $\vsone$ and $\vstwo$;
    \item \Proserenamelocalsstmt{$\vsone$}{$\vsonep$};
    \item \Proserenamelocalsstmt{$\vstwo$}{$\vstwop$};
    \item \Proseeqdef{$\news$}{a \sequencingstatementterm\ for $\vsonep$ and $\vstwop$}.
  \end{itemize}

  \item \AllApplyCase{s\_decl}
  \begin{itemize}
    \item $\vs$ is a \declarationstatement{$\ldk$}{$\ldi$}{$\tty$}{$\ve$};
    \item \Proserenamelocalsldi{$\ldi$}{$\ldip$};
    \item \Prosemapopt{$\renamelocalsty$}{$\tty$}{$\ttyp$};
    \item \Prosemapopt{$\renamelocalsexpr$}{$\ve$}{$\vep$};
    \item \Proseeqdef{$\news$}{a \declarationstatement{$\ldk$}{$\ldip$}{$\ttyp$}{$\vep$}}.
  \end{itemize}

  \item \AllApplyCase{s\_assign}
  \begin{itemize}
    \item $\vs$ is a \assignmentstatement{$\vle$}{$\ve$};
    \item \Proserenamelocalslexpr{$\vle$}{$\vlep$};
    \item \Proserenamelocalsexpr{$\ve$}{$\vep$};
    \item \Proseeqdef{$\news$}{an \assignmentstatement{$\vlep$}{$\vep$}}.
  \end{itemize}

  \item \AllApplyCase{s\_call}
  \begin{itemize}
    \item $\vs$ is a \callstatement{$\name$}{$\args$}{$\params$}{$\calltype$};
    \item \Proserenamelocalsargs{$\args$}{$\newargs$};
    \item \Proserenamelocalsnamedargs{$\params$}{$\newparams$};
    \item \Proseeqdef{$\news$}{\callstatement{$\name$}{$\newargs$}{$\newparams$}{$\calltype$}}.
  \end{itemize}

  \item \AllApplyCase{s\_return}
  \begin{itemize}
    \item $\vs$ is a \returnstatement{$\ve$};
    \item \Prosemapopt{$\renamelocalsexpr$}{$\ve$}{$\vep$};
    \item \Proseeqdef{$\news$}{the \returnstatement{$\vep$}}.
  \end{itemize}

  \item \AllApplyCase{s\_cond}
  \begin{itemize}
    \item $\vs$ is a \conditionalstatement{$\ve$}{$\vsone$}{$\vstwo$};
    \item \Proserenamelocalsexpr{$\ve$}{$\vep$};
    \item \Proserenamelocalsstmt{$\vsone$}{$\vsonep$};
    \item \Proserenamelocalsstmt{$\vstwo$}{$\vstwop$};
    \item \Proseeqdef{$\news$}{\conditionalstatement{$\vep$}{$\vsonep$}{$\vstwop$}}.
  \end{itemize}

  \item \AllApplyCase{s\_assert}
  \begin{itemize}
    \item $\vs$ is a \assertionstatement{$\ve$};
    \item \Proserenamelocalsexpr{$\ve$}{$\vep$};
    \item \Proseeqdef{$\news$}{\assertionstatement{$\vep$}}.
  \end{itemize}

  \item \AllApplyCase{s\_for}
  \begin{itemize}
    \item $\vs$ is a \forstatement{$\id$}{$\estart$}{$\dir$}{$\eend$}{$\vbody$}{$\velimitopt$};
    \item \Proserenamelocalsexpr{$\estart$}{$\estartp$};
    \item \Proserenamelocalsexpr{$\eend$}{$\eendp$};
    \item \Proserenamelocalsstmt{$\vbody$}{$\vbodyp$};
    \item \Prosemapopt{$\renamelocalsexpr$}{$\velimitopt$}{$\velimitoptp$};
    \item \Proseeqdef{$\news$}{\forstatement{$\id$}{\\ $\estartp$}{$\dir$}{$\eendp$}{$\vbodyp$}{$\velimitoptp$}}.
  \end{itemize}

  \item \AllApplyCase{s\_while}
  \begin{itemize}
    \item $\vs$ is a \whilestatement{$\ve$}{$\vlimit$}{$\vbody$};
    \item \Proserenamelocalsstmt{$\vbody$}{$\vbodyp$};
    \item \Proserenamelocalsexpr{$\ve$}{$\vep$};
    \item \Prosemapopt{$\renamelocalsexpr$}{$\vlimit$}{$\vlimitp$};
    \item \Proseeqdef{$\news$}{\whilestatement{$\vep$}{$\vlimitp$}{$\vbodyp$}}.
  \end{itemize}

  \item \AllApplyCase{s\_repeat}
  \begin{itemize}
    \item $\vs$ is a \repeatstatement{$\vs$}{$\ve$}{$\vlimit$};
    \item \Proserenamelocalsstmt{$\vs$}{$\vsp$};
    \item \Proserenamelocalsexpr{$\ve$}{$\vep$};
    \item \Prosemapopt{$\renamelocalsexpr$}{$\vlimit$}{$\vlimitp$};
    \item \Proseeqdef{$\news$}{\repeatstatement{$\vsp$}{$\vep$}{$\vlimitp$}}.
  \end{itemize}

  \item \AllApplyCase{s\_throw}
  \begin{itemize}
    \item $\vs$ is a \throwstatement{$\veopt$};
    \item \Prosemapopt{$\renamelocalsexpr$}{$\veopt$}{$\veoptp$};
    \item \Proseeqdef{$\news$}{\throwstatement{$\veoptp$}}.
  \end{itemize}

  \item \AllApplyCase{s\_try}
  \begin{itemize}
    \item $\vs$ is a \trystatementterm;
    \item this case is not implemented.
  \end{itemize}

  \item \AllApplyCase{s\_print}
  \begin{itemize}
    \item $\vs$ is a \printstatement{$\args$}{$\vnewline$};
    \item \Proseeqdef{$\newargs$}{the list obtained by applying $\renamelocalsexpr$ to each expression in $\args$};
    \item \Proseeqdef{$\news$}{\printstatement{$\newargs$}{$\vnewline$}}.
  \end{itemize}

  \item \AllApplyCase{s\_unreachable}
  \begin{itemize}
    \item $\vs$ is a \unreachablestatementterm;
    \item \Proseeqdef{$\news$}{\unreachablestatementterm}.
  \end{itemize}

  \item \AllApplyCase{s\_pragma}
  \begin{itemize}
    \item $\vs$ is a \pragmastatement{$\name$}{$\args$};
    \item \Proseeqdef{$\newargs$}{the list obtained by applying $\renamelocalsexpr$ to each expression in $\args$};
    \item \Proseeqdef{$\news$}{\pragmastatement{$\name$}{$\newargs$}}.
  \end{itemize}
\end{itemize}

\FormallyParagraph
\begin{mathpar}
\inferrule[s\_pass]{}{
  \renamelocalsstmt(\overname{\SPass}{\vs}) \astarrow \overname{\SPass}{\news}
}
\end{mathpar}

\begin{mathpar}
\inferrule[s\_seq]{
  \renamelocalsstmt(\vsone) \astarrow \vsonep\\
  \renamelocalsstmt(\vstwo) \astarrow \vstwop\\
}{
  \renamelocalsstmt(\overname{\SSeq(\vsone, \vstwo)}{\vs}) \astarrow \overname{\SSeq(\vsonep, \vstwop)}{\news}
}
\end{mathpar}

\begin{mathpar}
\inferrule[s\_decl]{
  \renamelocalsldi(\ldi) \astarrow \ldip\\
  \mapopt{\renamelocalsty}(\tty) \astarrow \ttyp\\
  \mapopt{\renamelocalsexpr}(\ve) \astarrow \vep
}{
  \renamelocalsstmt(\overname{\SDecl(\ldk, \ldi, \tty, \ve)}{\vs}) \astarrow \overname{\SDecl(\ldk, \ldip, \ttyp, \vep)}{\news}
}
\end{mathpar}

\begin{mathpar}
\inferrule[s\_assign]{
  \renamelocalslexpr(\vle) \astarrow \vlep\\
  \renamelocalsexpr(\ve) \astarrow \vep\\
}{
  \renamelocalsstmt(\overname{\SAssign(\vle, \ve)}{\vs}) \astarrow \overname{\SAssign(\vlep, \vep)}{\news}
}
\end{mathpar}

\begin{mathpar}
\inferrule[s\_call]{
  \renamelocalsargs(\args) \astarrow \newargs\\
  \renamelocalsnamedargs(\params) \astarrow \newparams
}{
  {
  \begin{array}{r}
    \renamelocalsstmt\left(\overname{\SCall\left(
      {
        \left\{
          \begin{array}{rcl}
         \callname &:& \name, \\
         \callparams &:& \params, \\
         \callargs &:& \args, \\
         \callcalltype &:& \calltype
        \end{array}
        \right\}
        }\right)}{\vs}\right) \astarrow \\
    \overname{\SCall\left(
      {
        \left\{
          \begin{array}{rcl}
         \callname &:& \name, \\
         \callparams &:& \newparams, \\
         \callargs &:& \newargs, \\
         \callcalltype &:& \calltype
        \end{array}
        \right\}
        }\right)}{\news}
  \end{array}
  }
}
\end{mathpar}

\begin{mathpar}
\inferrule[s\_return]{
  \mapopt{\renamelocals}(\ve) \astarrow \vep
}{
  \renamelocalsstmt(\overname{\SReturn(\ve)}{\vs}) \astarrow \overname{\SReturn(\vep)}{\news}
}
\end{mathpar}

\begin{mathpar}
\inferrule[s\_cond]{
  \renamelocalsexpr(\ve) \astarrow \vep\\
  \renamelocalsstmt(\vsone) \astarrow \vsonep\\
  \renamelocalsstmt(\vstwo) \astarrow \vstwop\\
}{
  \renamelocalsstmt(\overname{\SCond(\ve, \vsone, \vstwo)}{\vs}) \astarrow
  \overname{\SCond(\vep, \vsonep, \vstwop)}{\news}
}
\end{mathpar}

\begin{mathpar}
\inferrule[s\_assert]{
  \renamelocalsexpr(\ve) \astarrow \vep
}{
  \renamelocalsstmt(\overname{\SAssert(\ve)}{\vs}) \astarrow \overname{\SAssert(\vep)}{\news}
}
\end{mathpar}

\begin{mathpar}
\inferrule[s\_for]{
  \renamelocalsexpr(\estart) \astarrow \estartp\\
  \renamelocalsexpr(\eend) \astarrow \eendp\\
  \renamelocalsstmt(\vbody) \astarrow \vbodyp\\
  \mapopt{\renamelocalsexpr}(\velimitopt) \astarrow \velimitoptp\\
}{
  {
  \begin{array}{r}
  \renamelocalsstmt\left(\overname{
    \SFor\left\{
    \begin{array}{rcl}
    \Forindexname  &:& \id,\\
    \Forstarte     &:& \estart,\\
    \Fordir        &:& \dir,\\
    \Forende       &:& \eend,\\
    \Forbody       &:& \vbody,\\
    \Forlimit      &:& \velimitopt
    \end{array}
  \right\}
  }{\vs}\right) \astarrow \\
  \overname{
    \SFor\left\{
    \begin{array}{rcl}
    \Forindexname  &:& \id,\\
    \Forstarte     &:& \estartp,\\
    \Fordir        &:& \dir,\\
    \Forende       &:& \eendp,\\
    \Forbody       &:& \vbodyp,\\
    \Forlimit      &:& \velimitoptp
    \end{array}
  \right\}
  }{\news}
  \end{array}
  }
}
\end{mathpar}

\begin{mathpar}
\inferrule[s\_while]{
  \renamelocalsexpr(\ve) \astarrow \vep\\
  \mapopt{\renamelocalsexpr}(\vlimit) \astarrow \vlimitp\\
  \renamelocalsstmt(\vbody) \astarrow \vbodyp\\
}{
  \renamelocalsstmt(\overname{\SWhile(\ve, \vlimit, \vbody)}{\vs}) \astarrow
  \overname{\SWhile(\vep, \vlimitp, \vbodyp)}{\news}
}
\end{mathpar}

\begin{mathpar}
\inferrule[s\_repeat]{
  \renamelocalsstmt(\vs) \astarrow \vsp\\
  \renamelocalsexpr(\ve) \astarrow \vep\\
  \mapopt{\renamelocalsexpr}(\vlimit) \astarrow \vlimitp\\
}{
  \renamelocalsstmt(\overname{\SRepeat(\vs, \ve, \vlimit)}{\vs}) \astarrow
  \overname{\SRepeat(\vsp, \vep, \vlimitp)}{\news}
}
\end{mathpar}

\begin{mathpar}
\inferrule[s\_throw]{
  \mapopt{\renamelocalsexpr}(\veopt) \astarrow \veoptp
}{
  \renamelocalsstmt(\overname{\SThrow(\veopt)}{\vs}) \astarrow
  \overname{\SThrow(\veoptp)}{\news}
}
\end{mathpar}

\begin{mathpar}
\inferrule[s\_try]{
  \astlabel(\vs) = \STry
}{
  \renamelocalsstmt(\vs) \astarrow \tododefine{not implemented yet}
}
\end{mathpar}

\begin{mathpar}
\inferrule[s\_print]{
  \newargs = [\ve\in\args: \renamelocalsexpr(\ve)]
}{
  {
  \begin{array}{r}
  \renamelocalsstmt(\overname{\SPrint(\args, \vnewline)}{\vs}) \astarrow\\
  \overname{\SPrint(\newargs, \vnewline)}{\news}
  \end{array}
  }
}
\end{mathpar}

\begin{mathpar}
\inferrule[s\_unreachable]{}{
  \renamelocalsstmt(\overname{\SUnreachable}{\vs}) \astarrow \overname{\SUnreachable}{\news}
}
\end{mathpar}

\begin{mathpar}
\inferrule[s\_pragma]{
  \newargs \eqdef [\ve\in\args: \renamelocalsexpr(\ve)]
}{
  \renamelocalsstmt(\overname{\SPragma(\name, \args)}{\vs}) \astarrow \overname{\SPragma(\name, \newargs)}{\news}
}
\end{mathpar}
\CodeSubsection{\RenameLocalsStmtBegin}{\RenameLocalsStmtEnd}{../ASTUtils.ml}

\ASTRuleDef{RenameLocalsExpr}
\hypertarget{def-renamelocalsexpr}{}
The helper function
\[
\renamelocalsexpr(\overname{\expr}{\ve}) \aslto \overname{\expr}{\newe}
\]
renames the local storage elements appearing in the expression $\ve$,
yielding the expression $\newe$.

\ProseParagraph
\OneApplies
\begin{itemize}
  \item \AllApplyCase{e\_literal}
  \begin{itemize}
    \item $\ve$ is a \literalexpressionterm;
    \item \Proseeqdef{$\newe$}{$\ve$}.
  \end{itemize}

  \item \AllApplyCase{e\_var}
  \begin{itemize}
    \item $\ve$ is a \variableexpression{$\vs$};
    \item \Proserenamelocalsname{$\vx$}{$\vxp$};
    \item \Proseeqdef{$\newe$}{the \variableexpression{$\vxp$}}.
  \end{itemize}

  \item \AllApplyCase{e\_arbitrary}
  \begin{itemize}
    \item $\ve$ is a \arbitraryexpression{$\vt$};
    \item \Proserenamelocalsty{$\vt$}{$\vtp$};
    \item \Proseeqdef{$\newe$}{the \arbitraryexpression{$\vtp$}}.
  \end{itemize}

  \item \AllApplyCase{e\_atc}
  \begin{itemize}
    \item $\ve$ is a \atcexpression{$\veone$}{$\vt$};
    \item \Proserenamelocalsexpr{$\veone$}{$\veonep$};
    \item \Proserenamelocalsty{$\vt$}{$\vtp$};
    \item \Proseeqdef{$\newe$}{the \atcexpression{$\veonep$}{$\vtp$}}.
  \end{itemize}

  \item \AllApplyCase{e\_binop}
  \begin{itemize}
    \item $\ve$ is a \binopexpression{$\op$}{$\veone$}{$\vetwo$};
    \item \Proserenamelocalsexpr{$\veone$}{$\veonep$};
    \item \Proserenamelocalsexpr{$\vetwo$}{$\vetwop$};
    \item \Proseeqdef{$\newe$}{the \binopexpression{$\op$}{$\veonep$}{$\vetwop$}}.
  \end{itemize}

  \item \AllApplyCase{e\_unop}
  \begin{itemize}
    \item $\ve$ is a \unopexpression{$\op$}{$\veone$};
    \item \Proserenamelocalsexpr{$\veone$}{$\veonep$};
    \item \Proseeqdef{$\newe$}{the \unopexpression{$\op$}{$\veonep$}}.
  \end{itemize}

  \item \AllApplyCase{e\_call}
  \begin{itemize}
    \item $\ve$ is a \callexpression{$\name$}{$\params$}{$\args$}{$\calltype$};
    \item \Proseeqdef{$\newargs$}{the list obtained by applying $\renamelocalsexpr$ to each expression in $\args$};
    \item \Proseeqdef{$\newparams$}{the list obtained by applying $\renamelocalsexpr$ to each expression in $\params$};
    \item \Proseeqdef{$\newe$}{the \callexpression{$\name$}{$\newparams$}{$\newargs$}{$\calltype$}}.
  \end{itemize}

  \item \AllApplyCase{e\_slice}
  \begin{itemize}
    \item $\ve$ is a \sliceexpression{$\veone$}{$\vslices$};
    \item \Proserenamelocalsexpr{$\veone$}{$\veonep$};
    \item \Proseeqdef{$\vslicesp$}{the list obtained by applying $\renamelocalsslice$ to each slice in $\vslices$};
    \item \Proseeqdef{$\newe$}{the \sliceexpression{$\veonep$}{$\vslicesp$}}.
  \end{itemize}

  \item \AllApplyCase{e\_cond}
  \begin{itemize}
    \item $\ve$ is a \conditionalstatement{$\veone$}{$\vetwo$}{$\vethree$};
    \item \Proserenamelocalsexpr{$\veone$}{$\veonep$};
    \item \Proserenamelocalsexpr{$\vetwo$}{$\vetwop$};
    \item \Proserenamelocalsexpr{$\vethree$}{$\vethreep$};
    \item \Proseeqdef{$\newe$}{the \conditionalstatement{$\veonep$}{$\vetwop$}{$\vethreep$}}.
  \end{itemize}

  \item \AllApplyCase{e\_getarray}
  \begin{itemize}
    \item $\ve$ is a \getarrayexpression{$\veone$}{$\vetwo$};
    \item \Proserenamelocalsexpr{$\veone$}{$\veonep$};
    \item \Proserenamelocalsexpr{$\vetwo$}{$\vetwop$};
    \item \Proseeqdef{$\newe$}{the \getarrayexpression{$\veonep$}{$\vetwop$}}.
  \end{itemize}

  \item \AllApplyCase{e\_getenumarray}
  \begin{itemize}
    \item $\ve$ is a \getenumarrayexpression{$\veone$}{$\vetwo$};
    \item \Proserenamelocalsexpr{$\veone$}{$\veonep$};
    \item \Proserenamelocalsexpr{$\vetwo$}{$\vetwop$};
    \item \Proseeqdef{$\newe$}{the \getenumarrayexpression{$\veonep$}{$\vetwop$}}.
  \end{itemize}

  \item \AllApplyCase{e\_getfield}
  \begin{itemize}
    \item $\ve$ is a \getfieldexpression{$\veone$}{$\vf$};
    \item \Proserenamelocalsexpr{$\veone$}{$\veonep$};
    \item \Proseeqdef{$\newe$}{the \getfieldexpression{$\veonep$}{$\vf$}}.
  \end{itemize}

  \item \AllApplyCase{e\_getfields}
  \begin{itemize}
    \item $\ve$ is a \getfieldsexpression{$\veone$}{$\vli$};
    \item \Proserenamelocalsexpr{$\veone$}{$\veonep$};
    \item \Proseeqdef{$\newe$}{the \getfieldsexpression{$\veonep$}{$\vli$}}.
  \end{itemize}

  \item \AllApplyCase{e\_getitem}
  \begin{itemize}
    \item $\ve$ is a \getitemexpression{$\veone$}{$\vi$};
    \item \Proserenamelocalsexpr{$\veone$}{$\veonep$};
    \item \Proseeqdef{$\newe$}{the \getitemexpression{$\veonep$}{$\vi$}}.
  \end{itemize}

  \item \AllApplyCase{e\_record}
  \begin{itemize}
    \item $\ve$ is a \recordexpression{$\vt$}{$\vli$};
    \item \Proserenamelocalsty{$\vt$}{$\vtp$};
    \item \Proseeqdef{$\newe$}{the \recordexpression{$\vtp$}{$\vli$}}.
  \end{itemize}

  \item \AllApplyCase{e\_tuple}
  \begin{itemize}
    \item $\ve$ is a \tupleexpression{$\vli$};
    \item \Proseeqdef{$\vlip$}{the list obtained by the application of $\renamelocalsty$ to each type in $\vli$};
    \item \Proseeqdef{$\newe$}{the \tupleexpression{$\vlip$}}.
  \end{itemize}

  \item \AllApplyCase{e\_pattern}
  \begin{itemize}
    \item $\ve$ is a \patternexpressionterm;
    \item this case is not implemented.
  \end{itemize}
\end{itemize}

\FormallyParagraph
\begin{mathpar}
\inferrule[e\_literal]{
  \astlabel(\ve) = \ELiteral
}{
  \renamelocalsexpr(\ve) \astarrow \overname{\ve}{\newe}
}
\end{mathpar}

\begin{mathpar}
\inferrule[e\_var]{
  \renamelocalsname(\vx) \astarrow \vxp
}{
  \renamelocalsexpr(\overname{\EVar(\vx)}{\ve}) \astarrow \overname{\EVar(\vxp)}{\newe}
}
\end{mathpar}

\begin{mathpar}
\inferrule[e\_arbitrary]{
  \renamelocalsty(\vt) \astarrow \vtp
}{
  \renamelocalsexpr(\overname{\EArbitrary(\vt)}{\ve}) \astarrow \overname{\EArbitrary(\vtp)}{\newe}
}
\end{mathpar}

\begin{mathpar}
\inferrule[e\_atc]{
  \renamelocalsexpr(\veone) \astarrow \veonep\\
  \renamelocalsty(\vt) \astarrow \vtp
}{
  \renamelocalsexpr(\overname{\EATC(\veone, \vt)}{\ve}) \astarrow \overname{\EATC(\veonep, \vtp)}{\newe}
}
\end{mathpar}

\begin{mathpar}
\inferrule[e\_binop]{
  \renamelocalsexpr(\veone) \astarrow \veonep\\
  \renamelocalsexpr(\vetwo) \astarrow \vetwop\\
}{
  \renamelocalsexpr(\overname{\EBinop(\op, \veone, \vetwo)}{\ve}) \astarrow \overname{\EBinop(\op, \veonep, \vetwop)}{\newe}
}
\end{mathpar}

\begin{mathpar}
\inferrule[e\_unop]{
  \renamelocalsexpr(\veone) \astarrow \veonep
}{
  \renamelocalsexpr(\overname{\EUnop(\op, \veone)}{\ve}) \astarrow \overname{\EUnop(\op, \veonep)}{\newe}
}
\end{mathpar}

\begin{mathpar}
\inferrule[e\_call]{
  \newargs \eqdef [\ve\in\args: \renamelocalsexpr(\ve)]\\
  \newparams \eqdef [\ve\in\params: \renamelocalsexpr(\ve)]
}{
  {
  \begin{array}{r}
  \renamelocalsexpr(\overname{
    \ECall\left(
      {
        \left\{
          \begin{array}{rcl}
         \callname &:& \name, \\
         \callparams &:& \params, \\
         \callargs &:& \args, \\
         \callcalltype &:& \calltype
        \end{array}
        \right\}
        }\right)}{\ve}) \astarrow \\
    \overname{
      \ECall\left(
      {
        \left\{
          \begin{array}{rcl}
         \callname &:& \name, \\
         \callparams &:& \newparams, \\
         \callargs &:& \newargs, \\
         \callcalltype &:& \calltype
        \end{array}
        \right\}
        }\right)
    }{\newe}
      \end{array}
  }
}
\end{mathpar}

\begin{mathpar}
\inferrule[e\_slice]{
  \renamelocalsexpr(\veone) \astarrow \veonep\\
  \vslicesp \eqdef [\vs \in \vslices: \renamelocalsslice(\vs)]
}{
  \renamelocalsexpr(\overname{\ESlice(\veone, \vslices)}{\ve}) \astarrow \overname{\ESlice(\veonep, \vslicesp)}{\newe}
}
\end{mathpar}

\begin{mathpar}
\inferrule[e\_cond]{
  \renamelocalsexpr(\veone) \astarrow \veonep\\
  \renamelocalsexpr(\vetwo) \astarrow \vetwop\\
  \renamelocalsexpr(\vethree) \astarrow \vethreep\\
}{
  \renamelocalsexpr(\overname{\ECond(\veone, \vetwo, \vethree)}{\ve}) \astarrow \overname{\ECond(\veonep, \vetwop, \vethreep)}{\newe}
}
\end{mathpar}

\begin{mathpar}
\inferrule[e\_getarray]{
  \renamelocalsexpr(\veone) \astarrow \veonep\\
  \renamelocalsexpr(\vetwo) \astarrow \vetwop
}{
  \renamelocalsexpr(\overname{\EGetArray(\veone, \vetwo)}{\ve}) \astarrow \overname{\EGetArray(\veonep, \vetwop)}{\newe}
}
\end{mathpar}

\begin{mathpar}
\inferrule[e\_getenumarray]{
  \renamelocalsexpr(\veone) \astarrow \veonep\\
  \renamelocalsexpr(\vetwo) \astarrow \vetwop
}{
  \renamelocalsexpr(\overname{\EGetEnumArray(\veone, \vetwo)}{\ve}) \astarrow \overname{\EGetEnumArray(\veonep, \vetwop)}{\newe}
}
\end{mathpar}

\begin{mathpar}
\inferrule[e\_getfield]{
  \renamelocalsexpr(\veone) \astarrow \veonep
}{
  \renamelocalsexpr(\overname{\EGetField(\veone, \vf)}{\ve}) \astarrow \overname{\EGetField(\veone, \vf)}{\newe}
}
\end{mathpar}

\begin{mathpar}
\inferrule[e\_getfields]{
  \renamelocalsexpr(\veone) \astarrow \veonep
}{
  \renamelocalsexpr(\overname{\EGetFields(\veone, \vli)}{\ve}) \astarrow \overname{\EGetFields(\veonep, \vli)}{\newe}
}
\end{mathpar}

\begin{mathpar}
\inferrule[e\_getitem]{
  \renamelocalsexpr(\veone) \astarrow \veonep
}{
  \renamelocalsexpr(\overname{\EGetItem(\veone, \vi)}{\ve}) \astarrow \overname{\EGetItem(\veonep, \vi)}{\newe}
}
\end{mathpar}

\begin{mathpar}
\inferrule[e\_record]{
  \renamelocalsty(\vy) \astarrow \vtp
}{
  \renamelocalsexpr(\overname{\ERecord(\vt, \vli)}{\ve}) \astarrow \overname{\ERecord(\vtp, \vli)}{\newe}
}
\end{mathpar}

\begin{mathpar}
\inferrule[e\_tuple]{
  \vlip \eqdef [\ve\in\vli: \renamelocalsexpr(\ve)]
}{
  \renamelocalsexpr(\overname{\ETuple(\vli)}{\ve}) \astarrow \overname{\ETuple(\vlip)}{\newe}
}
\end{mathpar}

\begin{mathpar}
\inferrule[e\_pattern]{}{
  \renamelocalsexpr(\overname{\EPattern(\Ignore, \Ignore)}{\ve}) \astarrow \tododefine{not implemented yet}
}
\end{mathpar}

\CodeSubsection{\RenameLocalsExprBegin}{\RenameLocalsExprEnd}{../ASTUtils.ml}

\ASTRuleDef{RenameLocalsLexpr}
\hypertarget{def-renamelocalslexpr}{}
The helper function
\[
\renamelocalslexpr(\overname{\lexpr}{\vle}) \aslto \overname{\lexpr}{\newle}
\]
renames the local storage elements appearing in the \assignableexpression\ $\vle$,
yielding the \assignableexpression\ $\newle$.

\ProseParagraph
\OneApplies
\begin{itemize}
  \item \AllApplyCase{le\_discard}
  \begin{itemize}
    \item $\vle$ is a \discardlexprterm;
    \item \Proseeqdef{$\newle$}{$\vle$}.
  \end{itemize}

  \item \AllApplyCase{le\_var}
  \begin{itemize}
    \item $\vle$ is an \varlexpr{$\vx$};
    \item \Proserenamelocalsname{$\vx$}{$\vxp$};
    \item \Proseeqdef{$\newle$}{the \varlexpr{$\vxp$}}.
  \end{itemize}

  \item \AllApplyCase{le\_slice}
  \begin{itemize}
    \item $\vle$ is an \slicelexpr{$\vleone$}{$\vslices$};
    \item \Proserenamelocalslexpr{$\vleone$}{$\vleonep$};
    \item \Proseeqdef{$\vslicesp$}{the list obtained by applying $\renamelocalsslice$ to each slice in $\vslices$};
    \item \Proseeqdef{$\newle$}{the \slicelexpr{$\vleonep$}{$\vslicesp$}}.
  \end{itemize}

  \item \AllApplyCase{le\_setarray}
  \begin{itemize}
    \item $\vle$ is an \setarraylexpr{$\vleone$}{$\vi$};
    \item \Proserenamelocalslexpr{$\vle$}{$\vlep$};
    \item \Proserenamelocalsexpr{$\vi$}{$\vip$};
    \item \Proseeqdef{$\newle$}{the \setarraylexpr{$\vleonep$}{$\vip$}}.
  \end{itemize}

  \item \AllApplyCase{le\_setfield}
  \begin{itemize}
    \item $\vle$ is an \setfield{$\vleone$}{$\vf$};
    \item \Proserenamelocalslexpr{$\vle$}{$\vlep$};
    \item \Proseeqdef{$\newle$}{the \setfield{$\vleonep$}{$\vf$}}.
  \end{itemize}

  \item \AllApplyCase{le\_setfields}
  \begin{itemize}
    \item $\vle$ is an \setfieldslexpr{$\vleone$}{$\vfl$};
    \item \Proserenamelocalslexpr{$\vle$}{$\vlep$};
    \item \Proseeqdef{$\newle$}{the \setfieldslexpr{$\vleonep$}{$\vfl$}}.
  \end{itemize}

  \item \AllApplyCase{le\_destructuring}
  \begin{itemize}
    \item $\vle$ is an \destructuringlexpr{$\vles$};
    \item \Proseeqdef{$\vlesp$}{the list obtained by applying $\renamelocalslexpr$\\ to each assignable expression in $\vles$};
    \item \Proseeqdef{$\newle$}{the \destructuringlexpr{$\vlesp$}}.
  \end{itemize}
\end{itemize}

\FormallyParagraph
\begin{mathpar}
\inferrule[le\_discard]{}{
  \renamelocalslexpr(\overname{\LEDiscard}{\vle}) \astarrow \overname{\LEDiscard}{\newle}
}
\end{mathpar}

\begin{mathpar}
\inferrule[le\_var]{
  \renamelocalsname(\vx) \astarrow \vxp
}{
  \renamelocalslexpr(\overname{\LEVar(\vx)}{\vle}) \astarrow \overname{\LEVar(\vxp)}{\newle}
}
\end{mathpar}

\begin{mathpar}
\inferrule[le\_slice]{
  \renamelocalslexpr(\vleone) \astarrow \vleonep\\
  \vslicesp \eqdef [\vs\in\vslices: \renamelocalsslice(\vs)]
}{
  \renamelocalslexpr(\overname{\LESlice(\vleone, \vslices)}{\vle}) \astarrow \overname{\LESlice(\vlep, \vslicesp)}{\newle}
}
\end{mathpar}

\begin{mathpar}
\inferrule[le\_setarray]{
  \renamelocalslexpr(\vleone) \astarrow \vleonep\\
  \renamelocalsexpr(\vi) \astarrow \vip\\
}{
  \renamelocalslexpr(\overname{\LESetArray(\vleone, \vi)}{\vle}) \astarrow \overname{\LESetArray(\vleonep, \vip)}{\newle}
}
\end{mathpar}

\begin{mathpar}
\inferrule[le\_setfield]{
  \renamelocalslexpr(\vleone) \astarrow \vleonep
}{
  \renamelocalslexpr(\overname{\LESetField(\vleone, \vf)}{\vle}) \astarrow \overname{\LESetField(\vleonep, \vf)}{\newle}
}
\end{mathpar}

\begin{mathpar}
\inferrule[le\_setfields]{
  \renamelocalslexpr(\vleone) \astarrow \vleonep
}{
  \renamelocalslexpr(\overname{\LESetFields(\vleone, \vfl)}{\vle}) \astarrow \overname{\LESetFields(\vleonep, \vfl)}{\newle}
}
\end{mathpar}

\begin{mathpar}
\inferrule[le\_destructuring]{
  \vlesp \eqdef [\vl\in\vles: \renamelocalslexpr(\vl)]
}{
  \renamelocalslexpr(\overname{\LEDestructuring(\vles)}{\vle}) \astarrow \overname{\LEDestructuring(\vlesp)}{\newle}
}
\end{mathpar}

\ASTRuleDef{RenameLocalsLDI}
\hypertarget{def-renamelocalsldi}{}
The helper function
\[
\renamelocalsldi(\overname{\localdeclarationitem}{\ldi}) \aslto \overname{\localdeclarationitem}{\newldi}
\]
renames the local storage elements appearing in the local declaration item $\ldi$,
yielding the local declaration item $\newldi$.

\ProseParagraph
\OneApplies
\begin{itemize}
  \item \AllApplyCase{var}
  \begin{itemize}
    \item $\ldi$ is a local declaration item for the identifier $\vx$;
    \item \Proserenamelocalsname{$\vx$}{$\vxp$};
    \item \Proseeqdef{$\newldi$}{the local declaration item for the identifier $\vxp$}.
  \end{itemize}

  \item \AllApplyCase{tuple}
  \begin{itemize}
    \item $\ldi$ is a local declaration item for the tuple of identifiers $\names$;
    \item \Proseeqdef{$\namesp$}{the list obtained by applying $\renamelocalsname$ to each identifier in $\names$};
    \item \Proseeqdef{$\newldi$}{the local declaration item for the tuple of identifiers $\namesp$}.
  \end{itemize}
\end{itemize}

\FormallyParagraph
\begin{mathpar}
\inferrule[var]{}{
  \renamelocalsldi(\overname{\LDIVar(\vx)}{\ldi}) \astarrow \overname{\LDIVar(\vxp)}{\newldi}
}
\end{mathpar}

\begin{mathpar}
\inferrule[tuple]{
  \namesp \eqdef [\name\in\names: \renamelocalsname(\name)]
}{
  \renamelocalsldi(\overname{\LDITuple(\names)}{\ldi}) \astarrow \overname{\LDITuple(\namesp)}{\newldi}
}
\end{mathpar}

\ASTRuleDef{RenameLocalsConstraint}
\hypertarget{def-renamelocalsconstraint}{}
The helper function
\[
\renamelocalsconstraint(\overname{\intconstraint}{\vc}) \aslto \overname{\intconstraint}{\newc}
\]
renames the local storage elements appearing in the constraint $\vc$,
yielding the constraint $\newc$.

\ProseParagraph
\OneApplies
\begin{itemize}
  \item \AllApplyCase{exact}
  \begin{itemize}
    \item $\vx$ is a \Proseconstraintexact{$\ve$};
    \item \Proserenamelocals{$\ve$}{$\vep$};
    \item \Proseeqdef{$\newc$}{\Proseconstraintexact{$\vep$}}.
  \end{itemize}

  \item \AllApplyCase{range}
  \begin{itemize}
    \item $\vx$ is a \Proseconstraintrange{$\veone$}{$\vetwo$};
    \item \Proserenamelocals{$\veone$}{$\veonep$};
    \item \Proserenamelocals{$\vetwo$}{$\vetwop$};
    \item \Proseeqdef{$\newc$}{\Proseconstraintrange{$\veonep$}{$\vetwop$}}.
  \end{itemize}
\end{itemize}

\FormallyParagraph
\begin{mathpar}
\inferrule[exact]{
  \renamelocalsexpr(\ve) \astarrow \vep
}{
  \renamelocalsconstraint(\overname{\ConstraintExact(\ve)}{\vc}) \astarrow \overname{\ConstraintExact(\vep)}{\newc}
}
\end{mathpar}

\begin{mathpar}
\inferrule[range]{
  \renamelocalsexpr(\veone) \astarrow \veonep\\
  \renamelocalsexpr(\vetwo) \astarrow \vetwop\\
}{
  \renamelocalsconstraint(\overname{\ConstraintRange(\veone, \vetwo)}{\vc}) \astarrow \overname{\ConstraintRange(\veonep, \vetwop)}{\newc}
}
\end{mathpar}

\ASTRuleDef{RenameLocalsSlice}
\hypertarget{def-renamelocalsslice}{}
The helper function
\[
\renamelocalsslice(\overname{\slice}{\vslice}) \aslto \overname{\slice}{\vnewslice}
\]
renames the local storage elements appearing in the slice $\vslice$,
yielding $\vnewslice$.

\ProseParagraph
\OneApplies
\begin{itemize}
  \item \AllApplyCase{single}
  \begin{itemize}
    \item $\vslice$ is the slice of the single expression $\ve$;
    \item \Proserenamelocalsexpr{$\ve$}{$\vep$};
    \item \Proseeqdef{$\vnewslice$}{the slice of the single expression $\vep$};
  \end{itemize}

  \item \AllApplyCase{non\_single}
  \begin{itemize}
    \item $\vslice$ is a slice over two expression $\veone$ and $\vetwo$ with AST label \\
          $S \in \{\SliceLength, \SliceRange, \SliceStar\}$;
    \item \Proserenamelocalsexpr{$\veone$}{$\veonep$};
    \item \Proserenamelocalsexpr{$\vetwo$}{$\vetwop$};
    \item \Proseeqdef{$\vnewslice$}{the slice with AST label $S$ over the two expression $\veonep$ and $\vetwop$};
  \end{itemize}
\end{itemize}

\FormallyParagraph
\begin{mathpar}
\inferrule[single]{
  \renamelocalsexpr(\ve) \astarrow \vep
}{
\renamelocalsslice(\overname{\SliceSingle(\ve)}{\vslice}) \astarrow \overname{\SliceSingle(\vep)}{\vnewslice}
}
\end{mathpar}

\begin{mathpar}
\inferrule[non\_single]{
  S \in \{\SliceLength, \SliceRange, \SliceStar\}
  \renamelocalsexpr(\veone) \astarrow \veonep\\
  \renamelocalsexpr(\vetwo) \astarrow \vetwop\\
}{
\renamelocalsslice(\overname{S(\veone, \vetwo)}{\vslice}) \astarrow \overname{S(\veonep, \vetwop)}{\vnewslice}
}
\end{mathpar}

\ASTRuleDef{RenameLocalsName}
\hypertarget{def-renamelocalsname}{}
The helper function
\[
\renamelocalsname(\overname{\identifier}{\name}) \aslto \overname{\identifier}{\newname}
\]
renames the local identifier $\name$,
yielding the identifier $\newname$.

\ProseParagraph
\Proseeqdef{$\newname$}{the string concatenation of \texttt{\_\_stdlib\_local\_} and $\name$}.

\FormallyParagraph
\begin{mathpar}
\inferrule{}{
  \renamelocalsname(\name) \astarrow \overname{\texttt{\_\_stdlib\_local\_}\ \stringconcat\ \name}{\newname}
}
\end{mathpar}

%%%%%%%%%%%%%%%%%%%%%%%%%%%%%%%%%%%%%%%%%%%%%%%%%%%%%%%%%%%%%%%%%%%%%%%%%%%%%%%%
%% Beginning of rename locals-related functions
%%%%%%%%%%%%%%%%%%%%%%%%%%%%%%%%%%%%%%%%%%%%%%%%%%%%%%%%%%%%%%%%%%%%%%%%%%%%%%%%

\section{Marking Standard Library Functions}

\ASTRuleDef{SetBuiltin}
\hypertarget{def-setbuiltin}{}
The helper function
\[
  \setbuiltin(\overname{\decl}{\vdecl}) \aslto
  \overname{\decl}{\vdeclp}
  \cup \overname{\TBuildError}{\BuildErrorConfig}
\]
sets the $\funcbuiltin$ flag of a top-level function declaration, which is used to identify standard library functions in \TypingRuleRef{InsertStdlibParam}.
It produces a build error when given a top-level declaration which is not a function.

\subsubsection{Prose}
\AllApply
\begin{itemize}
  \item \Prosechecktrans{that $\vdecl$ is a function}{\BuiltinExpectedToBeFunction};
  \item view $\vdecl$ as $\DFunc(\funcsig)$;
  \item define $\funcsigp$ as $\funcsig$ with its $\funcbuiltin$ flag set to $\True$;
  \item define $\vdeclp$ as $\DFunc(\funcsigp)$.
\end{itemize}

\subsubsection{Formally}
\begin{mathpar}
\inferrule{
  \checktrans{\astlabel(\vdecl) = \DFunc}{\BuiltinExpectedToBeFunction} \typearrow \True \terminateas \BuildErrorConfig\\\\
  \vdecl \eqname \DFunc(\funcsig)\\
  \funcsigp \eqdef \funcsig[\funcbuiltin \mapsto \True]
}{
  \setbuiltin(\vdecl) \astarrow \DFunc(\funcsigp)
}
\end{mathpar}
