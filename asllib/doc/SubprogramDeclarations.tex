\chapter{Subprogram Declarations\label{chap:SubprogramDeclarations}}

Subprogram declarations are grammatically derived from $\Ndecl$ via the subset of productions shown in
\secref{SubprogramDeclarationsSyntax} and represented as ASTs via the production of $\decl$
shown in \secref{SubprogramDeclarationsAbstractSyntax}.
%
Subprogram declarations are typed via $\annotateanddeclarefunc$, which is defined in \nameref{sec:TypingRule.AnnotateAndDeclareFunc}.
%
Subprogram declarations have no associated semantics.

%%%%%%%%%%%%%%%%%%%%%%%%%%%%%%%%%%%%%%%%%%%%%%%%%%%%%%%%%%%%%%%%%%%%%%%%%%%%%%%%
\section{Syntax\label{sec:SubprogramDeclarationsSyntax}}
%%%%%%%%%%%%%%%%%%%%%%%%%%%%%%%%%%%%%%%%%%%%%%%%%%%%%%%%%%%%%%%%%%%%%%%%%%%%%%%%
\begin{flalign*}
\Ndecl  \derivesinline\ & \productionname{funcdecl}{func\_decl}\ \Tfunc \parsesep \Tidentifier \parsesep \Nparamsopt \parsesep \Nfuncargs \parsesep \Nreturntype \parsesep \Nfuncbody &\\
|\ & \productionname{proceduredecl}{procedure\_decl}\ \Tfunc \parsesep \Tidentifier \parsesep \Nparamsopt \parsesep \Nfuncargs \parsesep \Nfuncbody &\\
|\ & \productionname{getter}{getter}\ \Tgetter \parsesep \Tidentifier \parsesep \Nparamsopt \parsesep \Naccessargs \parsesep \Nreturntype \parsesep \Nfuncbody&\\
|\ & \productionname{noarggetter}{no\_arg\_getter}\ \Tgetter \parsesep \Tidentifier \parsesep \Nreturntype \parsesep \Nfuncbody &\\
|\ & \productionname{setter}{setter}\ \Tsetter \parsesep \Tidentifier \parsesep \Nparamsopt \parsesep \Naccessargs \parsesep \Teq \parsesep \Ntypedidentifier & \\
    & \wrappedline\ \parsesep \Nfuncbody &\\
|\ & \productionname{noargsetter}{no\_arg\_setter}\ \Tsetter \parsesep \Tidentifier \parsesep \Teq \parsesep \Ntypedidentifier \parsesep \Nfuncbody&\\
\end{flalign*}

\begin{flalign*}
\Nparamsopt \derivesinline\ & \emptysentence &\\
                   |\ & \Tlbrace \parsesep \Clist{\Nopttypedidentifier} \parsesep \Trbrace &\\
\Nopttypedidentifier \derivesinline\ & \Tidentifier \parsesep \option{\Nasty} &\\
\Nfuncargs          \derivesinline\ & \Tlpar \parsesep \Clist{\Ntypedidentifier} \parsesep \Trpar &\\
\Nreturntype        \derivesinline\ & \Tarrow \parsesep \Nty &\\
\Nfuncbody          \derivesinline\ & \Tbegin \parsesep \Nmaybeemptystmtlist \parsesep \Tend &\\
\Naccessargs        \derivesinline\ & \Tlbracket \parsesep \Clist{\Ntypedidentifier} \parsesep \Trbracket &\\
\Nmaybeemptystmtlist \derivesinline\ & \emptysentence \;|\; \Nstmtlist &
\end{flalign*}

%%%%%%%%%%%%%%%%%%%%%%%%%%%%%%%%%%%%%%%%%%%%%%%%%%%%%%%%%%%%%%%%%%%%%%%%%%%%%%%%
\section{Abstract Syntax\label{sec:SubprogramDeclarationsAbstractSyntax}}
%%%%%%%%%%%%%%%%%%%%%%%%%%%%%%%%%%%%%%%%%%%%%%%%%%%%%%%%%%%%%%%%%%%%%%%%%%%%%%%%
\begin{flalign*}
\decl \derives\ & \DFunc(\func) &
\end{flalign*}

\begin{flalign*}
\func \derives\ &
{
\left\{
  \begin{array}{rcl}
 \funcname &:& \Strings, \\
 \funcparameters &:& (\identifier, \ty?)^*,\\
 \funcargs &:& \typedidentifier^*,\\
 \funcbody &:& \subprogrambody,\\
 \funcreturntype &:& \ty?,\\
 \funcsubprogramtype &:& \subprogramtype
\end{array}
\right\}
} &\\
\typedidentifier \derives\ & (\identifier, \ty) &\\
\subprogrambody \derives\ & \SBASL(\stmt) \;|\; \SBPrimitive &\\
\subprogramtype \derives\ & \STProcedure \;|\; \STFunction &\\
                |\  & \STGetter \;|\; \STEmptyGetter &\\
                |\  & \STSetter \;|\; \STEmptySetter &
\end{flalign*}

\subsubsection{ASTRule.GlobalDecl}
The relation
\[
  \builddecl : \overname{\parsenode{\Ndecl}}{\vparsednode} \;\aslrel\; \overname{\decl}{\vastnode}
\]
transforms a parse node $\vparsednode$ into an AST node $\vastnode$.

\hypertarget{build-funcdecl}{}
\begin{mathpar}
\inferrule[func\_decl]{}{
    \builddecl(
      \overname{\Ndecl(
      \Tfunc, \Tidentifier(\name), \punnode{\Nparamsopt}, \punnode{\Nfuncargs}, \punnode{\Nreturntype}, \punnode{\Nfuncbody}
    )}{\vparsednode})
  \astarrow \\
  {
    \overname{
  \DFunc\left(\left\{
    \begin{array}{rcl}
            \funcname &:& \name,\\
            \funcparameters &:& \astof{\vparamsopt},\\
            \funcargs &:& \astof{\vfuncargs},\\
            \funcbody &:& \SBASL(\astof{\vfuncbody}),\\
            \funcreturntype &:& \langle \astof{\vreturntype} \rangle,\\
            \funcsubprogramtype &:& \STFunction
    \end{array}
  \right\}\right)}{\vastnode}
  }
}
\end{mathpar}

\hypertarget{build-procedureecl}{}
\begin{mathpar}
\inferrule[procedure\_decl]{}{
  \builddecl(
    \overname{\Ndecl(\Tfunc, \Tidentifier(\name), \punnode{\Nparamsopt}, \punnode{\Nfuncargs}, \punnode{\Nfuncbody})}{\vparsednode}
    )
  \astarrow \\
  {
    \overname{
  \DFunc\left(\left\{
    \begin{array}{rcl}
            \funcname &:& \name,\\
            \funcparameters &:& \astof{\vparamsopt},\\
            \funcargs &:& \astof{\vfuncargs},\\
            \funcbody &:& \SBASL(\astof{\vfuncbody}),\\
            \funcreturntype &:& \None,\\
            \funcsubprogramtype &:& \STProcedure
    \end{array}
  \right\}\right)
    }{\vastnode}
  }
}
\end{mathpar}

\hypertarget{build-getter}{}
\begin{mathpar}
\inferrule[getter]{}{
  {
    \begin{array}{r}
  \builddecl\left(\overname{\Ndecl\left(
    \begin{array}{l}
      \Tgetter, \Tidentifier(\name), \punnode{\Nparamsopt}, \punnode{\Naccessargs}, \\
      \wrappedline\ \punnode{\Nreturntype}, \punnode{\Nfuncbody}
    \end{array}
      \right)}{\vparsednode}\right)
  \astarrow \\
  \overname{
  \DFunc\left(\left\{
    \begin{array}{rcl}
            \funcname &:& \name,\\
            \funcparameters &:& \astof{\vparamsopt},\\
            \funcargs &:& \astof{\vaccessargs},\\
            \funcbody &:& \SBASL(\astof{\vfuncbody}),\\
            \funcreturntype &:& \langle \astof{\vreturntype}\rangle,\\
            \funcsubprogramtype &:& \STGetter
    \end{array}
  \right\}\right)
  }{\vastnode}
  \end{array}
  }
}
\end{mathpar}

\hypertarget{build-noarggetter}{}
\begin{mathpar}
\inferrule[no\_arg\_getter]{}{
  \builddecl(\overname{\Ndecl(\Tgetter, \Tidentifier(\name), \punnode{\Nreturntype}, \punnode{\Nfuncbody})}{\vparsednode})
  \astarrow \\
  {
    \overname{
  \DFunc\left(\left\{
    \begin{array}{rcl}
            \funcname &:& \name,\\
            \funcparameters &:& \emptylist,\\
            \funcargs &:& \emptylist,\\
            \funcbody &:& \SBASL(\astof{\vfuncbody}),\\
            \funcreturntype &:& \langle \astof{\vreturntype}\rangle,\\
            \funcsubprogramtype &:& \STEmptyGetter
    \end{array}
  \right\}\right)
    }{\vastnode}
  }
}
\end{mathpar}

\hypertarget{build-setter}{}
\begin{mathpar}
\inferrule[setter]{}{
  {
      \builddecl\left(\overname{\Ndecl\left(
        \begin{array}{r}
          \Tsetter, \Tidentifier(\name), \punnode{\Nparamsopt}, \punnode{\Naccessargs}, \Teq, \\
   \wrappedline\ \namednode{\vv}{\Ntypedidentifier}, \punnode{\Nfuncbody}
        \end{array}
   \right)}{\vparsednode}\right)
  } \astarrow
  \\
  {
    \overname{
  \DFunc\left(\left\{
    \begin{array}{rcl}
            \funcname &:& \name,\\
            \funcparameters &:& \astof{\vparamsopt},\\
            \funcargs &:& [\vv] \concat \astof{\vaccessargs},\\
            \funcbody &:& \SBASL(\astof{\vfuncbody}),\\
            \funcreturntype &:& \None,\\
            \funcsubprogramtype &:& \STSetter
    \end{array}
  \right\}\right)
    }{\vastnode}
  }
}
\end{mathpar}

\hypertarget{build-noargsetter}{}
\begin{mathpar}
\inferrule[no\_arg\_setter]{}{
  \builddecl(\overname{\Ndecl(\Tsetter, \Tidentifier(\name), \Teq, \namednode{\vv}{\Ntypedidentifier}, \punnode{\Nfuncbody})}{\vparsednode})
  \astarrow \\
  {
    \overname{
  \DFunc\left(\left\{
    \begin{array}{rcl}
            \funcname &:& \name,\\
            \funcparameters &:& \emptylist,\\
            \funcargs &:& [\vv],\\
            \funcbody &:& \SBASL(\astof{\vfuncbody}),\\
            \funcreturntype &:& \None,\\
            \funcsubprogramtype &:& \STEmptySetter
    \end{array}
  \right\}\right)
    }{\vastnode}
  }
}
\end{mathpar}

\subsubsection{ASTRule.TypedIdentifier \label{sec:ASTRule.TypedIdentifier}}
\hypertarget{build-typedidentifier}{}
The function
\[
\buildtypedidentifier(\overname{\parsenode{\Ntypedidentifier}}{\vparsednode}) \aslto \overname{(\identifier \times \ty)}{\vastnode}
\]
transforms a parse node $\vparsednode$ into an AST node $\vastnode$.

\begin{mathpar}
\inferrule{}{
  \buildtypedidentifier(\overname{\Ntypedidentifier(\Tidentifier(\id), \punnode{\Nasty})}{\vparsednode}) \astarrow \overname{(\id,\astof{\vasty})}{\vastnode}
}
\end{mathpar}

\subsubsection{ASTRule.OptTypedIdentifier \label{sec:ASTRule.OptTypedIdentifier}}
\hypertarget{build-opttypedidentifier}{}
The function
\[
\buildopttypedidentifier(\overname{\parsenode{\Nopttypedidentifier}}{\vparsednode}) \aslto \overname{(\identifier \times \langle\ty\rangle)}{\vastnode}
\]
transforms a parse node $\vparsednode$ into an AST node $\vastnode$.

\begin{mathpar}
\inferrule{
  \buildoption[\Nasty](\vastyopt) \astarrow \astversion{\vastyopt}
}{
  {
  \begin{array}{r}
  \buildopttypedidentifier(\overname{\Ntypedidentifier(\Tidentifier(\id), \namednode{\vastyopt}{\option{\Nasty}})}{\vparsednode}) \astarrow \\
  \overname{(\id, \astversion{\vastyopt})}{\vastnode}
  \end{array}
  }
}
\end{mathpar}

\subsubsection{ASTRule.ReturnType \label{sec:ASTRule.ReturnType}}
\hypertarget{build-returntype}{}
The function
\[
\buildreturntype(\overname{\parsenode{\Nreturntype}}{\vparsednode}) \aslto \overname{\ty}{\vastnode}
\]
transforms a parse node $\vparsednode$ into an AST node $\vastnode$.

\begin{mathpar}
\inferrule{}{
  \buildreturntype(\overname{\Nreturntype(\Tarrow, \punnode{\Nty})}{\vparsednode}) \astarrow
  \overname{\astof{\tty}}{\vastnode}
}
\end{mathpar}

\subsubsection{ASTRule.ParamsOpt \label{sec:ASTRule.ParamsOpt}}
\hypertarget{build-paramsopt}{}
The function
\[
\buildparamsopt(\overname{\parsenode{\Nparamsopt}}{\vparsednode}) \aslto
  \overname{(\identifier\times\langle\ty\rangle)^*}{\vastnode}
\]
transforms a parse node $\vparsednode$ into an AST node $\vastnode$.

\begin{mathpar}
\inferrule[empty]{}{
  \buildparamsopt(\overname{\Nparamsopt(\epsilonnode)}{\vparsednode}) \astarrow
  \overname{\emptylist}{\vastnode}
}
\end{mathpar}

\begin{mathpar}
\inferrule[non\_empty]{
  \buildclist[\Nopttypedidentifier](\ids) \astarrow \astversion{\ids}
}{
  \buildparamsopt(\overname{\Nparamsopt(\Tlbrace, \namednode{\ids}{\Clist{\Nopttypedidentifier}}, \Trbrace)}{\vparsednode}) \astarrow
  \overname{\astversion{\ids}}{\vastnode}
}
\end{mathpar}

\subsubsection{ASTRule.AccessArgs \label{sec:ASTRule.AccessArgs}}
\hypertarget{build-accessargs}{}
The function
\[
\buildaccessargs(\overname{\parsenode{\Naccessargs}}{\vparsednode}) \aslto
  \overname{(\identifier\times\ty)^*}{\vastnode}
\]
transforms a parse node $\vparsednode$ into an AST node $\vastnode$.

\begin{mathpar}
\inferrule{
  \buildclist[\Ntypedidentifier](\ids) \astarrow \astversion{\ids}
}{
  \buildaccessargs(\overname{\Naccessargs(\Tlbracket, \namednode{\ids}{\Clist{\Ntypedidentifier}}, \Trbracket)}{\vparsednode}) \astarrow
  \overname{\astversion{\ids}}{\vastnode}
}
\end{mathpar}

\subsubsection{ASTRule.FuncArgs \label{sec:ASTRule.FuncArgs}}
\hypertarget{build-funcargs}{}
The function
\[
\buildfuncargs(\overname{\parsenode{\Nfuncargs}}{\vparsednode}) \aslto
  \overname{(\identifier\times\ty)^*}{\vastnode}
\]
transforms a parse node $\vparsednode$ into an AST node $\vastnode$.

\begin{mathpar}
\inferrule{
  \buildclist[\Ntypedidentifier](\ids) \astarrow \astversion{\ids}
}{
  \buildfuncargs(\overname{\Nfuncargs(\Tlpar, \namednode{\ids}{\Clist{\Ntypedidentifier}}, \Trpar)}{\vparsednode}) \astarrow
  \overname{\astversion{\ids}}{\vastnode}
}
\end{mathpar}

\subsubsection{ASTRule.MaybeEmptyStmtList \label{sec:ASTRule.MaybeEmptyStmtList}}
\hypertarget{build-maybeemptystmtlist}{}
The function
\[
\buildmaybeemptystmtlist(\overname{\parsenode{\Nmaybeemptystmtlist}}{\vparsednode}) \aslto
  \overname{\stmt}{\vastnode}
\]
transforms a parse node $\vparsednode$ into an AST node $\vastnode$.

\begin{mathpar}
\inferrule[empty]{}{
  \buildmaybeemptystmtlist(\overname{\Nmaybeemptystmtlist(\epsilonnode)}{\vparsednode}) \astarrow
  \overname{\SPass}{\vastnode}
}
\end{mathpar}

\begin{mathpar}
\inferrule[non\_empty]{}{
  \buildmaybeemptystmtlist(\overname{\Nmaybeemptystmtlist(\Nstmtlist)}{\vparsednode}) \astarrow
  \overname{\astof{\vstmtlist}}{\vastnode}
}
\end{mathpar}

\subsubsection{ASTRule.FuncBody \label{sec:ASTRule.FuncBody}}
\hypertarget{build-funcbody}{}
The function
\[
\buildfuncargs(\overname{\parsenode{\Nfuncbody}}{\vparsednode}) \aslto
  \overname{\stmt}{\vastnode}
\]
transforms a parse node $\vparsednode$ into an AST node $\vastnode$.

\begin{mathpar}
\inferrule{}{
  {
  \begin{array}{r}
  \buildfuncbody(\overname{\Nfuncbody(\Tbegin, \namednode{\vstmts}{\Nmaybeemptystmtlist}, \Tend)}{\vparsednode}) \astarrow \\
  \overname{\astof{\vmaybeemptystmtlist}}{\vastnode}
  \end{array}
  }
}
\end{mathpar}

%%%%%%%%%%%%%%%%%%%%%%%%%%%%%%%%%%%%%%%%%%%%%%%%%%%%%%%%%%%%%%%%%%%%%%%%%%%%%%%%
\section{Typing}
%%%%%%%%%%%%%%%%%%%%%%%%%%%%%%%%%%%%%%%%%%%%%%%%%%%%%%%%%%%%%%%%%%%%%%%%%%%%%%%%

We also define the following helper rules:
\begin{itemize}
  \item TypingRule.AnnotateAndDeclareFunc (see \secref{TypingRule.AnnotateAndDeclareFunc})
  \item TypingRule.AnnotateFuncSig (see \secref{TypingRule.AnnotateFuncSig})
  \item TypingRule.UseFuncSig (\secref{TypingRule.UseFuncSig})
  \item TypingRule.GetUndeclaredDefining (see \secref{TypingRule.GetUndeclaredDefining})
  \item TypingRule.ScanForParams (see \secref{TypingRule.ScanForParams})
  \item TypingRule.AnnotateParams (see \secref{TypingRule.AnnotateParams})
  \item TypingRule.AnnotateOneParam (see \secref{TypingRule.AnnotateOneParam})
  \item TypingRule.ArgsAsParams (see \secref{TypingRule.ArgsAsParams})
  \item TypingRule.ArgAsParam (see \secref{TypingRule.ArgAsParam})
  \item TypingRule.AnnotateParamType (see \secref{TypingRule.AnnotateParamType})
  \item TypingRule.AnnotateArgs (see \secref{TypingRule.AnnotateArgs})
  \item TypingRule.AnnotateOneArg (see \secref{TypingRule.AnnotateOneArg})
  \item TypingRule.AnnotateReturnType (see \secref{TypingRule.AnnotateReturnType})
  \item TypingRule.DeclareOneFunc (see \secref{TypingRule.DeclareOneFunc})
  \item TypingRule.SubprogramClash (see \secref{TypingRule.SubprogramClash})
  \item TypingRule.AddNewFunc (see \secref{TypingRule.AddNewFunc})
  \item TypingRule.CheckSetterHasGetter (see \secref{TypingRule.CheckSetterHasGetter})
  \item TypingRule.AddSubprogram (see \secref{TypingRule.AddSubprogram})
\end{itemize}

\subsubsection{TypingRule.AnnotateAndDeclareFunc \label{sec:TypingRule.AnnotateAndDeclareFunc}}
\hypertarget{def-annotateanddeclarefunc}{}
The function
\[
  \annotateanddeclarefunc(\overname{\globalstaticenvs}{\tenv}, \overname{\func}{\funcsig})
  \aslto (\overname{\staticenvs}{\tenv} \times \overname{\func}{\newfuncsig})
  \cup \overname{\TTypeError}{\TypeErrorConfig}
\]
annotates a subprogram definition $\funcsig$ in the global static environment $\genv$,
yielding a new subprogram definition $\newfuncsig$ and modified static environment
$\newtenv$.
\ProseOtherwiseTypeError

\subsection{Prose}
All of the following apply:
\begin{itemize}
  \item annotating the signature of $\funcsig$ in $\genv$ as per
        \secref{TypingRule.AnnotateFuncSig} yields
        the environment $\tenvone$ and subprogram definition $\funcsigone$\ProseOrTypeError;
  \item declaring the subprogram defined by $\funcsigone$ in $\tenvone$
        as per \secref{TypingRule.DeclareOneFunc} yields the environment $\newtenv$
        and new $\func$ node $\newfuncsig$\ProseOrTypeError.
\end{itemize}
\CodeSubsection{\AnnotateAndDeclareFuncBegin}{\AnnotateAndDeclareFuncEnd}{../Typing.ml}

\subsection{Formally}
\begin{mathpar}
\inferrule{
  \annotatefuncsig(\genv, \funcsig) \typearrow (\tenvone, \funcsigone) \OrTypeError\\\\
  \declareonefunc(\tenvone, \funcsigone) \typearrow (\newtenv, \newfuncsig) \OrTypeError
}{
  \annotateanddeclarefunc(\genv, \funcsig) \typearrow (\newtenv, \newfuncsig)
}
\end{mathpar}

\subsubsection{TypingRule.AnnotateFuncSig\label{sec:TypingRule.AnnotateFuncSig}}
\hypertarget{def-annotatefuncsig}{}
The function
\[
  \annotatefuncsig(\overname{\globalstaticenvs}{\genv}, \overname{\func}{\funcsig})
  \aslto (\overname{\staticenvs}{\newtenv} \times \overname{\func}{\newfuncsig})
  \cup \overname{\TTypeError}{\TypeErrorConfig}
\]
annotates the signature of a function definition $\funcsig$ in the global static environment $\genv$,
yielding a new function definition $\newfuncsig$ and modified static environment
$\newtenv$.
\ProseOtherwiseTypeError

\subsection{Prose}
All of the following apply:
\begin{itemize}
  \item $\tenvone$ is the static environment comprised of the global static environment $\genv$ and an empty local environment;
  \item obtaining the variables appearing in the formal types in $\funcsig$ that may be parameter-defining
        in $\tenvone$ via $\getundeclareddefining$ yields \\ $\potentialparams$;
  \item annotating the parameters explicitly listed in $\funcsig$ ($\funcsig.\funcparameters)$ \\
        yields environment $\tenvtwo$, in which the parameters are declared,
        and the function \\
        $\declaredparams$, which binds parameter identifiers to their types\ProseOrTypeError;
  \item annotating arguments from $\funcsig$ that serve as parameters in $\tenvtwo$ yields the list of parameters
        $\argparams$ and modified environment $\tenvthree$\ProseOrTypeError;
  \item $\vparameters$ is the list $\declaredparams$ concatenated with $\argparams$ with each type
        transformed to an \optional\ type;
  \item annotating the arguments listed in $\funcsig$ in $\tenvthree$ with $\tenvtwo$ via \\
        $\annotateargs$ yields the list of annotated
        arguments $\vargs$ and modified environment $\tenvfour$\ProseOrTypeError;
  \item annotating the return type of $\funcsig$ in $\tenvfour$ with $\tenvthree$ via \\ $\annotatereturntype$ yields
        the annotated return type $\vreturntype$ and modified environment $\tenvfive$\ProseOrTypeError;
  \item $\newfuncsig$ is $\funcsig$ with the listed of parameters substituted with \\ $\vparameters$,
        the list of arguments substituted with $\vargs$, and return type substituted with $\vreturntype$;
  \item $\newtenv$ is $\tenvfive$.
\end{itemize}\CodeSubsection{\AnnotateFuncSigBegin}{\AnnotateFuncSigEnd}{../Typing.ml}

\subsection{Formally}
\begin{mathpar}
\inferrule{
  \withemptylocal(\genv) \typearrow \tenvone\\
  \getundeclareddefining(\tenvone, \funcsig) \typearrow \potentialparams\\
  {
    \begin{array}{r}
    \annotateparams(\tenvone, \potentialparams, \funcsig.\funcparameters, (\tenvone, \emptyfunc)) \typearrow \\
    (\tenvtwo, \declaredparams) \OrTypeError
    \end{array}
  }\\
  {
    \begin{array}{r}
    \argsasparams(\tenvtwo, \funcsig) \typearrow (\tenvthree, \argparams) \OrTypeError
    \end{array}
  }\\
  {
    \begin{array}{rl}
    \vparameters \eqdef & [(\id,\vt) \in \declaredparams: (\id, \langle\vt\rangle)]\ \concat \\
                     & [(\id,\vt) \in \argparams: (\id, \langle\vt\rangle)]
    \end{array}
  }\\
  \annotateargs(\tenvthree, \tenvtwo, \funcsig, \argparams) \typearrow (\tenvfour, \vargs) \OrTypeError\\\\
  {
    \begin{array}{r}
  \annotatereturntype(\tenvfour, \tenvthree, \funcsig.\funcreturntype) \typearrow \\
    (\tenvfive, \vreturntype) \OrTypeError
    \end{array}
  }\\\\
  {
    \begin{array}{rrcl}
      \newfuncsig \eqdef \{
                      & \funcname            &:& \funcsig.\funcname,\\
                      & \funcparameters      &:& \vparameters,\\
                      & \funcargs            &:& \vargs,\\
                      & \funcbody            &:& \funcsig.\funcbody,\\
                      & \funcreturntype      &:& \vreturntype,\\
                      & \funcsubprogramtype  &:&  \funcsig.\subprogramtype \\
                  \}  &  &&
        \end{array}
  }
}{
  \annotatefuncsig(\tenv, \funcsig) \typearrow (\overname{\tenvfive}{\newtenv}, \newfuncsig, \argparams)
}
\end{mathpar}

\subsubsection{TypingRule.UseFuncSig \label{sec:TypingRule.UseFuncSig}}
\hypertarget{def-usefuncsig}{}
The function
\[
\usefuncsig(\overname{\func}{\vf}) \aslto \overname{\pow{\identifier}}{\ids}
\]
returns the set of identifiers $\ids$ which the subprogram signature given
by $\vf$ depends on.

\subsection{Prose}
Define $\ids$ as the union of applying $\usety$ to every type of an argument of $\vf$
and applying $\usety$ to the \optional\ return type of $\vf$.

\subsection{Formally}
\begin{mathpar}
\inferrule{}{
  \usefuncsig(\vf) \typearrow \overname{\bigcup_{(\Ignore, \vt) \in \vf.\funcargs}\usety(\vt) \cup \usety(\vf.\funcreturntype)}{\ids}
}
\end{mathpar}

\subsubsection{TypingRule.GetUndeclaredDefining \label{sec:TypingRule.GetUndeclaredDefining}}
\hypertarget{def-getundeclareddefining}{}
The function
\[
\getundeclareddefining(\overname{\staticenvs}{\tenv} \aslsep \overname{\func}{\funcsig})
\aslto \overname{\pow{\identifier}}{\potentialparams}
\]
scans the list of types appearing in $\funcsig.\funcargs$ and in the return type
and returns the set of
identifiers that may be parameter-defining in $\tenv$.

\subsection{Example}
In the following specification, the set of identifiers that may correspond
to parameters of the function \texttt{signature\_example} is $\{\texttt{A}\}$,
since \texttt{A} appears in the type \texttt{bits(A)}
of the argument \texttt{bv}.

\VerbatimInput{\typingtests/TypingRule.AnnotateFuncSig.asl}

\CodeSubsection{\GetUndeclaredDefiningBegin}{\GetUndeclaredDefiningEnd}{../Typing.ml}

\subsection{Prose}
All of the following apply:
\begin{itemize}
  \item define $\formaltypes$ to consist of the types associated with the list of arguments in $\funcsig$
        and the return type in $\funcsig$, if one exists;
  \item scanning each type $\vt$ in $\formaltypes$ via $\scanforparams$ yields the set $\vparams_\vt$;
  \item $\potentialparams$ is the union of $\vparams_\vt$ for each type $\vt$ in $\formaltypes$.
\end{itemize}

\subsection{Formally}
\begin{mathpar}
\inferrule{
  {
    \begin{array}{rcl}
      \formaltypes &\eqdef&
    [(\Ignore, \vt) \in \funcsig.\funcargs: \vt]\ \concat \\
    && \choice{\funcsig.\funcreturntype = \langle\retty\rangle}{[\retty]}{\emptylist}
    \end{array}
  }\\
  \vt\in\formaltypes: \scanforparams(\tenv, \vt) \typearrow \vparams_\vt
}{
  \getundeclareddefining(\tenv, \funcsig) \typearrow \overname{\bigcup_{\vt \in \tys} \vparams_\vt}{\potentialparams}
}
\end{mathpar}

\subsubsection{TypingRule.ScanForParams \label{sec:TypingRule.ScanForParams}}
\hypertarget{def-scanforparams}{}
The function
\[
\scanforparams(\overname{\staticenvs}{\tenv} \aslsep \overname{\ty}{\tty})
\aslto \overname{\pow{\identifier}}{\potentialparams}
\]
scans a single type $\tty$ in $\tenv$ and returns the set of
identifiers that may be parameters in $\tenv$.

\subsection{Example}
Consider the following specification:
\VerbatimInput{\typingtests/TypingRule.AnnotateFuncSig.asl}

Scanning each type in the signature
of the function \texttt{signature\_example} yields the following results:
\begin{center}
\begin{tabular}{lll}
\textbf{Expression} & \textbf{Result} & \textbf{Reason}\\
\hline
\texttt{bits(A)} & $\{\texttt{A}\}$ & \texttt{A} is a variable expression \\
& & and \texttt{A} is not defined in the environment.\\
\texttt{bits(W)} & $\emptyset$ & \texttt{W} is defined in the environment.\\
\texttt{bits(A+B)} & $\emptyset$ & \texttt{A+B} is not a variable expression.\\
\end{tabular}
\end{center}

\subsection{Prose}
One of the following applies:
\begin{itemize}
  \item All of the following apply (\textsc{tbits\_evar}):
  \begin{itemize}
    \item $\tty$ is a bitvector type over a variable expression for $\vx$, that is, \\ $\TBits(\EVar(\vx), \Ignore)$;
    \item $\potentialparams$ is the singleton set consisting of $\vx$ if $\vx$ is not defined as a storage type in $\tenv$
          and the empty set, otherwise.
  \end{itemize}

  \item All of the following apply (\textsc{tbits\_other}):
  \begin{itemize}
    \item $\tty$ is a bitvector type where the bitwidth expression is not a variable expression;
    \item $\potentialparams$ is the empty set.
  \end{itemize}

  \item All of the following apply (\textsc{ttuple}):
  \begin{itemize}
    \item $\tty$ is a tuple type over a list of types $\tys$;
    \item obtaining the set of potential parameters for each type $\vt$ of $\tys$ in $\tenv$ yields $\vparams_\vt$;
    \item $\potentialparams$ is the union of sets $\vparams_\vt$, for each type $\vt$ of $\tys$.
  \end{itemize}

  \item All of the following apply (\textsc{other}):
  \begin{itemize}
    \item $\tty$ is neither a bitvector type or a tuple type;
    \item $\potentialparams$ is the empty set.
  \end{itemize}
\end{itemize}

\begin{mathpar}
  \inferrule[tbits\_evar]{
    \isundefined(\tenv, \vx) \typearrow \vb\\
    \potentialparams \eqdef \choice{\vb}{\{\vx\}}{\emptyset}
  }{
    \scanforparams(\tenv, \overname{\TBits(\EVar(\vx), \Ignore)}{\tty}) \typearrow \potentialparams
  }
  \and
  \inferrule[tbits\_other]{
    \astlabel(\ve) \neq \EVar
  }{
    \scanforparams(\tenv, \overname{\TBits(\ve, \Ignore)}{\tty}) \typearrow \overname{\emptyset}{\potentialparams}
  }
  \and
  \inferrule[ttuple]{
    \vt \in \tys: \scanforparams(\tenv, \vt) \typearrow \vparams_\vt
  }{
    \scanforparams(\tenv, \overname{\TTuple(\tys)}{\tty}) \typearrow \overname{\bigcup_{\vt \in \tys} \vparams_\vt}{\potentialparams}
  }
  \and
  \inferrule[other]{
    \astlabel(\tty) \not\in \{\TBits, \TTuple\}
  }{
    \scanforparams(\tenv, \tty) \typearrow \overname{\emptyset}{\potentialparams}
  }
  \end{mathpar}

\subsubsection{TypingRule.AnnotateParams \label{sec:TypingRule.AnnotateParams}}
\hypertarget{def-annotateparams}{}
The function
\[
\begin{array}{r}
\annotateparams(
  \overname{\staticenvs}{\tenv} \aslsep
  \overname{\pow{\identifier}}{\potentialparams} \aslsep
  \overname{(\identifier \times \langle\ty\rangle)^*}{\vparameters} \aslsep\
  (\overname{\staticenvs}{\tenvonep} \times \overname{\identifier\rightarrow\ty}{\acc})
) \aslto \\
(\overname{\staticenvs}{\newtenv} \times \overname{\identifier\rightarrow\ty}{\declaredparams})
\cup \overname{\TTypeError}{\TypeErrorConfig}
\end{array}
\]
scans the list of explicitly defined parameters $\vparameters$ with respect to the set of potential
parameters $\potentialparams$ in $\tenv$ and then updates a pair
consisting of an updated environment $\tenvonep$, which accumulates local storage declarations
for the parameters, and a function $\acc$, which maps identifiers corresponding to parameters
to their associated types.
The updated pair is given in $\newtenv$ and $\declaredparams$.
\ProseOtherwiseTypeError

\subsection{Example}
In the following specification, the list of explicitly defined parameters
of the function \texttt{signature\_example} is $\{\texttt{A}\}$.
Therefore, $\declaredparams$ binds \texttt{A} to the type \verb|integer{A}|
and $\newtenv$ effectively reflects an added declaration \verb|let A: integer{A}|.

\VerbatimInput{\typingtests/TypingRule.AnnotateFuncSig.asl}

\subsection{Prose}
One of the following applies:
\begin{itemize}
  \item All of the following apply (\textsc{empty}):
  \begin{itemize}
    \item $\vparameters$ is the empty list;
    \item $\newtenv$ is $\tenvonep$;
    \item $\declaredparams$ is $\acc$.
  \end{itemize}

  \item All of the following apply (\textsc{non\_empty}):
  \begin{itemize}
    \item $\vparameters$ is a list with $(\vx, \tyopt)$ as its \head\ and $\vparametersone$ as its \tail;
    \item applying $\annotateoneparam$ to the parameter $(\vx, \tyopt)$ in $\tenvone$ with \\
          $\potentialparams$ and the pair $(\tenvonep, \acc)$
          yields the pair \\
          $(\tenvonepp, \accp)$\ProseOrTypeError;
    \item annotating the parameter list $\vparametersone$ in $\tenvone$ with $\potentialparams$, starting with the pair
          $(\tenvonepp, \accp)$ yields the pair \\ $(\newtenv, \declaredparams)$.
  \end{itemize}
\end{itemize}

\subsection{Formally}
\begin{mathpar}
\inferrule[empty]{}{
  {
    \begin{array}{r}
  \annotateparams(\tenvone, \potentialparams, \overname{\emptylist}{\vparameters}, (\tenvonep, \acc)) \typearrow \\
  (\overname{\tenvonep}{\newtenv}, \overname{\acc}{\declaredparams})
    \end{array}
  }
}
\and
\inferrule[non\_empty]{
  \vparameters \eqname [(\vx,\tyopt)]\concat\vparametersone\\
  {
    \begin{array}{r}
  \annotateoneparam(\tenvone, \potentialparams, (\vx,\tyopt), (\tenvonep, \acc)) \typearrow \\ (\tenvonepp, \accp) \OrTypeError
    \end{array}
  }\\\\
  {
    \begin{array}{r}
  \annotateparams(\tenvone, \potentialparams, \vparametersone, (\tenvonepp, \accp)) \typearrow \\ (\newtenv, \declaredparams) \OrTypeError
\end{array}
}\\\\
}{
  {
    \begin{array}{r}
  \annotateparams(\tenvone, \potentialparams, \vparameters, (\tenvonep, \acc)) \typearrow \\
  (\newtenv, \declaredparams)
    \end{array}
}
}
\end{mathpar}

\subsubsection{TypingRule.AnnotateOneParam \label{sec:TypingRule.AnnotateOneParam}}
The function
\hypertarget{def-annotateoneparam}{}
\[
\begin{array}{r}
\annotateoneparam(
  \overname{\staticenvs}{\tenv} \aslsep
  \overname{\pow{\identifier}}{\potentialparams} \aslsep
  (\overname{\identifier}{\vx} \times \overname{\langle\ty\rangle}{\tyopt}) \aslsep
  (\overname{\staticenvs}{\tenvonep} \times \overname{\identifier\rightarrow\ty}{\acc})
) \\ \aslto
(\overname{\staticenvs}{\newtenv} \times \overname{\identifier\rightarrow\ty}{\declaredparams})
\cup \overname{\TTypeError}{\TypeErrorConfig}
\end{array}
\]
annotates the explicitly defined parameter given by $\vx$ and the \optional\ type $\tyoptp$
with respect to the set of potential
parameters $\potentialparams$ in $\tenv$ and then updates a pair
consisting of an updated environment $\tenvonep$, which accumulates local storage declarations
for the parameters, and a function $\acc$, which maps identifiers corresponding to parameters
to their associated types.
The updated pair is given in $\newtenv$ and $\declaredparams$.
\ProseOtherwiseTypeError

\subsection{Prose}
All of the following apply:
\begin{itemize}
  \item checking that $\vx$ is not defined as a variable in $\tenvonep$ yields $\True$\ProseOrTypeError;
  \item checking whether $\vx$ is included in the set $\potentialparams$ yields $\True$ or a type error
        indicating that each parameter must have a defining argument, thus short-circuiting the entire rule;
  \item One of the following applies:
  \begin{itemize}
    \item All of the following apply (\textsc{type\_parameterized}):
    \begin{itemize}
      \item $\tyopt$ is either $\None$ or a \parameterizedintegertype;
      \item $\vt$ is defined as the \parameterizedintegertype\ for the identifier $\vx$.
    \end{itemize}

    \item All of the following apply (\textsc{type\_annotated}):
    \begin{itemize}
      \item $\tyopt$ is the type $\vtone$, which is not the unconstrained integer type;
      \item annotating $\vtone$ in $\tenvone$ yields $\vt$\ProseOrTypeError.
    \end{itemize}
  \end{itemize}
  \item checking that $\vt$ is a constrained integer in $\tenvone$ via $\checkconstrainedinteger$
        yields $\True$\ProseOrTypeError;
  \item adding the local storage element given by the identifier $\vx$, type $\vt$, and local declaration keyword
        $\LDKLet$ in $\tenvonep$ yields $\newtenv$;
  \item $\declaredparams$ is $\acc$ updated by the binding of $\vx$ to $\vt$.
\end{itemize}

\subsection{Formally}
\begin{mathpar}
\inferrule[type\_parameterized]{
  \checkvarnotinenv{\tenvonep, \vx} \typearrow \True \OrTypeError\\\\
  \checktrans{\vx \in \potentialparams}{\ParameterWithoutDecl} \checktransarrow \True \OrTypeError\\\\
  (\tyopt = \None \lor \tyopt = \langle \unconstrainedinteger\rangle) \\
  \vt \eqdef \TInt(\parameterized(\vx))\\
  \checkconstrainedinteger(\tenvone, \vt) \typearrow \True \OrTypeError\\\\
  \addlocal(\tenvonep, \vx, \vt, \LDKLet) \typearrow \newtenv\\
  \declaredparams \eqdef \acc[\vx \mapsto \vt]
}{
  {
    \begin{array}{r}
  \annotateoneparam(\tenvone, \potentialparams, (\vx, \tyopt), (\tenvonep, \acc)) \typearrow \\ (\newtenv, \declaredparams)
    \end{array}
  }
}
\and
\inferrule[type\_annotated]{
  \checkvarnotinenv{\tenvonep, \vx} \typearrow \True \OrTypeError\\\\
  \checktrans{\vx \in \potentialparams}{\ParameterWithoutDecl} \checktransarrow \True \OrTypeError\\\\
  \vtone \neq \unconstrainedinteger\\
  \annotatetype{\False, \tenvone, \vtone} \typearrow \vt \OrTypeError\\\\
  \checkconstrainedinteger(\tenvone, \vt) \typearrow \True \OrTypeError\\\\
  \addlocal(\tenvonep, \vx, \vt, \LDKLet) \typearrow \newtenv\\
  \declaredparams \eqdef \acc[\vx \mapsto \vt]
}{
  {
    \begin{array}{r}
  \annotateoneparam(\tenvone, \potentialparams, (\vx, \overname{\langle\vtone\rangle}{\tyopt}),  (\tenvonep, \acc)) \typearrow \\ (\newtenv, \declaredparams)
    \end{array}
  }
}
  \end{mathpar}

\subsubsection{TypingRule.ArgsAsParams \label{sec:TypingRule.ArgsAsParams}}
\hypertarget{def-argsasparams}{}
The function
\[
\begin{array}{r}
\argsasparams(
  \overname{\staticenvs}{\tenvone} \aslsep
  \overname{\staticenvs}{\tenvtwo} \aslsep
  \overname{\func}{\funcsig} \aslsep
  \overname{\identifier\rightarrow\ty}{\declaredparams}
  )
\aslto \\
(\overname{\staticenvs}{\newtenv} \times \overname{\identifier\rightarrow\ty}{\argparams})
\cup \overname{\TTypeError}{\TypeErrorConfig}
\end{array}
\]
scans the list of arguments in $\func$ ($\func.\funcargs$) to find the ones that serve as
implicit parameters in $\tenvone$ and are not already included in the domain of $\declaredparams$.
The found parameters are added as local declarations to $\tenvtwo$, resulting in $\newtenv$,
and are used to update $\argparams$.
\ProseOtherwiseTypeError

\subsection{Prose}
One of the following applies:
\begin{itemize}
  \item All of the following apply (\textsc{empty}):
  \begin{itemize}
    \item the subprogram defined by $\funcsig$ has an empty list of arguments;
    \item $\newtenv$ is $\tenvtwo$;
    \item $\argparams$ is the empty list.
  \end{itemize}

  \item All of the following apply (\textsc{non\_empty}):
  \begin{itemize}
    \item the subprogram defined by $\funcsig$ has arguments $\varg_{1..k}$;
    \item obtaining the identifiers that can serve as parameters in the types of the formal arguments of $\funcsig$
          and in its return type yields $\vusedone$;
    \item the set $\vused$ contains all identifiers $\vs$ from $\vusedone$ that are undefined in $\tenvone$ and are not
          bound in $\declaredparams$;
    \item the following premises define the sequences $\argparams_{1..k}$ and $\tenvtwo_{0..k}$ as follows;
    \item $\argparams_1$ is $\declaredparams$;
    \item $\tenvtwo_0$ is $\tenvtwo$;
    \item for $\vi=1..k$, annotating the argument $\varg_\vi$ in $\tenvtwo$ with $\vused$ and the static environment $\tenvtwo_{\vi-1}$
          via $\argasparam$ yields $\tenvtwo_\vi$ and $\argparams_\vi$\ProseOrTypeError;
    \item $\newtenv$ is $\tenvtwo_k$;
    \item $\argparams$ is $\argparams_k$.
  \end{itemize}
\end{itemize}

\subsection{Example}
In the following specification, the argument \texttt{B}
of the function \texttt{signature\_example} is an implicit parameter
as it appears in the type \texttt{bits(A+B)}
(both for the argument \texttt{bv3} and as the return type)
and it is not listed as an explicit parameter.
Therefore, $\newtenv$ will effectively contain the declaration \verb|let B: integer{B}|.

\VerbatimInput{\typingtests/TypingRule.AnnotateFuncSig.asl}

\subsection{Formally}
\begin{mathpar}
\inferrule[empty]{
  \funcsig.\funcargs = \emptylist
}{
  {
    \begin{array}{r}
  \argsasparams(\tenvone, \tenvtwo, \funcsig, \declaredparams) \typearrow \\
  (\overname{\tenvtwo}{\newtenv}, \overname{\emptylist}{\argparams})
    \end{array}
  }
}
\end{mathpar}

\begin{mathpar}
\inferrule[non\_empty]{
  \funcsig.\funcargs \eqname \varg_{1..k}\\
  \usefuncsig(\funcsig) \typearrow \vusedone\\
  \vs\in\vusedone : \isundefined(\tenvone, \vs) \typearrow \vb_\vs\\
  \vused \eqdef \setcomprehension{\vs\in\vusedone \land \vb_\vs \land \declaredparams(\vs) = \bot}{\vs}\\
  \argparams_1 \eqdef \declaredparams\\
  \tenvtwo_0 \eqdef \tenvtwo\\
  \vi=1..k: \argasparam(\tenvtwo, \vused, \varg_\vi,\tenvtwo_{\vi-1}) \typearrow  (\tenvtwo_\vi, \argparams_\vi) \OrTypeError
}{
  \argsasparams(\tenvone, \tenvtwo, \funcsig, \declaredparams) \typearrow
  (\overname{\tenvtwo_k}{\newtenv}, \overname{\argparams_k}{\argparams})
}
\end{mathpar}

\subsubsection{TypingRule.ArgAsParam \label{sec:TypingRule.ArgAsParam}}
\hypertarget{def-argasparam}{}
The function
\[
\argasparam\left(
  \begin{array}{c}
  \overname{\staticenvs}{\tenvtwo} \aslsep\\
  \overname{\pow{\identifier}}{\vused} \aslsep\\
  (\overname{\identifier}{\vx} \times \overname{\ty}{\tty}) \aslsep\\
  (\overname{\staticenvs}{\tenvtwop} \times \overname{\identifier\rightarrow\ty}{\acc})
  \end{array}
  \right)
\aslto
(\overname{\staticenvs}{\newtenv} \times \overname{\identifier\rightarrow\ty}{\accp})
\cup \overname{\TTypeError}{\TypeErrorConfig}
\]
checks whether the argument given by $\vx$ and type $\tty$ is an implicit parameter
by checking whether it appears in $\vused$ and not in $\acc$.
If it is identified as an implicit parameter, it is used to update $\tenvtwop$ to yield $\newtenv$
and to update $\acc$ to yield $\accp$.
\ProseOtherwiseTypeError

\subsection{Prose}
One of the following applies:
\begin{itemize}
  \item All of the following apply (\textsc{not\_used}):
  \begin{itemize}
    \item $\vx$ is not a member of $\vused$;
    \item $\newtenv$ is $\tenvtwop$;
    \item $\accp$ is $\acc$.
  \end{itemize}

  \item All of the following apply (\textsc{used}):
  \begin{itemize}
    \item $\vx$ is a member of $\vused$;
    \item checking that $\vx$ is not declared in $\tenvtwop$ yields $\True$\ProseOrTypeError;
    \item annotating $\tty$ with identifier $\vx$ as a potential parameter type in $\tenvtwo$,
          which is an environment to which all explicit parameters have been added but implicit parameters were not added to,
          via $\annotateparamtype$ yields $\vt$\ProseOrTypeError;
    \item checking whether $\vt$ is a constrained integer type in $\tenvtwo$ yields $\True$\ProseOrTypeError;
    \item adding $\vx$ as a local storage element to $\tenvtwop$ with type $\vt$ and local declaration keyword $\LDKLet$ yields
          $\newtenv$;
    \item $\accp$ is $\acc$ updated by binding $\vx$ to $\vt$.
  \end{itemize}
\end{itemize}

\subsection{Formally}
\begin{mathpar}
\inferrule[not\_used]{
  \vx \not\in \vused
}{
  \argasparam(\tenvtwo, \vused, (\vx, \tty), (\tenvtwop, \acc))
  \typearrow (\overname{\tenvtwop}{\newtenv}, \overname{\acc}{\accp})
}
\and
\inferrule[used]{
  \vx \in \vused\\
  \checkvarnotinenv{\tenvtwop, \vx} \typearrow \True \OrTypeError\\\\
  \annotateparamtype(\tenvtwo, \tty, \vx) \typearrow \vt \OrTypeError\\\\
  \checkconstrainedinteger(\tenvtwo, \vt) \typearrow \True \OrTypeError\\\\
  \addlocal(\tenvtwop, \vx, \vt, \LDKLet) \typearrow \newtenv\\
  \accp \eqdef \acc[\vx\mapsto\vt]
}{
  \argasparam(\tenvtwo, \vused, (\vx, \tty), (\tenvtwop, \acc))
  \typearrow (\newtenv, \accp)
}
\end{mathpar}

\subsubsection{TypingRule.AnnotateParamType \label{sec:TypingRule.AnnotateParamType}}
\hypertarget{def-annotateparamtype}{}
The function
\[
  \annotateparamtype(\overname{\staticenvs}{\tenv} \aslsep \overname{\ty}{\tty} \aslsep \overname{\identifier}{\vx})
  \typearrow \overname{\ty}{\newty}
\]
annotates the type $\tty$ in $\tenv$, considering it as a subprogram parameter with identifier $\vx$,
yielding the type $\newtenv$.
It is assumed that $\tenv$ is an environment to which all explicitly defined parameters of the subprogram
in context were added to, but the implicitly defined parameters were not added to.
\ProseOtherwiseTypeError

\subsection{Prose}
One of the following applies:
\begin{itemize}
  \item All of the following apply (\textsc{tint\_unconstrained}):
  \begin{itemize}
    \item $\tty$ is the unconstrained integer type;
    \item $\newty$ is the \parameterizedintegertype\ for the identifier $\vx$.
  \end{itemize}

  \item All of the following apply (\textsc{other}):
  \begin{itemize}
    \item $\tty$ is not the unconstrained integer type;
    \item annotating the type $\tty$ in $\tenv$ yields $\newty$\ProseOrTypeError.
  \end{itemize}
\end{itemize}

\begin{mathpar}
\inferrule[tint\_unconstrained]{}
{
  \annotateparamtype(\tenv, \overname{\unconstrainedinteger}{\tty}, \vx) \typearrow \overname{\TInt(\parameterized(\vx))}{\newty}
}
\end{mathpar}

\begin{mathpar}
\inferrule[other]{
  \tty \neq \unconstrainedinteger\\
  \annotatetype{\tenv, \tty} \typearrow \newty \OrTypeError
}{
  \annotateparamtype(\tenv, \tty, \vx) \typearrow \newty
}
\end{mathpar}

\subsubsection{TypingRule.AnnotateArgs \label{sec:TypingRule.AnnotateArgs}}
\hypertarget{def-annotateargs}{}
The function
\[
\begin{array}{r}
\annotateargs(
  \overname{\staticenvs}{\tenvtwo} \aslsep
  \overname{\staticenvs}{\tenvthree} \aslsep
  \overname{\func}{\funcsig} \aslsep
  \overname{\identifier\rightarrow\ty}{\argparams}
)
\aslto \\
(\overname{\staticenvs}{\newtenv} \times \overname{(\identifier\times\ty)^*}{\vnewargs})
\cup \overname{\TTypeError}{\TypeErrorConfig}
\end{array}
\]
annotates the arguments listed in $\funcsig$ in the context of
of $\argparams$, which binds parameters to their types,
and two static environments:
$\tenvtwo$ --- the environment to which only explicit parameters were added, and
$\tenvthree$ --- the environment to which all parameters (explicit and implicit) were added.
The result is the environment $\newtenv$ where all arguments have been declared and
the annotated list of arguments $\vnewargs$.
\ProseOtherwiseTypeError

\subsection{Prose}
One of the following applies:
\begin{itemize}
  \item All of the following apply (\textsc{no\_args}):
  \begin{itemize}
    \item the function defined by $\funcsig$ has an empty list of arguments;
    \item $\newtenv$ is $\tenvthree$;
    \item $\vnewargs$ is the empty list.
  \end{itemize}

  \item All of the following apply (\textsc{some\_args}):
  \begin{itemize}
    \item the function defined by $\funcsig$ has arguments $\varg_{1..k}$;
    \item the following premises define the sequence of static environments $\tenvthree_{0..k}$ and
          list of typed identifiers $\vnewarg_{1..k}$;
    \item $\tenvthree_0$ is $\tenvthree$;
    \item annotating the argument $\varg_\vi$ in the context of $\tenvtwo$, $\tenvthree$, $\argparams$,
          and $\tenvthree_{\vi-1}$ via $\annotateonearg$ yields $(\tenvthree_\vi, \vnewarg_\vi)$\ProseOrTypeError;
    \item $\newtenv$ is $\tenvthree_k$;
    \item $\vnewargs$ is the list $\vnewarg_{1..k}$.
  \end{itemize}
\end{itemize}

\subsection{Example}
In the following specification, the annotated arguments are
\texttt{bv}, \texttt{bv2}, \texttt{bv3}, and \texttt{C}.
The argument \texttt{B} is not annotated as an argument since it is classified
and annotated as a parameter.

\VerbatimInput{\typingtests/TypingRule.AnnotateFuncSig.asl}

\subsection{Formally}
\begin{mathpar}
\inferrule[no\_args]{
  \funcsig.\funcargs = \emptylist
}{
  \annotateargs(\tenvtwo, \tenvthree, \funcsig, \argparams) \typearrow (\overname{\tenvthree}{\newtenv}, \overname{\emptylist}{\vnewargs})
}
\and
\inferrule[some\_args]{
  \funcsig.\funcargs \eqname \varg_{1..k}\\
  \tenvthree_{0} \eqdef \tenvthree\\
  {
    \begin{array}{r}
  \vi=1..k: \annotateonearg(\tenvtwo, \tenvthree, \argparams, (\tenvthree_{\vi-1}, \varg_{\vi})) \typearrow \\
    (\tenvthree_{\vi}, \vnewarg_\vi) \OrTypeError
    \end{array}
  }\\
  \vnewargs \eqdef [i=1..k: \vnewarg_{\vi}]
}{
  \annotateargs(\tenvtwo, \tenvthree, \funcsig, \argparams) \typearrow
  (\overname{\tenvthree_k}{\newtenv}, \vnewargs)
}
\end{mathpar}

\subsubsection{TypingRule.AnnotateOneArg \label{sec:TypingRule.AnnotateOneArg}}
\hypertarget{def-annotateonearg}{}
The function
\[
\begin{array}{r}
\annotateonearg(
  \overname{\staticenvs}{\tenvtwo} \aslsep
  \overname{\staticenvs}{\tenvthree} \aslsep
  \overname{\identifier\rightarrow\ty}{\argparams} \aslsep
  (\overname{\staticenvs}{\tenvthreep} \times (\overname{\vx}{\identifier} \times \overname{\ty}{\tty}))
) \aslto \\
(\overname{\staticenvs}{\newtenv} \times\ (\overname{\vx}{\identifier} \times \overname{\ty}{\ttyp}))
\cup \overname{\TTypeError}{\TypeErrorConfig}
\end{array}
\]
annotates the argument $\vx$ of type $\tty$ in the context of $\argparams$, which binds parameters to their types,
and the following static environments:
$\tenvtwo$ --- the environment to which only explicit parameters were added,
$\tenvthree$ --- $\tenvtwo$ with the addition of implicit parameters, and
$\tenvthreep$ --- same as $\tenvthree$ but updated with previously annotated arguments.
The result is the updated environment $\newtenv$ with the added declaration for the current argument
and the annotated argument, which has the same identifier $\vx$ and the annotated type $\vtp$.
\ProseOtherwiseTypeError

\subsection{Prose}
One of the following applies:
\begin{itemize}
  \item All of the following apply (\textsc{param}):
  \begin{itemize}
    \item $\vx$ is not bound in $\argparams$;
    \item annotating the type $\tty$ in $\tenvtwo$ yields $\ttyp$;
    \item $\newtenv$ is $\tenvthreep$.
  \end{itemize}

  \item All of the following apply (\textsc{not\_param}):
  \begin{itemize}
    \item $\vx$ is bound in $\argparams$;
    \item checking that $\vx$ is not defined in $\tenvthreep$ yields $\True$\ProseOrTypeError;
    \item annotating the type $\tty$ in $\tenvthree$ yields $\ttyp$;
    \item adding a local storage element $\vx$ with type $\ttyp$ and local declaration keyword $\LDKLet$ yields
          $\newtenv$.
  \end{itemize}
\end{itemize}

\subsection{Formally}
\begin{mathpar}
\inferrule[param]{
  \argparams(\vx) \neq \bot\\
  \annotatetype{\tenvtwo, \tty} \typearrow \ttyp \OrTypeError
}{
  \annotateonearg(\tenvtwo, \tenvthree, \argparams, (\tenvthreep, (\vx, \tty)))
  \typearrow (\overname{\tenvthreep}{\newtenv}, (\vx, \ttyp))
}
\end{mathpar}

\begin{mathpar}
\inferrule[not\_param]{
  \argparams(\vx) = \bot\\
  \checkvarnotinenv{\tenvthreep, \vx} \typearrow \True \OrTypeError\\\\
  \annotatetype{\tenvthree, \tty} \typearrow \ttyp \OrTypeError\\\\
  \addlocal(\tenvthreep, \vx, \ttyp, \LDKLet) \typearrow \newtenv
}{
  {
    \begin{array}{r}
  \annotateonearg(\tenvtwo, \tenvthree, \argparams, (\tenvthreep, (\vx, \tty)))
  \typearrow \\ (\newtenv, (\vx, \ttyp))
    \end{array}
  }
}
\end{mathpar}

\subsubsection{TypingRule.AnnotateReturnType \label{sec:TypingRule.AnnotateReturnType}}
\hypertarget{def-annotatereturntype}{}
The function
\[
\annotatereturntype(
  \overname{\staticenvs}{\tenvthree} \aslsep
  \overname{\staticenvs}{\tenvfour} \aslsep
  \overname{\langle\ty\rangle}{\returntype}
) \aslto
(\overname{\staticenvs}{\newtenv} \times \overname{\ty}{\newreturntype})
\cup \overname{\TTypeError}{\TypeErrorConfig}
\]
annotates the \optional\ return type $\returntype$ in the context of the static environment
$\tenvthree$ where all parameters have been added to, and
$\tenvfour$ where all parameters and arguments have been added to.
The result is the static environment $\newtenv$, which is $\tenvfour$ with the
annotated return type and the \optional\ annotated return type $\newreturntype$.
\ProseOtherwiseTypeError

\subsection{Prose}
One of the following applies:
\begin{itemize}
  \item All of the following apply (\textsc{no\_return\_type}):
  \begin{itemize}
    \item $\returntype$ is $\None$;
    \item $\newtenv$ is $\tenvfour$;
    \item $\newreturntype$ is $\None$.
  \end{itemize}

  \item All of the following apply (\textsc{has\_return\_type}):
  \begin{itemize}
    \item $\returntype$ is $\langle\tty\rangle$;
    \item annotating $\tty$ in $\tenvthree$ yields $\ttyp$\ProseOrTypeError;
    \item $\newreturntype$ is $\langle\ttyp\rangle$;
    \item $\newtenv$ is $\tenvfour$ with its local environment updated by binding its \\ $\funcreturntype$ field
          to $\newreturntype$.
  \end{itemize}
\end{itemize}

\subsection{Formally}
\begin{mathpar}
\inferrule[no\_return\_type]{}
{
  \annotatereturntype(\tenvthree, \tenvfour, \overname{\None}{\returntype}) \typearrow
  (\overname{\tenvfour}{\newtenv}, \overname{\None}{\newreturntype})
}
\and
\inferrule[has\_return\_type]{
  \annotatetype{\tenvthree, \tty} \typearrow \ttyp \OrTypeError\\\\
  \newreturntype \eqdef \langle\ttyp\rangle\\
  \newtenv \eqdef (G^\tenvfour, L^\tenvfour[\funcreturntype\mapsto\newreturntype])
}{
  \annotatereturntype(\tenvthree, \tenvfour, \overname{\langle\tty\rangle}{\returntype}) \typearrow
  (\newtenv, \newreturntype)
}
\end{mathpar}

\subsubsection{TypingRule.DeclareOneFunc \label{sec:TypingRule.DeclareOneFunc}}
\hypertarget{def-declareonefunc}{}
The function
\[
  \declareonefunc(\overname{\staticenvs}{\tenv}, \overname{\func}{\funcsig})
  \aslto (\overname{\staticenvs}{\newtenv} \times \overname{\func}{\newfuncsig})
  \cup \overname{\TTypeError}{\TypeErrorConfig}
\]
checks that a subprogram defined by $\funcsig$ can be added to the static environment $\tenv$,
resulting in an annotated function definition $\newfuncdef$ and new static environment $\newtenv$.
\ProseOtherwiseTypeError

\subsection{Prose}
All of the following apply:
\begin{itemize}
  \item $\funcsig$ has name $\name$, arguments $\vargs$, and type $\subprogramtype$, that is,
  \[
    \begin{array}{rrcl}
      \funcsig \eqdef \{
                      & \funcname            &:& \name,\\
                      & \funcparameters      &:& \vp,\\
                      & \funcargs            &:& \vargs,\\
                      & \funcbody            &:& \SBASL(\bd),\\
                      & \funcreturntype      &:& \vt,\\
                      & \funcsubprogramtype  &:& \subprogramtype \\
                  \}  & ; &&
        \end{array}
  \]
  \item adding a new subprogram with $\name$, $\vargs$, and $\subprogramtype$ to $\tenv$ yields the new
        environment $\tenvone$ and new name $\namep$\ProseOrTypeError;
  \item checking that $\namep$ is not already declared in the global environment of $\tenvone$
        yields $\True$\ProseOrTypeError;
  \item ensuring that each setter has a getter given $\funcsig$ in $\tenv$ yields $\True$\ProseOrTypeError;
  \item $\funcsigone$ is $\funcsig$ with $\name$ substituted by $\nameone$;
  \item adding a subprogram with name $\namep$ and definition $\funcsigone$ to $\tenvone$ yields $\newtenv$\ProseOrTypeError.
\end{itemize}
\CodeSubsection{\DeclareOneFuncBegin}{\DeclareOneFuncEnd}{../Typing.ml}

\subsection{Formally}
\begin{mathpar}
\inferrule{
  {
  \begin{array}{rrcl}
    \funcsig \eqdef \{
                    & \funcname            &:& \name,\\
                    & \funcparameters      &:& \vp,\\
                    & \funcargs            &:& \vargs,\\
                    & \funcbody            &:& \SBASL(\bd),\\
                    & \funcreturntype      &:& \vt,\\
                    & \funcsubprogramtype  &:& \subprogramtype \\
                \} &&&
      \end{array}
    }\\
  \addnewfunc(\tenv, \name, \vargs, \subprogramtype) \typearrow
  (\tenvone, \namep) \OrTypeError\\\\
  \checkvarnotingenv{G^\tenvone, \namep} \typearrow \True \OrTypeError\\\\
  \checksetterhasgetter(\tenvone, \funcsig) \typearrow \True \OrTypeError\\\\
  {
  \begin{array}{rrcl}
    \newfuncsig \eqdef \{
                    & \funcname            &:& \namep,\\
                    & \funcparameters      &:& \vp,\\
                    & \funcargs            &:& \vargs,\\
                    & \funcbody            &:& \SBASL(\bd),\\
                    & \funcreturntype      &:& \vt,\\
                    & \funcsubprogramtype  &:& \subprogramtype \\
                \} &&&
      \end{array}
    }\\
    \addsubprogram(\tenvone, \namep, \funcsigone) \typearrow \newtenv \OrTypeError
}{
  \declareonefunc(\tenv, \funcsig) \typearrow (\newtenv, \newfuncsig)
}
\end{mathpar}\lrmcomment{This relates to \identi{HJRD}, \identd{BTBR}, \identi{FSFQ}, \identi{PFGQ}, and \identr{PGFC}.}

\subsubsection{TypingRule.SubprogramClash \label{sec:TypingRule.SubprogramClash}}
\hypertarget{def-subprogramclash}{}
The function
\[
  \subprogramclash(
    \overname{\staticenvs}{\tenv} \aslsep
    \overname{\Strings}{\name} \aslsep
    \overname{\subprogramtype}{\subpgmtype} \aslsep
    \overname{\ty^*}{\formaltypes}
  )
  \aslto
  \overname{\Bool}{\vb} \cup \overname{\TTypeError}{\TypeErrorConfig}
\]
checks whether the unique subprogram associated with $\name$
clashes with another subprogram
that has subprogram type $\subpgmtype$ and list of formal types $\formaltypes$,
yielding a Boolean value in $\vb$.
\ProseOtherwiseTypeError

The function is only defined when there exists a binding for $\name$ in the
$\subprograms$ map of $\tenv$.

\subsection{Prose}
All of the following apply:
\begin{itemize}
  \item the subprogram type associated with the unique subprogram named by $\name$ is $\namesubpgmtype$;
  \item applying $\subprogramtypesclash$ to $\namesubpgmtype$ and $\subpgmtype$ yields \\
        $\True$\ProseTerminateAs{\False}
        (that is, if both $\namesubpgmtype$ and $\subpgmtype$ are $\STGetter$ or both are $\STSetter$ then the
        subprogram types are considered to be non-clashing and the entire rule short-circuits to $\False$);
  \item $\nameargs$ is the list of pairs of types and identifiers associated with the function definition
        of $\name$ in $\tenv$;
  \item determining whether there is an argument clash between $\formaltypes$ and \\
        $\nameformals$ in $\tenv$ yields $\vb$\ProseOrTypeError.
\end{itemize}
\subsection{Formally}
\hypertarget{def-subprogramtypeclash}{}
We first introduce the helper predicate
\[
  \subprogramtypesclash(
    \overname{\subprogramtype}{\subpgmtypeone} \aslsep
    \overname{\subprogramtype}{\subpgmtypetwo}
  ) \aslto \overname{\Bool}{\vb}
\]
which defines whether two subprogram types are considered to be clashing:
\begin{mathpar}
\inferrule{
  \vbone \eqdef
  {
    \begin{array}{l}
  (\subpgmtypeone=\STGetter \land \subpgmtypetwo=\STSetter)\ \lor\\
  (\subpgmtypeone=\STSetter \land \subpgmtypetwo=\STGetter)
    \end{array}
  }\\\\
  \vb \eqdef \neg\vbone
}{
  \subprogramtypesclash(\subpgmtypeone, \subpgmtypetwo) \typearrow \vb
}
\end{mathpar}

\begin{mathpar}
\inferrule{
  \namesubpgmtype \eqdef G^\tenv.\subprograms(\name).\subprogramtype\\
  \subprogramtypesclash(\namesubpgmtype, \subpgmtype) \typearrow \True \terminateas \False\\
  \nameargs \eqdef G^\tenv.\subprograms(\name).\funcargs\\
  \hasargclash(\formaltypes, \nameargs) \typearrow \vb
}{
  \subprogramclash(\tenv, \namep, \subpgmtype, \formaltypes) \typearrow \vb
}
\end{mathpar}\lrmcomment{This is related to \identd{BTBR}, \identi{FSFQ}, \identi{PFGQ}.}

\subsubsection{TypingRule.AddNewFunc \label{sec:TypingRule.AddNewFunc}}
\hypertarget{def-addnewfunc}{}
The function
\[
  \begin{array}{rl}
  \addnewfunc(
    \overname{\staticenvs}{\tenv} \aslsep
    \overname{\identifier}{\name} \aslsep
    \overname{\typedidentifier^*}{\formals} \aslsep
    \overname{\subprogramtype}{\subpgmtype})
  & \aslto\\
  (\overname{\staticenvs}{\newtenv} \times \overname{\Strings}{\newname})
  \cup \overname{\TTypeError}{\TypeErrorConfig} &
  \end{array}
\]
ensures that the subprogram given by the identifier $\name$, list of formals $\formals$,
and subprogram type $\subpgmtype$ has a unique name among all the potential subprograms
that overload $\name$.
The result is the unique subprogram identifier $\newname$, which is used to distinguish it in the set
of overloaded subprograms (that is, other subprograms that share the same name)
and the environment $\newtenv$, which is updated with $\newname$.
\ProseOtherwiseTypeError

\subsection{Prose}
One of the following applies:
\begin{itemize}
  \item All of the following apply (\textsc{first\_name}):
  \begin{itemize}
    \item the $\overloadedsubprograms$ map in the global environment of $\tenv$ does not have a binding for $\name$;
    \item $\newtenv$ is $\tenv$ with the $\overloadedsubprograms$ updated by binding $\name$ to the singleton set containing
          $\name$.
  \end{itemize}

  \item All of the following apply (\textsc{name\_exists}):
  \begin{itemize}
    \item the $\overloadedsubprograms$ map in the global environment of $\tenv$ binds $\name$ to the set of strings $\othernames$;
    \item $\newname$ is the unique name that will be associated with the subprogram given by the identifier $\name$, list of formals $\formals$,
          and subprogram type $\subpgmtype$. It is constructed by concatenating a hyphen (\texttt{-}) to $\name$, followed
          by a string corresponding to the number of strings in $\othernames$.
          Notice that this is not an ASL identifier, as ASL identifiers do not contain hyphens, which ensures that this string
          does not occur in any specification;
    \item $\formaltypes$ is the list of types that appear in $\formals$ in the same order;
    \item checking for each $\namep$ in $\othernames$ whether the subprogram associated with $\namep$ clashes
          with the subprogram type $\subpgmtype$ and list of types $\formaltypes$ yields $\False$
          or a type error that indicates there are multiply defined subprograms, which short-circuits the entire rule;
    \item $\newtenv$ is $\tenv$ with the $\overloadedsubprograms$ updated by binding $\name$ to the union of $\othernames$ and
          $\{\newname\}$.
  \end{itemize}
\end{itemize}
\subsection{Formally}
\newcommand\stringconcat[0]{\hyperlink{def-stringconcat}{\texttt{+}}}
\newcommand\stringofint[0]{\hyperlink{def-stringofnat}{\texttt{string\_of\_nat}}}

We use the following functions to construct a unique string for each subprogram:
\begin{itemize}
\hypertarget{def-stringconcat}{}
\item The function $\stringconcat : \Strings \times \Strings \rightarrow \Strings$
concatenates two strings.

\hypertarget{def-stringofnat}{}
\item The function $\stringofint : \N \rightarrow \Strings$ converts a natural number
to the corresponding string.
\end{itemize}

\begin{mathpar}
\inferrule[first\_name]{
  G^\tenv.\overloadedsubprograms(\name) = \bot\\
  \newtenv \eqdef (G^\tenv.\overloadedsubprograms[\name\mapsto\{\name\}],  L^\tenv)
}{
  \addnewfunc(\tenv, \name, \formals, \subpgmtype) \typearrow
  (\newtenv, \overname{\name}{\newname})
}
\end{mathpar}

\begin{mathpar}
\inferrule[name\_exists]{
  G^\tenv.\overloadedsubprograms(\name) = \othernames\\
  k \eqdef \cardinality{\othernames}\\
  \newname \eqdef \name\ \stringconcat\ \texttt{"-"}\ \stringconcat\ \stringofint(k)\\
  \formaltypes \eqdef [(\id,\vt) \in \formals : \vt]\\
  {
    \left(\begin{array}{l}
  \namep \in \othernames: \\ \subprogramclash(\tenv, \namep, \subpgmtype, \formaltypes) \typearrow
  \vb_{\namep} \OrTypeError
    \end{array}\right)
  }\\\\
  \namep \in \othernames: \checktrans{\neg\vb_{\namep}}{\SubrogramDeclaredMultipleTimes} \typearrow \True \OrTypeError\\\\
  \newtenv \eqdef (G^\tenv.\overloadedsubprograms[\name\mapsto \othernames \cup \{\newname\}],  L^\tenv)
}{
  \addnewfunc(\tenv, \name, \formals, \subpgmtype) \typearrow
  (\newtenv, \newname)
}
\end{mathpar}\lrmcomment{This is related to \identr{PGFC}.}

\CodeSubsection{\AddNewFuncBegin}{\AddNewFuncEnd}{../Typing.ml}

\subsubsection{TypingRule.CheckSetterHasGetter \label{sec:TypingRule.CheckSetterHasGetter}}
\hypertarget{def-checksetterhashgetter}{}
The function
\[
  \checksetterhasgetter(\overname{\staticenvs}{\tenv} \aslsep \overname{\func}{\funcsig})
  \aslto \overname{\True}{\vb} \cup \overname{\TTypeError}{\TypeErrorConfig}
\]
checks whether the setter procedure given by $\funcsig$ has a corresponding getter,
returning $\True$ if this condition holds and a type error otherwise.

\subsection{Prose}
All of the following apply:
\begin{itemize}
  \item checking that the subprogram type of $\funcsig$ is one of $\STSetter$ and \\ $\STEmptySetter$
        has one of two outcomes:
        $\False$, which satisfies the premise;
        or $\True$, which short-circuits the entire rule
        (since the subprogram is not any kind of setter and no getter is required);
  \item \view\ the list of arguments of $\funcsig$ (that is, $\funcsig.\funcargs$) as follows:
        the \head\ is an argument that has the type $\rettype$;
        the \tail\ is a list with arguments that have the types $\argtypes$;
  \item applying $\subprogramforname$ to look up $\tenv$ for a subprogram with the name given by $\funcsig$ (that is, $\funcsig.\funcname$)
        yields a subprogram definition AST node $\funcsigp$\ProseOrTypeError;
  \item define $\watendgettertype$ as $\STGetter$ if $\funcsig.\subprogramtype$ is \\ $\STSetter$ and
        $\STEmptyGetter$ otherwise (meaning, $\funcsig.\subprogramtype$ is $\STEmptySetter$);
  \item checking that $\watendgettertype$ is the same as $\funcsigp.\funcsubprogramtype$ \\ yields $\True$\ProseOrTypeError;
  \item define $\argtypesp$ as the list of types appearing in the signature of $\funcsigp$ (that is, in $\funcsigp.\funcargs$);
  \item checking, for each index $\vi$ in the indices for $\argtypes$, that the type at $\argtypes[\vi]$ and the type at $\argtypesp[\vi]$
        are \typeequivalent\ yields $\True$\ProseOrTypeError;
  \item checking that $\rettype$ and $\funcsigp.\funcreturntype$ are \typeequivalent\ yields \\
        $\True$\ProseOrTypeError;
  \item define $\vb$ as $\True$ (that is, unless the rule short-circuited with a type error).
\end{itemize}
\CodeSubsection{\CheckSetterHasGetterBegin}{\CheckSetterHasGetterEnd}{../Typing.ml}

\subsection{Formally}
We define the helper function
\newcommand\matchsettertype[0]{\hyperlink{def-matchsettertype}{\texttt{match\_setter\_type}}}
\hypertarget{def-matchsettertype}{}
\[
  \matchsettertype \triangleq [\STSetter \mapsto \STGetter, \STEmptySetter \mapsto \STEmptyGetter] \enspace.
\]

\begin{mathpar}
\inferrule{
  \issetter \eqdef \funcsig.\subprogramtype \in \{ \STSetter, \STEmptySetter\}\\
  \booltrans{\issetter} \booltransarrow \False \terminateas \True\\\\
  \funcsig.\funcargs \eqname (\Ignore, \rettype) \concat \vargs\\
  \argtypes \eqdef [(\Ignore, \vt) \in \vargs: \vt]\\
  \subprogramforname(\tenv, \funcsig.\funcname, \argtypes) \typearrow (\Ignore, \Ignore, \funcsigp) \OrTypeError\\\\
  \matchsettertype(\funcsig.\funcsubprogramtype) \typearrow \watendgettertype\\
  \checktrans{\watendgettertype = \funcsigp.\funcsubprogramtype}{\SetterWithoutGetter} \typearrow \True \OrTypeError\\\\
  \argtypesp \eqdef [(\Ignore, \vt) \in \funcsigp.\funcargs : \vt]\\
  \vi\in\listrange(\argtypes): \typeequal(\argtypes[\vi], \argtypesp[\vi]) \typearrow \vb_\vi \OrTypeError\\\\
  \vi\in\listrange(\argtypes): \checktrans{\vb_\vi}{\SetterWithoutGetter} \typearrow \True \OrTypeError\\\\
  \typeequal(\rettype, \funcsigp.\funcreturntype) \typearrow \vb_\ret \OrTypeError\\\\
  \checktrans{\vb_\ret}{\SetterWithoutGetter} \typearrow \True \OrTypeError
}{
  \checksetterhasgetter(\tenv, \funcsig) \typearrow \overname{\True}{\vb}
}
\end{mathpar}

\subsubsection{TypingRule.AddSubprogram \label{sec:TypingRule.AddSubprogram}}
\hypertarget{def-addsubprogram}{}
The function
\[
  \addsubprogram(
    \overname{\staticenvs}{\tenv} \aslsep
    \overname{\Strings}{\name} \aslsep
    \overname{\func}{\funcdef})
   \aslto
  \overname{\staticenvs}{\newtenv}
\]
updates the global environment of $\tenv$ by mapping the (unique) subprogram identifier $\name$
to the function definition $\funcdef$ in $\tenv$, resulting in a new static environment $\newtenv$.

\subsection{Prose}
$\newtenv$ is $\tenv$ with its $\subprograms$ component updated by binding $\name$ to $\funcdef$.

\subsection{Formally}
\begin{mathpar}
\inferrule{
  \newtenv \eqdef (G^\tenv.\subprograms[\name\mapsto\funcdef], L^\tenv)
}{
  \addsubprogram(\tenv, \name, \funcdef) \typearrow \newtenv
}
\end{mathpar}
