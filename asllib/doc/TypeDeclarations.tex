\chapter{Type Declarations\label{chap:TypeDeclarations}}
Type declarations are grammatically derived from $\Ndecl$ via the subset of productions shown in
\secref{TypeDeclarationsSyntax} and represented as ASTs via the production of $\decl$
shown in \secref{TypeDeclarationsAbstractSyntax}.
%
Typing type declarations is done via $\declaretype$, which is defined in \nameref{sec:TypingRule.DeclareType}.
%
Type declarations have no associated semantics.

%%%%%%%%%%%%%%%%%%%%%%%%%%%%%%%%%%%%%%%%%%%%%%%%%%%%%%%%%%%%%%%%%%%%%%%%%%%%%%%%
\section{Syntax\label{sec:TypeDeclarationsSyntax}}
%%%%%%%%%%%%%%%%%%%%%%%%%%%%%%%%%%%%%%%%%%%%%%%%%%%%%%%%%%%%%%%%%%%%%%%%%%%%%%%%
\begin{flalign*}
\Ndecl  \derives \ & \Ttype \parsesep \Tidentifier \parsesep \Tof \parsesep \Ntydecl \parsesep \Nsubtypeopt \parsesep \Tsemicolon&\\
|\ & \Ttype \parsesep \Tidentifier \parsesep \Nsubtype \parsesep \Tsemicolon&\\
\Nsubtypeopt           \derives \ & \option{\Nsubtype} &\\
\Nsubtype \derives \ & \Tsubtypes \parsesep \Tidentifier \parsesep \Twith \parsesep \Nfields &\\
            |\              & \Tsubtypes \parsesep \Tidentifier &\\
\Nfields \derives \ & \Tlbrace \parsesep \TClist{\Ntypedidentifier} \parsesep \Trbrace &\\
\Nfieldsopt \derives \ & \Nfields \;|\; \emptysentence &\\
\Ntypedidentifier \derives \ & \Tidentifier \parsesep \Nasty &\\
\Nasty \derives \ & \Tcolon \parsesep \Nty &
\end{flalign*}

%%%%%%%%%%%%%%%%%%%%%%%%%%%%%%%%%%%%%%%%%%%%%%%%%%%%%%%%%%%%%%%%%%%%%%%%%%%%%%%%
\section{Abstract Syntax\label{sec:TypeDeclarationsAbstractSyntax}}
%%%%%%%%%%%%%%%%%%%%%%%%%%%%%%%%%%%%%%%%%%%%%%%%%%%%%%%%%%%%%%%%%%%%%%%%%%%%%%%%
\begin{flalign*}
\decl \derives\ & \DTypeDecl(\identifier, \ty, (\identifier, \overtext{\Field^*}{with fields})?) &\\
\Field \derives\ & (\identifier, \ty) &
\end{flalign*}

\subsubsection{ASTRule.GlobalDecl}
The relation
\[
  \builddecl : \overname{\parsenode{\Ndecl}}{\vparsednode} \;\aslrel\; \overname{\decl}{\vastnode}
\]
transforms a parse node $\vparsednode$ into an AST node $\vastnode$.

\hypertarget{build-typedecl}{}
\begin{mathpar}
\inferrule[type\_decl]{}
{
  {
    \begin{array}{c}
      \builddecl(\overname{\Ndecl(\Ttype, \Tidentifier(\vx), \Tof, \punnode{\Ntydecl}, \Nsubtypeopt, \Tsemicolon)}{\vparsednode})
  \astarrow \\
  \overname{\DTypeDecl(\vx, \astof{\vt}, \astof{\vsubtypeopt})}{\vastnode}
    \end{array}
  }
}
\end{mathpar}

\hypertarget{build-subtypedecl}{}
\begin{mathpar}
\inferrule[subtype\_decl]{
  \buildsubtype(\vsubtype) \astarrow \vs\\
  \vs \eqname (\name, \vfields)
}{
  {
    \begin{array}{c}
      \builddecl(\overname{\Ndecl(\Ttype, \Tidentifier(\vx), \Tof, \punnode{\Nsubtype}, \Tsemicolon)}{\vparsednode})
  \astarrow \\
  \overname{\DTypeDecl(\vx, \TNamed(\name), \langle(\name, \vfields)\rangle)}{\vastnode}
    \end{array}
  }
}
\end{mathpar}

\subsubsection{ASTRule.Subtype \label{sec:ASTRule.Subtype}}
\hypertarget{build-subtype}{}
The function
\[
  \buildsubtype(\overname{\parsenode{\Nsubtype}}{\vparsednode}) \aslto \overname{(\identifier \times (\identifier\times \ty)^*)}{\vastnode}
\]
transforms a parse node $\vparsednode$ into an AST node $\vastnode$.

\hypertarget{build-subtype}{}
\begin{mathpar}
\inferrule[with\_fields]{}{
  {
    \begin{array}{r}
  \buildsubtype(\overname{\Nsubtype(
    \Tsubtypes, \Tidentifier(\id), \Twith, \punnode{\Nfields}
    )}{\vparsednode})
  \astarrow \\
  \overname{(\id, \astof{\vfields})}{\vastnode}
  \end{array}
  }
}
\end{mathpar}

\begin{mathpar}
  \inferrule[no\_fields]{}{
  \buildsubtype(\overname{\Nsubtype(
    \Tsubtypes, \Tidentifier(\id))}{\vparsednode})
  \astarrow
  \overname{(\id, \emptylist)}{\vastnode}
}
\end{mathpar}

\subsubsection{ASTRule.Subtypeopt \label{sec:ASTRule.Subtypeopt}}
\hypertarget{build-subtypeopt}{}
The function
\[
   \buildsubtypeopt(\overname{\parsenode{\Nsubtypeopt}}{\vparsednode}) \aslto
    \overname{\langle(\identifier \times \langle (\identifier\times \ty)^* \rangle)\rangle}{\vastnode}
\]
transforms a parse node $\vparsednode$ into an AST node $\vastnode$.

\begin{mathpar}
\inferrule[subtype\_opt]{
  \buildoption[\Nsubtype](\vsubtypeopt) \astarrow \vastnode
}{
  \buildsubtypeopt(\overname{\Nsubtypeopt(\namednode{\vsubtypeopt}{\option{\Nsubtype}})}{\vparsednode}) \astarrow \vastnode
}
\end{mathpar}

\subsubsection{ASTRule.Fields \label{sec:ASTRule.Fields}}
\hypertarget{build-fields}{}
The function
\[
  \buildfields(\overname{\parsenode{\Nfields}}{\vparsednode}) \;\aslto\; \overname{\Field^*}{\vastnode}
\]
transforms a parse node $\vparsednode$ into an AST node $\vastnode$.

\begin{mathpar}
\inferrule{
  \buildtclist[\buildtypedidentifier](\vfields) \astarrow \vfieldasts
}{
  \buildfields(\Nfields(\Tlbrace, \namednode{\vfields}{\TClist{\Ntypedidentifier}}, \Trbrace)) \astarrow
  \overname{\vfieldasts}{\vastnode}
}
\end{mathpar}

\subsubsection{ASTRule.FieldsOpt \label{sec:ASTRule.FieldsOpt}}
\hypertarget{build-fieldsopt}{}
The function
\[
  \buildfieldsopt(\overname{\parsenode{\Nfieldsopt}}{\vparsednode}) \;\aslto\; \overname{\Field^*}{\vastnode}
\]
transforms a parse node $\vparsednode$ into an AST node $\vastnode$.

\begin{mathpar}
\inferrule[fields]{}{
  \buildfieldsopt(\Nfieldsopt(\punnode{\Nfields})) \astarrow
  \overname{\astof{\vfields}}{\vastnode}
}
\end{mathpar}

\begin{mathpar}
\inferrule[empty]{}{
  \buildfieldsopt(\Nfieldsopt(\emptysentence)) \astarrow
  \overname{\emptylist}{\vastnode}
}
\end{mathpar}

%%%%%%%%%%%%%%%%%%%%%%%%%%%%%%%%%%%%%%%%%%%%%%%%%%%%%%%%%%%%%%%%%%%%%%%%%%%%%%%%
\section{Typing}
%%%%%%%%%%%%%%%%%%%%%%%%%%%%%%%%%%%%%%%%%%%%%%%%%%%%%%%%%%%%%%%%%%%%%%%%%%%%%%%%

We also define the following helper rules:
\begin{itemize}
  \item TypingRule.AnnotateTypeOpt (see \secref{TypingRule.AnnotateTypeOpt})
  \item TypingRule.AnnotateExprOpt (see \secref{TypingRule.AnnotateExprOpt})
  \item TypingRule.AnnotateInitType (see \secref{TypingRule.AnnotateInitType})
  \item TypingRule.AddGlobalStorage (see \secref{TypingRule.AddGlobalStorage})
  \item TypingRule.DeclareType (see \secref{TypingRule.DeclareType})
  \item TypingRule.AnnotateExtraFields (see \secref{TypingRule.AnnotateExtraFields})
  \item TypingRule.AnnotateEnumLabels (see \secref{TypingRule.AnnotateEnumLabels})
  \item TypingRule.DeclareConst (see \secref{TypingRule.DeclareConst})
\end{itemize}

\subsubsection{TypingRule.DeclareType \label{sec:TypingRule.DeclareType}}
\hypertarget{def-declaretype}{}
The function
\[
\declaretype(\overname{\globalstaticenvs}{\genv} \aslsep
            \overname{\identifier}{\name} \aslsep
            \overname{\ty}{\tty} \aslsep
            \overname{\langle(\identifier\times\field^*)\rangle}{\vs}
) \aslto \overname{\globalstaticenvs}{\newgenv}
\cup \overname{\TTypeError}{\TypeErrorConfig}
\]
declares a type named $\name$ with type $\tty$ and \optional\ additional fields
over another type $\vs$
in the global static environment $\genv$, resulting in the modified global static environment $\newgenv$.
\ProseOtherwiseTypeError

\subsubsection{Prose}
All of the following apply:
\begin{itemize}
  \item checking that $\name$ is not already declared in the global environment of $\genv$ yields $\True$\ProseOrTypeError;
  \item define $\tenv$ as the static environment whose global component is $\genv$ and its local component is the empty local
        static environment;
  \item annotating the \optional\ extra fields $\vs$ for $\tty$ in $\tenv$ yields via \\ $\annotateextrafields$
        yields the modified environment $\tenvone$ and type $\vtone$\ProseOrTypeError;
  \item annotating $\vtone$ in $\tenvone$ yields $(\vttwo, \vsest)$\ProseOrTypeError;
  \item applying $\maxtimeframe$ to $\vsest$ yields $\vtimeframe$;
  \item applying $\addtype$ to $\name$, $\vttwo$, and $\vtimeframe$ in $\tenv$ yields $\tenvtwo$;
  \item $\tenvtwo$ is $\tenvone$ with its $\declaredtypes$ component updated by binding $\name$ to $\vttwo$;
  \item One of the following applies:
  \begin{itemize}
    \item All of the following apply (\textsc{enum}):
    \begin{itemize}
      \item $\vttwo$ is an enumeration type with labels $\ids$, that is, $\TEnum(\ids)$;
      \item applying $\declareenumlabels$ to $\vttwo$ in $\tenvtwo$ $\newtenv$\ProseOrTypeError.
    \end{itemize}

    \item All of the following apply (\textsc{not\_enum}):
    \begin{itemize}
      \item $\vttwo$ is not an enumeration type;
      \item $\newtenv$ is $\tenvtwo$.
    \end{itemize}
  \end{itemize}
\end{itemize}

\subsubsection{Formally}
\begin{mathpar}
\inferrule[enum]{
  \checkvarnotingenv{\genv, \name} \typearrow \True \OrTypeError\\\\
  \withemptylocal(\genv) \typearrow \tenv\\\\
  \annotateextrafields(\tenv, \tty, \vs) \typearrow (\tenvone, \vtone)\\
  \annotatetype{\True, \tenvone, \vtone} \typearrow (\vttwo, \vsest) \OrTypeError\\\\
  \maxtimeframe(\vsest) \typearrow \vtimeframe\\
  \addtype(\tenvone, \name, \vttwo, \vtimeframe) \typearrow \tenvtwo\\
  \commonprefixline\\\\
  \vttwo = \TEnum(\ids)\\
  \declareenumlabels(\tenvtwo, \vttwo) \typearrow \newtenv \OrTypeError
}{
  \declaretype(\genv, \name, \tty, \vs) \typearrow \newtenv
}
\end{mathpar}

\begin{mathpar}
\inferrule[not\_enum]{
  \checkvarnotingenv{\genv, \name} \typearrow \True \OrTypeError\\\\
  \withemptylocal(\genv) \typearrow \tenv\\\\
  \annotateextrafields(\tenv, \tty, \vs) \typearrow (\tenvone, \vtone)\\
  \annotatetype{\True, \tenvone, \vtone} \typearrow (\vttwo, \vsest) \OrTypeError\\\\
  \maxtimeframe(\vsest) \typearrow \vtimeframe\\
  \addtype(\tenvone, \name, \vttwo, \vtimeframe) \typearrow \tenvtwo\\
  \commonprefixline\\\\
  \astlabel(\vttwo) \neq \TEnum
}{
  \declaretype(\genv, \name, \tty, \vs) \typearrow \overname{\tenvtwo}{\newtenv}
}
\end{mathpar}
\CodeSubsection{\DeclareTypeBegin}{\DeclareTypeEnd}{../Typing.ml}
\lrmcomment{This is related to \identr{DHRC}, \identd{YZBQ}, \identr{DWSP}, \identi{MZXL}, \identr{MDZD}, \identr{CHKR}.}

\subsubsection{TypingRule.AnnotateExtraFields \label{sec:TypingRule.AnnotateExtraFields}}
\hypertarget{def-annotateextrafields}{}
The function
\[
\begin{array}{r}
\annotateextrafields(\overname{\staticenvs}{\tenv} \aslsep
  \overname{\ty}{\tty} \aslsep
  \overname{\langle(\overname{\identifier}{\vsuper}\times\overname{\field^*}{\extrafields})\rangle}{\vs}
)
\aslto \\
(\overname{\staticenvs}{\newtenv} \times \overname{\ty}{\newty})
\cup \overname{\TTypeError}{\TypeErrorConfig}
\end{array}
\]
annotates the type $\tty$ with the \optional\ extra fields $\vs$ in $\tenv$, yielding
the modified environment $\newtenv$ and type $\newty$.
\ProseOtherwiseTypeError

\subsubsection{Prose}
One of the following applies:
\begin{itemize}
  \item All of the following apply (\textsc{none}):
  \begin{itemize}
    \item $\vs$ is $\None$;
    \item $\newtenv$ is $\tenv$;
    \item $\newty$ is $\tty$.
  \end{itemize}

  \item All of the following apply (\textsc{empty\_fields}):
  \begin{itemize}
    \item $\vs$ is $\langle(\vsuper, \extrafields)\rangle$;
    \item checking that $\tty$ \subtypesatisfies\ the named type $\vsuper$ (that is, \\ $\TNamed(\vsuper)$) yields
          $\True$\ProseOrTypeError;
    \item $\extrafields$ is the empty list;
    \item $\newtenv$ is $\tenv$;
    \item $\newty$ is $\tty$.
  \end{itemize}

  \item All of the following apply (\textsc{no\_super}):
  \begin{itemize}
    \item $\vs$ is $\langle(\vsuper, \extrafields)\rangle$;
    \item checking that $\tty$ \subtypesatisfies\ the named type $\vsuper$ (that is, \\ $\TNamed(\vsuper)$) yields
          $\True$\ProseOrTypeError;
    \item $\extrafields$ is not an empty list;
    \item $\vsuper$ is not bound to a type in $\tenv$;
    \item the result is a type error indicating that $\vsuper$ is not a declared type.
  \end{itemize}

  \item All of the following apply (\textsc{structured}):
  \begin{itemize}
    \item $\vs$ is $\langle(\vsuper, \extrafields)\rangle$;
    \item checking that $\tty$ \subtypesatisfies\ the named type $\vsuper$ (that is,\\ $\TNamed(\vsuper)$) yields
          $\True$\ProseOrTypeError;
    \item $\extrafields$ is not an empty list;
    \item $\vsuper$ is bound to a type $\vt$ in $\tenv$;
    \item checking that $\vt$ is a \structuredtype\ yields $\True$ or a type error
          indicating that a \structuredtype\ was expected, thereby short-circuiting the entire rule;
    \item $\vt$ has AST label $L$ and fields $\fields$;
    \item $\newty$ is the type with AST label $L$ and list fields that is the concatenation of $\fields$ and $\extrafields$;
    \item $\newtenv$ is $\tenv$ with its $\subtypes$ component updated by binding $\name$ to $\vsuper$.
  \end{itemize}
\end{itemize}

\subsubsection{Formally}
\begin{mathpar}
\inferrule[none]{}{
  \annotateextrafields(\tenv, \tty, \overname{\None}{\vs}) \typearrow (\overname{\tenv}{\newtenv}, \overname{\tty}{\newty})
}
\end{mathpar}

\begin{mathpar}
\inferrule[empty\_fields]{
  \subtypesat(\tty, \TNamed(\vsuper)) \typearrow \vb\\
  \checktrans{\vb}{TypeConflict} \typearrow \True \OrTypeError\\\\
  \extrafields = \emptylist
}{
  \annotateextrafields(\tenv, \tty, \overname{\langle(\vsuper, \extrafields)\rangle}{\vs}) \typearrow (\overname{\tenv}{\newtenv}, \overname{\tty}{\newty})
}
\end{mathpar}

\begin{mathpar}
\inferrule[no\_super]{
  \subtypesat(\tty, \TNamed(\vsuper)) \typearrow \vb\\
  \checktrans{\vb}{TypeConflict} \typearrow \True \OrTypeError\\\\
  \extrafields \neq \emptylist\\\\
  G^\tenv.\declaredtypes(\vsuper) = \bot
}{
  \annotateextrafields(\tenv, \tty, \overname{\langle(\vsuper, \extrafields)\rangle}{\vs}) \typearrow
  \TypeErrorVal{\UndefinedIdentifier}
}
\end{mathpar}

\begin{mathpar}
\inferrule[structured]{
  \subtypesat(\tty, \TNamed(\vsuper)) \typearrow \vb\\
  \checktrans{\vb}{TypeConflict} \typearrow \True \OrTypeError\\\\
  \extrafields \neq \emptylist\\\\
  G^\tenv.\declaredtypes(\vsuper) = (\vt, \Ignore)\\
  {
    \begin{array}{r}
  \checktrans{\astlabel(\vt) \in \{\TRecord, \TException\}}{ExpectedStructuredType} \typearrow \\ \True \OrTypeError
    \end{array}
  }\\
  \vt \eqname L(\fields)\\
  \newty \eqdef L(\fields \concat \extrafields)\\
  \newtenv \eqdef (G^\tenv.\subtypes[\name\mapsto\vsuper], L^\tenv)
}{
  \annotateextrafields(\tenv, \tty, \overname{\langle(\vsuper, \extrafields)\rangle}{\vs}) \typearrow (\newtenv, \newty)
}
\end{mathpar}

\subsubsection{TypingRule.AnnotateTypeOpt \label{sec:TypingRule.AnnotateTypeOpt}}
\hypertarget{def-annotatetypeopt}{}
The function
\[
\annotatetypeopt(\overname{\staticenvs}{\tenv} \aslsep \overname{\langle\overname{\ty}{\vt}\rangle}{\tyopt})
\typearrow \overname{\langle\ty\rangle}{\tyoptp}
\cup \overname{\TTypeError}{\TypeErrorConfig}
\]
annotates the type $\vt$ inside an \optional\ $\tyopt$, if there is one, and leaves it as is if $\tyopt$ is $\None$.
\ProseOtherwiseTypeError

\subsubsection{Prose}
One of the following applies:
\begin{itemize}
  \item All of the following apply (\textsc{none}):
  \begin{itemize}
    \item $\tyopt$ is $\None$;
    \item $\tyoptp$ is $\tyopt$.
  \end{itemize}

  \item All of the following apply (\textsc{some}):
  \begin{itemize}
    \item $\tyopt$ contains the type $\vt$;
    \item annotating $\vt$ in $\tenv$ yields $\vtone$\ProseOrTypeError;
    \item $\tyoptp$ is $\langle\vtone\rangle$.
  \end{itemize}
\end{itemize}

\subsubsection{Formally}
\begin{mathpar}
\inferrule[none]{}{
  \annotatetypeopt(\tenv, \overname{\None}{\tyopt}) \typearrow \overname{\tyopt}{\tyoptp}
}
\and
\inferrule[some]{
  \annotatetype{\tenv, \vt} \typearrow \vtone \OrTypeError
}{
  \annotatetypeopt(\tenv, \overname{\langle\vt\rangle}{\tyopt}) \typearrow\overname{\langle\vtone\rangle}{\tyoptp}
}
\end{mathpar}

\subsubsection{TypingRule.AnnotateExprOpt \label{sec:TypingRule.AnnotateExprOpt}}
\hypertarget{def-annotateexpropt}{}
The function
\[
  \annotateexpropt(\overname{\staticenvs}{\tenv} \aslsep \overname{\langle\ve\rangle}{\expropt})
  \aslto \overname{(\langle\expr\rangle \times \langle\ty\rangle)}{\vres}
  \cup \overname{\TTypeError}{\TypeErrorConfig}
\]
annotates the \optional\ expression $\expropt$ in $\tenv$ and returns a pair of \optional\ expressions
for the type and annotated expression in $\vres$.
\ProseOtherwiseTypeError

\subsubsection{Prose}
One of the following applies:
\begin{itemize}
  \item All of the following apply (\textsc{none}):
  \begin{itemize}
    \item $\expropt$ is $\None$;
    \item $\vres$ is $(\None, \None)$.
  \end{itemize}

  \item All of the following apply (\textsc{some}):
  \begin{itemize}
    \item $\expropt$ contains the expression $\ve$;
    \item annotating $\ve$ in $\tenv$ yields $(\vt, \vep)$\ProseOrTypeError;
    \item $\vres$ is $(\langle\vt\rangle, \langle\vep\rangle)$.
  \end{itemize}
\end{itemize}

\subsubsection{Formally}
\begin{mathpar}
\inferrule[none]{}{
  \annotateexpropt(\tenv, \overname{\None}{\expropt}) \typearrow (\None, \None)
}
\and
\inferrule[some]{
  \annotateexpr{\tenv, \ve} \typearrow (\vt, \vep)\OrTypeError
}{
  \annotateexpropt(\tenv, \overname{\langle\ve\rangle}{\expropt}) \typearrow \overname{(\langle\vt\rangle, \langle\vep\rangle)}{\vres}
}
\end{mathpar}

\subsubsection{TypingRule.AnnotateInitType \label{sec:TypingRule.AnnotateInitType}}
\hypertarget{def-annotateinittype}{}
The function
\[
  \annotateinittype(\overname{\staticenvs}{\tenv} \aslsep
  \overname{\langle\ty\rangle}{\initialvaluetype} \aslsep
  \overname{\langle\ty\rangle}{\typeannotation}
  )
  \aslto \overname{\ty}{\declaredtype}
  \cup \overname{\TTypeError}{\TypeErrorConfig}
\]
takes the \optional\ type associated with the initialization value of a global storage declaration --- $\initialvaluetype$ ---
and the \optional\ type annotation for the same global storage declaration --- $\typeannotation$ ---
and chooses the type that should be associated with the declaration --- $\declaredtype$ -- in $\tenv$.
\ProseOtherwiseTypeError

The ASL parser ensures that at least one of $\initialvaluetype$ and \\
$\typeannotation$ should not be $\None$.

\subsubsection{Prose}
One of the following applies:
\begin{itemize}
  \item All of the following apply (\textsc{both}):
  \begin{itemize}
    \item $\initialvaluetype$ is $\langle\vtone\rangle$ and $\typeannotation$ is $\langle\vttwo\rangle$;
    \item checking that $\vtone$ \typesatisfies\ $\vttwo$ in $\tenv$ yields $\True$\ProseOrTypeError;
    \item $\declaredtype$ is $\vtone$.
  \end{itemize}

  \item All of the following apply (\textsc{annotated}):
  \begin{itemize}
    \item $\initialvaluetype$ is $\None$ and $\typeannotation$ is $\langle\vttwo\rangle$;
    \item $\declaredtype$ is $\vttwo$.
  \end{itemize}

  \item All of the following apply (\textsc{initial}):
  \begin{itemize}
    \item $\initialvaluetype$ is $\langle\vtone\rangle$ and $\typeannotation$ is $\None$;
    \item $\declaredtype$ is $\vtone$.
  \end{itemize}
\end{itemize}

\begin{mathpar}
\inferrule[both]{
  \checktypesat(\tenv, \vtone, \vttwo) \typearrow \True \OrTypeError
}{
  \annotateinittype(\tenv, \overname{\langle\vtone\rangle}{\initialvaluetype}, \overname{\langle\vttwo\rangle}{\typeannotation})
  \typearrow \vttwo
}
\and
\inferrule[annotated]{}{
  \annotateinittype(\tenv, \overname{\None}{\initialvaluetype}, \overname{\langle\vttwo\rangle}{\typeannotation})
  \typearrow \vttwo
}
\and
\inferrule[initial]{}{
  \annotateinittype(\tenv, \overname{\langle\vtone\rangle}{\initialvaluetype}, \overname{\None}{\typeannotation})
  \typearrow \vtone
}
\end{mathpar}

\hypertarget{def-declaredtype}{}
\subsubsection{TypingRule.DeclaredType \label{sec:TypingRule.DeclaredType}}
The function
\[
  \declaredtype(\overname{\staticenvs}{\tenv} \aslsep \overname{\identifier}{\id}) \aslto \overname{\ty}{\vt} \cup \TTypeError
\]
retrieves the type associated with the identifier $\id$ in the static environment $\tenv$.
If the identifier is not associated with a declared type, a type error is returned.

\subsubsection{Prose}
One of the following applies:
\begin{itemize}
  \item All of the following apply (\textsc{exists}):
  \begin{itemize}
    \item $\id$ is bound in the global environment to the type $\vt$.
  \end{itemize}

  \item All of the following apply (\textsc{type\_not\_declared}):
  \begin{itemize}
    \item $\id$ is not bound in the global environment to any type;
    \item the result is a type error indicating the lack of a type declaration for $\id$ (\UndefinedIdentifier).
  \end{itemize}
\end{itemize}

\subsubsection{Formally}
\begin{mathpar}
\inferrule[exists]{
  G^\tenv.\declaredtypes(\id) = (\vt, \Ignore)
}{
  \declaredtype(\tenv, \id) \typearrow \vt
}
\end{mathpar}

\subsubsection{TypingRule.AnnotateEnumLabels \label{sec:TypingRule.AnnotateEnumLabels}}
\hypertarget{def-annotateenumlabels}{}
The function
\[
\declareenumlabels(\overname{\staticenvs}{\tenv} \aslsep
  \overname{\identifier}{\name} \aslsep
  \overname{\identifier^+}{\ids} \aslsep
  \aslto \overname{\staticenvs}{\newtenv}
  \cup \overname{\TTypeError}{\TypeErrorConfig}
)
\]
updates the static environment $\tenv$ with the identifiers $\ids$ listed by an enumeration type,
yielding the modified environment $\newtenv$.
\ProseOtherwiseTypeError

\subsubsection{Prose}
All of the following apply:
\begin{itemize}
  \item $\ids$ is the (non-empty) list of labels $\id_{1..k}$;
  \item $\tenv_0$ is $\tenv$;
  \item declaring the constant $\id_i$ with the type $\TNamed(\name)$ and literal $\llabel(\id_i)$ in $\tenv_{i-1}$
        via $\declareconst$
        yields $\tenv_i$, for $i=1 $ to $k$ (if $k>1$)\ProseOrTypeError;
  \item $\newtenv$ is $\tenv_k$.
\end{itemize}

\subsubsection{Formally}
\begin{mathpar}
\inferrule{
  \ids \eqname \id_{1..k}\\
  \tenv_0 \eqdef \tenv\\
  {
  \begin{array}{r}
    \vi=1..k: \declareconst(\tenv_{\vi-1}, \id_\vi, \TNamed(\name), \llabel(\id_\vi)) \typearrow \\
    \tenv_{\vi} \OrTypeError
  \end{array}
  }
}{
  \declareenumlabels(\tenv, \name, \ids) \typearrow \overname{\tenv_k}{\newtenv}
}
\end{mathpar}

\subsubsection{TypingRule.DeclareConst \label{sec:TypingRule.DeclareConst}}
\hypertarget{def-declareconst}{}
The function
\[
\declareconst(\overname{\globalstaticenvs}{\genv} \aslsep
              \overname{\identifier}{\name} \aslsep
              \overname{\ty}{\tty} \aslsep
              \overname{\literal}{vv})
              \aslto
              \overname{\globalstaticenvs}{\newgenv} \cup \overname{\TTypeError}{\TypeErrorConfig}
\]
adds a constant given by the identifier $\name$, type $\tty$, and literal $\vv$ to the
global static environment $\genv$, yielding the modified environment $\newgenv$.
\ProseOtherwiseTypeError

\subsubsection{Prose}
All of the following apply:
\begin{itemize}
  \item adding the global storage given by the identifier $\name$, global declaration keyword $\GDKConstant$,
        and type $\tty$ to $\genv$ yields $\genvone$;
  \item applying $\addglobalconstant$ to $\name$ and $\vv$ in $\genvone$ yields $\newgenv$.
\end{itemize}

\subsubsection{Formally}
\begin{mathpar}
\inferrule{
  \addglobalstorage(\genv, \name, \GDKConstant, \tty) \typearrow \genvone\\
  \addglobalconstant(\genvone, \name, \vv) \typearrow \newgenv
}{
  \declareconst(\genv, \name, \tty, \vv) \typearrow \newgenv
}
\end{mathpar}
