\chapter{Type Inference and Typechecking Definitions\label{chap:TypeChecking}}

The purpose of the ASL type system is to describe, in a formal and authoritative way,
which ASL specifications are considered \emph{well-typed}.
Whether a specification is well-typed is defined in terms of a \emph{type system}~\cite{TypeSystemsLucaCardelli}.
That is, a set of \emph{typing rules}.
Typing a specification consists of annotating the root of its AST with the rules defined
in the remainder of this document.

An ASL parser accepts an ASL specification and checks whether it is valid with respect to the syntax of ASL,
which is defined in \secref{ASLGrammar}.
If the specification is syntactically valid, the parser returns an \emph{abstract syntax tree} (AST, for short),
which represents the specification as a labelled structured tree. Otherwise, it returns a syntax error.
When an ASL specification is successfully parsed, we refer to the resulting AST as the \emph{untyped AST}.

A \emph{typechecker} is an implementation of the ASL type system, which accepts an untyped AST and applies the
rules of the type system to the untyped AST. If it is successful, the specification
is considered \emph{well-typed} and the result is a pair consisting of
a \emph{static environment} and a \emph{typed AST},
which are used in defining the ASL semantics (\chapref{Semantics}).
Otherwise, the typechecker returns a type error.

\hypertarget{def-annotaterel}{}
The type system of ASL is given by the relation $\annotaterel$, which is defined as the disjoint union
of the functions and relations defined in this reference.
The functions and relations in this reference are defined, in turn, via type system rules.

Types are represented by respective Abstract Syntax Trees derived from the non-terminal $\ty$.
Throughout this document we use $\tty$ to denote a type variable, which should not be confused with the abstract syntax variable $\ty$.

\section{Static Environments \label{sec:StaticEnvironments}}

A \emph{static environment} (also called a \emph{type environment}) is what the typing rules operate over:
a structure, which amongst other things, associates types to variables.
Intuitively, the typing of a specification makes an initial environment evolve, with new types as given by the
variable declarations of the specification.

\begin{definition}
\hypertarget{def-staticenvs}{}
Static environments, denoted as $\staticenvs$, are defined as follows (referring to symbols defined by the abstract syntax):
\[
\begin{array}{rcl}
\staticenvs 	          &\triangleq& \globalstaticenvs \times \localstaticenvs \hypertarget{def-globalstaticenvs}{}\\
\\
\globalstaticenvs &\triangleq& \left[
\begin{array}{lcl}
  \declaredtypes        &\mapsto& \identifier \partialto \ty \times \TTimeFrame\\
  \constantvalues       &\mapsto& \identifier \partialto \literal,\\
  \globalstoragetypes   &\mapsto& \identifier \partialto \ty \times \globaldeclkeyword,\\
  \exprequiv            &\mapsto& \identifier \partialto \expr,\\
  \subtypes             &\mapsto& \identifier \partialto \identifier,\\
  \subprograms          &\mapsto& \identifier \partialto \func \times \TSideEffectSet,\\
  \overloadedsubprograms  &\mapsto& \identifier \rightarrow \pow{\Strings}
\end{array}
\right]\\
\hypertarget{def-localstaticenvs}{}\\
\localstaticenvs &\triangleq& \left[
\begin{array}{lcl}
  \constantvalues       &\mapsto& \identifier \partialto \literal,\\
  \localstoragetypes    &\mapsto& \identifier \partialto \ty \times \globaldeclkeyword,\\
  \exprequiv            &\mapsto& \identifier \partialto \expr,\\
  \returntype           &\mapsto& \langle \ty \rangle
\end{array}
\right]\\
\end{array}
\]
\end{definition}

We use $\tenv$ and similar variable names (for example, $\tenvone$ and $\newtenv$) to range over static environments.

A static environment $\tenv=(G^\tenv, L^\tenv)$ consists of two
distinct components: the global environment $G^\tenv \in \globalstaticenvs$ --- pertaining to AST nodes
appearing outside of a given subprogram, and the local environment
$L^\tenv \in \localstaticenvs$ --- pertaining to AST nodes appearing inside a given subprogram.
This separation allows us to typecheck subprograms by using an empty local environment.

The intuitive meaning of each component is as follows:
\begin{itemize}
  \hypertarget{def-declaredtypes}{}
  \item $\declaredtypes$ assigns types to their declared names and \timeframeterm\
  (the \hyperlink{def-maxtimeframe}{maximal time frame} of any \sideeffectdescriptorterm\ inferred for
  the type definition);
  \hypertarget{def-constantvalues}{}
  \item $\constantvalues$ assigns literals to their declaring (constant) names;
  \hypertarget{def-globalstoragetypes}{}
  \item $\globalstoragetypes$ associates names of global storage elements to their inferred type
  and how they were declared --- as constants, configuration variables, \texttt{let} variables,
  or mutable variables;
  \hypertarget{def-localstoragetypes}{}
  \item $\localstoragetypes$ associates names of local storage elements to their inferred type
  and how they were declared --- as variables, constants, or as \texttt{let} variables;
  \hypertarget{def-exprequiv}{}
  \item $\exprequiv$ associates names of immutable storage elements to a simplified version
  of their initializing expression;
  \hypertarget{def-subtypes}{}
  \item $\subtypes$ associates type names to the names that their type subtypes;
  \hypertarget{def-subprograms}{}
  \item $\subprograms$ associates names of subprograms to the $\func$ AST node they were
  declared with and the set of \sideeffectdescriptorsterm\ inferred for them;
  \hypertarget{def-overloadedsubprograms}{}
  \item $\overloadedsubprograms$ associates names of subprograms to the set of overloading
  subprograms ---  $\func$ AST nodes that share the same name;
  \hypertarget{def-returntype}{}
  \item $\returntype$ contains the name of the type that a subprogram declares, if it is
  a function or a getter.
\end{itemize}

\hypertarget{def-emptytenv}{}
\begin{definition}[Empty Static Environment]
The \emph{empty static environment}, \\ denoted as $\emptytenv$, is defined as follows:
\[
\emptytenv \triangleq \left(
  \overname{
    \left[
\begin{array}{lcl}
  \declaredtypes        &\mapsto& \emptyfunc,\\
  \constantvalues       &\mapsto& \emptyfunc,\\
  \globalstoragetypes   &\mapsto& \emptyfunc,\\
  \exprequiv            &\mapsto& \emptyfunc,\\
  \subtypes             &\mapsto& \emptyfunc,\\
  \subprograms          &\mapsto& \emptyfunc,\\
  \overloadedsubprograms  &\mapsto& \emptyfunc
\end{array}
\right]}{\globalstaticenvs},
\overname{
 \left[
\begin{array}{lcl}
  \constantvalues       &\mapsto& \emptyfunc,\\
  \localstoragetypes    &\mapsto& \emptyfunc,\\
  \exprequiv            &\mapsto& \emptyfunc,\\
  \returntype           &\mapsto& \None
\end{array}
\right]}{\localstaticenvs}
\right)
\]
\end{definition}

The global environment and local environment consist of various components.
We use the notation $G^\tenv.m$ and $L^\tenv.m$ to access the $m$ component of a given environment.

To update a function component $f$ (e.g., $\declaredtypes$) of a global or local environment $E$
with a new mapping $x \mapsto v$, we use the notation $\tenv.f[x \mapsto v]$ to stand for $E[f \mapsto E.f[x \mapsto v]]$.

\identd{JRXM} \identi{ZTMQ}

\section{Typing Rule Configurations}
The output configurations of type system assertions have two flavors:
\begin{description}
  \item[Normal Outputs.]
  Configurations are typically tuples with different combinations
  of \emph{static environments}, types, and Boolean values.

  \hypertarget{def-typeerror}{}
  \item[Type Errors.] Configurations in $\TypeError(\Strings)$
  represent type errors, for example, using an integer type as a condition expression, as in \verb|if 5 then 1 else 2|.
  The ASL type system is designed such that when these \emph{type error configurations} appear,
  the typing of the entire specification terminates by outputting them.
\end{description}

We define the mathematical type of type error configurations
(which is needed to define the types of functions in the ASL type system)
as follows:
\hypertarget{def-ttypeerror}{}
\[
  \TTypeError \triangleq \{\TypeErrorVal{\vs} \;|\; \vs \in \Strings \} \enspace.
\]

\hypertarget{def-typeerrorconfig}{}
and the shorthand $\TypeErrorConfig \triangleq \TypeError(\vs)$ for type error configurations.

% Specifically,\ProseOrTypeError\ means: ``or a type error configuration $\TypeErrorConfig$, which short-circuits the rule,
% making it transition into the type error configuration $\TypeErrorConfig$.''.
%
When several \hyperlink{def-caserules}{case rules} for the same function use the same short-circuiting transition assertion,
we do not repeat the\ProseOrTypeError, but rather include it only in the first rule.

% \subsection*{Rule Example}
% The following rule is used to type a sequence of two statements:
% \[
% \inferrule{
%   \annotatestmt{\tenv, \vs1} = (\newsone, \tenvone)\\
%   \annotatestmt{\tenvone, \vs2} = (\newstwo, \tenvtwo)\\
% }
% {
%   \annotatestmt{\tenv, \SSeq(\vsone, \vstwo)} = (\SSeq(\newsone, \newstwo), \tenvtwo)
% }
% \]
% The rule uses the annotation function $\annotatestmt{\cdot}$, which
% accepts an environment $\tenv$ and two statements and returns a new statement and a new environment.
% The function returns a new statement in order to implement certain code transformations, such as
% inlining setter functions.
