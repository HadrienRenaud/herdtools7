\chapter{Global Declarations\label{chap:GlobalDeclarations}}
Global declarations are grammatically derived from $\Ndecl$ and represented as ASTs by $\decl$.

There are four kinds of global declarations:
\begin{itemize}
  \item Subprogram declarations, defined in \chapref{SubprogramDeclarations};
  \item Type declarations, defined in \chapref{TypeDeclarations};
  \item Global storage declarations, defined in \chapref{GlobalStorageDeclarations};
  \item Global pragmas.
\end{itemize}

The typing of global declarations is defined in \secref{GlobalDeclarationsTyping}.
As the only kind of global declarations that are associated with semantics are global storage declarations,
their semantics is given in \secref{GlobalStorageDeclarationsSemantics}.

%%%%%%%%%%%%%%%%%%%%%%%%%%%%%%%%%%%%%%%%%%%%%%%%%%%%%%%%%%%%%%%%%%%%%%%%%%%%%%%%
\section{Syntax}
%%%%%%%%%%%%%%%%%%%%%%%%%%%%%%%%%%%%%%%%%%%%%%%%%%%%%%%%%%%%%%%%%%%%%%%%%%%%%%%%
Subprogram declarations:
\begin{flalign*}
\Ndecl  \derivesinline\ & \productionname{funcdecl}{func\_decl}\ \Tfunc \parsesep \Tidentifier \parsesep \Nparamsopt \parsesep \Nfuncargs \parsesep \Nreturntype \parsesep \Nfuncbody &\\
|\ & \productionname{proceduredecl}{procedure\_decl}\ \Tfunc \parsesep \Tidentifier \parsesep \Nparamsopt \parsesep \Nfuncargs \parsesep \Nfuncbody &\\
|\ & \productionname{getter}{getter}\ \Tgetter \parsesep \Tidentifier \parsesep \Nparamsopt \parsesep \Nfuncargs \parsesep \Nreturntype \parsesep \Nfuncbody&\\
|\ & \productionname{setter}{setter}\ \Tsetter \parsesep \Tidentifier \parsesep \Nparamsopt \parsesep \Nfuncargs \parsesep \Teq \parsesep \Ntypedidentifier & \\
    & \wrappedline\ \parsesep \Nfuncbody &\\
\end{flalign*}

Type declarations:
\begin{flalign*}
\Ndecl  \derivesinline\ & \productionname{typedecl}{type\_decl}\ \Ttype \parsesep \Tidentifier \parsesep \Tof \parsesep \Ntydecl \parsesep \Nsubtypeopt \parsesep \Tsemicolon&\\
|\ & \productionname{subtypedecl}{subtype\_decl}\ \Ttype \parsesep \Tidentifier \parsesep \Nsubtype \parsesep \Tsemicolon&\\
\end{flalign*}

Global storage declarations:
\begin{flalign*}
\Ndecl  \derivesinline\ & \productionname{globalstorage}{global\_storage}\ \Nstoragekeyword \parsesep \Nignoredoridentifier \parsesep \option{\Tcolon \parsesep \Nty} \parsesep \Teq \parsesep &\\
    & \wrappedline\ \Nexpr \parsesep \Tsemicolon &\\
|\ & \productionname{globaluninitvar}{global\_uninit\_var}\ \Tvar \parsesep \Nignoredoridentifier \parsesep \Tcolon \parsesep \Nty \parsesep \Tsemicolon&\\
\end{flalign*}

Pragmas:
\begin{flalign*}
\Ndecl  \derivesinline\ & \productionname{globalpragma}{global\_pragma}\ \Tpragma \parsesep \Tidentifier \parsesep \Clist{\Nexpr} \parsesep \Tsemicolon&
\end{flalign*}

%%%%%%%%%%%%%%%%%%%%%%%%%%%%%%%%%%%%%%%%%%%%%%%%%%%%%%%%%%%%%%%%%%%%%%%%%%%%%%%%
\section{Abstract Syntax}
%%%%%%%%%%%%%%%%%%%%%%%%%%%%%%%%%%%%%%%%%%%%%%%%%%%%%%%%%%%%%%%%%%%%%%%%%%%%%%%%
\begin{flalign*}
\decl \derives\ & \DFunc(\func) & \\
  |\ & \DGlobalStorage(\globaldecl) & \\
  |\ & \DTypeDecl(\identifier, \ty, (\identifier, \overtext{\Field^*}{with fields})?) &
\end{flalign*}

\subsubsection{ASTRule.GlobalDecl\label{sec:ASTRule.GlobalDecl}}
\hypertarget{build-decl}{}
The relation
\[
  \builddecl : \overname{\parsenode{\Ndecl}}{\vparsednode} \;\aslrel\; \overname{\decl}{\vastnode}
\]
transforms a parse node $\vparsednode$ into an AST node $\vastnode$.

\hypertarget{build-globalpragma}{}
\begin{mathpar}
\inferrule[global\_pragma]{}
{
  \builddecl(\overname{\Ndecl(\Tpragma, \Tidentifier(\vx), \Clist{\Nexpr}, \Tsemicolon)}{\vparsednode}) \astarrow \overname{\tododefine{pragma\_node}}{\vastnode}
}
\end{mathpar}

%%%%%%%%%%%%%%%%%%%%%%%%%%%%%%%%%%%%%%%%%%%%%%%%%%%%%%%%%%%%%%%%%%%%%%%%%%%%%%%%
\section{Typing\label{sec:GlobalDeclarationsTyping}}
%%%%%%%%%%%%%%%%%%%%%%%%%%%%%%%%%%%%%%%%%%%%%%%%%%%%%%%%%%%%%%%%%%%%%%%%%%%%%%%%
\hypertarget{def-typecheckdecl}{}
The function
\[
  \typecheckdecl(
    \overname{\globalstaticenvs}{\genv} \aslsep
    \overname{\decl}{\vd}
  )
  \aslto (\overname{\decl}{\newd} \times \overname{\globalstaticenvs}{\newgenv})
  \cup \overname{\TTypeError}{\TypeErrorConfig}
\]
annotates a global declaration $\vd$ in the global static environment $\genv$,
yielding an annotated global declaration $\newd$ and modified global static environment $\newgenv$.
\ProseOtherwiseTypeError

\subsubsection{TypingRule.TypecheckFunc \label{sec:TypingRule.TypecheckFunc}}
\subsubsection{Prose}
All of the following apply:
\begin{itemize}
  \item $\vd$ is a subprogram AST node with a subprogram definition $\vf$, that is, $\DFunc(\vf)$;
  \item annotating and declaring the subprogram for $\vf$ in $\genv$ as per \secref{TypingRule.AnnotateAndDeclareFunc}
        yields the new environment $\newtenv$ and a subprogram definition $\vfone$\ProseOrTypeError;
  \item annotating the subprogram definition $\vfone$ in $\tenv$ as per \nameref{sec:TypingRule.Subprogram} yields
        the annotated subprogram definition $\vftwo$\ProseOrTypeError;
  \item define $\newd$ as the subprogram AST node with $\vftwo$, that is, $\DFunc(\vftwo)$;
  \item define $\newgenv$ as the global component of $\newtenv$.
\end{itemize}\CodeSubsection{\TypecheckFuncBegin}{\TypecheckFuncEnd}{../Typing.ml}

\subsubsection{Formally}
\begin{mathpar}
\inferrule{
  \vf \eqname \{\text{body}: \SBASL, \ldots\}\\
  \annotateanddeclarefunc(\genv, \vf) \typearrow (\newtenv, \vfone) \OrTypeError\\\\
  \annotatesubprogram{\newtenv, \vfone} \typearrow \vftwo \OrTypeError
}{
  \typecheckdecl(\genv, \overname{\DFunc(\vf)}{\vd})
  \typearrow (\overname{\DFunc(\vftwo)}{\newd}, \overname{G^\newtenv}{\newgenv})
}
\end{mathpar}

\subsubsection{TypingRule.Subprogram \label{sec:TypingRule.Subprogram}}
\hypertarget{def-annotatesubprogram}{}
The function
\[
  \annotatesubprogram{\overname{\staticenvs}{\tenv} \aslsep \overname{\func}{\vf}} \aslto \overname{\func}{\vfp}
  \cup \overname{\TTypeError}{\TypeErrorConfig}
\]
annotates a subprogram $\vf$ in an environment $\tenv$, resulting in an annotated subprogram $\vfp$.
\ProseOtherwiseTypeError

Note that the return type in $\vf$ has already been annotated by $\annotatefuncsig$.

\subsection{Prose}
One of the following applies:
\begin{itemize}
  \item All of the following apply (\textsc{procedure}):
  \begin{itemize}
    \item $\vf$ is a $\func$ AST node subprogram body $\body$;
    \item annotating the block $\body$ in $\tenv$ as per \secref{TypingRule.Block} yields $\newbody$\ProseOrTypeError;
    \item $\vf.\funcreturntype$ is $\None$;
    \item $\vfp$ is $\vf$ with the subprogram body substituted with $\newbody$.
  \end{itemize}

  \item All of the following apply (\textsc{function}):
  \begin{itemize}
    \item $\vf$ is a $\func$ AST node subprogram body $\body$;
    \item annotating the block $\body$ in $\tenv$ as per \secref{TypingRule.Block} yields $\newbody$\ProseOrTypeError;
    \item applying $\checkstmtreturnsorthrows$ to $\newbody$ in $\tenv$ yields $\True$\ProseOrTypeError;
    \item $\vfp$ is $\vf$ with the subprogram body substituted with $\newbody$.
  \end{itemize}
\end{itemize}

\subsection{Formally}
\begin{mathpar}
\inferrule[procedure]{
  \annotateblock{\tenv, \body} \typearrow \newbody \OrTypeError\\\\
  \vf.\funcreturntype = \None\\
  \vfp \eqdef \substrecordfield(\vf, \funcbody, \SBASL(\newbody))
}{
  \annotatesubprogram{\tenv, \vf} \typearrow \vfp
}
\end{mathpar}

\begin{mathpar}
\inferrule[function]{
  \annotateblock{\tenv, \body} \typearrow \newbody \OrTypeError\\\\
  \vf.\funcreturntype \neq \None\\
  \checkstmtreturnsorthrows(\tenv, \newbody) \typearrow \True \OrTypeError\\\\
  \vfp \eqdef \substrecordfield(\vf, \funcbody, \SBASL(\newbody))
}{
  \annotatesubprogram{\tenv, \vf} \typearrow \vfp
}
\end{mathpar}

\lrmcomment{
This is related to \identi{GHGK}, \identr{HWTV}, \identr{SCHV}, \identr{VDPC},
\identr{TJKQ}, \identi{LFJZ}, \identi{BZVB}, \identi{RQQB}.
}
\CodeSubsection{\SubprogramBegin}{\SubprogramEnd}{../Typing.ml}

\subsubsection{TypingRule.CheckStmtReturnsOrThrows\label{sec:TypingRule.CheckStmtReturnsOrThrows}}
\hypertarget{def-checkstmtreturnsorthrows}{}
The helper function
\[
  \checkstmtreturnsorthrows(\overname{\staticenvs}{\tenv} \aslsep \overname{\stmt}{\vs})
  \typearrow \{\True\} \cup \overname{\TTypeError}{\TypeErrorConfig}
\]
checks whether all control-flow paths defined by the statement $\vs$ in $\tenv$ terminate by either
a statement returning a value, a \texttt{throw} statement, or the \texttt{Unreachable()} statement.

\subsubsection{Prose}
All of the following apply:
\begin{itemize}
  \item applying $\controlflowfromstmt$ to $\vs$ in $\tenv$ yields a \controlflowsymbolterm\ $\vctrlflow$;
  \item checking that $\vctrlflow$ is different from $\maynotinterrupt$ yields $\True$\ProseTerminateAs{\NonReturningFunction};
  \item the result is $\True$.
\end{itemize}

\subsubsection{Formally}
\begin{mathpar}
\inferrule{
  \controlflowfromstmt(\tenv, \vs) \typearrow \vctrlflow\\
  \checktrans{\vctrlflow \neq \maynotinterrupt}{\NonReturningFunction} \typearrow \True \OrTypeError
}{
  \checkstmtreturnsorthrows(\tenv, \vs) \typearrow \True
}
\end{mathpar}

\subsubsection{TypingRule.ControlFlowFromStmt\label{sec:TypingRule.ControlFlowFromStmt}}
\hypertarget{def-controlflowsymbolterm}{}
We define \controlflowsymbolterm\ as follows:
\[
  \controlflowstate \eqdef \{\assertednotinterrupt, \interrupt, \maynotinterrupt\}
\]

\hypertarget{def-controlflowfromstmt}{}
The helper function
\[
  \controlflowfromstmt(\overname{\staticenvs}{\tenv} \aslsep \overname{\stmt}{\vs})
  \typearrow \overname{\controlflowstate}{\vctrlflow}
\]
statically analyzes the statement $\vs$ in the context of the static environment $\tenv$
and determines the \controlflowsymbolterm\ $\vctrlflow$ to be one of the following:
\begin{description}
  \item[$\assertednotinterrupt$] evaluating $\vs$ in any environment will evaluate \texttt{Unreachable()};
  \item[$\interrupt$] evaluating $\vs$ in any environment will end by either evaluating a return statement with an expression,
      or evaluating a \texttt{throw} statement;
  \item[$\maynotinterrupt$] evaluating $\vs$ in any environment may not end by evaluating either of the above mentioned statements.
\end{description}

\subsubsection{Prose}
One of the following applies:
\begin{itemize}
  \item All of the following apply (\textsc{falls\_through}):
  \begin{itemize}
    \item the AST label of $\vs$ is $\SPass$, $\SDecl$, $\SAssign$, $\SAssert$, $\SCall$, or $\SPrint$;
    \item $\vctrlflow$ is $\maynotinterrupt$;
  \end{itemize}

  \item All of the following apply (\textsc{unreachable}):
  \begin{itemize}
    \item $\vs$ is $\SUnreachable$;
    \item $\vctrlflow$ is $\assertednotinterrupt$;
  \end{itemize}

  \item All of the following apply (\textsc{return\_throw}):
  \begin{itemize}
    \item the AST label of $\vs$ is either $\SReturn$ or $\SThrow$;
    \item $\vctrlflow$ is $\interrupt$;
  \end{itemize}

  \item All of the following apply (\textsc{s\_seq}):
  \begin{itemize}
    \item $\vs$ is the sequence statement for $\vsone$ and $\vstwo$;
    \item applying $\controlflowfromstmt$ to $\vsone$ in $\tenv$ yields $\vctrlflowone$;
    \item applying $\controlflowfromstmt$ to $\vstwo$ in $\tenv$ yields $\vctrlflowtwo$;
    \item applying $\controlflowseq$ to $\vctrlflowone$ and $\vctrlflowtwo$ yields $\vctrlflow$.
  \end{itemize}

  \item All of the following apply (\textsc{s\_cond}):
  \begin{itemize}
    \item $\vs$ is the condition statement for the expression $\ve$ and statements $\vsone$ and $\vstwo$;
    \item applying $\controlflowfromstmt$ to $\vsone$ in $\tenv$ yields $\vctrlflowone$;
    \item applying $\controlflowfromstmt$ to $\vstwo$ in $\tenv$ yields $\vctrlflowtwo$;
    \item applying $\controlflowguard$ to $\ve$, $\vctrlflowone$ and $\vctrlflowtwo$ in $\tenv$ yields $\vctrlflow$.
  \end{itemize}

  \item All of the following apply (\textsc{s\_while}):
  \begin{itemize}
    \item $\vs$ is the \texttt{while} statement for the expression $\ve$ and statement $\vbody$;
    \item applying $\controlflowfromstmt$ to $\vbody$ in $\tenv$ yields $\vctrlflowone$;
    \item applying $\controlflowguard$ to $\ve$, $\vctrlflowone$ and $\maynotinterrupt$ in $\tenv$ yields $\vctrlflow$.
  \end{itemize}

  \item All of the following apply (\textsc{s\_repeat}):
  \begin{itemize}
    \item $\vs$ is the \texttt{repeat} statement for the statement $\vbody$;
    \item applying $\controlflowfromstmt$ to $\vbody$ in $\tenv$ yields $\vctrlflow$.
  \end{itemize}

  \item All of the following apply (\textsc{s\_try}):
  \begin{itemize}
    \item $\vs$ is the \texttt{try} statement for the statement $\vbody$;
    \item applying $\controlflowfromstmt$ to $\vbody$ in $\tenv$ yields $\vctrlflow$.
  \end{itemize}

  \item All of the following apply (\textsc{s\_for}):
  \begin{itemize}
    \item $\vs$ is the \texttt{for} statement for the direction $\dir$, start expression $\vstarte$,
          end $\vende$ and body statement $\vbody$;
    \item define the expression $\vcond$ as follows:
    \begin{itemize}
      \item the binary expression testing whether $\vstarte$ is less then or equal to $\vende$, if $\dir$ is $\UP$;
      \item the binary expression testing whether $\vstarte$ is greater or equal to $\vende$, if $\dir$ is $\DOWN$;
    \end{itemize}
    \item applying $\controlflowfromstmt$ to $\vbody$ in $\tenv$ yields $\vctrlflowone$;
    \item applying $\controlflowguard$ to $\vcond$, $\vctrlflowone$, and $\maynotinterrupt$ in $\tenv$ yields $\vctrlflow$.
  \end{itemize}
\end{itemize}

\subsubsection{Formally}
\begin{mathpar}
\inferrule[falls\_through]{
  \astlabel(\vs) \in \{\SPass, \SDecl, \SAssign, \SAssert, \SCall, \SPrint\}
}{
  \controlflowfromstmt(\tenv, \vs) \typearrow \overname{\maynotinterrupt}{\vctrlflow}
}
\end{mathpar}

\begin{mathpar}
\inferrule[unreachable]{}{
  \controlflowfromstmt(\tenv, \overname{\SUnreachable}{\vs}) \typearrow \overname{\assertednotinterrupt}{\vctrlflow}
}
\end{mathpar}

\begin{mathpar}
\inferrule[return\_throw]{
  \astlabel(\vs) \in \{\SReturn, \SThrow\}
}{
  \controlflowfromstmt(\tenv, \overname{\SUnreachable}{\vs}) \typearrow \overname{\interrupt}{\vctrlflow}
}
\end{mathpar}

\begin{mathpar}
\inferrule[s\_seq]{
  \controlflowfromstmt(\tenv, \vsone) \typearrow \vctrlflowone\\
  \controlflowfromstmt(\tenv, \vstwo) \typearrow \vctrlflowtwo\\
  \controlflowseq(\vctrlflowone, \vctrlflowtwo) \typearrow \vctrlflow
}{
  \controlflowfromstmt(\tenv, \overname{\SSeq(\vsone, \vstwo)}{\vs}) \typearrow \vctrlflow
}
\end{mathpar}

\begin{mathpar}
\inferrule[s\_cond]{
  \controlflowfromstmt(\tenv, \vsone) \typearrow \vctrlflowone\\
  \controlflowfromstmt(\tenv, \vstwo) \typearrow \vctrlflowtwo\\
  \controlflowguard(\tenv, \ve, \vctrlflowone, \vctrlflowtwo) \typearrow \vctrlflow
}{
  \controlflowfromstmt(\tenv, \overname{\SCond(\ve, \vsone, \vstwo)}{\vs}) \typearrow \vctrlflow
}
\end{mathpar}

\begin{mathpar}
\inferrule[s\_while]{
  \controlflowfromstmt(\tenv, \vbody) \typearrow \vctrlflowone\\
  \controlflowguard(\tenv, \ve, \vctrlflowone, \maynotinterrupt) \typearrow \vctrlflow
}{
  \controlflowfromstmt(\tenv, \overname{\SWhile(\ve, \Ignore, \vbody)}{\vs}) \typearrow \vctrlflow
}
\end{mathpar}

\begin{mathpar}
\inferrule[s\_repeat]{
  \controlflowfromstmt(\tenv, \vbody) \typearrow \vctrlflow
}{
  \controlflowfromstmt(\tenv, \overname{\SRepeat(\vbody, \Ignore, \Ignore)}{\vs}) \typearrow \vctrlflow
}
\end{mathpar}

\begin{mathpar}
\inferrule[s\_try]{
  \controlflowfromstmt(\tenv, \vbody) \typearrow \vctrlflow
}{
  \controlflowfromstmt(\tenv, \overname{\STry(\vbody, \Ignore, \Ignore)}{\vs}) \typearrow \vctrlflow
}
\end{mathpar}

\begin{mathpar}
\inferrule[s\_for]{
  \vcond \eqdef \choice{\dir = \UP}{\EBinop(\LEQ, \vstarte, \vende)}{\EBinop(\GEQ, \vstarte, \vende)}\\
  \controlflowfromstmt(\tenv, \vbody) \typearrow \vctrlflowone\\
  \controlflowguard(\tenv, \vcond, \vctrlflowone, \maynotinterrupt) \typearrow \vctrlflow
}{
  {
  \controlflowfromstmt\left(\tenv, \overname{
    \SFor\left\{
    \begin{array}{rcl}
    \Forindexname  &:& \Ignore,\\
    \Forstarte     &:& \vstarte,\\
    \Fordir        &:& \dir,\\
    \Forende       &:& \vende,\\
    \Forbody       &:& \vbody,\\
    \Forlimit      &:& \Ignore
    \end{array}\right\}
  }{\vs}\right) \typearrow \vctrlflow
  }
}
\end{mathpar}

\subsubsection{TypingRule.ControlFlowSeq\label{sec:TypingRule.ControlFlowSeq}}
\hypertarget{def-controlflowseq}{}
The helper function
\[
\controlflowseq(\overname{\controlflowstate}{\vtone} \aslsep \overname{\controlflowstate}{\vttwo})
\aslto \overname{\controlflowstate}{\vctrlflow}
\]
combines two \controlflowsymbolterm{s} considering them as part of a control flow path where the analysis of the
path prefix yields $\vtone$ and the analysis of the path suffix yields $\vttwo$,
into a single \controlflowsymbolterm\ $\vctrlflow$.

\subsubsection{Prose}
Define $\vctrlflow$ as $\vttwo$ if $\vtone$ is $\maynotinterrupt$ and $\vttwo$, otherwise.

\subsubsection{Formally}
\begin{mathpar}
\inferrule{
  \vctrlflow \eqdef \choice{\vtone = \maynotinterrupt}{\vttwo}{\vtone}
}{
  \controlflowseq(\vtone, \vttwo) \typearrow \vctrlflow
}
\end{mathpar}

\subsubsection{TypingRule.ControlFlowGuard\label{sec:TypingRule.ControlFlowGuard}}
\hypertarget{def-controlflowguard}{}
The helper function
\[
\controlflowguard(\overname{\staticenvs}{\tenv} \aslsep
                \overname{\expr}{\ve} \aslsep
                \overname{\controlflowstate}{\vtone} \aslsep
                \overname{\controlflowstate}{\vttwo})
\aslto \overname{\controlflowstate}{\vctrlflow}
\]

\subsubsection{Prose}
One of the following applies:
\begin{itemize}
  \item All of the following apply (\textsc{true}):
  \begin{itemize}
    \item applying $\staticeval$ to $\ve$ in $\tenv$ yields the literal $\True$\ProseOrTypeError;
    \item define $\vctrlflow$ as $\vtone$;
  \end{itemize}

  \item All of the following apply (\textsc{false}):
  \begin{itemize}
    \item applying $\staticeval$ to $\ve$ in $\tenv$ yields the literal $\False$\ProseOrTypeError;
    \item define $\vctrlflow$ as $\vttwo$;
  \end{itemize}

  \item All of the following apply (\textsc{non\_literal}):
  \begin{itemize}
    \item applying $\staticeval$ to $\ve$ in $\tenv$ yields $\CannotBeTransformed$\ProseOrTypeError;
    \item define $\vctrlflow$ according to the following cases:
    \begin{itemize}
      \item $\maynotinterrupt$ if either $\vtone$ or $\vttwo$ are $\maynotinterrupt$;
      \item $\vttwo$ if $\vtone$ is $\assertednotinterrupt$ and $\vttwo$ is not $\maynotinterrupt$;
      \item $\vtone$ if $\vtone$ is not $\maynotinterrupt$ and $\vttwo$ is $\assertednotinterrupt$;
      \item $\interrupt$ if both $\vtone$ and $\vttwo$ are $\interrupt$.
    \end{itemize}
  \end{itemize}
\end{itemize}

\subsubsection{Formally}
\begin{mathpar}
\inferrule[true]{
  \staticeval(\tenv, \ve) \typearrow \lbool(\True) \OrTypeError
}{
  \controlflowguard(\tenv, \ve, \vtone, \vttwo) \typearrow \overname{\vtone}{\vctrlflow}
}
\end{mathpar}

\begin{mathpar}
\inferrule[false]{
  \staticeval(\tenv, \ve) \typearrow \lbool(\False) \OrTypeError
}{
  \controlflowguard(\tenv, \ve, \vtone, \vttwo) \typearrow \overname{\vttwo}{\vctrlflow}
}
\end{mathpar}

\begin{mathpar}
\inferrule[non\_literal]{
  \staticeval(\tenv, \ve) \typearrow \CannotBeTransformed \OrTypeError\\\\
  {
    \vctrlflow \eqdef
    \begin{cases}
      \maynotinterrupt & \vtone = \maynotinterrupt \lor \vttwo = \maynotinterrupt\\
      \vttwo & \vtone = \assertednotinterrupt \land \vttwo \neq \maynotinterrupt\\
      \vtone & \vtone \neq \maynotinterrupt \land \vttwo = \assertednotinterrupt\\
      \interrupt & \vtone = \vttwo = \interrupt
    \end{cases}
  }
}{
  \controlflowguard(\tenv, \ve, \vtone, \vttwo) \typearrow \vctrlflow
}
\end{mathpar}

\subsubsection{TypingRule.TypecheckGlobalStorage\label{sec:TypingRule.TypecheckGlobalStorage}}
\subsubsection{Prose}
All of the following apply:
\begin{itemize}
  \item $\vd$ is a global storage declaration with description $\gsd$, that is, \\ $\DGlobalStorage(\gsd)$;
  \item declaring the global storage with description $\gsd$ in $\genv$ yields the new environment
        $\newgenv$ and new global storage description $\gsdp$\ProseOrTypeError;
  \item define $\newd$ as the global storage declaration with description $\gsdp$, that is, \\ $\DGlobalStorage(\gsdp)$.
\end{itemize}\CodeSubsection{\TypecheckGlobalStorageBegin}{\TypecheckGlobalStorageEnd}{../Typing.ml}

\subsubsection{Formally}
\begin{mathpar}
\inferrule{
  \declareglobalstorage(\genv, \gsd) \typearrow (\newgenv, \gsdp) \OrTypeError
}{
  {
    \begin{array}{r}
  \typecheckdecl(\genv, \overname{\DGlobalStorage(\gsd)}{\vd})
  \typearrow \\ (\overname{\DGlobalStorage(\gsdp)}{\newd}, \newgenv)
    \end{array}
  }
}
\end{mathpar}

\subsubsection{TypingRule.TypecheckTypeDecl\label{sec:TypingRule.TypecheckTypeDecl}}
\subsubsection{Prose}
All of the following apply:
\begin{itemize}
  \item $\vd$ is a type declaration with identifier $\vx$, type $\tty$,
        and \optional\ field initializers $\vs$, that is, $\DTypeDecl(\vx, \tty, \vs)$;
  \item declaring the type described by $(\vx, \tty, \vs)$ in $\genv$
        as per \secref{TypingRule.DeclaredType} yields the modified global static environment $\newgenv$\ProseOrTypeError;
  \item define $\newd$ as $\vd$.
  \end{itemize}
\CodeSubsection{\TypecheckTypeDeclBegin}{\TypecheckTypeDeclEnd}{../Typing.ml}

\subsubsection{Formally}
\begin{mathpar}
\inferrule{
  \declaretype(\genv, \vx, \tty, \vs) \typearrow \newgenv \OrTypeError
}{
  \typecheckdecl(\genv, \overname{\DTypeDecl(\vx, \tty, \vs)}{\vd}) \typearrow (\overname{\vd}{\newd}, \newgenv)
}
\end{mathpar}
