\chapter{Statements\label{chap:Statements}}
Statements update storage elements and determine the flow of control of a subprogram.

Statements are grammatically derived from $\Nstmt$ and represented as ASTs by $\stmt$.

\hypertarget{build-stmt}{}
The function
\[
\buildstmt(\overname{\parsenode{\Nstmt}}{\vparsednode}) \;\aslto\; \overname{\stmt}{\vastnode}
\]
transforms a statement parse node $\vparsednode$ into a statement AST node $\vastnode$.

\hypertarget{def-annotatestmt}{}
The function
\[
  \annotatestmt(\overname{\staticenvs}{\tenv} \aslsep \overname{\stmt}{\vs}) \aslto
  (\overname{\stmt}{\news}\aslsep \overname{\staticenvs}{\newtenv})
  \cup \overname{\TTypeError}{\TypeErrorConfig}
\]
annotates a statement $\vs$ in an environment $\tenv$, resulting in $\news$ ---
the \typedast\ for $\vs$, which is also known as the \emph{annotated statement} ---
and a modified environment $\newtenv$. \ProseOtherwiseTypeError

The relation
\hypertarget{def-evalstmt}{}
\[
  \evalstmt{\overname{\envs}{\env} \aslsep \overname{\stmt}{\vs}} \;\aslrel\;
  \left(
  \begin{array}{cl}
  \overname{\TReturning}{\Returning((\vvs,\newg), \newenv)} & \cup\\
  \overname{\TContinuing}{\Continuing(\newg,\newenv)} & \cup\\
  \overname{\TThrowing}{\ThrowingConfig} & \cup \\
  \overname{\TDynError}{\DynErrorConfig} &
  \end{array}
  \right)
\]
evaluates a statement $\vs$ in an environment $\env$, resulting in one of four types of configurations
(see more details in \secref{KindsOfSemanticConfigurations}):
\begin{itemize}
  \item returning configurations with values $\vvs$, execution graph $\newg$, and a modified environment $\newenv$;
  \item continuing configurations with an execution graph $\newg$ and modified environment $\newenv$;
  \item throwing configurations;
  \item error configurations.
\end{itemize}

We now define the syntax, abstract syntax, typing, and semantics for the following kinds of statements:
\begin{itemize}
\item Pass statements (see \secref{PassStatements})
\item Assignment statements (see \secref{AssignmentStatements})
\item Declaration statements (see \secref{DeclarationStatements})
\item Sequencing statements (see \secref{SeuqncingStatements})
\item Call statements (see \secref{CallStatements})
\item Conditional statements (see \secref{ConditionalStatements})
\item Case statements (see \secref{CaseStatements})
\item Assertion statements (see \secref{AssertionStatements})
\item While statements (see \secref{WhileStatements})
\item Repeat statements (see \secref{RepeatStatements})
\item For statements (see \secref{ForStatements})
\item Throw statements (see \secref{ThrowStatements})
\item Try statements (see \secref{TryStatements})
\item Return statements (see \secref{ReturnStatements})
\item Print statements (see \secref{PrintStatements})
\item Unreachable statements (see \secref{UnreachableStatement})
\item Pragma statements (see \secref{PragmaStatements})
\end{itemize}

\section{Pass Statements\label{sec:PassStatements}}
\subsection{Syntax}
\begin{flalign*}
\Nstmt \derivesinline\ & \Tpass \parsesep \Tsemicolon &
\end{flalign*}

\subsection{Abstract Syntax}
\begin{flalign*}
\stmt \derives\ & \SPass &
\end{flalign*}

\subsubsection{ASTRule.SPass}
\begin{mathpar}
\inferrule{}{
  \buildstmt(\overname{\Nstmt(\Tpass, \Tsemicolon)}{\vparsednode})
  \astarrow
  \overname{\SPass}{\vastnode}
}
\end{mathpar}

\subsection{Typing}
\subsubsection{TypingRule.SPass \label{sec:TypingRule.SPass}}
\subsubsection{Prose}
All of the following apply:
\begin{itemize}
  \item $\vs$ is a pass statement, that is, $\SPass$;
  \item $\news$ is $\vs$;
  \item $\newtenv$ is $\tenv$.
\end{itemize}
\subsubsection{Formally}
\begin{mathpar}
\inferrule{}{\annotatestmt(\tenv, \SPass) \typearrow (\SPass,\tenv)}
\end{mathpar}
\CodeSubsection{\SPassBegin}{\SPassEnd}{../Typing.ml}

\subsection{Semantics}
\subsubsection{SemanticsRule.SPass \label{sec:SemanticsRule.SPass}}
\subsubsection{Example}
In the specification:
\VerbatimInput{../tests/ASLSemanticsReference.t/SemanticsRule.SPass.asl}
\texttt{pass;} does nothing.

\subsubsection{Prose}
All of the following apply:
\begin{itemize}
\item $\vs$ is a \texttt{pass} statement, $\SPass$;
\item $\newg$ is the empty graph;
\item $\newenv$ is $\env$.
\end{itemize}

\subsubsection{Formally}
\begin{mathpar}
\inferrule{}{
  \evalstmt{\env, \SPass} \evalarrow \Continuing(\overname{\emptygraph}{\newg}, \overname{\env}{\newenv})
}
\end{mathpar}
\CodeSubsection{\EvalSPassBegin}{\EvalSPassEnd}{../Interpreter.ml}

\section{Assignment Statements\label{sec:AssignmentStatements}}
\subsection{Syntax}
\begin{flalign*}
\Nstmt \derivesinline\ & \Nlexpr \parsesep \Teq \parsesep \Nexpr \parsesep \Tsemicolon &
\end{flalign*}

\subsection{Abstract Syntax}
\begin{flalign*}
\stmt \derives\ & \SAssign(\lexpr, \expr) &
\end{flalign*}

\subsubsection{ASTRule.SAssign}
\begin{mathpar}
\inferrule{}{
  \buildstmt(\overname{\Nstmt(\punnode{\Nlexpr}, \Teq, \punnode{\Nexpr}, \Tsemicolon)}{\vparsednode})
  \astarrow
  \overname{\SAssign(\astof{\vlexpr}, \astof{\vexpr})}{\vastnode}
}
\end{mathpar}

\subsection{Typing}
\subsubsection{TypingRule.SAssign \label{sec:TypingRule.SAssign}}
\subsubsection{Prose}
All of the following apply:
\begin{itemize}
  \item $\vs$ is an assignment \texttt{le = re}, that is, $\SAssign(\vle, \vre)$;
  \item annotating the right-hand-side expression $\vre$ in $\tenv$ yields $(\vtre, \vreone)$\ProseOrTypeError;
  \item annotating the \assignableexpression\ $\vle$ with the type $\vtre$ in $\tenv$ yields $\vleone$\ProseOrTypeError;
  \item $\news$ is the assignment \texttt{le1 = re1}, that is, $\SAssign(\vleone, \vreone)$;
  \item $\newtenv$ is $\tenv$.
\end{itemize}
\subsubsection{Formally}
\begin{mathpar}
\inferrule{
  \annotateexpr{\tenv, \vre} \typearrow (\vtre, \vreone) \OrTypeError\\\\
  \annotatelexpr{\tenv, \vle, \vtre} \typearrow \vleone \OrTypeError
}{
  \annotatestmt(\tenv, \overname{\SAssign(\vle, \vre)}{\vs}) \typearrow (\overname{\SAssign(\vleone, \vreone)}{\news},\overname{\tenv}{\newtenv})
}
\end{mathpar}
\CodeSubsection{\SAssignBegin}{\SAssignEnd}{../Typing.ml}

\subsection{Semantics}
\subsubsection{SemanticsRule.SAssign \label{sec:SemanticsRule.SAssign}}
\subsubsection{Example}
In the specification:
\VerbatimInput{../tests/ASLSemanticsReference.t/SemanticsRule.SAssign.asl}
\texttt{x = 3;} binds \texttt{x} to $\nvint(3)$ in the environment where \texttt{x} is bound to
$\nvint(42)$, and $\newenv$ is such that \texttt{x} is bound to $\nvint(3)$.

\subsubsection{Prose}
All of the following apply:
\begin{itemize}
  \item $\vs$ is an assignment statement, $\SAssign(\vle, \vre)$;
  \item $\vre$ is not a call expression;
  \item evaluating the expression $\vre$ in $\env$ yields
        $\Normal(\vm, \envone)$ (here, $\vm$ is a pair consisting of a value and an execution graph)\ProseOrAbnormal;
  \item evaluating the \assignableexpression\ $\vle$ with $\vm$ in $\envone$,
        as per \chapref{AssignableExpressions}, yields $\Normal(\newg, \newenv)$\ProseOrAbnormal.
\end{itemize}

\subsubsection{Formally}
\begin{mathpar}
\inferrule{
  \astlabel(\vre) \neq \ECall\\
  \evalexpr{\env, \vre} \evalarrow \Normal(\vm, \envone) \OrAbnormal\\
  \evallexpr{\envone, \vle, \vm} \evalarrow \Normal(\newg, \newenv) \OrAbnormal
}{
  \evalstmt{\env, \SAssign(\vle, \vre)} \evalarrow \Continuing(\newg, \newenv)
}
\end{mathpar}
\CodeSubsection{\EvalSAssignBegin}{\EvalSAssignEnd}{../Interpreter.ml}

\subsubsection{Comments}
This rule covers all assignment statements, except the ones where the
right-hand side expression is a function call, which is covered by
\nameref{sec:SemanticsRule.SAssignCall}.  Although
the sequential semantics of both statements is the same,
SemanticsRule.SAssignCall generates a different execution graph.

Notice that this rule first produces a value for the right-hand side expression
and then completes the update via an appropriate rule for evaluating the
\assignableexpression, which in turn handles variables, tuples, bitvectors,
etc.

\subsubsection{SemanticsRule.SAssignCall \label{sec:SemanticsRule.SAssignCall}}
\subsubsection{Example}
\VerbatimInput{\semanticstests/SemanticsRule.SAssignCall.asl}
given that the function call \texttt{f(1)} returns a pair of values --- $\nvint(1)$ and $\nvint(2)$
(each with its own associated execution graph),
the statement \texttt{(a,b) = f(1)} assigns the value $\nvint(1)$ to the mutable variable \texttt{a}
and the value $\nvint(2)$ to the mutable variable~\texttt{b}.

\subsubsection{Prose}
All of the following apply:
\begin{itemize}
  \item $\vs$ assigns a \assignableexpression\ list from a subprogram call, \\
        $\SAssign(\LEDestructuring(\les),\ECall(\name, \args, \namedargs))$;
  \item $\les$ is a list of \assignableexpressions, each of which is either \\ a variable ($\LEVar(\Ignore)$)
        or a discarded variable (\LEDiscard);
  \item evaluating the subprogram call as per \chapref{SubprogramCalls} is
        $\Normal(\vms, \envone)$\ProseOrAbnormal;
  \item assigning each value in $\vms$ to the respective element of the tuple $\les$ is \\
        $\Normal(\vgtwo, \newg)$\ProseOrAbnormal.
\end{itemize}

\subsubsection{Formally}
\hypertarget{def-lexprisvar}{}
We first define the syntactic predicate
\[
  \lexprisvar(\lexpr) \aslto \True
\]
which holds when a left-hand side expression
represents a variable:
\begin{mathpar}
  \inferrule{}{ \lexprisvar(\LEVar(\Ignore)) \evalarrow \True}
  \and
  \inferrule{}{ \lexprisvar(\LEDiscard) \evalarrow \False}
\end{mathpar}

We now define the evaluation of assigning from a subprogram call:
\begin{mathpar}
\inferrule{
  \vles \eqdef \vle_{1..k}\\
  i=1..k: \lexprisvar(\vle_i) \evalarrow \True\\
  \evalcall{\env, \name, \args, \namedargs} \evalarrow \Normal(\vms, \envone) \OrAbnormal\\\\
  \evalmultiassignment(\envone, \vles, \vms) \evalarrow \Normal(\newg, \newenv) \OrAbnormal
}{
  \evalstmt{\env, \SAssign(\LEDestructuring(\les),\ECall(\name, \args, \namedargs))} \\
  \evalarrow \Continuing(\newg, \newenv)
}
\end{mathpar}
\CodeSubsection{\EvalSAssignCallBegin}{\EvalSAssignCallEnd}{../Interpreter.ml}

\section{Declaration Statements\label{sec:DeclarationStatements}}
\subsection{Syntax}
\begin{flalign*}
\Nstmt \derivesinline\ & \Nlocaldeclkeyword \parsesep \Ndeclitem \parsesep \Teq \parsesep \Nexpr \parsesep \Tsemicolon &\\
|\ & \Tvar \parsesep \Ndeclitem \parsesep \option{\Teq \parsesep \Nexpr} \parsesep \Tsemicolon &\\
|\ & \Tvar \parsesep \Clisttwo{\Tidentifier} \parsesep \Tcolon \parsesep \Nty \parsesep \Tsemicolon &\\
\end{flalign*}

\subsection{Abstract Syntax}
\begin{flalign*}
\stmt \derives\ & \SDecl(\localdeclkeyword, \localdeclitem, \expr?) &
\end{flalign*}

\subsubsection{ASTRule.SDecl}
\begin{mathpar}
\inferrule[let\_constant]{}{
  {
  \begin{array}{r}
  \buildstmt(\overname{\Nstmt(\Nlocaldeclkeyword, \Ndeclitem, \Teq, \punnode{\Nexpr}, \Tsemicolon)}{\vparsednode})
  \astarrow\\
  \overname{\SDecl(\astof{\vlocaldeclkeyword}, \astof{\vdeclitem}, \langle\astof{\vexpr}\rangle)}{\vastnode}
  \end{array}
  }
}
\end{mathpar}

\begin{mathpar}
\inferrule[var]{
  \buildoption[\buildexpr](\ve) \astarrow \astversion{\ve}
}{
  {
    \begin{array}{r}
  \buildstmt(\overname{\Nstmt(\Tvar, \Ndeclitem, \namednode{\ve}{\option{\Teq, \Nexpr}}, \Tsemicolon)}{\vparsednode})
  \astarrow \\
  \overname{\SDecl(\LDKVar, \astof{\vdeclitem}, \astversion{\ve})}{\vastnode}
\end{array}
}
}
\end{mathpar}

\begin{mathpar}
\inferrule[multi\_var]{
  \buildclist[\buildidentity](\vids) \astarrow \astversion{\vids}\\
  \vstmts \eqdef [\vx\in\astversion{\vids}: \SDecl(\LDKVar, \vx, \astof{\tty})]\\
  \stmtfromlist(\vstmts) \astarrow \vastnode
}{
  \buildstmt(\overname{\Nstmt(\Tvar, \namednode{\vids}{\Clisttwo{\Tidentifier}}, \Tcolon, \punnode{\Nty}, \Tsemicolon)}{\vparsednode})
  \astarrow
  \vastnode
}
\end{mathpar}

\subsection{Typing}
\subsubsection{TypingRule.SDecl\label{sec:TypingRule.SDecl}}
\subsubsection{Prose}
One of the following applies:
\begin{itemize}
  \item All of the following apply (\textsc{some}):
  \begin{itemize}
    \item $\vs$ is a declaration with local declaration keyword $\ldk$, local identifiers $\ldi$, and an expression $\ve$,
          that is, $\SDecl(\ldk, \ldi, \langle\ve\rangle)$;
    \item annotating the right-hand-side expression $\ve$ in $\tenv$ yields $(\vte,\vep)$\ProseOrTypeError;
    \item annotating the local declaration item $\ldi$ with local declaration keyword $\ldk$, the type $\vte$,
          and the initializing expression $\vep$, in $\tenv$
          yields $(\tenvone, \ldione)$;
    \item One of the following applies:
    \begin{itemize}
      \item All of the following apply (\textsc{constant}):
      \begin{itemize}
        \item $\ldk$ indicates a local constant declaration, that is, $\LDKConstant$;
        \item symbolically simplifying $\ve$ in $\tenvone$ yields the literal $\vv$\ProseOrTypeError;
        \item declaring a local constant of type $\vte$, literal $\vv$ and local declaration item $\ldione$ in $\tenvone$ yields $\newtenv$;
        \item $\news$ is a declaration with $\ldk$, $\ldione$ and an expression $\vep$.
      \end{itemize}

      \item All of the following apply (\textsc{non\_constant}):
      \begin{itemize}
        \item $\ldk$ indicates that this is not a local constant declaration, that is, $\ldk\neq\LDKConstant$;
        \item $\news$ is a declaration with $\ldk$, $\ldione$ and an expression $\vep$;
        \item $\newtenv$ is $\tenvone$.
      \end{itemize}
    \end{itemize}
  \end{itemize}

  \item All of the following apply (\textsc{none}):
  \begin{itemize}
  \item $\vs$ is a local declaration statement with a variable keyword and local identifiers $\ldi$, and no initial expression,
        that is, $\SDecl(\LDKVar, \ldi, \None)$ (local declarations of \texttt{let} variables and constants require
        an initializing expression, otherwise they are rejected by an ASL parser);
  \item annotating the uninitialised local declarations $\ldi$ in $\tenv$ yields \\
        $(\newtenv, \ldione, \veinit)$;
  \item $\news$ is a local declaration statement with variable keyword, local identifiers $\ldione$, and the initializing expression $\veinit$,
        that is, $\SDecl(\LDKVar, \ldione, \langle\veinit\rangle)$.
  \end{itemize}
\end{itemize}
\subsubsection{Formally}
\begin{mathpar}
\inferrule[constant]{
  \annotateexpr{\tenv, \ve} \typearrow (\vte, \vep) \OrTypeError\\\\
  \annotatelocaldeclitem{\tenv, \vte, \ldk, \langle\vep\rangle, \ldi} \typearrow (\tenvone, \ldione)\\\\
  \commonprefixline\\\\
  \ldk = \LDKConstant\\
  \reduceconstants(\tenvone, \ve) \typearrow \vv \OrTypeError\\\\
  \declarelocalconstant(\tenvone, \vv, \ldione) \typearrow \newtenv\\
  \news \eqdef \SDecl(\LDKConstant, \ldione, \langle\vep\rangle)
}{
  \annotatestmt(\tenv, \overname{\SDecl(\ldk, \ldi, \langle\ve\rangle)}{\vs}) \typearrow (\news, \newtenv)
}
\end{mathpar}

\begin{mathpar}
\inferrule[non\_constant]{
  \annotateexpr{\tenv, \ve} \typearrow (\vte, \vep) \OrTypeError\\\\
  \annotatelocaldeclitem{\tenv, \vte, \ldk, \langle\vep\rangle, \ldi} \typearrow (\tenvone, \ldione)\\\\
  \commonprefixline\\\\
  \ldk \neq \LDKConstant\\
  \news \eqdef \SDecl(\ldk, \ldione, \langle\vep\rangle)
}{
  \annotatestmt(\tenv, \overname{\SDecl(\ldk, \ldi, \langle\ve\rangle)}{\vs}) \typearrow (\news, \overname{\tenvone}{\newtenv})
}
\end{mathpar}
\lrmcomment{This is related to \identr{YSPM}.}

\begin{mathpar}
\inferrule[none]{
  \annotatelocaldeclitemuninit(\tenv, \ldi) \typearrow (\newtenv, \ldione, \veinit) \OrTypeError\\\\
  \news \eqdef \SDecl(\LDKVar, \ldione, \langle\veinit\rangle)
}{
  \annotatestmt(\tenv, \overname{\SDecl(\LDKVar, \ldi, \None)}{\vs}) \typearrow (\news, \newtenv)
}
\end{mathpar}
\CodeSubsection{\SDeclegin}{\SDeclEnd}{../Typing.ml}

\subsubsection{TypingRule.DeclareLocalConstant \label{sec:TypingRule.DeclareLocalConstant}}
\hypertarget{def-declarelocalconstant}{}
The helper function
\[
\declarelocalconstant(\overname{\staticenvs}{\tenv} \aslsep \overname{\literal}{\vv} \aslsep \overname{\localdeclitem}{\ldi})
\typearrow \overname{\staticenvs}{\newtenv}
\]
adds the literal $\vv$ with the local declaration item $\ldi$ as a constant to the local component of the static environment $\tenv$,
yielding the modified static environment $\newtenv$.

\subsubsection{Prose}
One of the following applies:
\begin{itemize}
  \item All of the following apply (\textsc{discard}):
  \begin{itemize}
    \item $\ldi$ corresponds to a discarding declaration, that is, $\LDIDiscard$;
    \item $\newtenv$ is $\tenv$.
  \end{itemize}

  \item All of the following apply (\textsc{var}):
  \begin{itemize}
    \item $\ldi$ corresponds to a variable declaration for $\vx$, that is, $\LDIVar(\vx)$;
    \item applying $\addlocalconstant$ to $\vx$ and $\vv$ in $\tenv$ yields $\newtenv$.
  \end{itemize}

  \item All of the following apply (\textsc{tuple}):
  \begin{itemize}
    \item $\ldi$ corresponds to a tuple declaration, that is, $\LDIVar(\Ignore)$;
    \item this case is not yet implemented.
  \end{itemize}

  \item All of the following apply (\textsc{typed}):
  \begin{itemize}
    \item $\ldi$ corresponds to a typed declaration of the local declaration item $\ldip$ and some type, that is, $\LDITyped(\ldip, \Ignore)$;
    \item applying $\declarelocalconstant$ to $\vv$ and $\ldip$ in $\tenv$ yields $\newtenv$.
  \end{itemize}
\end{itemize}
\subsubsection{Formally}
\begin{mathpar}
\inferrule[discard]{}{
  \declarelocalconstant(\tenv, \vv, \overname{\LDIDiscard}{\ldi}) \typearrow \overname{\tenv}{\newtenv}
}
\end{mathpar}

\begin{mathpar}
\inferrule[var]{
  \addlocalconstant(\tenv, \vx, \vv) \typearrow \newtenv
}{
  \declarelocalconstant(\tenv, \vv, \overname{\LDIVar(\vx)}{\ldi}) \typearrow \newtenv
}
\end{mathpar}

\begin{mathpar}
\inferrule[tuple]{}{
  \declarelocalconstant(\tenv, \vv, \overname{\LDITuple(\Ignore)}{\ldi}) \typearrow \tododefine{not implemented yet}
}
\end{mathpar}

\begin{mathpar}
\inferrule[typed]{
  \declarelocalconstant(\tenv, \vv, \ldip) \typearrow \newtenv
}{
  \declarelocalconstant(\tenv, \vv, \overname{\LDITyped(\ldip, \Ignore)}{\ldi}) \typearrow \newtenv
}
\end{mathpar}
\CodeSubsection{\DeclareLocalConstantBegin}{\DeclareLocalConstantEnd}{../Typing.ml}

\subsubsection{TypingRule.AnnotateLocalDeclItemUninit \label{sec:TypingRule.AnnotateLocalDeclItemUninit}}
\hypertarget{def-annotatelocaldeclitemuninit}{}
The helper function
\[
\begin{array}{r}
\annotatelocaldeclitemuninit(\overname{\staticenvs}{\tenv} \aslsep \overname{\localdeclitem}{\ldi})
\typearrow \\
(\overname{\staticenvs}{\newtenv} \times \overname{\localdeclitem}{\newldi} \times \overname{\expr}{\veinit})
\cup \overname{\TTypeError}{\TypeErrorConfig}
\end{array}
\]
annotates the local declaration for a variable declaration without an initializing expressions in the static environment $\tenv$,
yielding a triple consisting of the annotated local declaration item $\newldi$, the modified static environment $\newtenv$,
and an initializing expression $\veinit$.
\ProseOtherwiseTypeError

\subsubsection{Prose}
One of the following applies:
\begin{itemize}
  \item All of the following apply (\textsc{typed}):
  \begin{itemize}
    \item $\ldi$ corresponds to a variable declaration via the local declaration item $\ldip$ and type annotation $\vt$,
          that is, $\LDITyped(\ldip, \vt)$;
    \item applying $\basevalue$ to $\vtp$ in $\tenv$ yields $\veinit$\ProseOrTypeError;
    \item annotating $\vt$ in $\tenv$ yields $\vtp$\ProseOrTypeError;
    \item annotating the local declaration item $\ldip$ with the type $\vtp$ and local declaration keyword $\LDIVar$
          yields $(\newtenv, \newldip)$\ProseOrTypeError;
    \item define $\newldi$ as the typed local declaration item with local declaration item $\newldip$, type $\vtp$, and
          $\veinit$.
  \end{itemize}

  \item All of the following apply (\textsc{error}):
  \begin{itemize}
    \item $\ldi$ does not correspond to a typed declaration item;
    \item the result is a type error indicating that an initializing expression was expected.
  \end{itemize}
\end{itemize}

\subsubsection{Formally}
\begin{mathpar}
\inferrule[typed]{
  \annotatetype{\tenv, \vt} \typearrow \vtp \OrTypeError\\\\
  \basevalue(\tenv, \vtp) \typearrow \veinit \OrTypeError\\\\
  {
    \begin{array}{r}
      \annotatelocaldeclitem{\tenv, \vtp, \LDKVar, \None, \ldip} \typearrow \\
      (\newtenv, \newldip) \OrTypeError
    \end{array}
  }\\
  \newldi \eqdef \LDITyped(\newldip, \vtp, \veinit)
}{
  \annotatelocaldeclitemuninit(\tenv, \overname{\LDITyped(\ldip, \vt)}{\ldi}) \typearrow (\newtenv, \newldi)
}
\end{mathpar}

\begin{mathpar}
\inferrule[error]{
  \astlabel(\ldi) \in \{\LDIDiscard, \LDIVar, \LDITuple\}
}{
  \annotatelocaldeclitemuninit(\tenv, \ldi) \typearrow \TypeErrorVal{ExpectedInitializingExpression}
}
\end{mathpar}
\CodeSubsection{\AnnotateLocalDeclItemUninitBegin}{\AnnotateLocalDeclItemUninitEnd}{../Typing.ml}

\subsection{Semantics}
\subsubsection{SemanticsRule.SDeclSome\label{sec:SemanticsRule.SDeclSome}}
\subsubsection{Example (Declaration With an Initializing Value)}
The specification:
\VerbatimInput{../tests/ASLSemanticsReference.t/SemanticsRule.SDeclSome.asl}
\texttt{let x = 3;} binds \texttt{x} to $\nvint(3)$ in the empty environment.

\subsubsection{Example (Declaration Without an Initializing Value)}
In the specification:
\VerbatimInput{../tests/ASLSemanticsReference.t/SemanticsRule.SDeclNone.asl}
\texttt{var x : integer;} binds \texttt{x} in $\env$ to the base value of \texttt{integer}.

\subsubsection{Prose}
One of the following applies:
\begin{itemize}
  \item All of the following apply (\textsc{some}):
  \begin{itemize}
    \item $\vs$ is a declaration with an initial value,
    $\SDecl(\text{ldk}, \ldi, \langle\ve\rangle)$;
    \item evaluating $\ve$ in $\env$ is $\Normal(\vm, \envone)$\ProseOrAbnormal;
    \item evaluating the local declaration $\ldi$ with $\langle\vm\rangle$ as the initializing
    value in $\envone$ as per \chapref{LocalStorageDeclarations} is $\Normal(\newg, \newenv)$;
    \item the result of the entire evaluation is $\Continuing(\newg, \newenv)$.
  \end{itemize}

  \item All of the following apply (\textsc{none}):
  \begin{itemize}
    \item $\vs$ is a declaration without an initial value, $\SDecl(\Ignore, \ldi, \None)$;
    \item evaluating the local declaration $(\ldi, \None)$ as per \chapref{LocalStorageDeclarations}
    is \\ $\Normal(\newg, \newenv)$;
    \item the result of the entire evaluation is $\Continuing(\newg, \newenv)$.
  \end{itemize}
\end{itemize}
\subsubsection{Formally}
\begin{mathpar}
\inferrule[some]{
  \evalexpr{\env, \ve} \evalarrow \Normal(\vm, \envone) \OrAbnormal\\
  \evallocaldecl{\envone, \ldi, \langle\vm\rangle} \evalarrow \Normal(\newg, \newenv)\\
}{
  \evalstmt{\env, \SDecl(\Ignore, \ldi, \langle\ve\rangle)} \evalarrow \Continuing(\newg, \newenv)
}
\end{mathpar}

\begin{mathpar}
\inferrule[none]{
  \evallocaldecl{\env, \vs, \ldi, \None} \evalarrow \Normal(\newg, \newenv)\\
}{
  \evalstmt{\env, \SDecl(\Ignore, \ldi, \None)} \evalarrow \Continuing(\newg, \newenv)
}
\end{mathpar}
\CodeSubsection{\EvalSDeclBegin}{\EvalSDeclEnd}{../Interpreter.ml}

\section{Sequencing Statements\label{sec:SeuqncingStatements}}
\subsection{Syntax}
\begin{flalign*}
\Nstmtlist \derivesinline\ & \nonemptylist{\Nstmt} &
\end{flalign*}

\subsection{Abstract Syntax}
\begin{flalign*}
\stmt \derives\ & \SSeq(\stmt, \stmt) &
\end{flalign*}

\subsubsection{ASTRule.StmtList \label{sec:ASTRule.StmtList}}
\hypertarget{build-stmtlist}{}
The function
\[
  \buildstmtlist(\overname{\parsenode{\Nstmtlist}}{\vparsednode}) \;\aslto\; \overname{\stmt}{\vastnode}
\]
transforms a parse node $\vparsednode$ into an AST node $\vastnode$.

\begin{mathpar}
\inferrule{
  \buildlist[\Nstmt](\vstmts) \astarrow \vstmtlist\\
  \stmtfromlist(\vstmtlist) \astarrow \vastnode
}{
  \buildstmtlist(\Nstmtlist(\namednode{\vstmts}{\nonemptylist{\Nstmt}})) \astarrow \vastnode
}
\end{mathpar}

\subsubsection{ASTRule.StmtFromList \label{sec:ASTRule.StmtFromList}}
\hypertarget{def-stmtfromlist}{}
The helper function
\[
\stmtfromlist(\overname{\stmt^*}{\vstmts}) \aslto \overname{\stmt}{\news}
\]
builds a statement $\news$ from a possibly-empty list of statements $\vstmts$.

\begin{mathpar}
\inferrule[empty]{
}{
  \stmtfromlist(\overname{\emptylist}{\vstmts}) \astarrow \overname{\SPass}{\news}
}
\and
\inferrule[non\_empty]{
  \stmtfromlist(\vstmtsone) \astarrow \vsone\\
  \sequencestmts(\vs, \vsone) \astarrow \news
}{
  \stmtfromlist(\overname{[\vs] \concat \vstmtsone}{\vstmts}) \astarrow \news
}
\end{mathpar}

\subsubsection{ASTRule.SequenceStmts \label{sec:ASTRule.SequenceStmts}}
\hypertarget{def-sequencestmts}{}
The helper function
\[
\sequencestmts(\overname{\stmt}{\vsone}, \overname{\stmt}{\vstwo}) \aslto \overname{\stmt}{\news}
\]
Combines the statement $\vsone$ with $\vstwo$ into the statement $\news$, while filtering away
instances of $\SPass$.

\begin{mathpar}
\inferrule[s1\_spass]{}{
  \sequencestmts(\overname{\SPass}{\vsone}, \vstwo) \astarrow \overname{\vstwo}{\news}
}
\and
\inferrule[s2\_spass]{
  \vsone \neq \SPass
}{
  \sequencestmts(\vsone, \overname{\SPass}{\vstwo}) \astarrow \overname{\vsone}{\news}
}
\and
\inferrule[no\_spass]{
  \vsone \neq \SPass\\
  \vstwo \neq \SPass
}{
  \sequencestmts(\vsone, \vstwo) \astarrow \overname{\SSeq(\vsone, \vstwo)}{\news}
}
\end{mathpar}

\subsection{Typing}
\subsubsection{TypingRule.SSeq \label{sec:TypingRule.SSeq}}
\subsubsection{Prose}
All of the following apply:
\begin{itemize}
  \item $\vs$ is the AST node for the sequence of statements $\vsone$ and $\vstwo$, that is, $\SSeq(\vsone, \vstwo)$;
  \item annotating $\vsone$ in $\tenv$ yields $(\newsone, \tenvone)$\ProseOrTypeError;
  \item annotating $\vstwo$ in $\tenvone$ yields $(\newstwo, \newtenv)$\ProseOrTypeError;
  \item $\news$ is the AST node for the sequence of statements $\newsone$ and $\newstwo$, that is, $\SSeq(\newsone, \newstwo)$.
\end{itemize}
\subsubsection{Formally}
\begin{mathpar}
\inferrule{
  \annotatestmt(\tenv, \vs1) \typearrow (\newsone, \tenvone) \OrTypeError\\\\
  \annotatestmt(\tenvone, \vs2) \typearrow (\newstwo, \newtenv) \OrTypeError
}{
  \annotatestmt(\tenv, \overname{\SSeq(\vsone, \vstwo)}{\vs}) \typearrow (\overname{\SSeq(\newsone, \newstwo)}{\news}, \newtenv)
}
\end{mathpar}
\CodeSubsection{\SSeqBegin}{\SSeqEnd}{../Typing.ml}

\subsection{Semantics}
\subsubsection{SemanticsRule.SSeq\label{sec:SemanticsRule.SSeq}}
\subsubsection{Example}
In the specification:
\VerbatimInput{../tests/ASLSemanticsReference.t/SemanticsRule.SSeq.asl}
\texttt{let x = 3; let y = x + 1} evaluates \texttt{let x = 3} then \texttt{let y = x + 1}.

\subsubsection{Prose}
All of the following apply:
\begin{itemize}
  \item $\vs$ is a \emph{sequencing statement} \texttt{s1; s2}, that is, $\SSeq(\vsone, \vstwo)$;
  \item evaluating $\vsone$ in $\env$ is either $\Continuing(\vgone, \envone)$ in which case
  the evaluation continues,
  or a returning configuration ($\Returning((\vvs, \newg), \newenv)$)\ProseOrAbnormal;
  \item evaluating $\vstwo$ in $\envone$ yields a non-abnormal configuration \\
        (either $\Normal$ or $\Continuing$) $C$\ProseOrAbnormal;
  \item $\newg$ is the ordered composition of $\vgone$ and the execution graph of $C$ with the
  $\aslpo$ edge;
  \item $D$ is the configuration $C$ with the execution graph component replaced with $\newg$.
\end{itemize}
\subsubsection{Formally}
\begin{mathpar}
  \inferrule{
    \evalstmt{\env, \vsone} \evalarrow \Continuing(\vgone, \envone) \terminateas \ReturningConfig,\ThrowingConfig,\DynErrorConfig\\
    \evalstmt{\envone, \vstwo} \evalarrow C \OrAbnormal\\
    \newg \eqdef \ordered{\vgone}{\aslpo}{\graphof{C}}\\
    D \eqdef \withgraph{C}{\newg}
  }
  {
    \evalstmt{\env, \SSeq(\vsone, \vstwo)} \evalarrow D
  }
\end{mathpar}
\CodeSubsection{\EvalSSeqBegin}{\EvalSSeqEnd}{../Interpreter.ml}

\section{Call Statements\label{sec:CallStatements}}
\subsection{Syntax}
\begin{flalign*}
\Nstmt \derivesinline\ & \Tidentifier \parsesep \Plist{\Nexpr} \parsesep \Tsemicolon &
\end{flalign*}

\subsection{Abstract Syntax}
\begin{flalign*}
\stmt \derives\ & \SCall(\overtext{\identifier}{subprogram name}, \overtext{\expr^{*}}{actual arguments}) &
\end{flalign*}

\subsubsection{ASTRule.SCall}
\begin{mathpar}
\inferrule{
  \buildplist[\Nexpr](\vargs) \astarrow \astversion{\vargs}
}{
  \buildstmt(\overname{\Nstmt(\Tidentifier(\vx), \namednode{\vargs}{\Plist{\Nexpr}}, \Tsemicolon)}{\vparsednode})
  \astarrow
  \overname{\SCall(\vx, \astversion{\vargs})}{\vastnode}
}
\end{mathpar}

\subsection{Typing}
\subsubsection{TypingRule.SCall \label{sec:TypingRule.SCall}}
\subsubsection{Prose}
All of the following apply:
\begin{itemize}
  \item $\vs$ is a call to a subprogram named $\name$ with arguments $\vargs$;
  \item annotating the call to $\name$ with arguments $\vargs$, as a procedure (that is, with $\STProcedure$),
        as per \chapref{SubprogramCalls} (which makes sure that the call does not have a return type),
        yields $(\newname, \newargs, \eqs, \None)$\ProseOrTypeError;
  \item $\news$ is the call to a subprogram named $\newname$ with arguments
        $\newargs$ and parameter assignments $\neweqs$;
  \item $\newtenv$ is $\tenv$.
\end{itemize}
\subsubsection{Formally}
\begin{mathpar}
\inferrule{
  {
    \begin{array}{r}
      \annotatecall(\tenv, \name, \vargs, \STProcedure) \typearrow \\ (\newname, \newargs, \eqs, \None) \OrTypeError
    \end{array}
  }
}{
  {
    \begin{array}{r}
  \annotatestmt(\tenv, \overname{\SCall(\name, \vargs)}{\vs}) \typearrow \\
  (\overname{\SCall(\newname, \newargs, \eqs)}{\news}, \tenv)
    \end{array}
  }
}
\end{mathpar}
\CodeSubsection{\SCallBegin}{\SCallEnd}{../Typing.ml}

\subsubsection{Comments}
Notice that the input statement, which belongs to the untyped AST, has two children nodes ---
$\name$ and $\vargs$, whereas the output statement, which belongs to the typed AST has the additional
node $\neweqs$, which associates expressions with parameters.
\lrmcomment{This is related to \identd{VXKM}.}

\subsection{Semantics}
\subsubsection{SemanticsRule.SCall \label{sec:SemanticsRule.SCall}}
\subsubsection{Example}
In the specification:
\VerbatimInput{../tests/ASLSemanticsReference.t/SemanticsRule.SCall.asl}
\texttt{Zeros(3)} evaluates to \texttt{'000'}.

\subsubsection{Prose}
All of the following apply:
\begin{itemize}
  \item $\vs$ is a call statement, $\SCall(\name, \args, \namedargs)$;
  \item evaluating the subprogram call as per \chapref{SubprogramCalls} is
  \\ $\Normal(\newg, \newenv)$\ProseOrAbnormal;
  \item the result of the entire evaluation is $\Continuing(\newg, \newenv)$.
\end{itemize}

\subsubsection{Formally}
\begin{mathpar}
\inferrule{
  \evalcall{\env, \name, \args, \namedargs} \evalarrow \Normal(\newg, \newenv) \OrAbnormal
}{
  \evalstmt{\env, \overname{\SCall(\name, \args, \namedargs)}{\vs}} \evalarrow \Continuing(\newg, \newenv)
}
\end{mathpar}
\CodeSubsection{\EvalSCallBegin}{\EvalSCallEnd}{../Interpreter.ml}
% \lrmcomment{This is related to \identd{KCYT}:}

\section{Conditional Statements\label{sec:ConditionalStatements}}
\subsection{Syntax}
\begin{flalign*}
\Nstmt \derivesinline\ & \Tif \parsesep \Nexpr \parsesep \Tthen \parsesep \Nstmtlist \parsesep \Nselse \parsesep \Tend \parsesep \Tsemicolon &\\
\Nselse \derives\ & \Telseif \parsesep \Nexpr \parsesep \Tthen \parsesep \Nstmtlist \parsesep \Nselse &\\
        |\ & \Tpass &\\
        |\ & \Telse \parsesep \Nstmtlist &
\end{flalign*}

\subsection{Abstract Syntax}
\begin{flalign*}
\stmt \derives\ & \SCond(\expr, \stmt, \stmt)
\end{flalign*}

\subsubsection{ASTRule.SCond}
\begin{mathpar}
\inferrule{}{
  {
    \begin{array}{r}
  \buildstmt(\overname{\Nstmt(\Tif, \punnode{\Nexpr}, \Tthen, \punnode{\Nstmtlist}, \punnode{\Nselse}, \Tend, \Tsemicolon)}{\vparsednode})
  \astarrow \\
  \overname{\SCond(\astof{\vexpr}, \astof{\vstmtlist}, \astof{\velse})}{\vastnode}
    \end{array}
  }
}
\end{mathpar}

\subsubsection{ASTRule.SElse \label{sec:ASTRule.SElse}}
\hypertarget{build-selse}{}
The function
\[
  \buildselse(\overname{\parsenode{\Nselse}}{\vparsednode}) \;\aslto\; \overname{\stmt}{\vastnode}
\]
transforms a parse node $\vparsednode$ into an AST node $\vastnode$.

\begin{mathpar}
\inferrule[elseif]{}{
  {
    \begin{array}{r}
  \buildselse(\Nselse(\Telseif, \Nexpr, \Twhen, \Nstmtlist, \Nselse)) \astarrow \\
  \overname{\SCond(\astof{\vexpr}, \astof{\vstmtlist}, \astof{\vselse})}{\vastnode}
    \end{array}
  }
}
\end{mathpar}

\begin{mathpar}
\inferrule[pass]{}{
  \buildselse(\Nselse(\Tpass)) \astarrow \overname{\SPass}{\vastnode}
}
\end{mathpar}

\begin{mathpar}
\inferrule[else]{}{
  \buildselse(\Nselse(\Telse, \punnode{\Nstmtlist})) \astarrow \overname{\astof{\vstmtlist}}{\vastnode}
}
\end{mathpar}

\subsection{Typing}
\subsubsection{TypingRule.SCond \label{sec:TypingRule.SCond}}
\subsubsection{Prose}
All of the following apply:
\begin{itemize}
  \item $\vs$ is a condition $\ve$ with the statements $\vsone$ and $\vstwo$, that is, $\SCond(\ve, \vsone, \vstwo)$;
  \item annotating the right-hand-side expression $\ve$ in $\tenv$ yields $(\tcond, \econd)$\ProseOrTypeError;
  \item checking that $\tcond$ \typesatisfies\ $\TBool$ yields $\True$\ProseOrTypeError;
  \item annotating the statement $\vsone$ in $\tenv$ yields $\vsonep$\ProseOrTypeError;
  \item annotating the statement $\vstwo$ in $\tenv$ yields $\vstwop$\ProseOrTypeError;
  \item $\news$ is the condition $\econd$ with the statements $\vsonep$ and $\vstwop$, that is, \\ $\SCond(\econd, \vsonep, \vstwop)$;
  \item $\newtenv$ is $\tenv$.
\end{itemize}
\subsubsection{Formally}
\begin{mathpar}
\inferrule{
  \annotateexpr{\tenv, \ve} \typearrow (\tcond, \econd) \OrTypeError\\\\
  \checktypesat(\tenv, \tcond, \TBool) \typearrow \True \OrTypeError\\\\
  \annotateblock{\tenv, \vsone} \typearrow \vsonep \OrTypeError\\\\
  \annotateblock{\tenv, \vstwo} \typearrow \vstwop \OrTypeError
}{
  \annotatestmt(\tenv, \overname{\SCond(\ve, \vsone, \vstwo)}{\vs}) \typearrow
  (\overname{\SCond(\econd, \vsonep, \vstwop)}{\news}, \overname{\tenv}{\newtenv})
}
\end{mathpar}
\CodeSubsection{\SCondBegin}{\SCondEnd}{../Typing.ml}
\lrmcomment{This is related to \identr{NBDJ}.}

\subsection{Semantics}
subsubsection{SemanticsRule.SCond\label{sec:SemanticsRule.SCond}}
\subsubsection{Examples}
The specification:
\VerbatimInput{../tests/ASLSemanticsReference.t/SemanticsRule.SCond.asl}
does not result in any Assertion Error.

The specification:
\VerbatimInput{../tests/ASLSemanticsReference.t/SemanticsRule.SCond2.asl}

The specification:
\VerbatimInput{../tests/ASLSemanticsReference.t/SemanticsRule.SCond3.asl}

The specification:
\VerbatimInput{../tests/ASLSemanticsReference.t/SemanticsRule.SCond4.asl}

\subsubsection{Prose}
All of the following apply:
\begin{itemize}
\item $\vs$ is a condition statement, $\SCond(\ve, \vsone, \vstwo)$;
\item evaluating $\ve$ in $\env$ is $\Normal(\vv, \vgone)$\ProseOrAbnormal;
\item $\vv$ is a native Boolean for $\vb$;
\item the statement $\vsp$ is $\vsone$ is $\vb$ is $\True$ and $\vstwo$ otherwise
(so that $\vsone$ will be evaluated if the condition evaluates to $\True$ and otherwise
$\vstwo$ will be evaluated);
\item evaluating $\vsp$ in $\envone$ as per \chapref{Blocks} is a non-abnormal configuration
      (either $\Normal$ or $\Continuing$) $C$\ProseOrAbnormal;
\item $\vg$ is the ordered composition of $\vgone$ and the execution graph of the configuration $C$;
\item $D$ is the configuration $C$ with the execution graph component updated to be $\vg$.
\end{itemize}

\subsubsection{Formally}
\begin{mathpar}
\inferrule{
  \evalexpr{\env, \ve} \evalarrow \Normal((\vv, \vgone), \envone) \OrAbnormal\\
  \vv \eqname \nvbool(\vb)\\
  \vsp \eqdef \choice{\vb}{\vsone}{\vstwo}\\
  \evalblock{\envone, \vsp} \evalarrow C \OrAbnormal\\\\
  \vg \eqdef \ordered{\vgone}{\aslctrl}{\graphof{C}}\\
  D \eqdef \withgraph{C}{\vg}
}{
  \evalstmt{\env, \overname{\SCond(\ve, \vsone, \vstwo)}{\vs}} \evalarrow D
}
\end{mathpar}
\CodeSubsection{\EvalSCondBegin}{\EvalSCondEnd}{../Interpreter.ml}

\section{Case Statements\label{sec:CaseStatements}}
Case statements are considered syntactic sugar for compound condition statements,
as defined by \nameref{sec:TypingRule.DesugarCaseStmt}.

\subsection{Syntax}
\begin{flalign*}
\Nstmt \derivesinline\ & \Tcase \parsesep \Nexpr \parsesep \Tof \parsesep \Ncasealtlist \parsesep \Tend \parsesep \Tsemicolon &\\
\Ncasealtlist \derivesinline\ & \NClist{\Ncasealt} \parsesep &\\
                           |\ & \NClist{\Ncasealt} \parsesep \Ncaseotherwise &\\
\Ncasealt \derivesinline\ & \Twhen \parsesep \Npatternlist \parsesep \option{\Twhere \parsesep \Nexpr} \parsesep \Tarrow \parsesep \Nstmtlist &\\
\Ncaseotherwise \derivesinline\ & \Totherwise \parsesep \Tarrow \parsesep \Nstmtlist &
\end{flalign*}

\subsection{Abstract Syntax}
\begin{flalign*}
\stmt \derives\ & \SCase(\expr, \casealt^*) &
\end{flalign*}

\subsubsection{ASTRule.SCase}
\begin{mathpar}
\inferrule{
  \buildlist[\Ncasealt](\vcasealtlist) \astarrow \vcasealtlistast
}{
  {
    \begin{array}{r}
  \buildstmt(\overname{\Nstmt(\Tcase, \punnode{\Nexpr}, \Tof, \namednode{\vcasealtlist}{\Ncasealtlist}, \Tend, \Tsemicolon)}{\vparsednode})
  \astarrow \\
  \overname{\SCase(\astof{\vexpr}, \vcasealtlistast)}{\vastnode}
    \end{array}
  }
}
\end{mathpar}

\subsubsection{ASTRule.CaseAltList\label{sec:ASTRule.CaseAltList}}
\hypertarget{build-casealtlist}{}
The function
\[
\buildcasealtlist(\overname{\parsenode{\Ncasealtlist}}{\vparsednode}) \;\aslto\; \overname{\casealt^+}{\vastnode}
\]
transforms a parse node $\vparsednode$ into an AST node $\vastnode$.

\begin{mathpar}
\inferrule[no\_otherwise]{
  \buildclist[\buildcasealt](\vcases) \typearrow \vastnode
}{
  \buildcasealtlist(\overname{\Ncasealtlist(\vcases : \NClist{\Ncasealt})}{\vparsednode}) \astarrow \vastnode
}
\end{mathpar}

\begin{mathpar}
\inferrule[otherwise]{
  \buildclist[\buildcasealt](\vcases) \astarrow h\\
  \buildcasealt(\votherwise) \astarrow t
}{
  {
  \buildcasealtlist\left(\overname{
      \Ncasealtlist\left(
        \begin{array}{l}
          \vcases : \NClist{\Ncasealt}, \\
          \votherwise:\Ncaseotherwise
        \end{array}
    \right)
    }{\vparsednode}\right) \astarrow
  \overname{[h] \concat t}{\vastnode}
  }
}
\end{mathpar}

\subsubsection{ASTRule.CaseAlt\label{sec:ASTRule.CaseAlt}}
\hypertarget{build-casealt}{}
The function
\[
\buildcasealt(\overname{\parsenode{\Ncasealt}}{\vparsednode}) \;\aslto\; \overname{\casealt}{\vastnode}
\]
transforms a parse node $\vparsednode$ into an AST node $\vastnode$.

\begin{mathpar}
\inferrule{
  \buildoption[\buildexpr](\vwhereopt) \astarrow \vwhereast
}{
  {
    \begin{array}{r}
  \buildcasealt\left(\overname{\Ncasealt\left(
    \begin{array}{l}
    \Twhen, \punnode{\Npatternlist}, \\
    \wrappedline\ \namednode{\vwhereopt}{\option{\Twhere, \Nexpr}}, \Tarrow, \\
    \wrappedline\ \namednode{\vstmts}{\Nstmtlist}
    \end{array}
    \right)}{\vparsednode}\right)
  \astarrow \\
  \overname{\casealt(\CasePattern: \astof{\vpatternlist}, \CaseWhere: \vwhereast, \CaseStmt: \astof{\vstmtlist})}{\vastnode}
    \end{array}
  }
}
\end{mathpar}

\subsubsection{ASTRule.CaseOtherwise\label{sec:ASTRule.CaseOtherwise}}
\hypertarget{build-caseotherwise}{}
The function
\[
\buildcaseotherwise(\overname{\parsenode{\Ncaseotherwise}}{\vparsednode}) \;\aslto\; \overname{\casealt}{\vastnode}
\]
transforms a parse node $\vparsednode$ into an AST node $\vastnode$.

\begin{mathpar}
\inferrule{}{
  {
    \begin{array}{r}
  \buildcaseotherwise(\overname{\Ncasealt(\Totherwise, \Tarrow, \namednode{\vstmts}{\Nstmtlist})}{\vparsednode})
  \astarrow \\
  \overname{\casealt(\CasePattern: \PatternAll, \CaseWhere: \None, \CaseStmt: \astof{\vstmtlist})}{\vastnode}
    \end{array}
  }
}
\end{mathpar}

\subsubsection{ASTRule.OtherwiseOpt\label{sec:ASTRule.OtherwiseOpt}}
\hypertarget{build-otherwiseopt}{}
The function
\[
\buildotherwiseopt(\overname{\parsenode{\Notherwiseopt}}{\vparsednode}) \;\aslto\; \overname{\stmt?}{\vastnode}
\]
transforms a parse node $\vparsednode$ into an AST node $\vastnode$.

\begin{mathpar}
\inferrule{
  \buildoption[\buildstmtlist] (\vv)\astarrow \vastnode
}{
  {
  \begin{array}{r}
  \buildotherwiseopt(\overname{\Notherwiseopt(\namednode{\vv}{\option{\Totherwise, \Tarrow, \Nstmtlist}})}{\vparsednode})
  \astarrow \\
  \vastnode
  \end{array}
  }
}
\end{mathpar}

\subsection{Typing}
\subsubsection{TypingRule.DesugarCaseStmt\label{sec:TypingRule.DesugarCaseStmt}}
\hypertarget{def-desugarcasestmt}{}
The relation
\[
\desugarcasestmt(\overname{\stmt}{\vs}) \;\aslrel\; \overname{\stmt}{\news}
\]
transforms a case statement $\vs$ into a conditional statement $\news$.

\subsubsection{Prose}
All of the following apply:
\begin{itemize}
  \item $\vs$ is a \texttt{case} statement with expression $\ve$ and list of cases $\vcases$, that is, $\SCase(\ve, \vcases)$;
  \item One of the following applies:
  \begin{itemize}
    \item All of the following apply (\textsc{var}):
    \begin{itemize}
      \item $\ve$ is a variable expression;
      \item applying $\casestocond$ to $\ve$ and $\vcases$ yields $\news$.
    \end{itemize}

    \item All of the following apply (\textsc{non\_var}):
    \begin{itemize}
      \item $\ve$ is not a variable expression;
      \item let $\vx$ be a fresh identifier;
      \item define $\vdeclx$ the statement declaring $\vx$ as an immutable variable initialized by $\ve$;
      \item applying $\casestocond$ to the variable expression for $\vx$ ($\EVar(\vx)$) and $\vcases$ yields
            the condition statement $\vscond$;
      \item define $\news$ as the sequence statement for $\vdeclx$ and $\vscond$.
    \end{itemize}
  \end{itemize}
\end{itemize}

\subsubsection{Formally}
\begin{mathpar}
\inferrule[var]{
  \astlabel(\ve) = \EVar\\
  \casestocond(\ve, \vcases) \typearrow \news
}{
  \desugarcasestmt(\overname{\SCase(\ve, \vcases)}{\vs}) \typearrow \news
}
\end{mathpar}

\begin{mathpar}
\inferrule[non\_var]{
  \astlabel(\ve) \neq \EVar\\
  \vx \in \Identifiers \text{ is fresh}\\
  \vdeclx \eqname \SDecl(\LDKLet, \LDIVar(\vx), \langle\ve\rangle)\\
  \casestocond(\EVar(\vx), \vcases) \typearrow \vscond
}{
  \desugarcasestmt(\overname{\SCase(\ve, \vcases)}{\vs}) \typearrow \overname{\SSeq(\vdeclx, \vscond)}{\news}
}
\end{mathpar}

\subsubsection{TypingRule.CasesToCond\label{sec:TypingRule.CasesToCond}}
\hypertarget{def-casestocond}{}
The function
\[
\casestocond(\overname{\expr}{\ve} \aslsep \overname{\casealt^*}{\vcases})
\;\aslrel\; \overname{\stmt}{\news}
\]
transforms an expression $\ve$ and a list of \texttt{case} alternatives $\vcases$
into a statement $\news$.

\subsubsection{Prose}
One of the following applies:
\begin{itemize}
  \item All of the following apply (\textsc{last}):
  \begin{itemize}
    \item $\vcases$ is the list consisting of just $\vcase$;
    \item applying $\casetocond$ to $\ve$, $\vcase$, and $\SUnreachable$ yields $\news$.
  \end{itemize}

  \item All of the following apply (\textsc{not\_last}):
  \begin{itemize}
    \item $\vcases$ is the list with $\vcase$ as its \head\ and a non-empty list $\vcasesone$ as its \tail;
    \item applying $\casestocond$ to $\ve$ and $\vcasesone$ yields $\vsone$;
    \item applying $\casetocond$ to $\ve$, $\vcase$, and $\vsone$ yields $\news$.
  \end{itemize}
\end{itemize}

\subsubsection{Formally}
\begin{mathpar}
\inferrule[last]{
  \casetocond(\ve, \vcase, \SUnreachable) \typearrow \news
}{
  \casestocond(\ve, \overname{[\vcase]}{\vcases}) \typearrow \news
}
\end{mathpar}

\begin{mathpar}
\inferrule[not\_last]{
  \vcasesone \neq \emptylist\\
  \casestocond(\ve, \vcasesone) \typearrow \vsone\\
  \casetocond(\ve, \vcase, \vsone) \typearrow \news
}{
  \casestocond(\ve, \overname{[\vcase] \concat \vcasesone}{\vcases}) \typearrow
}
\end{mathpar}

\subsubsection{TypingRule.CaseToCond\label{sec:TypingRule.CaseToCond}}
\hypertarget{def-casetocond}{}
The function
\[
\casetocond(\overname{\expr}{\vezero} \aslsep \overname{\casealt}{\vcase} \aslsep \overname{\stmt}{\vtail})
\;\aslrel\; \overname{\stmt}{\news}
\]
transforms an expression $\vezero$ (the condition used for a \texttt{case} statement),
a single \texttt{case} alternative $\vcase$, and a statement $\vtail$, which represents
a list of \texttt{case} alternatives already converted to conditionals, into a condition statement $\news$.

\subsubsection{Prose}
All of the following apply:
\begin{itemize}
  \item view $\vcase$ as the \texttt{case} alternative with pattern $\vpattern$, optional \texttt{where} clause $\vwhere$,
        and statement $\vstmt$;
  \item define $\vepattern$ as the pattern expression for expression $\vezero$ and the pattern \\
        $\vpattern$;
  \item define $\vcond$ as the binary expression with operator $\BAND$ and expressions \\
        $\vepattern$ and $\vewhere$
        if $\vwhere$ is the expression $\vewhere$ and $\vepattern$, otherwise;
  \item define $\news$ as the condition statement with the condition expression $\vcond$ and statements $\vstmt$ and $\vtail$.
\end{itemize}

\subsubsection{Formally}
\begin{mathpar}
\inferrule{
  \vcase \eqname \{ \CasePattern : \vpattern, \CaseWhere : \vwhere, \CaseStmt : \vstmt \}\\
  \vepattern \eqdef \EPattern(\vezero, \vpattern)\\
  \vcond \eqdef \choice{\vwhere = \langle\vewhere\rangle}{\EBinop(\BAND, \vepattern, \vewhere)}{\vepattern}
}{
  \casetocond(\vezero, \vcase, \vtail) \typearrow \overname{\SCond(\vcond, \vstmt, \vtail)}{\news}
}
\end{mathpar}

\subsubsection{TypingRule.SCase\label{sec:TypingRule.SCase}}
\subsubsection{Prose}
All of the following apply:
\begin{itemize}
  \item $\vs$ is a case statement;
  \item applying $\desugarcasestmt$ to $\vs$ transforms $\vs$ to a conditional statement $\vsp$;
  \item annotating $\vsp$ in $\tenv$ yields $(\news, \newtenv)$\ProseOrTypeError.
\end{itemize}
\subsubsection{Formally}
\begin{mathpar}
\inferrule{
  \astlabel(\vs) = \SCase\\
  \desugarcasestmt(\vs) \typearrow \vsp\\
  \annotatestmt(\tenv, \vsp) \typearrow (\news, \newtenv) \OrTypeError
}{
  \annotatestmt(\tenv,\vs) \typearrow (\news, \newtenv)
}
\end{mathpar}
\CodeSubsection{\SCaseBegin}{\SCaseEnd}{../Typing.ml}
\lrmcomment{This is related to \identr{WGSY}.}

\subsection{Semantics}
Since case statements are transformed into conditional statements,
they do not appear in the typed AST and thus are not associated with a semantics.

\section{Assertion Statements\label{sec:AssertionStatements}}
\subsection{Syntax}
\begin{flalign*}
\Nstmt \derivesinline\ & \Tassert \parsesep \Nexpr \parsesep \Tsemicolon &
\end{flalign*}

\subsection{Abstract Syntax}
\begin{flalign*}
\stmt \derives\ & \SAssert(\expr) &
\end{flalign*}

\subsubsection{ASTRule.SAssert}
\begin{mathpar}
\inferrule{}{
  \buildstmt(\overname{\Nstmt(\Tassert, \Nexpr, \Tsemicolon)}{\vparsednode})
  \astarrow
  \overname{\SAssert(\astof{\vexpr})}{\vastnode}
}
\end{mathpar}

\subsection{Typing}
\subsubsection{TypingRule.SAssert \label{sec:TypingRule.SAssert}}
\subsubsection{Prose}
All of the following apply:
\begin{itemize}
  \item $\vs$ is an assert statement with expression $\ve$, that is, $\SAssert(\ve)$;
  \item annotating the right-hand-side expression $\ve$ in $\tenv$ yields $(\tep,\vep)$\ProseOrTypeError;
  \item checking that $\vtep$ \typesatisfies\ $\TBool$ in $\tenv$ yields $\True$\ProseOrTypeError;
  \item $\news$ is an assert statement with expression $\vep$, that is, $\SAssert(\vep)$;
  \item $\newtenv$ is $\tenv$.
\end{itemize}
\subsubsection{Formally}
\begin{mathpar}
\inferrule{
  \annotateexpr{\tenv, \ve} \typearrow (\vtep, \vep) \OrTypeError\\\\
  \checktypesat(\tenv, \vtep, \TBool) \typearrow \True \OrTypeError
}{
  \annotatestmt(\tenv, \overname{\SAssert(\ve)}{\vs}) \typearrow (\overname{\SAssert(\vep)}{\news}, \overname{\tenv}{\newtenv})
}
\end{mathpar}
\CodeSubsection{\SAssertBegin}{\SAssertEnd}{../Typing.ml}
\lrmcomment{This is related to \identr{JQYF}.}

\subsection{Semantics}
\subsubsection{SemanticsRule.SAssert \label{sec:SemanticsRule.SAssert}}
\subsubsection{Example}
In the specification:
\VerbatimInput{../tests/ASLSemanticsReference.t/SemanticsRule.SAssertOk.asl}
\texttt{assert (42 != 3);} ensures that \texttt{3} is not equal to \texttt{42}.

\subsubsection{Example}
In the specification:
\VerbatimInput{../tests/ASLSemanticsReference.t/SemanticsRule.SAssertNo.asl}
\texttt{assert (42 == 3);} results in an \texttt{AssertionFailed} error.
\subsubsection{Prose}
All of the following apply:
\begin{itemize}
  \item $\vs$ is an assertion statement, $\SAssert(\ve)$;
  \item one of the following holds:
  \begin{itemize}
    \item all of the following hold (\textsc{okay}):
    \begin{itemize}
      \item evaluating $\ve$ in $\env$ is $\Normal((\vv, \newg), \newenv)$\ProseOrAbnormal;
      \item $\vv$ is a native Boolean value for $\True$;
      \item the resulting configuration is $\Continuing(\newg, \newenv)$.
    \end{itemize}

    \item all of the following hold (\textsc{error}):
    \begin{itemize}
      \item evaluating $\ve$ in $\env$ is $\Normal((\vv, \newg), \newenv)$;
      \item $\vv$ is a native Boolean value for $\False$;
      \item an AssertionFailed error is returned.
    \end{itemize}
  \end{itemize}
\end{itemize}
\subsubsection{Formally}
\begin{mathpar}
\inferrule[okay]{
  \evalexpr{\env, \ve} \evalarrow \Normal((\vv, \newg), \newenv) \OrAbnormal\\\\
  \vv \eqname \nvbool(\True)
}{
  \evalstmt{\env, \SAssert(\ve)} \evalarrow \Continuing(\newg, \newenv)
}
\end{mathpar}

\begin{mathpar}
  \inferrule[error]{
  \evalexpr{\env, \ve} \evalarrow \Normal((\vv, \Ignore), \Ignore)\\
  \vv \eqname \nvbool(\False)
}{
  \evalstmt{\env, \SAssert(\ve)} \evalarrow \ErrorVal{AssertionFailed}
}
\end{mathpar}
\CodeSubsection{\EvalSAssertBegin}{\EvalSAssertEnd}{../Interpreter.ml}
\lrmcomment{This is related to \identd{QJYV}:}
\lrmcomment{This is related to \identr{WZSL}:}
\lrmcomment{This is related to \identr{WQRN}:}

\section{While Statements\label{sec:WhileStatements}}
\subsection{Syntax}
\begin{flalign*}
\Nstmt \derivesinline\ & \Twhile \parsesep \Nexpr \parsesep \Tdo \parsesep \Nstmtlist \parsesep \Tend \parsesep \Tsemicolon &\\
|\ & \Tlooplimit \parsesep \Tlpar \parsesep \Nexpr \parsesep \Trpar \parsesep \Twhile \parsesep \Nexpr \parsesep \Tdo \parsesep \Nstmtlist \parsesep \Tend \parsesep \Tsemicolon &\\
\end{flalign*}

\subsection{Abstract Syntax}
\begin{flalign*}
\stmt \derives\ & \SWhile(\overtext{\expr}{condition}, \overtext{\expr?}{loop limit}, \overtext{\stmt}{loop body}) &
\end{flalign*}

\subsubsection{ASTRule.SWhile}
\begin{mathpar}
\inferrule[no\_limit]{}{
  {
    \begin{array}{r}
  \buildstmt(\overname{\Nstmt(\Twhile, \punnode{\Nexpr}, \Tdo, \punnode{\Nstmtlist}, \Tend, \Tsemicolon)}{\vparsednode})
  \astarrow\\
  \overname{\SWhile(\astof{\vexpr}, \None, \astof{\vstmtlist})}{\vastnode}
\end{array}
}
}
\end{mathpar}

\begin{mathpar}
\inferrule[limit]{
  \buildexpr(\vlimitexpr) \astarrow \astversion{\vlimitexpr}
}{
  {
    \begin{array}{r}
  \buildstmt\left(\overname{\Nstmt\left(
    \begin{array}{r}
    \Tlooplimit, \Tlpar, \namednode{\vlimitexpr}{\Nexpr}, \Trpar, \Twhile,  \\
    \wrappedline\ \punnode{\Nexpr}, \Tdo, \punnode{\Nstmtlist}, \Tend, \Tsemicolon
    \end{array}
    \right)}{\vparsednode}\right)
  \astarrow\\
  \overname{\SWhile(\astof{\vexpr}, \langle\astversion{\vlimitexpr}\rangle, \astof{\vstmtlist})}{\vastnode}
\end{array}
}
}
\end{mathpar}

\subsection{Typing}
\subsubsection{TypingRule.SWhile \label{sec:TypingRule.SWhile}}
\subsubsection{Prose}
All of the following apply:
\begin{itemize}
\item $\vs$ is a \texttt{while} statement with expression $\veone$, optional limit expression $\vlimitone$,
      and statement block $\vsone$, that is, $\SWhile(\veone, \vsone)$;
\item annotating the right-hand-side expression $\veone$ in $\tenv$ yields $(\vt, \vetwo)$\ProseOrTypeError;
\item annotating the optional limit expression $\vlimitone$ via $\annotatelooplimit$ in $\tenv$ yields $\vlimittwo$\ProseOrTypeError;
\item checking that $\vt$ \typesatisfies\ $\TBool$ in $\tenv$ yields $\True$\ProseOrTypeError;
\item $\news$ is a \texttt{while} statement with expression $\vetwo$, optional limit expression $\vlimittwo$,
      and statement block $\vstwo$, that is, $\SWhile(\vetwo, \vstwo)$;
\item $\newtenv$ is $\tenv$.
\end{itemize}
\subsubsection{Formally}
\begin{mathpar}
\inferrule{
  \annotateexpr{\tenv, \veone} \typearrow (\vt, \vetwo) \OrTypeError\\\\
  \annotatelooplimit(\tenv, \vlimitone) \typearrow \vlimittwo \OrTypeError\\\\
  \checktypesat(\tenv, \vt, \TBool) \typearrow \True \OrTypeError\\\\
  \annotateblock{\tenv, \vsone} \typearrow \vstwo \OrTypeError
}{
  \annotatestmt(\tenv, \overname{\SWhile(\veone, \vlimitone, \vsone)}{\vs}) \typearrow
  (\overname{\SWhile(\vetwo, \vlimittwo, \vstwo)}{\news}, \overname{\tenv}{\newtenv})
}
\end{mathpar}
\CodeSubsection{\SWhileBegin}{\SWhileEnd}{../Typing.ml}
\lrmcomment{This is related to \identr{FTVN}.}

\subsubsection{TypingRule.AnnotateLoopLimit\label{sec:TypingRule.AnnotateLoopLimit}}
\hypertarget{def-annotatelooplimit}{}
The function
\[
  \annotatelooplimit(
    \overname{\staticenvs}{\tenv} \aslsep
    \overname{\langle\expr\rangle}{\ve} \aslsep
  ) \aslto
  \overname{\expr}{\vep} \cup\ \overname{\TTypeError}{\TypeErrorConfig}
\]
annotates an optional expression $\ve$ serving as the limit of a loop in $\tenv$,
yielding the optional loop expression $\vep$.
\ProseOtherwiseTypeError

\subsubsection{Prose}
One of the following applies:
\begin{itemize}
  \item All of the following apply (\textsc{none}):
  \begin{itemize}
    \item $\ve$ is $\None$;
    \item $\vep$ is $\None$.
  \end{itemize}

  \item All of the following apply (\textsc{some}):
  \begin{itemize}
    \item $\ve$ is $\langle\vlimit\rangle$;
    \item annotating $\vlimit$ in $\tenv$ yields $(\vt, \vlimitp)$\ProseOrTypeError;
    \item checking that $\vt$ is a constrained integer in $\tenv$ via \\
          $\checkconstrainedinteger$ yields $\True$\ProseOrTypeError;
    \item $\vep$ is $\langle\vlimitp\rangle$.
  \end{itemize}
\end{itemize}

\subsubsection{Formally}
\begin{mathpar}
\inferrule[none]{}{
  \annotatelooplimit(\tenv, \overname{\None}{\vlimit}) \typearrow \overname{\None}{\vlimitp}
}
\end{mathpar}
\begin{mathpar}
\inferrule[some]{
  \annotateexpr{\tenv, \vlimit} \typearrow (\vt, \vlimitp) \OrTypeError\\\\
  \checkconstrainedinteger(\tenv, \vt) \typearrow \True \OrTypeError
}{
  \annotatelooplimit(\tenv, \overname{\langle\vlimit\rangle}{\vlimit}) \typearrow \overname{\langle\vlimitp\rangle}{\vlimitp}
}
\end{mathpar}
\CodeSubsection{\AnnotateLoopLimitBegin}{\AnnotateLoopLimitEnd}{../Typing.ml}

\subsection{Semantics}
\subsubsection{SemanticsRule.SWhile \label{sec:SemanticsRule.SWhile}}
\subsubsection{Example}
The specification:
\VerbatimInput{../tests/ASLSemanticsReference.t/SemanticsRule.SWhile.asl}
prints \texttt{0123}.

\subsubsection{Prose}
Evaluation of the statement $\vs$ in an environment $\env$ is
the output configuration $C$ and all of the following apply:
\begin{itemize}
  \item $\vs$ is a \texttt{while} statement, $\SWhile(\ve, \Ignore, \vbody)$;
  \item evaluating the loop as per \nameref{sec:SemanticsRule.Loop} in an environment $\env$,
  with the arguments $\True$ (which conveys that this is a \texttt{while} statement), $\ve$, and $\vbody$
  results in $C$.
\end{itemize}
\subsubsection{Formally}
\begin{mathpar}
\inferrule{
  \evalloop{\env, \True, \ve, \vbody} \evalarrow C
}{
  \evalstmt{\env, \overname{\SWhile(\ve, \Ignore, \vbody)}{\vs}} \evalarrow C
}
\end{mathpar}
\CodeSubsection{\EvalSWhileBegin}{\EvalSWhileEnd}{../Interpreter.ml}

\subsubsection{SemanticsRule.Loop\label{sec:SemanticsRule.Loop}}
The relation
\hypertarget{def-evalloop}{}
\[
  \evalloop{\overname{\envs}{\env} \aslsep \overname{\Bool}{\iswhile} \aslsep \overname{\expr}{\econd} \aslsep \overname{\stmt}{\vbody}}
  \;\aslrel\;
  \left(
    \begin{array}{cl}
      \Continuing(\overname{\XGraphs}{\newg} \aslsep \overname{\envs}{\newenv}) & \cup \\
      \overname{\TReturning}{\ReturningConfig} & \cup \\
      \overname{\TThrowing}{\ThrowingConfig} & \cup \\
      \overname{\TDynError}{\DynErrorConfig} &
    \end{array}
  \right)
\]
to evaluate both \texttt{while} statements and \texttt{repeat} statements.

More specifically, $\evalloop{\env, \iswhile, \econd, \vbody}$
evaluates $\vbody$ in $\env$ as long as $\econd$ holds when $\iswhile$ is $\True$
or until $\econd$ holds when $\iswhile$ is $\False$.
The result is either the continuing configuration \\ $\Continuing(\newg,\newenv)$,
an early return configuration, or an abnormal configuration.

\subsubsection{Example}
The specification:
\VerbatimInput{../tests/ASLSemanticsReference.t/SemanticsRule.Loop.asl}
does not result in any Assertion Error and the specification terminates with exit
code $0$.

\CodeSubsection{\EvalLoopBegin}{\EvalLoopEnd}{../Interpreter.ml}
\subsection{Prose}
One of the following applies:
\begin{itemize}
\item all of the following apply (\textsc{exit}):
  \begin{itemize}
    \item evaluating $\econd$ in $\env$ is $\Normal(\condm, \newenv)$\ProseOrAbnormal;
    \item $\condm$ consists of a native Boolean for $\vb$ and an execution graph $\newg$;
    \item $\vb$ is not equal to $\iswhile$;
    \item the result of the entire evaluation is $\Continuing(\newg, \newenv)$
    and the loop is exited.
  \end{itemize}
\item all of the following apply (\textsc{continue}):
  \begin{itemize}
    \item evaluating $\econd$ in $\env$ is $\Normal(\condm, \envone)$;
    \item $\mcond$ consists of a native Boolean for $\vb$ and an execution graph $\vgone$;
    \item $\vb$ is equal to $\iswhile$;
    \item evaluating $\vbody$ in $\envone$ as per \chapref{eval_block} is either \\
    $\Continuing(\vgtwo, \envtwo)$\ProseTerminateAs{\ReturningConfig, \ThrowingConfig, \DynErrorConfig};
    \item evaluating $(\iswhile, \econd, \vbody)$ in $\envtwo$ as a loop is \\
    $\Continuing(\vgthree, \newenv)$\ProseTerminateAs{\ReturningConfig, \ThrowingConfig, \DynErrorConfig};
    \item $\newg$ is the ordered composition of $\vgone$ and $\vgtwo$ with the $\aslctrl$ label
    and then the ordered composition of the result and $\vgthree$ with the $\aslpo$ edge;
    \item the result of the entire evaluation is $\Continuing(\newg, \newenv)$.
  \end{itemize}
\end{itemize}

\subsubsection{Formally}
The premise $\vb \neq \iswhile$ is $\True$ in the case of a \texttt{while} loop
and the loop condition $\ve$ not holding, which is exactly when we want the
loop to exit. The opposite holds for a \texttt{repeat} loop.
The negation of the condition is used to decide whether to continue the loop iteration.

\begin{mathpar}
\inferrule[exit]{
  \evalexpr{\env, \econd} \evalarrow \Normal(\condm, \newenv) \OrAbnormal\\
  \condm \eqname (\nvbool(\vb), \newg)\\
  \vb \neq \iswhile
}{
  \evalloop{\env, \iswhile, \econd, \vbody} \evalarrow \Continuing(\newg, \newenv)
}
\end{mathpar}

\begin{mathpar}
\inferrule[continue]{
  \evalexpr{\env, \econd} \evalarrow \Normal(\condm, \envone)\\
  \condm \eqname (\nvbool(\vb), \vgone)\\
  \vb = \iswhile\\
  \evalblock{\envone, \vbody} \evalarrow \Continuing(\vgtwo, \envtwo) \terminateas \ReturningConfig, \ThrowingConfig, \DynErrorConfig\\
  \evalloop{\envtwo, \iswhile, \econd, \vbody} \evalarrow \Continuing(\vgthree, \newenv) \terminateas \ReturningConfig, \ThrowingConfig, \DynErrorConfig\\
  \newg \eqdef \ordered{\ordered{\vgone}{\aslctrl}{\vgtwo}}{\aslpo}{\vgthree}
}{
  \evalloop{\env, \iswhile, \econd, \vbody} \evalarrow \Continuing(\newg, \newenv)
}
\end{mathpar}

\section{Repeat Statements\label{sec:RepeatStatements}}
\subsection{Syntax}
\begin{flalign*}
\Nstmt \derivesinline\ & \Trepeat \parsesep \Nstmtlist \parsesep \Tuntil \parsesep \Nexpr \parsesep \Tsemicolon &\\
|\ & \Tlooplimit \parsesep \Tlpar \parsesep \Nexpr \parsesep \Trpar \parsesep \Trepeat \parsesep \Nstmtlist \parsesep \Tuntil \parsesep \Nexpr \parsesep \Tsemicolon &
\end{flalign*}

\subsection{Abstract Syntax}
\begin{flalign*}
\stmt \derives\ & \SRepeat(\overtext{\stmt}{loop body}, \overtext{\expr}{condition}, \overtext{\expr?}{loop limit}) &
\end{flalign*}

\subsubsection{ASTRule.SRepeat}
\begin{mathpar}
\inferrule[no\_limit]{}{
  {
    \begin{array}{r}
  \buildstmt(\overname{\Nstmt(\Trepeat, \Nstmtlist, \Tuntil, \Nexpr, \Tsemicolon)}{\vparsednode})
  \astarrow\\
  \overname{\SRepeat(\astof{\vstmtlist}, \astof{\vexpr}, \None)}{\vastnode}
    \end{array}
  }
}
\end{mathpar}

\begin{mathpar}
\inferrule[limit]{
  \buildexpr(\vlimitexpr) \astarrow \astversion{\vlimitexpr}
}{
  {
    \begin{array}{r}
  \buildstmt\left(\overname{\Nstmt\left(
    \begin{array}{r}
    \Tlooplimit, \Tlpar, \namednode{\vlimitexpr}{\Nexpr}, \Trpar, \Trepeat, \\
    \wrappedline\ \Nstmtlist, \Tuntil, \Nexpr, \Tsemicolon
    \end{array}
    \right)}{\vparsednode}\right)
  \astarrow\\
  \overname{\SRepeat(\astof{\vstmtlist}, \astof{\vexpr}, \langle\astversion{\vlimitexpr}\rangle)}{\vastnode}
    \end{array}
  }
}
\end{mathpar}

\subsection{Typing}
\subsubsection{TypingRule.SRepeat \label{sec:TypingRule.SRepeat}}
\subsubsection{Prose}
All of the following apply:
\begin{itemize}
  \item $\vs$ is a \texttt{repeat} statement with statement block $\vsone$,
        optional limit expression $\vlimitone$, and expression $\veone$, that is, $\SRepeat(\vsone, \veone, \vlimitone)$;
  \item annotating $\vsone$ as a block statement in $\tenv$ yields $\vstwo$\ProseOrTypeError;
  \item annotating the optional limit expression $\vlimitone$ via $\annotatelooplimit$ in $\tenv$ yields $\vlimittwo$\ProseOrTypeError;
  \item annotating the right-hand-side expression $\veone$ in $\tenv$ yields $(\vt, \vetwo)$\ProseOrTypeError;
  \item checking that $\vt$ \typesatisfies\ $\TBool$ in $\tenv$ yields $\True$\ProseOrTypeError;
  \item $\news$ is a \texttt{repeat} statement with statement block $\vstwo$, optional limit expression $\vlimittwo$,
        and condition expression $\vetwo$ and , that is, $\SRepeat(\vstwo, \vetwo, \vlimittwo)$;
  \item $\newtenv$ is $\tenv$.
\end{itemize}
\subsubsection{Formally}
\begin{mathpar}
\inferrule{
  \annotateblock{\tenv, \vsone} \typearrow \vstwo \OrTypeError\\\\
  \annotatelooplimit(\tenv, \vlimitone) \typearrow \vlimittwo \OrTypeError\\\\
  \annotateexpr{\tenv, \veone} \typearrow (\vt, \vetwo) \OrTypeError\\\\
  \checktypesat(\tenv, \vt, \TBool) \typearrow \True \OrTypeError
}{
  \annotatestmt(\tenv, \overname{\SRepeat(\vsone, \veone, \vlimitone)}{\vs}) \typearrow
  (\overname{\SRepeat(\vstwo, \vetwo, \vlimittwo)}{\news}, \overname{\tenv}{\newtenv})
}
\end{mathpar}
\CodeSubsection{\SRepeatBegin}{\SRepeatEnd}{../Typing.ml}
\lrmcomment{This is related to \identr{FTVN}.}

\subsection{Semantics}
\subsubsection{SemanticsRule.SRepeat \label{sec:SemanticsRule.SRepeat}}
\subsubsection{Example}
The specification:
\VerbatimInput{../tests/ASLSemanticsReference.t/SemanticsRule.SRepeat.asl}
prints
\begin{Verbatim}
  0
  1
  2
  3
\end{Verbatim}

\subsubsection{Prose}
Evaluation of the statement $\vs$ in an environment $\env$ is
either \\ $\Returning((\vvs, \newg), \newenv)$ or an output configuration $D$ and all of the following apply:
\begin{itemize}
  \item $\vs$ is a \texttt{repeat} statement, $\SRepeat(\ve, \vbody, \Ignore)$;
  \item evaluating $\vbody$ in $\env$ as per \chapref{Bloccks}
        yields $\Continuing(\vgone, \envone)$\ProseTerminateAs{\ReturningConfig,\ThrowingConfig,\DynErrorConfig};
  \item evaluating the loop as per \secref{SemanticsRule.Loop} in an environment $\envone$,
        with the arguments $\False$ (which conveys that this is a \texttt{repeat} statement), $\ve$, and $\vbody$
        results in $C$;
  \item $\vgtwo$ is the ordered composition of $\vgone$ and $\vgtwo$ with the $\aslpo$ edge;
  \item the output configuration $D$ is the output configuration $C$ with its execution graph
        substituted with $\vgtwo$.
\end{itemize}
\subsubsection{Formally}
\begin{mathpar}
\inferrule{
  \evalblock{\env, \vbody} \evalarrow \Continuing(\vgone, \envone) \terminateas \ReturningConfig,\ThrowingConfig,\DynErrorConfig\\\\
  \evalloop{\envone, \False, \ve, \vbody} \evalarrow C\\
  \vgtwo \eqdef \ordered{\vgone}{\aslpo}{\graphof{C}}\\
  D \eqdef \withgraph{C}{\vgtwo}
}{
  \evalstmt{\env, \overname{\SRepeat(\ve, \vbody, \Ignore)}{\vs}} \evalarrow D
}
\end{mathpar}
\CodeSubsection{\EvalSRepeatBegin}{\EvalSRepeatEnd}{../Interpreter.ml}

\section{For Statements\label{sec:ForStatements}}
\subsection{Syntax}
\begin{flalign*}
\Nstmt \derivesinline\ & \Tfor \parsesep \Tidentifier \parsesep \Teq \parsesep \Nexpr \parsesep \Ndirection \parsesep
                    \Nexpr \parsesep \Tdo \parsesep \Nstmtlist \parsesep \Tend \parsesep \Tsemicolon &\\
\Ndirection \derivesinline\ & \Tto \;|\; \Tdownto &
\end{flalign*}

\subsection{Abstract Syntax}
\begin{flalign*}
\fordirection \derives\ & \UP \;|\; \DOWN &\\
\stmt \derives\ & \SFor\left\{
      \begin{array}{rcl}
      \Forindexname  &:& \identifier,\\
      \Forstarte     &:& \expr,\\
      \Fordir        &:& \fordirection,\\
      \Forende       &:& \expr,\\
      \Forbody       &:& \stmt,\\
      \Forlimit      &:& \expr?
      \end{array}
    \right\} &
\end{flalign*}

\subsubsection{ASTRule.SFor}
\begin{mathpar}
\inferrule{
  \buildexpr(\vstarte) \astarrow \astversion{\vstarte}\\
  \buildexpr(\vende) \astarrow \astversion{\vende}\\
}{
  {
    \begin{array}{r}
      \buildstmt\left(\overname{\Nstmt\left(
        \begin{array}{l}
        \Tfor, \Tidentifier(\vindexname), \Teq, \namednode{\vstarte}{\Nexpr}, \Ndirection, \\
        \wrappedline\ \namednode{\vende}{\Nexpr}, \Tdo, \punnode{\Nstmtlist}, \Tend, \Tsemicolon
        \end{array}
        \right)}{\vparsednode}\right)
      \astarrow \\
        \overname{
        \SFor\left(\left\{
          \begin{array}{rcl}
            \Forindexname &:& \vindexname\\
            \Forstarte &:& \astversion{\vstarte}\\
            \Forende &:& \astversion{\vende}\\
            \Forbody &:& \astof{\vstmtlist}\\
            \Forlimit &:& \None\\
          \end{array}
            \right\}\right)
    }{\vastnode}
    \end{array}
  }
}
\end{mathpar}

\subsubsection{ASTRule.Direction \label{sec:ASTRule.Direction}}
\hypertarget{build-direction}{}
The function
\[
\builddirection(\overname{\parsenode{\Ndirection}}{\vparsednode}) \;\aslto\; \overname{\fordirection}{\vastnode}
\]
transforms a parse node $\vparsednode$ into an AST node $\vastnode$.

\begin{mathpar}
\inferrule[to]{}{
  \builddirection(\overname{\Ndirection(\Tto)}{\vparsednode}) \astarrow \overname{\UP}{\vastnode}
}
\end{mathpar}

\begin{mathpar}
\inferrule[downto]{}{
  \builddirection(\overname{\Ndirection(\Tdownto)}{\vparsednode}) \astarrow \overname{\DOWN}{\vastnode}
}
\end{mathpar}

\subsection{Typing}
\subsubsection{TypingRule.SFor \label{sec:TypingRule.SFor}}
\subsubsection{Prose}
All of the following apply:
\begin{itemize}
  \item $\vs$ is a \texttt{for} statement with index $\vindexname$,
        start expression $\vstarte$,
        direction $\dir$,
        end expression $\vende$,
        body statement (block) $\vbody$,
        and optional limit expression $\vlimit$,
        that is, $\SFor\left\{\begin{array}{rcl}
          \Forindexname &:& \vindexname\\
          \Forstarte &:& \vstarte\\
          \fordirection &:& \vdir\\
          \Forende &:& \vende\\
          \Forbody &:& \vbody\\
          \Forlimit &:& \vlimit
        \end{array}\right\}$;
  \item annotating the right-hand-side expression $\vstarte$ in $\tenv$ yields \\
        $(\vstartt, \vstartep)$\ProseOrTypeError;
  \item annotating the right-hand-side expression $\vende$ in $\tenv$ yields $(\vendt, \vendep)$\ProseOrTypeError;
  \item annotating the optional loop limit expression $\vlimit$ via $\annotatelooplimit$ in $\tenv$
        yields $\vlimitp$\ProseOrTypeError;
  \item obtaining the \underlyingtype\ of $\vstartt$ in $\tenv$ yields $\vstartstruct$\ProseOrTypeError;
  \item obtaining the \underlyingtype\ of $\vendt$ in $\tenv$ yields $\vendstruct$\ProseOrTypeError;
  \item applying $\getforconstraints$ to $\vstartstruct$, $\vendstruct$,
        $\vstartep$, $\vendep$, and $\dir$ in $\tenv$,
        to obtain the constraints on the loop index $\vindexname$,
        yields $\cs$\ProseOrTypeError;
  \item $\tty$ is the integer type with constraints $\cs$;
  \item checking that $\vindexname$ is not already declared in $\tenv$ yields $\True$\ProseOrTypeError;
  \item adding $\vindexname$ as a local immutable variable with type $\tty$ to $\tenv$ yields $\tenvp$;
  \item annotating $\vbody$ as a block statement in $\tenvp$ yields $\vbodyp$\ProseOrTypeError;
  \item $\news$ is the \texttt{for} statement with index $\vindexname$,
        start expression $\vstartep$, direction $\dir$,
        end expression $\vendep$,
        body statement (block) $\vbodyp$, and
        optional limit expression $\vlimit$;
  \item $\newtenv$ is $\tenv$ (notice that this means $\vindexname$ is only declared for annotating $\vbodyp$ but then goes
        out of scope).
\end{itemize}
\subsubsection{Formally}
\begin{mathpar}
\inferrule{
  \annotateexpr{\tenv, \vstarte} \typearrow (\vstartt, \vstartep) \OrTypeError\\\\
  \annotateexpr{\tenv, \vende} \typearrow (\vendt, \vendep) \OrTypeError\\\\
  \annotatelooplimit(\tenv, \vlimit) \typearrow \vlimitp \OrTypeError\\\\
  \makeanonymous(\tenv, \vstartt) \typearrow \vstartstruct \OrTypeError\\\\
  \makeanonymous(\tenv, \vendt) \typearrow \vendstruct \OrTypeError\\\\
  {
    \begin{array}{r}
  \getforconstraints(\tenv, \vstartstruct, \vendstruct, \vstartep, \vendep, \dir) \typearrow \\
    \cs \OrTypeError
    \end{array}
  }\\\\
  \tty \eqdef \TInt(\cs)\\
  \checkvarnotinenv{\tenv, \vindexname} \typearrow \True \OrTypeError\\\\
  \addlocal(\tenv, \tty, \vindexname, \LDKLet) \typearrow \tenvp\\
  \annotateblock{\tenvp, \vbody} \typearrow \vbodyp \OrTypeError
}{
  {
    \begin{array}{r}
  \annotatestmt\left(\tenv, \overname{\SFor\left\{\begin{array}{rcl}
    \Forindexname &:& \vindexname\\
    \Forstarte &:& \vstarte\\
    \fordirection &:& \vdir\\
    \Forende &:& \vende\\
    \Forbody &:& \vbody\\
    \Forlimit &:& \vlimit
  \end{array}\right\}}{\vs}\right) \typearrow \\
  \left(\overname{\SFor\left\{\begin{array}{rcl}
    \Forindexname &:& \vindexname\\
    \Forstarte &:& \vstartep\\
    \fordirection &:& \vdir\\
    \Forende &:& \vendep\\
    \Forbody &:& \vbodyp\\
    \Forlimit &:& \vlimitp
  \end{array}\right\}}{\news}, \overname{\tenv}{\newtenv}\right)
\end{array}
  }
}
\end{mathpar}
\CodeSubsection{\SForBegin}{\SForEnd}{../Typing.ml}
\lrmcomment{This is related to \identr{SSBD}, \identr{ZSND}, \identr{VTJW}.}

\subsubsection{TypingRule.SForConstraints\label{sec:TypingRule.SForConstraints}}
\hypertarget{def-getforconstraints}{}
The function
\[
  \getforconstraints(
    \overname{\staticenvs}{\tenv} \aslsep
    \overname{\ty}{\structone} \aslsep
    \overname{\ty}{\structtwo} \aslsep
    \overname{\expr}{\veonep} \aslsep
    \overname{\expr}{\vetwop} \aslsep
    \overname{\dir}{\dir}
  ) \aslto
  \overname{\constraintkind}{\vis} \cup\ \overname{\TTypeError}{\TypeErrorConfig}
\]
infers the integer constraints for a \texttt{for} loop index variable from the following:
\begin{itemize}
  \item the \wellconstrainedversion\ of the type of the start expression --- $\structone$
  \item the \wellconstrainedversion\ of the type of the end expression --- $\structtwo$
  \item the annotated start expression --- $\veonep$
  \item the annotated end expression --- $\vetwop$
  \item the loop direction --- $\dir$
\end{itemize}
The result is $\vis$.
\ProseOtherwiseTypeError

\subsubsection{Prose}
One of the following applies:
\begin{itemize}
  \item All of the following apply (\textsc{not\_integers}):
  \begin{itemize}
    \item at least one of $\structone$ and $\structtwo$ is not an integer type;
    \item the result is a type error indicating that the start expression and end expression of \texttt{for} loops
          must have the \structure\ of integer types.
  \end{itemize}

  \item All of the following apply (\textsc{unconstrained}):
  \begin{itemize}
    \item both of $\structone$ and $\structtwo$ are integer types;
    \item at least one of $\structone$ and $\structtwo$ is the unconstrained integer type;
    \item define $\vis$ as $\unconstrained$.
  \end{itemize}

  \item All of the following apply (\textsc{well\_constrained}):
  \begin{itemize}
    \item both of $\structone$ and $\structtwo$ are integer types;
    \item neither $\structone$ nor $\structtwo$ is the unconstrained integer type;
    \item symbolically simplifying $\veonep$ in $\tenv$ yields $\eonen$\ProseOrTypeError;
    \item symbolically simplifying $\vetwop$ in $\tenv$ yields $\etwon$\ProseOrTypeError;
    \item define $\icsup$ as the single range constraint with expressions $\eonen$ and $\etwon$;
    \item define $\icsdown$ as the single range constraint with expressions $\etwon$ and $\eonen$;
    \item define $\vis$ as $\icsup$ if $\dir$ is $\UP$ and $\icsdown$ otherwise.
  \end{itemize}
\end{itemize}

\subsubsection{Formally}
\begin{mathpar}
\inferrule[not\_integers]{
  \astlabel(\structone) \neq \TInt \lor \astlabel(\structtwo) \neq \TInt
}{
  \getforconstraints(\tenv, \structone, \structtwo, \veonep, \vetwop, \dir) \typearrow \TypeErrorVal{\RequireIntegerForLoopBounds}
}
\end{mathpar}

\begin{mathpar}
\inferrule[unconstrained]{
  \astlabel(\structone) = \TInt \land \astlabel(\structtwo) = \TInt\\
  \structone = \unconstrainedinteger \lor \structtwo = \unconstrainedinteger\\
}{
  \getforconstraints(\tenv, \structone, \structtwo, \veonep, \vetwop, \dir) \typearrow \overname{\unconstrained}{\vis}
}
\end{mathpar}

\begin{mathpar}
\inferrule[well\_constrained]{
  \astlabel(\structone) = \TInt \land \astlabel(\structtwo) = \TInt\\
  \structone \neq \unconstrainedinteger \land \structtwo \neq \unconstrainedinteger\\
  \normalize(\tenv, \veonep) \typearrow \eonen \OrTypeError\\\\
  \normalize(\tenv, \vetwop) \typearrow \etwon \OrTypeError\\\\
  \icsup \eqdef \wellconstrained([\ConstraintRange(\eonen, \etwon)])\\
  \icsdown \eqdef \wellconstrained([\ConstraintRange(\etwon, \eonen)])\\
  \vis \eqdef \choice{\dir=\UP}{\icsup}{\icsdown}
}{
  \getforconstraints(\tenv, \structone, \structtwo, \veonep, \vetwop, \dir) \typearrow \vis
}
\end{mathpar}

\subsection{Semantics}
\subsubsection{SemanticsRule.SFor\label{sec:SemanticsRule.SFor}}
Evaluating a \texttt{for} statement involves introducing an index variable to the
environment. The type system ensures, via TypingRule.SFor, that the index variable
is not already declared in the scope of the subprogram containing the \texttt{for}
statement.

\subsubsection{Example}
The specification:
\VerbatimInput{../tests/ASLSemanticsReference.t/SemanticsRule.SFor.asl}
prints
\begin{Verbatim}
  0
  1
  2
  3
\end{Verbatim}

\subsubsection{Prose}
All of the following apply:
\begin{itemize}
  \item $\vs$ is a \texttt{for} statement, $\SFor\left\{\begin{array}{rcl}
    \Forindexname &:& \vindexname\\
    \Forstarte &:& \vstarte\\
    \fordirection &:& \vdir\\
    \Forende &:& \vende\\
    \Forbody &:& \vbody\\
    \Forlimit &:& \Ignore
  \end{array}\right\}$;
  \item evaluating the side-effect-free expression $\veone$ in $\env$ is either
  $\Normal(\vvone, \vgone)$\ProseOrError;
  \item evaluating the side-effect-free expression $\vetwo$ in $\env$ is either
  $\Normal(\vvtwo, \vgtwo)$\ProseOrError;
  \item declaring the local identifier $\vindexname$ in $\env$ with value $\vvone$ is $(\vgthree, \envone)$;
  \item evaluating the \texttt{for} loop with arguments $(\vindexname, \veone, \dir, \vetwo, \vs)$ in $\envone$,
  as per \nameref{sec:SemanticsRule.SFor} is $\Normal(\vgfour, \envtwo)$\ProseOrAbnormal;
  \item removing the local $\vindexname$ from $\envtwo$ is $\envthree$;
  \item $\newg$ is formed as follows: taking the parallel composition of $\vgone$ and $\vgtwo$,
  then taking the ordered composition of the result with the $\asldata$ edge,
  and finally taking the ordered composition of the result with the $\aslpo$ edges;
  \item $\newenv$ is $\envthree$.
  \item the result of the entire evaluation is $\Continuing(\newg, \newenv)$.
\end{itemize}
\subsubsection{Formally}
Recall that the expressions for the \texttt{for} loop range are side-effect-free,
which is why they are evaluated via the rule for evaluating side-effect-free expressions.
\begin{mathpar}
\inferrule{
  \evalexprsef{\env, \vstarte} \evalarrow \Normal(\vstartv, \vgone) \terminateas \DynErrorConfig\\
  \evalexprsef{\env, \vende} \evalarrow \Normal(\vendv, \vgtwo) \terminateas \DynErrorConfig\\
  \declarelocalidentifier(\env, \vindexname, \vendv) \evalarrow (\vgthree,\envone)\\
  \evalfor{\envone, \vindexname, \vstartv, \dir, \vendv, \vbody} \evalarrow \Normal(\vgfour, \envtwo) \OrAbnormal\\
  \removelocal(\envtwo, \vindexname) \evalarrow \envthree\\
  \newg \eqdef \ordered{(\vgone \parallelcomp \vgtwo)}{\asldata}{ \ordered{\vgthree}{\aslpo}{\vgfour}   }\\
  \newenv \eqdef \envthree
}{
  {
  \begin{array}{r}
  \evalstmt{\env,
  \overname{
  \SFor\left\{\begin{array}{rcl}
    \Forindexname &:& \vindexname\\
    \Forstarte &:& \vstarte\\
    \fordirection &:& \vdir\\
    \Forende &:& \vende\\
    \Forbody &:& \vbody\\
    \Forlimit &:& \Ignore\\
  \end{array}\right\}}{\vs}} \evalarrow \\ \Continuing(\newg, \newenv)
  \end{array}
  }
}
\end{mathpar}
\CodeSubsection{\EvalSForBegin}{\EvalSForEnd}{../Interpreter.ml}

\subsubsection{SemanticsRule.EvalFor\label{sec:SemanticsRule.EvalFor}}
The relation
\hypertarget{def-evalfor}{}
\[
  \evalfor{\overname{\envs}{\env} \aslsep \overname{\Identifiers}{\vindexname} \aslsep \overname{\tint}{\vstart}
  \aslsep \overname{\{\UP, \DOWN\}}{\dir} \aslsep \overname{\tint}{\vend} \aslsep \overname{\stmt}{\vbody}}
  \;\aslrel\;
  \left(
    \begin{array}{cl}
    \overname{\TReturning}{\ReturningConfig} & \cup\\
    \overname{\TContinuing}{\ContinuingConfig} & \cup\\
    \overname{\TThrowing}{\ThrowingConfig} & \cup \\
    \overname{\TDynError}{\DynErrorConfig} &
    \end{array}
    \right)
\]
evaluates the \texttt{for} loop with the index variable $\vindexname$ starting from the value
$\vstart$ going in the direction given by $\dir$ until the value given by $\vend$,
executing $\vbody$ on each iteration.
%
The evaluation utilizes two helper relations: $\evalforstep$ and $\evalforloop$.

The helper relation
\[
  \evalforstep(
    \overname{\envs}{\env},
    \overname{\Identifiers}{\vindexname},
    \overname{\tint}{\vstart},
    \overname{\{\UP,\DOWN\}}{\dir})
    \;\aslrel\;
    ((\overname{\tint}{\vstep} \times \overname{\envs}{\newenv}) \times \overname{\XGraphs}{\newg})
\]
either increments or decrements the index variable,
returning the new value of the index variable, the modified environment,
and the resulting execution graph.

The helper relation
\[
  \evalforloop(\overname{
    \envs}{\env},
    \overname{\Identifiers}{\vindexname},
    \overname{\tint}{\vstart},
    \overname{\{\UP,\DOWN\}}{\dir},
    \overname{\tint}{\vend},
    \overname{\stmt}{\vbody}) \;\aslrel\;
    \left(
    \begin{array}{cl}
      \overname{\TContinuing}{\Continuing(\newg, \newenv)} & \cup\\
      \overname{\TReturning}{\ReturningConfig} & \cup\\
    \overname{\TThrowing}{\ThrowingConfig} & \cup \\
    \overname{\TDynError}{\DynErrorConfig} &
    \end{array}
    \right)
\]
executes one iteration of the loop body and then uses $\texttt{eval\_for}$ to execute the remaining
iterations.

\subsection{Prose}
\subsubsection{Stepping the Index Variable}
All of the following apply:
\begin{itemize}
  \item $\opfordir$ is either $\PLUS$ when $\dir$ is $\UP$ or $\MINUS$ when $\dir$ is $\DOWN$;
  \item reading $\vstart$ into the identifier $\vindexname$ gives $\vgone$;
  \item applying the binary operator $\opfordir$ to $\vstart$ and the native integer for $1$ is $\vstep$;
  \item the execution graph for writing $\vstep$ into the identifier $\vindexname$ gives $\vgtwo$;
  \item updating the local component of the dynamic environment of $\env$ by binding \\ $\vindexname$ to $\vstep$
  gives $\newenv$;
  \item $\newg$ is the ordered composition of $\vgone$ and $\vgtwo$ with the $\asldata$ edge.
\end{itemize}

\subsubsection{Running the Loop Body}
All of the following apply:
\begin{itemize}
  \item evaluating $\vbody$ as a block statement (see \chapref{eval_block}) in $\env$
  is \\ $\Continuing(\vgone, \envone)$\ProseTerminateAs{\ReturningConfig, \ThrowingConfig, \DynErrorConfig};
  \item stepping the index $\vindexname$ with $\vstart$ and the direction $\dir$ in $\envone$,
  that is, $\evalforstep(\envone, \vindexname, \vstart, \dir)$ gives $((\vstep, \envtwo), \vgtwo)$;
  \item evaluating the \texttt{for} loop with $(\vindexname, \vstep, \dir, \vend, \vbody)$
  in $\envtwo$ results in a continuing configuration $\Continuing(\vgthree, \newenv)$\ProseTerminateAs{\ReturningConfig, \ThrowingConfig, \DynErrorConfig};
  \item $\newg$ is the ordered composition of $\vgone$, $\vgtwo$, and $\vgthree$ with the $\aslpo$
  edge.
\end{itemize}

\subsubsection{Overall Evaluation}
\subsubsection{Example}
The specification:
\VerbatimInput{../tests/ASLSemanticsReference.t/SemanticsRule.For.asl}
does not result in any assertion error, and the specification terminates with exit-code $0$.

Evaluating $(\vindexname, \vstart, \dir, \vend, \vbody)$ in $\env$ is either
a continuing configuration $\Continuing(\newg, \newenv)$ or a returning configuration
(in case the body of the loop results in an early return)
or an abnormal configuration,
and All of the following apply:
\begin{itemize}
  \item $\compfordir$ is either $\LT$ when $\dir$ is $\UP$ or $\GT$ when $\dir$ is $\DOWN$;
  \item reading $\vstart$ into the identifier $\vindexname$ gives $\vgone$;
  \item One of the following applies:
    \begin{itemize}
    \item All of the following apply (\textsc{return}):
    \begin{itemize}
      \item using $\compfordir$ to compare $\vend$ to $\vstart$ gives the native Boolean for $\True$;
      \item $\newg$ is $\vgone$;
      \item $\newenv$ is $\env$;
      \item the result of the entire evaluation is $\Continuing(\newg, \newenv)$.
    \end{itemize}
    \item All of the following apply (\textsc{continue}):
    \begin{itemize}
      \item using $\compfordir$ to compare $\vend$ to $\vstart$ gives the native Boolean for $\False$;
      \item evaluating the loop body via $\evalforloop$ with \\ $(\vindexname, \vstart, \dir, \vend, \vbody)$
      in $\env$ is \\ $\Continuing(\vgtwo, \newenv)$\ProseTerminateAs{\ReturningConfig, \ThrowingConfig, \DynErrorConfig};
      \item $\newg$ is the ordered composition of $\vgone$ and $\vgtwo$ with the $\aslctrl$ label.
    \end{itemize}
  \end{itemize}
\end{itemize}

\subsubsection{Formally}
Advancing the loop counter one step towards the end of its range is achieved via the following rule:
\begin{mathpar}
\inferrule{
  \opfordir \eqdef \choice{\dir = \UP}{\PLUS}{\MINUS}\\
  \readidentifier(\vindexname, \vstart) \evalarrow \vgone\\
  \binoprel(\opfordir, \vstart, \nvint(1)) \evalarrow \vstep\\
  \writeidentifier(\vindex, \vstep) \evalarrow \vgtwo\\
  \env \eqname (\tenv, \denv)\\
  \newenv \eqdef (\tenv, (G^\denv, L^\denv[\vindexname\mapsto\vstep]))\\
  \newg \eqdef \ordered{\vgone}{\asldata}{\vgtwo}
}{
  \evalforstep(\env, \vindexname, \vstart, \dir) \evalarrow ((\vstep, \newenv), \newg)
}
\end{mathpar}

Running the loop body is achieved via the following rule:
\begin{mathpar}
\inferrule{
  \evalblock{\env, \vbody} \evalarrow \Continuing(\vgone, \envone) \terminateas \ReturningConfig, \ThrowingConfig, \DynErrorConfig\\
  \evalforstep(\envone, \vindexname, \vstart, \dir) \evalarrow ((\vstep, \envtwo), \vgtwo)\\
  \evalfor{\envtwo, \vindexname, \vstep, \dir, \vend, \vbody} \evalarrow \Continuing(\vgthree, \newenv) \terminateas \ReturningConfig, \ThrowingConfig, \DynErrorConfig\\
  \newg \eqdef \ordered{\ordered{\vgone}{\aslpo}{\vgtwo}}{\aslpo}{\vgthree}
}{
  \evalforloop(\env, \vindexname, \vstart, \dir, \vend, \vbody) \evalarrow \Continuing(\newg, \newenv)
}
\end{mathpar}

Finally, the rules for evaluating a \texttt{for} loop utilize both $\evalforstep$
and $\evalforloop$ (the latter in a mutually recursive manner):
\begin{mathpar}
\inferrule[return]{
  \compfordir \eqdef \choice{\dir = \UP}{\LT}{\GT}\\
  \readidentifier(\vindexname, \vstart) \evalarrow \vgone\\
  \binoprel(\compfordir, \vend, \vstart) \evalarrow \nvbool(\True)\\
  \newg \eqdef \vgone\\
  \newenv = \env
}
{
  \evalfor{\env, \vindexname, \vstart, \dir, \vend, \vbody} \evalarrow \Continuing(\newg, \newenv)
}
\end{mathpar}

\begin{mathpar}
\inferrule[continue]{
  \compfordir \eqdef \choice{\dir = \UP}{\LT}{\GT}\\
  \readidentifier(\vindexname, \vstart) \evalarrow \vgone\\
  \binoprel(\compfordir, \vend, \vstart) \evalarrow \nvint(\False)\\
  \evalforloop(\env, \vindexname, \vstart, \dir, \vend, \vbody) \evalarrow \\
  \Continuing(\vgtwo, \newenv) \terminateas \ReturningConfig, \ThrowingConfig, \DynErrorConfig\\\\
  \newg \eqdef \ordered{\vgone}{\aslctrl}{\vgtwo}
}{
  \evalfor{\env, \vindexname, \vstart, \dir, \vend, \vbody} \evalarrow \Continuing(\newg, \newenv)
}
\end{mathpar}
\CodeSubsection{\EvalForBegin}{\EvalForEnd}{../Interpreter.ml}

\section{Throw Statements\label{sec:ThrowStatements}}
\subsection{Syntax}
\begin{flalign*}
\Nstmt \derivesinline\ & \Tthrow \parsesep \Nexpr \parsesep \Tsemicolon &\\
|\ & \Tthrow \parsesep \Tsemicolon &
\end{flalign*}

\subsection{Abstract Syntax}
\begin{flalign*}
\stmt \derives\ & \SThrow(\expr?) &
\end{flalign*}

\subsubsection{ASTRule.SThrow}
\begin{mathpar}
\inferrule[throw\_some]{}{
  \buildstmt(\overname{\Nstmt(\Tthrow, \Nexpr, \Tsemicolon)}{\vparsednode})
  \astarrow
  \overname{\SThrow(\langle\astof{\vexpr}\rangle)}{\vastnode}
}
\end{mathpar}

\begin{mathpar}
\inferrule[throw\_none]{}{
  \buildstmt(\overname{\Nstmt(\Tthrow, \Tsemicolon)}{\vparsednode})
  \astarrow
  \overname{\SThrow(\None)}{\vastnode}
}
\end{mathpar}

\subsection{Typing}
\subsubsection{TypingRule.SThrow\label{sec:TypingRule.SThrow}}
\subsubsection{Prose}
One of the following applies:
\begin{itemize}
  \item All of the following apply (\textsc{none}):
  \begin{itemize}
    \item $\vs$ is a throw statement with no expression, that is, $\SThrow(\None)$;
    \item $\news$ is $\vs$;
    \item $\newtenv$ is $\tenv$.
  \end{itemize}

  \item All of the following apply (\textsc{some}):
  \begin{itemize}
    \item $\vs$ is a throw statement with expression $\ve$, that is, $\SThrow(\langle\ve\rangle)$;
    \item annotating the right-hand-side expression $\ve$ in $\tenv$ yields $(\vte, \vep)$\ProseOrTypeError;
    \item checking that $\vte$ has the structure of an exception type yields $\True$\ProseOrTypeError;
    \item $\news$ is a throw statement with expression $\vep$ and type $\vte$, that is, \\
          $\SThrow(\langle (\vep, \vte) \rangle)$;
    \item $\newtenv$ is $\tenv$.
  \end{itemize}
\end{itemize}
\subsubsection{Formally}
\begin{mathpar}
\inferrule[none]{}{
  \annotatestmt(\tenv, \overname{\SThrow(\None)}{\vs}) \typearrow (\overname{\SThrow(\None)}{\news}, \overname{\tenv}{\newtenv})
}
\end{mathpar}
\lrmcomment{Note that \identr{BRCJ} is done in~\cite[SemanticsRule.TopLevel]{ASLSemanticsReference}.}

\begin{mathpar}
\inferrule[some]{
  \annotateexpr{\tenv, \ve} \typearrow (\vte, \vep) \OrTypeError\\\\
  \checkstructurelabel(\tenv, \vte, \TException) \typearrow \True \OrTypeError
}{
  \annotatestmt(\tenv, \overname{\SThrow(\langle\ve\rangle)}{\vs}) \typearrow
  (\overname{\SThrow(\langle (\vep, \vte) \rangle)}{\news}, \overname{\tenv}{\newtenv})
}
\end{mathpar}
\lrmcomment{This is related to \identr{NXRC}.}
\CodeSubsection{\SThrowBegin}{\SThrowEnd}{../Typing.ml}

\subsection{Semantics}
subsubsection{SemanticsRule.SThrow\label{sec:SemanticsRule.SThrow}}
\subsubsection{Example (Throwing Without an Exception)}
The specification:
\VerbatimInput{../tests/ASLSemanticsReference.t/SemanticsRule.SThrowNone.asl}
throws a \texttt{MyException} exception.

\subsubsection{Example (Throwing a Typed Exception)}
The specification:
\VerbatimInput{../tests/ASLSemanticsReference.t/SemanticsRule.SThrowSomeTyped.asl}
terminates successfully. That is, no dynamic error occurs.

\subsubsection{Prose}
One of the following applies:
\begin{itemize}
  \item All of the following apply (\textsc{none}):
  \begin{itemize}
  \item $\vs$ is a \texttt{throw} statement that does not provide an expression, $\SThrow(\None)$;
  \item $\newenv$ is $\env$;
  \item $\vex$ is $\None$;
  \item $\newg$ is the empty graph;
  \item an exception is thrown with $\newenv$.
  \end{itemize}

  \item All of the following apply (\textsc{some}):
  \begin{itemize}
    \item $\vs$ is a \texttt{throw} statement that provides an expression and a type, \\
          $\SThrow(\langle(\ve, \vt)\rangle)$;
    \item evaluating $\ve$ in $\env$ is $\Normal((\vv, \vgone), \newenv)$\ProseOrAbnormal;
    \item $\name$ is a fresh identifier (which conceptually holds the exception value);
    \item $\vgtwo$ is a Write Effect to $\name$;
    \item $\newg$ is the ordered composition of $\vgone$ and $\vgtwo$ with the $\asldata$ edge;
    \item $\vex$ consists of the exception value $\vv$, the name of the variable holding it ---
          $\name$, and the type annotation for the exception --- $\vt$;
    \item the result of the entire evaluation is $\Throwing((\vex, \newg), \env)$.
  \end{itemize}
\end{itemize}
\subsubsection{Formally}
\begin{mathpar}
\inferrule[none]{}
{
  \evalstmt{\env, \SThrow(\None)} \evalarrow \Throwing((\None, \emptygraph), \env)
}
\end{mathpar}
\begin{mathpar}
\inferrule[some]{
  \evalexpr{\env, \ve} \evalarrow \Normal((\vv, \vgone), \newenv) \OrAbnormal\\
  \name\in\Identifiers \text{ is fresh}\\
  \vgtwo \eqdef \WriteEffect(\name)\\
  \newg \eqdef \ordered{\vgone}{\asldata}{\vgtwo}\\
  \vex \eqdef \langle(\valuereadfrom(\vv, \name),\vt)\rangle
}{
  \evalstmt{\env, \SThrow(\langle(\ve, \vt)\rangle)} \evalarrow
  \Throwing((\vex, \newg), \newenv)
}
\end{mathpar}
\CodeSubsection{\EvalSThrowBegin}{\EvalSThrowEnd}{../Interpreter.ml}

\section{Try Statements\label{sec:TryStatements}}
\subsection{Syntax}
\begin{flalign*}
\Nstmt \derivesinline\ & \Ttry \parsesep \Nstmtlist \parsesep \Tcatch \parsesep \nonemptylist{\Ncatcher} \parsesep \Notherwiseopt \parsesep \Tend \parsesep \Tsemicolon &
\end{flalign*}

\subsection{Abstract Syntax}
\begin{flalign*}
\stmt \derives\ & \STry(\stmt, \catcher^*, \overtext{\stmt?}{otherwise}) &
\end{flalign*}

\subsubsection{ASTRule.STry}
\begin{mathpar}
\inferrule{
  \buildlist[\Ncatcher] \astarrow \astversion{\vcatcherlist}
}{
  {
    \begin{array}{r}
  \buildstmt\left(\overname{\Nstmt\left(
    \begin{array}{r}
    \Ttry, \Nstmtlist, \Tcatch,  \\
    \wrappedline\ \namednode{\vcatcherlist}{\nonemptylist{\Ncatcher}}, \\
    \wrappedline\ \Notherwiseopt, \Tend, \Tsemicolon
    \end{array}
    \right)}{\vparsednode}\right)
  \astarrow \\
  \overname{\STry(\astof{\vstmtlist}, \astversion{\vcatcherlist}, \astof{\votherwiseopt})}{\vastnode}
\end{array}
}
}
\end{mathpar}

\subsection{Typing}
\subsubsection{TypingRule.STry \label{sec:TypingRule.STry}}
\subsubsection{Prose}
All of the following apply:
\begin{itemize}
  \item $\vs$ is a try statement with statement $\vsp$, list of catchers $\catchers$ and an \optional\ \texttt{otherwise} block;
  \item annotating the statement $\vsp$ as a block statement yields $\vspp$\ProseOrTypeError;
  \item annotating each catcher $\catchers[\vi]$, for each $\vi$ in $\listrange(\catchers)$ in $\tenv$ yields $\vc\_\vi$\ProseOrTypeError;
  \item $\catchersp$ is the list of annotated catchers $\vc\_\vi$ for each $\vi\in\listrange(\catchers)$;
  \item One of the following applies:
  \begin{itemize}
    \item All of the following apply (\textsc{no\_otherwise}):
    \begin{itemize}
      \item there is no \texttt{otherwise} statement;
      \item $\news$ is a try statement with statement $\vspp$, list catchers $\catchersp$ and no \texttt{otherwise} statement,
            that is \\
            $\STry(\vspp, \catchersp, \None)$;
    \end{itemize}

    \item All of the following apply (\textsc{otherwise}):
    \begin{itemize}
      \item there is an \texttt{otherwise} statement $\otherwise$;
      \item annotating the statement $\otherwise$ as a block statement in $\tenv$ yields $\otherwisep$\ProseOrTypeError;
      \item $\news$ is a try statement with statement $\vspp$, list catchers $\catchersp$ and \texttt{otherwise} statement
            $\otherwisep$, that is \\
            $\STry(\vspp, \catchersp, \langle\otherwisep\rangle)$;
    \end{itemize}
  \end{itemize}
  \item $\newtenv$ is $\tenv$.
\end{itemize}
\subsubsection{Formally}
\begin{mathpar}
\inferrule[no\_otherwise]{
  \annotateblock{\tenv, \vsp} \typearrow \vspp \OrTypeError\\\\
  \vi\in\listrange(\catchers): \annotatecatcher{\tenv, \catchers[\vi]} \typearrow \vc_\vi \OrTypeError\\\\
  \catchersp \eqdef [\vi\in\listrange(\catchers) : \vc_\vi]\\\\
  \commonprefixline\\\\
  \news \eqdef \STry(\vspp, \catchersp, \None)
}{
  \annotatestmt(\tenv, \overname{\STry(\vsp, \catchers, \None)}{\vs}) \typearrow (\news, \overname{\tenv}{\newtenv})
}
\and
\inferrule[otherwise]{
  \annotateblock{\tenv, \vsp} \typearrow \vspp \OrTypeError\\\\
  \vi\in\listrange(\catchers): \annotatecatcher{\tenv, \catchers[\vi]} \typearrow \vc_\vi \OrTypeError\\\\
  \catchersp \eqdef [\vi\in\listrange(\catchers) : \vc_\vi]\\\\
  \commonprefixline\\\\
  \annotateblock{\tenv, \otherwise} \typearrow \otherwisep \OrTypeError\\\\
  \news \eqdef \STry(\vspp, \catchersp, \otherwise')
}{
  \annotatestmt(\tenv, \overname{\STry(\vsp, \catchers, \langle\otherwise\rangle)}{\vs}) \typearrow (\news, \overname{\tenv}{\newtenv})
}
\end{mathpar}
\CodeSubsection{\STryBegin}{\STryEnd}{../Typing.ml}
\lrmcomment{This is related to \identr{WVXS}.}

\subsection{Semantics}
\subsubsection{SemanticsRule.STry \label{sec:SemanticsRule.STry}}
\subsubsection{Example}
The specification:
\VerbatimInput{../tests/ASLSemanticsReference.t/SemanticsRule.STry.asl}
does not result in any Assertion error, and the specification terminates with the exit code $0$.

\subsubsection{Prose}
All of the following apply:
\begin{itemize}
  \item $\vs$ is a \texttt{try} statement, $\STry(\vs, \catchers, \otherwiseopt)$;
  \item evaluating $\vsone$ in $\env$ as per \chapref{Blocks}
  is a non-abnormal (that is, either $\Normal$ or $\Continuing$) configuration $\sm$\ProseOrAbnormal;
  \item evaluating $(\catchers, \otherwiseopt, \sm)$ as per \chapref{CatchingExceptions}
  is $C$, which is the result of the entire evaluation.
\end{itemize}
\subsubsection{Formally}
\begin{mathpar}
\inferrule{
  \evalblock{\env, \vsone} \evalarrow \sm \OrAbnormal\\
  \evalcatchers{\env, \catchers, \otherwiseopt, \sm} \evalarrow C
}{
  \evalstmt{\env, \STry(\vsone, \catchers, \otherwiseopt)} \evalarrow C
}
\end{mathpar}
\CodeSubsection{\EvalSTryBegin}{\EvalSTryEnd}{../Interpreter.ml}

\section{Return Statements\label{sec:ReturnStatements}}
\subsection{Syntax}
\begin{flalign*}
\Nstmt \derivesinline\ & \Treturn \parsesep \option{\Nexpr} \parsesep \Tsemicolon &
\end{flalign*}

\subsection{Abstract Syntax}
\begin{flalign*}
  \stmt \derives\ & \SReturn(\expr?) &
\end{flalign*}

\subsubsection{ASTRule.SReturn}
\begin{mathpar}
\inferrule{
  \buildoption[\Nexpr](\vexpr) \astarrow \astversion{\vexpr}
}{
  \buildstmt(\overname{\Nstmt(\Treturn, \namednode{\vexpr}{\option{\Nexpr}}, \Tsemicolon)}{\vparsednode})
  \astarrow
  \overname{\SReturn(\astversion{\vexpr})}{\vastnode}
}
\end{mathpar}

\subsection{Typing}
\subsubsection{TypingRule.SReturn\label{sec:TypingRule.SReturn}}
\subsubsection{Prose}
One of the following applies:
\begin{itemize}
  \item All of the following apply (\textsc{error}):
  \begin{itemize}
    \item $\vs$ is a \texttt{return} statement with an optional expression $\veopt$, that is, \\
          $\SReturn(\veopt)$;
    \item the condition that $\veopt$ is $\None$ if and only if the enclosing subprogram does not have a return type
          (that is, $\returntype$ in the local static environment is $\None$) does not hold;
    \item the result is an error indicating the mismatch between the declared (existence of the) return type
          and the (existence of the) return expression.
  \end{itemize}

  \item All of the following apply (\textsc{none}):
  \begin{itemize}
    \item $\vs$ is a \texttt{return} statement with no expression, that is, $\SReturn(\None)$;
    \item the enclosing subprogram does not have a \texttt{return} type (it is either a setter
          or a procedure);
    \item $\news$ is a \texttt{return} statement with no expression, that is, $\SReturn(\None)$;
    \item $\newtenv$ is $\tenv$.
  \end{itemize}

  \item All of the following apply (\textsc{some}):
  \begin{itemize}
    \item $\vs$ is a \texttt{return} statement with an expression $\ve$, that is, $\SReturn(\langle \vep \rangle)$;
    \item the enclosing subprogram has a return type $\vt$;
    \item annotating the right-hand-side expression $\ve$ in $\tenv$ yields $(\tep,\vep)$\ProseOrTypeError;
    \item checking whether $\vtep$ \typesatisfies\ $\vt$ in $\tenv$ yields $\True$\ProseOrTypeError;
    \item $\news$ is a \texttt{return} statement with value $\vep$, that is, $\SReturn(\langle \vep \rangle)$;
    \item $\newtenv$ is $\tenv$.
  \end{itemize}
\end{itemize}
\subsubsection{Formally}
\begin{mathpar}
\inferrule[error]{
  L^\tenv.\returntype \neq \veopt
}{
  \annotatestmt(\tenv, \overname{\SReturn(\veopt)}{\vs}) \typearrow \TypeErrorVal{InvalidReturnStmt}
}
\end{mathpar}

\begin{mathpar}
\inferrule[none]{
  L^\tenv.\returntype = \None
}{
  \annotatestmt(\tenv, \overname{\SReturn(\None)}{\vs}) \typearrow (\overname{\SReturn(\None)}{\news}, \overname{\tenv}{\newtenv})
}
\end{mathpar}

\begin{mathpar}
\inferrule[some]{
  L^\tenv.\returntype = \langle \vt \rangle\\
  \annotateexpr{\tenv, \ve} \typearrow (\vtep, \vep) \OrTypeError\\\\
  \checktypesat(\tenv, \vtep, \vt) \typearrow \True \OrTypeError
}{
  \annotatestmt(\tenv, \overname{\SReturn(\langle \ve \rangle)}{\vs}) \typearrow
  (\overname{\SReturn(\langle \vep \rangle)}{\news}, \overname{\tenv}{\newtenv})
}
\end{mathpar}
\CodeSubsection{\SReturn}{\SReturnEnd}{../Typing.ml}
\lrmcomment{This is related to \identr{FTPK}.}

\subsection{Semantics}
\subsubsection{SemanticsRule.SReturn\label{sec:SemanticsRule.SReturn}}
\subsubsection{Example (No Return Value)}
The specification:
\VerbatimInput{../tests/ASLSemanticsReference.t/SemanticsRule.SReturnNone.asl}
exits the current procedure.

\subsubsection{Example (Returning a Single Value)}
In the specification:
\VerbatimInput{../tests/ASLSemanticsReference.t/SemanticsRule.SReturnOne.asl}
\texttt{return 3;} exits the current subprogram with value \texttt{3}.

\subsubsection{Example (Returning Multiple Values)}
In the specification:
\VerbatimInput{../tests/ASLSemanticsReference.t/SemanticsRule.SReturnSome.asl}
\texttt{return (3, 42);} exits the current subprogram with value \texttt{(3, 42)}.

\subsubsection{Prose}
One of the following applies:
\begin{itemize}
  \item All of the following apply (\textsc{none}):
  \begin{itemize}
    \item $\vs$ is a \texttt{return} statement, $\SReturn(\None)$;
    \item $\vvs$ is the empty list, $\emptylist$;
    \item $\newg$ is the empty graph;
    \item $\newenv$ is $\env$.
  \end{itemize}

  \item All of the following apply (\textsc{one}):
  \begin{itemize}
    \item $\vs$ is a \texttt{return} statement;
    \item $\vs$ is a \texttt{return} statement for a single expression, $\SReturn(\langle\ve\rangle)$;
    \item evaluating $\ve$ in $\env$ is $\Normal((\vv, \vgone), \newenv)$\ProseOrAbnormal;
    \item $\vvs$ is $[\vv]$;
    \item $\vgtwo$ is the result of adding a Write Effect for a fresh identifier and the value $\vv$;
    \item $\newg$ is the ordered composition of $\vgone$ and $\vgtwo$ with the $\asldata$ edge.
  \end{itemize}

  \item All of the following apply (\textsc{tuple}):
  \begin{itemize}
    \item $\vs$ is a \texttt{return} statement for a list of expressions, $\SReturn(\langle\ETuple(\es)\rangle)$;
    \item evaluating each expression in $\es$ separately as per \secref{SemanticsRule.EExprListM}
    is \\ $\Normal(\ms, \newenv)$\ProseOrAbnormal;
    \item writing the list of values in $\vms$ results in $(\vvs, \newg)$.
  \end{itemize}
\end{itemize}

\begin{mathpar}
\inferrule[none]{}
{
  \evalstmt{\env, \SReturn(\None)} \evalarrow \Returning((\emptylist, \emptygraph), \env)
}
\end{mathpar}

\begin{mathpar}
\inferrule[one]{
  \evalexpr{\env, \ve} \evalarrow \Normal((\vv, \vgone), \newenv) \OrAbnormal\\\\
  \wid \in \Identifiers\text{ is fresh}\\
  \writeidentifier(\wid, \vv) \evalarrow \vgtwo\\
  \newg \eqdef \ordered{\vgone}{\asldata}{\vgtwo}
}{
  \evalstmt{\env, \SReturn(\langle\ve\rangle)} \evalarrow \Returning(([\vv], \newg), \newenv)
}
\end{mathpar}

\begin{mathpar}
\inferrule[tuple]{
  \evalexprlistm(\env, \es) \evalarrow \Normal(\ms, \newenv) \OrAbnormal\\
  \writefolder(\ms) \evalarrow (\vvs, \newg)
}{
  \evalstmt{\env, \SReturn(\langle\ETuple(\es)\rangle)} \evalarrow \Returning((\vvs, \newg), \newenv)
}
\end{mathpar}
\CodeSubsubsection{\SReturnBegin}{\EvalSReturnEnd}{../Interpreter.ml}

\subsubsection{SemanticsRule.EExprListM \label{sec:SemanticsRule.EExprListM}}
The helper relation
\[
  \evalexprlistm(\overname{\envs}{\env} \aslsep \overname{\expr^*}{\vEs}) \;\aslrel\;
          \Normal(\overname{(\vals\times\XGraphs)^* }{\vms} \aslsep \overname{\envs}{\newenv}) \cup
          \overname{\TThrowing}{\ThrowingConfig} \cup \overname{\TDynError}{\DynErrorConfig}
\]
evaluates a list of expressions $\vEs$ in left-to-right in the initial environment $\env$
and returns the list of values associated with graphs $\vms$ and the new environment $\newenv$.
If the evaluation of any expression terminates abnormally then the abnormal configuration is returned.

\subsubsection{Prose}
One of the following applies:
\begin{itemize}
  \item All of the following apply (\textsc{empty}):
  \begin{itemize}
    \item $\vEs$ is an empty list;
    \item $\vms$ is then empty list.
  \end{itemize}

  \item All of the following apply (\textsc{non\_empty}):
  \begin{itemize}
    \item $\vEs$ is a list with \head\ $\ve$ and \tail\ $\vesone$;
    \item evaluating $\ve$ in $\env$ yields $\Normal(\vmone, \envone)$\ProseOrAbnormal;
    \item evaluating $\vesone$ in $\envone$ via $\evalexprlistm$ yields \\
          $\Normal(\vmsone, \newenv)$\ProseOrAbnormal;
    \item the result is the normal configuration with the list consisting of $\vmone$ as its \head\ and $\vmsone$
          as its \tail\ and $\newenv$.
  \end{itemize}
\end{itemize}

\subsubsection{Formally}
\begin{mathpar}
\inferrule[empty]{}
{
  \evalexprlistm(\env, \overname{\emptylist}{\vEs}) \evalarrow \Normal(\overname{\emptylist}{\vms}, \overname{\env}{\newenv})
}
\end{mathpar}

\subsubsection{Semantics}
\begin{mathpar}
\inferrule[non\_empty]{
  \vEs \eqname [\ve] \concat \vesone\\
  \evalexpr{\env, \ve} \evalarrow \Normal(\vmone, \envone) \OrAbnormal\\
  \evalexprlistm(\envone, \vesone) \evalarrow \Normal(\vmsone, \newenv) \OrAbnormal
}{
  \evalexprlistm(\env, \vEs) \evalarrow \Normal([\vmone]\concat\vmsone, \newenv)
}
\end{mathpar}

\subsubsection{SemanticsRule.WriteFolder\label{sec:SemanticsRule.WriteFolder}}
\hypertarget{def-writefolder}{}
The helper relation
\[
  \writefolder(\overname{(\vals\times\XGraphs)^*}{\vms}) \aslrel (\overname{\vals^*}{\vvs}, \overname{\XGraphs}{\newg}) \enspace,
\]
concatenates the input values in $\vms$ and generates an execution graph
by composing the graphs in $\vms$ with Write Effects for the respective values.

\begin{mathpar}
\inferrule[empty]{}{
  \writefolder(\emptylist) \evalarrow (\emptylist, \emptygraph)
}
\and
\inferrule[nonempty]{
  \vms \eqname [\vm] \concat \vmsone\\
  \vm \eqdef (\vv, \vg)\\
  \wid \in \Identifiers\text{ is fresh}\\
  \writeidentifier(\wid, \vv) \evalarrow \vgone\\
  \writefolder(\vmsone, \vgone) \evalarrow (\vvsone, \vgtwo)\\
  \vvs \eqdef [\vv] \concat \vvsone\\
  \newg \eqdef \ordered{\vgone}{\asldata}{\vgtwo}
}{
  \writefolder(\vms) \evalarrow (\vvs, \ordered{\vg}{\aslpo}{\newg})
}
\end{mathpar}

\section{Print Statements\label{sec:PrintStatements}}
\subsection{Syntax}
\begin{flalign*}
\Nstmt \derivesinline\ & \Tprint \parsesep \Plist{\Nexpr} \parsesep \Tsemicolon &
\end{flalign*}

\subsection{Abstract Syntax}
\begin{flalign*}
\stmt \derives\ & \SPrint(\overtext{\expr^*}{args}, \overtext{\Bool}{debug}) &
\end{flalign*}

\subsubsection{ASTRule.SPrint}
\begin{mathpar}
\inferrule{
  \buildplist[\Nexpr](\vargs) \astarrow \astversion{\vargs}
}{
  \buildstmt(\overname{\Nstmt(\Tprint, \namednode{\vargs}{\Plist{\Nexpr}}, \Tsemicolon)}{\vparsednode})
  \astarrow
  \overname{\SPrint(\astversion{\vargs})}{\vastnode}
}
\end{mathpar}

\subsection{Typing}
\subsection{Semantics}

\section{The Unreachable Statement\label{sec:UnreachableStatement}}
\subsection{Syntax}
\begin{flalign*}
\Nstmt \derivesinline\ & \Tunreachable \parsesep \Tlpar \parsesep \Trpar \parsesep \Tsemicolon &
\end{flalign*}

\subsection{Abstract Syntax}
\begin{flalign*}
\stmt \derives\ & \SUnreachable &
\end{flalign*}

\subsubsection{ASTRule.SUnreachable}
\begin{mathpar}
\inferrule{}{
  \buildstmt(\overname{\Nstmt(\Tunreachable, \Tlpar, \Trpar, \Tsemicolon)}{\vparsednode})
  \astarrow
  \overname{\SUnreachable}{\vastnode}
}
\end{mathpar}

\subsubsection{TypingRule.SUnreachable}
\subsubsection{TypingRule.SUnreachable\label{sec:TypingRule.SUnreachable}}
\subsubsection{Prose}
Annotating $\SUnreachable$ in the static environment $\tenv$ yields $(\SUnreachable, \tenv)$.

\subsubsection{Formally}
\begin{mathpar}
\inferrule{}{
  \annotatestmt(\tenv, \SUnreachable) \typearrow (\SUnreachable, \tenv)
}
\end{mathpar}

\subsubsection{SemanticsRule.SUnreachable\label{sec:SemanticsRule.SUnreachable}}
\subsubsection{Prose}
Evaluating $\SUnreachable$ in an environment $\env$ results in a dynamic error indicating this ($\UnreachableError$).
\subsubsection{Formally}
\begin{mathpar}
\inferrule{}{
  \evalstmt{\env, \SUnreachable} \evalarrow \DynamicErrorVal{\UnreachableError}
}
\end{mathpar}

\section{Pragma Statements\label{sec:PragmaStatements}}
\subsection{Syntax}
\begin{flalign*}
\Nstmt \derivesinline\ & \Tpragma \parsesep \Tidentifier \parsesep \Clist{\Nexpr} \parsesep \Tsemicolon &
\end{flalign*}

\subsection{Abstract Syntax}
\subsection{Typing}
\subsection{Semantics}
