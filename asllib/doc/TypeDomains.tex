%%%%%%%%%%%%%%%%%%%%%%%%%%%%%%%%%%%%%%%%%%%%%%%%%%%%%%%%%%%%%%%%%%%%%%%%%%%%%%%%%%%%
\section{Domain of Values for Types\label{sec:DomainOfValuesForTypes}}
%%%%%%%%%%%%%%%%%%%%%%%%%%%%%%%%%%%%%%%%%%%%%%%%%%%%%%%%%%%%%%%%%%%%%%%%%%%%%%%%%%%%
This section formalizes the concept of the set of values for a given type.
The formalism is given in the form of rules.
%
The section also defines the concept of checking whether the set of values
for one type is included in the set of values for another type.

\subsection{Dynamic Domain of a Type\label{sec:DynDomain}}
\hypertarget{def-dyndomain}{}

We now define the concept of a \emph{dynamic domain} of a type
and the \emph{static domain} of a type.
Intuitively, domains assign potentially infinite sets of \nativevalues\ to types.
Dynamic domains are used by the semantics to evaluate expressions of the form \texttt{UNKNOWN: t}
by choosing a single value from the dynamic domain of $\vt$.
Static domains are used to define subtype satisfaction in \secref{TypingRule.SubtypeSatisfaction}.

Formally, the partial function
\[
  \dynamicdomain : \overname{\envs}{\env} \times \overname{\ty}{\vt}
  \partialto \overname{\pow{\vals}}{\vd}
\]
assigns the set of values that a type $\vt$ can hold in a given environment $\env$.
%
We say that $\dynamicdomain(\env, \vt)$ is the \emph{dynamic domain} of $\vt$
in the environment $\env$.
%
The \emph{static domain} of a type is the set of values which storage elements of that type may hold
\underline{across all possible dynamic environments}.
%
The reason for this distinction is that the sets of values
of integer types, bitvector types, and array types can depend on the dynamic values of variables.

Types that do not refer to variables whose values are only known dynamically have
a static domain that is equal to any of their dynamic domains.
In those cases, we simply refer to their \emph{domain}.

Associating a set of values to a type is done by evaluating any expression appearing
in the type definitions. Evaluation is defined by the relation $\evalexprsef$.
which evaluates side-effect-free expressions and either returns
a configuration of the form $\Normal(\vv,\vg)$ or a dynamic error configuration $\DynErrorConfig$.
In the first case, $\vv$ is a \nativevalue\ and $\vg$
is an \emph{execution graph}. Execution graphs are related to the concurrent semantics
and can be ignored in the context of defining dynamic domains.
In the latter case (which can occur if, for example, an expression attempts to divide
\texttt{8} by \texttt{0}), a dynamic error configuration, for which we use the notation
$\DynErrorConfig$, is returned.
%
The dynamic domain is empty in cases where evaluating side-effect-free expressions
results in a dynamic error.
%
The dynamic domain is undefined if the type $\vt$ is not well-typed in $\tenv$.
That is, if $\annotatetype{\tenv, \vt} \typearrow \TypeErrorConfig$.

As part of the definition, we also associate dynamic domains to integer constraints
by overloading $\dynamicdomain$:
\[
  \dynamicdomain : \overname{\envs}{\env} \times \overname{\intconstraint}{\vc}
  \partialto \overname{\pow{\vals}}{\vd}
\]

\subsection{Prose}
For an environment $\env \in \envs$ and a type $\vt$, the domain is $\vd$ and one of the following applies:
\begin{itemize}
  \item All of the following apply (\textsc{t\_bool}):
  \begin{itemize}
    \item $\vt$ is the Boolean type, $\TBool$;
    \item $\vd$ is the set of native Boolean values, $\tbool$.
  \end{itemize}

  \item All of the following apply (\textsc{t\_string}):
  \begin{itemize}
    \item $\vt$ is the string type, $\TString$;
    \item $\vd$ is the set of all native string values, $\tstring$.
  \end{itemize}

  \item All of the following apply (\textsc{t\_real}):
  \begin{itemize}
    \item $\vt$ is the real type, $\TReal$;
    \item $\vd$ is the set of all native real values, $\treal$.
  \end{itemize}

  \item All of the following apply (\textsc{t\_enumeration}):
  \begin{itemize}
    \item $\vt$ is the enumeration type with labels $\id_{1..k}$, that is $\TEnum(\id_{1..k})$;
    \item $\vd$ is the set of all native integer values for $1..k$.\\
    \textbf{Why represent enumeration domains via integers:}
    Conceptually, enumeration labels carry two pieces of information --- the identifiers themselves
    and their position in the list of identifiers, which are used for accessing arrays.
    For the purpose of type-checking, we use the identifiers, but for the purpose of the semantics
    and the domain of values, only the positions are relevant.
  \end{itemize}

  \item All of the following apply (\textsc{t\_int\_unconstrained}):
  \begin{itemize}
    \item $\vt$ is the unconstrained integer type, $\unconstrainedinteger$;
    \item $\vd$ is the set of all native integer values, $\tint$.
  \end{itemize}

  \item All of the following apply (\textsc{t\_int\_well\_constrained}):
  \begin{itemize}
    \item $\vt$ is the well-constrained integer type $\TInt(\wellconstrained(\vc_{1..k}))$;
    \item $\vd$ is the union of the dynamic domains of each of the constraints $\vc_{1..k}$ in $\env$.
  \end{itemize}

  \item All of the following apply (\textsc{constraint\_exact\_okay}):
  \begin{itemize}
    \item $\vc$ is a constraint consisting of a single side-effect-free expression $\ve$, that is, $\ConstraintExact(\ve)$;
    \item evaluating $\ve$ in $\env$, results in a configuration with the native integer for $n$;
    \item $\vd$ is the set containing the single native integer value for $n$.
  \end{itemize}

  \item All of the following apply (\textsc{constraint\_exact\_dynamic\_error}):
  \begin{itemize}
    \item $\vc$ is a constraint consisting of a single side-effect-free expression $\ve$, that is, $\ConstraintExact(\ve)$;
    \item evaluating $\ve$ in $\env$, results in a dynamic error configuration;
    \item $\vd$ is the empty set.
  \end{itemize}

  \item All of the following apply (\textsc{constraint\_range\_okay}):
  \begin{itemize}
    \item $\vc$ is a range constraint consisting of a two side-effect-free expressions $\veone$ and $\vetwo$, that is, $\ConstraintRange(\veone, \vetwo)$;
    \item evaluating $\veone$ in $\env$, results in a configuration with the native integer for $a$;
    \item evaluating $\vetwo$ in $\env$, results in a configuration with the native integer for $b$;
    \item $\vd$ is the set containing all native integer values for integers greater or equal to $a$ and less than or equal to $b$.
  \end{itemize}

  \item All of the following apply (\textsc{constraint\_range\_dynamic\_error1}):
  \begin{itemize}
    \item $\vc$ is a range constraint consisting of a two side-effect-free expressions $\veone$ and $\vetwo$, that is, $\ConstraintRange(\veone, \vetwo)$;
    \item evaluating $\veone$ in $\env$, results in a dynamic error configuration;
    \item $\vd$ is the empty set.
  \end{itemize}

  \item All of the following apply (\textsc{constraint\_range\_dynamic\_error2}):
  \begin{itemize}
    \item $\vc$ is a range constraint consisting of a two side-effect-free expressions $\veone$ and $\vetwo$, that is, $\ConstraintRange(\veone, \vetwo)$;
    \item evaluating $\veone$ in $\env$, results in a configuration with the native integer for $a$;
    \item evaluating $\vetwo$ in $\env$, results in a dynamic error configuration;
    \item $\vd$ is the empty set.
  \end{itemize}

  \item All of the following apply (\textsc{t\_int\_parameterized}):
  \begin{itemize}
    \item $\vt$ is a \parameterizedintegertype\ for parameter $\id$, \\ $\TInt(\parameterized(\id))$;
    \item the \nativevalue\ associated with $\id$ in the local dynamic environment is the native integer value for $n$;
    \item $\vd$ is the set containing the single integer value for $n$.
  \end{itemize}

  \item All of the following apply (\textsc{t\_bits\_dynamic\_error}):
  \begin{itemize}
    \item $\vt$ is a bitvector type with size expression $\ve$, $\TBits(\ve, \Ignore)$;
    \item evaluating $\ve$ in $\env$, results in a dynamic error configuration;
    \item $\vd$ is the empty set.
  \end{itemize}

  \item All of the following apply (\textsc{t\_bits\_negative\_width\_error}):
  \begin{itemize}
    \item $\vt$ is a bitvector type with size expression $\ve$, $\TBits(\ve, \Ignore)$;
    \item evaluating $\ve$ in $\env$, results in a configuration with the native integer for $k$;
    \item $k$ is negative;
    \item $\vd$ is the empty set.
  \end{itemize}

  \item All of the following apply (\textsc{t\_bits\_empty}):
  \begin{itemize}
    \item $\vt$ is a bitvector type with size expression $\ve$, $\TBits(\ve, \Ignore)$;
    \item evaluating $\ve$ in $\env$, results in a configuration with the native integer for $0$;
    \item $\vd$ is the set containing the single \nativevalue\ for an empty bitvector.
  \end{itemize}

  \item All of the following apply (\textsc{t\_bits\_non\_empty}):
  \begin{itemize}
    \item $\vt$ is a bitvector type with size expression $\ve$, $\TBits(\ve, \Ignore)$;
    \item evaluating $\ve$ in $\env$, results in a configuration with the native integer for $k$;
    \item $k$ is greater than $0$;
    \item $\vd$ is the set containing all \nativevalues\ for bitvectors of size exactly $k$.
  \end{itemize}

  \item All of the following apply (\textsc{t\_tuple}):
  \begin{itemize}
    \item $\vt$ is a tuple type over types $\vt_i$, for $i=1..k$, $\TTuple(\vt_{1..k})$;
    \item the domain of each element $\vt_i$ is $D_i$, for $i=1..k$;
    \item evaluating $\ve$ in $\env$, results in a configuration with the native integer for $k$;
    \item $\vd$ is the set containing all native vectors of $k$ values, where the value at position $i$
    is from $D_i$.
  \end{itemize}

  \item All of the following apply (\textsc{t\_array\_dynamic\_error}):
  \begin{itemize}
    \item $\vt$ is an array type with length expression $\ve$ and element type $\vt_i$, for $i=1..k$, $\TArray(\ve, \vtone)$;
    \item evaluating $\ve$ in $\env$, results in a dynamic error configuration;
    \item $\vd$ is the empty set.
  \end{itemize}

  \item All of the following apply (\textsc{t\_array\_negative\_length\_error}):
  \begin{itemize}
    \item $\vt$ is an array type with length expression $\ve$ and element type $\vt_i$, for $i=1..k$, $\TArray(\ve, \vtone)$;
    \item evaluating $\ve$ in $\env$, results in a configuration with the native integer for $k$;
    \item $k$ is negative;
    \item $\vd$ is the empty set.
  \end{itemize}

  \item All of the following apply (\textsc{t\_array\_okay}):
  \begin{itemize}
    \item $\vt$ is an array type with length expression $\ve$ and element type $\vt_i$, for $i=1..k$, $\TArray(\ve, \vtone)$;
    \item evaluating $\ve$ in $\env$, results in a configuration with the native integer for $k$;
    \item $k$ is greater than or equal to $0$;
    \item the domain of $\vtone$ is $D_\vtone$;
    \item $\vd$ is the set containing all native vectors of $k$ values taken from $D_\vtone$.
  \end{itemize}

  \item All of the following apply (\textsc{t\_structured}):
  \begin{itemize}
    \item $\vt$ is a \structuredtype\ with typed fields $(\id_i, \vt_i$, for $i=1..k$, that is $L([i=1..k: (\id_i,\vt_i))]$
    where $L\in\{\TRecord, \TException\}$;
    \item the domain of each type $\vt_i$ is $D_i$, for $i=1..k$;
    \item $\vd$ is the set containing all native records where $\id_i$ is mapped to a value taken from $D_i$.
  \end{itemize}

  \item All of the following apply (\textsc{t\_named}):
  \begin{itemize}
    \item $\vt$ is a named type with name $\id$, $\TNamed(\id)$;
    \item the type associated with $\id$ in $\tenv$ is $\tty$;
    \item $\vd$ is the domain of $\tty$ in $\env$.
  \end{itemize}
\end{itemize}

\subsubsection{Formally}

\begin{mathpar}
\inferrule[t\_bool]{}{ \dynamicdomain(\env, \overname{\TBool}{\vt}) = \overname{\tbool}{\vd} }
\and
\inferrule[t\_string]{}{ \dynamicdomain(\env, \overname{\TString}{\vt}) = \overname{\tstring}{\vd} }
\and
\inferrule[t\_real]{}{ \dynamicdomain(\env, \overname{\TReal}{\vt}) = \overname{\treal}{\vd} }
\and
\inferrule[t\_enumeration]{}{
  \dynamicdomain(\env, \overname{\TEnum(\id_{1..k})}{\vt}) = \overname{\{\nvint(1),\ldots,\nvint(k)\}}{\vd}
}
\end{mathpar}

\begin{mathpar}
  \inferrule[t\_int\_unconstrained]{}{
  \dynamicdomain(\env, \overname{\unconstrainedinteger}{\vt}) = \overname{\tint}{\vd}
}
\end{mathpar}

\begin{mathpar}
\inferrule[t\_int\_well\_constrained]{}{
  \dynamicdomain(\env, \overname{\TInt(\wellconstrained(\vc_{1..k}))}{\vt}) = \overname{\bigcup_{i=1}^k \dynamicdomain(\env, \vc_i)}{\vd}
}
\end{mathpar}

\begin{mathpar}
\inferrule[constraint\_exact\_okay]{
  \evalexprsef{\env, \ve} \evalarrow \Normal(\nvint(n), \Ignore)
}{
  \dynamicdomain(\env, \overname{\ConstraintExact(\ve)}{\vc}) = \overname{\{ \nvint(n) \}}{\vd}
}
\and
\inferrule[constraint\_exact\_dynamic\_error]{
  \evalexprsef{\env, \ve} \evalarrow \DynErrorConfig
}{
  \dynamicdomain(\env, \overname{\ConstraintExact(\ve)}{\vc}) = \overname{\emptyset}{\vd}
}
\end{mathpar}

\begin{mathpar}
\inferrule[constraint\_range\_okay]{
  \evalexprsef{\env, \veone} \evalarrow \Normal(\nvint(a), \Ignore)\\
  \evalexprsef{\env, \vetwo} \evalarrow \Normal(\nvint(b), \Ignore)
}{
  \dynamicdomain(\env, \overname{\ConstraintRange(\veone, \vetwo)}{\vc}) = \overname{\{ \nvint(n) \;|\;  a \leq n \land n \leq b\}}{\vd}
}
\and
\inferrule[constraint\_range\_dynamic\_error1]{
  \evalexprsef{\env, \veone} \evalarrow \DynErrorConfig
}{
  \dynamicdomain(\env, \overname{\ConstraintRange(\veone, \vetwo)}{\vc}) = \overname{\emptyset}{\vd}
}
\and
\inferrule[constraint\_range\_dynamic\_error2]{
  \evalexprsef{\env, \veone} \evalarrow \Normal(\Ignore, \Ignore)\\
  \evalexprsef{\env, \vetwo} \evalarrow \DynErrorConfig
}{
  \dynamicdomain(\env, \overname{\ConstraintRange(\veone, \vetwo)}{\vc}) = \overname{\emptyset}{\vd}
}
\end{mathpar}

The notation $L^\denv(\id)$ denotes the \nativevalue\ associated with the identifier $\id$
in the \emph{local dynamic environment} of $\denv$.
\begin{mathpar}
  \inferrule[t\_int\_parameterized]{
  L^\denv(\id) = \nvint(n)
}{
  \dynamicdomain(\env, \overname{\TInt(\parameterized(\id))}{\vt}) = \overname{\{ \nvint(n) \}}{\vd}
}
\end{mathpar}

\begin{mathpar}
\inferrule[t\_bits\_dynamic\_error]{
  \evalexprsef{\env, \ve} \evalarrow \DynErrorConfig
}{
  \dynamicdomain(\env, \overname{\TBits(\ve, \Ignore)}{\vt}) = \overname{\emptyset}{\vd}
}
\and
\inferrule[t\_bits\_negative\_width\_error]{
  \evalexprsef{\env, \ve} \evalarrow \Normal(\nvint(k), \Ignore)\\
  k < 0
}{
  \dynamicdomain(\env, \overname{\TBits(\ve, \Ignore)}{\vt}) = \overname{\emptyset}{\vd}
}
\and
\inferrule[t\_bits\_empty]{
  \evalexprsef{\env, \ve} \evalarrow \Normal(\nvint(0), \Ignore)
}{
  \dynamicdomain(\env, \overname{\TBits(\ve, \Ignore)}{\vt}) = \overname{\{ \nvbitvector(\emptylist) \}}{\vd}
}
\and
\inferrule[t\_bits\_non\_empty]{
  \evalexprsef{\env, \ve} \evalarrow \Normal(\nvint(k), \Ignore)\\
  k > 0
}{
  \dynamicdomain(\env, \overname{\TBits(\ve, \Ignore)}{\vt}) = \overname{\{ \nvbitvector(\vb_{1..k}) \;|\; \vb_1,\ldots,\vb_k \in \{0,1\} \}}{\vd}
}
\end{mathpar}

\begin{mathpar}
\inferrule[t\_tuple]{
  i=1..k: \dynamicdomain(\env, \vt_i) = D_i
}{
  \dynamicdomain(\env, \overname{\TTuple(\vt_{1..k})}{\vt}) =
  \overname{\{ \nvvector{\vv_{1..k}} \;|\; \vv_i \in D_i \}}{\vd}
}
\end{mathpar}

\begin{mathpar}
\inferrule[t\_array\_dynamic\_error]{
  \evalexprsef{\env, \ve} \evalarrow \DynErrorConfig
}{
  \dynamicdomain(\env, \overname{\TArray(\ve, \vtone)}{\vt}) = \overname{\emptyset}{\vd}
}
\and
\inferrule[t\_array\_negative\_length\_error]{
  \evalexprsef{\env, \ve} \evalarrow \Normal(\nvint(k), \Ignore)\\
  k < 0
}{
  \dynamicdomain(\env, \overname{\TArray(\ve, \vtone)}{\vt}) = \overname{\emptyset}{\vd}
}
\and
\inferrule[t\_array\_okay]{
  \evalexprsef{\env, \ve} \evalarrow \Normal(\nvint(k), \Ignore)\\
  k \geq 0\\
  \dynamicdomain(\env, \vtone) = D_\vtone
}{
  \dynamicdomain(\env, \overname{\TArray(\ve, \vtone)}{\vt}) =
  \overname{\{ \nvvector{\vv_{1..k}} \;|\; \vv_{1..k} \in D_{\vtone} \}}{\vd}
}
\end{mathpar}

\begin{mathpar}
\inferrule[structured]{
  L \in \{\TRecord, \TException\}\\
  i=1..k: \dynamicdomain(\env, \vt_i) = D_i
}{
  \dynamicdomain(\env, \overname{L([i=1..k: (\id_i,\vt_i))]}{\vt}) = \\
  \overname{\{ \nvrecord{\{i=1..k: \id_i\mapsto \vv_i\}} \;|\; \vv_i \in D_i \}}{\vd}
}
\end{mathpar}

\begin{mathpar}
\inferrule[t\_named]{
  G^\tenv.\declaredtypes(\id)=\tty
}{
  \dynamicdomain(\env, \overname{\TNamed(\id)}{\vt}) = \overname{\dynamicdomain(\env, \tty)}{\vd}
}
\end{mathpar}

\subsubsection{Examples}
The domain of \texttt{integer} is the infinite set of all integers.

The domain of \verb|integer {2,16}| is the set $\{\nvint(2), \nvint(16)\}$.

The domain of \verb|integer{1..3}| is the set $\{\nvint(1), \nvint(2), \nvint(3)\}$.

The domain of \verb|integer{10..1}| is the empty set as there are no integers that are
both greater than $10$ and smaller than $1$.

The domain of \texttt{bits(2)} is the set $\{\nvbitvector(00)$, $\nvbitvector(01),$
$\nvbitvector(10)$, $\nvbitvector(11)\}$.

The domain of \verb|enumeration {GREEN, ORANGE, RED}| is the set \\
$\{\nvint(1), \nvint(2), \nvint(3)\}$ and so is the domain
of \\
\verb|type TrafficLights of enumeration {GREEN, ORANGE, RED}|.

The domain of \texttt{bits({2,16})} is the set containing native bitvectors of all 2-bit and all 16-bit binary sequences.

The domain of \texttt{(integer, integer)} is the set containing all pairs of native integer values.

The domain of \verb|record {a: integer;  b: boolean}| contains all native records
that map \texttt{a} to a native integer value and \texttt{b} to a native Boolean value.

The dynamic domain of a subprogram parameter \texttt{N: integer} is the (singleton) set containing
the native integer value $c$,
which is assigned to \texttt{N} by a given dynamic environment. The static domain of that parameter
is the infinite set of all native integer values.

\lrmcomment{
This is related to \identd{BMGM}, \identr{PHRL}, \identr{PZNR},
\identr{RLQP}, \identr{LYDS}, \identr{SVDJ}, \identi{WLPJ}, \identr{FWMM},
\identi{WPWL}, \identi{CDVY}, \identi{KFCR}, \identi{BBQR}, \identr{ZWGH},
\identr{DKGQ}, \identr{DHZT}, \identi{HSWR}, \identd{YZBQ}.
}

\subsection{Subsumption Testing\label{sec:subsumptiontesting}}
Whether an assignment statement is well-typed depends on whether the dynamic domain of the
right hand side type is contained in the dynamic domain of the left hand side type,
for any given dynamic environment
(see \secref{TypingRule.SubtypeSatisfaction} where this is checked).

\begin{definition}[Subsumption]
For any given types $\vt$ and $\vs$ and static environment $\tenv$,
we say that $\vt$ \emph{subsumes} $\vs$ in $\tenv$,
if the following condition holds:
\hypertarget{def-subsumes}{}
\begin{equation}
  \subsumes(\tenv, \vt, \vs) \triangleq \forall \denv\in\dynamicenvs.\ \dynamicdomain((\tenv, \denv), \vt) \supseteq \dynamicdomain((\tenv, \denv), \vs) \enspace.
\end{equation}
\end{definition}

For example, consider the assignment
\begin{center}
\verb|var x : integer{1,2,3} = UNKNOWN : integer{1,2};|
\end{center}

It is legal, since (in any static environment), the domain of \verb|integer{1,2,3}|
is \\
$\{\nvint(1), \nvint(2), \nvint(3)\}$, which subsumes
the domain of \verb|integer{1,2}|, which is \\ $\{\nvint(1), \nvint(2)\}$.

Since dynamic domains are potentially infinite, this requires \emph{symbolic reasoning}.
Furthermore, since any (statically evaluable) expressions may appear inside integer and bitvector
types, testing subsumption is undecidable.
We therefore approximate subsumption testing \emph{conservatively} via the predicate $\symsubsumes(\tenv, \vt, \vs)$.

\hypertarget{def-soundsubsumptiontest}{}
\begin{definition}[Sound Subsumption Test]
A predicate
\[
  \symsubsumes(\overname{\staticenvs}{\tenv} \aslsep \overname{\ty}{\vt} \aslsep \overname{\ty}{\vs}) \aslto \Bool
\]
is \emph{sound} if the following condition holds:
\begin{equation}
  \begin{array}{l}
  \forall \vt,\vs\in\ty.\ \tenv\in\staticenvs. \\
  \;\;\;\; \symsubsumes(\tenv, \vt, \vs) \typearrow \True \;\Longrightarrow\; \subsumes(\tenv, \vt, \vs)  \enspace.
  \end{array}
\end{equation}
\end{definition}

That is, if a sound subsumption test returns a positive answer, it means that
$\vt$ definitely \emph{subsumes} $\vs$ in the static environment $\tenv$.
This is referred to as a \emph{true positive}.
However, a negative answer means one of two things:
\begin{description}
  \item[True Negative:] indeed, $\vt$ does not subsume $\vs$ in the static environment $\tenv$; or
  \item[False Negative:] the symbolic reasoning is unable to decide.
\end{description}

In other words, $\symsubsumes(\tenv, \vt, \vs)$ errs on the \emph{safe side} ---
it never answers $\True$ when the real answer is $\False$, which would (undesirably)
determine the following statement as well-typed:
\begin{center}
  \verb|var x : integer{1,2} = UNKNOWN: integer;|
\end{center}

A sound but trivial subsumption test is one that always returns $\False$.
However, that would make all assignments be considered as not well-typed.
Indeed, it has the maximal set of false negatives.
Reducing the set of false negatives requires stronger symbolic reasoning algorithms,
which inevitably leads to higher computational complexity.
%
The symbolic subsumption test in \chapref{SymbolicSubsumptionTesting}
attempts to accept a large enough set of true positives, based on empirical trial and error,
while maintaining the computational complexity of the symbolic reasoning relatively low.
%
In particular, it serves as the definitive subsumption test that must be utilized
by any implementation of the ASL type system.
